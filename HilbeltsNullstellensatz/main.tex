\documentclass[a4paper]{jsarticle}
\usepackage{../math_note}
\usepackage[all]{xy}

\newcommand{\kx}{k[\mathbf{x}]}

\title{Hilbert's Nullstellensatz}
\author{七条彰紀}

\begin{document}
    この記事では文字$\mathbf{x}$を一貫して不定元を表すために使う.
    $\mathbf{x}=(x_1, \dots, x_d)$とする.

%   TODO: どうせなら使用したNoether normalization theoremと\cite{atimac}, 
%   Prop5.7の証明も書いてself-containedにしたい.
%   重要定理だけあって,証明の方針はかなり多い.
%   Jacobson ringであることを用いる証明,
%   幾何的なNoether normalization theoremを用いる証明,
%   Rabinowitsch's Trickを用いる証明,
%   Artin-Tate Lemmaを用いる証明(\cite{atimac}, Prop7.9),
%   valuation ringを用いる証明(\cite{atimac}, Cor5.24),
%   Chevalley's theoremを用いる証明,
%   TaoとEnrique Arrondoによる終結式を用いる証明,
%   モデル理論的な証明.

\section{The Statement}
    \begin{Thm}[Hilbert's Nullstellensatz (weak form)]
        $k$を代数閉体とする.この時,以下で定まる対応$\mu$は全単射である.
        \begin{defmap}
            \mu:& k^d& \to& \Max(\kx) \\ 
            {}& (a_1, \dots, a_d)& \mapsto& (x_1-a_1, \dots, x_d-a_d)
        \end{defmap}
    \end{Thm}

    \begin{Thm}[Hilbert's Nullstellensatz (strong form)]
        $k$を代数閉体とする.任意のイデアル$\I{a} \subsetneq \kx$に対して
        \[ \defs \zeros(\I{a})=\sqrt{\I{a}} \]が成立する.
    \end{Thm}

    \subsection{Another Forms of The Two Statements}
    以上の二つの定理が「弱形」「強形」と並べられる理由は今ひとつ理解りにくい.
    Terence Taoは自身のブログ``What's New"にHilbert's Nullstellensatzを扱った記事を掲載している\cite{tao}.
    それによると,以上の二つのステートメントはそれぞれ次のように言い換えられる.

    \begin{Thm}[Hilbert's Nullstellensatz (weak form) by Terence Tao]
        $k$を代数閉体とし,多項式$P_1, \dots, P_m \in \kx$をとる.
        この時,以下のちょうど一方が成立する.
        \begin{enumerate}[1.]
            \item 方程式系$P_1(\mathbf{x})=\ldots=P_m(\mathbf{x})=0$が解$\mathbf{x}=(a_1, \dots, a_d) \in k^d$を持つ.
            \item$P_1 Q_1 + \ldots + P_m Q_m=1$を満たす多項式$Q_1,\ldots,Q_m \in k[x]$が存在する.
        \end{enumerate}
    \end{Thm}

    \begin{Thm}[Hilbert's Nullstellensatz (strong form) by Terence Tao]
        $k$を代数閉体とし,多項式$P_1, \dots, P_m, R\in \kx$をとる.
        この時,以下のちょうど一方が成立する.
        \begin{enumerate}[1.]
            \item 方程式系$P_1(\mathbf{x})=\ldots=P_m(\mathbf{x})=0, R(\mathbf{x}) \neq 0$が解$\mathbf{x}=(a_1, \dots, a_d) \in k^d$を持つ.
            \item$P_1 Q_1 + \ldots + P_m Q_m=R^r$を満たす多項式$Q_1,\ldots,Q_m \in \kx$と非負整数$r$が存在する.
        \end{enumerate}
    \end{Thm}
    このようにweak formはstrong formで$R=1$とした場合であることは明白である.
    したがってstrong form $\implies$ weak formが分かる.

\section{Prepare for The Proofs}
    
\begin{Lemma}[Zariski's Lemma]
体$k$上の有限生成代数$K$が体ならば,$K$は$k$の有限次代数拡大体である.
\end{Lemma}

\paragraph{Noether normalization theoremを使うもの.}
\begin{proof}
    Noether normalization theoremにより,
    有限生成代数$K$が$R:=k[y_1, \dots, y_m]$の整拡大となり,
    しかも$k$上代数独立であるような元$y_1, \dots, y_m$が存在する.
    
    $m>0$とする.
    $y_1 \in K$ (::field)なので$y_1^{-1} \in K$.
    したがって$y_1^{-1}$は$R$上整であるから,
    以下が成立するような$f \in R$と非負整数$n$が存在する.
    \[ (y_1^{-1})^n+f(y_1, \dots, y_m) (y_1^{-1})^{n-1}=0 \]
    この両辺に$y_1^{n}$を掛けると,
    \[ 1+f(y_1, \dots, y_m) y_1=0 \]
    となり,これは$y_1, \dots, y_m$が$k$上代数独立
    \footnote
    {
        「$y_1, \dots, y_m$が$k$上代数独立」の定義:
        $f(y_1, \dots, y_m)=0$となる0でない多項式$f \in k[x_1, \dots, x_m]$が存在しない.
    } 
    であることに矛盾する.
    よって$m=0$.

    以上より,$K$は$k$の整拡大,すなわち代数拡大となる.
    再び$K$が$k$上有限生成代数な体であることから,$K$は$k$の有限次代数拡大体.
\end{proof}

\paragraph{整従属性を使うもの.}
(\cite{atimac}, Ex5.18と\cite{oneline}を参照)
\begin{proof}
    $k$代数としての$K$の生成元を$x_1,\dots,x_n$とする.
    すなわち,
    \[ K=k[x_1,\dots,x_n]. \]
    $n=1$ならば定理の成立は自明なので$n>1$としよう.
    示したいことは$x_1,\dots,x_n$のすべてが$k$上代数的であること.
    なので帰納的に考えて,
    $x_2,\dots,x_n$が$k(x_1)$
    \footnote{$k(x_1)$は$k$と$x_1$を含む明らかな体.}
    上代数的ならば
    $x_1,\dots,x_n$が$k$上代数的であることを示せば良い
    \footnote
    {
        言い換えれば
        $K=k(x_1,\dots,x_{n-2})(x_{n-1})[x_n] 
        \implies
        K=k(x_1,\dots,x_{n-3})(x_{n-2})[x_{n-1},x_n]
        \implies
        \dots
        \implies K=k(x_1)[x_2,\dots,x_n]$.
    }.
    そこで,$x_1$が$k$上代数的でなく同時に
    $x_2,\dots,x_n$が$k(x_1)$上代数的であると仮定し,背理法を用いる.

    $x_2,\dots,x_n$が$k[x_1]_f(=k[x_1][1/f])$上代数的であるような$f \in k[x_1]$が存在する.
    実際,$x_i$が$k(x_1)$上代数的であることから,
    次の式を満たす$f_{j}^{(i)},g_{j}^{(i)} \in k[x_1]$が存在する.
    \[
        x_i^{d_i}+\left( \frac{g_{d_i-1}^{(i)}}{f_{d_i-1}^{(i)}} \right) x_2^{d_i-1}+\dots+\left(\frac{g_{0}^{(i)}}{f_{0}^{(i)}}\right)=0
        \mwhere d_i>0, f_{j}^{(i)},g_{j}^{(i)} \in k[x_1], g_{j}^{i} \neq 0.
    \]
    $\frac{g_{d_i-1}^{(i)}}{f_{d_i-1}^{(i)}}$から
    $\frac{g_{0}^{(i)}}{f_{0}^{(i)}}$までを通分すると,
    各$x_i$は$k[x_1]\left[1/\prod_{j=0}^{d_i}f_{j}^{(i)}\right]$上整であることが分かる.
    したがって
    \[ f=\prod_{i=2}^n\prod_{j} f_{j}^{(i)} \in k[x_1] \]とおくと,
    $x_2,\dots,x_n$は$k[x_1]\left[1/f\right]=k[x_1]_f$
    上代数的であると言える.

    $K=k[x_1][x_2,\dots,x_n]$であり,
    $x_2,\dots,x_n$は$k[x_1]_f$上整だから,$K$は$k[x_1]_f$上整.
    この整従属関係と$K$が体であることから$k[x_1]_f$も体(\cite{atimac}, Prop5.7).
    $k[x_1] \subseteq k[x_1]_f \subseteq k(x_1)$かつ
    $k(x_1)$が$k[x_1]$を含む最小の体(商体)であることから
    $k(x_1)=k[x_1]_f$.
    しかし実際は$k[x_1]_f \neq k(x_1)$となる
    \footnote
    {
    実際,仮定から$x_1$は$k$上超越的だから,
    $f$は$k[x_1]$の有限個の既約多項式の積に分解され,$k[x_1]$は無数の既約多項式を持つ.
    なので$f$と互いに素な既約多項式$g \in k[x_1]$が存在する.
    $1/g=h/f^n$となる$n>0, h \in k[x_1]$が存在すれば,$gh=f^n=f \cdot f^{n-1} \in (g)$.
    $g$は素元だから$f \in (g)$となり,$f,g$が互いに素であることに反する.
    よって$1/g \not \in k[x_1]_f$.
    }.
    よって矛盾が生じた.
\end{proof}


    \input{noether_normalization_thm}

\section{Proof of The Weak Form}
    \subsection{From Zariski Lemma.}
    \paragraph{$(x_1-a_1, \dots, x_d-a_d) \in \Max(\kx)$.}
    各$x_i$を$x_i \mapsto a_i$と写す写像を考える.
    明らかにこれは全射で,$\ker = (x_1-a_1, \dots, x_d-a_d)$.
    準同型定理から$\kx/(x_1-a_1, \dots, x_d-a_d) \cong k$が得られる.
    剰余環が体になったので,$(x_1-a_1, \dots, x_d-a_d)$は$\Max(\kx)$の元.

    \paragraph{$\mu$ :: injective.}
    自明である.

    \paragraph{$\mu$ :: surjective.}
    $\I{m} \in \Max(\kx)$を任意に取る.
    この時$L=\kx/\I{m}$は体.
    しかも$\tilde{a}_i=x_i+\I{m}$とおけば$L=k[\{\tilde{a}_i\}_{i=1}^d]$と書けるから,$L$は有限生成$k$-代数.
    Zariski's Lemmaより,$L/k$は有限代数拡大である.
    $k$は代数的閉体であったから,$L \cong k$となり,よって各$\tilde{a}_i$は$k$の元$a_i$に対応する.
    こうして点$\mathbf{a}=(a_1,\dots,a_d)$が得られた.
    再び$x_i \mapsto a_i$という写像(像は$k[\{a_i\}_{i=1}^d]=k$)に準同型定理を用いれば,
    \[ \kx/\mu(\mathbf{a}) \cong k[\{a_i\}_{i=1}^d] \cong k[\{\tilde{a}_i\}_{i=1}^d]=\kx/\I{m} \]
    という同型が構成できる.
    したがって$\I{m}=\mu(\mathbf{a})$.

\section{Proof of The Strong Form}
    \subsection{From Zariski Lemma.}
    $\sqrt{\I{a}} \subseteq \defs \zeros (\I{a})$は明らか.
    逆に$f \not \in \sqrt{\I{a}}$として$f \not \in \defs \zeros (\I{a})$を示す.

    \paragraph{素イデアル$\I{p}$の存在.}
    $\sqrt{\I{a}}$は$\I{a}$を含む素イデアル全体の共通部分であるから,
    この時$\I{a} \subseteq \I{p}, f \not \in \I{p}$なる素イデアル$\I{p}$が存在する.

    \paragraph{体$L$の構成.}
    $\bar{f}=f+\I{p} (\neq 0)$とし,$C=(\kx/\I{p})_f=(\kx/\I{p})[1/\bar{f}]$とする.
    さらに$\I{m}$を$C$の極大イデアルとおく.
    すると体$L=C/\I{m}=(\kx/\I{p})_f/\I{m}$は
    $\tilde{a}_i=\frac{x_i+\I{p}}{1}+\I{m}$で生成される有限生成$k$-代数.

    \paragraph{$\mathbf{a} \in \zeros(\I{a})$かつ$f(\mathbf{a}) \neq 0$なる点$\mathbf{a}$を得る.}
    Zariski's Lemmaより,$L/k$は有限代数拡大である.
    $k$は代数的閉体であったから,$L \cong k$となり,
    よって各$\tilde{a}_i$は$k$の元$a_i$に対応する.
    こうして点$\mathbf{a}=(a_1,\dots,a_d)$が得られた.
    ここで以下の準同型を考える.
    \[
        \phi: \kx \to \kx/\I{p} \to (\kx/\I{p})_f=C \to C/\I{m} \cong k;
    ~~ x_i \mapsto x_i+\I{p} \mapsto \frac{x_i+\I{p}}{1} \mapsto \frac{x_i+\I{p}}{1}+\I{m}=\tilde{a}_i \mapsto a_i.
    \]
    これは代入写像.
    繋いでいる写像はすべて準同型なので,
    $g \in \I{p}$は$C$の零元$\frac{0+\I{p}}{1}$へ写り,最終的に零元$0$へ写る.
    同様に,$f$は$C$の単元$\frac{f+\I{p}}{1}$へ写り,最終的に単元へ写る.
    つまり$g \in \I{p}$について$\phi(g)=g(\mathbf{a})=0$で,$\phi(f)=f(\mathbf{a})$は単元.
    よって$\mathbf{a} \in \zeros(\I{p}) \subset \zeros(\I{a})$かつ$f(\mathbf{a}) \neq 0$.

\begin{thebibliography}{9}
    \bibitem{atimac}
        M.F.Atiyah, I.G.MacDonald
        ''Introduction to Commutative Algebra"

    \bibitem{tao}
        Terence Tao (2007/11/27)
        ''Hilbert’s nullstellensatz"
        \url{https://terrytao.wordpress.com/2007/11/26/hilberts-nullstellensatz/}

    \bibitem{oneline}
        Alborz Azarang
        ''A one-line undergraduate proof of Zariski's lemma and Hilbert's nullstellensatz"
        \url{http://arxiv.org/abs/1506.08376}
\end{thebibliography}

\end{document}
