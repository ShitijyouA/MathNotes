
\begin{Lemma}[Zariski's Lemma]
体$k$上の有限生成代数$K$が体ならば,$K$は$k$の有限次代数拡大体である.
\end{Lemma}

\paragraph{Noether normalization theoremを使うもの.}
\begin{proof}
    Noether normalization theoremにより,
    有限生成代数$K$が$R:=k[y_1, \dots, y_m]$の整拡大となり,
    しかも$k$上代数独立であるような元$y_1, \dots, y_m$が存在する.
    
    $m>0$とする.
    $y_1 \in K$ (::field)なので$y_1^{-1} \in K$.
    したがって$y_1^{-1}$は$R$上整であるから,
    以下が成立するような$f \in R$と非負整数$n$が存在する.
    \[ (y_1^{-1})^n+f(y_1, \dots, y_m) (y_1^{-1})^{n-1}=0 \]
    この両辺に$y_1^{n}$を掛けると,
    \[ 1+f(y_1, \dots, y_m) y_1=0 \]
    となり,これは$y_1, \dots, y_m$が$k$上代数独立
    \footnote
    {
        「$y_1, \dots, y_m$が$k$上代数独立」の定義:
        $f(y_1, \dots, y_m)=0$となる0でない多項式$f \in k[x_1, \dots, x_m]$が存在しない.
    } 
    であることに矛盾する.
    よって$m=0$.

    以上より,$K$は$k$の整拡大,すなわち代数拡大となる.
    再び$K$が$k$上有限生成代数な体であることから,$K$は$k$の有限次代数拡大体.
\end{proof}

\paragraph{整従属性を使うもの.}
(\cite{atimac}, Ex5.18と\cite{oneline}を参照)
\begin{proof}
    $k$代数としての$K$の生成元を$x_1,\dots,x_n$とする.
    すなわち,
    \[ K=k[x_1,\dots,x_n]. \]
    $n=1$ならば定理の成立は自明なので$n>1$としよう.
    示したいことは$x_1,\dots,x_n$のすべてが$k$上代数的であること.
    なので帰納的に考えて,
    $x_2,\dots,x_n$が$k(x_1)$
    \footnote{$k(x_1)$は$k$と$x_1$を含む明らかな体.}
    上代数的ならば
    $x_1,\dots,x_n$が$k$上代数的であることを示せば良い
    \footnote
    {
        言い換えれば
        $K=k(x_1,\dots,x_{n-2})(x_{n-1})[x_n] 
        \implies
        K=k(x_1,\dots,x_{n-3})(x_{n-2})[x_{n-1},x_n]
        \implies
        \dots
        \implies K=k(x_1)[x_2,\dots,x_n]$.
    }.
    そこで,$x_1$が$k$上代数的でなく同時に
    $x_2,\dots,x_n$が$k(x_1)$上代数的であると仮定し,背理法を用いる.

    $x_2,\dots,x_n$が$k[x_1]_f(=k[x_1][1/f])$上代数的であるような$f \in k[x_1]$が存在する.
    実際,$x_i$が$k(x_1)$上代数的であることから,
    次の式を満たす$f_{j}^{(i)},g_{j}^{(i)} \in k[x_1]$が存在する.
    \[
        x_i^{d_i}+\left( \frac{g_{d_i-1}^{(i)}}{f_{d_i-1}^{(i)}} \right) x_2^{d_i-1}+\dots+\left(\frac{g_{0}^{(i)}}{f_{0}^{(i)}}\right)=0
        \mwhere d_i>0, f_{j}^{(i)},g_{j}^{(i)} \in k[x_1], g_{j}^{i} \neq 0.
    \]
    $\frac{g_{d_i-1}^{(i)}}{f_{d_i-1}^{(i)}}$から
    $\frac{g_{0}^{(i)}}{f_{0}^{(i)}}$までを通分すると,
    各$x_i$は$k[x_1]\left[1/\prod_{j=0}^{d_i}f_{j}^{(i)}\right]$上整であることが分かる.
    したがって
    \[ f=\prod_{i=2}^n\prod_{j} f_{j}^{(i)} \in k[x_1] \]とおくと,
    $x_2,\dots,x_n$は$k[x_1]\left[1/f\right]=k[x_1]_f$
    上代数的であると言える.

    $K=k[x_1][x_2,\dots,x_n]$であり,
    $x_2,\dots,x_n$は$k[x_1]_f$上整だから,$K$は$k[x_1]_f$上整.
    この整従属関係と$K$が体であることから$k[x_1]_f$も体(\cite{atimac}, Prop5.7).
    $k[x_1] \subseteq k[x_1]_f \subseteq k(x_1)$かつ
    $k(x_1)$が$k[x_1]$を含む最小の体(商体)であることから
    $k(x_1)=k[x_1]_f$.
    しかし実際は$k[x_1]_f \neq k(x_1)$となる
    \footnote
    {
    実際,仮定から$x_1$は$k$上超越的だから,
    $f$は$k[x_1]$の有限個の既約多項式の積に分解され,$k[x_1]$は無数の既約多項式を持つ.
    なので$f$と互いに素な既約多項式$g \in k[x_1]$が存在する.
    $1/g=h/f^n$となる$n>0, h \in k[x_1]$が存在すれば,$gh=f^n=f \cdot f^{n-1} \in (g)$.
    $g$は素元だから$f \in (g)$となり,$f,g$が互いに素であることに反する.
    よって$1/g \not \in k[x_1]_f$.
    }.
    よって矛盾が生じた.
\end{proof}

