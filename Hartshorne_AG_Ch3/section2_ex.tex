\documentclass[a4paper]{jsarticle}
\usepackage{../math_note, exercise}
\usepackage[all]{xy}
\renewcommand{\thesection}{Ex2.\arabic{section}}

\begin{document}
\section{Grothendieck Vanithing Thm is Best Possible.} %% Ex2.1 
    \subsection{Let $X=\affine^1, U=X-\{P,Q\}$. $H^1(X, \Z_U) \neq 0$.}
    $k$ :: infinite field,
    $X=\affine^1_k$,
    $P, Q \in X$ :: distinct closed points,
    $Y=\{P,Q\}, U=X-Y$
    とおく.
    $\#Y<\#k=\infty$なので$\#U=\infty.$
    
    次の完全列が成り立つ.
    \[\xymatrix{ 0 \ar[r]& \Z_U \ar[r]& \Z_X \ar[r]& \Z_Y \ar[r]& 0. }\]
    ここから次の長完全列が誘導される.
    \[
    \xymatrix
    {
        0  \ar[r]& \Gamma(X, \Z_U) \ar[r]& \Gamma(X, \Z_X) \ar[r]& \Gamma(X, \Z_Y) \\
        {} \ar[r]& H^1(X, \Z_U) \ar[r]& H^1(X, \Z_X) \ar[r]& H^1(X, \Z_Y) \\
        {} \ar[r]& \dots
    }
    \]
    完全列であるから,
    $H^1(X, \Z_U)=0$は
    $\Gamma(X, \Z_X) \to \Gamma(X, \Z_Y)$が全射であることと同値である.
    一方,$\#U=\infty, \#Y=2$であるから,
    \[
        \Gamma(X, \Z_X)=\Z,
        \qquad
        \Gamma(X, \Z_Y)=\Gamma(\{P\} \sqcup \{Q\}, \Z_Y)=\Z \oplus \Z.
    \]
    ($\sqcup$はdisjoint unionを意味する.)
    よって全射にはなり得ない.
    したがって$H^1(X, \Z_U) \neq 0$.

    \subsection{Generalization for $\affine^n$.}
    $k$ :: infinite field,
    $X=\affine^n_k$,
    $H_1,\dots,H_{n+1}$ :: hyperplanes,
    $Y=\bigcup_{i=1}^{n+1} H_i, U=X-Y$
    とおく.
    $H^n(X, \Z_U) \neq 0$を示す.
    $n=1$の場合については(a)で示したから,
    $n>1$とする.

    完全列
    \[\xymatrix{ 0 \ar[r]& \Z_U \ar[r]& \Z_X \ar[r]& \Z_Y \ar[r]& 0 }\]
    から誘導される長完全列の一部は次の様になる.
    \[\xymatrix{ H^{n-1}(X, \Z_X) \ar[r]& H^{n-1}(X, \Z_Y) \ar[r]& H^{n}(X, \Z_U) \ar[r]& H^{n}(X, \Z_X). }\]
    また,$\Z_X$ :: constant sheaf on irreducible spaceなので
    $\Z_X$ :: flasque (II, Ex1.16a), acyclic (Prop2.5).
    今$n>1$だから$H^{n-1}(X, \Z_X)=H^{n}(X, \Z_X)=0$.
    よって$H^{n-1}(X, \Z_Y) \iso H^{n}(X, \Z_U)$が得られる.
    なので我々は$H^{n-1}(X, \Z_Y) \neq 0$を示すことにする.

%    次元についての帰納法を用いる.
%    以下の主張が成立していると仮定しよう.
%    \begin{Claim}
%        $\affine^{n-1}$から$n$個のhyperplaneを取り除いた空間を$V$とすると,
%        $H^{n-1}(\affine^{n-1}, \Z_V) \neq 0$.
%    \end{Claim}
%    まず$H_{n+1} \iso \affine^{n-1}$に注意する.
%    $Y_n=\bigcup_{i=1}^{n} H_i$とおくと,
%    $H_{n+1}-Y_n$は仮定における$V$だとみなせる.
%    したがって
%    \[ H^{n-1}(H_{n+1}, \Z_{H_{n+1}-Y_n}) \iso H^{n-1}(X, \Z_{H_{n+1}-Y_n}) \neq 0. \]
%    次の完全列が成り立つ.
%    \[\xymatrix{ 0 \ar[r]& \Z_{H_{n+1}-Y_n} \ar[r]& \Z_{H_{n+1}} \ar[r]& \Z_{H_{n+1} \cap Y_n} \ar[r]& 0. }\]

\section{$\shO_{\proj^1_k}$ :: acyclic.} %% Ex2.2 

\section{Cohomology with Supports.} %% Ex2.3 
    $X$ :: topological space,
    $Y$ :: closed subset of $X$,
    $U=X-Y$,
    $\shF$ :: sheaf of abelian group on $X$
    とする.
    この時,$\Gamma_Y(X, \shF)$を以下で定める.
    \[
        \Gamma_Y(X, \shF)
        =\{ s \in \Gamma(X, \shF) \mid \Supp s \subseteq Y \}
        =\{ s \in \Gamma(X, \shF) \mid s|_{X-Y}=0 \}.
    \]

    \subsection{$\Gamma_Y(X,-): \catAb(X) \to \catAb$ :: left exact functor.}
    $\catAb(X)$の完全列を考える.
    \[\xymatrix{ 0 \ar[r]& \shF' \ar[r]& \shF \ar[r]& \shF''. }\]
    ここから誘導される以下の列が完全であることはII, Ex1.8で示した.
    \[\xymatrix{ 0 \ar[r]& \Gamma(X,\shF') \ar[r]^-{f}& \Gamma(X,\shF) \ar[r]^-{g}& \Gamma(X,\shF''). }\]
    $\Gamma$を$\Gamma_Y$に付け替えても完全列であることを示す.

    \paragraph{Exact at $\Gamma_Y(X,\shF')$.}
    示すべきことは次のこと.
    \[ \Forall{s \in \Gamma_Y(X,\shF')} f(s)=0 \implies s=0. \]
    $f: \Gamma(X,\shF') \to \Gamma(X,\shF)$が単射なので,
    その制限$f|_{\Gamma_Y(X,\shF')}$も単射.
    よって主張が示せた.

    \paragraph{Exact at $\Gamma_Y(X,\shF)$.}
    示すべきは次のこと.
    \[ \Forall{t \in \Gamma_Y(X,\shF)} g(t)=0 \iff \Exists{u \in \Gamma_Y(X,\shF')} f(u)=t. \]
    $\impliedby$は$g \circ f=0$から直ちに分かる.
    $\implies$は次のように示す.
    元の完全列があるため,$f(u)=t$を満たす$u$が$\Gamma(X,\shF')$からはとれる.
    今$t \in \Gamma_Y(X,\shF)$から$t$は以下を満たす.
    \[ \Forall{P \in U} t_P=(f(u))_P=f_P(u_P)=0. \]
    $f_P$は単射であるから,任意の点$P \in U$について$u_P=0$.
    よって$u \in \Gamma_Y(X,\shF')$.

    \subsection{$\shF'$ :: flasque then
        $0 \to \Gamma_Y(X, \shF') \to \Gamma_Y(X, \shF) \to \Gamma_Y(X, \shF'') \to 0$ is exact.}
    次の完全列は成立する(II, Ex1.16b).
    \[\xymatrix{ 0 \ar[r]& \Gamma(X,\shF') \ar[r]^-{f}& \Gamma(X,\shF) \ar[r]^-{g}& \Gamma(X,\shF'') \ar[r]& 0. }\]
    これらの$\Gamma$を$\Gamma_Y$に取り替えても良いことを示す.
    (a)で示したことを合わせれば,次のことを示せば良いということになる.
    \[ \Forall{t \in \Gamma_Y(X,\shF'')} \Exists{u \in \Gamma_Y(X,\shF)} g(u)=t. \]
    元の完全列から,
    $g(u)=t$を満たす$u$が$\Gamma(X,\shF)$からはとれる.
    今$t \in \Gamma_Y(X,\shF'')$から$t$は以下を満たす.
    \[ t|_U=(g(u))|_U=g|_U(u|_U)=0. \]
    元の完全列があるため,
    $f|_{U}(\tilde{s})=u|_{U}$となる$\tilde{s} \in \Gamma(U,\shF')$がとれる.
    $\shF'$ :: flasqueなので$s|_U=\tilde{s}$となる
    $s \in \Gamma(X, \shF')$が存在する.
    構成から
    \[ f(s)|_U=u|_U \iff \bar{u}:=u-f(s) \in \Gamma_Y(X,\shF). \]
    $g \circ f=0$なので$g(\bar{u})=g(u-f(s))=g(u)=t$.
    よって$\bar{u}$が条件を満たす.

    \subsection{injective sheaves are acyclic for $\Gamma_Y(X, -)$.}
    Prop2.5(injective sheaves are acyclic for $\Gamma(X,-)$)の証明がそのまま使える.
    この証明では(b)の内容の他にはderived functorの性質しか使わない.

    \subsection{$\shF$ :: flasque then
        $0 \to \Gamma_Y(X, \shF) \to \Gamma(X, \shF) \to \Gamma(X-Y, \shF) \to 0$ is exact.}
    引き続き$U:=X-Y$とする.
    $\res_X^U: \Gamma(X, \shF) \to \Gamma(U, \shF)$は$\shF$ :: flasqueゆえに全射.
    この写像のkernelは次のような集合である.
    \[ \ker \res_X^U=\{ s \in \Gamma(X, \shF) \mid s|_U=0 \}. \]
    これは$\Gamma_Y(X, \shF)$に他ならない.

    \subsection{there is a long exact seq ::
        $\cdots \to H_Y^i(X,\shF) \to H^i(X, \shF) \to H^i(U, \shF|_U) \to \cdots$. }
    $\shF$のinjective resolution :: $0 \to \shF \to \shI^*$をとる.
    このとき$\shI^*$はflasque resolutionでもある.
    すると(d)より次の図式が得られる.
    \[
    \xymatrix@R=20pt@C=40pt
    {
        {} & 0 \ar[d]& 0 \ar[d]& \dots \\
        0 \ar[r]& \Gamma_Y(X, \shI^0) \ar[r]^-{(d^0_{X})|_{\subseteq Y}}\ar@{^(->}[d]&
                  \Gamma_Y(X, \shI^1) \ar[r]^-{(d^1_{X})|_{\subseteq Y}}\ar@{^(->}[d]& \dots \\
        0 \ar[r]& \Gamma(X, \shI^0)   \ar[r]^-{d^0_X}\ar[d]^-{\res_X^U}&
                  \Gamma(X, \shI^1)   \ar[r]^-{d^1_X}\ar[d]^-{\res_X^U}& \dots \\
        0 \ar[r]& \Gamma(U, \shI^0)   \ar[r]^-{d^0_U}\ar[d]&
                  \Gamma(U, \shI^1)   \ar[r]^-{d^1_U}\ar[d]& \dots \\
        {} & 0 & 0 & \dots \\
    }
    \]
    ここで$\shI^*$の微分射を$d^i: \shI^i \to \shI^{i+1}$とした.
    また$(d^i_{X})|_{\subseteq Y}=(d^i_{X})|_{\Gamma_Y(X, \shI^i)}$とした.
    以上の写像の定義から,それぞれの四角形が可換であることが確認できる.

    まとめると,次のexact sequence of complexsが存在する.
    \[ 0 \to \Gamma_Y(X, \shI^*) \to \Gamma(X, \shI^*) \to \Gamma(X-Y, \shI^*) \to 0. \]
    よってderived functorの一般論から主張のlong exact sequenceが得られる.

    \subsection{$Y \subseteq V$ :: open in $X$ then $H^i_Y(X, \shF) \iso H^i_Y(V, \shF|_V)$.}
    \begin{Claim}
        $Y$を$X$の閉集合,$V$を$Y$を含む$X$の開集合とする.
        任意の$\shF$ :: sheaf on $X$ of abelian groupについて次が成り立つ.
        \[ \Gamma_Y(X, \shF)  \iso \Gamma_Y(V, \shF|_V). \]
    \end{Claim}
    \begin{proof}
        同型写像を次のように定める.
        \begin{defmap}
            \rho:& \Gamma_Y(X, \shF)& \to& \Gamma_Y(V, \shF|_V) \\
            {}& s& \mapsto& s|_V \\
            {}& \epsilon& \mapedfrom& t
        \end{defmap}
        ただし$\epsilon$は次のようなsectionである:
        \[
            \Forall{P \in X}
            \epsilon_P=
            \begin{cases}{}
                t_P & (P \in V) \\
                0 & (P \in Y^c).
            \end{cases}
        \]
        $V \cap Y^c$において$t_P=0$なのでこれはwell-defined.
        また,$\epsilon$は
        $t \in \Gamma(V, \shF)$と$0 \in \Gamma(Y^c, \shF)$の張り合わせであるから,
        gluability axiomより$\epsilon \in \Gamma(X, \shF)$.
        よって$\rho$全体もwell-defiend.
    \end{proof}

    $0 \to \shF \to \shI^*$ :: injective resolution of $\shF$をとる.
    この時,
    $0 \to \shF|_V \to \shI^*|_V$はinjective/flasque resolution of $\shF|_V$.
    主張から$\Gamma_Y(X, \shI^*) \iso \Gamma_Y(V, \shI^*|_V)$なので,
    求めるcohomology groupの同型が得られる.

\section{Mayer--Vietoris Sequence.} %% Ex2.4 
    $Y_1, Y_2$ :: closed subset of $X$とする.
    $\shF$ :: sheaf of abelian group on $X$について
    次の完全列が存在することを証明する.
    \[
    \xymatrix@R=10pt
    {
        \dots \ar[r]&
            H_{Y_1 \cap Y_2}^i(X, \shF) \ar[r]&
            H_{Y_1}^i(X, \shF) \oplus H_{Y_2}^i(X, \shF) \ar[r]&
            H_{Y_1 \cup Y_2}^i(X, \shF) \\
            {} \ar[r]& H_{Y_1 \cap Y_2}^i(X, \shF) \ar[r] & \dots
    }
    \]

    \begin{Claim}
        $\shF$ :: flasque sheaf on $X$について,
        次の完全列が存在する.
        \[
        \xymatrix@R=10pt
        {
            0 \ar[r]&
                \Gamma_{Y_1 \cap Y_2}(X, \shF) \ar[r]&
                \Gamma_{Y_1}(X, \shF) \oplus \Gamma_{Y_2}(X, \shF) \ar[r]&
                \Gamma_{Y_1 \cup Y_2}(X, \shF) \ar[r]&
            0.
        }
        \]
    \end{Claim}
    \begin{proof}
        9-Lemmaを用いる.
        \[
        \xymatrix
        {
            {} & 0 \ar[d]& 0 \ar[d]& 0 \ar[d]& {} \\
            0 \ar[r]&
            \Gamma_{Y_1 \cap Y_2}(X, \shF) \ar[r]\ar[d]&
                \Gamma_{Y_1}(X, \shF) \oplus \Gamma_{Y_2}(X, \shF) \ar[r]\ar[d]&
                \Gamma_{Y_1 \cup Y_2}(X, \shF) \ar[r]\ar[d]&
            0 \\
            0 \ar[r]&
                \Gamma(X, \shF) \ar[r]\ar[d]&
                \Gamma(X, \shF) \oplus \Gamma(X, \shF) \ar[r]\ar[d]&
                \Gamma(X, \shF) \ar[r]\ar[d]&
            0 \\
            0 \ar[r]&
            \Gamma(X-(Y_1 \cap Y_2), \shF) \ar[r]\ar[d]&
                \Gamma(X-Y_1, \shF) \oplus \Gamma(X-Y_2, \shF) \ar[r]\ar[d]&
                \Gamma(X-(Y_1 \cup Y_2), \shF) \ar[r]\ar[d]&
            0 \\
            {} & 0 & 0 & 0 & {}
        }
        \]
        各列はEx2.3dからexact.
        真ん中の行は明らかにsplit exact sequenceで,
        一番下の行は以下の写像でexactになる.
        よって一番上の行もexact.
        \[
        \xymatrix@R=5pt
        {
            0 \ar[r]&
            \Gamma(X-(Y_1 \cap Y_2), \shF) \ar[r]&
                \Gamma(X-Y_1, \shF) \oplus \Gamma(X-Y_2, \shF) \ar[r]&
                \Gamma(X-(Y_1 \cup Y_2), \shF) \ar[r]&
            0 \\
            {} & a \ar@{|->}[r]& (a|_{X-Y_1},-a|_{X-Y_2})& {} & {} \\
            {} & {}& (a,b) \ar@{|->}[r]& a|_{X-(Y_1 \cup Y_2)}+b|_{X-(Y_1 \cup Y_2)} & {} \\
        }
        \]
    \end{proof}
    主張から次のexact sequence of complexsが得られる.
    \[
    \xymatrix@R=10pt
    {
        0 \ar[r] & 
        \Gamma_{Y_1 \cap Y_2}(X, \shI^*) \ar[r]&
            \Gamma_{Y_1}(X, \shI^*) \oplus \Gamma_{Y_2}(X, \shI^*) \ar[r]&
            \Gamma_{Y_1 \cup Y_2}(X, \shI^*) \ar[r]&
        0.
    }
    \]
    これからderived functorの一般論によって所望の完全列が得られる.

\section{$H^i_P(X, \shF)=H^i_P(X_P, \shF_P)$.} %% Ex2.5 

\section{$\{\shI_{\alpha} \}$ :: direct system of injective sheaves
    then $\lim \shI_{\alpha}$ :: injective sheaf.} %% Ex2.6 
    $X$ :: noetherian topological space,
    $\{ \shI_{\alpha} \}_{\alpha \in A}$ :: direct system of injective sheaves of abelian groups on $X$
    とする.
    この時,$\varinjlim_{\alpha \in A} \shI_{\alpha}$ :: injectiveであることを示す.

    \begin{Claim}
        $\shI$ :: sheaf of abelian groups on $X$とする.
        $\shI$ :: injectiveと,
        次の可換図式で,$\bar{f}$が存在することは同値である.
        \[
        \xymatrix
        {
            {}^{\forall}\shR \ar@{^{(}->}[r] \ar[d]_-{{}^{\forall} f}&
            \Z_{{}^{\forall}U} \ar[ld]^-{{}^{\exists} \bar{f}} \\
            \shI
        }
        \]
    \end{Claim}
    \begin{proof}
        injectiveならば$\bar{f}$が存在することは
        injectiveの定義から従う.
        逆を示そう.

        $\alpha \in A$と$\shF_{\alpha}$をThm2.7の証明のStep 3と同様に定義する.
        また$\alpha' \subsetneq \alpha$と$f: \shR \to \shI$について,
        $f_{\alpha'}=f|_{\shR \cap \shF_{\alpha'}}$と置く.
        $\sum:  \bigoplus_{\alpha' \subsetneq \alpha} \shF_{\alpha'} \to \shF_{\alpha}$を
        $\sum \left(\bigoplus_{\alpha'} s_{\alpha'}\right)=\sum_{\alpha'} s_{\alpha'}$
        と定めると,これは全射に成る.
        (2)において$f_{\alpha'}$の拡張$\overline{f_{\alpha'}}$が存在すれば,
        $\bar{f}=(\bigoplus_{\alpha'} \overline{f_{\alpha'}}) \circ \sum^{-1}$と置くことで
        (TODO: この$\bar{f}$はwell-defined?)
        $f$の拡張が得られる(TODO) .
        \[
        (1)
        \xymatrix@C=30pt@M=5pt
        {
            {}& {} & {} \\
            \shR \ar@{^{(}->}[rr] \ar[rd]^-{f}& {} & \shF_{\alpha}\\
            {} & \shI & {}
        }
        \qquad
        \qquad
        (2)
        \xymatrix@C=5pt@M=5pt
        {
            \bigoplus_{\alpha' \subsetneq \alpha} (\shR \cap \shF_{\alpha'})
            \ar@{^{(}->}[rr]\ar[d] \ar`ld[dd][rdd]_-{\bigoplus_{\alpha'} f_{\alpha'}}&
                {} &
                \bigoplus_{\alpha' \subsetneq \alpha} \shF_{\alpha'} \ar@{->>}[d]^-{\sum}\\
            \shR \ar@{^{(}->}[rr] \ar[rd]^-{f}& {} & \shF_{\alpha}\\
            {} & \shI & {}
        }
        \]
        \[
        (3)
        \xymatrix@C=5pt@M=5pt
        {
            \bigoplus_{\alpha' \subsetneq \alpha} (\shR \cap \shF_{\alpha'})
                    \ar@{^{(}->}[rr]\ar[d] \ar`ld[dd][rdd]_-{\bigoplus_{\alpha'} f_{\alpha'}}&
                {} &
                \bigoplus_{\alpha' \subsetneq \alpha} \shF_{\alpha'}
                    \ar@{->>}[d]_-{\sum}\ar`rd[dd][ldd]^-{\bigoplus_{\alpha'} \overline{f_{\alpha'}}}\\
                    \shR \ar@{^{(}->}[rr] \ar[rd]^-{f}& {} & \shF_{\alpha} \ar[ld]_-{\bar{f}}\\
            {} & \shI & {}
        }
        \]
        $\shF=\varinjlim_{\alpha \in A} \shF_{\alpha}$と
        $|\alpha|$についての帰納法を用いれば,
        主張が得られる.
    \end{proof}

    $U \subseteq X$ :: open subset, 
    $\shR \subseteq \Z_{U}$ :: subsheafを任意に取る.
    $U$をirreducible componentに分解し
    $U=\bigsqcup_i U_i$とする.
    $X$ :: noetherianゆえ$U_i$は有限個しかない.

    今,$\shR(U_i) \subseteq \Z_U(U_i)=\Z$なので
    $\shR(U_i)$は$s_i \in \Z$で生成されるabelian groupである.
    $V$ :: open subset of $U_i$を任意に取った時,
    $\shR(V)$は$(s_i)|_V$で生成される.
    実際,$\shR(V) \subseteq \Z_U(V)$の生成元を$g \in \Z_U(V)$とすると,
    $\Z_{U_i}$ :: flasque (II, Ex1.16a)なので$\bar{g}|_V=g$なる
    $\bar{g} \in \Z_U(U_i)$が存在する.
    したがって$\bar{g}=r \cdot s_i$となるが,
    $(s_i)|_V, r \cdot (s_i)|_V \in \shR(V)$であり,
    $\shR(V)$の生成元はただひとつなので,結局$r=1$となる.
    すなわち$(s_i)|_V$が$\shR(V)$の生成元.
    以上から$\shR$は有限個の元$\{s_i\}_i$で生成される.
    
    $f: \shR \to \varinjlim \shI_{\alpha}$を任意に取る.
    すると全ての$i$について
    $\Gamma(U_i, f)(s_i) \in \Gamma(U_i, \varinjlim \shI_{\alpha})$なので
    \[ \Gamma(U_i, f)(s_i) \in \Gamma(U_i, \shI_{\alpha_i}) \]
    なる$\alpha_i$が存在する.
    したがって$f|_{U_i}: \shR|_{U_i} \to \shI_{\alpha_i}|_{U_i}$が得られる.
    これは$\shI_{\alpha_i}$ :: injectiveゆえ拡張できて
    \[ (f|_{U_i})\sidebar: \Z_{U_i} \to \shI_{\alpha_i}|_{U_i} \to (\varinjlim \shI_{\alpha})|_{U_i} \]
    が得られる.
    貼り合わせれば$\bar{f}: \shR \to \varinjlim \shI_{\alpha}$となる.

\section{$H^1(S^1, \Z_{S^1}) \iso \Z$.} %% Ex2.7 

\end{document}
