\documentclass[a4paper]{jsarticle}
\usepackage{../math_note, exercise}
\usepackage[all]{xy}
\renewcommand{\thesection}{Ex2.\arabic{section}}

\begin{document}
\section{Grothendieck Vanithing Thm is Best Possible.} %% Ex2.1 
    \subsection{Let $X=\affine^1, U=X-\{P,Q\}$. $H^1(X, \Z_U) \neq 0$.}
    $k$ :: infinite field,
    $X=\affine^1_k$,
    $P, Q \in X$ :: distinct closed points,
    $Y=\{P,Q\}, U=X-Y$
    とおく.
    $\#Y<\#k=\infty$なので$\#U=\infty.$
    
    次の完全列が成り立つ.
    \[\xymatrix{ 0 \ar[r]& \Z_U \ar[r]& \Z_X \ar[r]& \Z_Y \ar[r]& 0. }\]
    ここから次の長完全列が誘導される.
    \[
    \xymatrix
    {
        0  \ar[r]& \Gamma(X, \Z_U) \ar[r]& \Gamma(X, \Z_X) \ar[r]& \Gamma(X, \Z_Y) \\
        {} \ar[r]& H^1(X, \Z_U) \ar[r]& H^1(X, \Z_X) \ar[r]& H^1(X, \Z_Y) \\
        {} \ar[r]& \dots
    }
    \]
    完全列であるから,
    $H^1(X, \Z_U)=0$は
    $\Gamma(X, \Z_X) \to \Gamma(X, \Z_Y)$が全射であることと同値である.
    一方,$\#U=\infty, \#Y=2$であるから,
    \[
        \Gamma(X, \Z_X)=\Z,
        \qquad
        \Gamma(X, \Z_Y)=\Gamma(\{P\} \sqcup \{Q\}, \Z_Y)=\Z \oplus \Z.
    \]
    ($\sqcup$はdisjoint unionを意味する.)
    よって全射にはなり得ない.
    したがって$H^1(X, \Z_U) \neq 0$.

    \subsection{Generalization for $\affine^n$.}
    $k$ :: infinite field,
    $X=\affine^n_k$,
    $H_1,\dots,H_{n+1}$ :: hyperplanes,
    $Y=\bigcup_{i=1}^{n+1} H_i, U=X-Y$
    とおく.
    $H^n(X, \Z_U) \neq 0$を示す.
    $n=1$の場合については(a)で示したから,
    $n>1$とする.

    完全列
    \[\xymatrix{ 0 \ar[r]& \Z_U \ar[r]& \Z_X \ar[r]& \Z_Y \ar[r]& 0 }\]
    から誘導される長完全列の一部は次の様になる.
    \[\xymatrix{ H^{n-1}(X, \Z_X) \ar[r]& H^{n-1}(X, \Z_Y) \ar[r]& H^{n}(X, \Z_U) \ar[r]& H^{n}(X, \Z_X). }\]
    また,$\Z_X$ :: constant sheaf on irreducible spaceなので
    $\Z_X$ :: flasque (II, Ex1.16a), acyclic (Prop2.5).
    今$n>1$だから$H^{n-1}(X, \Z_X)=H^{n}(X, \Z_X)=0$.
    よって$H^{n-1}(X, \Z_Y) \iso H^{n}(X, \Z_U)$が得られる.
    なので我々は$H^{n-1}(X, \Z_Y) \neq 0$を示すことにする.

%    次元についての帰納法を用いる.
%    以下の主張が成立していると仮定しよう.
%    \begin{Claim}
%        $\affine^{n-1}$から$n$個のhyperplaneを取り除いた空間を$V$とすると,
%        $H^{n-1}(\affine^{n-1}, \Z_V) \neq 0$.
%    \end{Claim}
%    まず$H_{n+1} \iso \affine^{n-1}$に注意する.
%    $Y_n=\bigcup_{i=1}^{n} H_i$とおくと,
%    $H_{n+1}-Y_n$は仮定における$V$だとみなせる.
%    したがって
%    \[ H^{n-1}(H_{n+1}, \Z_{H_{n+1}-Y_n}) \iso H^{n-1}(X, \Z_{H_{n+1}-Y_n}) \neq 0. \]
%    次の完全列が成り立つ.
%    \[\xymatrix{ 0 \ar[r]& \Z_{H_{n+1}-Y_n} \ar[r]& \Z_{H_{n+1}} \ar[r]& \Z_{H_{n+1} \cap Y_n} \ar[r]& 0. }\]

\section{$\shO_{\proj^1_k}$ :: acyclic.} %% Ex2.2 

\section{Cohomology with Supports.} %% Ex2.3 
    $X$ :: topological space,
    $Y$ :: closed subset of $X$,
    $U=X-Y$,
    $\shF$ :: sheaf of abelian group on $X$
    とする.
    この時,$\Gamma_Y(X, \shF)$を以下で定める.
    \[
        \Gamma_Y(X, \shF)
        =\{ s \in \Gamma(X, \shF) \mid \Supp s \subseteq Y \}
        =\{ s \in \Gamma(X, \shF) \mid s|_{X-Y}=0 \}.
    \]

    \subsection{$\Gamma_Y(X,-): \catAb(X) \to \catAb$ :: left exact functor.}
    $\catAb(X)$の完全列を考える.
    \[\xymatrix{ 0 \ar[r]& \shF' \ar[r]& \shF \ar[r]& \shF''. }\]
    ここから誘導される以下の列が完全であることはII, Ex1.8で示した.
    \[\xymatrix{ 0 \ar[r]& \Gamma(X,\shF') \ar[r]^-{f}& \Gamma(X,\shF) \ar[r]^-{g}& \Gamma(X,\shF''). }\]
    $\Gamma$を$\Gamma_Y$に付け替えても完全列であることを示す.

    \paragraph{Exact at $\Gamma_Y(X,\shF')$.}
    示すべきことは次のこと.
    \[ \Forall{s \in \Gamma_Y(X,\shF')} f(s)=0 \implies s=0. \]
    $f: \Gamma(X,\shF') \to \Gamma(X,\shF)$が単射なので,
    その制限$f|_{\Gamma_Y(X,\shF')}$も単射.
    よって主張が示せた.

    \paragraph{Exact at $\Gamma_Y(X,\shF)$.}
    示すべきは次のこと.
    \[ \Forall{t \in \Gamma_Y(X,\shF)} g(t)=0 \iff \Exists{u \in \Gamma_Y(X,\shF')} f(u)=t. \]
    $\impliedby$は$g \circ f=0$から直ちに分かる.
    $\implies$は次のように示す.
    元の完全列があるため,$f(u)=t$を満たす$u$が$\Gamma(X,\shF')$からはとれる.
    今$t \in \Gamma_Y(X,\shF)$から$t$は以下を満たす.
    \[ \Forall{P \in U} t_P=(f(u))_P=f_P(u_P)=0. \]
    $f_P$は単射であるから,任意の点$P \in U$について$u_P=0$.
    よって$u \in \Gamma_Y(X,\shF')$.

    \subsection{$\shF'$ :: flasque then
        $0 \to \Gamma_Y(X, \shF') \to \Gamma_Y(X, \shF) \to \Gamma_Y(X, \shF'') \to 0$ is exact.}
    次の完全列は成立する(II, Ex1.16b).
    \[\xymatrix{ 0 \ar[r]& \Gamma(X,\shF') \ar[r]^-{f}& \Gamma(X,\shF) \ar[r]^-{g}& \Gamma(X,\shF'') \ar[r]& 0. }\]
    これらの$\Gamma$を$\Gamma_Y$に取り替えても良いことを示す.
    (a)で示したことを合わせれば,次のことを示せば良いということになる.
    \[ \Forall{t \in \Gamma_Y(X,\shF'')} \Exists{u \in \Gamma_Y(X,\shF)} g(u)=t. \]
    元の完全列から,
    $g(u)=t$を満たす$u$が$\Gamma(X,\shF)$からはとれる.
    今$t \in \Gamma_Y(X,\shF'')$から$t$は以下を満たす.
    \[ t|_U=(g(u))|_U=g|_U(u|_U)=0. \]
    元の完全列があるため,
    $f|_{U}(\tilde{s})=u|_{U}$となる$\tilde{s} \in \Gamma(U,\shF')$がとれる.
    $\shF'$ :: flasqueなので$s|_U=\tilde{s}$となる
    $s \in \Gamma(X, \shF')$が存在する.
    構成から
    \[ f(s)|_U=u|_U \iff \bar{u}:=u-f(s) \in \Gamma_Y(X,\shF). \]
    $g \circ f=0$なので$g(\bar{u})=g(u-f(s))=g(u)=t$.
    よって$\bar{u}$が条件を満たす.

    \subsection{injective sheaves are acyclic for $\Gamma_Y(X, -)$.}
    Prop2.5(injective sheaves are acyclic for $\Gamma(X,-)$)の証明がそのまま使える.
    この証明では(b)の内容の他にはderived functorの性質しか使わない.

    \subsection{$\shF$ :: flasque then
        $0 \to \Gamma_Y(X, \shF) \to \Gamma(X, \shF) \to \Gamma(X-Y, \shF) \to 0$ is exact.}
    引き続き$U:=X-Y$とする.
    $\res_X^U: \Gamma(X, \shF) \to \Gamma(U, \shF)$は$\shF$ :: flasqueゆえに全射.
    この写像のkernelは次のような集合である.
    \[ \ker \res_X^U=\{ s \in \Gamma(X, \shF) \mid s|_U=0 \}. \]
    これは$\Gamma_Y(X, \shF)$に他ならない.

    \subsection{there is a long exact seq ::
        $\cdots \to H^i(X,\shF) \to H^i(X, \shF) \to H^i(U, \shF|_U) \to \cdots$. }
    TODO: injective sheaves are flasqueと(d)を使うと思われる.

    \subsection{$Y \subseteq V$ :: open in $X$ then $H^i_Y(X, \shF) \iso H^i_Y(V, \shF|_V)$.}

\section{Mayer--Vietoris Sequence.} %% Ex2.4 

\section{$H^i_P(X, \shF)=H^i_P(X_P, \shF_P)$.} %% Ex2.5 

\section{$\{\shI_{\alpha} \}$ :: direct system of injective sheaves
    then $\lim \shI_{\alpha}$ :: injective sheaf.} %% Ex2.6 

\section{$H^1(S^1, \Z_{S^1}) \iso \Z$.} %% Ex2.7 

\end{document}
