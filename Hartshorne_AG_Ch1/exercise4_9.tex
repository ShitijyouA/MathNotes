\documentclass[a4paper]{jsarticle}
\usepackage{../math_note, exercise}
\usepackage[all]{xy}

\renewcommand{\thesection}{Ex4.\arabic{section}}
\setcounter{section}{8}

\newcommand{\cond}{\mathcal{C}}

\begin{document}
\section{Suitable Stereographic Projection Gives Birational Map.}
    $X$ :: projective variety in $\proj^n_k$とし,
    $r=\dim X \leq n-2$とする.
    また$H$ :: hyperplane in $\proj^n$とし,
    適宜$\proj^{n-1}$と同一視する.
    適切に点$P \not \in X$をとれば,
    $P$から$H$へのstereographic projection :: $\pi : X \to \proj^{n-1}$が
    $X$と$\pi(X)$の間のbirational mapになることを示す.

    あるmorphismがbirational mapであるかどうかというのはlocalな問題なので,
    $X$のaffine open subsetに絞って考える.
    $X$は射影変換によって
    $(1:\dots:1) \in X$かつ$(1:0:\dots:0) \not \in X$であるように出来るので
    そのようにし,
    $Y:=X \cap (\zerosp(x_0))^c \subseteq \affine^n$
    とおく.
    すると$Y \neq \emptyset, (0,\dots,0) \not \in Y$となる.

    $I=\defsa(Y) \subseteq k[y_1,\dots, y_n]$とし,
    $\bar{y}_i=y_i \bmod I, K:=K(Y)=k(\bar{y}_1,\dots,\bar{y}_n)$とおくと,
    $K=K(X)$.
    Thm4.8より拡大$K/k$はfinitely and separably generated.
    Thm4.7より,$\{\bar{y}_i\}_{i=1}^n$は
    separating transcendence baseを部分集合として含む.
    そこで番号を付け替えて,
    $\{\bar{y}_i\}_{i=1}^n$に含まれる
    separating transcendence baseを$\{\bar{y}_i\}_{i=1}^{r}$としよう.
    baseの濃度が$r(=\dim X=\dim Y)$であることはThm3.2による.
    そして以下の拡大はfinite generated extensionである.
    \[ k(\{\bar{y}_i\}_{i=1}^{n})/k(\{\bar{y}_i\}_{i=1}^{r}). \]
    $J=k(\{\bar{y}_i\}_{i=1}^{r})$とおけばこの拡大は$K/J$と書ける.
    Thm4.6から,この拡大は以下のような元$\eta$で生成することが出来る.
    \[
        \eta=\sum_{i=r+1}^{n} \eta_i x_i
        \mwhere
        \eta_{r+1},\dots,\eta_{n} \in J.
    \]

    stereographic projectionの像 :: 
    $\pi(Y) \subseteq H$のfunction fieldを$L$とする.
    $\pi$から誘導される準同型$\pi^*$を次で定める.
    \begin{defmap}
        \pi^*:& L& \to& K \\
        {}& f& \mapsto& f \circ \pi
    \end{defmap}
    $\pi$は$Q \in Y$を直線 :: $tP+Q$と$H$の交点へ写す写像であった.
    ($P \not \in H$なので$P=1 \cdot P+0 \cdot Q$は予め除いている.)
    したがって$R \in \pi(Y)$をとると$(\pi^* f)(tP+R)$は$t \in k$について定数.
    この値は$f(R)$であるから$\pi^*$は単射である.
    逆に$g \in K$から得られる関数$g(tP+Q)$が$t$について定数ならば,
    $f(R) \ (R \in \pi(Y))$を$g(\pi^{-1}(R))$
    \footnote
    {
        これは$\{ g(tP+R) \mid t \in k, tP+R \in Y \}$に等しい.
        単元集合なので関数$f$を定めることが出来る.
    }と置くことで
    $g=\pi^*f$となる$f \in L$が取れる.
    以上から,$K$の任意の元$g$について次の条件$\cond(g)$が成立すれば$\pi^*$は同型写像と成る:
    任意の$Q \in X$に対し$g(tP+Q)$は$t \in k$について定数である.

    さて,既に分かっている通り
    $K=k(\bar{y}_1,\dots,\bar{y}_r, \eta)$であった.
    なので$\cond(\bar{y}_1),\dots,\cond(\bar{y}_r), \cond(\eta)$の
    全てが成立すれば良い.

    引き続き$Q \in Y$とする.
    $P=(p_1, \dots, p_n), Q=(q_1, \dots, q_n)$とすると
    \[ tP+Q=(tp_1+q_1, \dots, tp_n+q_n). \]
    なので$p_1=\dots=p_r=0$すなわち
    $P \in \zerosa(y_1,\dots,y_r) \subseteq \affine^n$であれば
    $\cond(\bar{y}_1),\dots,\cond(\bar{y}_r)$は成立する.
    以下,$P$はこのようにとる.
    $tP+Q \in X$であるような$t$について$\eta(tP+Q)$は次のように成る.
    \[
        \eta(tP+Q)
        =\sum_{i=r+1}^{n} \eta_i(q_1,\dots,q_r) (tp_i+q_i)
        =\left( \sum_{i=r+1}^{n} \eta_i(q_1,\dots,q_r)p_i \right)t+\left( \sum_{i=r+1}^{n} \eta_i(q_1,\dots,q_r)q_i \right)
    \]
    よって$p_{r+1}=\dots=p_{n}=0$であれば$\cond(\eta)$も成立する.
    結局,$P=(0,\dots,0)$であれば良い.
    最初に$(0,\dots,0) \not \in Y$としていたから,
    これは正しくstereographic projectionを定める.
    このstereographic projectionはもとの射影空間で言うと
    $P=(1:0:\dots:0) \not \in X$から定まるものに一致する.

\end{document}
