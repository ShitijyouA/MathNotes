\documentclass[a4paper]{jsarticle}
\usepackage{../math_note, exercise}
\usepackage[all]{xy}

\newcommand{\bimap}{\xrightarrow{\iso}}
\newcommand{\mon}{\mbox{ on }}
\renewcommand{\thesection}{Ex4.\arabic{section}}

\begin{document}
\section{Term ``is defined" for Regular Functions.} %% Ex4.1 
    $X$::varietyと$\reg{U}{f}, \reg{V}{g} \in K(X)$について,
    $f=g \mon U \cap V$とする.
    このとき,$U$上で$f$,$V$上で$g$であるような写像$h$が
    $U \cup V$上のregular functionであることを示そう.
    つまり,regular functionの定義域を接続する.

    $P \in U \cup V$をとる.
    $P \in U \setminus V$ならば単に$P \in U$と考えて開近傍をとり,$h$は$P$でregularであることが分かる.
    $P \in V \setminus U$でも同様に$h$は$P$でregularであることがわかる.
    $P \in U \cap V$の時は$P$の適当な開近傍上で$f,g$がそれぞれ有理関数表示をとるが,
    $f=g \mon U \cap V$により,その有理関数表示も等しい.
    正確には,$P$の開近傍$Z$で$f=f_n/f_d, g=g_n/g_d$と同時にとれたとすると,
    $f=g$より$f_n g_d-f_d g_n=0 \mon Z$.
    左辺は多項式であり,$Z$は無限集合であるから,左辺は零多項式である.
    なので$h$の$P$近傍での有理関数表示としては$f_n/f_d, g_n/g_d$のいずれをとっても同じである.
    よって$h$は$U \cap V$でwell-defined.

    以上のように$\reg{U}{f}$の定義域を拡大していくと,定義域(開集合)の集合が出来る.
    ネーター空間上で議論しているので,これは極大元を持つ.
    もしも二つ極大な定義域が存在すれば,どちらも$U$を含むので接続が出来る.
    したがって定義域の拡張でできる極大な定義域はただ一つである.

\section{Term ``is defined" Rational Maps.} %% Ex4.2 
    $\phi:U \mand U' \to V$がrational mapであるとする.
    このとき$\phi$は明らかに$U \cup U'$で連続.
    また,regular funciton $f:Z \to k$を任意にとった時,
    $f \circ \phi: \phi^{-1}(Z) \to k$がregularであることは
    $f_1 \circ \phi: \phi^{-1}(Z) \cap U \to k, f_2 \circ \phi: \phi^{-1}(Z) \cap U' \to k$の
    両方がregularであることから明らか.

\section{Example of ``defined"} %% Ex4.3 
    \subsection{Open subset where $f=x_1/x_0$ is defiend.}
    \subsection{}
\section{``Rational"} %% Ex4.4
    $Y$ ::varietyがある$\proj^n$とbirationalであるとき,
    $Y$はrationalであるという.
    同値な条件として,$K(Y)/k$が純超越拡大であるとき$Y$はrationalである.

    \subsection{Any conic in $\proj^2$ is a rational curve.}
    $\proj^2$内の任意のconic curveは$\proj^1$に同型.
    したがってconic curve全体から$\proj^1$全体へのmorphismが存在するのでrationalである.

    \subsection{$C:y^2-x^3=0$ is a rational curve.}
    まず,$C$がパラメータ表示$\gamma(t)=(t^2,t^3)$を持つことを言っておく.
    $U=C \setminus \{(0,0)\}$という$C$の開部分集合をとると,
    パラメータ表示から以下はbirationalである.
    \begin{defmap}
        \phi:& U& \bimap& \proj^1 \\
        {}& (x,y)& \mapsto& (1:t)
    \end{defmap}
    明らかにこれは$U$とアフィン開被覆$U_0$の間のisomorphismであるから$C$はbirational.

    \subsection{Projection of $Y:y^2 z - x^2 (x + z)=0$.}
    $P=(0:0:1)$から$z=0$への射影を$\phi$とする.
    このとき,$\phi(x:y:z)=(x:y)$.
    ここから以下の写像が得られる.
    \begin{align*}
        \bar{\phi}:
            Y \cap U_0 &\to \{(1:s) ~|~ s^2 \neq 1\} \subset \proj^1 \\
            (1:s:t) &\mapsto (1:s) \\
            \left( 1:s:\frac{1}{s^2-1} \right) &\mapedfrom (1:s)\\
    \end{align*}
    $\{(1:s) ~|~ s \neq 1\}=U_0 \cap (\zerosp(x^2-y^2))^c$は開集合である.
    また,像,原像ともにaffineであるから,
    Lemma 3.6によって$\bar{\phi}, \bar{\phi}^{-1}$の両方がmorphismであることが分かる.
    よってこれはbirational map.

\section{$Q:xy-zw=0$ is birational to $\proj^2$ but not isomorphic.} %% Ex4.5 
    \paragraph{$Q$ is birational to $\proj^2$.}
    $Q_3=Q \cap U_3$を考えると,
    \[ \phi:(x:y:z:w) \mapsto \left( \frac{x}{w}:\frac{y}{w}:\frac{z}{w} \right) \]
    という写像が得られる.
    これは直ちに逆写像が得られるので,birational map $\phi: Q \cap U_3 \bimap \proj^2$が得られた.

    \paragraph{$Q$ is not isomorphic to $\proj^2$.}
    Ex3.7より,$\proj^2$の任意の曲線は交わる.
    しかしEx2.15より二つの直線$L_t, L_u (t \neq u)$は交わらない.
    よって$Q$と$\proj^2$は同相でなく,したがって同型でもない.

\section{Plane Cremona Transformations.} %% Ex4.6 
    $\proj^2$から自分自身へのbirational mapはplane Cremona transformationと呼ばれる.
    Quadratic transformation.
    \[ \phi: \proj^2 \to \proj^2;~~ (a_0:a_1:a_2) \mapsto (a_1 a_2:a_0 a_2:a_0 a_1) \]
    ここで$a_0,a_1,a_2$のいずれか二つは0でない.

    \subsection{$\phi$ itself is its inverse as a rational map.}
    $\phi$を2回適用する.
    \[ (a_0:a_1:a_2) \mapsto (a_1 a_2:a_0 a_2:a_0 a_1) \mapsto (a_0^2 a_1 a_2:a_0 a_1^2 a_2:a_0 a_1 a_2^2)=(a_0:a_1:a_2) \]
    したがって$\phi$は$U=(\zerosp(x_0 x_1 x_2))^c$から$U$自身へのisomorphismである.
    定義域がこれ以上拡大出来ないことは明らか.

    \subsection{Find $U,V \in \proj^2 \mwhere U \overset{\phi}{\iso} V$}
    すでに述べた.

    \subsection{Find opensets where $\phi$ and $\phi^{-1}$ are defiend.}
    すでに述べた.

\section{$\mathcal{O}_{P,X} \iso \mathcal{O}_{Q,Y}$ $\implies$ $\Exists{\psi} \psi:X \bimap Y; P \mapsto Q$} %% Ex4.7 
    \paragraph{We can assume $X,Y$::affine.}
    Prop4.3より,$P \in Z \subset X$なる$Z$::affine open subsetが存在する.
    このとき$\mathcal{O}_{P,X}$と$\mathcal{O}_{P,Z}$は
    $\reg{U}{f} \mapsto \reg{U \cap Z}{f}; \reg{V}{g} \mapedfrom \reg{V}{g}$なる写像で同型である.
    なので$X,Y$::affineと仮定して良い.

    \paragraph{Make $\phi_{\ast}$ and $\psi$.}
    今,仮定から$\phi: A(X)_{\I{m}_P} \isomap A(Y)_{\I{m}_Q}$なる同型写像が存在する.
    同型の両辺で$\Quot$を取ることで$\phi_{\ast}:K(X) \isomap K(Y)$ ::isomorphismが得られる.
    \footnote{$\phi_{\ast}(a/s)=\frac{\phi(a/1)}{\phi(s/1)}$とすれば良い.Thm3.2の議論とAti-Mac Ex3.3を参照.}
    $\phi_{\ast}(x_i+\defs(X)) \in \Quot(A(Y))=K(Y)$は有理関数であるから,
    以下の写像が定義できる開集合$U \subset Y$が存在する.
    \[ \psi: U \to X;~ S \mapsto (\phi_{\ast}(x_0+\defs(X))(S), \dots, \phi_{\ast}(x_n+\defs(X))(S)) \]
    逆写像も同様に作れるため,これはbirational mapである.
    $f \in A(X)_{\I{m}_P}$についてあきらかに$\phi_{\ast} \circ f=f \circ \psi$.
    \footnote{実際は$(a/s)(P)=0 \iff a(P)=0$なので$f \in A(X)$についてのみこの等式を言えば良い.}

    \paragraph{Paraphrasing of $\psi^{-1}: P \mapsto Q$.}
    $\phi_{\ast}$によって極大イデアル$\bar{\I{m}}_P \subset A(X)_{\I{m}_P} \subset K(X)$は
    極大イデアル$\bar{\I{m}}_Q \subset A(Y)_{\bar{\I{m}}_Q} \subset K(Y)$に写され,
    $\zerosa(\bar{\I{m}}_Q)=\zerosa(\phi_{\ast}(\bar{\I{m}}_P))=\{Q\}$.
    $\phi_{\ast} \circ f=f \circ \psi$から以下のように$\psi^{-1}: P \mapsto Q$が得られる.
    \begin{align*}
        {}&     Q \in \zerosa(\phi_{\ast}(\bar{\I{m}}_P))=\zerosa(\bar{\I{m}}_Q) \\
        \iff&   \Forall{f \in \bar{\I{m}}_P} \phi_{\ast}(f)(Q)=0 \\
        \iff&   \Forall{f \in \bar{\I{m}}_P} f(\psi(Q))=0 \\
        \iff&   \psi(Q) \in \zerosa(\bar{\I{m}}_P)=\{P\} \\
        \iff&   \psi^{-1}(P)=Q
    \end{align*}
    なお,証明には全て$\implies$で十分.

\section{Cardinality and Homeomorphism of Curves} %% Ex4.8
    \subsection*{Lemma}
    念の為に以下を証明しておく.
    \begin{Lemma}
        体$k$の代数閉包を$\bar{k}$とする.
        $k$が有限体ならば$|\bar{k}|=\aleph_0$であり,
        $k$が無限体ならば$|\bar{k}|=|k|$である.
    \end{Lemma}
    \begin{proof}
        $k$上の$n$次多項式は次のように$k^n$の元と一対一対応する.
        \[ x^n+c_{n-1} x^{n-1}+\dots+c_0 \leftrightarrow (c_{n-1}, \dots, c_0) \]
        $n$次多項式の根は高々$n$個だから,以下のように濃度が計算できる.
        \[
            |\bar{k}|
            \leq \sum_{n \in \N}{n |k^n|}
            =\sum_{n \in \N}{n |k|^n}
            =|\{(i,j,x) ~|~ i \in \N, 1 \leq j \leq i, x \in k^i \}|
        \]
        以降は$|k|$が有限かどうかで計算が変わる.
        \paragraph{Case: $|k| < \aleph_0$}
            $|k|$が有限ならば$n |k|^n$も有限なので
            \[ |\bar{k}| \leq |\{(i,j) ~|~ i \in \N, 0 \leq j \leq i |k|^i \}| \leq |\N \times \N|=\aleph_0. \]
            任意の自然数$d \in \N$に対して$d$次既約多項式が存在することが知られているので$|\bar{k}| \geq \aleph_0$.
            よって$|k|=\aleph_0$.

        \paragraph{Case: $|k| \geq \aleph_0$}
            $|k|$が無限ならば$n|k|^n=n|k|=|k|$なので
            \footnote{無限濃度$\kappa$について$\kappa^2=\kappa$は選択公理と同値.}
            \[ |\bar{k}| \leq |\N \times k| \leq |k^2|=|k|. \]
            $k \subseteq \bar{k}$から$|k| \leq |\bar{k}|$なので証明が完成した.
    \end{proof}
    \subsection{For any variety $X$ whose dimention $\geq 1$, $|X|=|k|$.}
    $k=\bar{k}$とする.

    \paragraph{$|\affine^n|=|\proj^n|=|k|$}
    $\affine^n=k^n$なので$|k|$が無限濃度であることと合わせて$|\affine^n|=|k|$.
    また,$\proj^n=(\affine^{n+1} \setminus \{O\})/\sim$なので,$|\proj^n| \leq |\affine^{n+1} \setminus \{O\}|=|k|$.
    $\affine^n \equiv U_0 \subset \proj^n$を考えて$|\proj^n| \geq |k|$.
    まとめて$|\affine^n|=|\proj^n|=|k|$.

    \paragraph{Start of step I: case of $\dim X=1$.}
    $\dim X=1$の時,
    $X$::variety, $\dim X=1$を考える.
    Prop 4.9より,$X$からhypersurface $H (\subset \proj^{2})$へのbirational mapが存在する.
    $H$の定義多項式を$h$としておこう.

    \paragraph{$\phi$ is surjective.}
    一次斉次多項式$F$を,
    $H \cup \zerosp(F)=\zerosp(\langle h, F \rangle)$が$\proj^{2}$全体でないように取る.
    すると$P \not \in H \cup \zerosp(F)$を適当に取ることができる.
    この点$P$からの射影$\phi: \proj^2 \setminus P \to \zerosp(F)$を考えよう.
    明らかに$\phi(H) \subseteq \zerosp(F)=\proj^1$かつ$|\zerosp(F)|=|\proj^1|=|k|$である.
    これが全射であることを示せば$|H| \geq |k|$が分かる.
    $R \in \zerosp(F)$を任意にとり,$P$と$R$を通る直線を$L(P,R)$としよう.
    Ex3.7(a)より,$L(P,R) \cap H \neq \emptyset$(ここで$\dim H=1$を用いる).
    したがって$Q \in L(P,R) \cap H$を取ることができて,
    構成法から\footnote{つまり$P,Q,R$が一直線上にあり,$Q \in H, R \in \zerosp(F)$だから.}$\phi(Q)=R$が成立する.
    よって$\phi$は全射.
    \footnote{$L(P,R) \cap H$が有限集合であることは,
    $M(R;t)=P+tR$とすると$h(M(R;t))$が$t$の1変数多項式であり,
    したがって根は高々$\deg h$個であることから得られる.}

    \paragraph{Conclusion of step I}
    以上で$|H| \geq |k|$が示された.
    $H \subset \proj^2$より$|H| \leq |k|$なので$|H|=|k|$.
    さて,$X$と$H$はbirationalなので,
    2つのある開集合$U \subset X, V \subset H$の間に全単射が存在する.
    $H$が1次元であることから$V$は$H$から有限個の点を除いたものであり,したがって$|V|=|H|=|k|$.
    \footnote{$H$に含まれる既約閉集合は1点のみであり,$H$の任意の閉集合はProp1.5から有限個の点である.}
    あわせて$|X| \geq |U|=|V|=|k|$.
    $X \subset \proj^n$より$|X| \leq |k|$なので,Case Iの証明が終わった.

    \paragraph{Next Step}
    $\dim X \geq 2$ならば,次元の定義より,$X$は1次元の既約閉集合$C$を含む.
    したがってCase Iより$|X| \geq |C|=|k|$.
    $|X| \leq |\proj^n|=|k|$なので一般の次元でも証明が得られた.

    \subsection{Any two curves over $k$ are homeomorphic.}
    Ex3.1dから$\affine^2 \not \iso \proj^2$であることに注意.

    二つの曲線$C,D$をとろう.
    (a)の結果から$|C|=|k|=|D|$なので全単射$\phi:C \to D$が存在する.
    $C,D$は1次元なので$C,D$上の閉集合は空集合,有限個の点,曲線全体しかない.
    明らかに$\phi$は空集合,点,曲線全体をそれぞれ空集合,点,曲線全体に写すので,
    同相写像である.

    \subsection{Another proof for $|\text{any curve}| \geq |k|$}
    任意の曲線$C$を取ると,これはhypersurface $H \subset \proj^2$とbirational.
    $H$の定義多項式を$h \in k[x,y,z]$とする.
    このとき,写像$\iota: k \to H$が構成できる.
    これは単純に$f(1,a,z) \in k[z]$の零点$z=b$を一つ選び\footnote{選択公理を用いる.}
    $\iota: a \mapsto (1:a:b)$とすれば良い.
    代数閉体で考えているので零点は必ず有限個存在する.
    これは明らかに単射だから$|C| \geq |k|$.

\section{Stereographic projection can induce a birational morphism.} %% Ex4.9
    別のPDFファイル``exercise4\_9.pdf"に解答を書いた.

\section{Blowing up of $C:y^2-x^3=0$ at $O=(0,0)$.} %% Ex4.10
    $V_0=\affine^2 \times U_0, V_1=\affine^2 \times U_1$とおく.
    これらはそれぞれ$\affine^2 \times \affine^1$とみなすことが出来る.

    \paragraph{Blowing up to $V_0$.}
    $C$の$V_0$へのblow-upは,$y^2-x^3=0, y=xu$の連立方程式を解くことで得られる.
    計算すると$x^2(u^2-x)=0$.
    よって$E_0=(0,0) \times (1:u), \tilde{C}_0=\zerosa(u^2-x) \subset V_0$.
    $E_0 \cap \tilde{C}_0=(0,0) \times (1:0)$が得られる.
    また,$\tilde{C}_0$は$u \mapsto (u^2,u^3) \times (1:u)$により$\affine^1$と同型である.

    \paragraph{Blowing up to $V_1$.}
    同様に$y^2-x^3=0, x=ty$を解いて$y^2 (1-t^3 y)=0$.
    よって$E_1=(0,0) \times (t:1), \tilde{C}_1=\zerosa(1-t^3 y) \subset V_1$となる.
    $E_1 \cap \tilde{C}_1$は空である.
    また,$\tilde{C}_1$は$t \mapsto (t^2,t^3) \times (1:t)$により$\affine^1$と同型である.

    \paragraph{Summarize.}
    \begin{align*}
        E &= O \times \proj^1 \\
        \tilde{C} &=\{ (t^2,t^3) \times (1:t) ~|~ t \in k \}
    \end{align*}
    $\tilde{C}$は直ちに$\affine^3$の曲線と見ることが出来る. \\
    gnuplotでのコードは\texttt{set parametric; splot u**2, u**3, u}なので試すと良い.

\end{document}
