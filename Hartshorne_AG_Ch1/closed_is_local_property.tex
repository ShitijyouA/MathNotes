\documentclass{jsarticle}
\usepackage{../math_note}

\begin{document}
    \begin{Prop}
        位相空間$X$とその部分集合$Y$を考える.
        ある$X$の被覆$\{ U_{\lambda} \}_{\lambda \in \Lambda}$であって,
        任意の$\lambda$で$Y \cap U_{\lambda}$が$U_{\lambda}$において閉であるようなものが有るとき,
        $Y$が閉であることを示せ.
    \end{Prop}
    一言で言えば,この命題は「閉集合である,ということは局所的な性質である」ということである.

    \begin{proof}
    対偶を示す.
    $Y$が閉でなければ,$x \not \in Y$かつ$x \in \cl_{X}(Y)$なる点$x$が存在する.
    すると$\{ U_{\lambda} \}_{\lambda \in \Lambda}$は$X$の被覆だから,
    この中に$x$の開近傍が少なくともひとつ存在する.それを$U$としよう.
    $U \cap Y$が$U$において閉集合でないことを示す.
    $x \not \in Y$から,$x \not \in U \cap Y$が得られる.
    あとは$x \in \cl_{U}(U \cap Y)$が得られれば証明は終わる.

    まず,$X$での閉集合$Z$を用いて$\cl_{U}(U \cap Y)=U \cap Z$と書く.
    $x \not \in \cl_{U}(U \cap Y)=U \cap Z$と仮定すると,$x \in U$なので$x \not \in Z$が得られる.
    すると$U \cap (X \setminus Z)$は$x$の開近傍となる.
    $x \not \in Y$かつ$x \in \cl_{X}(Y)$から$x$は$Y$の集積点だから$U \cap (X \setminus Z)$は($x$と異なる)$Y$の点$y$も含む.
    $y \in Y$かつ$y \in U \cap (X \setminus Z)$なので$y \in U \cap Y$かつ$y \not \in Z$.
    しかし$y \in U \cap Y \subseteq U \cap Z$なので$y \in U \cap Z$.
    したがって$x \not \in \cl_{U}(U \cap Y)$とすると$y \not \in Z$と$y \in Z$が同時に得られ,矛盾.
    よって$x \in \cl_{U}(U \cap Y)$かつ$x \not \in U \cap Y$が示され,
    $U \cap Y$が$U$における閉集合では無いことが示される.
    \end{proof}
    \begin{proof}
        \begin{align*}
            {}&         Y \cap U_{\lambda} \text{ :: closed in } U_{\lambda} \\
            \iff&       Y^c \cap U_{\lambda} \text{ :: open in } U_{\lambda} \\ 
            \iff&       Y^c \cap U_{\lambda} \text{ :: open in } X \\ 
            \implies&   {\textstyle \bigcup_{\lambda}{(Y^c \cap U_{\lambda})}} \text{ :: open in } X \\ 
            \iff&       Y^c \text{ :: open in } X \\ 
            \iff&       Y \text{ :: closed in } X
        \end{align*}
        ただし2つめの$\iff$は$U_{\lambda}$::open in $X$から.
    \end{proof}
\end{document}
