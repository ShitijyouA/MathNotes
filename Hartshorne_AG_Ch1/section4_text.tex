\documentclass[a4paper]{jarticle}
\usepackage{../math_note, braket}

\begin{document}
\section{Lemma 4.1}
    \subsection{``We may assume that $Y \subset \proj^n$ for some $n$"}
    affine, quasi-affineは$(a_1,\dots,a_n) \mapsto (1,a_1,\dots,a_n)$
    という写像で$\proj^n$へ埋め込める.
    この写像が同相写像であることはProp 2.2で示されているので,
    irreducibleは反映される.

    \subsection{$\Delta=\zerosp(\{ x_i y_j=x_j y_i \})$}
    $P, Q \in \proj^n$が$P=Q$を満たすとき,以下が成り立つ.
    \[ \Exists{\lambda \in k^{\times}} \Forall{i} p_i=\lambda q_i \]
    なので$\lambda=p_j/q_j$とすれば,分母を払って$p_i q_j=p_j q_i$が得られる.
    最後の式は$q_j=0$でも成立する.

    \subsection{$(\phi \times \psi)(X) \subseteq \Delta$}
    $q=\phi \times \psi$とすると,これはmorphismなので連続.
    $q(U) \subseteq \Delta$なので$U \subset q^{-1}(\Delta)$が成り立ち,
    $\Delta$が閉なので$q^{-1}(\Delta)$も閉.
    $U$は$X$でdenseなので
    \[ X=\cl_X(U) \subseteq \cl_X(q^{-1}(\Delta))=q^{-1}(\Delta). \]
    よって$q(X)=(\phi \times \psi)(X) \subseteq \Delta$.

\section{$Y (\subseteq X)$ :: open affine subset, $K(X) \cong K(Y)$.}
    以下の同型写像がある.
    \begin{defmap}
        \kappa:& K(X)& \to& K(Y) \\ 
        {}& \reg{U}{f}& \mapsto& \reg{U \cap Y}{f} \\
        {}& \reg{V}{g}& \mapedfrom& \reg{V}{g}
    \end{defmap}
    これが同型写像であることは,$X$がirreducibleであることから来ている.
    $Y$が何らかのaffine varietyと同型であることから,
    $K(Y)$はThem3.2に従う.

\end{document}
