\documentclass[a4paper]{jsarticle}
\usepackage{../math_note, exercise}
\usepackage[all]{xy}

\newcommand{\mon}{\mbox{ on }}
\newcommand{\Sing}{\operatorname{Sing}}
\newcommand{\regs}{\mathcal{O}}
\newcommand{\Aut}{\operatorname{Aut}}
\renewcommand{\thesection}{Ex6.\arabic{section}}

\begin{document}
\section{Nonsingular Curve which Is Birational but Not Isomorphic to $\proj^1$.} %% Ex6.1 
    $Y$を$\proj^1$と同型でないnonsingular rational curveとする.

    \subsection{$Y$ is isomorphic to an open subset of $\affine^1$.}
    Cor6.12を用いる.
    仮定より,$K(Y) \cong K(\proj^1)$.
    $Y$があるnonsingular projective curveと同型ならそれは$\proj^1$と同型になってしまう
    \footnote{Cor6.12の(iii)$\simeq$(i).}
    したがって$Y$はあるnonsingular projective curve $\bar{Y}$の真の開部分集合と同型である.
    \footnote{Cor6.12の(iii)$\simeq$(ii).}
    再びCor6.12によって$\bar{Y} \iso \proj^1$なので\footnote{Cor6.12の(iii)$\simeq$(i).},
    結局$Y$は$\proj^1$の真の開部分集合と同型である.

    $\proj^1$の真の開部分集合$U$は$\affine^1$の開部分集合と同型であることを示す.
    $(a:b) \in \proj^1 \setminus U$と,$(a:b)$と異なる点$(c:d)$を$ad-bc \neq 0$であるように取る.
    \begin{defmap}
        f:& U& \to& \proj^1-\{(1:0)\} \\ 
        {}& (x:y)& \mapsto& (dx-cy:bx-ay)=(u:v)
    \end{defmap}
    $ad-bc \neq 0$なのでこれは同型写像である.
    同型写像は同相写像であり,かつ
    $\im f \subseteq (\zerosp(v))^c \iso \affine^1$なので,
    $U$は$\affine^1$の開部分集合と同型である.

    \subsection{$Y$ is affine.}
    (a)より$Y \iso \affine^1 \setminus \{P_1,\dots,P_s\}$となる$\{P_1,\dots,P_s\}$が存在する.
    この時$f=(x-P_1)\cdots(x-P_s)$とすれば,$Y \iso \affine^1 \setminus \zerosa(f)$.
    なのでLemma4.2より$Y$はaffine.

    \subsection{$A(Y)$ is UFD.}
    $\affine^2$のcoordinate variableを$x,y$とする.
    Lemma4.2より,$A(Y)=k[x]_f$.
    Ati-Mac Prop3.11より,局所化$k[x]_f$のイデアルはすべて拡大イデアルである.
    すなわち,$k[x]_f$の任意のイデアル$\I{a}$に対して$k[x]_f$のイデアル$\I{a}'$が存在し,
    $\I{a}$は$\I{a}'$の元を$x \mapsto x/1$で写したもので生成される.
    $k[x]$はPIDだから,$k[x]_f$もPID.
    PIDならばUFDであることは森田『代数概論』にもある.

\section{An Elliptic Curve} %% Ex6.2 
    $f=y^2-x^3+x$を考えよう.
    $Y=\zerosa(f) \subset \affine^2$とし,
    考える体$k$の標数は2でないとする.
    また,$K=K(Y), A=A(Y), \bar{x}=x+(f), \bar{y}=y+(f)$とする.

    \subsection{$Y$ :: nonsingular, and $A$ :: integrally closed domain.}
    $Y$がnonsingular affine curveであることは次の連立方程式が解を持たないことと同値である.
    \[ y^2-x^3+x=-3x^2+1=2y=0. \]
    $\fchar k \neq 2$なので,これは以下と同値.
    \[ x(x+1)(x-1)=-3x^2+1=0. \]
    これは明らかに解を持たない.

    さらにaffine curve $Y$については以下のような同値な命題の列があるので,
    $A$がintegrally closed domainであることがわかる.
    \begin{align*}
        {}&     Y \text{ :: nonsingular affine curve} \\
        \iff&   \Forall{P \in Y} \mathcal{O}_{P,Y} \text{ :: regular local ring} \\
        \iff&   \Forall{P \in Y} \mathcal{O}_{P,Y} \text{ :: integrally closed domain} \\
        \iff&   Y \text{ :: normal affine curve} \\
        \iff&   A (=A(Y)) \text{ :: integrally closed domain}
    \end{align*}
    Them 5.1, Them6.2, Ex3.17dを用いた.
    
    \subsection{$k[\bar{x}]$ :: polynomial ring, and $A$ is the integral closure of $k[\bar{x}]$ in $K$.}
    $\bar{x}$が$k$上超越的であることを示そう.
    仮に超越的でない,すなわち代数的であるとすると,多項式$p(X) \in k[X]$が存在して$p(\bar{x})=0$となる.
    これは$p(x) \in (f)$と同値.したがって$\zerosa(p) \subseteq \zerosa(f)=Y$と同値である.
    仮に$p(x)$の$k$における解の1つを$\alpha$とすると,$p(x)$が$y$を含まないので,以下が成り立つことが必要.
    \[ \zerosa(x-\alpha) \subseteq Y. \]
    なので$y^2-\alpha^3+\alpha=0$が恒等式になる.しかしこれは不可能.
    よって$\bar{x}$は$k$上超越的であり,したがって$k[\bar{x}]$はpolynomial ringである.

    $k[\bar{x}]$はpolynomial ringだから,$k[\bar{x}]$は$Y$の任意の局所環に含まれる.
    Ati-Mac Cor5.22より$k[\bar{x}]$のintegral closureは$k[\bar{x}]$を含むすべての付値環の共通集合.
    $Y$はnonsingularであるからThem6.2より$Y$の任意の局所環は付置環である.
    したがって以下が言える.
    \[ \text{the integral closure of }k[\bar{x}] = \bigcap_{P \in Y} A_{\I{m}_P}=\bigcap_{\I{m} \in \Max(A)} A_{\I{m}}. \]
    最右辺はThem3.2の証明で述べられているように$A$に等しいから,$k[\bar{x}]$のintegral closureは$A$.

    \subsection{Properties of the Norm.}
    $A$の自己準同型$\sigma$を$\bar{x} \mapsto \bar{x}; \bar{y} \mapsto -\bar{y}$で定義する.
    $\sigma$は明らかに$\sigma^2=\id{}$で,巡回群をなす.
    これを用いてnorm $N$を$a \in A \mapsto a \cdot \sigma(a)$と定義する.
    これは体拡大$K/k(\bar{x})$のnorm(対論で用いられる)である.

    $\sigma$でfixされる元がなす$A$の部分環を考える.
    $\sigma(\bar{x}^m \bar{y}^n)=(-1)^n \bar{x}^m \bar{y}^n$と$-1 \neq 1$より,
    $\bar{y}$の次数が偶数であるような単項式がfixされる.
    よって,
    \[ A^{\langle \sigma \rangle}=k[\bar{x}, \bar{y}^2]=k[\bar{x}, \bar{x}^3-\bar{x}]=k[\bar{x}]. \]

    $\sigma(N(a))=\sigma(a) \cdot \sigma^2(a)=N(a)$より,$\im N \subset k[\bar{x}]$.
    また$N(1)=1 \cdot \sigma(1)=1$,$N(ab)=ab \cdot \sigma(a) \sigma(b)=N(a)N(b)$が成り立つ.

    \subsection{The units of $A$ $=k^{\times}$, and $\bar{x}, \bar{y}$ :: irreducible elements.}
    単元$u \in A^{\times}$をとる.
    (b)で示したことから,以下がわかる.
    \[ 1=N(u u^{-1})=N(u) N(u^{-1}) \in k[\bar{x}]. \]
    したがって$N(u)$は$k[\bar{x}]$の単元であるが,
    すでに示したとおり$k[\bar{x}]$はpolynomial ringなので$N(u)=u \cdot \sigma(u) \in k^{\times}$.
    $\zerosa(N(u))$を考えると,
    \[ \zerosa(N(u))=\zerosa(u) \cup \zerosa(\sigma(u))=\emptyset. \]
    なので$\zerosa(u)=\emptyset$であり,したがって$u \in k^{\times}$.

    $\bar{x}$がirreducibleでないと仮定しよう.
    すると$\bar{x}=u v$となる非単元$u,v \in A \setminus k^{\times}$がある.
    \[ u=\alpha \bar{x}+\beta \bar{y} \mwhere \alpha,\beta \in A \]
    とおこう.
    すると$(\alpha v-1) \bar{x}+\beta v \bar{y}=0$が成り立つ.
    $v \in A$は非単元だから$\alpha v \neq 1$.
    よって$\bar{x}=\frac{\beta v}{1- \alpha v} \bar{y}$となる.
    このことから$Y$は$y$軸全体を含むか,または$x$軸との交点は$(0,0)$のみとなるか,どちらかになる.
    しかし実際はどちらでもなく,矛盾.
    よって$\bar{x}$はirreducible.

    $\bar{y}$がirreducibleでないと仮定すると,
    同様にして以下にできる.
    \[ \bar{y}=\omega \bar{x} \mwhere \omega \in k(\bar{x}, \bar{y}). \]
    これを$f$に代入すると,$\bar{x}(\bar{x}^2+\omega^2 \bar{x}-1)=0$が得られる.
    $Y$は$y$軸全体($=\zerosa(x)$)を含まないので
    \[ \bar{x}^2+\omega^2 \bar{x}-1=0. \]
    これは$Y$上の任意の点で成り立つ方程式であるから$Y$は$x=0$となる点を持たない.
    しかし実際は$(0,0) \in Y$なので矛盾.

%    また$\beta=0$とすると$\frac{u}{\bar{x}} \in A, 1=\frac{u}{\bar{x}} v$となり,
%    これも$v$が非単元であることに反するから$\beta \neq 0$.
%    $\bar{x}=\frac{\beta v}{1- \alpha v} \bar{y}$かつ$\frac{\beta v}{1- \alpha v} \neq 0$となるので,
%    $Y$と$x$軸,あるいは$y$軸の交点は$(0,0)$のみであるということになる.

    \subsection{$Y$ :: not rational curve.}
    $A$において,以下の等式が成り立つ.
    \[ \bar{y}^2=\bar{y} \cdot \bar{y}=\bar{x} \cdot (\bar{x}-1) \cdot (\bar{x}-1). \]
    $\bar{x}, \bar{y}$は既約元であるから,これは$\bar{y}^2$に2つの既約元分解を与えている.
    よって$A=A(Y)$はUFDでなく,同時に$Y$は明らかに$\proj^1$と同型でない.
    これらのことからEx6.1cより$Y$はrationalでない.

\section{Give counterexample to Prop6.8.} %% Ex6.3 
%Prop6.8
%Let X be an abstract nonsingular curve, let P \in X, let Y be
%a projective variety, and let \phi:X - P \to Y be a morphism. Then there
%exists a unique morphism \phi: X \to Y extending \phi.
    \paragraph{The Extension of Birational map between Nonsingular Projective Curves is Isomorphism.}
    この問題とEx6.7で用いるので明確に述べておく.
    Cor6.12より,互いにbirationalなnonsingular projective curveは同型である.

    これはProp6.8から直接示すこともできる.
    $X,Y$ :: nonsingular projective curves, $\phi: X \bimap Y$ :: birational mapとする.
    Cor4.5から$U$ :: open in $X$, $V$ :: open in $Y$が存在して
    isomorphism $\phi|_U^V: U \isomap V$ができる.
    Prop6.8から,$\phi$には拡張が一意に存在する.
    これをvarietyとmorphismの圏における可換図式にすると以下のようになる.
    \[
    \xymatrix
    {
    {} & U \ar@{>->}[dl] \ar[dr]^-{\phi|_U} \ar@{<->}^-{\phi|_U^V}[rr] & {} & V \ar@{>->}[dl] \ar[dr]^-{(\phi^{-1})|_V} & {} \\
    X \ar@{-->}_-{\overline{\phi}}[rr] & {} & Y \ar@{-->}_-{\overline{\phi^{-1}}}[rr]& {} & X
    }
    \]
    すると$U$から右下の$X$へ向かう2つのパスが可換であることから,以下が成り立つ.
    \[ (\overline{\phi^{-1}} \circ \overline{\phi})|_U=(\phi^{-1})|_V \circ \phi|_U^V=\id{U}. \]
    Lemma4.1から$\overline{\phi^{-1}} \circ \overline{\phi}=\id{X}$となる.
    $\overline{\phi} \circ \overline{\phi^{-1}}=\id{Y}$も同様に示すことができるので,
    $\overline{\phi}$がisomorphismになることがわかった.

    \paragraph{If $\dim X \geq 2$.}
    自然数$n \geq 3$を固定する.
    $f=x_0^2-\left( \sum_{i=1}^n x_i^2 \right), X=\zerosp(f), P=(1:0:\dots:0:1)$とする.
    この時,$\dim X=n-1 \geq 2, P \in X$である.
    まず$X-P$から$H=\zerosp(x_n)$へのstereographic projectionを$\phi$とすると,これは全単射になる.
    \begin{defmap}
        \phi:& X-P& \to& H \\ 
        {}& (a_0:\dots:a_n)& \mapsto& (a_0-a_n:a_1:\dots:a_{n-1}:0) \\
        {}& \left( 2b_0b_0-\alpha: 2b_0b_1: \dots: 2b_0b_{n-1}: -\alpha\right)& \mapedfrom& (b_0:\dots:b_{n-1}:0)
    \end{defmap}
    ただし$\alpha=f(b_0,\dots,b_{n-1},0)$とした.
    Prop6.8がこの場合にも成立したと仮定しよう.
    すると拡張$\bar{\phi}:X \to H$は全単射である.
    しかし$\phi$がすでに全単射なので,$P$の$\bar{\phi}$による像は別の点$P'$の像でもある.
    すなわち$\bar{\phi}^{-1}(\bar{\phi}(P))$は二点集合$\{P,P'\}$となる.
    これは$\bar{\phi}$が単射であることに反する.

    \paragraph{If $Y$ :: not projective variety.}
    $X=\proj^1, P=(1:0), Y=\affine^1$としよう.
    すると$X-P$と$Y$には$(a:b) \mapsto a/b \mapsto (a/b:1)$という標準的な全単射が存在する.
    Prop6.8がこの場合にも成立したと仮定すると,前段落と同様に矛盾が生じる.

%    これを拡張したものを$\bar{\phi}$とすると,Lemma 3.6より,これはひとつのregular functionで表せる.
%    regular function$\bar{\phi}$は開集合$X-P$上ではもとの全単射と一致しているから,
%    Remark3.1.1より,$\bar{\phi}$は$X$全体で$(a:b) \mapsto a/b$と一致する.
%    しかし明らかにこれは$P$で定義できない.
%    よって矛盾が生じ,拡張は存在しない.

\section{Make surjective morphism $\phi: Y \to \proj^1$ from nonconstant rational function.} %% Ex6.4 
    $Y$ :: nonsingular projective curveとする.
    $Y$上の任意の定数でないrational function $f=g/h$に対して以下のように写像を定める.
    \begin{defmap}
        \phi:& Y& \to& \proj^1 \\ 
        {}& P& \mapsto& (1:f(P))=(h(P):g(P))
    \end{defmap}

    多項式に多項式を代入したものは多項式だから,これがmorphismであることは自明.

    surjectiveであることを示そう.
    $(a:b) \in \proj^1$を任意に取る.
    $\phi(P)=(h(P):g(P))=(a:b)$とすると,
    \[ (ag-bh)(P)=0. \]
    となる.
    これを満たす$P$全体が$\phi^{-1}(a:b)$であり,
    $ag-bh$は多項式であることから$\phi^{-1}(a:b)$は高々有限集合である.
    $\phi^{-1}(a:b)$が空でないことは次のようにわかる.
    $\phi^{-1}(a:b)$が空であるとき$ag-bh$は0でない定数$c$である.
    $(a:b) \in \proj^1$なので$a \neq 0$と仮定すると\[ g=(b/a)h+c. \]
    $g$は斉次であるから$\deg (b/a)h=\deg c=0$.
    しかしそうなると$\deg g=\deg h=0$となり,$f=g/h$は定数であることになる.
    これは$f$のとり方に矛盾する.

\section{Subvariety which is nonsingular projective curve is closed subset.} %% Ex6.5 
    $X,Y$ :: (quasi-projective) variety, $X \subset Y \subset \proj^n$とする.
    $X$ :: nonsingular projective curveであるときに$X$ :: closed in $Y$であることを示そう.
    $\bar{X}=\cl_{Y}(X)$とすると,これはnonsingular projective curve.
    したがって,$X$は$\bar{X}$の開集合である.
    $\bar{X}=X$を示そう.

    埋め込み$i: X \to \bar{X}$は$\bar{i}: \bar{X} \to \bar{X}$へ拡張できる.
    しかし$X$は予めprojective curveなので$i$は最初からmorphismで,$i=\bar{i}$.
    よって$X=\im i^{-1}=\im \bar{i}^{-1}=\bar{X}$.

\section{Automorphisms of $\proj^1$.} %% Ex6.6 
    $\proj^1=\affine^1+\{\infty\}$を考える.
    Fractional linear transformation of $\proj^1$を以下のような写像と定める.
    \[ x \mapsto \frac{ax+b}{cx+d} \mwhere a,b,c,d \in k, ad-bc \neq 0. \]
    Fractional linear transformation of $\proj^1$全体を$PGL(1)$と書く.

    \subsection{$(ax+b)/(cx+d) \in PGL(1)$ induces an automorphism of $\proj^1$.}
    $\frac{ax+b}{cx+d}$の逆写像は$\frac{-dx+b}{cx-a}$である.
    これらは明らかに$\proj^1$の定数でないrational function.
    ($\proj^1=\affine^1+\{\infty\}$と考えていることに注意.必要なら$x=u/v$と斉次化せよ.)
    Ex6.4より,$\proj^1$のautomorphismが誘導される.

    \subsection{$\Aut \proj^1 \cong \Aut k(x)$.}
    $\phi \in \Aut \proj^1$を任意に取ると,
    以下のように$k(x)$の自己同型写像が誘導される.
    \begin{defmap}
        \phi^*:& k(x)& \to& k(x) \\ 
        {}& \frac{g}{h}(x)& \mapsto& \left( \frac{g}{h} \circ \phi \right)(x)
    \end{defmap}
    これが自己同型であることはmorphismの定義から明らか.

    逆に$\psi \in \Aut k(x)$を任意に取ると,
    以下のように$\proj^1$の自己同型写像が誘導される.
    \begin{defmap}
        \psi_*:& \proj^1& \to& \proj^1 \\ 
        {}& a& \mapsto& (\psi(x))(a)
    \end{defmap}
    $\psi(x) \in k(x)$に注意.

    \subsection{$\Aut k(x)=PGL(1)$, and then $\Aut \proj^1 \cong PGL(1)$.}
    $k(x)$の自己準同型は$x$の像で決定されることは明らか.
    そこで$\Aut k(x)$のある元$\psi$は$x$を$f/g \in k(x)$に写すとしよう.
    この時,$f,g$は高々1次式でなくては$\psi$がinverse morphismを持たないことを示す.

    $X=f/g$としよう.
    \[ X=\frac{a_n x^n+\dots+a_0}{b_n x^n+\dots+b_0} \mwhere \{a_i\}, \{b_j\} \subset k. \]
    分子分母の次数は高々$n$であることに注意せよ.
    $X$の値が与えられた時,$\psi^{-1}(X)$は以下の方程式の解集合である.
    \[ (a_n-X b_n)x^n+(a_{n-1}-X b_{n-1}) x^{n-1}+\dots+(a_0-X b_0)=0. \]
    $\psi$が全単射ならば任意の$X$について$\psi^{-1}(X)$は一点集合である.
    よって$n=1$,すなわち$\Aut k(x) \subseteq PGL(1)$.
    逆の包含関係は明らかだから,主張が示せた.

    \section{If $\affine^1 - \{P_n\}_{n=1}^{s} \iso \affine^1 - \{Q_n\}_{n=1}^{t}$ then $s=t$.Converse?} %% Ex6.7
    $\proj^1=\affine^1+\{\infty\}$と考えることで$\affine^1-\{P_i\} \iso \proj^1-\{P_i,\infty\}$が自然に得られる.
    なので同型写像$\phi: \affine^1 - \{P_n\}_{n=1}^{s} \isomap \affine^1 - \{Q_n\}_{n=1}^{t}$から
    birational map $\phi': \proj^1 \bimap \proj^1$が得られる.
    このbirational mapを拡張すると,Ex6.3で述べたように,$\proj^1$の自己同型写像$\bar{\phi'}$が得られる.
    このことから主張が示される.
    \[ \bar{\phi'}(\{P_n\}_{n=1}^{s})=\proj^1 - \bar{\phi'}(\proj^1 - \{P_n\}_{n=1}^{s})=\proj^1 - (\proj^1 - \{Q_n\}_{n=1}^{t})=\{Q_n\}_{n=1}^{t}. \]
    
    逆を考える.
    ここまでの議論では$\affine^1 - \{P_n\}_{n=1}^{s} \iso \affine^1 - \{Q_n\}_{n=1}^{t}$の同型から$\proj^1$の自己同型を作った.
    しかも出来上がった自己同型は$\{P_n\}_{n=1}^{s}$を$\{Q_n\}_{n=1}^{t}$へ写すものであった.
    一方,Ex6.6より$\Aut \proj^1 \cong PGL(1)$であり,
    $PGL(1)$の元は3点をどの3点に移すかで決定される.
    $\affine^1$は無限集合なので,$PGL(1)$の任意の元で互いに写せないような
    $\{P_n\}_{n=1}^{s}, \{Q_n\}_{n=1}^{s}$を選べる.
    \footnote{具体的には,$k$を$\C$に埋め込み,$\affine^3$を見る.$\{(P_m, Q_n, P_mQ_n)\}$のうちのどの4点も同一平面上に乗らなければ良い.}
    このような点達について同様の議論をすると矛盾が生じる.

\end{document}
