\documentclass[a4paper, dvipdfmx]{jsarticle}

\usepackage[]{../math_note, enumitem}
\usepackage{xcolor}
\usepackage{graphics}
\usepackage[all, pdf, 2cell, cmtip]{xy}
\usepackage{tikz}
\usetikzlibrary{cd, positioning, arrows}
%% for hyperref {{{
\usepackage[dvipdfmx, colorlinks=true, linkcolor=black]{hyperref}
\usepackage{pxjahyper}
%% }}}

%% environment: question/problem {{{
\usepackage{chngcntr}
\makeatletter
    \newcounter{c@question}
    \counterwithin{c@question}{section}
    \newenvironment{question}[0]%
    {\stepcounter{c@question}\begin{itembox}[l]{問\arabic{section}.\arabic{c@question}}}%
    {\end{itembox}}%
    \newenvironment{question*}[0]%
    {\stepcounter{c@question}\begin{itembox}[l]{問}}% 
    {\end{itembox}}%
\makeatother

\makeatletter
    \newcounter{c@problem}
    \counterwithin*{c@problem}{section}
    \newenvironment{problem}[0]%
    {\stepcounter{c@problem}\begin{itembox}[l]{問題\arabic{section}.\arabic{c@problem}}}%
    {\end{itembox}}%
    \newenvironment{problem*}[0]%
    {\stepcounter{c@problem}\begin{itembox}[l]{問題}}% 
    {\end{itembox}}%
\makeatother
%% }}}

\newenvironment{myenum}[1][\roman*]
{\hfill \vspace{-0.8cm}\begin{enumerate}[label=(#1), labelindent=1cm]}
{\end{enumerate}}

\newcommand{\step}[1]{\paragraph{\bf #1}}

\setenumerate{label=(\roman*),itemsep=3pt,topsep=7pt}

%% category
\newcommand{\Sch}{\mathbf{Sch}}
\newcommand{\Sets}{\mathbf{Sets}}
\newcommand{\Ring}{\mathbf{Ring}}
\newcommand{\Alg}{\mathbf{Alg}}
\newcommand{\Cat}{\mathbf{Cat}}
\newcommand{\Sh}{\mathbf{Sh}}
\newcommand{\PSh}{\mathbf{PSh}}

\newcommand{\Fib}[1]{\cat{Fib}(\cat{#1})}
\newcommand{\cFib}[1]{\cat{cFib}(\cat{#1})}
\newcommand{\sFib}[1]{\cat{sFib}(\cat{#1})}
\newcommand{\FibBP}[1]{\cat{Fib}^{\mathrm{bp}}(\cat{#1})}
\newcommand{\CFG}[1]{\cat{CFG}(\cat{#1})}
\newcommand{\Shv}[1]{\cat{Shv}(\cat{#1})}

\newcommand{\Comma}[2]{#1{\downarrow}#2}
\newcommand{\Lim}{\operatorname{lim}}
\newcommand{\Colim}{\operatorname{colim}}

\newcommand{\kiso}[1][{}]{\overset{#1}{\iso}}
\newcommand{\kequiv}[1][{}]{\overset{#1}{\simeq}}

%% derivation
\newcommand{\shDer}{\Omega}
\newcommand{\modDer}{\Omega}
\newcommand{\Der}{\mathrm{Der}}

%% functor
\newcommand{\ftor}[1]{\underline{#1}}
\newcommand{\ftorSh}{\mathit{Shff}}
\newcommand{\ftorFgt}{\mathit{Fgt}}

%% sites
\newcommand{\Cov}{\operatorname{Cov}}
\newcommand{\et}{\mathrm{et}}
\newcommand{\Et}{\mathrm{Et}}
\newcommand{\ET}{\mathrm{ET}}

%% covers
\newcommand{\covU}{\mathcal{U}}
\newcommand{\covV}{\mathcal{V}}
\newcommand{\covW}{\mathcal{W}}

%% utility
\newcommand{\mnewline}{\mbox{}\newline}
\newcommand{\tp}[2]{\texorpdfstring{#1}{#2}}
\newcommand{\parto}[2]{\mathrel{\mathop{\rightrightarrows}^{#1}_{#2}}}

%% {{{ fibered categories
\newcommand{\fib}[1]{\mathscr{#1}}
\newcommand{\fibA}{\fib{A}}
\newcommand{\fibB}{\fib{B}}
\newcommand{\fibC}{\fib{C}}
\newcommand{\fibD}{\fib{D}}
\newcommand{\fibE}{\fib{E}}
\newcommand{\fibF}{\fib{F}}
\newcommand{\fibG}{\fib{G}}
\newcommand{\fibH}{\fib{H}}
\newcommand{\fibI}{\fib{I}}
\newcommand{\fibJ}{\fib{J}}
\newcommand{\fibK}{\fib{K}}
\newcommand{\fibL}{\fib{L}}
\newcommand{\fibM}{\fib{M}}
\newcommand{\fibN}{\fib{N}}
\newcommand{\fibO}{\fib{O}}
\newcommand{\fibP}{\fib{P}}
\newcommand{\fibQ}{\fib{Q}}
\newcommand{\fibR}{\fib{R}}
\newcommand{\fibS}{\fib{S}}
\newcommand{\fibT}{\fib{T}}
\newcommand{\fibU}{\fib{U}}
\newcommand{\fibV}{\fib{V}}
\newcommand{\fibW}{\fib{W}}
\newcommand{\fibX}{\fib{X}}
\newcommand{\fibY}{\fib{Y}}
\newcommand{\fibZ}{\fib{Y}}
%% }}}

%% {{{ stacks 
\newcommand{\st}[1]{\mathcal{#1}}
\newcommand{\stA}{\st{A}}
\newcommand{\stB}{\st{B}}
\newcommand{\stC}{\st{C}}
\newcommand{\stD}{\st{D}}
\newcommand{\stE}{\st{E}}
\newcommand{\stF}{\st{F}}
\newcommand{\stG}{\st{G}}
\newcommand{\stH}{\st{H}}
\newcommand{\stI}{\st{I}}
\newcommand{\stJ}{\st{J}}
\newcommand{\stK}{\st{K}}
\newcommand{\stL}{\st{L}}
\newcommand{\stM}{\st{M}}
\newcommand{\stN}{\st{N}}
\newcommand{\stO}{\st{O}}
\newcommand{\stP}{\st{P}}
\newcommand{\stQ}{\st{Q}}
\newcommand{\stR}{\st{R}}
\newcommand{\stS}{\st{S}}
\newcommand{\stT}{\st{T}}
\newcommand{\stU}{\st{U}}
\newcommand{\stV}{\st{V}}
\newcommand{\stW}{\st{W}}
\newcommand{\stX}{\st{X}}
\newcommand{\stY}{\st{Y}}
\newcommand{\stZ}{\st{Z}}
%% }}}


%% changemargin {{{
\def\changemargin#1#2{\list{}{\rightmargin#2\leftmargin#1}\item[]}
\let\endchangemargin=\endlist 
%% }}}

\begin{document}
\title{ゼミノート \#11.5 \\ Artin Stackのpresentationについての短い概要}
\author{七条彰紀}
\maketitle
\tableofcontents
\vspace{10pt}

一般にartin stackはalgebraic spaceのgroupoidの商として表現することが出来る.
これをpresentation of an artin stackと呼ぶ.
このことについて,基本的な定義と命題をまとめておく.
(ほとんど\cite{SP}の和訳程度になるだろう.)

\section*{Stacks Project\tp{\cite{SP}}{}の記法について}
    \cite{SP} section 88.16, 88.17とchapter 72に
    presentation of an artin stackについての命題が書かれているが,
    これらだけで読むと良く分からない記法が有るので,意味をまとめた.

    \begin{itemize}
    \item 
        algebraic space :: $F$について$\mathcal{S}_F$は
        $F$からgrothandieck constructionで得られるstackを意味する(\cite{SP} 04M7).

    \item
        自然変換の間の演算$\ast$はhorizontal compositionで
        $\circ$はvertical composition (\cite{SP} 044T).
    \end{itemize}

\section{Algebraic Groupoid Space}
\newpage

\begin{Def}[\cite{Olsson16}, p.80, \cite{LMB} 2.4.3, \url{https://en.wikipedia.org/wiki/Groupoid_object}]
    圏$\cat{C}$をfinite fiber productを持つ圏とする.
    圏$\cat{C}$の対象$X_0, X_1$と,
    次の$5$つの射の組 :: $(X_0,X_1, s,t,c,e,i)$を考える.
    (この組をしばしば$(X_0,X_1,s,t)$や$X_0 \parto{s}{t} X_1$と略す.)
    \begin{center}
    \begin{tabular}{ll}
        \textbf{source and target} & $s,t \colon X_1 \to X_0$ \\ \hline
        \textbf{composition}       & $c \colon X_1 \times_{t,X_0,s} X_1 \to X_1$ \\ \hline
        \textbf{identity}          & $e \colon X_0 \to X_1$, \\ \hline
        \textbf{inversion}         & $i \colon X_1 \to X_1$.
    \end{tabular}
    \end{center}

    これらが次を満たす時,groupoid in $\cat{C}$と呼ぶ.
    なお,以下では$\times_{s,X_0, t}, \times_{s, X_0, \id},..$等を$\times$と略す.
    \begin{enumerate}[label=(\Alph*)]
    \item 
        \begin{itemize}
            \item $s\circ e=t\circ e=\id[X_0]$
            \item $s\circ m=s\circ \pr_{0}$
            \item $t\circ m=t\circ \pr_{1}$
        \end{itemize}
        ここで$\pr_i \colon X_1\times _{t,X_0,s} X_1\to X_1$は射影.
    \item
        (Associativity)
        $m\circ (\id[X_1]\times m)=m\circ (m\times \id[X_1]),\ \ 
        m\circ (\id[X_1]\times m)=m\circ (m\times \id[X_1]),$
    \item
        (Identity)
        $m\circ (e\circ s,\id[X_1])=m\circ (\id[X_1],e\circ t)=\id[X_1]$
    \item
        (Inverse)
        \begin{itemize}
            \item $i\circ i=\id[X_1]$
            \item $s\circ i=t,\,t\circ i=s$
            \item $m\circ (\id[X_1],i)=e\circ s,\,m\circ (i,\id[X_1])=e\circ t$
            \item $m\circ (\id[X_1],i)=e\circ s,\,m\circ (i,\id[X_1])=e\circ t$
        \end{itemize}
    \end{enumerate}
\end{Def}

\begin{Def}
    $B$ :: algebraic space over a scheme $S$とする.
    algebraic space over $B$の圏におけるgroupoid対象を,
    単にgroupoid in algebraic spaces over $B$という.
\end{Def}

groupoid in algebraic spaces over $B$には
$(U, R, s,t,c,e,i)$ ( $(U, R, s,t), U \parto{s}{t} R$ )という記号が使われることが多い.

\section{Quotients of Algebraic Space by Groupoid}
\begin{Def}
    任意のscheme over $B$ :: $T$について,
    組$(U(T), R(T), s_T, t_T, c_T, e_T, i_T)$から次の様に
    圏$\{U(T)/R(T)\}$ (あるいは$[U/_pR]$と書く)が構成できる.
    \begin{description}[labelindent=0.5cm, leftmargin=1.5cm]
        \item[Object] \mnewline
            $U(T)$の元

        \item[Arrow] \mnewline
            $u, u' \in U(T)$について,
            $\Hom_{\{U/R\}(T)}(u, u')=\{\xi \in R(T) \mid s(\xi)=u, t(\xi)=u' \}.$

        \item[Identity Morphism] \mnewline
            対象$u \in U(T)$のidentity morphismは$e_T(u) \in R(T)$.

        \item[Composition of Morphisms] \mnewline
            射 :: $\xi \colon u \to u', \eta \colon u' \to u''$の合成$\eta \circ \xi$は
            $(\eta, \xi) \in R(T) \times_{s_T, U(T), t_T} R(T)$の$c_T$による像.

        \item[Inverse Morphism] \mnewline
            射$\xi \colon u \to u'$の逆射は$i_T(\xi) \in R(T)$.
    \end{description}

    関手$\{U(-)/T(-)\} \colon \Sch/B \to (\mathbf{Groupoids})$を
    Grothendieck constructionでfibered categoryにしたものを$\{U/R\}$と書く.
    さらにこれをstackificationしたものを$[U/R]$と書き,
    quotient stack of $U$ by $R$と呼ぶ.
\end{Def}

一般にquotient stackはalgebraic stackでない.

なお,組$(U,R,s,t,c,e,i)$がgroupoid in algebraic spaces over $B$であることは,
以下で構成する圏$\{U(T)/R(T)\}$がgroupoidとなることと同値である.

\section{Artin StackからPresentation of an Artin Stackへ}
    artin stack over a scheme $S$ :: $\stX$とatlas :: $f \colon U \to \stX$をとる.
    この時,$R:=U \times_{\stX} U$が($s:=\pr_1, t=\pr_2$とすれば)groupoidになっている.
    特にatlasがsmoothであるから,$s,t \colon R \to U$がsmoothになっている.

    \begin{Thm}[\cite{SP} 04T4, 04T5]
        artin stack over a scheme $S$ :: $\stX$とatlas :: $f \colon U \to \stX$をとる.
        するとgroupoid space :: $(U, R, s,t,c,i)$が構成できる.
        さらに標準的な射 :: $f_{can} \colon [U/R] \to \stX$が存在し,
        これが圏同値となる.
    \end{Thm}

    証明の準備として以下の補題を置く.

    \begin{Lemma}[\cite{SP} 04T4 (1)--(3)]\label{lemm:fiberprod}
        artin stack over a scheme $S$ :: $\stX$とatlas :: $f \colon U \to \stX$をとる.
        この時,fiber productに関する次の命題が成り立つ.
        \begin{enumerate}
            \item
                $R:=U \times_{f,\stX,f} U$ :: algebraic space.
            \item
                標準的圏同型 $U \times_{\stX} U \times_{\stX} U=R \times_{U} R$が成り立つ.
            \item
                $U \times U \times U$の第$0$成分と第$2$成分から$R$への射影 :: $\pr_{0,2}$は
                (ii)の同型により射$\pr_{0,2} \colon R \times R \to R$を誘導する.
        \end{enumerate}
    \end{Lemma}

    \begin{Lemma}[\cite{SP} 04T4 (4), 04T5 (1)]
        この時,groupoid :: $(U, R, s,t,c,i)$が構成できる.
        より詳しく,$R, s,t,c,i$を次のようにすれば良い.
        \begin{itemize}
            \item $R:=U \times_{f,\stX,f} U$,
            \item $s:=\pr_0 \colon R \to U$,
            \item $t:=\pr_1 \colon R \to U$,
            \item $c:=\pr_{0,2} \colon R \times R \to R$,
            \item $e \colon U \ni u  \mapsto (u,u,\id[f(u)]) \in R$,
            \item $i \colon R \ni (u,u',\alpha) \mapsto (u',u,\alpha^{-1}) \in R$,
        \end{itemize}
        fiber product of stacksの構成から,
        $R$の対象は$U(T)\ (T \in \Sch/S)$の二つの対象 :: $u, u'$と
        $\stX$内の同型$\alpha \colon f(u) \to f(u')$からなる$3$つ組であることに注意.
        また,$s,t$ :: smoothにも注意.
    \end{Lemma}

    \begin{Lemma}[\cite{SP} 04T4 (5)]
        $f \colon U \to \stX$から
        標準的な射 :: $f_{can} \colon [U/R] \to \stX$が誘導される.
    \end{Lemma}
    \begin{proof}
        (\cite{SP} 04T4)の証明内,
        ``This proves that Groupoids in Spaces, Lemma 044U applies"で始まる段落に
        $f_{can}$の具体的な構成が記述されている.
    \end{proof}

\printbibliography[title=参考文献]
\end{document}
