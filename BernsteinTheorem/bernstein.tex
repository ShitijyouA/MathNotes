\documentclass[a4j, 10pt]{jarticle}
\usepackage{../math_note}
\newcommand{\powerset}{\mathcal{P}}

\title{Cantor-Schr\"{o}der-Bernsteinの定理}
\author{七条 彰紀}

\begin{document}
    \maketitle
    
    以下で述べる証明は,\cite{thebook}にある証明を徹底的に詳述したものである.
    教育的な証明とも言えるだろう.
    この証明の歴史については\cite{cbt}のPart IVが参考になる.
    特に,ここでの証明はChapter 35(pp.343-353)にある
    Tarski’s Fixed-Point Theoremを応用したものに近いと思われる.

    \section{定理}
    \begin{Thm}[表現1]
        集合M, Nについてそれぞれの濃度を$|M|, |N|$のように表す.
        $|M| \leq |N|$かつ$|M| \geq |N|$ならば$|M|=|N|$である.
    \end{Thm}

    \begin{Thm}[表現2]
        M, Nを集合とし,単射$f:M \to N$と単射$g: N \to M$があったとする.
        この時全単射$h:M \to N$が存在する.
    \end{Thm}

    表現1はこの定理を証明するモチベーションの出処がよく分かる.
    表現2からは定理が写像についての基本的な定理であることが分かる.

    \section{証明}
    この定理が述べているのは全単射$h$の存在であるから,証明の方針は二つ有る.
    一つは別の存在定理の帰結として$h$の存在を示す方針,
    もう一つは全単射$h$を実際に構成する方針である.
    このレポートでの証明は前者と後者の両方\footnote{なので証明の方針は三つある.}を使う.
    すなわち,ある集合の存在を示し,それを用いて写像$h$を構成する.

    \subsection{準備.}
    単射$f: M \to N$と任意の部分集合$S,T \subseteq M$について以下が成り立つ.
    \[
        f^{-1}(f(S))=S,~~~
        f(M \setminus S)=N \setminus f(S),~~~
        f(S) \cap f(T)=f(S \cap T).
    \]
    これらの命題は集合と写像について書いた本なら
    どれにでも書かれているようなものである.

    \subsection{全単射$h$の構成方法}
    $f, g^{-1}; M \to N$とする.
    $M$から$N$への全単射$h$を作る方法として,
    $M$の各元が$M$の部分集合$S$に属すか属さないかによって
    $M$の元を送る先を$f(x)$か$g^{-1}(x)$にする方法がある.
    \[
        h(x)=
        \begin{cases}
            f(x) & (x \in S) \\
            g^{-1}(x) & (x \in M \setminus S) \\
        \end{cases}
    \]
    ただし$S \subset M$は以下の3つの条件を満たす.
    \begin{enumerate}[(a)]
    \setlength{\itemindent}{3em}
        \item $f(S) \cup g^{-1}(M \setminus S)=N$.
        \item $M \setminus S \subseteq g(N)$.
    \end{enumerate}

    $f$は$M$全体で定義されるが,像が$N$全体とは限らない.
    一方$g^{-1}$は$M$全体で定義されるとは限らないが,像は$N$全体である.
    この二つを組み合わせることで全単射が作れそうだ,というのがこの構成の着想である.
    しかし$f, g^{-1}$が互いをうまく補い合えるかどうかは自明でない.

    \subsection{写像$F$}
    写像$F$を以下のように定義する.
    \begin{defmap}
        F:& \powerset(M)& \to& \powerset(N) \\ 
        {}& S & \mapsto& M \setminus g(N \setminus f(S))
    \end{defmap}
    そして$F(S_0)=S_0$となる集合$S_0$,すなわち$F$の不動点が存在すると仮定する.
    この時,以下のようにして(a),(b)が成立することが分かる.
    \begin{align*}
        {}                      S_0=&M \setminus g(N \setminus f(S_0)) \\
        \iff\hspace{5.7em}      M \setminus S_0=&g(N \setminus f(S_0)) \subseteq g(N) \tag{b} \\
        \implies\hspace{3.5em}  g^{-1}(M \setminus S_0)=&N \setminus f(S_0) \\
        \implies                f(S_0) \cup g^{-1}(M \setminus S_0)=&N \tag{a}
    \end{align*}
    したがって,我々は$F$の不動点$S_0$を構成すれば良い.

    \subsection{写像$F$はどうやって着想されるのか}
    (a) $f(S) \cup g^{-1}(M \setminus S)=N$に注目する.
    この式は$g^{-1}$を用いているが,これを取り除く方針で式変形をする.
    これは丁度直前のsubsectionにおける変形を逆転させたものである.
    \begin{align*}
        f(S_0) \cup g^{-1}(M \setminus S_0)=&N \\
        g^{-1}(M \setminus S_0)=&N \setminus f(S_0) \\
        M \setminus S_0=&g(N \setminus f(S_0)) \\
        S_0=&M \setminus g(N \setminus f(S_0))
    \end{align*}
    実際に上から下が示されるためには
    $f(S) \cap g^{-1}(M \setminus S)=\emptyset$($1$行目$\implies$$2$行目)と
    $g(g^{-1}(M \setminus S_0))=M \setminus S_0$($2$行目$\implies$$3$行目)が必要である.
    さらに,(a)から(b) $M \setminus S \subseteq g(N)$を示すことも出来ない.

    しかし,最終的に現れる$S_0=M \setminus g(N \setminus f(S_0))$を
    「$S_0$は不動点」と解釈することで,
    直前のsubsectionのような結果が得られる.
    示したい命題(a)からはじめて荒っぽい式変形を行い着想を得る,
    という,数学者の仕事が見える証明である.

    \subsection{$F$の不動点}
    以上の議論から,我々は写像$F$の不動点さえ求めれば全単射$h$が構成できることがわかった.
    また,$F$について次が成り立つ.
    \[ \Forall{I,J \subseteq M} I \subseteq J \implies F(I) \subseteq F(J) \]
    これは次のように示せば良い.
    \begin{align*}
        {}&         I \subseteq J \\
        \implies&   f(I) \subseteq f(J) \\
        \iff&       N \setminus f(I) \supseteq N \setminus f(J) \\
        \implies&   g(N \setminus f(I)) \supseteq g(N \setminus f(J)) \\
        \iff&       M \setminus g(N \setminus f(I)) \subseteq M \setminus g(N \setminus f(J)) \\
        \iff&       F(I) \subseteq F(J)
    \end{align*}
    このことを用いて2つの証明を述べる.

    \subsection{$F$の不動点が存在することの証明(非構成的)}
    Tarski’s Fixed-Point Theoremを用いる.

    \subsection{$F$の不動点が存在することの証明(構成的)}
    明らかに$M \supseteq F(M)$なので,両辺を繰り返し$F$で写せば
    \[ M \supseteq F(M) \supseteq F^2(M) \supseteq \cdots. \]
    が得られる.
    さて,以下の集合が$F$の不動点である.
    \[ S_0=\bigcap_{i \geq 0}{F^{i}(M)}=M \cap F(M) \cap F(F(M)) \cap \cdots. \]
    これが不動点であることは以下のように示される.
    \begin{align*}
    F(S_0)
    =&F \left(\bigcap_{i \geq 0}{F^{i}(M)} \right) \\
    =&M \setminus g \left(N \setminus f \left(\bigcap_{i \geq 0}{F^{i}(M)} \right) \right) \\
    =&M \setminus g \left(N \setminus \bigcap_{i \geq 0}{f(F^{i}(M))} \right)\\
    =&M \setminus g \left(\bigcup_{i \geq 0}{N \setminus f(F^{i}(M))} \right)\\
    =&M \setminus \bigcup_{i \geq 0}{g(N \setminus f(F^{i}(M)))}\\
    =& \bigcap_{i \geq 0}{M \setminus g(N \setminus f(F^{i}(M)))}\\
    =& \bigcap_{i \geq 0}{F(F^{i}(M))} \\
    =& \bigcap_{i \geq 1}{F^{i}(M)} \\
    =& M \cap \bigcap_{i \geq 1}{F^{i}(M)} \\
    =& \bigcap_{i \geq 0}{F^{i}(M)} \\
    =& S_0
    \end{align*}
    $M \supseteq F(M) \supseteq \cdots$の他に,$f$が単射であることを用いた.

    \begin{thebibliography}{99}
        \bibitem{thebook}
            マーティン・アイグナー(英語版),ギュンター・ツィーグラー(英語版)
            『天書の証明』 蟹江幸博 訳,丸善出版,2012年9月(原著2002年12月),縮刷版.

        \bibitem{cbt}
        Arie Hinkis (2013) 
        ``Proofs of the Cantor-Bernstein theorem. A mathematical excursion''
        Science Networks. Historical Studies 45, Springer Basel.
    \end{thebibliography}

\end{document}
