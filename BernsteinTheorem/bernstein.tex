\documentclass[a4j, 10pt]{jarticle}
\usepackage{../math_note}

\title{Cantor-Schr\"{o}der-Bernsteinの定理}
\author{七条 彰紀}

\begin{document}
    \maketitle
    
    以下で述べる証明は,\cite{thebook}にある証明を再構成したものである.
    元の証明では写像$F$が現れる理由が明記されていないため,
    その部分を補った.
    この証明の歴史については\cite{cbt}のPart IVが参考になる.
    特に,ここでの証明はChapter 35(pp.343-353)にある
    Tarski’s Fixed-Point Theoremを応用したものに近いと思われる.

    \section{定理}
    \begin{Them}[表現1]
        集合A, Bについてそれぞれの濃度を$|A|, |B|$のように表す.
        $|A| \leq |B|$かつ$|A| \geq |B|$ならば$|A|=|B|$である.
    \end{Them}

    \begin{Them}[表現2]
        A, Bを集合とし,単射$f:A \to B$と単射$g: B \to A$があったとする.
        この時全単射$h:A \to B$が存在する.
    \end{Them}

    表現1はこの定理を証明するモチベーションの出処がよく分かる.
    表現2からは定理が写像についての基本的な定理であることが分かる.

    \section{証明}
    この定理が述べているのは全単射$h$の存在であるから,証明の方針は二つ有る.
    一つは別の存在定理の帰結として$h$の存在を示す方針,
    もう一つは全単射$h$を実際に構成する方針である.
    このレポートでは後者の方針を取る.
    前者の方針での証明は知らない.可能かどうかもわからない.

    \subsection{全単射hの構成方法}
    $f, g^{-1}; M \to N$とする.
    $M$から$N$への全単射$h$を作る方法として,
    $M$の各元が$M$の部分集合$S$に属すか属さないかによって
    $M$の元を送る先を$f(x)$か$g^{-1}(x)$にする方法がある.
    \[
        h(x)=
        \begin{cases}
            f(x) & (x \in S) \\
            g^{-1}(x) & (x \in M \setminus S) \\
        \end{cases}
    \]
    ただし$S \subset M$は以下の3つの条件を満たす.
    \begin{enumerate}[(a)]
    \setlength{\itemindent}{3em}
        \item $f(S) \cup g^{-1}(M \setminus S)=N$.
        \item $f(S) \cap g^{-1}(M \setminus S)=\emptyset$.
        \item $M \setminus S \subseteq g(N)$.
    \end{enumerate}

    $f$は$M$全体で定義されるが,像が$N$全体とは限らない.
    一方$g^{-1}$は$M$全体で定義されるとは限らないが,像は$N$全体である.
    この二つを組み合わせることで全単射が作れそうだ,というのがこの構成の着想である.
    しかし$f, g^{-1}$が互いをうまく補い合えるかどうかは自明でない.
    (b)が必要な理由は今の段階ではわかりにくいが,
    後に必要なので,集合$S$の性質としてここにまとめて書いておく.

    \subsection{部分集合$S$に課された条件の整理.}
    実は上で$S$に要求した3つの条件のうち,(b), (c)は残る(a)から導かれる.

    \paragraph{(a) $\implies$ (b).}
    まず,(a) $\implies$ (b)が成り立つ.
    そのことを示すため,(a)式の両辺を$g$で写す.
    \[ g \circ f(S) \cup (M \setminus S)=g(N). \]
    この時,$g \circ f$が単射であることから,$g \circ f(S) \subseteq S$が成り立つ.
    したがって$g \circ f(S) \cap (M \setminus S)=\emptyset$.
    再び両辺を$g^{-1}$で写して,$f(S) \cap g^{-1}(M \setminus S)=\emptyset$が得られる.

    \paragraph{(a) $\implies$ (c)}
    上で見た式$g \circ f(S) \cup (M \setminus S)=g(N)$より明らかである.

    \subsection{写像$F$}
    再び(a) $f(S) \sqcup g^{-1}(M \setminus S)=N$に注目する.
    この式は$g^{-1}$を用いているが,これを取り除く方針で式変形をする.
    \begin{eqnarray*}
        f(S) \sqcup g^{-1}(M \setminus S)&=&N \\
        g^{-1}(M \setminus S)&=&N \setminus f(S) \\
        M \setminus S&=&g(N \setminus f(S)) \\
        S&=&M \setminus g(N \setminus f(S))
    \end{eqnarray*}
    一行目から二行目への変形で
    (b) $f(S) \cap g^{-1}(M \setminus S)=\emptyset$を用いた.
    そこで,写像$F$を次のように定義する.
    \begin{eqnarray*}
        F : \mathcal{P}(M) &\to& \mathcal{P}(N) \\
            S &\mapsto& M \setminus g(N \setminus f(S))
    \end{eqnarray*}
    すると,部分集合$S$が満たすべき3つの条件は,
    $F(S)=S$すなわち「$S$の写像$F$の不動点である」ということと同値になる.

    \subsection{$F$の不動点}
    以上の議論から,我々は写像$F$の不動点さえ求めれば全単射$h$が構成できることがわかった.
    また,$F$が写像であることから,
    \[ \Forall{I \subseteq J \subseteq M} F(I) \subseteq F(J) \]
    が成り立つ.
    このことを用いて2つの証明を述べる.

    \subsection{$F$の不動点が存在することの証明(非構成的)}

    \subsection{$F$の不動点が存在することの証明(構成的)}
    明らかに$M \supseteq F(M)$なので,両辺を繰り返し$F$で写せば
    \[ M \supseteq F(M) \supseteq F^2(M) \supseteq \cdots. \]
    が得られる.
    さて,以下の集合が$F$の不動点である.
    \[ S=\bigcap_{i \geq 0}{F^{i}(M)}=M \cap F(M) \cap F(F(M)) \cap \cdots. \]
    これが不動点であることは以下のように示される.
    \begin{align*}
    F(S)
    =&F \left(\bigcap_{i \geq 0}{F^{i}(M)} \right) \\
    =&M \setminus g \left(N \setminus f \left(\bigcap_{i \geq 0}{F^{i}(M)} \right) \right) \\
    =&M \setminus g \left(N \setminus \bigcap_{i \geq 0}{f(F^{i}(M))} \right)\\
    =&M \setminus g \left(\bigcup_{i \geq 0}{N \setminus f(F^{i}(M))} \right)\\
    =&M \setminus \bigcup_{i \geq 0}{g(N \setminus f(F^{i}(M)))}\\
    =& \bigcap_{i \geq 0}{M \setminus g(N \setminus f(F^{i}(M)))}\\
    =& \bigcap_{i \geq 0}{F(F^{i}(M))} \\
    =& \bigcap_{i \geq 1}{F^{i}(M)} \\
    =& M \cap \bigcap_{i \geq 1}{F^{i}(M)} \\
    =& \bigcap_{i \geq 0}{F^{i}(M)} \\
    =& S
    \end{align*}
    $M \supseteq F(M) \supseteq \cdots$の他に,$f$が単射であることを用いた.

    \begin{thebibliography}{99}
        \bibitem{thebook}
            マーティン・アイグナー(英語版),ギュンター・ツィーグラー(英語版)
            『天書の証明』 蟹江幸博 訳,丸善出版,2012年9月(原著2002年12月),縮刷版.

        \bibitem{cbt}
        Arie Hinkis (2013) 
        ``Proofs of the Cantor-Bernstein theorem. A mathematical excursion''
        Science Networks. Historical Studies 45, Springer Basel.
    \end{thebibliography}

\end{document}
