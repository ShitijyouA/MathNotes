\documentclass[a4paper]{jsarticle}
\usepackage{../math_note}
\usepackage{enumitem}

\newcommand{\Stab}{\operatorname{Stab}}
\newcommand{\Fix}{\operatorname{Fix}}
\newcommand{\Norm}{\operatorname{N}}
\newcommand{\Syl}{\operatorname{Syl}}

\title{Sylow Theorems}
\author{七条 彰紀}

\begin{document}
\maketitle
これはKeith Conrad氏のノート``THE SYLOW THEOREMS"
\footnote{\url{http://www.math.uconn.edu/~kconrad/blurbs/grouptheory/sylowpf.pdf}}
を元にした,Sylowの定理の証明とその応用についてのノートである.

\begin{Thm*}[Sylow Theorems]
    任意の有限群$G$について,
    その位数が素数$p$,$p$に互いに素な正数$m$,そして非負整数$n$によって
    $|G|=p^{n}m$と表されるとする.
    更に群$G$の$p$-Sylow部分群全体の集合を$\Syl_p(G)$とおく.以下が成り立つ.
    \begin{enumerate}[label=\Roman*.]
        \item
            $G$は$p$-Sylow部分群を持ち
            \footnote{すなわち$\Syl_p(G) \neq \emptyset$},
            またこれは$G$の任意の$p$-部分群を含む.
        \item $G$の全ての$p$-Sylow部分群は互いに共役である.
        \item $|\Syl_p(G)| \equiv 1 \mod p$
        \item 任意の$p$-Sylow部分群$P$について$|\Syl_p(G)| = [G:\Norm_G(P)]$
        \item $|\Syl_p(G)|$は$m$の約数である.
    \end{enumerate}
\end{Thm*}

\section{Prepares for The proof}
    群$G$は集合$X$に作用するものとする.
    この時,$g \in G, x \in X$について,
    $g$が$x$に作用したものを(このセクションでは)$g \ast x$で表す.
    \begin{Def}
        安定化群$\Stab_G$,不変元の全体$\Fix_G$を以下で定める.
        \[
            \Stab_G(X)=\{ g \in G \mid g \ast x=x \},~~~
            \Fix_G(X)=\{x \in X \mid \Forall{g \in G} g \ast x=x \}.
        \]
        また正規化群$\Norm_G$を部分集合$S \subseteq G$について次のように定める.
        \[ \Norm_G(S)=\{ g \in G \mid gSg^{-1}=S \}. \]
    \end{Def}
    $\Stab_G, \Norm_G$がそれぞれ群であることは使う.
    また,$S \subseteq G$が部分群である時に
    $\Norm_G(S)$は$S$を正規部分群に持つ最大の部分群であることも
    (特にsection 4で)使う.

    \begin{Thm}[Orbit-Stabilizer Theorem]
        群$G$は集合$X$に作用するものとする.以下が成り立つ.
        \[ \Forall{x \in  X} |G / \Stab_G(x)|=|G \ast x|. \]
    \end{Thm}
    \begin{proof}
        次の写像を考える(準同型とは限らない).
        \begin{defmap}
            \phi:& G / \Stab_G(x)& \to& G \ast x \\
            {}& g \Stab_G(x)& \mapsto& g \ast x
        \end{defmap}
        明らかに$\phi$は全射.
        あとはwell-definedであることと単射であることを見れば良い.
        $g, h \in G$を任意に取る.
        \begin{align*}
        {}&     g \ast x=h \ast x \\
        \iff&   (h^{-1} g) \ast x=x \\
        \iff&   h^{-1} g \in \Stab_G(x) \\
        \iff&   (h^{-1} g) \Stab_G(x)=\Stab_G(x) \\
        \iff&   h^{-1} (g \Stab_G(x))=\Stab_G(x) \\
        \iff&   g \Stab_G(x)= h \Stab_G(x)
        \end{align*}
        上から順に見ればこれは$\phi$が単射であることの証明であり,
        下から順に見ればwell-definedであることの証明になっている.
    \end{proof}

    \begin{Thm}[Lagrange's Theorem]
        任意の群$G$とその部分群$U$の位数について以下が成り立つ.
        \[ |G/U||U|=|G|. \]
        これは$|G|,|U|$が無限群であっても成り立つ.
    \end{Thm}
    \begin{proof}
        \textbf{この段落は『天書の証明』より引用する.}
        二項関係\[ a \sim b \iff ba^{-1} \in U \]を考える.
        群の公理から$\sim$が同値関係であることがわかる.
        元$a$を含む同値類はコセット\[ aU=\{ ax \mid x \in U \} \]に一致する.
        明らかに$|aU|=|U|$なので,
        $G$は全ての大きさが$|U|$である同値類に分解される.
        それゆえ,$|U|$は$|G|$を割る.
        \[ |G|=\sum_{i=1}^{|G/U|}{|a_i U|}=\sum_{i=1}^{|G/U|}{|U|}=|G/U||U|. \]
        こうして主張の等式が得られる.
    \end{proof}

    \begin{Lemma}[Fixed Points Congurance]
        群$G$は集合$X$に作用するものとする.
        $|G|$が素数$p$の倍数ならば,以下が成り立つ.
        \[ |X| \equiv |\Fix_G(X)| \mod p\]
    \end{Lemma}
    \begin{proof}
        $X$の$G$による軌道分解を考える.
        \[ X= \bigsqcup_{x \in X}{Gx} \]
        するとOrbit-Stabilizer TheoremとLagrange's Theoremより,
        \[ |X| = \sum_{x \in X}{|G/ \Stab_G(x)|} = \sum_{x \in X}{|G|/|\Stab_G(x)|}\]

        $\Stab_G(x)$は$G$の部分群だから,$|\Stab_G(x)|$も$p$の倍数か1.
        したがって$|G|/|\Stab_G(x)|$は$p$の倍数か1である.
        しかも$|G|/|\Stab_G(x)|=1$すなわち$\Stab_G(x)=G$の時は$x \in \Fix_G(X)$となっている.
        よって,$|X|=|\Fix_G(X)|+(p\mbox{の倍数}) \equiv |\Fix_G(X)| \mod p$
        
    \end{proof}

    \begin{Thm}[Cauchy's Group Theorem]
        群$G$の位数が$p$の倍数ならば,$G$は位数$p$の巡回部分群を含む.
    \end{Thm}
    \begin{proof}
        位数$p$の元の存在を示せば良い.
        この元は求める巡回群の生成元である.
        ここではJames H. McKayによる論文
        ``Another Proof of Cauchy's Group Theorem"
        \footnote{\url{http://www.jstor.org/stable/2310010}}
        を紹介するに留める.
    \end{proof}

\section{Proof of Sylow Theorem I.}
    \paragraph{整理と方針}
    ステートメントは定義から次のように論理式で表される.
    \[ \Forall{i=0,1,\dots,n} \Exists{H \text{ :: subgroup of }G} |H|=p^i \]
    これを$i$に関する帰納法で証明しよう.
    まず,$i=0$の時は$H=\{e\}$が条件を満たす.
    以下では$n > 0$とし,
    $i=k<n$の時$|H|=p^k$となる部分群$H$が存在するならば,
    $H \subset H'$かつ$|H'|=p^{k+1}$,すなわち$[H':H]=p$となる部分群$H'$が存在することを示す.

    \paragraph{$\Fix_{H}(G/H)$の定義.}
    中心となるアイデアは,
    集合$G/H$の元であって,
    $H$による左からの積作用によって不変なものを考える,ということである.
    このような元全体を$\Fix_{H}(G/H)$と置く.
    \[
        \Fix_{H}(G/H) := \{ gH \in G/H \mid \Forall{h \in H} hgH=gH \}
    \]

    \paragraph{$\Fix_{H}(G/H)=\Norm_G(H)/H$.}
    この$\Fix_{H}(G/H)$を別の表現にしよう.
    \begin{eqnarray*}
        &{}&    gH \in \Fix_H(G/H) \\
        &\iff&  \Forall{h \in H} hgH=gH \\
        &\iff&  \Forall{h \in H} (g^{-1}hg)H=H \\
        &\iff&  \Forall{h \in H} (g^{-1}hg) \in H \\
        &\iff&  g^{-1}Hg = H \\
        &\iff&  g \in \Norm_G(H)
    \end{eqnarray*}
    以上から,$\Fix_{H}(G/H)=\{ gH \mid g \in \Norm_G(H) \} = \Norm_G(H)/H$となる.
    $\Fix_{H}(G/H)$が群だから,$H$は$\Norm_G(H)$の正規部分群である.

    \paragraph{$\Norm_G(H)/H$は$p$群.}
    さて,補題から次が成り立つ.
    \[ |G/H| \equiv |\Fix_{H}(G/H)| \mod p \]
    $k<n$という条件とLagrange's Theoremから,
    $|G/H|=|G|/|H|=p^{n-k}m$は$p$の倍数.
    したがって$|\Fix_{H}(G/H)|=|\Norm_G(H)/H|$も$p$の倍数.

    \paragraph{Cauchy's Group Theoremから$H'$が存在.}
    $|\Norm_G(H)/H|$が$p$の倍数であるということは,
    Cauchy's Group Theoremから,$\Norm_G(H)/H$は位数$p$の巡回部分群を含む.
    それは群$H' \subset \Norm_G(H)$によって$H'/H$と表される.
    $|H'/H|=[H':H]=p$だから,帰納法が完成した.
    
    \section{Proof of Sylow Theorem II}
    \paragraph{$\Fix_Q(G/P)$は空でない.}
    群$G$の$p$-Sylow部分群$P$,$Q$をとり,これらが共役であることを示す.
    $Q$は$p$-部分群なので以下が成り立つ.
    \[ |G/P|=[G:P] \equiv |\Fix_Q(G/P)| \mod p \]
    $|P|=p^n$から,$|G/P|=|G|/|P|=m$は$p$の倍数でない.
    特に$|G/P| \equiv |\Fix_Q(G/P)| \not \equiv 0$
    すなわち$\Fix_Q(G/P)$は空集合でない.

    \paragraph{$\Fix_Q(G/P)$の元の定義から結論へ.}
    $\Fix_Q(G/P)$の元を一つ取って$gP$とおく.
    定義から,全ての$Q$の元$q$に対して
    \[ qg \cdot P=gP \implies qg \cdot e \in gP \iff qg \in gP \iff Q \subset gPg^{-1} \]となる.
    $P,Q$はどちらも$p$-Sylow部分群で,位数は同じ.したがって$Q=gPg^{-1}$

    \section{Proof of Sylow Theorem III}
    \paragraph{方針.}
    $\Syl_p(G)$の元を一つとり$P$とする.そして集合$\Syl_p(G)$への$P$の共役作用を考える.
    $P$は$p$-部分群なので,
    \[ |\Syl_p(G)| \equiv |\Fix_P(\Syl_p(G))| \mod p. \]
    以下で$|\Fix_P(\Syl_p(G))|=1$を示す.

    \paragraph{不動点$Q$を取る.}
    $\Fix_P(\Syl_p(G))$に$P$が属すことは,
    任意の$p \in P$について$p^{-1}Pp=P$であることから自明.
    なので$\Fix_P(\Syl_p(G))$からもうひとつ元をとって$Q$とする.

    \paragraph{$P, Q, \Norm_G(Q)$の関係.}
    この時,$P,Q \subset \Norm_G(Q) \subset G$だから
    \footnote
    {
        念の為.$\Norm_G(Q):=\{ g \in G \mid g^{-1}Qg=Q \}$であり,
        $Q \in \Fix_P(\Syl_p(G))$から任意の$p \in P$について$p^{-1}Qp=Q$が成立する.
    },
    位数を考えれば$P,Q \in \Syl_p(\Norm_G(Q))$も成り立つ.
    また,$Q$は$\Norm_G(Q)$の正規部分群(Sylow Theorem Iでも触れた)である.

    \paragraph{$P=Q$を示す.}
    Sylow Theorem IIから$P,Q$は$\Norm_G(Q)$の部分群として互いに共役.
    ところが正規部分群の定義より,$\Norm_G(Q)$の部分群で$Q$と共役なものは$Q$自身しか無い.
    よって$P=Q$となり,$\Fix_P(\Syl_p(G))=\{ P \}$が示された.

    \section{Proof of Sylow Theorem IV}
    Orbit-Stabilizer Theoremを集合$\Syl_p(G)$とこれに共役作用する群$G$に用いる.
    Sylow Theorem IIから$\Syl_p(G)$の元は互いに共役だから,
    $G$の共役作用による軌道は一つしか無い.
    \[ \Forall{P \in \Syl_p(G)} |G/\Stab_G(P)|=|G \ast P|=|\Syl_p(G)|. \]
    $\Stab_G(P)=\{ g \in G \mid g^{-1}Pg=P \}=\Norm_G(P)$なので,
    \[ \Forall{P \in \Syl_p(G)} |\Syl_p(G)|=[G:\Norm_G(P)]. \]

    \section{Proof of Sylow Theorem V}
    Sylow Theorem Vから$|\Syl_p(G)|$は素数$p$と互いに素.
    また,Sylow Theorem IVとLagrange's Theoremから
    $|\Syl_p(G)|=[G:\Norm_G(P)]=|G|/|\Norm_G(P)|$なので
    $|\Syl_p(G)|$は$|G|=p^n m$の約数である.
    よって$|\Syl_p(G)|$は$m$の約数.

    \section{Applications}
    \begin{Lemma}[Frattini's Argument]
        $H$を群$G$の正規部分群,$P$を$H$の$p$-Syllow部分群とすると,
        $G=\Norm_G(P)H$である.
    \end{Lemma}
    \begin{proof}
        $H,\Norm_G(H) \subseteq G$から$G \supseteq \Norm_G(P)H$.
        逆の包含関係を示す.
        $G$の任意の元$g$を取る.$H$は$G$の正規部分群だから,
        \[ g^{-1}Pg \subset g^{-1}Hg=H \]
        したがって$g^{-1}Pg$も$H$の$p$-Syllow部分群である.
        するとSylow Theorem IIより,ある$h \in H$が存在して
        \[ hg^{-1} P gh^{-1} = (gh^{-1})^{-1} P gh^{-1} = P. \]
        $\Norm_G(P)$の定義から,$gh^{-1} \in \Norm_G(P)$.
        よって$g \in \Norm_G(P)H$が成立.
    \end{proof}

    \begin{Prop}[Sylow's test]
        $n$を素数でない正の整数とし,$p$を$n$の素因数とする.
        もし$n$の約数の中で$p$を法として1と合同なものが$1$のみであれば,
        位数$n$の単純群は存在しない.
    \end{Prop}
    \begin{proof}
        位数$n$の任意の群を$G$とする.
        $n$が素数の冪数ならば$G$は非自明な中心を持つ.したがって単純群でない.

        $n$は素数の冪数でないとする.
        すると$G$の任意の$p$-Sylow群は真部分群である.すなわち$\Syl_p(G) \not \ni G$である.
        そしてSylow Theorem IIIより$|\Syl_p(G)| \mod p=1$であるが,
        Sylow Theorem Vと仮定から,このような$|\Syl_p(G)|$は1しか無い.
        よって$G$の$p$-Sylow部分群は唯1つであり,
        $\Syl_p(G) \not \ni G$とSylow Theorem IIから,これは$G$の正規部分群.
        よって$G$は単純群でない.
    \end{proof}

\end{document}
