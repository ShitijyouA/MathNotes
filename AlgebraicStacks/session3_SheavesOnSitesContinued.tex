\documentclass[a4paper]{jsarticle}
\usepackage{macros}

\begin{document}
\title{ゼミノート \#3 \\ Sheaves on Sites, continued.}
\author{七条彰紀}
\maketitle

\section{Propositions : Sheaves.}
\begin{Thm}
    $\cat{C}$ :: siteとする.
    忘却関手
    \[
        \ftorFgt \colon
        \text{(Category of Sheaves on $\cat{C}$)}
            \to \text{(Category of Presheaves on $\cat{C}$)}.
    \]
    はleft adjoint functor :: $\ftorSh$を持つ.
\end{Thm}
\begin{proof}
\end{proof}

\begin{Prop}
    $X$ :: schemeとする.
    representable sheaf :: $\ftor{X}$はfppf topologyを備えるsite上でsheafである.
\end{Prop}
\begin{proof}
\end{proof}

\begin{Thm}
    
\end{Thm}

\begin{Prop}
    任意のpresheafはcolimit of representable sheavesとして表現できる
    (Kan拡張に関連して得られる.).
\end{Prop}
\begin{proof}
\end{proof}

\section{Definitions : Points and Stalks.}
以下はsmall/big etale siteのみで使われるものである.
\begin{Def}[Geometric Point, Etale Neighborhood, \cite{ASS} 1.3.15.]
\begin{myenum}{\roman*}
    \item 
    $X$ :: schemeに対し,
    $k$ :: separabely closed fieldを用いて
    $\bar{x} \colon \Spec k \to X$と表される射をgeometric pointと呼ぶ.

    \item
    geometric point :: $\bar{x} \colon \Spec k \to X$について,
    $\bar{x}$のetale neighborhoodとは
    $U \to X$がetaleであるような以下の可換図式のことである.
    \[\xymatrix{
        {} & U \ar[d]\\
        \Spec k \ar[r]_-{\bar{x}} \ar[ru]& X
    }\]

    \item
    geometric point :: $\bar{x} \colon \Spec k \to X$について,
    $\bar{x}$の$2$つのetale neighborhood :: $U_1, U_2$を考える.
    この時,$U_1$と$U_2$の間の射とは,
    以下の図式を可換にするmorphism of schemes :: $\eta \colon U_1 \to U_2$のことである.
    \begin{equation*}
    \begin{xy}
        (0,17.32) *{U_1}="U1", (20, 17.32) *{U_2}="U2",
        (-20, 3) *{\Spec k}="k", (10,0) *{X}="X",
        \ar "U1";"X" \ar "U2";"X" \ar "U1";"U2"^{\eta}
        \ar "k";"U1" \ar|(.59)\hole "k";"U2" \ar "k";"X"
    \end{xy}
    \end{equation*}
\end{myenum}
\end{Def}

\begin{Remark}
    geometric pointの定義に
    separabely closed fieldでなくalgebraically closed fieldを用いることもある.

\end{Remark}

\begin{Remark}
    より一般的なpoint of siteの定義が存在する(\cite{StacksProj} Tag 04JU).
    これはetaleか否かに依らず採用できる.
    しかしこの一般的な定義は複雑であるし,
    我々はsmall/big etale siteしか扱わないので,
    我々は以上の定義のみ用いる.
\end{Remark}

\begin{Def}[Stalk, \cite{ASS} 1.3.15.]
    $X$ :: scheme,
    $\shF \in \Et(X)$あるいは$\shF \in \ET(X)$とする.
    さらに$\bar{x} \colon \Spec k \to X$ :: geometric pointとする.
    $\bar{x}$に対して$\bar{x}$のetale neighborhoodが成す圏を$I_{\bar{x}}$とする,

    \begin{enumerate}[label=(\roman*)]
        \item 
        $I_{\bar{x}}$を用いてstalk of $\shF$ at $\bar{x}$を
        \[ \shF_{\bar{x}}:=\varinjlim_{U \in I_{\bar{x}}} \shF(U) \]
        と定義する.

        \item
        $U \in I_{\bar{x}}$について,$\shF(U)$から$\shF_{\bar{x}}$への標準的射がある.
        この射による$s \in \shF(U)$の像を$s_{\bar{x}}$と表し,
        germ of $s$ at $\bar{x}$と呼ぶ.
    \end{enumerate}
\end{Def}

\section{Definitions : Morphism of Shaves.}
\begin{Def}[Injective, Surjective]
\end{Def}

\begin{Def}[Representable Morphism.]
\end{Def}

\section{Examples : Morphism of Shaves.}

\section{Propositions : Morphism of Shaves.}
\begin{Prop}
    inj/surj/iso at stalk.
\end{Prop}

\bibliographystyle{jplain}
\bibliography{reference}
\end{document}
