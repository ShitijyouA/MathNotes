\documentclass[a4paper, dvipdfmx]{jsarticle}
\usepackage{macros}

\begin{document}
\title{ゼミノート \#12 \\ Quotients of Algebraic Spaces}
\author{七条彰紀}
\maketitle
\tableofcontents
\vspace{10pt}

このノートでは一貫して,
$S$ :: algebraic spaceとし,
groupoid in algebraic $S$-space :: $R \parto{s}{t} X$を考える.

\section{Notes on Topology}
\subsection{Constructible Topology}
以下を参考にした.
\begin{itemize}
    \item D.Rydh(2010) ``Submersions and effective descent of étale morphisms"
    \item Lecture note by B.Conrad,
        \url{http://virtualmath1.stanford.edu/~conrad/Perfseminar/Notes/L3.pdf}
    \item \cite{SP} 08YF \url{https://stacks.math.columbia.edu/tag/08YF}
\end{itemize}

\begin{Def}
    $X$ :: topological spaceとする.
    \begin{enumerate}
    \item 
        $X$のlocally closed subsetとは,
        closed subsetとopen subsetの共通部分で表せるsubsetである.

    \item
        $X$のconstructible setとは,
        $X$の有限個のlocally closed subsetの和集合で表せるsubsetのことである.

    \item
        $U \subseteq X$が$X$のlocally constructible setであるとは,
        $U$のある開被覆$\{U_i\}$について,
        各$U \cap U_i$がconstructible setである,ということ.

    \item
        $X$のconstructible topologyとは,
        $X$のconstructible setを開基とする位相のことである.

    \item
        有限個とは限らない$X$のconstructible setの,
        和集合をind-constructible subsetと呼び,
        共通部分をpro-constructible subsetと呼ぶ
        \footnote{ ``ind-"はinductive limitを意味し,``pro-"はprojective limitを意味する. }.
    \end{enumerate}
\end{Def}

次の命題は直ちに分かることなので証明しない.
\begin{Prop}
    $X$ :: topological spaceとする.
    \begin{enumerate}
    \item
        $X$のopen subsetとclosed subsetはconstructible setである.

    \item
        有限個のconstructible setの和,共通部分はconstructible setである.
        constructible setの補集合もconstructible setである.

    \item
        $X$のconstructible topologyに於けるopen subsetはind-constructible subsetに限る.
        同様に,closed subsetはpro-constructible subsetに限る.
    \end{enumerate}
\end{Prop}

\begin{Prop}
    \begin{enumerate}
    \item 
        qcqs($=$quasi-compact and quasi-separated) schemeのpro-constructible subsetは,
        affine schemeからの射の像に限る(Rydh10, Prop1.1).

    \item
        locally of finite presentation morphismはconstructible topologyにおいてsubmersive.

    \item
         quasi-compact morphismはconstructible topologyにおいてsubmersive.
    \end{enumerate}
\end{Prop}

constructible topologyはspectral space
\footnote
{
    spectral spaceとは,以下の性質をもつ位相空間:
    sober, quasi-compact,
    the intersection of two quasi-compact opens is quasi-compact,
    and the collection of quasi-compact opens forms a basis for the topology
    (\cite{SP} 08FG).
}
と共に扱われることが多い.
例えばqcqs schemeのunderlying spaceはspectralである.

\subsection{Equivalence Relation on Topological Space Induced by Groupoid}

\section{Quotients}
\subsection{Definitions}

\begin{Def}[equivariant morphism]
\end{Def}

\begin{Def}[$j, j_{Y}$]
\end{Def}
stabilizerはまたの機会に定義する.

\begin{description}[labelindent=1cm, leftmargin=1.5cm]
    \item[Zariski quotient             ]
    \item[Constructible quotient       ]
    \item[Topological quotient         ]
    \item[Strongly topological quotient]
    \item[Geometric quotient           ]
    \item[Strongly geometric quotient  ]
\end{description}

\begin{Def}[universal, uniform quotient]
\end{Def}

\begin{Remark}
    geometric quotient in \cite{GIT}
    \begin{itemize}
        \item 
    \end{itemize}
\end{Remark}

\subsection{Propositions : Paraphase}

\bibliographystyle{jplain}
\bibliography{../references/stacks_reference}
\end{document}
