\documentclass[a4paper, dvipdfmx]{jsarticle}
\usepackage{macros}

\newcommand{\cons}{\mathrm{cons}}

\begin{document}
\title{ゼミノート \#12 \\ Quotients of Algebraic Spaces}
\author{七条彰紀}
\maketitle
\tableofcontents
\vspace{10pt}


\section{Notes on Topology}
\subsection{Constructible Topology}
以下を参考にした.
\begin{itemize}
    \item D.Rydh(2010) ``Submersions and effective descent of étale morphisms"
    \item Lecture note by B.Conrad,
        \url{http://virtualmath1.stanford.edu/~conrad/Perfseminar/Notes/L3.pdf}
    \item \cite{SP} 08YF \url{https://stacks.math.columbia.edu/tag/08YF}
\end{itemize}

\begin{Def}
    $X$ :: topological spaceとする.
    \begin{enumerate}
    \item 
        $X$のlocally closed subsetとは,
        closed subsetとopen subsetの共通部分で表せるsubsetである.

    \item
        $X$のconstructible setとは,
        $X$の有限個のlocally closed subsetの和集合で表せるsubsetのことである.

    \item
        $U \subseteq X$が$X$のlocally constructible setであるとは,
        $U$のある開被覆$\{U_i\}$について,
        各$U \cap U_i$がconstructible setである,ということ.

    \item
        $X$のconstructible topologyとは,
        $X$のconstructible setを開基とする位相のことである.
        $X$のunderlying setに$X$のconstructible topologyを
        与えた位相空間を$X_{\cons}$と書く.

    \item
        有限個とは限らない$X$のconstructible setの,
        和集合をind-constructible subsetと呼び,
        共通部分をpro-constructible subsetと呼ぶ
        \footnote{ ``ind-"はinductive limitを意味し,``pro-"はprojective limitを意味する. }.

    \item
        map of topological spaces :: $f \colon X \to Y$について,
        $f^{\cons}$をconstructible topologyでのmapとする.
        (map of setsとしては$f=f^{\cons}$である.)
    \end{enumerate}
\end{Def}

\begin{Prop}
    $X$ :: topological spaceとする.
    \begin{enumerate}
    \item
        $X$のopen subsetとclosed subsetはconstructible setである.

    \item
        有限個のconstructible setの和,共通部分はconstructible setである.
        constructible setの補集合もconstructible setである.

    \item
        $X$のconstructible topologyに於けるopen subsetはind-constructible subsetに限る.
        同様に,closed subsetはpro-constructible subsetに限る.

    \item
        map of topological spaces :: $f \colon X \to Y$について,
        $f^{\cons}$ :: continuous.
    \end{enumerate}
\end{Prop}
\begin{proof}
    自明.   
\end{proof}

\begin{Prop}
    \begin{enumerate}
    \item 
        qcqs($=$quasi-compact and quasi-separated) schemeのpro-constructible subsetは,
        affine schemeからの射の像に限る().

    \item
        locally of finite presentation morphismはconstructible topologyにおいてopen.

    \item
         quasi-compact morphismはconstructible topologyにおいてclosed.
    \end{enumerate}
\end{Prop}
\begin{proof}
    (i)はRydh10のProp1.1である.
    (ii)はChevalley's theoremからの帰結.
    (iii)はlocallyに調べれば容易に分かる.
\end{proof}

\begin{Prop}{\cite{Rydh10} Prop1.7}
    morphism of schemes :: $f \colon X \to Y, g \colon Y' \to Y$を考え,
    $f$の$g$によるpullbackを$f'$と書く.
    \begin{enumerate}
    \item 
        $P$をopen, closed, submersiveのいずれかとする.
        $g$がsubmersiveならば,
        $f'$ :: Pと$f$ :: Pは同値.
    \item 
        $P$をuniversally open, universally closed, universally submersive, separatedのいずれかとする.
        $g$がuniversally submersiveならば,
        $f'$ :: Pと$f$ :: Pは同値.
    \item 
        $g^{\cons}$がuniversally submersiveならば,
        $f'$ :: quasi-compactと$f$ :: quasi-compactは同値.
    \end{enumerate}
\end{Prop}
\begin{proof}
    (TODO)
    (iii)だけ証明を与える.
\end{proof}

constructible topologyはspectral space
\footnote
{
    spectral spaceとは,以下の性質をもつ位相空間:
    sober, quasi-compact,
    the intersection of two quasi-compact opens is quasi-compact,
    and the collection of quasi-compact opens forms a basis for the topology
    (\cite{SP} 08FG).
}
と共に扱われることが多い.
例えばqcqs schemeのunderlying spaceはspectralである.

\subsection{Equivalence Relation on Topological Space Induced by Groupoid}

\section{Quotients}
以降は$S$ :: algebraic spaceとし,
groupoid in algebraic $S$-space :: $R \parto{s}{t} X$を考える.

\subsection{Definitions}
\begin{Def}[equivariant morphism]
    morphism :: $q \colon X \to Y$について,
    $q \circ s=q \circ t$であるとき,
    $q$をequivariant morphismという.
\end{Def}

\begin{Def}[$j, j_{Y}$]
    $s,t \colon R \to X$から$X \times_{S} X$の普遍性により得られる
    射 :: $R \to X \times_{S} X$を$j$と書く.

    また,equivariant morphism :: $q \colon X \to Y$について,
    $s,t$から$X \times_{Y} X$の普遍性により得られる
    射 :: $R \to X \times_{Y} X$を$j_{/Y}$と書く.
\end{Def}

\begin{Remark}
    fiber productの普遍性から,
    $j_{/Y}$に$X \times_{Y} X \to X \times_{S} X$を合成すると$j$に一致する.
\end{Remark}

stabilizerはまたの機会に定義する.

\begin{description}[labelindent=1cm, leftmargin=1.5cm]
    \item[Zariski quotient             ]

    \item[Constructible quotient       ]

    \item[Topological quotient         ]

    \item[Strongly topological quotient]

    \item[Geometric quotient           ]

    \item[Strongly geometric quotient  ]

\end{description}

\begin{Def}[universal, uniform quotient]
\end{Def}

\begin{Remark}
    geometric quotient in \cite{GIT}
    \begin{itemize}
        \item $q$ :: surjective and equivariant.
        \item $\shO_{Y}=(q_* \shO_X)^{R}$.
        \item 任意の点$y \in Y$について,$q^{-1}(y)$はただ一つのorbitからなる.
        \item
            $W_1, W_2 \subseteq X$ :: disjoint closed subsetについて
            $\cl_Y(q(W_1)), \cl_Y(q(W_2))$ :: disjoint.
    \end{itemize}

    以下のように言い換えても良い.
    \begin{itemize}
        \item $q$ :: Zariski quotient.
        \item $j_{/Y}$ :: surjective.
        \item $\shO_{Y}=(q_* \shO_X)^{R}$.
        \item $q$のopen immersionによるpullbackも上記を満たす.
    \end{itemize}
    ref. D. Mumford's Geometric Invariant Theory(E.Viehweg)
\end{Remark}

\subsection{Propositions : Paraphase}

\bibliographystyle{jplain}
\bibliography{../references/stacks_reference}
\end{document}
