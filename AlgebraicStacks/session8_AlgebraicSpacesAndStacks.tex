\documentclass[a4paper, dvipdfmx]{jsarticle}
\usepackage{macros}

\newcommand{\Diag}{\Delta}
\newcommand{\rest}{\vspace{5pt}}
\newcommand{\rep}{{\color{blue}\#}}
\newcommand{\arpb}{\ar[lu, phantom, "\text{p.b.}"]}

\begin{document}
\title{ゼミノート \#8 \\ Algebraic-ness of Spaces and Stacks}
\author{七条彰紀}
\maketitle
\tableofcontents

\vspace{10pt}

affine scheme, scheme. algebraic space, algebraic stackという貼り合わせの連なりを意識した定義をした後,
algebraic stackがschemeの貼り合わせとして定義できることを示す.
algebraic spaceとalgebraic stackの定義は全く平行に行われる.
そのことが分かりやすい記述を志向する.

\section{Fiber Product}
\begin{Prop}
    任意のsite :: $\cat{C}$について,
    $\cat{C}$上のsheafの圏$\Shv{C}$はfiber poductを持つ.
\end{Prop}
\begin{proof}
    二つの射$\shF \to \shH, \shG \to \shH$をとる.
    \[ \cat{C} \ni U \mapsto \shF(U) \times_{\shH(U)} \shG(U) \]
    とすれば,これはfiber productとなる.
\end{proof}

$\cat{B}$ :: categoryとする.
この時,$\FibBP{B}$は以下のような圏であった.
\begin{description}[labelindent=1cm]
    \item[Objects:] fibered categories over $\cat{B}$.
    \item[Arrows:]  base-preserving natural transtormation.
\end{description}
新たに圏$\CFG{B}$を以下のように定義する.
\begin{description}[labelindent=1cm]
    \item[Objects:] categories fibered in groupoids(CFG) over $\cat{B}$.
    \item[Arrows:] base-preserving natural transtormation.
\end{description}

重要なのは次の存在命題である.
\begin{Prop}[\cite{Gomez} p.10]
    任意の圏$\cat{C}$について,
    $\FibBP{B}$と$\CFG{B}$はfibered productを持つ.
\end{Prop}
%% {{{
\begin{proof}
    $\FibBP{B}$の射$F \colon \fibX \to \fibZ$と$F \colon \fibY \to \fibZ$をとり,
    $F, G$のfiber productを実際に構成する.

    \step{圏$\cat{P}$の構成}
    圏$\cat{P}$を以下のように定義する.
    \begin{description}
        \item[Objects:]
            以下の$4$つ組
            \begin{enumerate}
                \item $b \in \cat{B}$,
                \item $x \in \fibX(b)$,
                \item $y \in \fibY(b)$,
                \item $\fibZ$の恒等射上の同型射$\alpha \colon Fx \to Gy$.
            \end{enumerate}
        
        \item[Arrows:] \mnewline
            射$(b, x, y, \alpha) \to (b', x', y', \alpha')$は,
            二つの射$\phi_{\fibX} \colon x \to x', \phi_{\fibY} \colon y \to y'$であって以下を満たすもの:
            $\phi_{\fibX}, \phi_{\fibY}$は同じ射$b' \to b$上の射で,
            以下の可換図式を満たすもの.
            \[
            \begin{tikzcd}
                Fx \ar[d, red, "F\phi_{\fibX}"']\ar[r, "\alpha"]& Gy \ar[d, red, "G\phi_{\fibY}"]\\
                Fx' \ar[r, "\alpha'"']& Gy'
            \end{tikzcd}
            \]
    \end{description}

    \step{Cartesian Lifting in $\cat{P}$.}
    この圏は$\pi \colon (b,x,y,\alpha) \mapsto b$によってfibered categoryと成る.
    $f \colon b' \to b$の$\xi=(b, x, y, \alpha)$に関する
    Cartesian Lifting :: $f^*\xi \to \xi$は次のように定義される.
    \[
        \chi_{\xi}=(f^*x \xrightarrow{\chi_x} x, f^*y \xrightarrow{\chi_y} y)
        \colon
        f^*\xi=(b', f^*x, f^*y, \bar{\alpha}) \to \xi.
    \]
    ここで$\chi_x \colon f^*x \to x$は$f$の$x$に関するCartesian Liftingである.
    $\chi_y$も同様.
    さらに$\bar{\alpha}$は以下のTriangle Liftingで得られる射である
    \footnote
    {
        $f^*\alpha \colon f^*Fx \to f^*Gy$とは異なる.
        同型$Ff^*x \to F^*Fx, Gf^*y \to f^*Gy$と$f^*\alpha \colon Fx \to Gy$を
        合成しても$\bar{\alpha}$は得られる.
    }.
    \begin{center}
    \begin{tikzpicture}[mybox/.style={draw, inner sep=5pt}]
    \node[mybox] (X) at (0,3){%
        \begin{tikzcd}
            Ff^*x \ar[d, "F\chi_x"']\ar[r, red, "\bar{\alpha}"]& Gf^*y \ar[d, "G\chi_y"]\\
            Fx \ar[r, "\alpha"']& Gy
        \end{tikzcd}
    };
    \node[mybox] (B) at (0,0){%
        \begin{tikzcd}
            b' \ar[d, "f"']\ar[r, red, "\id"]& b' \ar[d, "f"]\\
            b \ar[r, "\id"]& b
        \end{tikzcd}
    };

    \node [above=5pt of X] {in $\fibZ$};
    \node [below=5pt of B] {in $\cat{B}$};
    \draw [->, line width=1.5pt] (X) edge (B);
    \node at (0.3,1.55) {$\pi_{\fibZ}$};
    \end{tikzpicture}
    \end{center}
    fibered categoryの間の射はcartesian arrowを保つので$F\chi_x, G\chi_y$もcartesian.
    したがってTriangle Liftingが出来る.
    $\bar{\alpha}$が同型であることはTriangle Liftingの一意性を用いて容易に証明できる.
    また,この可換図式から$\chi_{\xi}$が$\cat{P}$の射であることも分かる.

    \step{$\fibX, \fibY, \fibZ$がcategory fibered in groupoids(CFG)ならば$\cat{P}$もCFG}
    $\fibX, \fibY, \fibZ$がCFGならば$\cat{P}$もCFGとなる.
    実際,
    $\phi_{\fibX} \colon x \to x'$と$\phi_{\fibY} \colon y \to y'$の両方がcartesianならば
    $(\phi_{\fibX}, \phi_{\fibY}) \colon (b, x, y, \alpha) \to (b', x', y', \alpha')$はcartesianである.

    \step{$\cat{P}$からの射影写像.}
    定義から明らかに$\pr_1 \colon \cat{P} \to \fibX, \pr_2 \colon \cat{P} \to \fibY$が定義できる.
    射の定義にある可換図式は,
    以下の$A$がnatural transformationであることを意味している.
    \begin{defmap}
        A \colon & F\pr_1& \to& G\pr_2 \\
        {}& (F\pr_1)((b, x, y, \alpha))=Fx& \mapsto& \alpha(Fx)=\alpha((F\pr_1)((b, x, y, \alpha)))
    \end{defmap}
    $A$がbase-preservingであることは$\alpha$が恒等射上のもの(i.e. $\pi_{\fibZ}(\alpha)=\id$)であることから,
    isomorphismであることは$\alpha$が同型であることから示される.

    \step{$\cat{P}$ :: fiber product.}
    今,
    $\fibW \in \FibBP{B}$と
    射$S \colon \fibW \to \fibX, T \colon \fibW \to \fibY$及び
    base-preserving isomorphism :: $\delta \colon FS \to GT$をとる.
    base-preservingなので,任意の$w \in \fibW$について$\pi_{\fibZ}(FS(w))=\pi_{\fibZ}(GT(w))$.
    そこで次のように関手が定義できる.
    \begin{defmap}
        H\colon & \fibW& \to& \cat{P} \\
        \mathbf{Object}& w& \mapsto& (\pi_{\fibZ}(FS(w)), Sw, Tw, \delta_{w}) \\
        \mathbf{Arrow}& [\phi \colon w \to w']& \mapsto& (S\phi \colon Sw \to Sw', T\phi \colon Tw \to Tw')
    \end{defmap}
    このように置くと,$S=\pr_1 H, T=\pr_2 H$となる.
    逆に$S \iso \pr_1 H', T \iso \pr_2 H'$となる
    関手$H' \colon \fibW \to \cat{P}$は$H$と同型に成ることが直ちに分かる.
\end{proof}
%% }}}
\begin{Remark}
    session4 命題4.5より,CFGの恒等射上の射は同型射である.
    したがって$\alpha \colon Fx \to Gy$に課せられた条件は,
    $\fibZ$がCFGならば一つしか無い.
\end{Remark}

\begin{Example}
    representable fibered categoryのfiber product.
    
    sheafに対応するfibered categoryの
    fiber product $\int \shF \times_{\int \shH} \int \shG$に対応するsheafは,
    sheafのfiber productに対応する.
    \[
        \left( \int \shF \times_{\int \shH} \int \shG \right)(-)
        =\shF \times_{\shH} \shG
        \quad \in \Shv{C}
    \]
\end{Example}

我々が扱うのはstackであるから,
stackという性質がfiber productで保たれていて欲しいが,果たしてそうなる.
\begin{Prop}[\cite{ASS} Prop 4.6.4]
    $\fibX, \fibY, \fibZ$ :: stack over $\cat{C}$とし,
    morphism of stacks :: $F \colon \fibX \to \fibZ, G \colon \fibY \to \fibZ$をとる.
    この時,$F, G$についてのfiber product :: $\fibX \times_{\fibZ} \fibY$はstackである.
\end{Prop}
したがって結局$\FibBP{B}, \CFG{B}$と,
stackの圏及びstack in groupoidsの圏はfiber productを持つ.
我々が実際に扱うのはstack in groupoidsである.
%% {{{
\begin{proof}
    $\fibP=\fibX \times_{\fibZ} \fibY$とおく.
    $U \in \cat{C}, \covU=\{\phi_i \colon U_i \to U\} \in \Cov(U)$を任意にとり,
    $\epsilon_{\covU} \colon \fibP(U) \to \fibP(\covU)$を計算する.

    \step{$\epsilon_{\covU}(\xi)$.}
    $\xi=(b, x, y, \alpha)$をとり,$\epsilon_{\covU}(\xi)$を計算する.
    まず$\{\phi_i^*\xi\}_i$は既に詳しく説明した.
    注意が必要なのは同型$\sigma_{ij} \colon \pr_2^*\phi_i^*\xi \to \pr_1^*\phi_j^*\xi$である.
    可換性は以下の図式から分かる.
    \begin{center}
    \begin{tikzpicture}[mybox/.style={draw, inner sep=5pt}]
    \node[mybox] (X) at (0,5){%
        \begin{tikzcd}
            F\pr_2^*\phi_j^*x \ar[d, "\sigma_{ij}^{x}"']\ar[r, red, "\bar{\alpha}"]&
            G\pr_2^*\phi_j^*y \ar[d, "\sigma_{ij}^{y}"]\\
            F\pr_1^*\phi_i^*x \ar[d]\ar[r, red, "\bar{\alpha}"']& G\pr_1^*\phi_i^*y \ar[d]\\
            Fx \ar[r, "\alpha"']& Gy
        \end{tikzcd}
    };
    \node[mybox] (B) at (0,0){%
        \begin{tikzcd}
            b' \ar[r, red, equal]\ar[d, equal]& b' \ar[d, equal] \\
            b' \ar[r, red, equal]\ar[d, "\phi_j \circ\, \pr_2"']& b' \ar[d, "\phi_i \circ\, \pr_1"]\\
            b \ar[r, equal]& b
        \end{tikzcd}
    };

    \node [above=5pt of X] {in $\fibZ$};
    \node [below=5pt of B] {in $\cat{B}$};
    \draw [->, line width=1.5pt] (X) edge (B);
    \node at (0.3,2.45) {$\pi_{\fibZ}$};
    \end{tikzpicture}
    \end{center}
    
    \step{$\epsilon_{\covU}(\kappa)$.}
    (TODO)

\end{proof}
%% }}}
\section{Diagonal Map}
\begin{Remark}
    以降は$S$ :: schemeをとり,
    $\cat{C}=\ET(S)$ :: big etale site over $S$上の
    sheafあるいはstack in groupoidsのみを考える.
\end{Remark}

\begin{Def}[Diagonal Map]
    sheafあるいはstack in groupoids over $S$:: $\fibX/S$(すなわち射$\fibX \to S$)
    のdiagonal map :: $\Diag$とは,
    以下の可換図式に収まる射のことである.
    \[\begin{tikzcd}
            \fibX \ar[rrd, bend left, "\id"]\ar[rdd, bend right, "\id"']\ar[rd, "\Diag"]&
                                                            & \\
                                                            &
        \fibX \times \fibX \ar[r]\ar[d]\ar[rd, phantom, "\text{p.b.}"]& \fibX \ar[d]\\
          &\fibX \ar[r]& S
    \end{tikzcd}\]
\end{Def}

\section{Local Property of Scheme/Morphism of Them}
\begin{Def}[\cite{IrrOfMg} p.100, Local Property for the topology.]
    $S$ :: schemeとし,$(\Sch/S)$上のsite :: $\cat{C}$を考える.
    $X, Y$ :: schemeとし,
    $\{\phi_i \colon X_i \to X\} \in \Cov(X), \{\psi_i \colon Y_i \to Y\} \in \Cov(Y)$を任意に取る.
    \begin{enumerate}
        \item 
            $P$をschemeの性質とする.
            $P$がlocal for the topologyであるとは,以下が成り立つということ: \mnewline
            $X$が$P$であることは,全ての$U_i$が$P$であることと同値.
        \item
            $P$をschemeの射の性質とする.
            $P$がlocal on the sourceであるとは,以下が成り立つということ:\mnewline
            $f \colon X \to Y$が$P$であることは,
                全ての$f \circ \phi_i$が$P$であることと同値.
        \item
            $P$をschemeの射の性質とする.
            $P$がlocal on the targetであるとは,以下が成り立つということ:\mnewline
            $f \colon X \to Y$が$P$であることは,
                全ての$\pr_2 \colon X \times_Y Y_i \to Y_i$が$P$であることと同値.
        \item
            (\cite{ASS} 5.1.3)
            $P$をschemeの射の性質とする.
            以下が全て成り立つ時,$P$はstableであると呼ばれる.
            \begin{itemize}
                \item 任意の同型は$P$.
                \item $P$は,射の合成で保たれる.
                \item $P$は,任意の$\cat{C}$の射によるbase changeで保たれる.
                \item local on the target.
            \end{itemize}

        \item
            (\cite{ChAlg} p.33, \cite{Gomez} p.16)
            $P$をschemeの射の性質とする.
            以下が全て成り立つ時,$P$はlocal on the source and targetであると呼ばれる.:
            $X' \to Y' \times X, Y' \to Y$がcoverであるような,
            次の形の任意の可換図式をとる.
            \[
            \begin{tikzcd}
                X' \ar[r] \ar[rd, "f'"']& Y' \times X \ar[r]\ar[d]& X \ar[d, "f"]\\
                {} & Y' \ar[r]& Y \ar[lu, phantom, "\text{p.b.}"]
            \end{tikzcd}
            \]
            この時,$f$が$P$であることは$f'$が$P$であることと同値.
    \end{enumerate}
\end{Def}

\begin{Example}[\cite{StacksProj} Tag 0238]
    etale topologyでの定義を挙げる.
    \begin{description}
    \item[local for the topologyである性質の例] \mnewline
        locally Noetherian, reduced, normal, regular.

    \item[local on the sourceである性質の例] \mnewline
        flat, locally of finite presentation, locally of finite type, open,
        smooth, etale, unramified, locally quasi-finite.

    \item[local on the targetである性質の例] \mnewline
        quasi-compact, quasi-separated, universally closed, separated, surjective,
        locally of finite type, locally of finite presentation,
        proper, smooth, etale, unramified, flat.

    \item[local on the source and targetである性質の例] \mnewline
        flat, locally of finite presentation, locally finite type,
        smooth, etale, unramified...
    \end{description}
\end{Example}

\begin{Remark}
    ``local on the source and target"は,
    algebraic space, algebraic stackの射について性質を定めるときに必要に成る.
    この定義は文献に寄って数種類ある.
    私が知る限りのものを以下に列挙する.
    \begin{description}
        \item[SP] \mnewline
            \cite{StacksProj} Tag 04QZ.

        \item[DM] \mnewline
            $X, Y$をschemeとし,
            $\{\phi_i \colon X_i \to X\} \in \Cov(X), \{\psi_i \colon Y_i \to Y\} \in \Cov(Y)$を任意のcoverとする.
            $\{ f_i \colon X_i \to Y_i \}$を以下の可換図式を満たす射の族とする.
            \[
            \begin{tikzcd}
                X_i \ar[r, "\phi_i"]\ar[d, "f_i"']&  X \ar[d, "f"]\\
                Y_i \ar[r, "\psi_i"']& Y
            \end{tikzcd}
            \]
            この時,射$f$が$P$であることと,全ての射$f_i$が$P$であることは同値.
            \cite{IrrOfMg}, p.100より.
        
        \item[DM'] \mnewline
            以下のschemeの可換図式が成立しているとする.
            \[
            \begin{tikzcd}
                X' \ar[r]\ar[d, "f'"']&  X \ar[d, "f"]\\
                Y' \ar[r]& Y
            \end{tikzcd}
            \]
            ただし$X' \to X, Y' \to Y$はcoverである.
            この時,射$f$が$P$であることと,射$f'$が$P$であることは同値.
            \cite{StacksProj} Tag 04R4でDeligne-Mumfordの定義として参照されている.
        
        \item[ST] \mnewline
            local on the sourceかつlocal on the target.

        \item[ST+] \mnewline
            次の5条件を合わせたもの.
            \begin{itemize}
                \item 同型について成立する,
                \item stable under composition,
                \item stable under base change,
                \item local on the source,
                \item local on the target.
            \end{itemize}
            \cite{ASS} Def 5.1.3, 5.4.11で採用されている.

        \item[La] \mnewline
            $X, Y$ :: schemeとし,射$Y' \to Y, X' \to Y' \times X$をcoverとする.
            この時,$f \colon X \to Y$と合わせると次の可換図式が得られる.
            \[
            \begin{tikzcd}
                X' \ar[r] \ar[rd, "f'"']& Y' \times X \ar[r]\ar[d]& X \ar[d, "f"]\\
                {} & Y' \ar[r]& Y \ar[lu, phantom, "\text{p.b.}"]
            \end{tikzcd}
            \]
            この時,$f$が$P$であることは$f'$が$P$であることと同値.
            \cite{ChAlg} p.33, \cite{Gomez} p.16で採用されている.
        
        \item[La'] \mnewline
            $X, Y$ :: schemeとし,射$Y' \to Y, X' \to X$をcoverとする.
            この時,$f \colon X \to Y$と合わせると次の可換図式が得られる.
            \[
            \begin{tikzcd}[row sep=30pt]
                X' \times_Y Y' \ar[rr]\ar[d, "f'"']& {} & Y' \ar[d]\\
                X' \ar[r]& X \ar[r, "f"']& Y \ar[llu, phantom, "\text{p.b.}"]
            \end{tikzcd}
            \]
            この時,$f$が$P$であることは$f'$が$P$であることと同値.
    \end{description}
    
    強弱関係は次の通り.
    \[
    \begin{tikzcd}[every arrow=Rightarrow]
        ST+ \ar[r, Rightarrow]& SP \ar[r, Rightarrow]&
        DM \ar[r, Rightarrow]\ar[rd, Rightarrow]& DM' \ar[r, Rightarrow]& La \ar[r, Rightarrow]& La' \\
                                                & & & ST \ar[u, Rightarrow]& &
    \end{tikzcd}
    \]
    SP$\implies$DM$\implies$STは\cite{StacksProj} Tag 04R4による.
    SP$\implies$DM$\implies$ST, DM'$\implies$La$\implies$La'の
    それぞれの$\implies$は逆が成り立たないことも分かっている.
    また,DM'とlocal on the targetを合わせたものはDMと同値である.
    
    我々としては,「便利な性質」をもち,かつ弱い定義を取りたい.
    後に示すとおり,Laを仮定すれば十分「便利」である.
\end{Remark}

\section{Algebraic Space}
\subsection{Representable Ones.}
    \begin{Def}[Representable Space]
        stack :: $\fibX$がrepresentable (by scheme)であるとは,
        あるscheme :: $X$が存在し,$\fibX \iso X=\Sch/X$であるということ.
    \end{Def}

    \begin{Def}[Representable Morphism of Spaces]
        morphism of spaces :: $f \colon \shX \to \shY$がrepresentable (by scheme)であるとは,
        任意の$S$-scheme :: $U$と$\cat{C}$の射$U \to \shY$について,
        fiber product :: $U \times_{\shY} \shX$がrepresentable (by scheme)であるということ.
    \end{Def}

    \begin{Prop}[Representable Diagonal Morphism]
        $\fibF$ :: stack on $\tau(S)$
        以下は同値である.
        \begin{enumerate}[label=(\roman*)]
            \item
                $\Diag \colon \shX \to \shX \times_S \shX$は表現可能.
            \item
                任意のscheme :: $U$と射$U \to \shX$について,
                $U \to \fibX$ :: representable.
            \item
                任意のscheme :: $U, V$と射$u \colon U \to \shX, v \colon V \to \shX$について
                $U \times_{\shX} V$ :: representable.
        \end{enumerate}
    \end{Prop}
    \begin{proof}
        (ii)$\iff$(iii)はrepresentable morphismの定義から直ちに分かる.
        (i)$\iff$(iii)は以下がpullback diagramであることから分かる.
        \[
            \begin{tikzcd}[sep=30pt]
            U \times_{\shX} V \ar[r]\ar[d]& U \times_{S} V \ar[d, "u \times v"]\\
            \shX \ar[r, "\Diag"']& \shX \times_{S} \shX \ar[lu, phantom, "\text{p.b.}"]
        \end{tikzcd}
        \]
        (TODO: もう少し詳しく.)
    \end{proof}

    \begin{Def}[Property of Representable Spaces/Morphism of Them]
        \enumfix
    \begin{enumerate}
    \item
        $\mathcal{P}$をschemeの性質でlocal for etale topologyであるものとする.
        この時,representable space :: $\shX$が性質$\mathcal{P}$を持つとは,
        $\shX$をrepresentするschemeが性質$\mathcal{P}$を持つということである.

    \item
        $\mathcal{P}$をmorphism of schemesの性質で
        local on the targetかつstable under base changeであるものとする.
        この時,representable morphism of spaces :: $f \colon \fibX \to \fibY$が性質$\mathcal{P}$を持つとは,
        任意の$U \in \cat{C}$と射$U \to \fibY$について,
        $\pr \colon \fibX \times_{\fibY} U \to U$
        (に対応するmorphism of algebraic schemes)が性質$\mathcal{P}$を持つということである.
    \end{enumerate}
    \end{Def}

\subsection{Definition of Algebraic Space}
    \begin{Def}[Algebraic Space]
        $S$ :: schemeとし,$\shX$をspace over $S$(すなわちbig etale site $\Et(S)$上のsheaf)とする.
        $\shX$がalgebraicであるとは,次が成り立つということである.
    \begin{enumerate}
        \item diagonal morphism :: $\Diag \colon \shX \to \shX \times_{S} \shX$がrepresentableである.
        \item scheme :: $U$からのetale surjective morphism :: $U \to \shX$が存在する.
    \end{enumerate}
        Algebraic spaceの射はspaceとしてのものである.
    \end{Def}
    
以下ではschemeの性質とschemeの射の性質をalgebraic spaceへ拡張する.

\subsection{Properties of Algebraic Space/Morphism of Algebraic Spaces}
    \begin{Def}[Property of Algebraic Spaces]
        \enumfix
    \begin{enumerate}
    \item
        $\mathcal{P}$をschemeの性質であって,
        local for etale topologyであるものとする.
        この時,algebraic stack :: $\shX$が性質$\mathcal{P}$を持つとは,
        $\shX$のあるatlasが性質$\mathcal{P}$を持つということである.

    \item
        algebraic stack :: $\shX$がquasi-compact
        \footnote{ 明らかに,これはlocal for etale topologyではない. }であるとは,
        $\shX$のあるatlasが性質$\mathcal{P}$を持つということである.
    \end{enumerate}
    \end{Def}

    \begin{Def}[Property of Morphism of Algebraic Spaces]
        $\mathcal{P}$をmorphism of schemeの性質であって,
        local on the source and targetであるものとする.
        以下の可換図式で,$V \to \shY, U \to V \times \shX$はcoverであるとする.
        \[
        \begin{tikzcd}
            U \ar[r] \ar[rd, "f'"']& V \times \shX \ar[r]\ar[d]& \shX \ar[d, "f"]\\
            {} & V \ar[r]& \shY \arpb
        \end{tikzcd}
        \eqno{(PM)}
        \]
        この時,\underline{morphism of algebraic spaces} :: $f \colon \shX \to \shY$が
        性質$\mathcal{P}$を持つとは,
        この可換図式にある$f'$(に対応するmorphism of scheme)が
        性質$\mathcal{P}$を持つということである.
    \end{Def}

    \begin{Lemma}
    \enumfix
    \begin{enumerate}[label=(\alph*)]
        \item
        $\shX$をrepresentable spaceとし,
        $P$をalgebraic spaceの性質とする.
        $f$がalgebraic spaceとして性質$P$を持つことと,
        representable spaceとして性質$P$を持つことは同値.

        \item 
        $f \colon \shX \to \shY$をrepresentable morphismとし,
        $P$をalgebraic spaceの射の性質とする.
        $f$がalgebraic spaceの射として性質$P$を持つことと,
        representable morphismとして性質$P$を持つことは同値.
    \end{enumerate}
    \end{Lemma}
    \begin{proof}
        $\shX$がscheme :: $X$で表現されるならば
        $X \to \Sch/X \iso \shX$がatlasなので(a)が成立する.

        以下の図式で$\id \colon V \to V$がetale surjectiveなので
        $U \to V \times \shX$もetale surjectiveである.
        またrepresentable morphismの性質は,
        schemeの射の性質としてlocal on the targetであるものに限っていた.
        したがって(b)が成立する.
        \[
        \begin{tikzcd}
            U \ar[r] \ar[d, "f''"']& V \times \shX \ar[r]\ar[d, "f'"]& \shX \ar[d, "f"]\\
            V \ar[r, "\id"']& V \ar[r]\arpb& \shY \arpb
        \end{tikzcd}
        \]
    \end{proof}

    \begin{Lemma}
        \enumfix
        \begin{enumerate}[label=(\alph*)]
        \item
            $P$をschemeの性質でlocal for etale topologyなものとする.\mnewline
            一つのetale surjective morphism :: $U \to \shX$について$U$が性質$P$を持つならば,\mnewline
            任意のetale surjective morphism :: $U \to \shX$について$U$が性質$P$を持つ.
        \item
            $P$をmorphism of schemeの性質であって,
            local on the source and targetであるものとする.\mnewline
            一つの$V \to \shY, U \to U \times \shX$の組み合わせについて
                図式$(PM)$の$f'$が性質$P$を持つならば,\mnewline
            任意の$V \to \shY, U \to U \times \shX$の組み合わせについて
                図式$(PM)$の$f'$が性質$P$を持つ.
    \end{enumerate}
    \end{Lemma}
    \begin{proof}
        \cite{StacksProj} Tag 06FM
    \end{proof}

    \begin{Lemma}
            $P$をalgebraic spaceの射の性質とする.
            \begin{enumerate}[label=(\alph*)]
        \item
            $P$がschemeの射の性質としてstable under base changeならば,
            algebraic spaceの射の性質としてもstable under base change.
        \item
            $P$がschemeの射の性質としてstable under compositionならば,
            algebraic spaceの射の性質としてもstable under composition.
    \end{enumerate}
    \end{Lemma}
    %% {{{
    \begin{proof}
        (a)は\cite{StacksProj} Tag 0CIIを参考にすれば良い.
        
        (b)を示す.準備として次を示す.
        \begin{Claim}
            $U$ :: schemeとする.
            $f \colon U \to \shX, g \colon \shX \to \shY$がetale, surjectiveならば,
            合成$g \circ f \colon U \to \shX \to \shY$もetale, surjectiveである.
        \end{Claim}
        \begin{proof}
            etale, surjectiveはschemeの射の性質として
            stable under base changeかつstable under compositionであることに注意する.
            
            $V \to \shY, W \to V \times_{\shY} \shX$をschemeからのetale surjective (e.s.)射とする.
            この時fiber productを組み合わせて以下の可換図式が得られる.
            (pullback lemmaを暗黙のうちに用いている.)
            \[
            \begin{tikzcd}
                W \times_{\shX} U \ar[r]\ar[d]& V \times_{\shX} U \ar[r]\ar[d]& U \ar[d, "f"]\\
                W \ar[r]\ar[dr]& V \times_{\shY} \shX \ar[r]\ar[d]\arpb&
                    \shX \ar[d, "g"]\arpb\\
                {} & V \ar[r]& \shY \arpb
            \end{tikzcd}
            \]
            この時,
            以下のように$W \times U \to W \to V$と$W \times U \to V \times U$がe.s.であることが示せる.
            \begin{itemize}
                \item $f \colon U \to \shX$ :: e.s.かつrepresentable $\implies$ $W \times U \to W$ :: e.s.
                \item $\shX \to \shY$ :: e.s. $\implies$ $W \to V$ :: e.s.
                \item $W \times U \to W, W \to V$ :: e.s. $\implies$ $W \times U \to W \to V$ :: e.s.
                \item $W \to V \times \shX$ :: e.s.かつrepresentable
                            $\implies$ $W \times U=W \times (V \times U) \to V \times U$ :: e.s.
            \end{itemize}
            etaleはlocal on the sourceな性質なので$V \times U \to V \times X \to V$もetale.
            またsurjectiveの圏論的な性質からsurjectiveであることも分かる.
            この二つから,representable morphism :: $g \circ f \colon U \to \shX \to \shY$はe.s.である.
        \end{proof}
        この主張を用いて(b)を示す.

        etale surjective射 :: $W \to \shZ, V \to W \times \shY, U \to V \times \shX$から次の可換図式が得られる.
        \[
        \begin{tikzcd}
            U \ar[r]\ar[rd, "f'"']& V \times_{\shY} \shX \ar[r]\ar[d]&
                W \times_{\shZ} \shX \ar[r]\ar[d]& \shX \ar[d, "f"]\\
            {} & V \ar[r]\ar[rd, "g'"']& W \times_{\shZ} \shY \ar[r]\ar[d]\arpb& \shY \ar[d, "g"]\arpb\\
            {} & {} & W \ar[r]& \shZ \arpb
        \end{tikzcd}
        \]
        定義から$g$が$P$であることと$g'$が$P$であることは同値.
        また,主張から$W \to W \times \shY \to \shY$はetale surjective射である.
        したがって再び定義から,
        $f$が$P$であることと$f'$が$P$であることは同値.
        最後に,$U \to V \times \shX \to W \times \shX$もetale surjectiveであるから,
        $g \circ f$が$P$であることと$g' \circ f'$が$P$であることは同値である.
    \end{proof}
    %% }}}

\section{Algebraic Stack}
節\ref{sec:def-algst}以外はalgebraic spaceの節にある定義文を
\begin{itemize}
    \item ``Space" $\to$ ``Stack",
    \item ``Scheme" $\to$ ``Algebraic Space"
\end{itemize}
と置換しただけで得られるので読み飛ばして構わない.

\subsection{Representable Ones}
    \begin{Def}[\rep Representable Stack]
        stack :: $\fibX$がrepresentableであるとは,
        あるalgebraic space :: $\shX$が存在し,$\fibX \iso \shX=\int \shX$であるということ.
    \end{Def}

    \begin{Def}[\rep Representable Morphism of Stacks]
        morphism of stacks :: $f \colon \fibX \to \fibY$がrepresentableであるとは,
        任意の$S$-algebraic space :: $U$と$\cat{C}$の射$U \to \fibY$について,
        fiber product :: $U \times_{\fibY} \fibX$がrepresentableであるということ.
    \end{Def}

    \begin{Lemma}[\rep]\label{lem:repdiag_stack}
        $\fibX$ :: stack in groupoids on $\cat{C}$とする.
        以下は同値である.
        \begin{enumerate}[label=(\roman*)]
            \item
                $\Diag \colon \fibX \to \fibX \times_S \fibX$は表現可能.
            \item
                任意のalgebraic space :: $U$と射$U \to \fibX$について,
                $U \to \fibX$ :: representable.
            \item
                任意のalgebraic space :: $U, V$と射$U \to \fibX, V \to \fibX$について
                $U \times_{\fibX} V$ :: representable.
        \end{enumerate}
    \end{Lemma}

    \begin{Def}[\rep Property of Representable Stacks/Morphism of Them]
        \enumfix
    \begin{enumerate}
    \item
        $\mathcal{P}$をschemeの性質でlocal for etale topologyであるものとする.
        この時,representable stack :: $\fibX$が性質$\mathcal{P}$を持つとは,
        $\fibX$をrepresentするalgebraic spaceが性質$\mathcal{P}$を持つということである.

    \item
        $\mathcal{P}$をmorphism of schemeの性質で
        local on the targetかつstable under base changeであるものとする.
        この時,representable morphism of stacks :: $f \colon \fibX \to \fibY$が性質$\mathcal{P}$を持つとは,
        任意の$U \in \cat{C}$と射$U \to \fibY$について,
        $\pr \colon \fibX \times_{\fibY} U \to U$
        (に対応するmorphism of algebraic algebraic spaces)が性質$\mathcal{P}$を持つということである.
    \end{enumerate}
    \end{Def}

    \subsection{Definition of Algebraic Stack}\label{sec:def-algst}
    \begin{Def}[Algebraic Stack (Artin Stack)]
        $S$ :: algebraic spaceとし,
        $\fibX$をstack over $S$(すなわちbig etale site $\Et(S)$上のsheaf)とする.
        $\fibX$がalgebraicであるとは,次が成り立つということである.
    \begin{enumerate}[label=(\alph*)]
        \item diagonal morphism :: $\Diag \colon \fibX \to \fibX \times_{S} \fibX$がrepresentableである.
        \item algebraic space :: $U$からの\underline{smooth} surjective morphism :: $U \to \fibX$が存在する.
    \end{enumerate}
        射はstack in groupoidsとしての射である.    
    \end{Def}
    補題(\ref{lem:repdiag_stack})から,二つの条件は意味を成す.

    \begin{Def}[\rep Deligne-Mumford(DM) Stack]
        $S$ :: algebraic spaceとし,
        $\fibX$をstack over $S$(すなわちbig etale site $\Et(S)$上のsheaf)とする.
        $\fibX$がalgebraicであるとは,次が成り立つということである.
    \begin{enumerate}[label=(\alph*)]
        \item diagonal morphism :: $\Diag \colon \fibX \to \fibX \times_{S} \fibX$がrepresentableである.
        \item algebraic space :: $U$からの\underline{etale} surjective morphism :: $U \to \fibX$が存在する.
    \end{enumerate}
        射はstack in groupoidsとしての射である.    
    \end{Def}

    \rest
    以下,Algebraic Stackと言った時はDMかArtinかを限定しない.
    \begin{Remark}
        我々が採用するalgebraic stackの定義は,しばしばArtin stackの定義として参照される.

        歴史的には,DM stackの方が先に定義された.
        これは1969年の論文\cite{IrrOfMg}でのことである.
        動機はalgebraic stack $\bar{\mathscr{M}}_g$を通して,
        coarse moduli schemeの性質を調べることだった.
        しばしばDM stackの定義として
        $\Diag \colon \fibX \to \fibX \times \fibX$はquasi-compactかつseparatedであるものとする.
        しかしこれは\cite{IrrOfMg}では要求されて居ない.

        一方,Artin stackは1974年の論文\cite{VerDefAlgSt}でDM stackの一般化として定義された.
        我々が扱うAlgebraic stackの定義(したがって多くの文献での``Artin stack"の定義)は,
        原論文のものとは異なる.
    \end{Remark}
    \rest

\subsection{Properties of Algebraic Stack/Morphism of Algebraic Stacks}
以下ではschemeの性質とschemeの射の性質をalgebraic stackへ拡張する.

\begin{Def}[\rep Property of Algebraic Stack]
    \enumfix
\begin{enumerate}
\item
    $\mathcal{P}$をschemeの性質であって,
    local for etale topologyであるものとする.
    この時,algebraic stack :: $\fibX$が性質$\mathcal{P}$を持つとは,
    $\fibX$のあるatlasが性質$\mathcal{P}$を持つということである.

\item
    algebraic stack :: $\fibX$がquasi-compact
    \footnote{ 明らかに,これはlocal for etale topologyではない. }であるとは,
    $\fibX$のあるatlasが性質$\mathcal{P}$を持つということである.
\end{enumerate}
\end{Def}

\begin{Def}[\rep Property of Morphism of Algebraic Stack]
    $\mathcal{P}$をmorphism of schemeの性質であって,
    local on the source and targetであるものとする.
    以下の可換図式で,$V \to \fibY, U \to V \times \fibX$はcoverであるとする.
    \[
    \begin{tikzcd}
        U \ar[r] \ar[rd, "f'"']& V \times \fibX \ar[r]\ar[d]& \fibX \ar[d, "f"]\\
        {} & V \ar[r]& \fibY \ar[lu, phantom, "\text{p.b.}"]
    \end{tikzcd}
    \eqno{(PM)}
    \]
    この時,\underline{morphism of algebraic spaces} :: $f \colon \fibX \to \fibY$が
    性質$\mathcal{P}$を持つとは,
    この可換図式にある$f'$(に対応するmorphism of scheme)が
    性質$\mathcal{P}$を持つということである.
\end{Def}

\begin{Lemma}[\rep]
\enumfix
\begin{enumerate}
    \item
    $\shX$をrepresentable stackとし,
    $P$をalgebraic stackの性質とする.
    $f$がalgebraic stackとして性質$P$を持つことと,
    representable stackとして性質$P$を持つことは同値.

    \item 
    $f \colon \shX \to \shY$をrepresentable morphismとし,
    $P$をalgebraic stackの射の性質とする.
    $f$がalgebraic stackの射として性質$P$を持つことと,
    representable morphismとして性質$P$を持つことは同値.
\end{enumerate}
\end{Lemma}

\begin{Lemma}[\rep]
    \enumfix
\begin{enumerate}
    \item
        $P$をschemeの性質でlocal for etale topologyなものとする.\mnewline
        一つのetale surjective morphism :: $U \to \fibX$について$U$が性質$P$を持つならば,\mnewline
        任意のetale surjective morphism :: $U \to \fibX$について$U$が性質$P$を持つ.
    \item
        $P$をmorphism of schemeの性質であって,
        local on the source and targetであるものとする.\mnewline
        一つの$V \to \fibY, U \to U \times \fibX$の組み合わせについて図式$(PM)$の$f'$が性質$P$を持つならば,\mnewline
        任意の$V \to \fibY, U \to U \times \fibX$の組み合わせについて図式$(PM)$の$f'$が性質$P$を持つ.
\end{enumerate}
\end{Lemma}
\begin{proof}
    \cite{StacksProj} Tag 06FM
\end{proof}

\begin{Lemma}[\rep]
        $P$をalgebraic stackの射の性質とする.
\begin{enumerate}
    \item
        $P$がschemeの射の性質としてstable under base changeならば,
        algebraic stackの射の性質としてもstable under base change.
    \item
        $P$がschemeの射の性質としてstable under compositionならば,
        algebraic stackの射の性質としてもstable under composition.
\end{enumerate}
\end{Lemma}
\begin{proof}
    algebraic spaceの場合の繰り返しである.
\end{proof}

\bibliographystyle{jplain}
\bibliography{reference}
\end{document}
