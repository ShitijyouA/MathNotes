\documentclass[a4paper, dvipdfmx]{jsarticle}
\usepackage{macros}
\setenumerate{label=(\roman*),itemsep=3pt,topsep=7pt}

\newcommand{\Diag}{\Delta}
\newcommand{\CFG}[1]{\cat{CFG}(\cat{#1})}
\begin{document}
\title{ゼミノート \#8 \\ Algebraic-ness of Spaces and Stacks}
\author{七条彰紀}
\maketitle
affine scheme, scheme. algebraic space, algebraic stackという貼り合わせの連なりを意識した定義をした後,
algebraic stackがschemeの貼り合わせとして定義できることを示す.
algebraic spaceとalgebraic stackの定義は全く平行に行われる.
そのことが分かりやすい記述を志向する.

\section{Fiber Product of Fibered Categories}
    $\cat{B}$ :: categoryとする.
    この時,$\FibBP{B}$は以下のような圏であった.
    \begin{description}[labelindent=1cm]
        \item[Objects:] fibered categories over $\cat{B}$.
        \item[Arrows:]  base-preserving natural transtormation.
    \end{description}
    新たに圏$\CFG{B}$を以下のように定義する.
    \begin{description}[labelindent=1cm]
        \item[Objects:] categories fibered in groupoids over $\cat{B}$.
        \item[Arrows:] base-preserving natural transtormation.
    \end{description}

重要なのは次の存在命題である.
\begin{Prop}[\cite{Gomez} p.10]
    $\FibBP{B}$と$\CFG{B}$はfibered productを持つ.
\end{Prop}
%% {{{
\begin{proof}
    $\FibBP{B}$の射$F \colon \fibX \to \fibZ$と$F \colon \fibY \to \fibZ$をとり,
    $F, G$のfiber productを実際に構成する.

    \step{圏$\cat{P}$の構成}
    圏$\cat{P}$を以下のように定義する.
    \begin{description}
        \item[Objects:]
            以下の$4$つ組
            \begin{enumerate}
                \item $b \in \cat{B}$,
                \item $x \in \fibX(b)$,
                \item $y \in \fibY(b)$,
                \item $\fibZ$の恒等射上の同型射$\alpha \colon Fx \to Gy$.
            \end{enumerate}
        
        \item[Arrows:] \mnewline
            射$(b, x, y, \alpha) \to (b', x', y', \alpha')$は,
            二つの射$\phi_{\fibX} \colon x \to x', \phi_{\fibY} \colon y \to y'$であって以下を満たすもの:
            $\phi_{\fibX}, \phi_{\fibY}$は同じ射$b' \to b$上の射で,
            以下の可換図式を満たすもの.
            \[
            \begin{tikzcd}
                Fx \ar[d, red, "F\phi_{\fibX}"']\ar[r, "\alpha"]& Gy \ar[d, red, "G\phi_{\fibY}"]\\
                Fx' \ar[r, "\alpha'"']& Gy'
            \end{tikzcd}
            \]
    \end{description}

    \step{$\cat{P}$はfibered category / category fibered by groupoids.}
    この圏は$(b,x,y,\alpha) \mapsto b$によってfibered categoryと成る.
    特に$\fibX, \fibY, \fibZ$がCFGならば$\cat{P}$もC.F.Gとなる.
    このことは次のことから分かる:
    $\phi_{\fibX} \colon x \to x'$と$\phi_{\fibY} \colon y \to y'$の両方がcartesianならば
    $(\phi_{\fibX}, \phi_{\fibY}) \colon (b, x, y, \alpha) \to (b', x', y', \alpha')$はcartesianである.

    \step{$\cat{P}$からの射影写像.}
    定義から明らかに$\pr_1 \colon \cat{P} \to \fibX, \pr_2 \colon \cat{P} \to \fibY$が定義できる.
    射の定義にある可換図式は,
    以下の$A$がnatural transformationであることを意味している.
    \begin{defmap}
        A \colon & F\pr_1& \to& G\pr_2 \\
        {}& (F\pr_1)((b, x, y, \alpha))=Fx& \mapsto& \alpha(Fx)=\alpha((F\pr_1)((b, x, y, \alpha)))
    \end{defmap}
    $A$がbase-preservingであることは$\alpha$が恒等射上のもの(i.e. $\pi_{\fibZ}(\alpha)=\id$)であることから,
    isomorphismであることは$\alpha$が同型であることから示される.

    \step{$\cat{P}$ :: fiber product.}
    今,
    $\fibW \in \CFG{B}$と
    射$S \colon \fibW \to \fibX, T \colon \fibW \to \fibY$及び
    base-preserving isomorphism :: $\delta \colon FS \to GT$をとる.
    base-preservingなので,任意の$w \in \fibW$について$\pi_{\fibZ}(FS(w))=\pi_{\fibZ}(GT(w))$.
    そこで次のように関手が定義できる.
    \begin{defmap}
        H\colon & \fibW& \to& \cat{P} \\
        \mathbf{Object}& w& \mapsto& (\pi_{\fibZ}(FS(w)), Sw, Tw, \delta_{w}) \\
        \mathbf{Arrow}& [\phi \colon w \to w']& \mapsto& (S\phi \colon Sw \to Sw', T\phi \colon Tw \to Tw')
    \end{defmap}
    このように置くと,$S=\pr_1 H, T=\pr_2 H$となる.
    逆に$S \iso \pr_1 H', T \iso \pr_2 H'$となる
    関手$H' \colon \fibW \to \cat{P}$は$H$と同型に成ることが直ちに分かる.
\end{proof}
%% }}}
\begin{Remark}
    session4 命題4.5より,CFGの恒等射上の射は同型射である.
    したがって$\alpha \colon Fx \to Gy$に課せられた条件は,
    $\fibZ$がCFGならば一つしか無い.
\end{Remark}

\begin{Example}
    representable fibered categoryのfiber product.
\end{Example}

我々が扱うのはstackであるから,
stackという性質がfiber productで保たれていて欲しいが,果たしてそうなる.
\begin{Prop}[\cite{ASS} Prop 4.6.4]
    $\fibX, \fibY, \fibZ$ :: stack over $\cat{C}$とし,
    morphism of stacks :: $F \colon \fibX \to \fibZ, G \colon \fibY \to \fibZ$をとる.
    この時,$F, G$についてのfiber product :: $\fibX \times_{\fibZ} \fibY$はstackである.
\end{Prop}
\begin{proof}
    (TODO)
\end{proof}

\section{Representable Morphism}
\begin{Remark}
    以下,$S$ :: schemeとし,$\Sch/S$上のsiteを$\cat{C}$と書く
    ($(\Sch/S)_{\tau}$といった表記も見かける).
    また,stackといえばstack in groupoidに限る.
\end{Remark}

\begin{Def}[Representability of Morphism of Spaces/Stacks]
    \enumfix
\begin{enumerate}
\item
    morphism of spaces :: $f \colon \shX \to \shY$がrepresentable( by scheme)であるとは,
    任意の$S$-scheme :: $U$と射$U \to \shY$について,
    fiber product :: $U \times_{\shY} \shX$(これはspace)がrepresentable by schemeであるということ.

\item
    morphism of stacks :: $f \colon \fibX \to \fibY$がrepresentable( by space)であるとは,
    任意の$S$-space :: $U$と射$U \to \fibY$について,
    fiber product :: $U \times_{\fibY} \fibX$(これはstack)がrepresentable by spaceであるということ.
\end{enumerate}
\end{Def}

\begin{Lemma}
    morphism of stacks :: $f \colon \fibX \to \fibY$がrepresentable by spaceであることは,
    任意の$S$-scheme :: $U$と射$U \to \fibY$について,
    fiber product :: $U \times_{\fibY} \fibX$
    (これはstack)がrepresentable by spaceであることと同値.
\end{Lemma}

\section{Diagonal Map}
\begin{Def}[Diagonal Map]
    $\fibX/S$(すなわち射$\fibX \to S$)のdiagonal map :: $\Diag$とは,
    以下の可換図式に収まる射のことである.
    \[\begin{tikzcd}
            \fibX \ar[rrd, bend left, "\id"]\ar[rdd, bend right, "\id"']\ar[rd, "\Diag"]&
                                                            & \\
                                                            &
        \fibX \times \fibX \ar[r]\ar[d]\ar[rd, phantom, "\text{p.b.}"]& \fibX \ar[d]\\
          &\fibX \ar[r]& S
    \end{tikzcd}\]
\end{Def}

\begin{Prop}
    $\fibF$ :: stack on $\tau(S)$
    以下は同値である.
    \begin{enumerate}[label=(\roman*)]
        \item $\Diag \colon \fibX \to \fibX \to \fibX$ :: representable.
        \item 任意のscheme :: $U$について,$U \to \fibX$ :: representable.
        \item 任意のscheme :: $U, V$と射$U \to \fibX, V \to \fibX$について$U \times_{\fibX} V$ :: representable.
    \end{enumerate}
\end{Prop}
\begin{proof}
    (TODO)
\end{proof}

\section{Property of Representable Space/Stack/Morphism}
\begin{Def}
    まずspaceとmorphism of spacesについて定義する.
\begin{enumerate}
\item
    $\mathcal{P}$をschemeの性質とする.
    この時,representable space :: $\shX$が性質$\mathcal{P}$を持つとは,
    $\shX$をrepresentするschemeが性質$\mathcal{P}$を持つということである.

\item
    $\mathcal{P}$をmorphism of algebraic schemesの性質とする.
    この時,representable morphism of spaces :: $f \colon \fibX \to \fibY$が性質$\mathcal{P}$を持つとは,
    任意の$U \in \cat{C}$と射$U \to \fibY$について,
    $\pr \colon \fibX \times_{\fibY} \fibX \to U$
    (に対応するmorphism of algebraic schemes)が性質$\mathcal{P}$を持つということである.
\end{enumerate}

    次にstackとmorphism of stacksについて定義する.
    これらは上の定義を殆ど機械的に置換すれば得られる.
\begin{enumerate}
\item
    $\mathcal{P}$をspace (resp. scheme)の性質とする.
    この時,representable stack :: $\shX$が性質$\mathcal{P}$を持つとは,
    $\shX$をrepresentするspace(resp. scheme)が性質$\mathcal{P}$を持つということである.

\item
    $\mathcal{P}$をmorphism of algebraic spacesの性質とする.
    この時,representable morphism of stacks :: $f \colon \fibX \to \fibY$が性質$\mathcal{P}$を持つとは,
    任意の$U \in \cat{C}$と射$U \to \fibY$について,
    $\pr \colon \fibX \times_{\fibY} \fibX \to U$
    (に対応するmorphism of algebraic spaces)が性質$\mathcal{P}$を持つということである.
\end{enumerate}
\end{Def}

\section{Algebraic-ness}
\subsection{Definition}
\begin{Def}[Algebraic Space]
    \enumfix
\begin{enumerate}
    \item diagonal morphism :: $\Diag \colon \shX \to \shX \times_{S} \shX$がrepresentableである.
    \item schemeからのetale surjective morphism :: $U \to \shX$が存在する.
\end{enumerate}
\end{Def}

\begin{Def}[Deligne-Mumford / Artin Stack]
    \enumfix
\begin{enumerate}
    \item diagonal morphism :: $\Diag \colon \fibX \to \fibX \times_{S} \fibX$がrepresentableである.
    \item algebraic spaceからのetale surjective morphism :: $U \to \fibX$が存在する.
\end{enumerate}
    以上の定義はDeligne-Mumford stackと呼ばれる.
    \textbf{algebraic stackと言えばこちらを指す.}
    etaleでなくsmoothを要求するものに弱めたものはArtin stackと呼ばれる.
\end{Def}

\bibliographystyle{jplain}
\bibliography{reference}
\end{document}
