\documentclass[a4paper]{jsarticle}
\usepackage{macros}

\begin{document}
\title{ゼミノート \#5 \\ Categorical Part of Descent Theory, and Stacks}
\author{七条彰紀}
\maketitle

今回のノートで一貫して用いる記号と記法を定める.

$\cat{C}$ :: site, 
$\pi \colon \shF \to \cat{C}$ :: fibered categoryを考える
\footnote{ ほとんどfiber of $\pi$しか扱わないので,psuedo-functor $\cat{C} \to \Cat$をとっても構わない. }.

記法を定める.
$U \in \cat{C}, \covU=\{\phi_i \colon U_i \to U\}_{i \in I} \in \Cov(U)$について,
\[ U_{ij}:=U_i \times_U U_j, \quad U_{ijk}:=U_i \times_U U_j \times_U U_k \ \ (i,j,k \in I) \]
と書くことにする.
また,添字$a,b=i \mor j \mor k$について,
fiber productからの射影を
\[ \pr_{a} \colon U_{ij} (\mor U_{ijk}) \to U_{a}, \qquad \pr_{a,b} \colon U_{ijk} \to U_{ab} \]
とする.
さらに$\pr_{i} \colon U_{ij} \to U_{i}$によるpullbackを$(-)|_{U_{ij}}$などと書く.

\section{The Category of Descent Data}
\subsection{Definition}
\begin{Def}[$\shF(\covU)$, \cite{ASS} 4.2.4, \cite{NoteGroTop} Def4.2]
    圏$\shF(\covU)$を次のように定める.
    \begin{description}
        \item[Object.] \hfill \vspace{-0.2cm}
            \begin{itemize}
                \item $\xi_i \in \shF(U_i)$なる対象のclass $\{\xi_i\}_{i \in I}$と,
                \item
                    $\shF(U_{ij})$中の同型$\sigma_{ij} \colon \xi_j|_{U_{ij}} \to \xi_i|_{U_{ij}}$の
                    class $\{\sigma_{ij}\}_{i,j \in I}$
            \end{itemize}
            の組$(\{\xi_i\}, \{\sigma_{ij}\})$であって,
            以下で述べるcocycle conditionを満たすもの.
            このような組をobject with descent dataと呼ぶ
            \footnote{ 同型のclass $\{\sigma_{ij}\}$がdescent dataと呼ばれる. }.

        \item[Arrow.] \mnewline
            射$\{\alpha_i\} \colon (\{\xi_i\}, \{\sigma_{ij}\}) \to (\{\eta_i\}, \{\tau_{ij}\})$とは,
            $\shF(U_i)$の射$\alpha_i \colon \xi_i \to \eta_i$のclassであって,\mnewline
            $\sigma_{ij}, \tau_{ij}$と整合的であるもの.
            すなわち,任意の$i, j \in I$について以下の図式が可換であるもの.
            \[\xymatrix{
                \ar[d]_-{\sigma_{ij}} \xi_j|_{U_{ij}} \ar[r]^-{\alpha_j|_{U_{ij}}}&
                \eta_j|_{U_{ij}} \ar[d]^-{\tau_{ij}}\\
                 \xi_i|_{U_{ij}} \ar[r]_-{\alpha_i|_{U_{ij}}}& \eta_i|_{U_{ij}}
            }\]
    \end{description}

    \step{cocycle condition}
    組$(\{\xi_i\}, \{\sigma_{ij}\})$がcocycle conditionを満たすとは,
    任意の$i,j,k \in I$について以下が成り立つということ.
    \[ \sigma_{ik}|_{U_{ijk}}=(\sigma_{ij}|_{U_{ijk}}) \circ (\sigma_{jk}|_{U_{ijk}}). \]
    図式でかけば,圏$\shF(U_{ijk})$における以下の図式が可換であることと同値.
    \[\xymatrix@R=70pt{
         \xi_k|_{U_{ijk}} \ar[rd]_-{\sigma_{ik}|_{U_{ijk}}} \ar[rr]^-{\sigma_{jk}|_{U_{ijk}}}
            & {}
            & \xi_j|_{U_{ijk}} \ar[ld]^-{\sigma_{ij}|_{U_{ijk}}} \\
        {} & \xi_i|_{U_{ijk}} & {}
    }\]
\end{Def}

\begin{Remark}
    この定義に於いてfiber products :: $U_{ij}, U_{ijk}$を暗黙のうちに選択している.
    たが,どのように選択しても得られる圏は同型に成る.
    $U_{ij}, U_{ijk}$の選択も込めて
    $(\{\xi_i\}, \{\xi_{ij}\}, \{\xi_{ijk}\})$を$\shF(\covU)$の対象とする
    定義の仕方も有るが,ここでは述べない.
    詳細は\cite{NoteGroTop} Remark 4.3にある.
\end{Remark}

\begin{Def}[\cite{NoteGroTop} p.72] \label{def:epsilon}
    $\xi \in \shF(U), \covU=\{\phi_i \colon U_i \to U\} \in \Cov(U)$について,
    $\shF(\covU)$の元を以下のデータに対応させる:
    \begin{itemize}
        \item $\xi_i:=\phi_i^*\xi$のclass $\{\xi_i\}_{i \in I}$.
        \item
            $\xi_i|_{U_{ij}}$と$\xi_j|_{U_{ij}}$が,
            いずれも
            \[ \phi_i \circ \pr_i=\phi_j \circ \pr_j \colon U_{ij} \to U \]による$\xi$のpullback
            であることから得られる
            標準的同型のclass $\{ \sigma_{ji} \colon \xi_j|_{U_{ij}} \to \xi_i|_{U_{ij}} \}_{i,j}$.
    \end{itemize}
    このデータをまとめて$(\{\phi_i^*\xi\}, \mathrm{cano})$などと書く.
    この対応を$\epsilon_{\covU} \colon \shF(U) \to \shF(\covU)$と書く.
    $\shF(U)$の射$\xi \to \eta$から,
    $\phi_i$に沿ったpullbackによって$(\{\phi_i^*\xi\}, \mathrm{cano}) \to (\{\phi_i^*\eta\}, \mathrm{cano})$
    が得られるので,
    対応$\epsilon_{\covU}$は関手である.
\end{Def}

\subsection{Example}
\begin{Example}[\cite{ASS}, 4.2.1]
    一つの射から成るcover :: $\covU=\{f \colon V \to U\}$について$\shF(\covU)$を考えてみる.
    この圏の対象は,
    \begin{itemize}
        \item 対象$E \in \shF(V)$
        \item $\shF(V \times_U V)$の中の同型射$\sigma \colon \pr_1^*E \to \pr_2^*E$
    \end{itemize}
    の組である.
\end{Example}

\bibliographystyle{jplain}
\bibliography{reference}
\end{document}
