\documentclass[a4paper, dvipdfmx]{jsarticle}
\usepackage{macros}

\newcommand{\Diag}{\Delta}
\newcommand{\rest}{\vspace{5pt}}
\newcommand{\arpb}{\ar[lu, phantom, "\text{p.b.}"]}
\newcommand{\Isom}{\cat{Isom}}

\begin{document}
\title{ゼミノート \#9 \\ Quotient Stacks}
\author{七条彰紀}
\maketitle
\tableofcontents
\vspace{10pt}

Algebraic stackの具体例としてQuotient stackを扱う.
この例を通じて特に,
「diagonal morphism $\Diag \colon \fibX \to \fibX \times_S \fibX$が表現可能とはどういうことか」
ということを考えたい.
参考文献として\cite{ChAlg} 1.3.2, \cite{IrrOfMg} Example 4.8, \cite{ASS} Example 8.1.12を参照する.

\section{Definitions}
\subsection{\tp{$\shG$}{G}-torsor}
\begin{Def}[Equivariant Morphism]
    一般のsite :: $\cat{C}$をとり,$\shG$を$\cat{C}$上のsheaf of groupsとする.
    sheaf :: $\shF$と,
    $\shG$からの左作用$\alpha \colon \shG \times \shF \to \shF$を組にして
    $(\shF, \alpha)$と書く.
    $\shG$からの左作用を持つsheafの間の射$(\shF, \alpha) \to (\shF', \alpha')$とは,
    sheafの射$f \colon \shF \to \shF'$であって以下が可換図式であるもの.
    \[
        \begin{tikzcd}
            \shG \times \shF \ar[r, "\id \times f"]\ar[d, "\alpha"']& \shG \times \shF' \ar[d, "\alpha'"]\\
            \shF \ar[r, "f"']& \shF'
        \end{tikzcd}
    \]
    このような射$f$は$\shG$-equivariant morphism($\shG$同変写像)と呼ばれる.
\end{Def}

\begin{Def}[$\shG$-Torsor, \cite{ASS} 4.5.1, \cite{StacksProj} Tag 04UJ]
    一般のsite :: $\cat{C}$をとり,$\shG$を$\cat{C}$上のsheaf of groupsとする.
    $\cat{C}$上の$\shG$-torsorとは,
    $\cat{C}$上のsheaf :: $\shP$と左作用$\alpha \colon \shG \times \shP \to \shP$の組であって,
    次を満たすもの.
    \begin{description}
        \item[T1]
            任意の$X \in \cat{C}$についてcover of $X$ :: $\{X_i \to X\}$が存在し,
            $\shP(X_i) \neq \emptyset$.
        \item[T2]
            写像
            \[
                \langle \pr_2, \alpha \rangle \colon \shG \times \shP \to \shP \times \shP;
                \quad (p, g) \mapsto (p, \alpha(g, p))
            \]
            は同型.
            ただし,
            $\langle \pr_1, \alpha \rangle$は$\shP \times \shP$の普遍性と
            $\pr_1, \alpha \colon \shP \times \shG \to \shP$から得られる射である.
    \end{description}
    $\shG$-torsorの射は$\shG$-equivariant morphismである.

    $(\shP, \alpha)$が$\shG$-torsor :: $(\shG, m)$
    (ただし$m \colon \shG \times \shG \to \shG$は積写像)と同型である時
    $\shG$-torsor :: $(\shP, \alpha)$は自明(trivial)であると言う.
\end{Def}

\begin{Remark}
    $\shG, \shP$の両方がschemeで表現できる場合には,
    $\shG$-torsorはprincipal bundleと呼ばれる.
    group schemeに対応するrepresentable sheafが
\end{Remark}

\begin{Remark}
    任意の$X \in \cat{C}$について$\shP(X) \neq \emptyset$である場合には,
    条件T2は作用$\alpha$が単純推移的であることを意味する.
    すなわち,任意の$p, q \in \shP(X)$について
    ただ一つの$g \in \shG(X)$が存在し,$q=g \ast q=\alpha(g, p)$となる.
\end{Remark}

\begin{Lemma}[\cite{StacksProj} Tag 03AI, \cite{ASS} 4.5.1]
    \enumfix
    \begin{enumerate}
        \item 
        $\shG$-torsor :: $(\shP, \alpha)$が自明であることと,
        $\shP$がglobal section\footnote
        {
            前層の圏$\PSh(\cat{C})$のterminal objectから$\shP$への射のこと(\cite{StacksProj} Tag 06UN).
            $\PSh(\cat{C})$のterminal objectは自明群で定まるconstant sheafである.
        }
        を持つことと同値.

        \item $\shP(X) \neq \emptyset$ならば制限$\shP|_{X}$はtrivial.

        \item 同型$\shG|_{X} \to \shP|_{X}$と$\shP(X)$の元は一対一に対応する.
    \end{enumerate}
\end{Lemma}
\begin{proof}
    $(\shP, \alpha)$が自明であると仮定すると,
    次のようにglobal sectionが得られる.
    \[ 1 \to \shG \iso \shP; \quad * \mapsto e \]
    ただし$e$は$\shG$の単位元である.

    $p$を$\shP$のglobal sectionとすると,
    \[ \shG \to \shP; \quad g \mapsto \alpha(g, p) \]
    という射が定義できる.
    これは定義にある条件T2から同型である.

    $s \in \shP(X)$をとれば,schemeの任意の射$\phi \colon U \to X$について
    \[ 1 \to (\shP|_{X})(U)=\shP(U);\quad * \mapsto \phi^*s \]
    のようにglobal section :: $1 \to \shP|_{X}$が定まる.
\end{proof}

\begin{Cor}
    $\shG$-torsorの任意の射は同型.
\end{Cor}
\begin{proof}
    isomorphismはetale local on the targetなので(TODO),
    条件T1にあるようなetale cover $\{\phi_i \colon U_i \to X\}$を取れば
    主張は「trivial $(\phi_i)^*\shG$-torsorの射は同型」という命題に帰着される.

    $\shG$の単位元(射$e \colon 1 \to \shG$の像)を$e$と書くことにすると,
    射$(\phi_i)^*\shG \to (\phi_i)^*\shG$は,$g \mapsto g \cdot f(e)$と書ける.
    $f(e) \in (\phi_i)^*\shG$も群の元なので逆元が存在する.
    なので$g' \mapsto g' \cdot f(e)^{-1}$とすれば逆射が作れる.
\end{proof}

\subsection{Quotient Stack}
\begin{Def}[Quotient Stack, \cite{ASS} Example 8.1.12]
    $X$ :: algebraic space,
    $G$ :: smooth group scheme over $S$, acting on $X$とする.
    すなわち左作用$\alpha \colon \ftor{G} \times X \to X$が存在するものとする.
    この時,fibered category :: $[X/G](\to \ET(S))$を以下で定める.

    \begin{description}
        \item[Object]
            以下の$3$つ組.
            \begin{itemize}
                \item $S$-scheme :: $U$,
                \item $G_{U}(:=G \times_{S} U)$-torsor on $\ET(U)$ :: $\shP$,
                \item $\ftor{G_U}$-torsorの射$\pi \colon \shP \to X_U(:=X \times_{S} U)$.
            \end{itemize}
        \item[Arrow]
            射$(U, \shP, \pi) \to (U', \shP', \pi')$は
            二つの射の組$(f \colon U \to U', f^{\flat} \colon \shP \to f^*\shP')$であって,
            以下が可換となるもの.
            \[
            \begin{tikzcd}
                \shP \ar[rr, "f^{\flat}"]\ar[rd, "\pi"']& {} & f^*\shP' \ar[ld, "f^*\pi'"]\\
                {} & X_{U} & {}
            \end{tikzcd}
            \]
            $\pi$と$f^*\pi'$のcodomain,すなわち$X_U$と$f^*X_{U'}$が一致していることに注意.
    \end{description}
    fibrationは$(U, \shP, \pi) \mapsto U, (f, f^{\flat}) \mapsto f$で与えられる.
\end{Def}

\begin{Remark}\label{rem:GU}
    任意の$[U \to S] \in \Sch/S$について,$G_{U}(:=G \times U)$は群になる.
    単位セクション$e_U \colon 1 \to G_{U}$,
    $e \colon 1 \to G$のpullbackから得られる.
    積$m_U$なども同様.
    特に射影$\pr_{U} \colon G_{U} \to U$は,
    smooth morphism :: $G \to S$のpullbackなのでsmooth.
\end{Remark}

\begin{Lemma}
    $S$ :: scheme,
    $X$ :: algebraic space,
    $G$ :: smooth group scheme over $S$, acting on $X$とする.
    Quotient stack :: $[X/G]$はstack in groupoidsである.
\end{Lemma}
\begin{proof}
    stackであることはsheafの貼り合わせが可能であることに拠る.
    詳しくは\cite{ASS} 4.2.12, \cite{StacksProj} Tag 04UKを参照せよ.
    $[X/G]$がcategory fibered in groupoids(CFG)であることを確かめる.
    これは恒等射上の$[X/G]$の射が同型射であることを確かめれば良い.

    $U \in \ET(S)$を固定し,
    射$(\id[U], f^{\flat}) \colon (U, \shP, \pi) \to (U, \shP', \pi')$を考える.
    定義から,次が可換である.
    \[
        \begin{tikzcd}[row sep=20pt]
        \shP \ar[rr, "f^{\flat}"]\ar[rd, "\pi"']& {} & \shP' \ar[ld, "\pi'"]\\
        {} & X_{U} & {}
    \end{tikzcd}
    \]
\end{proof}

\section{Aim of This Session}
\begin{Thm}
    $X$ :: algebraic space,
    $G$ :: smooth group scheme over $S$, acting on $X$とする.
    Quotient Stack :: $[X/G]$はArtin stackである.
\end{Thm}

\section{準備}
\subsection{Definition of \tp{$\Isom(X, Y)$}{Stack of Isomorphisms}}
最初に$\fibX$のcleavageを選択せずとも出来る$\Isom$の構成を述べる.
後の注意で特にsplittingを選択した場合の構成も述べておく.
\begin{Def}[$\Isom(X, Y)$]
    stackとは限らないfibration :: $\fibX \to \cat{B}$と,
    $U \in \cat{B}$及び$U$上の対象$X, Y \in \fibX$をとる.
    この時,CFG over $\cat{B}/U$:: $\Isom(X, Y)$を以下のように定める.
    \begin{description}
        \item[Object.]
            以下の$4$つ組.
            \begin{itemize}
                \item $\cat{B}/U$の対象$f \colon V \to U$.
                \item $f$のcartesian lifting :: $f^*X \to X, f^*Y \to Y$.
                \item 同型$\phi \colon f^*X \to f^*Y$.
            \end{itemize}

        \item[Arrow.]
            射
            \[
                (V \xrightarrow{f} U, f^*X \to X, f^*Y \to Y, f^*X \xrightarrow{\phi} f^*Y)
                \to
                (W \xrightarrow{g} U, g^*X \to X, g^*Y \to Y, g^*X \xrightarrow{\psi} g^*Y)
            \]
            は,以下の$2$つからなる.
            \begin{itemize}
                \item $\cat{B}/U$の射$h \colon V \to W$(したがって$g \circ h=f$が成立),
                \item 射$h^*\psi, \phi$の間のcanonicalな同型射$(h^*g^*X \to f^*X, h^*g^*Y \to f^*Y)$.
            \end{itemize}
    \end{description}
    $(h^*g^*X \to f^*X, h^*g^*Y \to f^*Y)$を選択することで,
    $h^*g^*X \to X, h^*g^*Y \to Y$が定まる.
    またTriangle Liftingにより$h^*\psi$も定まる.以下の図式を参考にすると良い.
    \begin{center}
    \begin{tikzpicture}[mybox/.style={draw, inner sep=5pt}]
    \node[mybox] (X) at (0,5){%
        \begin{tikzcd}
            {} & h^*g^*X\ar[ld] \ar[r, "h^*\psi"]\ar[dd]& h^*g^*Y \ar[dd]\ar[rd]& {} \\
            X & {} & {} & Y \\
            {} & f^*X\ar[lu]\ar[r, "\phi"']& f^*Y \ar[ru]& {}
        \end{tikzcd}
    };
    \node[mybox] (B) at (0,0){%
        \begin{tikzcd}[row sep=8pt]
            {} & V \ar[ld, "g \circ h"']\ar[r, "\id"]\ar[dd, "\id"']& V \ar[dd, "\id"]\ar[rd, "g \circ h"]& {} \\
            U & {} & {} & U \\
            {} & V \ar[lu, "f"]\ar[r, "\id"']& V \ar[ru, "f"']& {}
        \end{tikzcd}
    };

    \node [above=5pt of X] {in $\fibZ$};
    \node [below=5pt of B] {in $\cat{B}$};
    \draw [->, line width=1.5pt] (X) edge (B);
    \node at (0.3,2.45) {$\pi_{\fibZ}$};
    \end{tikzpicture}
    \end{center}

    fibrationは次のように与えられる.
    \begin{defmap}
        \pi \colon & \Isom(X, Y)& \to& \cat{B}/U \\
        \textbf{Objects:}& (f \colon V \to U, f^*X, f^*Y, \phi \colon f^*X \to f^*Y)& \mapsto& f \\
        \textbf{Arrows:}& (h \colon V \to W, h^*g^*X \to f^*X, h^*g^*Y \to f^*Y)& \mapsto& h \\
    \end{defmap}
\end{Def}

\begin{Remark}\label{rem:simple_isom}
    $\fibX \to \cat{B}$のsplittingを選んだ場合には$\Isom(X, Y)$の定義は次のように簡単に成る.
    \begin{description}[labelindent=1cm]
        \item[Object.]
             $\cat{B}/U$の対象$f \colon V \to U$と同型$\phi \colon f^*X \to f^*Y$の組.

        \item[Arrow.]
            射$(f, \phi) \to (g, \psi)$は,
            $g \circ h=f$を満たす$\cat{B}/U$の射$h$.
    \end{description}
    以下では$\Isom(X, Y)$がalgebraic space(これはsheaf)と
    同型であるかどうかを考えるので,
    こちらの定義だけを覚えていても問題はない.
\end{Remark}

\subsection{Propositions}
\begin{Lemma}
    任意の$U \in \cat{B}$と$X, Y \in \fibX(U)$について,
    $\Isom(X, Y)$はcategory fibered in sets.
\end{Lemma}
\begin{proof}
    恒等射上の射は恒等射しかないことを確かめれば良い.
    $\Isom(X, Y)$の射の定義から,恒等射上の射は次の形になっている.
    \[ (\id[U], f^*X \to f^*X, f^*Y \to f^*Y) \colon (f, f^*X, f^*Y, \phi) \to (f, f^*X, f^*Y, \psi) \]
    $f^*X \to f^*X, f^*Y \to f^*Y$はTriangle Liftingから得られるcanonicalなものなので,
    恒等射である.
\end{proof}

$\fibX$ :: stackの場合は($\fibX \to \cat{B}$のsplittingを選べば)$\Isom(X, Y)$はsheafになる.
\begin{Lemma}
    一般のsite :: $\cat{C}$とCFG :: $\fibX \to \cat{C}$をとる.
    さらに$\fibX$はsplit fibered categoryであるとする.
    以下の二つは互いに同値.
    \begin{enumerate}
        \item $\fibX$はprestackである.
        \item 任意の$X, Y \in \fibX$について$\Isom(X, Y)$のfiberはsheafである.
    \end{enumerate}
\end{Lemma}
\begin{proof}
    (TODO: 出典)
\end{proof}

\subsection{Representability of Diagonal Morphism.}
\begin{Remark}
    以下,scheme $S$を固定し,
    特に断らない限りbig etale site :: $\ET(S)$上のstack in groupoidsのみ考える.
\end{Remark}

\begin{Lemma}
    $\fibX$ :: stack in groupoids on $\cat{C}$($=\ET(S)$)とする.
    この時,$\Diag \colon \fibX \to \fibX \times_{S} \fibX$が表現可能であることと,
    任意の$U \in \cat{C}$と任意の$X, Y \in \fibX(U)$について
    $\Isom(X, Y)$がalgebraic spaceであることは同値.
\end{Lemma}
%% {{{
\begin{proof}
    $x,y \colon \Sch/U(=U) \to \fibX$を,
    $2$-Yoneda Lemmaにより得られる$X, Y \in \fibX(U)$に対応する射とする
    \footnote{ 例えば$x$は$f \in \Sch/U$をcartesian lifting $f^*X$へ写す. }.

    以下の図式がpullback diagramであることから分かる.
    \[
    \begin{tikzcd}[row sep=40pt]
        \Isom(X, Y) \ar[r, "\pr_{U}"]\ar[d, "\pr_{\fibX}"']& \Sch/U \ar[d, "x \times y"]\\
        \fibX \ar[r, "\Diag"'] & \fibX \times_{S} \fibX
        \ar[from=2-1, to=1-2, shorten >= 20pt, shorten <= 20pt, Rightarrow, "a"]
    \end{tikzcd}
    \]
    任意の射$\Sch/U \to \fibX \times \fibX$が$x \times y$の形で表されることは,
    $\fibX \times \fibX$の普遍性から得られる.

    まず,射と自然同型を定義する.
    $\Isom(X, Y)$から伸びる射は次の関手である.
    ただし
    $\xi=(f \colon V \to U, f^*X, f^*Y, \phi \colon f^*X \to f^*Y),
    \eta=(g \colon W \to U, g^*X, g^*Y, \psi \colon g^*X \to f^*Y)$とした.
    \begin{defmap}
        \pr_{U}& \Isom(X,Y)& \to& \Sch/U \\
        \textbf{Objects:}& \xi& \mapsto& f \\
        \textbf{Arrows:}& [\xi \to \eta]& \mapsto& h \\
        \hfill \\
        \pr_{\fibX}& \Isom(X, Y)& \to& \fibX \\
        \textbf{Objects:}& \xi& \mapsto& f^*X \\
        \textbf{Arrows:}& [\xi \to \eta]& \mapsto& f^*X \to h^*g^*X
    \end{defmap}
    自然同型$a$は次で定める.
    \begin{defmap}
        a_{\xi}\colon & ((x \times y) \pr_{U})(\xi)& \to& (\Diag \pr_{\fibX})(\xi) \\
        {}& (f \colon V \to U, f^*X, f^*X, \alpha)& \mapsto& (\id[f^*X], \phi)
    \end{defmap}

    $\Isom(X, Y)$がpullbackであることは,
    $\Isom(X, Y)$が普遍性を持つことを通して確かめる.
    (TODO)
\end{proof}
%% }}}

\section{証明}
\subsection{\tp{$\Diag$}{The Diagonal Morphism} is Representable.}
示した補題から,
任意の$U \in \cat{C}$と任意の$G_U$-torsor :: $\shP_1, \shP_2 \in \fibX(U)$について
$\Isom(\shP_1, \shP_2)$がalgebraic spaceであることは同値である.
これは次のようにして自明な場合に帰着できる.

\step{$\shP_1, \shP_2$が自明な場合に帰着させる.}
\begin{Lemma}[\cite{ASS} Exc 5.G]\label{lem:F_is_algsp}
    $U$ :: schemeをとる.
    sheaf on $\ET(U)$ :: $\shF$とetale surjective morphism :: $V \to U$に対し,
    $V \times_{U} \shF$がalgebraic spaceならば,$\shF$はalgebraic space.
\end{Lemma}

\begin{Lemma}\label{lem:UIsom}
    $X, Y \in \fibX(U)$と$v \colon V \to U$について
    \[ V \times_{U} \Isom(X, Y) \iso \Isom(v^*X, v^*Y). \]
\end{Lemma}

\begin{proof}[(補題\ref{lem:F_is_algsp}の証明)]
    $\shF':=(V \times_{U} \shF) \times_{V} (V \times_{U} \shF)$とおく.

    まずdiagonal morphismの表現可能性を考える.
    pullback lemmaから次が分かる.
    \[
        (V \times_{U} \shF) \times_{V} (V \times_{U} \shF)
        \iso (V \times_{U} \shF) \times_{U} \shF
        \iso V \times_{U} (\shF \times_{U} \shF).
    \]
    このことから,$\Diag \colon \shF \to \shF \times_{U} \shF$を$V \to Y$でpullbackしたものが
    $\Diag' \colon \shF' \to \shF' \times \shF'$だと分かる.

    atlasの存在は次のように分かる.
    $A \to \shF'$を$\shF'$のatlasとする.
    \[
    \begin{tikzcd}
        A \ar[r]& \shF' \ar[r]\ar[d]& \shF \ar[d]\\
        {} & V \ar[r, "\text{etale, surj.}"']& U
    \end{tikzcd}
    \]
    $V \to Y$がetale surjectiveなので$\shF' \to \shF$もetale surjective.
    今$A \to \shF'$がetale surjectiveなので,併せて$A \to \shF$がetale surjectiveと分かる.
\end{proof}

\begin{proof}[(補題\ref{lem:UIsom}の証明)]
    定義を変形するだけである.
    \begin{align*}
        {}& (V \times_{U} (\Isom(X, Y)))(W) \\
        =&  V(W) \times_{U(W)} \Isom(X, Y)(W) \\
        =&  \{ (w \colon W \to V, f \colon W \to U, \rho \colon f^*X \isomap f^*Y) \mid u \circ v=f \} \\
        =&  \{ (w \colon W \to V, \rho \colon w^* u^*X \isomap w^* u^*Y) \} \\
        =&  \Isom(v^*X, v^*Y)(V)
    \end{align*}
\end{proof}

$\shP_1, \shP_2$が自明に成るetale cover :: $\covV$
\footnote{ i.e. $\Forall{V \in \covV} \shP_1(V), \shP_2(V) \neq \emptyset$. }
をとり,$v \colon V=\bigsqcup_{V \in \covV} V \to X$とすれば,
$v^*\shP, v^*\shP_2$は自明な$G_U$-torsorとなる.
こうして$\shP_1, \shP_2$が自明な場合に議論を帰着させることが出来る.

同型$G_Y \iso \shP_1, G_Y \iso \shP_2$を固定する.
これらと$\pi_1, \pi_2$を合成して
\[ \rho_1 \colon G_Y \to X_Y, \quad \rho_2 \colon G_Y \to X_Y \]を得る.

\step{$\shP_1, \shP_2$が自明な場合について証明する.}

$\Isom(\ (G_Y, \rho_1), (G_Y, \rho_2)\ )$がどのようなsheafか考える.
trivial torsorの任意の自己同型$\phi \colon G_Y \to G_Y$は,
これがequivariantであることから,$\phi(e)$の左からの積になっている.
逆に任意の$G_Y$の元を取れば左からの積が自己同型になるから,
集合$\Isom(\ (G_Y, \rho_1), (G_Y, \rho_2)\ )$は$G_Y$の部分集合である.
そこで,$g \in G_Y$ (i.e. $g \colon 1 \to G_Y$)から得られる
自己同型$(\cdot g) \colon G_Y \to G_Y$が満たすべき条件を考える.

$[X/G]$の定義から,次が可換である.
\[
\begin{tikzcd}
    G_Y \ar[rr, "(\cdot g)"]\ar[rd, "\rho_1"']& {} & G_Y \ar[ld, "\rho_2"]\\
    {} & X_Y & {}
\end{tikzcd}
\]
$(\cdot g)$はequivariantだから,この図式が可換であることは$\rho_1(e)=\rho_2(g)$と同値である.
したがって
\begin{align*}
    {}& \Isom(\ (G_Y, \rho_1), (G_Y, \rho_2)\ )(V) \\
    =&  \{ g \in G_Y(V) \mid \rho_1(e)=\rho_2(g) \} \\
    =&  \{ (g, x) \in G_Y(V) \times X_Y(V) \mid  (\rho_1(e), \rho_2(g))=(x, x)  \}
\end{align*}
これは次のように,fiber productで表現出来る.
\[
\begin{tikzcd}
    \Isom(\ (G_Y, \rho_1), (G_Y, \rho_2)\ ) \ar[r]\ar[d]& G_Y \ar[d]\\
    X_Y \ar[r, "\Diag"']& X_Y \times_{Y} X_Y
\end{tikzcd}
\]
射$G_Y \to X_Y \times_{Y} X_Y$は$g \mapsto (\rho_1(e), \rho_2(g))$である.
この図式がpullback diagramであることは
$\Isom(\ (G_Y, \rho_1), (G_Y, \rho_2)\ )$がalgebraic spaceであることを意味している.

\subsection{\tp{$[X/G]$}{The Quotient Stack} has an Atlas.}
射$a \colon X \to [X/G]$を
trivial $G_X$-torsor :: $(G_X, m_X) \in [X/G](X)$に
($2$-Yoneda Lemmaによって)対応する射とする.
この射$a$がatlasであることを示す.
$a$がrepresentableであることは$\Diag$がrepresentableであることから分かる.
$a$がsmooth, surjectiveであることを示す.

以下のpullback diagramを考える.
\[
\begin{tikzcd}
    \cat{P} \ar[r]\ar[d]& X \ar[d, "{(G_X, m_X)}"]\\
    Y \ar[r, "{(\shP, \pi)}"']& {[X/G]}
    \ar[from=2-2, to=1-1, phantom, "\text{p.b.}"]
\end{tikzcd}
\]
ただし$Y \to [X/G]$は$a$と同様に$(\shP, \pi) \in [X/G](Y)$に対応する射である.
$\shP$は作用$\alpha \colon G_Y \times \shP \to \shP$を持つとする.

この時,$\cat{P}$はsheafとして$\shP$と同型である.
これは次のように同型が得られる.
まず$\cat{P}(U)$は以下の$3$つの組の集合である.
\begin{itemize}
    \item $x \colon U \to X$,
    \item $y \colon U \to Y$,
    \item $\rho \colon x^*(G_X, m_X) \isomap y^*(\shP, \pi))$.
\end{itemize}
ただし$x,y$について以下が可換.
\[
\begin{tikzcd}
    U \ar[r, "x"]\ar[d, "y"']& X \ar[d]\\
    Y \ar[r]& S
\end{tikzcd}
\]
上のような$x,y$を一つとり,$x_0, y_0$と名前を付ける.
すると$(y_0)^*G_Y=(x_0)^*G_X$となる.
$\cat{P}(U)$の元$(x,y, \rho)$が存在するならば,
$x^*X=y^*X_Y$ゆえに$x,y$は上の可換図式を満たすことに注意せよ.

次のように同型を定める.
\begin{defmap}
    {}&\cat{P}(U)& \to& \shP(U) \\
    {}&(x, y, \rho)& \mapsto& \rho_{U}(e) \\
    {}&(x_0, y_0, (y_0)^*\alpha(-, p))& \mapsfrom& p
\end{defmap}
$(f^*\shP)(U)=\shP(U)$は$f^*$のcolomitを用いた定義から得られる.

最後に$\shP \to Y$がetale surjectiveであることを確かめる.
$\shP(Y_i) \neq \emptyset$となる$Y$のcover :: $\{Y_i \to Y\}$をとる.
$\shP|_{Y_i}$はtrivial torsor :: $G_{Y_i}$と同型となる.
$\pr_{Y_i} \colon G_{Y_i} \to Y_i$はsmooth surjectiveであり,
smooth, surjectiveはどちらもetale local on targetな性質なので,
$\shP \to Y$もetale, surjective.
(ついでに$\shP$ :: algebraic spaceも分かる.)

\bibliographystyle{jplain}
\bibliography{reference}
\end{document}
