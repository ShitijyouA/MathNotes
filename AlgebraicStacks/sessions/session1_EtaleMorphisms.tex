\documentclass[a4paper]{jsarticle}
\usepackage{macros, enumitem}
%\setlist[description]{style=nextline}
\newcommand{\diag}{\Delta}
\newcommand{\Et}{\mathrm{Et}}
\newcommand{\nl}{\hfill \vspace{-0.5cm}}

\begin{document}
\title{ゼミノート \#1 \\ Etale Morphisms}
\author{七条彰紀}
\maketitle
\section{定義}
\begin{Def}[Infinitesimal Thickening, Formally Smooth/Unramified/Etale]
    \begin{enumerate}[label=(\roman*), leftmargin=*]
    \item
    $i \colon Y_0' \inclmap Y'$ :: closed embeddingについて,
    defining ideal :: $\ker i^{\#}$がnilpotent \footnote{i.e. $\Exists{n > 0} (\ker i^{\#})^n=0$}であるとき,
    $Y_0'$を$Y'$のinfinitesimal thickening(無限小肥大?)と呼ぶ.
    あるいは$i$をinfinitesimal thickeningと呼ぶ.

    \item
    $Y'$ :: affine $Y$-scheme, $Y_0' (\inclmap Y')$ :: infinitesimal thickening of $Y'$とする.
    $f \colon X \to Y$について,以下の図式を見よ.
    \[\xymatrix{
        Y_0' \ar@{_{(}->}[d]_-{\text{inf. thi.}} \ar@[red][r]& X \ar[d]^-{f} \\
        Y' \ar[r] \ar@[blue][ru]& Y
    }\]
    この時,次の写像が定まる.
    \begin{defmap}
        {}& \Hom_Y(Y', X)& \to& \Hom_Y(Y_0', X) \\
        {}& {\color{blue}\to}& \mapsto& {\color{red}\to}
    \end{defmap}
    この写像がsurjective injective,bijectiveであるとき,
    それぞれformally smooth, formally unramified, formally etaleという.
\end{enumerate}
\end{Def}

\begin{Def}[(Locally) Of Finite Presented Module/Algebra/Sheaf/Morphism]
    \begin{enumerate}[label=(\roman*), leftmargin=*] \hfill \vspace{-0.5cm}
    \item
        $R$-module :: $M$がfinitely presented moduleであるとは,
        次の完全列が存在すること.
        \[\xymatrix{
            A^{\oplus r} \ar[r]& A^{\oplus s} \ar[r]& M \ar[r]& 0
        }\]

    \item
        surjective ring homomorphism :: $\phi \colon R[x_1, \dots, x_s] \to A$が存在し,
        $\ker \phi$がfinitely generated idealであるとき,
        $A$ :: finitely presented $R$-algebra (of finite presentation over $R$)という.

    \item
        $\shF$ :: quasi-coherent sheaf on a scheme $X$とする.
        $\shF$ :: locally finitely presentedとは,
        任意のaffine open subscheme of $X$ :: $\Spec A \subseteq X$について,
        $\Gamma(\Spec B, \shF)$がfinitely presented $B$-\underline{module}であること.

    \item
        $f \colon X \to Y$ :: locally of finite presentationであるとは,
        任意の$\Spec B \subseteq Y$と$\Spec A \subseteq f^{-1}(\Spec B)$について,
        $A$ :: finitely presented $B$-algebraであるということ.
        あるいは(同値な条件として),
        affine open cover of $Y$ :: $Y=\bigcup_i \Spec B_i$が存在して,
        任意の$\Spec A_{ij} \subseteq f^{-1}(\Spec B_i)$について,
        $A_{ij}$ :: finitely presented $B_{i}$-algebraであるということ.

    \item
        $f \colon X \to Y$がquasi-compactであるとは,
        任意のaffine open subset of $X$ :: $\Spec A$について$f^{-1}(\Spec A)$ :: quasi-compactであること.
        あるいは(同値な条件として),
        affine open cover of $Y$ :: $Y=\bigcup_i \Spec B_i$が存在して,
        $f^{-1}(\Spec B_i)$ :: quasi-compactであること.
        
    \item
        $f \colon X \to Y$がquasi-separatedであるとは.
        またdiagonal morphism :: $\Delta \colon X \to X \times_{Y} X$
        \footnote
        {
            $\Delta$は以下のようにpullbackの普遍性から得られる射である.
            \[\xymatrix{
                    X \ar@/^5mm/[rr]\ar[rd]\ar@{-->}[r]^-{\Delta}& X \times X \ar[d]\ar[r] & X\ar[d]^-{f} \\
                {}& X\ar[r]_-{f} & Y \ar@{}[lu]|{\text{p.b.}}
            }\]
        }
        がquasi-compactであること.

    \item
        $f \colon X \to Y$がlocally of finite presentationかつquasi-compactかつquasi-separatedである時,
        $f$ :: finitely presentedという.
\end{enumerate}
\end{Def}
環$R$やscheme :: $Y$をnoetherianとすれば,
(locally) of finite presentationと(locally) of finite typeは同値に成る.
一般に(locally) of finite presentationの方が強い条件である(例を参照せよ).

\begin{Def}[Smooth/Unramified/Etale]
    morphism :: $f \colon X \to Y$は,
    formally smooth / unramified / etaleかつfinitely presentedならば
    smooth / unramified / etaleという.
\end{Def}
unramifiedについては,finite typeのみ要求する定義もある.
finitely presentedを要求するのはEGAからのもので,
我々が主に参照している\cite{ASS}もこの定義を取っている.

\section{定義に対する例}

\begin{Example}\label{example:not_qsep}
    locally of finite presentationかつquasi-compactだがNOT quasi-separatedである例を挙げる.

    以下のように設定する.
    \begin{itemize}
        \item $k$ :: field,
        \item $Y=\Spec k[x_1,x_2,\dots]$,
        \item $z=(x_1, x_2, \dots) \in Y$,
        \item $U=Y-\{z\}$.
    \end{itemize}
    この時,$U$はquasi-compactでない.
    これは$U$ :: quasi-compact $\iff$ $z$ :: finitely generatedからわかる
    \footnote
        {
            私のノート: \url{https://github.com/ShitijyouA/MathNotes/blob/master/Hartshorne_AG_Ch2/section2_ex.pdf}
            補題 Ex2.13.2 (II)に証明がある.
        }.

    $X$を,二つの$Y$のコピーを$U$で貼り合わせたものとし,
    $X_1, X_2 \subseteq X$をその$Y$のコピーとする.
    すなわち$X_1, X_2 \iso Y$.
    この同型を$\phi_i \colon X_i \to Y$と名付ける.
    このとき,$f \colon X \to Y$を$\phi_1, \phi_2$の$U$に沿った貼り合わせとする.
    こうすると$f|_{X_i}=\phi_i$となる.

    \paragraph{$f$ :: locally of finite presentation.}
    $Y$ :: affine schemeで,$f^{-1}(Y)=X_1 \cup X_2$であり,
    $X_1, X_2 \iso Y$であった.
    なので$f$ :: locally of finite presentation.

    \paragraph{$f$ :: quasi-compact.}
    同じく,$X_1, X_2$ :: quasi-compactなので$f^{-1}(Y)=X_1 \cup X_2$がquasi-compact.

    \paragraph{$f$ :: NOT quasi-separated.}
    $\basesp (X \times_Y X)$と$\diag \colon X \to X \times_Y X$を考えると次のように成る.
    \[ \diag \colon x \mapsto (\phi_1^{-1}(x), \phi_2^{-1}(x)). \]
    一方,$X_1 \times_{Y} X_2 (\subset X \times X)$は,$X_1, X_2 (\iso Y)$がaffineなのでaffine.
    そこで逆像$\diag^{-1}(X_1 \times_{Y} X_2)$を取ると,これは$U$である.
    既に述べたとおり,これはNOT quasi-compact.
\end{Example}

\begin{Example}[Smooth (BUT NOT Etale) Morphism.]
    次のように定める.
    \begin{defmap}
        f\colon& \Spec k[x,y]& \to& \Spec k[t] \\
        {}& (x,y)& \mapsto& x+y
    \end{defmap}
    これはaffine schemeの間の射なのでquasi-separated.
    $f^{-1}(\Spec k[t])=\Spec k[x, y]$がnoetherian schemeなのでfinitely presented.
    あとはformally smoothであることを示せば良い.
    (TODO)
\end{Example}
    
\begin{Example}[Unramified (BUT NOT Etale) Morphism.]
    次のように定める:
    \begin{defmap}
        h\colon& \Spec \frac{\C[x,y]}{(x^2-y^3)}& \to& \Spec \C[x] \\
        {}& (x, y)& \mapsto& x
    \end{defmap}
    $f$の場合と同様に,formally unramifiedだけ示せば良い.
    (TODO)
\end{Example}

\begin{Example}[Etale Morphism.]
    次のように定める:
    \begin{defmap}
        h\colon& \Spec \Q[u, u^{-1}, y]/(y^d-u)& \to& \Spec \Q[t, t^{-1}] \\
        {}& (u, y)& \mapsto& u
    \end{defmap}
    $A=\Q[t, t^{-1}], B=\Q[u, u^{-1}, y]/(y^d-u)$とおくと.
    $h$に対応する環準同型は$h^{\#} \colon A \to B; t \mapsto ua \bmod (y^d-u)$.
    $f$の場合と同様に,formally etaleだけ示せば良い.
    
    以下の図式を考える.
    \[\xymatrix{
            B \ar@[red][r]^-{\alpha}\ar@[blue][rd]^-{\beta}& R/I \\
            A \ar[u]^-{h^{\#}} \ar[r]_-{\phi}& R \ar[u]_-{\pi}
    }\]
    ここで$I \subseteq R$はイデアルで,$I^N=0$となる整数$N>0$が存在する.
    与えられた$\alpha$から図式を可換にする$\beta$を構成し,
    このような$\beta$が$\alpha$に対し唯一つであることを示す.
    まず$\beta$は$t \in B$の像のみで定まることに注意する.
    図式が可換であることと,次が成立することは同値.
    \[ \beta h^{\#}(t)=\beta(u)=\phi(t), \qquad \pi \beta(u)=\alpha(u) \]
    よって$\beta(u)=\phi(t)$で$\beta$を定めれば良い.
    このように定めれば後者も成立する.
    また,この構成から明らかに$\beta$はただ一つ.
\end{Example}

\begin{Example}[Formally Etale BUT NOT Etale Morphism.]
    例(\ref{example:not_qsep})のmorphism :: $f \colon X \to Y$がそうである.
    このことを示すには,Formally etaleであることだけ確かめれば十分.
\end{Example}

\section{命題}
%%% {{{
\begin{Prop} \label{prop:stable_property}
%    $f \colon X \to Y, g \colon Y \to Z$の両方が以下のいずれかの性質$P$を持つとする.
    以下に列挙する性質は,stable under base exchangeかつstable under composition.
    \begin{enumerate}[label=(\arabic*)]
        \item locally of finite presentation,
        \item quasi-compact,
        \item quasi-separated,
        \item of finite presenration,
        \item formally smooth,
        \item formally unramified,
        \item formally etale,
        \item smooth,
        \item unramified,
        \item etale.
    \end{enumerate}    
\end{Prop}
\begin{proof}
    証明が必要なのは(1), (2), (3)と(5), (6), (7)である.
    いずれも簡単なのでここでは証明しない.
\end{proof}

\begin{Def}[smooth of relative dimension $0$]
    morphism :: $f \colon X \to Y$について以下が成立する時,
    $f$ :: smooth of relative dimension $0$と呼ぶ.
    \begin{enumerate}
        \item $f$ :: finite type over $k$,
        \item $f$ :: flat,
        \item
            $X' \subseteq X$, $Y' \subseteq Y$を$f(X') \subseteq Y'$を満たすirreducible componentとする. \mbox{} \\
            この時$\dim X'=\dim Y'+n$,
        \item 任意の$x \in X$について$\dim_{k(x)} (\shDer_{X/Y} \otimes k(x))=n$.
    \end{enumerate}
\end{Def}

\begin{Thm}[\cite{HarAG} Ex.III.10.3] \label{thm:etale_equiv}
    morphism :: $f \colon X \to Y$について以下は同値.
    \begin{enumerate}
        \item $f$ :: etale,
        \item $f$ :: flat and unramified, 
        \item $f$ :: smooth of relative dimension $0$.
    \end{enumerate}
\end{Thm}
\begin{proof}
    (TODO)
\end{proof}

\begin{Prop}
    formally smooth/unramified/etaleはlocally on codomainな性質である.
\end{Prop}
\begin{proof}
    (TODO)
\end{proof}
%% }}}

\begin{Prop}[\cite{ASS} Prop1.3.6 (i)] \label{prop:dermod}
    $f \colon X \to Y$をfinite presentation, quasi-separated morphismとする.
    この時$\shDer_{X/Y}$は次のように成る.
\begin{enumerate}[label=(\roman*)]
    \item $f$ :: smooth \quad \ \,$\implies \,\shDer_{X/Y}$ :: locally free sheaf of finite rank.
    \item $f$ :: unramified $\iff \shDer_{X/Y}=0$.
    \item $f$ :: etale \qquad \ \ $\implies \, \shDer_{X/Y}=0$.
\end{enumerate}
\end{Prop}
%% {{{
\begin{proof}
    証明は\cite{Mat} \S 25の内容を一部使う.
    特に\S 25始めからThm25.1の直前までがわかっていれば良い.

    主張はlocalなものだから,$X=\Spec B, Y=\Spec A$と仮定して良い.
    $f$ :: smoothより$B$ :: finitely presented $A$-algebra.
    $f$に対応する準同型を$\phi \colon A \to B$とする.

    (i)を示すために,$\modDer_{B/A}$ :: projective $B$-moduleを示す
    (projectiveならばlocally freeであることは\cite{StacksProj} section 10.84に証明がある).
    これはすなわち,$B$-moduleの以下の図式に対し,
    図式を可換にする$\tilde{D} \colon \modDer_{B/A} \to M$が存在するということである.
    \[\xymatrix{
        {} & \modDer_{B/A} \ar[d]^-{D}\\
        M \ar@{->>}[r]^-{t}& N
    }\]
    ここで$t$ :: surj.
    
    次の図式を考える.
    \[\xymatrix{
        B \ar[r]^-{f_D}& B[N] \\
        A \ar[r]\ar[u]^-{\phi}& B[M]\ar@{->>}[u]
    }\]
    ここで$B[M]$は\cite{Mat} \S 25でいう$B \ast M$である
    \footnote
    {
        これらは$B$-algebraで,加群としては$B \oplus M$で,
        乗法は$(b, m) \cdot (b', m')=(bb', bm'+b'm)$で定まる.
        重要な特性として,$\pi_M: B[M] \to B; (b, m) \mapsto b$のkernelはsquare-zeroで,
        $\pi_M$の$A$-algebra section (section which is $A$-albgebra morphism)と
        $A$-derivation $B \to M$が一対一に対応する.
    }.
    $B[N]$も同様.
    $f_D$は$A$-derivation :: $D$に対応する射$b \mapsto (b, D(b))$である.
    $B[M] \to B[N]$は$(b, m) \mapsto (b, t(m))$で与えられる射で,
    したがって全射であり核は$0 \oplus (\ker t)$.
    これはsquare-zero idealである.
    そして$\phi$ :: formally smoothであるから,
    図式を可換にする$B \to B[M]$が存在する.
    これに対応する$A$-derivationが所望の$\tilde{D}$である.

    (ii)を示す.
    $R$ :: ring, $I \subseteq R$ :: idealを$I^2=0$を満たすものとする.
    以下が可換図式だったとしよう.
    \[\xymatrix{
        B \ar[r]^-{\theta} \ar[rd]_-{\lambda}& R/I \\
        A \ar[r]\ar[u]^-{\phi}& R\ar[u]_-{\pi}
    }\]
    この時,$\lambda$をlifting of $\theta$と呼ぶ.
    \cite{Mat} \S 25より
    \footnote
    {
        あるいは私のノート
        \url{https://github.com/ShitijyouA/MathNotes/blob/master/Hartshorne_AG_Ch2/section8_ex.pdf}
        のEx8.6(a)の解答より.
    },
    \[
        \Hom_A(\modDer_{B/A}, I)
        =\Der_{A} (B, I)
        =\{\lambda−\lambda′ | \lambda, \lambda' \text{ :: lifting of $\theta$} \}
    \]
    となっている.
    $\phi$ :: formally unramifiedなので,lifting of $\theta$は一つしか無い.
    よって$\Hom_A(\modDer_{B/A}, I)=0$.
    任意の$R, I$についてこれが成立するので,これは$\modDer_{B/A}=0$と同値.

    formally etale $\implies$ formally unramifiedなので(ii)$\implies$(iii)は明らか.
\end{proof}
%% }}}
\begin{Prop}[\cite{ASS} Prop1.3.6 (iii)] \label{prop:smooth_exact}\hfill \vspace{-7mm}
    \begin{itemize}
        \item $g \colon X \to Y$をsmooth morphism,
        \item $i \colon Z \to X$をlocally of finite presented closed embedding
    \end{itemize}
    とする.
    この時,$f=i \circ g \colon Z \to Y$がsmoothであることと,
    以下の列が完全かつlocally splitであることは同値である.
    \[\xymatrix{
        0 \ar[r]& i^*\shI_Z \ar[r]& i^*\shDer_{X/Y} \ar[r]& \shDer_{Z/Y} \ar[r]& 0
    }\]
    ただし$\shI_Z=\ker i^{\#}$.
\end{Prop}
%% {{{
\begin{proof}
    問題はlocalなものであるから,
    $X=\Spec B, Y=\Spec A, Z=\Spec R=\Spec B/I$とする.
    この時,主張にある完全列は次のように成る.
    \[\xymatrix{
            0 \ar[r]& I/I^2 \ar[r]^-{\delta}& \shDer_{B/A} \otimes_A R \ar[r]& \shDer_{R/A} \ar[r]& 0.
    }\]
    ただし$\delta \colon i \bmod I^2 \mapsto d_{B/A}(i) \otimes 1_R$.
    $d_{B/A}$はderivationである.

    \paragraph{方針.}
    左端の$0$を除いたものはSecond Fundamental Exact Sequenceとして知られ,
    \cite{Mat} Thm25.2などで証明されているとおり,常に成立する.
    また,明らかに$f$ :: locally finitely presented.
    したがって我々は,
    \begin{enumerate}[label=(\alph*), leftmargin=*]
        \item $Z=\Spec B/I \to \Spec A=Y$ :: formally smoothと,
        \item $\delta$ :: splitが
    \end{enumerate}
    同値であることを示せば良い.

    \paragraph{(b)の言い換え : $\delta^*$ :: surj.}
    最初に(a) $\implies$ (b)を示す.
    これは任意の$R$-module :: $N$について以下が成立することを示せば良い.
    \[
        \delta^*=(- \circ \delta) \colon
            \Hom_R(\shDer_{B/A} \otimes_A R, N) \iso \Hom_R(\shDer_{B/A}, N) \to \Hom_R(I/I^2, N)
            \text{ :: surj. }
    \]
    ($\iso$はテンソル積の随伴性から得られる.)
    実際,$N=I/I^2$とすると,
    $\delta^*$ :: surjから$\delta$のretractionの存在が言える.
    したがって$\delta$ :: split.
    splitはinjを意味することに注意.

    \paragraph{問題のさらなる言い換え.}
    $\delta^*$を具体的に計算すると,示すべきは次のことであることが分かる.
    \begin{Claim}
        任意の$\phi \in \Hom_R(I/I^2, N)$に対し,
        以下を満たす射$\psi \colon B \to R[N]$が存在する.
        \[\xymatrix@R=10pt{
                B \ar[r]^-{\psi}& R[N] \ar@{->>}[r]^-{\pr_1}& R & =
                    & B \ar@{->>}[r]^-{\bmod I}& R \\
                I \ar@{^{(}->}[r]& B \ar[r]^-{\psi}& R[N] & =
                    & I \ar[r]^-{\bmod I^2}& I/I^2 \ar[r]^-{\phi}& N \ar@{>->}[r]& R[N]
        }\]
    \end{Claim}
    一行目の等号は$\psi$が$\pi_N$のsectionであることを意味し,
    二行目の等号は$\pr_2 \circ \psi \colon B \to N$が$\delta^*(\phi)$に等しいことを意味する.

    \paragraph{$\delta^*$ :: surj.}
    $\psi$は次のように構成する.
    まず次の可換図式を考える.
    \[\xymatrix{
        R \ar@{=}[r]& R \\
        A \ar[r]\ar[u]& B/I^2 \ar@{->>}[u]_-{\pi}
    }\]
    $\pi$は$I/I^2$による剰余をとる写像である.
    $A \to R$ :: formally smoothから,図式を可換にする射$\sigma \colon R \to B/I^2$が存在する.
    この射から次のように同型$R[I/I^2] \iso B/I^2$が作れる.
    \begin{defmap}
        q\colon & B/I^2& \to& R[I/I^2] \\
        {}& \tilde{b}& \mapsto& (\pi(\tilde{b}), (\id-\sigma \pi)( \tilde{b} )) \\
        {}& \pi(r)+\tilde{i}& \mapedfrom& (r, \tilde{i})
    \end{defmap}
    これを元に$\psi$を構成する.
    \[\xymatrix{ B \ar[r]& B/I^2 \ar[r]^{q}& R[I/I^2] \ar[r]^-{\id \oplus \phi}& R[N] }\]
    これが主張の条件を満たすことは自明.

    \paragraph{(b) $\implies$ (a)の言い換え : $I \subseteq \ker \tilde{h}$にする.}
    話を切り替えて(b) $\implies$ (a)を証明しよう
    (これは\cite{Mat}で言及されていない).
    $C$ :: $A$-algebra, $J \subset C$ :: ideal with $J^2=0$とし,
    $D=C/J$とおく.
    以下の図式(1)を考える.
    \[
    (1)\xymatrix{
        R \ar[r]^-{h}& D \\
        B \ar[u]& {} \\
        A \ar[u]\ar[r]& C \ar@{-_>}[uu]
    }
    \qquad
    (2)\xymatrix{
        R \ar[r]^-{h}& D \\
        B \ar[u]\ar@{-->}[rd]^-{\tilde{h}}& {} \\
        A \ar[u]\ar[r]& C \ar[uu]
    }
    \]
    $A \to B$ :: formally smoothなので,図式(2)の破線の射$\tilde{h}$を得る.
    我々の目的は図式を可換にする$R \to C$を見つけることである.
    これには,
    $I \subseteq \ker \tilde{h}$であれば準同型定理から$\tilde{h}=B \to R=B/I \to C$が得られる.
    
    \paragraph{問題の言い換え : $\im \kappa_{h}=0$にする.}
    $\ker (B \to R=B/I)=I$なので$\tilde{h}(I) \subseteq J$.
    また$J^2=0$なので,
    $\tilde{h}$から$\kappa_{\tilde{h}} \colon I/I^2 \to J$が誘導される.
    構成から分かるとおり,示したい$\tilde{h}(I)=0$と$\im \kappa_{\tilde{h}}=0$は同値である.
    そして我々は,以下の通り,$\im \kappa_{j}=0$となる$h$を選ぶことが出来る.

    \paragraph{$h$の構成.}
    仮定より$\tilde{\alpha} \circ \delta=\kappa_{\tilde{h}}$を満たす
    $\tilde{\alpha} \colon \modDer_{B/A} \otimes_A R \to J$が存在する.
    この$\tilde{\alpha}$を以下のように拡張し,$\alpha \colon B \to J$とする:
%    \[\xymatrix{
%        B \ar[r]& B \otimes B \ar[r]& (B \otimes B) \otimes D \ar[r]& (B \oplus \modDer_{B/A}) \otimes D \ar[r]& C \\
%        b \ar@{|->}[r]& b \otimes 1 \ar@{|->}[r]& (b \otimes 1) \otimes 1_D \ar@{|->}[r]&
%        (b \otimes 1_D) \oplus \delta(b \bmod J^2) \ar@{|->}[r]& (\alpha \circ \delta) (b \bmod J^2)
%    }\]
    \begin{defmap}
        \alpha \colon & B& \to& J \\
        {}& b& \mapsto& \tilde{\alpha}(d_{B/A}(b) \otimes 1_R)
    \end{defmap}
    $h=\tilde{h}-\alpha$と置いて,$\kappa_h(I/I^2)$を計算する.
    $i \in I$をとる.
    \begin{align*}
        \kappa_{h}(i \bmod I^2)
        =&\kappa_{\tilde{h}}(i \bmod I^2)-\alpha(d(i) \otimes 1_R) \\
        =&\kappa_{\tilde{h}}(i \bmod I^2)-(\alpha \circ \delta)(i \bmod I^2) \\
        =&\kappa_{\tilde{h}}(i \bmod I^2)-\kappa_{\tilde{h}}(i \bmod I^2) \\
        =&0.
    \end{align*}
    以上より,$h(I)=0$が得られる.
\end{proof}
%% }}}

\begin{Prop}[\cite{StacksProj}, Tag 02G7] \label{prop:unram_qfinite}
    $f \colon X \to Y$ :: finitely presentation and quasi-separatedが
    unramified morphismであることと,以下の条件 (*)は同値:
    (*)
    任意の$y \in Y$について,fiber of $f$ at $y$ :: $X_y$は
    disjoint union of spectra of finite separable field extensions of $k(y)$.
\end{Prop}
%% {{{
\begin{proof}
    \paragraph{(*) $\implies$ $f$ :: unramified.}
    最初は$f$ :: unramifiedであることを仮定する.
    unramifiedはstable under base excange(命題\ref{prop:stable_property})なので,
    $Y=\Spec k$の場合を示せば良い.

    定理(\ref{thm:etale_equiv})と\cite{HarAG} Cor III.9.6より,
    $\dim X=\dim \Spec k=0$.
    また$X$はfinite type over $k$なのでnoetherian.
    したがって$X$ :: artinian
    \footnote{ noetherian schemeの定義にある語``noetherian"を
                ``artinian"に書き換えたのがartinian schemeの定義である. }.
    artinian ringの構造定理(\cite{Ati-Mac} Thm8.7)と$X$ :: quasi-compactより,$X$は
    \[ X=\bigsqcup \Spec A_i \]
    とdisjointに分解でき,各$A_i$はlocal artinian ringである.

    $\Spec A_i \to \Spec k$ :: unramifiedゆえに$\modDer_{A_i/k}=0$なので,
    $A_i$ :: reduced
    \footnote
    {
        $x \in A_i$が$x^{N}=0$を満たすとする.
        $I=\ker(A_i \otimes_k A_i \to A_i)$とすると,
        $x \otimes x^{N-1} \in I$かつ$\not \in I^2$.
        ゆえに$\modDer_{A_i/k}=I/I^2 \neq 0$.
    }.
    $A_i$の素イデアルは唯一つであるから,$A_i$ :: field.
    unramifiedはstable under base extensionなので,
    同様にして$A_i$ :: separable over $k$.

    $A_i$ :: finiely generated $k$-algebraかつ体なので,
    Zariski's Lemmaにより,$A_i/k$ :: finite algebraic extension.
    $A_i$ :: separable field over $k$なので,
    定義より$A_i$ :: finite separable extension.

    \paragraph{$f$ :: unramified $\implies$ (*).}
    次に命題の条件を仮定する.

    まず$K/k$がfinite separable field extensionであるとする.
    するとprimitive element theoremより,
    $\alpha \in K$を用いて$K=k(\alpha)$と書ける.
    $f \in k[x]$を$\alpha$の最小多項式とすると,
    以下が成り立つ.
    \[ 0=d(f(\alpha))=f'(\alpha) \cdot d(\alpha) \in \modDer_{K/k}. \]
    $K/k$ :: separableより,
    $f(\alpha)=0$かつ$f'(\alpha) \neq 0$となる.
    したがって$f'(\alpha)$ :: unit in $\modDer_{K/k}$なので$d(\alpha)=0$.
    よって$\modDer_{K/k}=K \cdot d(\alpha)=0$.

    $k$-algebra :: $A=\bigoplus_i K_i$を
    $K_i$ :: finite separable field extension of $k$の直和とする.
    $\I{p}_i$を第$i$成分以外は$0$である元からなる素イデアルとすると,
    \[ (\modDer_{A/k})_{\I{p}_i}=\modDer_{A_{\I{p}_i}/k}=\modDer_{K_i/k}=0. \]
    $\basesp \Spec A=\{ \I{p}_i \}_i$なので,これで$\modDer_{A/k}=0$が示せた.

    最後に,一般のschemeで考える.
    証明は命題(\ref{prop:dermod})を利用する.
    $\Spec A \subseteq Y$と$\Spec B \subseteq f^{-1}(\Spec A)$をとり,
    $\I{p} \in \Spec A$における$f$のfiberを考える.
    仮定(*)より
    $B \otimes_{A} k(\I{p})$ :: direct sum of finite separable field extensions of $k$.
    ここで,上述のことから
    \[ \modDer_{B \otimes_{A} k(\I{p})/k(\I{p})}=\modDer_{B/A} \otimes_{B} k(\I{p})=0. \]
    $\I{q} \in \Spec (B \otimes_A k(\I{p})) \homeo f^{-1}(\I{p})$を任意に取ると,
    \[
        ( \modDer_{B/A} \otimes_{B} k(\I{p}) )_{\I{q}}
        =(\modDer_{B/A})_{\I{q}} \otimes_{B} k(\I{p})
        =(\modDer_{B/A})_{\I{q}}/\I{q}(\modDer_{B/A})_{\I{q}}
        =0.
    \]
    これは局所環$B_{\I{q}} \otimes_A k(\I{p})$上の加群であるから,
    Nakayama's Lemmaより,$(\modDer_{B/A})_{\I{q}}=0$.
    $\I{q} \in \Spec (B \otimes_A k(\I{p}))$は任意に取っていたから,
    これは$\modDer_{B/A}=0$を意味する.
    命題(\ref{prop:dermod})より,これは$f$ :: unramifiedと同値.
\end{proof}
%% }}}

\begin{Thm}[\cite{StacksProj}, Tag 04HM] \label{thm:impfunc}
    $f \colon X \to Y$をseparated etale morphismとする.
    $y \in Y$に対し$f^{-1}(y)=\{x_1,\dots,x_n \}$とする
    (点が有限個であることは命題(\ref{prop:unram_qfinite})による).
    etale neighbourhood :: $\nu: (U, u) \to (Y, y)$が存在し,
    $X_U=X \times_{\nu, Y, f} U$のdisjoint union decomposition :
    \[ X_U=\bigsqcup_{i, j} V_{i, j} \]
    について$V_{i, j} \iso U$.
\end{Thm}

\begin{Remark}
この定理は代数幾何学版の陰関数定理とも呼べる定理である.
この現象を根拠にしたスローガン「etale morphismは代数幾何学版locally homeomorphism」がある.
証明は,この定理だけのために必要な命題が幾つかあるため,
後の節に行う.
\end{Remark}

\begin{Prop}(Jacobian criterian) \label{prop:jac_cri}
    $f \colon X \to Y$を,locally of finite presentationとする.
    $f$ :: smoothと次の条件(+)は同値である:
    
    任意の点$x \in X$について,
    $x$と$y=f(x) \in Y$の間にaffine neighborhood
    \[
        x \in \Spec B \subset X, \qquad y=f(x) \in \Spec A \subseteq Y
        \qquad (\text{with  }f(\Spec B) \subseteq \Spec A)
    \]
    が存在し,ある$n, s\ (s \leq n)$と$f_1, \dots, f_s \in A[x_1, \dots, x_n]$について
    \[ B \iso \frac{A[x_1, \dots, x_n]}{(f_1, \dots, f_s)}. \]
    さらに,Jacobian matrix ($n \times s$-matrix)
    \[ J=\mat{ \frac{\partial f_i}{\partial x_j} }_{i, j} \]
    は,右逆行列をもつ(i.e. $\Exists{J'} JJ'=I$).
    同値な条件として,
    regular(i.e.行列式がunit element of $B$)な部分$s$正方行列を$J$は持つ.

    さらに,$f$ :: etaleと,この条件で$n=s$であることは同値である.
\end{Prop}
%% {{{
\begin{proof}
    まずは$(f(x) \in) \Spec A$なる$\Spec A$と
    $f(\Spec B) \subseteq \Spec A$かつ$x \in \Spec B$なる$\Spec B$を適当に取る.
    $f$ :: locally of finite presentationから,
    surjective homomorphism
    \[ A[x_1, \dots, x_n] \to B \]
    であってkernel :: $I$が有限生成であると仮定できる.

    命題(\ref{prop:smooth_exact})を
    $\Spec B \to \Spec A[x_1,\dots,x_n]=\affine^n_A \to \Spec A$に適用すれば,
    $f$ :: smoothと以下がsplit exact sequenceであることが同値だと分かる.
    \[\xymatrix{
        0 \ar[r]& I/I^2 \ar[r]^-{\delta}& \modDer_{A[x_1,\dots,x_n]/A} \otimes_A B \ar[r]& \modDer_{B/A} \ar[r]& 0.
    }\]
    特に$\delta$ :: split injectiveが同値である.

    \paragraph{$f$ :: smooth $\implies$ (+).}
    $\{ f_i \}_{i=1}^s$を$I$の生成元とする.
    この時,$\delta$は次のように作用する.
    \[ \mat{\delta(\bar{f}_1) \\ \vdots \\ \delta(f_s)}
    =\mat{\frac{\partial f_i}{\partial x_j}}_{i,j} \mat{dx_1 \\ \vdots \\ dx_n} \]
    ここで現れた行列は主張にあるJacobian matrixである.
    今,命題(\ref{prop:dermod}) (i) と,
    このexact sequenceの存在から$I/I^2$ :: free moduleである.
    そしてこのとき$\{ \bar{f}_i \}_i \ (\bar{f}_i=f_i \bmod I^2)$は$I/I^2$の基底と成る.
    したがって$\{\bar{f}_i\}_i$と$\{dx_j\}_j$はそれぞれの自由加群の基底であるから,
    $\delta$がsplit injectiveであることは,主張にあるJacobian matrixの条件と同値である.

    \paragraph{$f$ :: smooth $\impliedby$ (+).}
    $J$がright inverseを持つならば,明らかに$\delta$がsectionを持つ.
    よって$\delta$ :: split injective.
    
    \paragraph{$f$ :: etale $\iff$ (+) and $n=s$.}
    最初に述べたexact sequenceから,
    $\modDer_{B/A}$の階数は以下のように成る.
    \[
        \rank \modDer_{B/A}
        =\rank (\modDer_{A[x_1,\dots,x_n]/A} \otimes_A B)-\rank (I/I^2)
        =n-s.
    \]
    したがって$n=s \iff \modDer_{B/A}=0 \iff f$ :: formally unramified.
    etale$=$smooth and formally etaleなので,証明できた.
\end{proof}
%% }}}

\begin{Thm}
    scheme :: $S$について,
    category :: $\Et(S)$を以下のように出さめる.
    \begin{description}
        \item[Objects] etale morphism :: $Z \to S$,
        \item[Arrows] $S$-morphism :: $Z \to Z'$.
    \end{description}
    
    $i \colon S_0 \to S$ :: infinitesimal thickeningについて,
    関手$F$を以下で定める.
    \begin{defmap}
        F\colon & \Et(S)& \to& \Et(S_0) \\
        {}& [Z \to S]& \mapsto& [Z \times_S S_0 \to S_0]
    \end{defmap}
    このとき,$F$は圏同値である.
\end{Thm}

\section{定理(\ref{thm:impfunc})の証明.}
\begin{Lemma}
    $R$ :: ring, $f \in R[x]$とする.
    以下のring homomorphismから誘導される$\Spec$間の射は
    etaleである.
    \begin{defmap}
        \phi\colon & R& \to& R[x, 1/f']/(f) \\
        {}& r& \mapsto& r \cdot f'
    \end{defmap}
\end{Lemma}
\begin{proof}
    命題(\ref{prop:jac_cri}) (Jacobian criterion)を利用する.
    この場合,Jacobian :: $\det \mat{f'}=f'$がunit in $R[x, 1/f'](f)$であるから,
    この補題が成り立つ.
\end{proof}

\bibliographystyle{jplain}
\bibliography{reference}
\end{document}
