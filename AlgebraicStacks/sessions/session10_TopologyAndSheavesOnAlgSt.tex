\documentclass[a4paper, dvipdfmx]{jsarticle}
\usepackage{macros}
\newcommand{\vsubseteq}{\text{\rotatebox{-90}{$\subseteq$}}}

\begin{document}
\title{ゼミノート \#10 \\ Topology and Shaves on Algebraic Stacks}
\author{七条彰紀}
\maketitle
\tableofcontents
\vspace{10pt}

ここまででartin stackが定義できたが,
これはschemeで言えばstructure sheafだけ定義したような状態である.
artin stackのZariski位相空間と,
(Grothendieck topology上の)sheafを導入する.

\section{Points of Artin Stack}
いずれも\cite{SP} Tag 04XE, \cite{LMB00} section 5.を参照せよ.

\begin{Def}[\cite{LMB00} section 5]
    体の$\Spec$からの射
    $x_1 \colon \Spec k_1 \to \fibX, x_2 \colon \Spec k_2 \to \fibX$について
    $x_1 \sim x_2$であるとは,
    ある$k_{12}$ ::  field と
    以下の$2$-可換図式が存在すること.

    \[
    \begin{tikzcd}
        \Spec k_{12} \ar[d]\ar[r]& \Spec k_{1} \ar[d, "x_1"] \\
        \Spec k_{2} \ar[r, "x_2"']& \fibX
    \end{tikzcd}
    \]
\end{Def}

\begin{Prop}[ \cite{SP} 04XF ]
    ここで定義した$\sim$は同値関係である.
\end{Prop}
\begin{proof}
    $\sim$は反射律,対称律を満たすことは自明なので,推移律の成立を示す.

    体から$\fibX$への$3$つの射 :: $x_1,x_2,x_3$を考える.
    これらが$x_1 \sim x_2, x_2 \sim x_3$を同時に満たすとは,
    体$k_{12}, k_{23}$と次の$2$-可換図式が存在するということである.
    \[
    \begin{tikzcd}
        \Spec k_{12} \ar[r] \ar[d] & \Spec k_2 \ar[d, "x_2"] &  \Spec k_{23} \ar[l] \ar[d] \\
        \Spec k_1 \ar[r, "x_1"'] &  \fibX &  \Spec k_3 \ar[l, "x_3"]
    \end{tikzcd}
    \]
    この時,$k_{12}, k_{23}$の合成体(すなわち最小の共通の拡大体)を$k_{123}$とする.
    $k_{12} \cap k_{23}$は$k_{123}$の部分体として$k_{2}$に一致する
    (あるいは,一致するように$2$つの準同型$k_{12}, k_{23} \to k_{123}$を選ぶ).
    すると可換図式は次のように拡張される.
    \[
    \begin{tikzcd}
        {} & \ar[ld] \Spec k_{123} \ar[rd]& {} \\
        \Spec k_{12} \ar[r] \ar[d] & \Spec k_2 \ar[d, "x_2"] &  \Spec k_{23} \ar[l] \ar[d] \\
        \Spec k_1 \ar[r, "x_1"'] &  \fibX &  \Spec k_3 \ar[l, "x_3"]
    \end{tikzcd}
    \]
    上の新たな四辺形はschemeの図式として可換なので,
    このartin stackの拡張後の図式も可換.
\end{proof}

\begin{Remark}
    以上の定義はschemeの点に対応している.
    scheme :: $X$について,
    体の$Spec$ :: $\Spec k$から$X$への射は
    点$x \in X$と体の準同型 :: $\phi \colon \kappa(x) \to k$に対応する
    ( \cite{HarAG} ch II, Ex2.7, \cite{SP} 01J5).
    ここで$\kappa(x)$はresidue fieldである.
    したがって一点$x$に対応する射は$\kappa(x)$から体への準同型の数だけ有る.
    これらを全て同値なものとする同値関係を定めたい.

    体から$X$への二つの射
    \[ x_1 \colon \Spec k_1 \to \fibX, \quad x_2 \colon \Spec k_2 \to \fibX \]
    について,以下は同値.
    \begin{enumerate}[label=(\alph*), leftmargin=*]
        \item 
        位相空間の$2$つの写像$|x_1|, |x_2|$の像が$x$である.

        \item
        すなわち,体$k_{12}$と次の可換図式が存在する.
        \[
        \begin{tikzcd}
            \Spec k_{12} \ar[d]\ar[r]& \Spec k_{1} \ar[d, "x_1"] \\
            \Spec k_{2} \ar[r, "x_2"']& X
        \end{tikzcd}
        \]
    \end{enumerate}

    (a) $\implies$ (b)は明らか.
    (a) $\impliedby$ (b)は次のように示す.
    まず$k_{12}$は合成体$k_1k_2$と置けば良い.
    すると包含射$k_1 \inclmap k_{12}, k_2 \inclmap k_{12}$が存在する.
    体の準同型は単射しか無いから,
    $x_1, x_2$からそれぞれ得られる$\kappa(x) \to k_1, \kappa(x) \to k_2$は包含射に取り替えられる.
    包含関係は推移律を満たすから,以下が可換ということに成る.
    \[
    \begin{tikzcd}
        k_{12} & k_{1} \ar[l, hookrightarrow] \\
        k_{2} \ar[u, hookrightarrow]& \kappa(x) \ar[l, hookrightarrow]\ar[u, hookrightarrow]
    \end{tikzcd}
    \]
    上で述べた,体から$X$への射と$\kappa(x)$から体への射の対応より,
    これは(b)の可換図式が存在することを意味する.
\end{Remark}

\begin{Def}[$|\fibX|, |f|$, \cite{SP} 04XG and the below paragraph]
    points of $\fibX$とは,fieldの$\Spec$から$\fibX$の射の,
    $\sim$による同値類のことである.
    set of points of $\fibX$を$|\fibX|$と表す.
    すなわち,
    \[ |\fibX|= \{ \Spec k \to \fibX \mid k\text{ :: algebraically closed field } \} / \sim. \]

    射$f \colon \fibX \to \fibY$について,$|f|$を次で定義する.
    \begin{defmap}
        |f|\colon & |\fibX|& \to& |\fibY| \\
        {}& x & \mapsto& f \circ x
    \end{defmap}
\end{Def}

\section{Zariski Topology of Artin Stack}

\subsection{Atlases of Artin Stacks}
    下準備としてartin stackのatlasについて幾つか命題を述べる.
    最初は読み飛ばして構わない.

    \begin{Lemma}
        任意のartin stackはatlas by a scheme,
        すなわちschemeからのsmooth surjective射を持つ.
    \end{Lemma}
    \begin{proof}
        この証明では
        \textbf{``smooth surjective"を``sm.surj.",``etale surjective"を``et.surj."と略す.}

        artin stackとalgebraic spaceの定義より,
        \begin{itemize}
            \item algebraic spaceからartin stackへのsm.surj.射 :: $\alpha \colon X \to \fibX$,
            \item schemeからalgebraic spaceへのet.surj.射 :: $a \colon U \to X$
        \end{itemize}
        が存在する.
        合成すればschemeからartin stackへのsm.surj.射 :: $\alpha \circ a \colon U \to \fibX$
        が得られる.

        $\alpha$と$a$ではそれぞれ``smooth surjective",``etale surjective"の
        定義の方法が異なるので,
        射$\alpha \circ a$がsm.surj.であることは調べる必要が有る.
        schemeからのsm.surj.射 :: $V \to \fibX$をとり,
        以下のpullback図式を考える.
        \[
        \begin{tikzcd}
            U \times_{\fibX} V \ar[r]\ar[d]& U \ar[d, "a"]\\
            X \times_{\fibX} V \ar[r]\ar[d]& X \ar[d, "\alpha"]\\
            V \ar[r]& \fibX
        \end{tikzcd}
        \]
        この図式から次の$3$つが分かる.
        \begin{itemize}
            \item $V \to \fibX$はschemeからのsm. surj.射,
            \item $a \colon U \to X$がsm. surj.なので$U \times V \to X \times V$もsm. surj.,
            \item $\alpha \colon X \to \fibX$もsm. surj.射.
        \end{itemize}

        artin stackの射の性質の定義($\alpha$がsm. surj.であることの定義)から,
        $U \times V \to X \times V \to V$はsm. surj..
        two pullback lemmaも合わせて考えれば,
        これは$\alpha \circ a$ :: sm. surj.を意味する.
    \end{proof}

    \begin{Lemma}[\cite{SP} tag 04T1]
        artin stack :: $\fibX, \fibY$と$\fibY$のatlas :: $V \to \fibY$をとる.
        morphism of artin stacks :: $f \colon \fibX \to \fibY$に対して,
        $\fibX$のatlas :: $U \to \fibX$とatlasの間の射 :: $\bar{f} \colon U \to V$が存在し,
        以下が可換図式となる.
        \[
        \begin{tikzcd}
            {}^{\exists} U \ar[r, "{}^{\exists}\bar{f}"]\ar[d]& {}^{\forall} V \ar[d]\\
            \fibX \ar[r, "{}^{\forall} f"']& \fibY
        \end{tikzcd}
        \]

        schemeの射の性質$P$を,
        smooth surjective morphismによるcompositionとbase changeで保たれるものとする.
        \footnote
        {
            例えば$P=$ smooth, surjective, flat, locally finite presentation, universally open. 
            \cite{SP} tag 01V4.
        }
        $f$が性質$P$を持つならば$\bar{f}$も性質$P$を持つ.
    \end{Lemma}
    \begin{proof}
        atlas of $\fibX$ :: $U \to \fibX$を適当にとり,次のfiber productをとる.
        \[
        \begin{tikzcd}
            U \times_{\fibY} V \ar[rr]\ar[d]& {} & V \ar[d]\\
            U \ar[r]& \fibX \ar[r, "f"']& \fibY
        \end{tikzcd}
        \]
        artin stackの定義から$U \times_{\fibY} V$ :: alg. sp.である.
        またsmooth, surjectiveはstable under base change/compositionなので
        $U \times_{\fibY} V \to U \to \fibX$はsmooth surjective.
        よって$\bar{V}=U \times_{\fibY} V, \bar{f}=\pr \colon U \times_{\fibY} V \to V$
        と置けばこれらが主張の条件を満たす.
        また,この証明から最後の段落の主張は明らかである.
    \end{proof}

\subsection{Definitions.}

\begin{Def}[Zariski Topology on Points of Scheme/Algebraic Space/Artin Stack]
\begin{myenum}
    \item
        scheme :: $X$とする.
        $U \subseteq |X|$が(Zariski) openであるとは,
        あるopen subscheme of $X$ :: $\bar{U}$が存在して$U=|\bar{U}|$であること.

    \item
        algebraic spspace :: $X$とし,
        $A$がschemeであるatlas :: $a \colon A \to X$をとる.
        $U \subseteq |X|$が(Zariski) open subsetであるとは,
        $a^{-1}(U)$が$|A|$のopen subsetであること.

    \item
        artin stactk :: $\fibX$とし,
        $A$がschemeであるatlas :: $a \colon A \to \fibX$をとる.
        $U \subseteq |\fibX|$が(Zariski) open subsetであるとは,
        $a^{-1}(U)$が$|A|$のopen subsetであること.
\end{myenum}
\end{Def}

\subsection{Propositions}

\begin{Prop}[\cite{SP} 04XL]
\begin{myenum}
    \item
        artin stack間の任意の射$f \colon \fibX \to \fibY$について,
        $|f| \colon |\fibX| \to |\fibY|$はcontinuous.
    \item
        algebraic spaceからのflat and locally of finite presentation射 :: $f \colon U \to \fibX$
        に対して,$|f|$はcontinuousかつopen.
\end{myenum}
\end{Prop}

(i)の証明は簡単.(ii)はTag  042Sを用いる.

\subsubsection{Open sub-stack maps to open subset bijectively.}
\begin{Remark}
    artin stackの射にも``open immersion"であるものは存在するのだから,
    これを用いてもopen morphismなどの概念が定義できる.
    この流儀でのopen morphism等の概念と,
    我々のpoints of artin stack :: $|\fibX|$を使う流儀でのopen morphism等の概念は
    同値なものである,
    ということを次の命題(\ref{prop:Uto|U|})で示す.

    points of artin stackを使うと,台集合を$|\fibX|$とする,
    通常の意味での位相空間が定義できる.
    その為,位相空間に関する概念を全て取り扱うことが出来る,
    というのが我々の流儀のアドバンテージである.
\end{Remark}

\begin{Def}
    artin stack :: $\fibX$のopen sub-stackとは,
    $\fibX$の \textbf{strictly full} sub-category :: $\fibU$
    \footnote{ すなわち$\fibU$は$\fibX$の全ての対象と同型射を含んでいる. }
    でartin stackであるものであって
    かつ$\fibX$へのinclusion :: $\fibU \to \fibX$がopen immersionであるもの.
\end{Def}

\begin{Remark}
    equivalence of artin stacks :: $f \colon \fibX \to fibY$があっても,
    open sub-stack of $\fibX$の$f$による像が
    strictly full sub-categoryであるとは限らないことに注意.
\end{Remark}

\begin{Prop}[\cite{SP} 06FJ, \cite{LMB00} Cor5.6.1]\label{prop:Uto|U|}
\begin{myenum}
    \item $\fibU$がopen sub-stack of $\fibX$ならば$|\fibU|$は$|\fibX|$のopen sub-set.
    \item open sub-stack of $\fibX$の集まりからの対応$\fibU \mapsto |\fibU|$は一対一.
\end{myenum}
\end{Prop}

\subsubsection{Surjectivity.}
\begin{Lemma}
    任意のartin stack :: $\fibX, \fibY, \fibZ$について,
    \[ |\fibZ \times_{\fibY} \fibX| \to |\fibZ| \times _{|\fibY|} |\fibX| \]
    は全射である.
\end{Lemma}

\begin{Lemma}
    $f \colon \fibX \to \fibY$が全射であることと,
    $|f|\colon |\fibX| \to |\fibY|$が全射であることは同値である.
\end{Lemma}

\subsubsection{Topological Property of \tp{$|\fibX|$}{|X|}.}
\begin{Prop}[\cite{LMB00} 5.6.1(iii), 5.7.2]
    atrin stack :: $\fibX$を考える.
    位相空間$|\fibX|$について次が成り立つ.
    \begin{enumerate}
        \item $|\fibX|$ :: quasi-compact.
        \item $|\fibX|$ :: sober
            (すなわち,任意のirreducible componentはただ一つのgeneric pointを持つ.)
    \end{enumerate}
\end{Prop}

\begin{Prop}[\cite{LMB00} 5.7]
    artin stackのquasi-compact射 :: $f \colon \fibX \to \fibY$について,
    $|f|(\fibX)$ :: stable under specialization.
\end{Prop}

\section{Sheaves on Algebraic Stacks}


\bibliographystyle{jplain}
\bibliography{../reference}
\end{document}
