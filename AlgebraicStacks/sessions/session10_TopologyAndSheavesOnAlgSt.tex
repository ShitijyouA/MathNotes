\documentclass[a4paper, dvipdfmx]{jsarticle}
\usepackage{macros}
\newcommand{\vsubseteq}{\text{\rotatebox{-90}{$\subseteq$}}}
\newcommand{\xto}[1]{\xrightarrow{#1}}

\begin{document}
\title{ゼミノート \#10 \\ Topology and Shaves on Algebraic Stacks}
\author{七条彰紀}
\maketitle
\tableofcontents
\vspace{10pt}

ここまででartin stackが定義できたが,
これはschemeで言えばstructure sheafだけ定義したような状態である.
artin stackのZariski位相空間と,
(Grothendieck topology上の)sheafを導入する.

\section{Points of Artin Stack}
いずれも\cite{SP} Tag 04XE, \cite{LMB00} section 5.を参照せよ.

\begin{Def}[\cite{LMB00} section 5]
    体の$\Spec$からの射
    $x_1 \colon \Spec k_1 \to \fibX, x_2 \colon \Spec k_2 \to \fibX$について
    $x_1 \sim x_2$であるとは,
    ある$k_{12}$ ::  field と
    以下の$2$-可換図式が存在すること.

    \[
    \begin{tikzcd}
        \Spec k_{12} \ar[d]\ar[r]& \Spec k_{1} \ar[d, "x_1"] \\
        \Spec k_{2} \ar[r, "x_2"']& \fibX
    \end{tikzcd}
    \]
\end{Def}

\begin{Prop}[ \cite{SP} 04XF ]
    ここで定義した$\sim$は同値関係である.
\end{Prop}
\begin{proof}
    $\sim$は反射律,対称律を満たすことは自明なので,推移律の成立を示す.

    体から$\fibX$への$3$つの射 :: $x_1,x_2,x_3$を考える.
    これらが$x_1 \sim x_2, x_2 \sim x_3$を同時に満たすとは,
    体$k_{12}, k_{23}$と次の$2$-可換図式が存在するということである.
    \[
    \begin{tikzcd}
        \Spec k_{12} \ar[r] \ar[d] & \Spec k_2 \ar[d, "x_2"] &  \Spec k_{23} \ar[l] \ar[d] \\
        \Spec k_1 \ar[r, "x_1"'] &  \fibX &  \Spec k_3 \ar[l, "x_3"]
    \end{tikzcd}
    \]
    この時,$k_{12}, k_{23}$の合成体(すなわち最小の共通の拡大体)を$k_{123}$とする.
    $k_{12} \cap k_{23}$は$k_{123}$の部分体として$k_{2}$に一致する
    (あるいは,一致するように$2$つの準同型$k_{12}, k_{23} \to k_{123}$を選ぶ).
    すると可換図式は次のように拡張される.
    \[
    \begin{tikzcd}
        {} & \ar[ld] \Spec k_{123} \ar[rd]& {} \\
        \Spec k_{12} \ar[r] \ar[d] & \Spec k_2 \ar[d, "x_2"] &  \Spec k_{23} \ar[l] \ar[d] \\
        \Spec k_1 \ar[r, "x_1"'] &  \fibX &  \Spec k_3 \ar[l, "x_3"]
    \end{tikzcd}
    \]
    上の新たな四辺形はschemeの図式として可換なので,
    このartin stackの拡張後の図式も可換.
\end{proof}

\begin{Remark}\label{rem:sch-points}
    以上の定義はschemeの点に対応している.
    scheme :: $X$について,
    体の$Spec$ :: $\Spec k$から$X$への射は
    点$x \in X$と体の準同型 :: $\phi \colon \kappa(x) \to k$に対応する
    ( \cite{HarAG} ch II, Ex2.7, \cite{SP} 01J5).
    ここで$\kappa(x)$はresidue fieldである.
    したがって一点$x$に対応する射は$\kappa(x)$から体への準同型の数だけ有る.
    これらを全て同値なものとする同値関係を定めたい.

    体から$X$への二つの射
    \[ x_1 \colon \Spec k_1 \to \fibX, \quad x_2 \colon \Spec k_2 \to \fibX \]
    について,以下は同値.
    \begin{enumerate}[label=(\alph*), leftmargin=*]
        \item 
        位相空間の$2$つの写像$|x_1|, |x_2|$の像が$x$である.

        \item
        すなわち,体$k_{12}$と次の可換図式が存在する.
        \[
        \begin{tikzcd}
            \Spec k_{12} \ar[d]\ar[r]& \Spec k_{1} \ar[d, "x_1"] \\
            \Spec k_{2} \ar[r, "x_2"']& X
        \end{tikzcd}
        \]
    \end{enumerate}

    (a) $\implies$ (b)は明らか.
    (a) $\impliedby$ (b)は次のように示す.
    まず$k_{12}$は合成体$k_1k_2$と置けば良い.
    すると包含射$k_1 \inclmap k_{12}, k_2 \inclmap k_{12}$が存在する.
    体の準同型は単射しか無いから,
    $x_1, x_2$からそれぞれ得られる$\kappa(x) \to k_1, \kappa(x) \to k_2$は包含射に取り替えられる.
    包含関係は推移律を満たすから,以下が可換ということに成る.
    \[
    \begin{tikzcd}
        k_{12} & k_{1} \ar[l, hookrightarrow] \\
        k_{2} \ar[u, hookrightarrow]& \kappa(x) \ar[l, hookrightarrow]\ar[u, hookrightarrow]
    \end{tikzcd}
    \]
    上で述べた,体から$X$への射と$\kappa(x)$から体への射の対応より,
    これは(b)の可換図式が存在することを意味する.
\end{Remark}

\begin{Def}[$|\fibX|, |f|$, \cite{SP} 04XG and the below paragraph]
    points of $\fibX$とは,fieldの$\Spec$から$\fibX$の射の,
    $\sim$による同値類のことである.
    set of points of $\fibX$を$|\fibX|$と表す.
    すなわち,
    \[ |\fibX|= \{ \Spec k \to \fibX \mid k\text{ :: algebraically closed field } \} / \sim. \]

    射$f \colon \fibX \to \fibY$について,$|f|$を次で定義する.
    \begin{defmap}
        |f|\colon & |\fibX|& \to& |\fibY| \\
        {}& x & \mapsto& f \circ x
    \end{defmap}
\end{Def}

\section{Zariski Topology of Artin Stack}

\subsection{Atlases of Artin Stacks}
    下準備としてartin stackのatlasについて幾つか命題を述べる.
    最初は読み飛ばして構わない.

    \begin{Lemma}
        任意のartin stackはatlas by a scheme,
        すなわちschemeからのsmooth surjective射を持つ.
    \end{Lemma}
    \begin{proof}
        この証明では
        \textbf{``smooth surjective"を``sm.surj.",``etale surjective"を``et.surj."と略す.}

        artin stackとalgebraic spaceの定義より,
        \begin{itemize}
            \item algebraic spaceからartin stackへのsm.surj.射 :: $\alpha \colon X \to \fibX$,
            \item schemeからalgebraic spaceへのet.surj.射 :: $a \colon U \to X$
        \end{itemize}
        が存在する.
        合成すればschemeからartin stackへのsm.surj.射 :: $\alpha \circ a \colon U \to \fibX$
        が得られる.

        $\alpha$と$a$ではそれぞれ``smooth surjective",``etale surjective"の
        定義の方法が異なるので,
        射$\alpha \circ a$がsm.surj.であることは調べる必要が有る.
        schemeからのsm.surj.射 :: $V \to \fibX$をとり,
        以下のpullback図式を考える.
        \[
        \begin{tikzcd}
            U \times_{\fibX} V \ar[r]\ar[d]& U \ar[d, "a"]\\
            X \times_{\fibX} V \ar[r]\ar[d]& X \ar[d, "\alpha"]\\
            V \ar[r]& \fibX
        \end{tikzcd}
        \]
        この図式から次の$3$つが分かる.
        \begin{itemize}
            \item $V \to \fibX$はschemeからのsm. surj.射,
            \item $a \colon U \to X$がsm. surj.なので$U \times V \to X \times V$もsm. surj.,
            \item $\alpha \colon X \to \fibX$もsm. surj.射.
        \end{itemize}

        artin stackの射の性質の定義($\alpha$がsm. surj.であることの定義)から,
        $U \times V \to X \times V \to V$はsm. surj..
        two pullback lemmaも合わせて考えれば,
        これは$\alpha \circ a$ :: sm. surj.を意味する.
    \end{proof}

    \begin{Lemma}[\cite{SP} tag 04T1] \label{lemm:induced_mor_of_atlases}
        artin stack :: $\fibX, \fibY$と$\fibY$のatlas :: $V \to \fibY$をとる.
        morphism of artin stacks :: $f \colon \fibX \to \fibY$に対して,
        $\fibX$のatlas :: $U \to \fibX$とatlasの間の射 :: $\bar{f} \colon U \to V$が存在し,
        以下が可換図式となる.
        \[
        \begin{tikzcd}
            {}^{\exists} U \ar[r, "{}^{\exists}\bar{f}"]\ar[d]& {}^{\forall} V \ar[d]\\
            \fibX \ar[r, "{}^{\forall} f"']& \fibY
        \end{tikzcd}
        \]

        schemeの射の性質$P$を,
        smooth surjective morphismによるcompositionとbase changeで保たれるものとする.
        \footnote
        {
            例えば$P=$ smooth, surjective, flat, locally finite presentation, universally open. 
            \cite{SP} tag 01V4.
        }
        $f$が性質$P$を持つならば$\bar{f}$も性質$P$を持つ.
    \end{Lemma}
    \begin{proof}
        atlas of $\fibX$ :: $U \to \fibX$を適当にとり,次のfiber productをとる.
        \[
        \begin{tikzcd}
            U \times_{\fibY} V \ar[rr]\ar[d]& {} & V \ar[d]\\
            U \ar[r]& \fibX \ar[r, "f"']& \fibY
        \end{tikzcd}
        \]
        artin stackの定義から$U \times_{\fibY} V$ :: alg. sp.である.
        またsmooth, surjectiveはstable under base change/compositionなので
        $U \times_{\fibY} V \to U \to \fibX$はsmooth surjective.
        よって$\bar{V}=U \times_{\fibY} V, \bar{f}=\pr \colon U \times_{\fibY} V \to V$
        と置けばこれらが主張の条件を満たす.
        また,この証明から最後の段落の主張は明らかである.
    \end{proof}

\subsection{Definitions.}

\begin{Def}[Zariski Topology on Points of Scheme/Algebraic Space/Artin Stack]
\begin{myenum}
    \item
        scheme :: $X$とする.
        $|X|$の(Zariski) open subsetとは,
        あるopen subscheme of $X$ :: $i \colon U \to X$によって
        $|i|(|U|)$とかける集合のこと.

    \item
        algebraic space :: $X$とし,
        $A$がschemeであるatlas :: $a \colon A \to X$をとる.
        $U \subseteq |X|$が(Zariski) open subsetであるとは,
        $|a|^{-1}(U)$が$|A|$のopen subsetであること.

    \item
        artin stactk :: $\fibX$とし,
        $A$がschemeであるatlas :: $a \colon A \to \fibX$をとる.
        $U \subseteq |\fibX|$が(Zariski) open subsetであるとは,
        $|a|^{-1}(U)$が$|A|$のopen subsetであること.
\end{myenum}
\end{Def}

\begin{Def}
    $P$を位相空間の性質(e.g. irreducible, connected, quasi-compact, ...)とする.
    artin stack :: $|\fibX|$が$P$であるとは,
    $|\fibX|$が$P$であるということ.

    $Q$を位相空間の射の性質(e.g. open, closed, dense, ...)とする.
    artin stackの射 :: $f \colon \fibX \to \fibY$が$Q$であるとは,
    $|f|$が$Q$であるということ.
\end{Def}

\begin{Lemma}
    $X$ :: schemeについて,$|X|$は$X$の台位相空間と一致する.
    さらにschemeの射 :: $f \colon X \to Y$について,
    $|f|$は$X$と$Y$の間の台位相空間の射と一致する.
\end{Lemma}
\begin{proof}
    注意(\ref{rem:sch-points})より明らか.
\end{proof}

\begin{Lemma}
    artin stactk :: $\fibX$について,
    $|\fibX|$のZariski topologyはatlasに関わらず一意である.
\end{Lemma}
\begin{proof}
    schemeから$\fibX$へのsmooth surjective morphismを二つとり,
    それらのfiber productを作る.
    \[
    \begin{tikzcd}
        W \ar[r]\ar[d]& U \ar[d, "u"]\\
        V \ar[r, "v"']& \fibX \ar[lu, phantom, "\text{p.b.}"]
    \end{tikzcd}
    \]
    artin stackの定義から,$W$ :: scheme.
    またsmoothならばuniversally openである(\cite{SP} 04XL)から,
    $W \to U, W \to V$はcontinuous, surjective, open.
    よって集合の間の射$|W| \to |U|, |W| \to |V|$もcontinuous, surjective, open.

    なので以上の可換図式をたどれば,
    任意の$O \subseteq |\fibX|$について,
    $|u|^{-1}(O) \subseteq |U|$がopenであることと$|v|^{-1}(O) \subseteq |V|$がopenであることが
    同値であると分かる.
    これは
    $|\fibX|$のZariski topologyはatlasに関わらず一意であることを意味する.
\end{proof}

\subsection{Propositions}

\begin{Prop}[\cite{SP} 04XL]
\begin{myenum}
    \item
        artin stack間の任意の射$f \colon \fibX \to \fibY$について,
        $|f| \colon |\fibX| \to |\fibY|$はcontinuous.
    \item
        algebraic spaceからのuniversally open射 :: $f \colon U \to \fibX$
        に対して,$|f|$はcontinuousかつopen.
\end{myenum}
    なお,smooth射はflat and locally of finite presentation射であり,
    したがってuniversally openである
    (\cite{SP} tag 01VE, 01VF, 01UA).
\end{Prop}
\begin{proof}
    (i)は補題(\ref{lemm:induced_mor_of_atlases})を用いれば容易に分かる.
\end{proof}

\subsubsection{Surjectivity.}
\begin{Lemma}
    任意のartin stack :: $\fibX, \fibY, \fibZ$について,
    \[ |\fibZ \times_{\fibY} \fibX| \to |\fibZ| \times _{|\fibY|} |\fibX| \]
    は全射である.
\end{Lemma}

\begin{Lemma}
    $f \colon \fibX \to \fibY$が全射であることと,
    $|f|\colon |\fibX| \to |\fibY|$が全射であることは同値である.
\end{Lemma}

\subsubsection{Open sub-stack maps to open subset bijectively.}
\begin{Remark}
    artin stackの射にも``open immersion"であるものは存在するのだから,
    これを用いてもopen morphismなどの概念が定義できる.
    この流儀でのopen morphism等の概念と,
    我々のpoints of artin stack :: $|\fibX|$を使う流儀でのopen morphism等の概念は
    同値なものである,
    ということを次の命題(\ref{prop:Uto|U|})で示す.

    points of artin stackを使うと,台集合を$|\fibX|$とする,
    通常の意味での位相空間が定義できる.
    その為,位相空間に関する概念を全て取り扱うことが出来る,
    というのが我々の流儀のアドバンテージである.
\end{Remark}

\begin{Def}[\cite{SP} 04YM]
    artin stack :: $\fibX$のopen sub-stackとは,
    $\fibX$の \textbf{strictly full} sub-category :: $\fibU$
    \footnote{ すなわち$\fibU$は$\fibX$の全ての対象と同型射を含んでいる. }
    でartin stackであるものであって
    かつ$\fibX$へのinclusion :: $\fibU \to \fibX$がopen immersionであるもの.
    closed sub-stackも同様である.
\end{Def}

\begin{Remark}
    equivalence of artin stacks :: $f \colon \fibX \to \fibY$があっても,
    open sub-stack of $\fibX$の$f$による像が
    strictly full sub-categoryであるとは限らないことに注意.
\end{Remark}

\begin{Prop}[\cite{SP} 06FJ, \cite{LMB00} Cor5.6.1]\label{prop:Uto|U|}
\begin{myenum}
    \item $\fibU$がopen sub-stack of $\fibX$ならば$|\fibU|$は$|\fibX|$のopen sub-set.
    \item open sub-stack of $\fibX$の集まりからの対応$\fibU \mapsto |\fibU|$は一対一.
\end{myenum}
    これらはclosedについても同様である.
\end{Prop}
\begin{proof}
    (TODO)
\end{proof}

\subsubsection{Topological Property of \tp{$|\fibX|$}{|X|}.}
\begin{Prop}[\cite{LMB00} 5.6.1(iii), 5.7.2]
    atrin stack :: $\fibX$を考える.
    位相空間$|\fibX|$について次が成り立つ.
    \begin{enumerate}
        \item $|\fibX|$ :: quasi-compact.
        \item $|\fibX|$ :: sober
            (すなわち,任意のirreducible componentはただ一つのgeneric pointを持つ.)
    \end{enumerate}
\end{Prop}

\begin{Prop}[\cite{LMB00} 5.7]
    artin stackのquasi-compact射 :: $f \colon \fibX \to \fibY$について,
    $|f|(\fibX)$ :: stable under specialization.
\end{Prop}

\section{Sheaves on Algebraic Stacks}

\subsection{Definitions}
artin stack over $S$ :: $\fibX$について,
schemeからなるbig etale site :: $\ET(\fibX)=(\Sch/\fibX)_{\ET}$を
我々は考える.
$\ET(\fibX)$のsheafとtoposは通常と同様に定義する.

\begin{Remark}
    歴史的にはsmooth morphismをunderlying categoryにとり
    etale topologyを備え付けるsite, lisse-etale siteが使われている.
    これはetale cohomologyの点で有利だが,
    artin stackの射から誘導されるfunctor :: $f^*, f^{-1}, f_*$などが
    exactで無いといった不満点が有る.
    このせいでsheaves on schemeの時のアナロジーも働かなくなる.
    定義も少々面倒なので,我々はbig etale siteを使う.
\end{Remark}

\begin{Def}
    site :: $\mathcal{S}$上のsheaf :: $\shF$について,
    set of global sectionsを以下で定める.
    \[ \Gamma(\mathcal{S}, \shF)=\Hom_{\Sh(\mathcal{S})}(*, \shF). \]
    ただし$*$はcategory of sheaves of sets :: $\Sh(\mathcal{S})$の
    terminal objectである.
\end{Def}

scheme :: $U$について,$\Gamma(U, \shF)$と$\shF(U)$は異なるものであることに注意せよ.

\begin{Remark}
    category of sheaves of setsのterminal objectは,
    単元集合のconstact sheafである.
    したがってterminal objectとして
    特に単元集合$\{ \ast \}$のconstant sheafをとると,
    $s \in \Gamma(\shF)$は
    \[ \mathcal{S} \ni U \mapsto s_U( \ast ) \in \shF(U) \]
    という対応を成す.
    $\mathcal{S}$はscheme上のsiteであれば,
    $\Gamma(\shF)$の元が自然な方法でglobal sectionに一対一対応する.
\end{Remark}

\begin{Def}[\cite{SP} 06TU]
    $\fibX$のstructure sheaf :: $\shO_{\fibX}$を次で定める.
    \[ \ET(\fibX) \ni U \mapsto \shO_{U}(U). \]
    $\shO_U$はscheme :: $U$のstructure sheafである.
\end{Def}

$\shO_{\fibX}$が確かにsheafであることは\cite{SP} 03DTで証明されている.

\begin{Def}[ $u^{p}, {}_p u$ in \cite{SP} 00VC, 00XF ]
    artin stack over a scheme $S$ :: $\fibX, \fibY$の
    間の射 :: $f \colon \fibX \to \fibY$を考える.
    この$f$からtoposの間の射 :: $\fibX_{\ET} \to \fibY_{\ET}$が誘導される.

    まずsheaf :: $\shG \in \fibY_{\ET}$について,
    \[ (f^{-1} \shG)(\ (U, u)\ )=\shG(\ (U, f \circ u)\ ) \mwhere (U, u) \in \ET(\fibX) \]
    で$f^{-1}\shG \in \fibX_{\ET}$を定める.
    同じく,sheaf :: $\shF \in \fibX_{\ET}$について,
    \[
        (f_*\shF)(\ (V, v)\ )=\Lim \left( I_{(V,v)}^{\mathrm{op}} \xto{\pr_2} \Sch/V \xto{\shF} \Sets \right)
        \mwhere (V, v) \in \ET(\fibY)
    \]
    で$f_*\shF \in \fibY_{\ET}$を定める.
    
    ここで$I_{(V,v)}$は次の圏である.
    \begin{description}
        \item[\textbf{ Objects: }] 
        以下の可換図式を満たす射の組$(U \to \fibX, U \to V)$:
        \[
        \begin{tikzcd}[sep=15pt]
            U \ar[r]\ar[d]& V \ar[d, "v"]\\
            \fibX \ar[r, "f"']& \fibY
        \end{tikzcd}
        \]

        \item[\textbf{ Arrows: }] 
        射$(U \to \fibX, U \to V) \to (U' \to \fibX, U' \to V)$は,
        以下を可換にする射$U \to U'$である.
        \[
        \begin{tikzcd}[sep=15pt]
            U \ar[rd]\ar[d]& {} \\
            U' \ar[r]\ar[d]& V \ar[d, "v"]\\
            \fibX \ar[r, "f"']& \fibY
        \end{tikzcd}
        \]
    \end{description}
    $(f_*\shG)(\ (V, v)\ )$の定義に有る$I_{(V,v)}^{\mathrm{op}} \xto{\pr_2} \Sch/V$は
    \[ (U \to \fibX, U \to V) \mapsto [U \to V] \in \Sch/V \]
    で与えられる.
\end{Def}

\subsection{Propositions}
\begin{Lemma}[\cite{SP} 06NW]
    artin stack :: $\fibX, \fibY$と$\shF \in \ET(\fibX), \shG \in \ET(\fibY)$をとる.
    任意の射$f \colon \fibX \to \fibY$について
    $f^{-1}\shF, f_*\shG$は確かにsheafである.
\end{Lemma}

\begin{Prop}[\cite{SP} 00XF]
    $f^{-1}$は$f_*$のleft adjoint functorである.
\end{Prop}
%% {{{
\begin{proof}
    $(f \circ) \colon (\Sch/\fibX) \to (\Sch/\fibY)$を
    $f \colon \fibX \to \fibY$の合成で得られる関手とする.
    すると関手$f^{-1}$は関手(sheaf) $(\Sch/\fibY) \to \Sets$と$(f \circ)$の合成としてかける.

    これを用いると上の定義は次のように変形できる.
    \[
        (f_*\shF)(\ (V,v)\ )
        =\Lim\Big(  ((f \circ) \downarrow (V,v))_{\et}^{\mathrm{op}} \xto{\pi_1} (\Sch/\fibY) \xto{\shF} \Sets \Big)
    \]
    $((f \circ) \downarrow (V,v))_{\et}^{\mathrm{op}}$は
    圏$(f \circ) \downarrow (V,v)$の双対圏にetale Grothendieck topologyを与えてできるsiteである.
    また$\pi_1 \colon [(f \circ)(\ (U,u)\ ) \to (V,v)] \mapsto (U,u)$.
    
    右辺は各点右Kan拡張$(\Ran_{(f \circ)} \shF)(\ (V,v)\ )$なので,
    Kan拡張の一般論により随伴性が分かる.
\end{proof}
%% }}}

\begin{Prop}[\cite{SP} 06WS]
    artin stack over a scheme $S$ :: $\fibX, \fibY$を考える.
    射 :: $f \colon \fibX \to \fibY$とsheaf :: $\shF \in \fibX_{\ET}$について,
    \[ (f_*\shF)(\ (V, v)\ )=\Gamma( \ET(V \times_{y, \fibY, f} \fibX), \pr_2^{-1} \shF  ). \]
    ただし$\pr_2 \colon V \times_{y, \fibY, f} \fibX \to \fibX$は射影である.
\end{Prop}
%% {{{
\begin{proof}
    一般的な次の命題を用いる.証明は省略する.
    \begin{Lemma}[\url{https://ncatlab.org/nlab/show/limit\#limit_of_a_setvalued_functor}]
        site :: $\mathcal{S}$上のset-value sheaf :: $\shF$を考える.
        この時,
        \[ \Lim \big( \mathcal{S}^{\mathrm{op}} \xto{\shF} \Sets \big)=\Gamma( \mathcal{S}, \shF). \]
    \end{Lemma}
    
    今,一つ前の命題と合わせて
    \[
        (f_*\shF)(\ (V, v)\ )
        =\Lim\Big( ((f \circ) \downarrow (V,v))_{\et}^{\mathrm{op}} \xto{\shF \circ \pi_1} \Sets \Big)
        =\Gamma( ((f \circ) \downarrow (V,v))_{\et}, \shF \circ \pi_1).
    \]
    なので$((f \circ) \downarrow (V,v))_{\et}^{\mathrm{op}}=\ET(V \times_{y, \fibY, f} \fibX)$と
    $\shF \circ \pi_1=\pr_2^{-1}\shF$を確かめれば良い.

    \paragraph{$((f \circ) \downarrow (V,v))_{\et}^{\mathrm{op}}=\ET(V \times_{y, \fibY, f} \fibX)$.}
    $\ET$や${}^{\mathrm{op}}$の部分は単にsiteとして必要な部分なので,
    \[ (f \circ) \downarrow (V,v)=\Sch/(V \times_{y, \fibY, f} \fibX) \]
    を示せば十分.
    $2$-Yoneda embeddingを用いて証明する.

    対象から見ていく.
    図式
    \[
        \begin{tikzcd}[sep=15pt]
            U \ar[r]\ar[d]& V \ar[d, "v"]\\
            \fibX \ar[r, "f"']& \fibY
        \end{tikzcd}
    \]
    を図式$(*)$と呼ぶことにする.
    圏$(f \circ) \downarrow (V,v)$の対象は
    \begin{itemize}
        \item 射$U \to \fibX \xto{f} \fibY$と,
        \item 図式$(*)$を$2$-可換にする射$U \to V$
    \end{itemize}
    の組である.
    一方,$V \times_{\fibY} \fibX$の対象$(U, x, y, \alpha)$の対象は
    $2$-Yoneda embeddingによって次のように対応する:
    \begin{itemize}
        \item $U \in \Sch/S$がそのままscheme :: $U$に,
        \item $x \in (\Sch/V)(U)$が射$U \to V$に,
        \item $y \in \fibX(U)$が射$U \to \fibX$に,
        \item $\alpha \colon v(x) \isomap f(y)$の存在が図式$(*)$の$2$-可換性に対応する.
    \end{itemize}

    射についても見ていく.
    圏$(f \circ) \downarrow (V,v)$の
    射$(U \to \fibX \to \fibY, U \to V) \to (U' \to \fibX \to \fibY, U' \to V)$は,
    以下(の特に$U$を頂点とする二つの三角形)を$1$-可換にする射$U \to U'$である.
    \[
    \begin{tikzcd}[sep=15pt]
        U \ar[rrd, bend left=30]\ar[rd]\ar[rdd, bend right=30]& {} & {} \\
        {} & U' \ar[r]\ar[d]& V \ar[d, "v"]\\
        {} & \fibX \ar[r, "f"']& \fibY
    \end{tikzcd}
    \]
    $V \times_{\fibY} \fibX$の対象の射
    \[ (U \to U', \phi_{V} \colon x \to x', \phi_{\fibX} \colon y \to y') \]
    は,それぞれ
    \begin{itemize}
        \item $U \to U'$はそのまま$U \to U'$に,
        \item $\phi_{V} \colon x \to x'$は$U, U', V$の可換な三角形に,
        \item $\phi_{\fibX} \colon y \to y'$は$U, U', V$の可換な三角形に
    \end{itemize}
    対応する.
    これらの射が満たす条件は,
    二つの三角形の可換性と右下の四角形の$2$-可換性とが整合的であることを意味している.

    以上より,所望の圏同値が得られた.

    \paragraph{$\shF \circ \pi_1=\pr_2^{-1}\shF$.}
    $\pi_1 \colon (f \circ) \downarrow (V,v) \to (\Sch/\fibX), \pr_2 \colon V \times_{\fibY} \fibX \to \fibX$は
    それぞれ次のように定義されている.
    \[
        \pi_1 \colon [(f \circ)(\ (U,u)\ ) \to (V,v)] \mapsto (U,u \colon U \to \fibX), \quad
        \pr_2 \colon (U, x, y, \alpha) \mapsto y \colon U \to \fibX
    \]
    よって$\shF \circ \pi_1=\shF \circ \pr_2=\pr_2^{-1}\shF$.
    な,$\shF \circ \pi_1$は関手の合成,
    $\pr_2^{-1}\shF$は関手$\pr_2^{-1}$による像であるが,結局同じものである.
\end{proof}
%% }}}

\begin{Remark}
    lisse-etale siteを採用する場合は,
    $f^{-1}$はcolimitとして定義される(\cite{LMB00} section 12.5).
    これは \cite{SP} 00XFにおける$u_p$であるが,
    $f_*$とadjointではない($f^{-1} \dashv f_*$は上で述べた).
\end{Remark}

\bibliographystyle{jplain}
\bibliography{../reference}
\end{document}
