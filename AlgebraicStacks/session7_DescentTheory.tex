\documentclass[a4paper, dvipdfmx]{jsarticle}
\usepackage{macros}
\begin{document}
\title{ゼミノート \#7 \\ Descent Theory}
\author{七条彰紀}
\maketitle

\newcommand{\QCoh}{\mathbf{QCoh}}
\newcommand{\vsim}{\text{\rotatebox{-90}{$\sim$}}}

\section{Motivation}
    (TODO)

\section{Definition}
\begin{Def}
    関手$\epsilon_{\covU} \colon \shF(U) \to \shF(\covU)$を用いて以下のように定義する.
    \begin{enumerate}[label=(\roman*)]
        \item
            $\epsilon_{\covU}$ :: equivalenceとなる$\covU$をof effective descent for $\shF$と呼ぶ.
        \item
            $\epsilon_{\covU}$の像と同型である$\shF(\covU)$の対象を,effective dataという.
    \end{enumerate}
\end{Def}

\section{Criterion for fpqc Stacks}
\begin{Lemma}[\cite{ASS} Lemma 4.25]
    $S$ :: scheme,
    $\shF \to (\Sch/S)$ :: fibrationとする.
    以下が成り立つとする.
    \begin{enumerate}[label=(\alph*)]
        \item $\shF$はZariski topologyでのstackである.
        \item
            任意のflat surjective morphism of affine $S$-scheme :: $V \to U$について,\mnewline
            $\epsilon_{\{V \to U\}} \colon \shF(U) \to \shF(\{V \to U\})$は圏同値.
    \end{enumerate}
    この時,$\shF$はfpqc topologyでのstackである.
\end{Lemma}

証明のために段階を踏む.
\subsection{Step 1 / \tp{$\epsilon_{\covU}$} :: faithfull.}
    (TODO: いらないかも)

\subsection{Step 2 / single morphism coverの場合に帰着させる.}
    次を示す.
    \begin{Lemma}[\cite{NoteGroTop} p.87]
        $\covU=\{\phi_i \colon V_i \to U\}, V'=\bigsqcup V_i$とする.
        さらに,$f \colon V' \to U$を$\covU$から誘導される射とする.

        このとき,圏同値 $E \colon \shF(U' \to U) \to \shF(\covU)$が存在し,
        合成
        \[\xymatrix{
            \shF(U) \ar[r]^-{\epsilon_{\{f\}}}& \shF(V' \to U) \ar[r]^-{E}& \shF(\covU)
        }\]
        が関手$\epsilon_{\covU} \colon \shF(U) \to \shF(\covU)$と同型と成る.
    \end{Lemma}

    \begin{proof}
        まず関手$E$を構成し,
        その後$E$が圏同値であること,補題の性質を満たすことを確認する.
    \end{proof}

    \begin{Remark}
        ここで,各$\phi_i$がquasi-compact(特にfpqc)であったとしても,
        誘導される射$f \colon V' \to U$が必ずしもquasi-compactでないことに注意する.
        例えば$\{\Spec k[x_i] \to \bigsqcup_i \Spec k[x_i]\}_{i \in \N}$を考えれば分かる.

        以上のことに注意すると,我々は次のことを証明することに成る:
        \begin{Lemma}
            条件(a), (b)が成り立つならば,
            以下の条件$(*)$を満たす任意のflat surjective morphism :: $f \colon V \to U$について,
            $\epsilon_{\{f\}} \colon \shF(U) \to \shF(f \colon V \to U)$ :: equivalence.
            \begin{description}
                \item[$(*)$]
                    affine Zariski cover :: $U=\bigcup_i U_i$と,
                    各$i$について$f^{-1}(U_i)$のaffine Zariski cover :: $f^{-1}(U_i)=\bigcup_i V_{ij}$が存在し,
                    $V_{ij}$ :: quasi-compactかつ$f(V_{ij})=U_i$となる.
            \end{description}
        \end{Lemma}
        条件$(*)$は$U, V$ :: locally noetherianであるような
        任意のfppf morphismについて成立する(\cite{ASS} Cor1.1.6).
    \end{Remark}

    \begin{Remark}
        以下,$\shF$ :: split fibered categoryとする.
        session 4.5 定理1.2より,このように仮定しても一般性を失わない.
    \end{Remark}

\subsection{Step 3 / affine schemeへのquasi-compact morphismの場合.}
    $f \colon V \to U$を$U$ :: affineであるquasi-compact morphismとする.
    $\{ V_i \}_i$を$V$のopen affine coverとし,$V'=\bigsqcup_i V_i$とおく.
    $V'$ :: affineなので,
    仮定(b)から圏同値$\shF(U) \simeq \shF(V' \to U)$が存在する.

    以下の図式$(1)$を考える.
    $\leftrightarrow$は圏同値を意味する.
    \[
    \begin{tikzcd}[row sep=1cm]
        \shF(U) \ar[r, "\epsilon_{f}"]\ar[d, leftrightarrow, "\epsilon_{\{V_i \to U\}}"']&
        \shF(f \colon V \to U) \ar[ld, "E"]\\
        \shF(\{V_i \to U\}) & \\
        \shF(V' \to U) \ar[u, leftrightarrow]
        \ar[uu, leftrightarrow, shift left=2mm, bend left=90, rounded corners, ]
    \end{tikzcd}
    \eqno{(1)}
    \]
    ここで関手$\epsilon_{f}, E$は次で与えられる.
    ただし$\pr_k \colon V \times_U V \to V$は第$k$成分への射影である.
    \begin{defmap}
        \epsilon_{f}\colon & \shF(U)& \to& \shF(f \colon V \to U) \\
        \mathbf{Objects:}& \xi& \mapsto& (f^*\xi, \sigma)\\
        \mathbf{Arrows:}& \alpha& \mapsto& f^*\alpha
    \end{defmap}
    ここで$\sigma \colon \pr_2^*f^*\xi \to \pr_1^*f^*\xi \in \Arr(\shF(V \times_U V))$は
    $\pr_2^*f^*\xi,\pr_1^*f^*\xi$が
    いずれも$f \circ \pr_2=f \circ \pr_1$による$\xi$のpullbackであることから得られる同型射である.
    \begin{defmap}
        E\colon & \shF(f \colon V \to U)& \to& \shF(\{V_i \to U\}) \\
        \mathbf{Objects:}& (\eta, \sigma)& \mapsto& (\{\eta|_{V_i}\}_i, \{(\gamma_{ij})^*\sigma\}) \\
        \mathbf{Arrows:}& \beta& \mapsto& \{\beta|_{V_i}\}
    \end{defmap}
    ここで$\gamma_{ij}$は以下の可換図式のように,
    fiber productの一意性から得られる射である.
    \begin{center}
        \begin{tikzcd}[row sep=1.2cm, column sep=0.7cm]
            V_i \times_U V_j \ar[rr, "\pr_1"]\ar[dd, "\pr_2"']\ar[rd, "\gamma_{ij}"]& & V_i \ar[d, hookrightarrow]\\
                                       & V \times_U V \ar[r, "\pr_1"]\ar[d, "\pr_2"']& V \ar[d, "f"]\\
        V_j \ar[r, hookrightarrow]& V \ar[r, "f"']& U
    \end{tikzcd}
    \end{center}

    この図式の可換性から,
    関手の同型$E \circ \epsilon_{f} \iso \epsilon_{\{V_i \to U\}}$が得られる
    ($\shF$ :: split fibered categoryを利用する).
    (よって上の図式$(1)$は可換である.)
    したがって$\epsilon_{f}$のpsuedo-inverse functor ::
    $(\epsilon_{\{V_i \to U\}})^{-1} \circ E$が得られた.

\subsection{Step 4 / 条件\tp{$(*)$}{(*)}を満たすaffine schemeへの射の場合.}
    (\cite{NoteGroTop} p.88)
    仮定$(*)$より,
    Zariski cover :: $\{\iota_i \colon V_i \to V\}$が存在し,$V_i$ :: quasi-compact.
    \begin{Remark}
        前段の議論のうち,図式$(1)$の$\shF(U) \to \shF(V' \to U)$が圏同値でない.
        なので新しい議論が必要である.
    \end{Remark}

    \[
    (2)
    \begin{tikzcd}[row sep=10mm]
        \shF(U) \ar[r]\ar[d]& \shF(f \colon V \to U) \ar[ld]\\
        \shF(V_i \to U)
    \end{tikzcd}
    \qquad
    (3)
    \begin{tikzcd}[row sep=10mm]
        \shF(U) \ar[r]\ar[d]& \shF(V_i \cup V_j \to U) \ar[ld]\\
        \shF(V_i \to U)
    \end{tikzcd}
    \]

\subsection{Step 5 / 一般の場合.}
    条件$(*)$を満たす任意の射$f \colon V \to U$をとり,
    affine Zarisiki cover :: $\{U_i \to U\}$をとる.
    さらに$V_i:=f^{-1}(U_i)$とおき,$\phi_i=f|_{V_i}$とおく.

    今,
    \[ \Phi_i=\epsilon_{V_i \to U_i} \colon \shF(U_i) \to \shF(V_i \to U_i) \]と置く.
    同様に$\Phi_{ij}=\epsilon_{V_{ij} \to U_{ij}}, \Phi_{ijk}=\epsilon_{V_{ijk} \to U_{ijk}}$と置く.
    この時,以下は厳密な可換図式である.
    \[
    \begin{tikzcd}
        \shF(U_i) \ar[r, "\res"]\ar[d, "\Phi_{i}"']&
            \shF(U_{ij}) \ar[d, "\Phi_{ij}"]\ar[r, "\res"] & \shF(U_{ijk}) \ar[d, "\Phi_{ijk}"]\\
        \shF(V_i \to U_i) \ar[r, "\res"']& \shF(V_{ij} \to U_{ij}) \ar[r, "\res"']& \shF(V_{ijk} \to U_{ijk})
    \end{tikzcd}
    \eqno{(4)}
    \]
    ここで,各$\Phi_{*}$はいずれもStep 4から圏同値である.

    次の関手を考える.
    \begin{defmap}
        P_{i}\colon & \shF(f \colon V \to U)& \to& \shF(V_i \to U_i) \\
        \mathbf{Objects}& (\eta, \sigma)& \mapsto& (\eta|_{V_i}, (\gamma_{ii})^*\sigma) \\
        \mathbf{Arrows}& \alpha& \mapsto& \alpha|_{V_i} \\
    \end{defmap}
    同様に$P_{ij} \colon \shF(f) \to \shF(V_{ij} \to U_{ij})$を定義する.
    するとstep 4の結果から$\shF(U_i) \simeq \shF(V_i \to U_i)$なので,
    $\xi_i \in \shF(U_i)$と同型
    \[ \alpha_i \colon \Phi_i(\xi_i) \isomap P_i((\eta, \sigma)) \]
    が得られる.
    上の図式(4)が可換であることから,
    \[
        \alpha_i|_{V_{ij}} \colon
        \Phi_{ij}(\xi_i|_{V_{ij}})=(\Phi_i(\xi_i))|_{V_{ij}} \isomap P_{ij}((\eta, \sigma))
    \]
    すると,
    \[ \alpha_i^{-1}\alpha_j \colon \Phi_{ij}(\xi_j|_{V_{ij}}) \to \Phi_{ij}(\xi_i|_{V_{ij}}) \]
    $\Phi_{ij}$ :: equivalenceなので,
    この同型射の逆像として$\sigma_{ij} \colon \xi_j|_{V_{ij}} \to \xi_i|_{V_{ij}}$が得られる.
    
    以上で得られる$(\{\xi_i\}, \{\sigma_{ij}\})$はcocycle conditionを満たすため,
    $\shF(\{U_i \to U\})$のobjectである.
    これは$\shF(U)$と圏同値なので,$\xi$が得られる.

\section{Application : \tp{$\QCoh/S \to \Sch/S$}{QCoh/S -> Sch/S} is a fpqc stack.}

\bibliographystyle{jplain}
\bibliography{reference}
\end{document}
