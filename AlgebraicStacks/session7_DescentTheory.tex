\documentclass[a4paper, dvipdfmx]{jsarticle}
\usepackage{macros}
\begin{document}
\title{ゼミノート \#7 \\ Descent Theory}
\author{七条彰紀}
\maketitle

\newcommand{\QCoh}{\mathbf{QCoh}}
\newcommand{\Mod}{\mathbf{Mod}}
\newcommand{\vsim}{\text{\rotatebox{-90}{$\sim$}}}

\section{Motivation}
    (TODO)

\section{Definition}
\begin{Def}
    関手$\epsilon_{\covU} \colon \shF(U) \to \shF(\covU)$を用いて以下のように定義する.
    \begin{enumerate}[label=(\roman*)]
        \item
            $\epsilon_{\covU}$ :: equivalenceとなる$\covU$をof effective descent for $\shF$と呼ぶ.
        \item
            $\epsilon_{\covU}$の像と同型である$\shF(\covU)$の対象を,effective dataという.
    \end{enumerate}
\end{Def}

\section{Criterion for fpqc Stacks}
\begin{Thm}[\cite{ASS} Lemma 4.25]\label{thm:fpqccriterion}
    $S$ :: scheme,
    $\shF \to (\Sch/S)$ :: fibrationとする.
    以下が成り立つとする.
    \begin{enumerate}[label=(\alph*)]
        \item $\shF$はZariski topologyでのstackである.
        \item
            任意のflat surjective morphism of affine $S$-scheme :: $V \to U$について,\mnewline
            $\epsilon_{\{V \to U\}} \colon \shF(U) \to \shF(\{V \to U\})$は圏同値.
    \end{enumerate}
    この時,$\shF$はfpqc topologyでのstackである.
\end{Thm}

\begin{Remark}
    ``flat"という条件は以下の証明では利用されない.
\end{Remark}

\subsection{Step 1 / 準備}
以前示した命題から,
$\shF$ :: split fibered categoryと仮定しても一般性を失わないので,
以下そのように仮定する.

\begin{Lemma}\label{lemma:key}
    $\cat{C}$をsiteとし,
    $\shF \to \cat{C}$を\underline{split} fibrationとする.
    さらに$U \in \cat{C}$,$\covU=\{\phi_i \colon U_i \to U\} \in \Cov(U)$と
    $\covU$の細分
    \footnote
    {
        細分の定義を確認しておく:
        任意の$\covV$の元$V_{ij} \to U$に対して$\covU$の元$U_i \to U$が存在し,
        $V_{ij} \to U$が$U_i \to U$を通して$V_{ij} \to U_i \to U$と分解する.
        特に射$V_{ij} \to U_{i}$が存在する.
    }
    $\covV=\{\psi_{ij} \colon V_{ij} \to U\}$をとる.

    この時,関手$R_{\covU}^{\covV} \colon \shF(\covU) \to \shF(\covV)$が存在し,
    以下は厳密な可換図式である。
    \[
        \begin{tikzcd}
        \shF(U) \ar[r, "\epsilon_{\covU}"] \ar[d, "\epsilon_{\covV}"']&
        \shF(\covU) \ar[ld, "R_{\covU}^{\covV}"] \\
        \shF(\covV)
    \end{tikzcd}
    \]
\end{Lemma}
%% {{{
\begin{proof}
    \step{関手$\shF(\covU) \to \shF(\covV)$の構成.}
    細分の定義から,各$i,k$について以下が可換に成る射$\iota_{ik} \colon V_{ik} \to U_{i}$が存在する.
    \[
    \begin{tikzcd}
        V_{ik} \ar[r, "\iota_{ik}"'] \ar[rr, bend left, "\psi_{ik}"]& U_{i} \ar[r, "\phi_i"']& U
    \end{tikzcd}
    \]
    この射$\iota_{ik}$を用いて,
    関手$R_{\covU}^{\covV}$を次のように定義する。
    \begin{defmap}
        R_{\covU}^{\covV} \colon& \shF(\covU)& \to& \shF(\covV) \\
        \mathbf{Objects}& (\{\eta_i\}, \{\sigma_{ij}\})& \mapsto&
            \left(\{(\iota_{ik})^*\eta_i\}, \left\{\left(\iota_{ik}^{jl}\right)^*\sigma_{ij}\right\}\right) \\
        \mathbf{Arrows}& \{\alpha_i\}& \mapsto& \{(\iota_{ik})^*\alpha_i\}
    \end{defmap}
    ここで$\iota_{ik}^{jl}$は,
    以下の可換図式のようにfiber productの一意性から得られる射である.
    \begin{center}
    \begin{tikzcd}[row sep=1.4cm, column sep=0.7cm]
        V_{ik} \times_U V_{jl}
        \ar[rr, "\pr_1"]\ar[dd, "\pr_2"']\ar[rd, red, "\iota_{ik}^{jl}"]&&
        V_{ik} \ar[d, "\iota_{ik}"'] \ar[dd, bend left, "\psi_{ik}"]\\
        {}& U_{i} \times_U U_{j} \ar[r, "\pr_1"]\ar[d, "\pr_2"']& U_{i} \ar[d, "\phi_i"']\\
        V_{jl} \ar[r, "\iota_{jl}"]\ar[rr, bend right, "\psi_{jl}"']& U_{j} \ar[r, "\phi_j"]& U
    \end{tikzcd}
    \end{center}
    $\{\sigma_{ij}\}$がcocycle conditionを満たすので,
    $\left\{\left(\iota_{ik}^{jl}\right)^*\sigma_{ij}\right\}$もcocycle conditionを満たす
    \footnote
    {
        証明はfiber productの普遍性から得られる
        射$V_{il} \times V_{jm} \times V_{kn} \to U_{i} \times U_{j} \times U_{k}$を用いれば良い.
    }.
    同様に$\{(\iota_{ik})^*\alpha_i\}$が$\shF(\covV)$の射であることも確認できる.

    \step{対象について$R_{\covU}^{\covV}\epsilon_{\covU}=\epsilon_{\covV}.$}
    $R_{\covU}^{\covV}\epsilon_{\covU}$を計算する.
    まず$\xi \in \shF(U)$をとり,$R_{\covU}^{\covV}\epsilon_{\covU}(\xi)$を計算しよう.
    \begin{align*}
        R_{\covU}^{\covV}\epsilon_{\covU}(\xi)
        =&  R_{\covU}^{\covV}\Big( (\{\phi_i^*\xi\}, \{\sigma_{ij}\}) \Big) \\
        =&  \left(\{(\iota_{ik})^*\phi_i^*\xi\}, \left\{\left(\iota_{ik}^{jl}\right)^*\sigma_{ij} \right\}\right) \\
        =&  \left(\{(\psi_{ik})^*\xi\}, \left\{\left(\iota_{ik}^{jl}\right)^*\sigma_{ij} \right\}\right)
    \end{align*}
    今,$\shF$ :: split fibered categoryとしているので,
    \[ \pr_2^*\phi_j^* \xi=(\phi_j \circ \pr_2)^*\xi=(\phi_i \circ \pr_1)^*\xi=\pr_1^*\phi_i^* \xi. \]
    $\sigma_{ij}$はfiber productの普遍性から得られる
    $\pr_2^*\phi_j^* \xi$から$\pr_1^*\phi_i^* \xi$への同型であるから,
    $\sigma_{ij}=\id$.
    このことと$\shF$ :: splitから$\left(\iota_{ik}^{jl}\right)^*\sigma_{ij}=\id$も分かる.
    まとめると,
    $R_{\covU}^{\covV}\epsilon_{\covU}(\xi)=\left(\{(\psi_{ik})^*\xi\}, \{\id[(\psi_{ik})^*\xi]\}\right)$.

    一方,
    \[
        \left(\iota_{ik}^{jl}\right)^*\pr_2^*\phi_j^* \xi
        =(\psi_{jl} \circ \pr_2)^*\xi
        =\pr_2^*(\psi_{jl})^*\xi
        =\pr_1^*(\psi_{ik})^*\xi
        =(\psi_{ik} \circ \pr_1)^*\xi
        =\left(\iota_{ik}^{jl}\right)^*\pr_1^*\phi_i^* \xi.
    \]
    なので,fiber productの普遍性から得られる
    $\pr_2^*(\psi_{jl})^*\xi$から$\pr_1^*(\psi_{ik})^*\xi$への同型は$\id$である.
    したがって
    $\epsilon_{\covV}(\xi)=\left(\{(\psi_{ik})^*\xi\}, \{\id[(\psi_{ik})^*\xi]\}\right)$となり,
    $R_{\covU}^{\covV}\epsilon_{\covU}(\xi)=\epsilon_{\covV}(\xi)$.
    
    \step{射について$R_{\covU}^{\covV}\epsilon_{\covU}=\epsilon_{\covV}.$}
    $\shF(U)$の射$\alpha \colon \xi_1 \to \xi_2$をとる.
    すると$\shF$ :: splitなので
    \[
        R_{\covU}^{\covV}\epsilon_{\covU}(\alpha)
        =\{(\iota_{ik})^*\phi_i^*\alpha\}=\{(\phi_i \circ \iota_{ik})^*\alpha\}=\{\psi_{ik}^*\alpha\}
        =\epsilon_{\covV}(\alpha).
    \]
\end{proof}
%% }}}
\begin{Remark}
    $\shF$ :: splitを仮定しない場合,
    可換図式が厳密であることを主張できないのは明白であろう.
    実はさらに,$2$圏の意味でも図式が可換にならない.
    なぜなら自然変換$\pr_1^* \phi_i^* \to (\phi_i \circ \pr_1)^*$などが
    存在する保証がないからである.
    同型$\pr_1^* \phi_i^*\xi \to (\phi_i \circ \pr_1)^*\xi$は$\xi$毎に存在が保証されているのみで,
    それらが自然であることは保証されない.
\end{Remark}

%\begin{Lemma}
%    $\cat{C}, \shF$等を上の補題のようにする.
%    $U \in \cat{C}$をとり,$\covU \in \Cov(U)$とする.
%    さらに$\covV$を$\covU$の細分,$\covW$を$\covV$の細分とする
%    (したがって$\covW$は$\covU$の細分である).
%    この時,次が成立する.
%    \[ R_{\covV}^{\covW} \circ R_{\covU}^{\covV}=R_{\covU}^{\covW}. \]
%\end{Lemma}
%\begin{proof}
%    明らかなので証明は略す.
%\end{proof}

\subsection{Step 2 / single morphism coverの場合に帰着させる.}
    \begin{Cor}[\cite{NoteGroTop} p.87]
        $\cat{C}, \shF$等を補題(\ref{lemma:key})の様にとる.
        $\covU=\{\phi_i \colon V_i \to U\}, V'=\bigsqcup V_i$とする.
        さらに,$f \colon V' \to U$を$\covU$から誘導される射とする.

        このとき,圏同値 $R_{V' \to U}^{\covU} \colon \shF(V' \to U) \to \shF(\covU)$が存在し,
        合成
        \[\xymatrix{
            \shF(U) \ar[r]^-{\epsilon_{\{f\}}}& \shF(V' \to U) \ar[r]^-{E}& \shF(\covU)
        }\]
        が関手$\epsilon_{\covU} \colon \shF(U) \to \shF(\covU)$と厳密に一致する.
    \end{Cor}
    \begin{proof}
        $\covU$は$\{U' \to U\} \in \Cov(U)$の細分であるから,
        補題(\ref{lemma:key})から明らか.
    \end{proof}

    \begin{Remark}
        ここで,各$\phi_i$がquasi-compact(特にfpqc)であったとしても,
        誘導される射$f \colon V' \to U$が必ずしもquasi-compactでないことに注意する.
        例えば$\{\Spec k[x_i] \to \bigsqcup_i \Spec k[x_i]\}_{i \in \N}$を考えれば分かる.

        以上のことに注意すると,我々は次のことを証明することに成る:
        \begin{Claim}
            条件(a), (b)が成り立つならば,
            以下の条件$(*)$を満たす任意のflat surjective morphism :: $f \colon V \to U$について,
            $\epsilon_{\{f\}} \colon \shF(U) \to \shF(f \colon V \to U)$ :: equivalence.
            \begin{description}
                \item[$(*)$]
                    affine Zariski cover :: $U=\bigcup_i U_i$と,
                    各$i$について$f^{-1}(U_i)$のaffine Zariski cover :: $f^{-1}(U_i)=\bigcup_i V_{ij}$が存在し,
                    $V_{ij}$ :: quasi-compactかつ$f(V_{ij})=U_i$となる.
            \end{description}
        \end{Claim}
        条件$(*)$は$U, V$ :: locally noetherianであるような
        任意のfppf morphismについて成立する(\cite{ASS} Cor1.1.6).
    \end{Remark}

    \begin{Remark}
        以下,$\shF$ :: split fibered categoryとする.
        session 4.5 定理1.2より,このように仮定しても一般性を失わない.
    \end{Remark}

\subsection{Step 3 / affine schemeへのquasi-compact morphismの場合.}
    $f \colon V \to U$を$U$ :: affineであるquasi-compact morphismとする.
    $\{ V_i \}_i$を$V$のopen affine coverとし,$V'=\bigsqcup_i V_i$とおく.
    $V'$ :: affineなので,
    仮定(b)から圏同値$\shF(U) \simeq \shF(V' \to U)$が存在する.

    以下の図式$(1)$を考える.
    $\leftrightarrow$は圏同値を意味する.
    \[
    \begin{tikzcd}[row sep=1cm]
        \shF(U) \ar[r, "\epsilon_{f}"]\ar[d, leftrightarrow, "\epsilon_{\{V_i \to U\}}"']&
        \shF(f \colon V \to U) \ar[ld, "R_{f}^{V_i \to U}"]\\
        \shF(\{V_i \to U\}) & \\
        \shF(V' \to U) \ar[u, leftrightarrow, "R_{V' \to U}^{\{V_i \to U\}}"']
        \ar[uu, leftrightarrow, shift left=2mm, bend left=90, rounded corners, "\epsilon_{V' \to U}"]
    \end{tikzcd}
    \eqno{(1)}
    \]
    ここで関手$\epsilon_{f}$は次で与えられる.
    ただし$\pr_k \colon V \times_U V \to V$は第$k$成分への射影である.
    \begin{defmap}
        \epsilon_{f}\colon & \shF(U)& \to& \shF(f \colon V \to U) \\
        \mathbf{Objects:}& \xi& \mapsto& (f^*\xi, \sigma)\\
        \mathbf{Arrows:}& \alpha& \mapsto& f^*\alpha
    \end{defmap}
    ここで$\sigma \colon \pr_2^*f^*\xi \to \pr_1^*f^*\xi \in \Arr(\shF(V \times_U V))$は,
    恒等射$\id[\pr_2^*f^*\xi]$である.
    これは
    $\pr_2^*f^*\xi,\pr_1^*f^*\xi$が
    いずれも$f \circ \pr_2=f \circ \pr_1$による$\xi$のpullbackであることと,
    $\shF$ :: splitから得られる

    この図式の可換性から,
    関手の同型$(R_{f}^{V_i \to U}) \circ \epsilon_{f}=\epsilon_{\{V_i \to U\}}$が得られる
    ($\shF$ :: split fibered categoryを利用する).
    (よって上の図式$(1)$は可換である.)
    したがって$\epsilon_{f}$のpsuedo-inverse functor ::
    $(\epsilon_{\{V_i \to U\}})^{-1} \circ (R_{f}^{V_i \to U})$が得られた.

\subsection{Step 4 / 条件\tp{$(*)$}{(*)}を満たすaffine schemeへの射の場合.}
    (\cite{NoteGroTop} p.88)
    仮定$(*)$より,
    Zariski cover :: $\{\iota_i \colon V_i \to V\}$が存在し,$V_i$ :: quasi-compact.
    \begin{Remark}
        前段の議論のうち,図式$(1)$の$\shF(U) \to \shF(V' \to U)$が圏同値でない.
        なので新しい議論が必要である.
    \end{Remark}

    補題(\ref{lemma:key})から得られる以下の可換図式を考える.
    \[
    \begin{tikzcd}[row sep=10mm]
        \shF(U) \ar[r, "\epsilon_{f}"]
        \ar[d, leftrightarrow, "\substack{\epsilon_{V_i \to U} \\ \text{\color{red}equivalence}}"']&
        \shF(f \colon V \to U)
        \ar[ld, "\substack{R_{f}^{V_i \to U} \\ \text{\color{red}essentially surjective \& full}}"]\\
        \shF(V_i \to U)
    \end{tikzcd}
    \]
    左にある縦の射はStep 3から圏同値である.
    したがって$\shF(V \to U) \to \shF(V_i \to U)$(定義はおおよそ関手$E$と同様に与えられる)は
    essentially surjectiveかつfullである.
    なのでこの関手が更にfaithfullであることが証明できれば,
    図式の可換性から$\epsilon_{f}$が圏同値であることが証明できる.

    $\shF(V \to U)$の射$\beta, \beta'$が,$\beta|_{V_i}=\beta'|_{V_i}$を満たすとする.
    この時,$\beta=\beta'$を証明すれば良い.
    まず,以下の厳密な可換図式から,
    任意の添字$j$について
    $R_{V_i \cup V_j \to U}^{V_i \to U} \colon \shF(V_i \to U) \to \shF(V_i \cup V_j \to U)$が圏同値だと分かる.
    この関手を略して$R$を書くことにする.
    \[
    \begin{tikzcd}[row sep=10mm]
        \shF(U)
        \ar[r, leftrightarrow, "\epsilon_{V_i \cup V_j \to U}"]
        \ar[d, leftrightarrow, "\epsilon_{V_i \to U}"']& 
        \shF(V_i \cup V_j \to U) \ar[ld, red, "R\,:=R_{V_i \cup V_j \to U}^{V_i \to U}"]\\
        \shF(V_i \to U)
    \end{tikzcd}
    \]
    したがって$R^{-1}$が存在する.
    関手$R$は
    $\shF(V_i \cup V_j \to U)$の射$\beta|_{V_i \cup V_j}$を$\beta|_{V_i}$に一対一に写すのだから,
    $R^{-1}$は$\beta|_{V_i}$を$\beta|_{V_i \cup V_j}$に一対一に写す.

    さて,以下の関手の合成で$\beta, \beta'$を写す.
    \[
    \begin{tikzcd}[column sep=20mm]
        \shF(V \to U) \ar[r, "R_{V \to U}^{V_i \to U}"]&
        \shF(V_i \to U) \ar[r, "R^{-1}"]&
        \shF(V_i \cup V_j \to U) \ar[r, "R_{V_i \cup V_j \to U}^{V_j \to U}"]&
        \shF(V_j \to U) 
    \end{tikzcd}
    \]
    $\beta|_{V_i}=\beta'|_{V_i}$を$R^{-1}$で写して
    $\beta|_{V_i \cup V_j}=\beta'|_{V_i \cup V_j}$が得られる.
    よって,任意の$j$について
    \[ \beta|_{V_j}=(\beta|_{V_i \cup V_j})|_{V_j}=(\beta'|_{V_i \cup V_j})|_{V_j}=\beta'|_{V_j} \]
    が成立する.
    $\shF$ :: Zariski stackなので,$\beta=\beta'$.

\subsection{Step 5 / 一般の場合.}
    条件$(*)$を満たす任意の射$f \colon V \to U$をとり,
    affine Zarisiki cover :: $\{U_i \to U\}$をとる.
    さらに$V_i:=f^{-1}(U_i)$とおき,$\phi_i=f|_{V_i}$とおく.

    今,
    \[ \Phi_i=\epsilon_{V_i \to U_i} \colon \shF(U_i) \to \shF(V_i \to U_i) \]と置く.
    同様に$\Phi_{ij}=\epsilon_{V_{ij} \to U_{ij}}, \Phi_{ijk}=\epsilon_{V_{ijk} \to U_{ijk}}$と置く.
    この時,以下は厳密な可換図式である.
    \[
    \begin{tikzcd}
        \shF(U_i) \ar[r, "\res"]\ar[d, "\Phi_{i}"']&
            \shF(U_{ij}) \ar[d, "\Phi_{ij}"]\ar[r, "\res"] & \shF(U_{ijk}) \ar[d, "\Phi_{ijk}"]\\
        \shF(V_i \to U_i) \ar[r, "\res"']& \shF(V_{ij} \to U_{ij}) \ar[r, "\res"']& \shF(V_{ijk} \to U_{ijk})
    \end{tikzcd}
    \eqno{(4)}
    \]
    ここで,各$\Phi_{*}$はいずれもStep 4から圏同値である.

    次の関手を考える.
    \begin{defmap}
        P_{i}\colon & \shF(f \colon V \to U)& \to& \shF(V_i \to U_i) \\
        \mathbf{Objects}& (\eta, \sigma)& \mapsto& (\eta|_{V_i}, (\gamma_{ii})^*\sigma) \\
        \mathbf{Arrows}& \alpha& \mapsto& \alpha|_{V_i} \\
    \end{defmap}
    同様に$P_{ij} \colon \shF(f) \to \shF(V_{ij} \to U_{ij})$を定義する.
    するとstep 4の結果から$\shF(U_i) \simeq \shF(V_i \to U_i)$なので,
    $\xi_i \in \shF(U_i)$と同型
    \[ \alpha_i \colon \Phi_i(\xi_i) \isomap P_i((\eta, \sigma)) \]
    が得られる.
    上の図式(4)が可換であることから,
    \[
        \alpha_i|_{V_{ij}} \colon
        \Phi_{ij}(\xi_i|_{V_{ij}})=(\Phi_i(\xi_i))|_{V_{ij}} \isomap P_{ij}((\eta, \sigma))
    \]
    すると,
    \[ \alpha_i^{-1}\alpha_j \colon \Phi_{ij}(\xi_j|_{V_{ij}}) \to \Phi_{ij}(\xi_i|_{V_{ij}}) \]
    $\Phi_{ij}$ :: equivalenceなので,
    この同型射の逆像として$\sigma_{ij} \colon \xi_j|_{V_{ij}} \to \xi_i|_{V_{ij}}$が得られる.
    
    以上で得られる$(\{\xi_i\}, \{\sigma_{ij}\})$はcocycle conditionを満たすため(TODO),
    $\shF(\{U_i \to U\})$のobjectである.
    $\shF$ :: Zariski stackなので,これは$\shF(U)$と圏同値.
    よって$\xi$が得られる.
    (TODO: 射についても)

\section{Application : \tp{$\QCoh/S \to \Sch/S$}{QCoh/S -> Sch/S} is a fpqc stack.}

    \begin{Def}
    $S \in \Sch$に対し,圏$\QCoh/S$を以下のように定める.
    \begin{description}
        \item[Objects.] \mnewline
            $\mathrm{Fpqc}(S)$
            \footnote{圏$\Sch/S$にfpqc topologyを備えたもの.}の対象 :: $U$と,
            $U$上のquasi-coherent sheaf (on fpqc topology):: $\shU$の組.
        \item[Arrows.] \mnewline
            射$(U, \shU) \to (V, \shV)$は,
            $\Sch/S$の射$f \colon U \to V$と,
            morphism of sheaves on $V$ :: $f^{\#} \colon \shV \to f_*\shU$の組.
    \end{description}
    この時,$\QCoh/S \to \Sch/S; \ (U, \shU) \mapsto U$はfibrationである.
    \end{Def}

%    \begin{Def}
%        $\phi \colon A \to B$を環準同型とする.
%        $A$-module :: $M$について,
%        \[
%            c_M \colon M \otimes_A B \to B \otimes M;\ m \otimes b \mapsto b \otimes m,\qquad 
%            p_M \colon M \to B \otimes M;\ m \mapsto 1_B \otimes m
%        \]
%        と定義する.

%        この時,$\Mod_{\phi}$を以下のように定義する.
%        \begin{description}
%            \item[Objects.] $A$-module :: $N$と同型$\psi \colon N \otimes B \to B \otimes N$であって,
%            \item[Arrows.]
%        \end{description}
%        さらに,$\Mod_A \to \Mod_{\phi}$を以下のように定義する.
%        \begin{defmap}
%            \epsilon_{\phi}\colon & & \to&  \\
%            \mathbf{Objects}& & \mapsto& \\
%            \mathbf{Arrows}& & \mapsto& 
%        \end{defmap}
%    \end{Def}
    
    $\Mod_A, \Mod_{\phi}, \Mod_A \to \Mod_{\phi}$の定義は
    \cite{NoteGroTop} \S4.2.1を参照せよ.

    $f \colon V \to U$をflat surjective morphism of $S$-schemesとし,
    $\phi \colon A \to B$を$f$に対応するfaithfully flatな環準同型とする.
    この時,$\QCoh(U) \simeq \Mod_A$はよく知られている
    \footnote{この命題は\cite{HarAG} Cor5.5で詳しく述べられている}.
    \begin{Claim}
        $\QCoh(f \colon V \to U) \simeq \Mod_{\phi}$.
    \end{Claim}
    したがって$\epsilon_{f} \colon \QCoh(U) \to \QCoh(f \colon V \to U)$は
    関手$\Mod_A \to \Mod_{\phi}$に対応する.
    この関手は,可換環論によって圏同値であることが証明される.

\bibliographystyle{jplain}
\bibliography{reference}
\end{document}
