\documentclass[a4paper]{jsarticle}
\usepackage{macros}

\newcommand{\HOM}{\operatorname{HOM}}
\newcommand{\inj}{\mathrm{inj}}
\newcommand{\QCoh}{\mathbf{QCoh}}
\newcommand{\Mod}{\mathbf{Mod}}
\newcommand{\vsim}{\text{\rotatebox{-90}{$\sim$}}}

\begin{document}
\title{第4章 \\ Stacks and Descent Theory}
\author{七条彰紀}
\maketitle

\section{The Category of Descent Data.}
    今回のノートで一貫して用いる記号と記法を定める.

    $\cat{C}$ :: site, 
    $\pi \colon \fibF \to \cat{C}$ :: fibered categoryを考える
    \footnote{ ほとんどfiber of $\pi$しか扱わないので,psuedo-functor $\cat{C} \to \Cat$をとっても構わない. }.

    記法を定める.
    $U \in \cat{C}, \covU=\{\phi_i \colon U_i \to U\}_{i \in I} \in \Cov(U)$について,
    \[ U_{ij}:=U_i \times_U U_j, \quad U_{ijk}:=U_i \times_U U_j \times_U U_k \ \ (i,j,k \in I) \]
    と書くことにする.
    また,添字$a,b=i \mor j \mor k$について,
    fiber productからの射影を
    \[ \pr_{a} \colon U_{ij} (\mor U_{ijk}) \to U_{a}, \qquad \pr_{a,b} \colon U_{ijk} \to U_{ab} \]
    とする.
    さらに$\pr_{i} \colon U_{ij} \to U_{i}$によるpullbackを$(-)|_{U_{ij}}$などと書く.

\subsection{Definition}
\begin{Def}[$\fibF(\covU)$, \cite{ASS} 4.2.4, \cite{NoteGroTop} Def4.2]
    圏$\fibF(\covU)$を次のように定める.
    \begin{description}
        \item[Object.] \hfill \vspace{-0.2cm}
            \begin{itemize}
                \item $\xi_i \in \fibF(U_i)$なる対象のclass $\{\xi_i\}_{i \in I}$と,
                \item
                    $\fibF(U_{ij})$中の同型$\sigma_{ij} \colon \xi_j|_{U_{ij}} \to \xi_i|_{U_{ij}}$の
                    class $\{\sigma_{ij}\}_{i,j \in I}$
            \end{itemize}
            の組$(\{\xi_i\}, \{\sigma_{ij}\})$であって,
            以下で述べるcocycle conditionを満たすもの.
            このような組をobject with descent dataと呼ぶ
            \footnote{ 同型のclass $\{\sigma_{ij}\}$がdescent dataと呼ばれる. }.

        \item[Arrow.] \mnewline
            射$\{\alpha_i\} \colon (\{\xi_i\}, \{\sigma_{ij}\}) \to (\{\eta_i\}, \{\tau_{ij}\})$とは,
            $\fibF(U_i)$の射$\alpha_i \colon \xi_i \to \eta_i$のclassであって,\mnewline
            $\sigma_{ij}, \tau_{ij}$と整合的であるもの.
            すなわち,任意の$i, j \in I$について以下の図式が可換であるもの.
            \[\xymatrix{
                \ar[d]_-{\sigma_{ij}} \xi_j|_{U_{ij}} \ar[r]^-{\alpha_j|_{U_{ij}}}&
                \eta_j|_{U_{ij}} \ar[d]^-{\tau_{ij}}\\
                 \xi_i|_{U_{ij}} \ar[r]_-{\alpha_i|_{U_{ij}}}& \eta_i|_{U_{ij}}
            }\]
    \end{description}

    \step{cocycle condition}
    組$(\{\xi_i\}, \{\sigma_{ij}\})$がcocycle conditionを満たすとは,
    任意の$i,j,k \in I$について以下が成り立つということ.
    \[ \sigma_{ik}|_{U_{ijk}}=(\sigma_{ij}|_{U_{ijk}}) \circ (\sigma_{jk}|_{U_{ijk}}). \]
    図式でかけば,圏$\fibF(U_{ijk})$における以下の図式が可換であることと同値.
    \[\xymatrix@R=70pt{
         \xi_k|_{U_{ijk}} \ar[rd]_-{\sigma_{ik}|_{U_{ijk}}} \ar[rr]^-{\sigma_{jk}|_{U_{ijk}}}
            & {}
            & \xi_j|_{U_{ijk}} \ar[ld]^-{\sigma_{ij}|_{U_{ijk}}} \\
        {} & \xi_i|_{U_{ijk}} & {}
    }\]
\end{Def}

\begin{Remark}
    この定義に於いてfiber products :: $U_{ij}, U_{ijk}$を暗黙のうちに選択している.
    たが,どのように選択しても得られる圏は同型に成る.
    $U_{ij}, U_{ijk}$の選択も込めて
    $(\{\xi_i\}, \{\xi_{ij}\}, \{\xi_{ijk}\})$を$\fibF(\covU)$の対象とする
    定義の仕方も有るが,ここでは述べない.
    詳細は\cite{NoteGroTop} Remark 4.3にある.
\end{Remark}

\begin{Def}[\cite{NoteGroTop} p.72] \label{def:epsilon}
    $\xi \in \fibF(U), \covU=\{\phi_i \colon U_i \to U\} \in \Cov(U)$について,
    $\fibF(\covU)$の元を以下のデータに対応させる:
    \begin{itemize}
        \item $\xi_i:=\phi_i^*\xi$のclass $\{\xi_i\}_{i \in I}$.
        \item
            $\xi_i|_{U_{ij}}$と$\xi_j|_{U_{ij}}$が,
            いずれも
            \[ \phi_i \circ \pr_i=\phi_j \circ \pr_j \colon U_{ij} \to U \]による$\xi$のpullback
            であることから得られる
            標準的同型のclass $\{ \sigma_{ji} \colon \xi_j|_{U_{ij}} \to \xi_i|_{U_{ij}} \}_{i,j}$.
    \end{itemize}
    このデータをまとめて$(\{\phi_i^*\xi\}, \mathrm{cano})$などと書く.
    この対応を$\epsilon_{\covU} \colon \fibF(U) \to \fibF(\covU)$と書く.
    $\fibF(U)$の射$\xi \to \eta$から,
    $\phi_i$に沿ったpullbackによって$(\{\phi_i^*\xi\}, \mathrm{cano}) \to (\{\phi_i^*\eta\}, \mathrm{cano})$
    が得られるので,
    対応$\epsilon_{\covU}$は関手である.
\end{Def}

\subsection{Example}
\begin{Example}[\cite{ASS}, 4.2.1]
    一つの射から成るcover :: $\covU=\{f \colon V \to U\}$について$\fibF(\covU)$を考えてみる.
    この圏の対象は,
    \begin{itemize}
        \item 対象$E \in \fibF(V)$
        \item $\fibF(V \times_U V)$の中の同型射$\sigma \colon \pr_1^*E \to \pr_2^*E$
    \end{itemize}
    の組である.
\end{Example}

\section{Prestack / Stack.}
\subsection{Definitions.}
\begin{Def}[Prestack, Stack]
    関手$\epsilon_{\covU} \colon \fibF(U) \to \fibF(\covU)$を用いて以下のように定義する.
    \begin{enumerate}[label=(\roman*)]
    \item
        任意の$U \in \cat{C}, \covU \in \Cov(U)$について$\epsilon_{\covU}$ :: fully faithfullである時,
        fibered category $\fibF \to \cat{C}$はprestackである,という.
    \item
        任意の$U \in \cat{C}, \covU \in \Cov(U)$について$\epsilon_{\covU}$ :: equivalenceである時,
        fibered category $\fibF \to \cat{C}$はstackである,という.
    \end{enumerate}

    (pre)stacksの間の射は,fibered categoryとしての射である.
\end{Def}

\begin{Remark}
    prestackの定義は以下のように言い換えられる:
    任意の$U \in \cat{C}, \covU=\{\phi_i \colon U_i \to U\} \in \Cov(U)$をとる.
    さらに$\xi, \eta \in \fibF(U)$をとり,$\epsilon_{\covU}$による像を
    \[
        \epsilon_{\covU}(\xi)=(\{\xi_i\}, \{\sigma_{ij}\}),\quad
        \epsilon_{\covU}(\eta)=(\{\eta_i\}, \{\tau_{ij}\})
            \in \fibF(\covU) 
    \]とする.
    この時,任意の射
    $\{\alpha_i\} \colon \epsilon_{\covU}(\xi) \to \epsilon_{\covU}(\xi)$
    (射$\alpha_i \colon \xi_i \to \eta_i$の集合)について,
    $\fibF(U)$の射$\alpha \colon \xi \to \eta$が\underline{一意に存在し},
    $\alpha_i=(\phi_i)^*\alpha (\iff \{\alpha_i\}=\epsilon_{\covU}(\alpha))$となる.

    標語的に言えば,prestackは「貼り合わせられる射を持つpsuedo-functor」となる.
    同型射の貼り合わせは同型射であるから,
    prestackは「貼り合わせが(存在すれば)一意な対象を持つpsuedo-functor」である.
\end{Remark}

\begin{Remark}
    このノートでは,
    fiberが条件を満たすfibered categoryとして(pre)stackは定義されている
    (fiberを用いずに(pre)stackを定義することも出来るが,今回は採用しなかった).
    なので形式上,(pre)stackはfibered categoryを経由せず,特別なpsuedo-functorとして定義できる.
    しかし実際にそのように定義されることは少ない.

    ではpsuedo-functorとして定義しない積極的な理由はと言うと,
    実用上,元のfibered categoryにも言及する場合が多いからであると思われる.
    fiberだけでなく元のfibered categoryに言及する理由については,
    このセミナーのノート session 4.5
    \footnote{URL : \url{https://github.com/ShitijyouA/MathNotes/blob/master/AlgebraicStacks/session4_5_FiberedCategoriesContinued.pdf}},
    注意2.8を参考にして欲しい.
\end{Remark}

\begin{Def}[Sub(pre)stack]
    stack :: $\pi \colon \fibF \to \cat{C}$のsub(pre)stackとは,
    $\fibF$の部分圏$\fibG$であって,
    $\pi$と包含関手の合成$\fibG \inclmap \fibF \xrightarrow{\pi} \cat{C}$がfibrationであり,
    さらにそのfiberが(pre)stackであるもの.
\end{Def}

\subsection{Examples.}
\begin{Prop}[\cite{NoteGroTop} Prop4.9]
    \begin{myenum}
        \item separated presheaf of sets is a prestack.
        \item sheaf of sets is a stack.
    \end{myenum}
\end{Prop}
\begin{proof}
    $\cat{C}$ :: site, $\fibF \colon \cat{C}^{op} \to \Sets$ :: presheafとする.
    $U \in \cat{C}, \covU=\{U_i \to U\} \in \Cov(U)$を任意に取る.

    今,圏$\fibF(U), \fibF(\covU)$は集合(離散圏)である.
    なので関手$\epsilon_{\covU} \colon \fibF(U) \to \fibF(\covU)$は\underline{写像}である.
    さらに射$\sigma_{ij}$も恒等射しかないから,
    $\fibF(\covU)$の対象は,
    任意の$i,j$について$\xi_i|_{U_{ij}}=\xi_j|_{U_{ij}}$を満たす
    $\xi_i \in \fibF(U_i)$の族$\{\xi_i\}_i$であると考えて良い.
    このセミナーノートのsession3の記号を用いれば,
    $\fibF(\covU)=H^0(\covU, \fibF)$ということに成る.

    二つのデータ$\{\xi_i\}, \{\eta_i\}$の間の射もやはり恒等射しかないから,
    「関手$\epsilon_{\covU}$がfully faithfulである」という仮定は
    「写像$\epsilon_{\covU}$が単射である」と言い換えられる.
    これはすなわち,$\fibF$がseparated presheafであるということである.

    「関手$\epsilon_{\covU}$がessentially surjectiveである」という仮定は
    「写像$\epsilon_{\covU}$が全射である」と言い換えられるから.
    $\epsilon_{\covU}$がequivalenceであることは
    $\fibF(\covU)=H^0(\covU, \fibF)$と$\fibF(U)$の間に全単射が存在するということである.
    これはすなわち,$\fibF(U)$がsheafであるということである.
\end{proof}

\begin{Remark}
    この命題で分かるとおり,
    prestackはpresheafの抽象化ではなく,separated presheafの抽象化である.
    そうすると,
    我々はpsuedo-functor $\cat{B}^{op} \to \Cat$をprestackと呼び,
    今prestackと呼んでいるものはseparated prestackと呼ぶべきなのかも知れない.
    我々がそうしないのは,後に定義される``separated stack"との混乱を避けるためである.
\end{Remark}

以下の二つの例は後にセミナーでも証明を扱う.

\begin{Example}
    $X \in \cat{C}$に対し,圏$\cat{Shv}/X$を以下のように定める.
    \begin{description}
        \item[Objects.] \mnewline
            $X$への射を持つような$\cat{C}$の対象 :: $U$と,
            $U$上のsheaf :: $\shU$の組.
        \item[Arrows.] \mnewline
            射$(U, \shU) \to (V, \shV)$は,
            $\cat{C}$の射$f \colon U \to V$と,
            morphism of sheaves on $V$ :: $f^{\#} \colon \shV \to f_*\shU$の組.
    \end{description}
    この時,fibered category :: $\cat{Shv}/X \to \cat{C}/X; \ (U, \shU) \mapsto U$はstackである.
    この例で考えるsheafをquasi-coherent sheafに制限してて得られる
    fibered category :: $\cat{QCoh}/X \to \cat{C}/X$もstackである.
    この二つの例については,このセミナーでも後に証明を扱う.
\end{Example}

\begin{Example}
    $X \in \Sch$に対し,圏$\cat{QCoh}/X$を以下のように定める.
    \begin{description}
        \item[Objects.] \mnewline
            $\mathrm{Fpqc}(X)$
            \footnote{圏$\Sch/X$にfpqc topologyを備えたもの.}の対象 :: $U$と,
            $U$上のsheaf (on fpqc topology):: $\shU$の組.
        \item[Arrows.] \mnewline
            射$(U, \shU) \to (V, \shV)$は,
            $\cat{C}$の射$f \colon U \to V$と,
            morphism of sheaves on $V$ :: $f^{\#} \colon \shV \to f_*\shU$の組.
    \end{description}
    この時,fibered category :: $\cat{QCoh}/X \to \cat{C}/X; \ (U, \shU) \mapsto U$はstackである.
\end{Example}

\begin{Example}[\cite{ASS} 4.4.1]
    以下で定まるfibered categoryはstackである.
    \begin{defmap}
    {}& \left\{\parbox{4cm}{\begin{center} pair of scheme over $S$ :: $Y$ \\ and closed imm. $W \inclmap Y$ \end{center}}\right\}& \to& \mathrm{Fppf}(S) \\
        {}& (Y, W \inclmap Y)& \mapsto& Y
    \end{defmap}
\end{Example}

\begin{Example}[\cite{ASS} 4.4.4]
    以下で定まるfibered categoryはstackである.
    \begin{defmap}
    {}& \left\{\parbox{4cm}{\begin{center} pair of scheme over $S$ :: $Y$ \\ and open imm. $W \inclmap Y$ \end{center}}\right\}& \to& \mathrm{Fppf}(S) \\
        {}& (Y, W \inclmap Y)& \mapsto& Y
    \end{defmap}
\end{Example}

以下の二つの例は後に一般化される.

\begin{Example}[\cite{NoteGroTop} \S 4.3.1]
    arrow category :: $\Sch^{\rightarrow}$の対象を
    affine morphismに制限したものを圏$\cat{Aff}$とする.
    以下で定まるfibered categoryはstackである.
    \begin{defmap}
        {} & \cat{Aff}& \to& \mathrm{Fppf}(\Spec \Z) \\
        {}& [ X \to Y ]& \mapsto& Y
    \end{defmap}
\end{Example}

\begin{Example}[\cite{ASS} 4.4.15]
    quasi-compact open imbeddingの後にaffine morphismを合成した射のことを
    quasi-affine morphismという.
    arrow category :: $\Sch^{\rightarrow}$の対象を
    quasi-affine morphismに制限したものを$\cat{QAff}$とする.
    以下で定まるfibered categoryはstackである.
    \begin{defmap}
        {} & \cat{QAff}& \to& \mathrm{Fppf}(\Spec \Z) \\
        {}& [ X \to Y ]& \mapsto& Y
    \end{defmap}
\end{Example}

\subsection{Propositions.}

\begin{Prop}[\cite{ASS} Prop4.12]
    二つのequivalentなfibered categoryがあり,
    かつ一方がstackならば,もう一方もstackである.
\end{Prop}
\begin{proof}
    $\fibF, \fibG$ :: fibered categories over $\cat{C}$とし,
    $F \colon \fibF \to \fibG$ :: morphism of fibered categoriesとする.
    この時cover of $U \in \cat{C}$ :: $\covU=\{U_i \to U\}$について$F_{\covU}$を定義する.
    \begin{defmap}
        F_{\covU}\colon & \fibF(\covU)& \to& \fibG(\covU) \\
        \mathbf{Objects:}& (\{\xi_i\}, \{\sigma_{ij}\})& \mapsto& (\{F\xi_i\}, \{F\sigma_{ij}\}) \\
        \mathbf{Arrows:}& \{\alpha_i\}& \mapsto& \{F\alpha_i\} \\
    \end{defmap}
    更に二つの射$F, G \colon \fibF \to \fibG$と
    その間のbase-preserving natural transformation :: $\rho \colon F \to G$に対し,
    $\rho_{\covU} \colon F_{\covU} \to G_{\covU}$を次のように定義する.
    \[ (\rho_{\covU})_{(\{\xi_i\}, \{\sigma_{ij}\})}=\{\rho_{\xi_i}\}. \]

    以上から,$F$がequivalenceならば$F_{\covU}$もquivalenceである.
    したがって以下のcommutative diagram of weak $2$-category
    \footnote{ 射の合成の間にnatural isomorphismが存在するという意味で可換. }
    が得られる.
    \[\xymatrix{
            \fibF(U) \ar[r]^-{\epsilon_{\covU}}\ar[d]_-{F}& \fibF(\covU) \ar[d]^-{F_{\covU}}\\
            \fibG(U) \ar[r]_-{\epsilon_{\covU}}& \fibF(\covU)
    }\]
    この可換図式から,主張が得られる.
\end{proof}

\begin{Prop}[\cite{ASS} Exc 4.I]
    $\fibF, \fibF'$ :: stack on $\cat{C}$,
    $f \colon \fibF \to \fibF'$ :: morphism of stacksとする.
    $f$ :: isomorphismは以下の2条件が成立することと同値.
    \begin{enumerate}[label=(\alph*)]
        \item
            任意の$X \in \cat{C}$について,
            fiberの間の射$f_X \colon \fibF(X) \to \fibF'(X)$はfully-faithful.
        \item
            任意の$X \in \cat{C}$と$x \in \fibF'(X)$について,
            covering of $X$ :: $\{\phi_i \colon X_i \to X\} \in \Cov(X)$が存在し,
            全ての$x$のpullback :: $\phi_i^*x \in \fibF'(X_i)$が$\fibF(X_i)$のessential imageに属す.
    \end{enumerate}
\end{Prop}
\begin{proof}
    (TODO)
\end{proof}

\begin{Lemma}
    site :: $\cat{C}$を,
    空集合のcoverとして空集合を持つ($\emptyset \in \Cov(\emptyset)$)ものとする.
    $\pi \colon \fibF \to \cat{C}$ :: stackについて,以下の圏同値が成立する.
    \[ \fibF( \emptyset ) \simeq \cat{1}. \]
    特に,$\fibF(\emptyset)$の任意の二つの対象の間には,ただ一つの同型射が存在する.
\end{Lemma}
\begin{proof}
    category of descent data :: $\fibF(\covU)$の対象を考える.
    これは$\covU$で添字付けられた対象の族の二つ組である.
    なので$\covU=\emptyset$について,
    $\fibF(\emptyset)$の対象は$(\emptyset, \emptyset)$しかない.
    射も$\covU$で添字付けられた族であるから,非自明な射は存在しない.
\end{proof}

この補題の仮定は奇妙に見えるかも知れないが,
以下の通り,このように仮定しても問題はないし,
我々が扱う殆どのsiteはこの仮定を満たす.

\begin{Claim}
    圏$\cat{C}$の任意の対象$U \in \cat{C}$について,
    命題「$\emptyset \in \Cov(U)$」はGrothendieck topologyの公理(定義)と独立である.
    すなわち,$\emptyset \in \Cov(U)$としてもしなくても矛盾は生じない.
\end{Claim}
\begin{proof}
    命題「$\emptyset \in \Cov(U)$」を$P$と書く.
    Grothendieck topologyの定義を見直そう.
    cover of $\emptyset$ :: $\covU \in \Cov(U)$が満たすべき条件を記号で書き下す.
    \begin{enumerate}[label=(\alph*)]
        \item
            $\Forall{[V \to U] \in \cat{C}/U}
            \lbra{ \Forall{[U' \to U] \in \covU} {}^{\exists} U' \times_{U} V }
            \implies \{ U' \times_{U} V \to V \mid [U' \to U] \in \covU \} \in \Cov(V)$.

        \item $\Forall{\covV:=\{ \covU'_{U'} \mid \covU'_{U'} \in \Cov(U')\}_{U' \in \covU}}
                \{ U'' \to U' \to U \mid
                    [U' \to U] \in \covU, [U'' \to U'] \in \covU'_{U'} \} \in \Cov(U)$.
    \end{enumerate}
    クラス$X$と述語$F$について``$\Forall{x \in X} F(x)$"という文は
    ``$\Forall{x} \lbra{x \in X \implies F(x)}$の省略形である.
    したがって,$X=\emptyset$であるとき,
    ``$\Forall{x \in X} F(x)$"という文は任意の$F$について真.
    また,$\{ f(x) \mid x \in \emptyset \}=\emptyset$.

    なので,以上の文を$\covU=\emptyset$の場合に考えると(すなわち$P$を仮定すると),
    いずれも$P$と同値に成る.
    よって$P \implies P$ということになる.
    一方,否定$\lnot P$を仮定しても矛盾が生じないことは明らか.
\end{proof}

\begin{Example}
    圏$\cat{C}$を$\Sch$の部分圏や$\Sch/S$\ ($S$ :: scheme)とする.
    morphism of schemesのクラス$\mathcal{P} \subset \Arr(\cat{C})$をとり,
    以下のように$\cat{C}$上の$\Cov$を定めたとする:
    \begin{align*}
        \Cov(U)
        =&\{
            \covU \mid
            \covU \text{ :: jointly surjective family and }
            \Forall{\phi \in \covU} \phi \in \mathcal{P}
        \} \\
        =&\left\{
            \covU ~\middle|~
            \bigsqcup_{U' \in \covU}U' \to U \text{ :: surjective}
            \mand
            \Forall{\phi} \lbra{ \phi \in \covU \implies \phi \in \mathcal{P} }
        \right\}.
    \end{align*}
    この時,$\bigsqcup_{U' \in \emptyset}U'=\emptyset$なので$\emptyset \in \Cov(\emptyset)$.

    このセミナーで定義したZariski site, etale site, ...などは
    全てこの主張のように定義されている.
\end{Example}

\begin{Lemma}
    圏$\cat{C}$を$\Sch$の部分圏や$\Sch/S$\ ($S$ :: scheme)とする.
    $U \in \cat{C}, \{U_i \to U\} \in \Cov(U)$をとり,
    $V=\bigsqcup_{i} U_i$と置く.
    
    $\left\{ U_i \to V \right\} \in \Cov(V)$と仮定する
    \footnote{ 例えば,Zariski topologyより細かい位相ならばこの仮定は成立する.}と
    $\pi \colon \fibF \to \cat{C}$ :: stackについて,
    圏同値(TODO: strict $2$-equivalence? ここは$\epsilon$と圏同型の合成)
    \[ \fibF \left( \bigsqcup_i U_i \right) \simeq \prod_i \fibF(U_i) \]
    が成立する.
\end{Lemma}
\begin{proof}
    瑣末なことでは有るが: 
    $\{U_i \to V\}$の添字について,
    $i \neq j$ならば$U_i \neq U_j$である,と仮定して一般性を失わない.

    $\covU=\{ \inj_i \colon U_i \to V \}(\in \Cov(V))$と置く.
    次の関手が圏同値であることを示す.
    \begin{defmap}
        E \colon& \im \epsilon_{\covU}& \to& \prod_i \fibF(U_i) \\
        \mathbf{Objects}\colon&
            (\{(\inj_i)^*\xi\}_i, \{\sigma_{ij}\}_{i, j})& \mapsto& ((\inj_i)^*\xi)_i\\
        \mathbf{Arrows}\colon& \{\alpha_i\}_i& \mapsto& (\alpha_i)_i
    \end{defmap}
    $\xi_i:=(\inj_i)^*$とおく.
    これが示せれば,$\fibF$ :: stackなので主張も得られる.

    まず,仮定の状況では,
    injection map (coprojection) :: $U_i \to V$についてのfiber productは
    次のように成る.
    \[
        U_{ij}
        =U_i \times_V U_j =
        \begin{cases}{}
            U_i & (U_i=U_j) \\
            \emptyset & (U_i \neq U_j).
        \end{cases}
    \]
    
    \step{$E$ :: essentially surjective.}
    $i \neq j$の時,
    $\sigma_{ij}$は$\fibF(U_{ij})=\fibF(\emptyset)$の射であるから,
    $\xi_i|_{\emptyset}$と$\xi_j|_{\emptyset}$の間に存在するただ一つの射である.
    一方,$i=j$の時は,
    $(\pr_i)^*(\inj_i)^*\xi$と$(\pr_j)^*(\inj_j)^*\xi$が完全に等しいので,
    $\sigma_{ij}=\id[\xi_i]$.
    以上より,対象同士の次の対応は$\mathrm{Ob}(E)$の逆対応と成る.
    \begin{defmap}
        {} & \mathrm{Ob}\left( \prod_i \fibF(U_i) \right)& \to&
            \mathrm{Ob}\left( \im \epsilon_{\covU} \right) \\
        {}& (\xi_i)_i& \mapsto&
            (\{\xi_i\}_i, \{\sigma_{ij}\}_{i \neq j} \cup \{\id[\xi_i]\}_{i}) \\
    \end{defmap}
    これは$E$ :: essentially surjectiveを意味する.

    \step{$E$ :: fully-faithfull.}
    また,上で述べたことから,
    射$\{\alpha_i\} \colon \epsilon_{\covU}(\xi) \to \epsilon_{\covU}(\eta)$に課された条件が,
    任意の射の組み合わせ$(\alpha_i \colon \xi_i \to \eta_i)_i$について成立することも分かる
    ($i \neq j$と$i=j$の場合で証明は異なる).
    なので全単射
    \[
        \Hom(\epsilon_{\covU}(\xi), \epsilon_{\covU}(\eta))
        =\Hom((\xi_i)_i, (\eta_i)_i)
        =\Hom(E(\epsilon_{\covU}(\xi)), E(\epsilon_{\covU}(\eta)))
    \]
    が得られる.
    これは$E$ :: fully-faithfullを意味する.
\end{proof}

\begin{Thm}[Stackification of category fibered by groupoids.]
    $\cat{C}$ :: site,
    $\fibF$ :: category fibered by groupoids over $\cat{C}$とする.
    この時,
    $\bar{\fibF}$ :: stack in groupoids over $\cat{C}$と
    $\theta \colon \fibF \to \bar{\fibF}$ :: morphism of fibered categoryが存在し,
    \[ (-) \circ \theta \colon \HOM_{\cat{C}}(\bar{\fibF}, -) \to \HOM_{\cat{C}}(\fibF, -) \]
    が圏同値となる.
\end{Thm}
(TODO: これは\cite{ASS} Thm4.6.5からとった.
しかし,\url{https://stacks.math.columbia.edu/tag/02ZM}に一般の場合が書かれている.
出来ればこちらを理解したい)

\begin{Example}
    presheafのstackificationはsheafificationと一致する.
\end{Example}

\begin{Example}[arXiv:math/0305243v1, Prop3.6]
    $S$ :: scheme,
    $\shM$ :: algebraic stack over $\Sch/S$,
    $\fibG$ :: sheaf in groups over $\Sch/S$, acting on $\shM$とする.
    この時,$\shM$の$\fibG$によるcategorical quotient :: $\shM/\fibG$は,
    以下のprestack($2$-functorとして定義する):: $\shP$ のstackificationとして定義される.
    \begin{description}[labelindent=1cm]
        \item[Objects of $\shP(U)$.] $\shM(U)$の対象と同じ.
        \item[Arrows of $\shP(U)$.]  $g \in \fibG(T)$と$\shM(U)$の射$g \ast x \to y$の組.
    \end{description}
    ただし$U \in \Sch/S$は任意.
\end{Example}

\section{Descent Theory.}
\subsection{Motivation.}
    (TODO)

\subsection{Definitions.}
\begin{Def}
    関手$\epsilon_{\covU} \colon \fibF(U) \to \fibF(\covU)$を用いて以下のように定義する.
    \begin{enumerate}[label=(\roman*)]
        \item
            $\epsilon_{\covU}$ :: equivalenceとなる$\covU$をof effective descent for $\fibF$と呼ぶ.
        \item
            $\epsilon_{\covU}$の像と同型である$\fibF(\covU)$の対象を,effective dataという.
    \end{enumerate}
\end{Def}

\section{Criterion for fpqc Stacks.}
\begin{Thm}[\cite{ASS} Lemma 4.25]\label{thm:fpqccriterion}
    $S$ :: scheme,
    $\fibF \to (\Sch/S)$ :: fibrationとする.
    以下が成り立つとする.
    \begin{enumerate}[label=(\alph*)]
        \item $\fibF$はZariski topologyでのstackである.
        \item
            任意のflat surjective morphism of affine $S$-scheme :: $V \to U$について,\mnewline
            $\epsilon_{\{V \to U\}} \colon \fibF(U) \to \fibF(\{V \to U\})$は圏同値.
    \end{enumerate}
    この時,$\fibF$はfpqc topologyでのstackである.
\end{Thm}

\begin{Remark}
    ``flat"という条件は以下の証明では利用されない.
\end{Remark}

\subsection{Step 1 / 準備}
以前示した命題から,
$\fibF$ :: split fibered categoryと仮定しても一般性を失わないので,
以下そのように仮定する.

\begin{Lemma}\label{lemma:key}
    $\cat{C}$をsiteとし,
    $\fibF \to \cat{C}$を\underline{split} fibrationとする.
    さらに$U \in \cat{C}$,$\covU=\{\phi_i \colon U_i \to U\} \in \Cov(U)$と
    $\covU$の細分
    \footnote
    {
        細分の定義を確認しておく:
        任意の$\covV$の元$V_{ij} \to U$に対して$\covU$の元$U_i \to U$が存在し,
        $V_{ij} \to U$が$U_i \to U$を通して$V_{ij} \to U_i \to U$と分解する.
        特に射$V_{ij} \to U_{i}$が存在する.
    }
    $\covV=\{\psi_{ij} \colon V_{ij} \to U\}$をとる.

    この時,関手$R_{\covU}^{\covV} \colon \fibF(\covU) \to \fibF(\covV)$が存在し,
    以下は厳密な可換図式である。
    \[
        \begin{tikzcd}
        \fibF(U) \ar[r, "\epsilon_{\covU}"] \ar[d, "\epsilon_{\covV}"']&
        \fibF(\covU) \ar[ld, "R_{\covU}^{\covV}"] \\
        \fibF(\covV)
    \end{tikzcd}
    \]
\end{Lemma}
%% {{{
\begin{proof}
    \step{関手$\fibF(\covU) \to \fibF(\covV)$の構成.}
    細分の定義から,各$i,k$について以下が可換に成る射$\iota_{ik} \colon V_{ik} \to U_{i}$が存在する.
    \[
    \begin{tikzcd}
        V_{ik} \ar[r, "\iota_{ik}"'] \ar[rr, bend left, "\psi_{ik}"]& U_{i} \ar[r, "\phi_i"']& U
    \end{tikzcd}
    \]
    この射$\iota_{ik}$を用いて,
    関手$R_{\covU}^{\covV}$を次のように定義する。
    \begin{defmap}
        R_{\covU}^{\covV} \colon& \fibF(\covU)& \to& \fibF(\covV) \\
        \mathbf{Objects}& (\{\eta_i\}, \{\sigma_{ij}\})& \mapsto&
            \left(\{(\iota_{ik})^*\eta_i\}, \left\{\left(\iota_{ik}^{jl}\right)^*\sigma_{ij}\right\}\right) \\
        \mathbf{Arrows}& \{\alpha_i\}& \mapsto& \{(\iota_{ik})^*\alpha_i\}
    \end{defmap}
    ここで$\iota_{ik}^{jl}$は,
    以下の可換図式のようにfiber productの一意性から得られる射である.
    \begin{center}
    \begin{tikzcd}[row sep=1.4cm, column sep=0.7cm]
        V_{ik} \times_U V_{jl}
        \ar[rr, "\pr_1"]\ar[dd, "\pr_2"']\ar[rd, red, "\iota_{ik}^{jl}"]&&
        V_{ik} \ar[d, "\iota_{ik}"'] \ar[dd, bend left, "\psi_{ik}"]\\
        {}& U_{i} \times_U U_{j} \ar[r, "\pr_1"]\ar[d, "\pr_2"']& U_{i} \ar[d, "\phi_i"']\\
        V_{jl} \ar[r, "\iota_{jl}"]\ar[rr, bend right, "\psi_{jl}"']& U_{j} \ar[r, "\phi_j"]& U
    \end{tikzcd}
    \end{center}
    $\{\sigma_{ij}\}$がcocycle conditionを満たすので,
    $\left\{\left(\iota_{ik}^{jl}\right)^*\sigma_{ij}\right\}$もcocycle conditionを満たす
    \footnote
    {
        証明はfiber productの普遍性から得られる
        射$V_{il} \times V_{jm} \times V_{kn} \to U_{i} \times U_{j} \times U_{k}$を用いれば良い.
    }.
    同様に$\{(\iota_{ik})^*\alpha_i\}$が$\fibF(\covV)$の射であることも確認できる.

    \step{対象について$R_{\covU}^{\covV}\epsilon_{\covU}=\epsilon_{\covV}.$}
    $R_{\covU}^{\covV}\epsilon_{\covU}$を計算する.
    まず$\xi \in \fibF(U)$をとり,$R_{\covU}^{\covV}\epsilon_{\covU}(\xi)$を計算しよう.
    \begin{align*}
        R_{\covU}^{\covV}\epsilon_{\covU}(\xi)
        =&  R_{\covU}^{\covV}\Big( (\{\phi_i^*\xi\}, \{\sigma_{ij}\}) \Big) \\
        =&  \left(\{(\iota_{ik})^*\phi_i^*\xi\}, \left\{\left(\iota_{ik}^{jl}\right)^*\sigma_{ij} \right\}\right) \\
        =&  \left(\{(\psi_{ik})^*\xi\}, \left\{\left(\iota_{ik}^{jl}\right)^*\sigma_{ij} \right\}\right)
    \end{align*}
    今,$\fibF$ :: split fibered categoryとしているので,
    \[ \pr_2^*\phi_j^* \xi=(\phi_j \circ \pr_2)^*\xi=(\phi_i \circ \pr_1)^*\xi=\pr_1^*\phi_i^* \xi. \]
    $\sigma_{ij}$はfiber productの普遍性から得られる
    $\pr_2^*\phi_j^* \xi$から$\pr_1^*\phi_i^* \xi$への同型であるから,
    $\sigma_{ij}=\id$.
    このことと$\fibF$ :: splitから$\left(\iota_{ik}^{jl}\right)^*\sigma_{ij}=\id$も分かる.
    まとめると,
    $R_{\covU}^{\covV}\epsilon_{\covU}(\xi)=\left(\{(\psi_{ik})^*\xi\}, \{\id[(\psi_{ik})^*\xi]\}\right)$.

    一方,
    \[
        \left(\iota_{ik}^{jl}\right)^*\pr_2^*\phi_j^* \xi
        =(\psi_{jl} \circ \pr_2)^*\xi
        =\pr_2^*(\psi_{jl})^*\xi
        =\pr_1^*(\psi_{ik})^*\xi
        =(\psi_{ik} \circ \pr_1)^*\xi
        =\left(\iota_{ik}^{jl}\right)^*\pr_1^*\phi_i^* \xi.
    \]
    なので,fiber productの普遍性から得られる
    $\pr_2^*(\psi_{jl})^*\xi$から$\pr_1^*(\psi_{ik})^*\xi$への同型は$\id$である.
    したがって
    $\epsilon_{\covV}(\xi)=\left(\{(\psi_{ik})^*\xi\}, \{\id[(\psi_{ik})^*\xi]\}\right)$となり,
    $R_{\covU}^{\covV}\epsilon_{\covU}(\xi)=\epsilon_{\covV}(\xi)$.
    
    \step{射について$R_{\covU}^{\covV}\epsilon_{\covU}=\epsilon_{\covV}.$}
    $\fibF(U)$の射$\alpha \colon \xi_1 \to \xi_2$をとる.
    すると$\fibF$ :: splitなので
    \[
        R_{\covU}^{\covV}\epsilon_{\covU}(\alpha)
        =\{(\iota_{ik})^*\phi_i^*\alpha\}=\{(\phi_i \circ \iota_{ik})^*\alpha\}=\{\psi_{ik}^*\alpha\}
        =\epsilon_{\covV}(\alpha).
    \]
\end{proof}
%% }}}
\begin{Remark}
    $\fibF$ :: splitを仮定しない場合,
    さらにbase preserving isomorphism ::
    $R_{\covU}^{\covV}\epsilon_{\covU} \to \epsilon_{\covV}$を構成する必要がある.
\end{Remark}

%\begin{Lemma}
%    $\cat{C}, \fibF$等を上の補題のようにする.
%    $U \in \cat{C}$をとり,$\covU \in \Cov(U)$とする.
%    さらに$\covV$を$\covU$の細分,$\covW$を$\covV$の細分とする
%    (したがって$\covW$は$\covU$の細分である).
%    この時,次が成立する.
%    \[ R_{\covV}^{\covW} \circ R_{\covU}^{\covV}=R_{\covU}^{\covW}. \]
%\end{Lemma}
%\begin{proof}
%    明らかなので証明は略す.
%\end{proof}

\subsection{Step 2 / single morphism coverの場合に帰着させる.}
    \begin{Cor}[\cite{NoteGroTop} p.87]
        $\cat{C}, \fibF$等を補題(\ref{lemma:key})の様にとる.
        $\covU=\{\phi_i \colon V_i \to U\}, V'=\bigsqcup V_i$とする.
        さらに,$f \colon V' \to U$を$\covU$から誘導される射とする.

        このとき,圏同値 $R_{V' \to U}^{\covU} \colon \fibF(V' \to U) \to \fibF(\covU)$が存在し,
        合成
        \[\xymatrix{
            \fibF(U) \ar[r]^-{\epsilon_{\{f\}}}& \fibF(V' \to U) \ar[r]^-{E}& \fibF(\covU)
        }\]
        が関手$\epsilon_{\covU} \colon \fibF(U) \to \fibF(\covU)$と厳密に一致する.
    \end{Cor}
    \begin{proof}
        $\covU$は$\{U' \to U\} \in \Cov(U)$の細分であるから,
        補題(\ref{lemma:key})から明らか.
    \end{proof}

    \begin{Remark}
        ここで,各$\phi_i$がquasi-compact(特にfpqc)であったとしても,
        誘導される射$f \colon V' \to U$が必ずしもquasi-compactでないことに注意する.
        例えば$\{\Spec k[x_i] \to \bigsqcup_i \Spec k[x_i]\}_{i \in \N}$を考えれば分かる.

        以上のことに注意すると,我々は次のことを証明することに成る:
        \begin{Claim}
            条件(a), (b)が成り立つならば,
            以下の条件$(*)$を満たす任意のflat surjective morphism :: $f \colon V \to U$について,
            $\epsilon_{\{f\}} \colon \fibF(U) \to \fibF(f \colon V \to U)$ :: equivalence.
            \begin{description}
                \item[$(*)$]
                    affine Zariski cover :: $U=\bigcup_i U_i$と,
                    各$i$について$f^{-1}(U_i)$のaffine Zariski cover :: $f^{-1}(U_i)=\bigcup_i V_{ij}$が存在し,
                    $V_{ij}$ :: quasi-compactかつ$f(V_{ij})=U_i$となる.
            \end{description}
        \end{Claim}
        条件$(*)$は$U, V$ :: locally noetherianであるような
        任意のfppf morphismについて成立する(\cite{ASS} Cor1.1.6).
    \end{Remark}

    \begin{Remark}
        以下,$\fibF$ :: split fibered categoryとする.
        session 4.5 定理1.2より,このように仮定しても一般性を失わない.
    \end{Remark}

\subsection{Step 3 / affine schemeへのquasi-compact morphismの場合.}
    $f \colon V \to U$を$U$ :: affineであるquasi-compact morphismとする.
    $\{ V_i \}_i$を$V$のopen affine coverとし,$V'=\bigsqcup_i V_i$とおく.
    $V'$ :: affineなので,
    仮定(b)から圏同値$\fibF(U) \simeq \fibF(V' \to U)$が存在する.

    以下の図式$(1)$を考える.
    $\leftrightarrow$は圏同値を意味する.
    \[
    \begin{tikzcd}[row sep=1cm]
        \fibF(U) \ar[r, "\epsilon_{f}"]\ar[d, leftrightarrow, "\epsilon_{\{V_i \to U\}}"']&
        \fibF(f \colon V \to U) \ar[ld, "R_{f}^{V_i \to U}"]\\
        \fibF(\{V_i \to U\}) & \\
        \fibF(V' \to U) \ar[u, leftrightarrow, "R_{V' \to U}^{\{V_i \to U\}}"']
        \ar[uu, leftrightarrow, shift left=2mm, bend left=90, rounded corners, "\epsilon_{V' \to U}"]
    \end{tikzcd}
    \eqno{(1)}
    \]
    ここで関手$\epsilon_{f}$は次で与えられる.
    ただし$\pr_k \colon V \times_U V \to V$は第$k$成分への射影である.
    \begin{defmap}
        \epsilon_{f}\colon & \fibF(U)& \to& \fibF(f \colon V \to U) \\
        \mathbf{Objects:}& \xi& \mapsto& (f^*\xi, \sigma)\\
        \mathbf{Arrows:}& \alpha& \mapsto& f^*\alpha
    \end{defmap}
    ここで$\sigma \colon \pr_2^*f^*\xi \to \pr_1^*f^*\xi \in \Arr(\fibF(V \times_U V))$は,
    恒等射$\id[\pr_2^*f^*\xi]$である.
    これは
    $\pr_2^*f^*\xi,\pr_1^*f^*\xi$が
    いずれも$f \circ \pr_2=f \circ \pr_1$による$\xi$のpullbackであることと,
    $\fibF$ :: splitから得られる

    この図式の可換性から,
    関手の同型$(R_{f}^{V_i \to U}) \circ \epsilon_{f}=\epsilon_{\{V_i \to U\}}$が得られる
    ($\fibF$ :: split fibered categoryを利用する).
    (よって上の図式$(1)$は可換である.)
    したがって$\epsilon_{f}$のpsuedo-inverse functor ::
    $(\epsilon_{\{V_i \to U\}})^{-1} \circ (R_{f}^{V_i \to U})$が得られた.

\subsection{Step 4 / 条件\tp{$(*)$}{(*)}を満たすaffine schemeへの射の場合.}
    (\cite{NoteGroTop} p.88)
    仮定$(*)$より,
    Zariski cover :: $\{\iota_i \colon V_i \to V\}$が存在し,$V_i$ :: quasi-compact.
    \begin{Remark}
        前段の議論のうち,図式$(1)$の$\fibF(U) \to \fibF(V' \to U)$が圏同値でない.
        なので新しい議論が必要である.
    \end{Remark}

    補題(\ref{lemma:key})から得られる以下の可換図式を考える.
    \[
    \begin{tikzcd}[row sep=10mm]
        \fibF(U) \ar[r, "\epsilon_{f}"]
        \ar[d, leftrightarrow, "\substack{\epsilon_{V_i \to U} \\ \text{\color{red}equivalence}}"']&
        \fibF(f \colon V \to U)
        \ar[ld, "\substack{R_{f}^{V_i \to U} \\ \text{\color{red}essentially surjective \& full}}"]\\
        \fibF(V_i \to U)
    \end{tikzcd}
    \]
    左にある縦の射はStep 3から圏同値である.
    したがって$\fibF(V \to U) \to \fibF(V_i \to U)$(定義はおおよそ関手$E$と同様に与えられる)は
    essentially surjectiveかつfullである.
    なのでこの関手が更にfaithfullであることが証明できれば,
    図式の可換性から$\epsilon_{f}$が圏同値であることが証明できる.

    $\fibF(V \to U)$の射$\beta, \beta'$が,$\beta|_{V_i}=\beta'|_{V_i}$を満たすとする.
    この時,$\beta=\beta'$を証明すれば良い.
    まず,以下の厳密な可換図式から,
    任意の添字$j$について
    $R_{V_i \cup V_j \to U}^{V_i \to U} \colon \fibF(V_i \to U) \to \fibF(V_i \cup V_j \to U)$が圏同値だと分かる.
    この関手を略して$R$を書くことにする.
    \[
    \begin{tikzcd}[row sep=10mm]
        \fibF(U)
        \ar[r, leftrightarrow, "\epsilon_{V_i \cup V_j \to U}"]
        \ar[d, leftrightarrow, "\epsilon_{V_i \to U}"']& 
        \fibF(V_i \cup V_j \to U) \ar[ld, red, "R\,:=R_{V_i \cup V_j \to U}^{V_i \to U}"]\\
        \fibF(V_i \to U)
    \end{tikzcd}
    \]
    したがって$R^{-1}$が存在する.
    関手$R$は
    $\fibF(V_i \cup V_j \to U)$の射$\beta|_{V_i \cup V_j}$を$\beta|_{V_i}$に一対一に写すのだから,
    $R^{-1}$は$\beta|_{V_i}$を$\beta|_{V_i \cup V_j}$に一対一に写す.

    さて,以下の関手の合成で$\beta, \beta'$を写す.
    \[
    \begin{tikzcd}[column sep=20mm]
        \fibF(V \to U) \ar[r, "R_{V \to U}^{V_i \to U}"]&
        \fibF(V_i \to U) \ar[r, "R^{-1}"]&
        \fibF(V_i \cup V_j \to U) \ar[r, "R_{V_i \cup V_j \to U}^{V_j \to U}"]&
        \fibF(V_j \to U) 
    \end{tikzcd}
    \]
    $\beta|_{V_i}=\beta'|_{V_i}$を$R^{-1}$で写して
    $\beta|_{V_i \cup V_j}=\beta'|_{V_i \cup V_j}$が得られる.
    よって,任意の$j$について
    \[ \beta|_{V_j}=(\beta|_{V_i \cup V_j})|_{V_j}=(\beta'|_{V_i \cup V_j})|_{V_j}=\beta'|_{V_j} \]
    が成立する.
    $\fibF$ :: Zariski stackなので,$\beta=\beta'$.

\subsection{Step 5 / 一般の場合.}
    条件$(*)$を満たす任意の射$f \colon V \to U$をとり,
    affine Zarisiki cover :: $\{U_i \to U\}$をとる.
    さらに$V_i:=f^{-1}(U_i)$とおき,$\phi_i=f|_{V_i}$とおく.

    今,
    \[ \Phi_i=\epsilon_{V_i \to U_i} \colon \fibF(U_i) \to \fibF(V_i \to U_i) \]と置く.
    同様に$\Phi_{ij}=\epsilon_{V_{ij} \to U_{ij}}, \Phi_{ijk}=\epsilon_{V_{ijk} \to U_{ijk}}$と置く.
    この時,以下は厳密な可換図式である.
    \[
    \begin{tikzcd}
        \fibF(U_i) \ar[r, "\res"]\ar[d, "\Phi_{i}"']&
            \fibF(U_{ij}) \ar[d, "\Phi_{ij}"]\ar[r, "\res"] & \fibF(U_{ijk}) \ar[d, "\Phi_{ijk}"]\\
        \fibF(V_i \to U_i) \ar[r, "\res"']& \fibF(V_{ij} \to U_{ij}) \ar[r, "\res"']& \fibF(V_{ijk} \to U_{ijk})
    \end{tikzcd}
    \eqno{(4)}
    \]
    ここで,各$\Phi_{*}$はいずれもStep 4から圏同値である.

    次の関手を考える.
    \begin{defmap}
        P_{i}\colon & \fibF(f \colon V \to U)& \to& \fibF(V_i \to U_i) \\
        \mathbf{Objects}& (\eta, \sigma)& \mapsto& (\eta|_{V_i}, (\gamma_{ii})^*\sigma) \\
        \mathbf{Arrows}& \alpha& \mapsto& \alpha|_{V_i} \\
    \end{defmap}
    同様に$P_{ij} \colon \fibF(f) \to \fibF(V_{ij} \to U_{ij})$を定義する.
    するとstep 4の結果から$\fibF(U_i) \simeq \fibF(V_i \to U_i)$なので,
    $\xi_i \in \fibF(U_i)$と同型
    \[ \alpha_i \colon \Phi_i(\xi_i) \isomap P_i((\eta, \sigma)) \]
    が得られる.
    上の図式(4)が可換であることから,
    \[
        \alpha_i|_{V_{ij}} \colon
        \Phi_{ij}(\xi_i|_{V_{ij}})=(\Phi_i(\xi_i))|_{V_{ij}} \isomap P_{ij}((\eta, \sigma))
    \]
    すると,
    \[ \alpha_i^{-1}\alpha_j \colon \Phi_{ij}(\xi_j|_{V_{ij}}) \to \Phi_{ij}(\xi_i|_{V_{ij}}) \]
    $\Phi_{ij}$ :: equivalenceなので,
    この同型射の逆像として$\sigma_{ij} \colon \xi_j|_{V_{ij}} \to \xi_i|_{V_{ij}}$が得られる.
    
    以上で得られる$(\{\xi_i\}, \{\sigma_{ij}\})$はcocycle conditionを満たすため(TODO),
    $\fibF(\{U_i \to U\})$のobjectである.
    $\fibF$ :: Zariski stackなので,これは$\fibF(U)$と圏同値.
    よって$\xi$が得られる.
    (TODO: 射についても)

\section{Application : \tp{$\QCoh/S \to \Sch/S$}{QCoh/S -> Sch/S} is a fpqc stack.}

    \begin{Def}
    $S \in \Sch$に対し,圏$\QCoh/S$を以下のように定める.
    \begin{description}
        \item[Objects.] \mnewline
            $\mathrm{Fpqc}(S)$
            \footnote{圏$\Sch/S$にfpqc topologyを備えたもの.}の対象 :: $U$と,
            $U$上のquasi-coherent sheaf (on fpqc topology):: $\shU$の組.
        \item[Arrows.] \mnewline
            射$(U, \shU) \to (V, \shV)$は,
            $\Sch/S$の射$f \colon U \to V$と,
            morphism of sheaves on $V$ :: $f^{\#} \colon \shV \to f_*\shU$の組.
    \end{description}
    この時,$\QCoh/S \to \Sch/S; \ (U, \shU) \mapsto U$はfibrationである.
    \end{Def}

%    \begin{Def}
%        $\phi \colon A \to B$を環準同型とする.
%        $A$-module :: $M$について,
%        \[
%            c_M \colon M \otimes_A B \to B \otimes M;\ m \otimes b \mapsto b \otimes m,\qquad 
%            p_M \colon M \to B \otimes M;\ m \mapsto 1_B \otimes m
%        \]
%        と定義する.

%        この時,$\Mod_{\phi}$を以下のように定義する.
%        \begin{description}
%            \item[Objects.] $A$-module :: $N$と同型$\psi \colon N \otimes B \to B \otimes N$であって,
%            \item[Arrows.]
%        \end{description}
%        さらに,$\Mod_A \to \Mod_{\phi}$を以下のように定義する.
%        \begin{defmap}
%            \epsilon_{\phi}\colon & & \to&  \\
%            \mathbf{Objects}& & \mapsto& \\
%            \mathbf{Arrows}& & \mapsto& 
%        \end{defmap}
%    \end{Def}
    
    $\Mod_A, \Mod_{\phi}, \Mod_A \to \Mod_{\phi}$の定義は
    \cite{NoteGroTop} \S4.2.1を参照せよ.

    $f \colon V \to U$をflat surjective morphism of $S$-schemesとし,
    $\phi \colon A \to B$を$f$に対応するfaithfully flatな環準同型とする.
    この時,$\QCoh(U) \simeq \Mod_A$はよく知られている
    \footnote{この命題は\cite{HarAG} Cor5.5で詳しく述べられている}.
    \begin{Claim}
        $\QCoh(f \colon V \to U) \simeq \Mod_{\phi}$.
    \end{Claim}
    したがって$\epsilon_{f} \colon \QCoh(U) \to \QCoh(f \colon V \to U)$は
    関手$\Mod_A \to \Mod_{\phi}$に対応する.
    この関手は,可換環論によって圏同値であることが証明される.

\bibliographystyle{jplain}
\bibliography{reference}
\end{document}
