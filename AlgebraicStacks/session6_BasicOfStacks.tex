\documentclass[a4paper]{jsarticle}
\usepackage{macros}

\newcommand{\HOM}{\operatorname{HOM}}
\newcommand{\inj}{\mathrm{inj}}

\begin{document}
\title{ゼミノート \#6 \\ Basic of Stacks}
\author{七条彰紀}
\maketitle
\tableofcontents

以下,特に改めて指定がなければ
$\cat{C}$ :: site, 
$\pi \colon \fibF \to \cat{C}$ :: fibered categoryを考える.

\section{Definition : Prestack / Stack }
\begin{Def}[Prestack, Stack]
    関手$\epsilon_{\covU} \colon \fibF(U) \to \fibF(\covU)$を用いて以下のように定義する.
    \begin{enumerate}[label=(\roman*)]
    \item
        任意の$U \in \cat{C}, \covU \in \Cov(U)$について$\epsilon_{\covU}$ :: fully faithfullである時,
        fibered category $\fibF \to \cat{C}$はprestackである,という.
    \item
        任意の$U \in \cat{C}, \covU \in \Cov(U)$について$\epsilon_{\covU}$ :: equivalenceである時,
        fibered category $\fibF \to \cat{C}$はstackである,という.
    \end{enumerate}

    (pre)stacksの間の射は,fibered categoryとしての射である.
\end{Def}

\begin{Remark}
    prestackの定義は以下のように言い換えられる:
    任意の$U \in \cat{C}, \covU=\{\phi_i \colon U_i \to U\} \in \Cov(U)$をとる.
    さらに$\xi, \eta \in \fibF(U)$をとり,$\epsilon_{\covU}$による像を
    \[
        \epsilon_{\covU}(\xi)=(\{\xi_i\}, \{\sigma_{ij}\}),\quad
        \epsilon_{\covU}(\eta)=(\{\eta_i\}, \{\tau_{ij}\})
            \in \fibF(\covU) 
    \]とする.
    この時,任意の射
    $\{\alpha_i\} \colon \epsilon_{\covU}(\xi) \to \epsilon_{\covU}(\xi)$
    (射$\alpha_i \colon \xi_i \to \eta_i$の集合)について,
    $\fibF(U)$の射$\alpha \colon \xi \to \eta$が\underline{一意に存在し},
    $\alpha_i=(\phi_i)^*\alpha (\iff \{\alpha_i\}=\epsilon_{\covU}(\alpha))$となる.

    標語的に言えば,prestackは「貼り合わせられる射を持つpsuedo-functor」となる.
    同型射の貼り合わせは同型射であるから,
    prestackは「貼り合わせが(存在すれば)一意な対象を持つpsuedo-functor」である.
\end{Remark}

\begin{Remark}
    このノートでは,
    fiberが条件を満たすfibered categoryとして(pre)stackは定義されている
    (fiberを用いずに(pre)stackを定義することも出来るが,今回は採用しなかった).
    なので形式上,(pre)stackはfibered categoryを経由せず,特別なpsuedo-functorとして定義できる.
    しかし実際にそのように定義されることは少ない.

    ではpsuedo-functorとして定義しない積極的な理由はと言うと,
    実用上,元のfibered categoryにも言及する場合が多いからであると思われる.
    fiberだけでなく元のfibered categoryに言及する理由については,
    このセミナーのノート session 4.5
    \footnote{URL : \url{https://github.com/ShitijyouA/MathNotes/blob/master/AlgebraicStacks/session4_5_FiberedCategoriesContinued.pdf}},
    注意2.8を参考にして欲しい.
\end{Remark}

\begin{Def}[Sub(pre)stack]
    stack :: $\pi \colon \fibF \to \cat{C}$のsub(pre)stackとは,
    $\fibF$の部分圏$\fibG$であって,
    $\pi$と包含関手の合成$\fibG \inclmap \fibF \xrightarrow{\pi} \cat{C}$がfibrationであり,
    さらにそのfiberが(pre)stackであるもの.
\end{Def}

\section{Example : Stack}
\begin{Prop}[\cite{NoteGroTop} Prop4.9]
    \begin{myenum}
        \item separated presheaf of sets is a prestack.
        \item sheaf of sets is a stack.
    \end{myenum}
\end{Prop}
\begin{proof}
    $\cat{C}$ :: site, $\fibF \colon \cat{C}^{op} \to \Sets$ :: presheafとする.
    $U \in \cat{C}, \covU=\{U_i \to U\} \in \Cov(U)$を任意に取る.

    今,圏$\fibF(U), \fibF(\covU)$は集合(離散圏)である.
    なので関手$\epsilon_{\covU} \colon \fibF(U) \to \fibF(\covU)$は\underline{写像}である.
    さらに射$\sigma_{ij}$も恒等射しかないから,
    $\fibF(\covU)$の対象は,
    任意の$i,j$について$\xi_i|_{U_{ij}}=\xi_j|_{U_{ij}}$を満たす
    $\xi_i \in \fibF(U_i)$の族$\{\xi_i\}_i$であると考えて良い.
    このセミナーノートのsession3の記号を用いれば,
    $\fibF(\covU)=H^0(\covU, \fibF)$ということに成る.

    二つのデータ$\{\xi_i\}, \{\eta_i\}$の間の射もやはり恒等射しかないから,
    「関手$\epsilon_{\covU}$がfully faithfulである」という仮定は
    「写像$\epsilon_{\covU}$が単射である」と言い換えられる.
    これはすなわち,$\fibF$がseparated presheafであるということである.

    「関手$\epsilon_{\covU}$がessentially surjectiveである」という仮定は
    「写像$\epsilon_{\covU}$が全射である」と言い換えられるから.
    $\epsilon_{\covU}$がequivalenceであることは
    $\fibF(\covU)=H^0(\covU, \fibF)$と$\fibF(U)$の間に全単射が存在するということである.
    これはすなわち,$\fibF(U)$がsheafであるということである.
\end{proof}

\begin{Remark}
    この命題で分かるとおり,
    prestackはpresheafの抽象化ではなく,separated presheafの抽象化である.
    そうすると,
    我々はpsuedo-functor $\cat{B}^{op} \to \Cat$をprestackと呼び,
    今prestackと呼んでいるものはseparated prestackと呼ぶべきなのかも知れない.
    我々がそうしないのは,後に定義される``separated stack"との混乱を避けるためである.
\end{Remark}

以下の二つの例は後にセミナーでも証明を扱う.

\begin{Example}
    $X \in \cat{C}$に対し,圏$\cat{Shv}/X$を以下のように定める.
    \begin{description}
        \item[Objects.] \mnewline
            $X$への射を持つような$\cat{C}$の対象 :: $U$と,
            $U$上のsheaf :: $\shU$の組.
        \item[Arrows.] \mnewline
            射$(U, \shU) \to (V, \shV)$は,
            $\cat{C}$の射$f \colon U \to V$と,
            morphism of sheaves on $V$ :: $f^{\#} \colon \shV \to f_*\shU$の組.
    \end{description}
    この時,fibered category :: $\cat{Shv}/X \to \cat{C}/X; \ (U, \shU) \mapsto U$はstackである.
    この例で考えるsheafをquasi-coherent sheafに制限してて得られる
    fibered category :: $\cat{QCoh}/X \to \cat{C}/X$もstackである.
    この二つの例については,このセミナーでも後に証明を扱う.
\end{Example}

\begin{Example}
    $X \in \Sch$に対し,圏$\cat{QCoh}/X$を以下のように定める.
    \begin{description}
        \item[Objects.] \mnewline
            $\mathrm{Fpqc}(X)$
            \footnote{圏$\Sch/X$にfpqc topologyを備えたもの.}の対象 :: $U$と,
            $U$上のsheaf (on fpqc topology):: $\shU$の組.
        \item[Arrows.] \mnewline
            射$(U, \shU) \to (V, \shV)$は,
            $\cat{C}$の射$f \colon U \to V$と,
            morphism of sheaves on $V$ :: $f^{\#} \colon \shV \to f_*\shU$の組.
    \end{description}
    この時,fibered category :: $\cat{QCoh}/X \to \cat{C}/X; \ (U, \shU) \mapsto U$はstackである.
\end{Example}

\begin{Example}[\cite{ASS} 4.4.1]
    以下で定まるfibered categoryはstackである.
    \begin{defmap}
    {}& \left\{\parbox{4cm}{\begin{center} pair of scheme over $S$ :: $Y$ \\ and closed imm. $W \inclmap Y$ \end{center}}\right\}& \to& \mathrm{Fppf}(S) \\
        {}& (Y, W \inclmap Y)& \mapsto& Y
    \end{defmap}
\end{Example}

\begin{Example}[\cite{ASS} 4.4.4]
    以下で定まるfibered categoryはstackである.
    \begin{defmap}
    {}& \left\{\parbox{4cm}{\begin{center} pair of scheme over $S$ :: $Y$ \\ and open imm. $W \inclmap Y$ \end{center}}\right\}& \to& \mathrm{Fppf}(S) \\
        {}& (Y, W \inclmap Y)& \mapsto& Y
    \end{defmap}
\end{Example}

以下の二つの例は後に一般化される.

\begin{Example}[\cite{NoteGroTop} \S 4.3.1]
    arrow category :: $\Sch^{\rightarrow}$の対象を
    affine morphismに制限したものを圏$\cat{Aff}$とする.
    以下で定まるfibered categoryはstackである.
    \begin{defmap}
        {} & \cat{Aff}& \to& \mathrm{Fppf}(\Spec \Z) \\
        {}& [ X \to Y ]& \mapsto& Y
    \end{defmap}
\end{Example}

\begin{Example}[\cite{ASS} 4.4.15]
    quasi-compact open imbeddingの後にaffine morphismを合成した射のことを
    quasi-affine morphismという.
    arrow category :: $\Sch^{\rightarrow}$の対象を
    quasi-affine morphismに制限したものを$\cat{QAff}$とする.
    以下で定まるfibered categoryはstackである.
    \begin{defmap}
        {} & \cat{QAff}& \to& \mathrm{Fppf}(\Spec \Z) \\
        {}& [ X \to Y ]& \mapsto& Y
    \end{defmap}
\end{Example}

\section{Proposition : Stack}

\begin{Prop}[\cite{ASS} Prop4.12]
    二つのequivalentなfibered categoryがあり,
    かつ一方がstackならば,もう一方もstackである.
\end{Prop}
\begin{proof}
    $\fibF, \fibG$ :: fibered categories over $\cat{C}$とし,
    $F \colon \fibF \to \fibG$ :: morphism of fibered categoriesとする.
    この時cover of $U \in \cat{C}$ :: $\covU=\{U_i \to U\}$について$F_{\covU}$を定義する.
    \begin{defmap}
        F_{\covU}\colon & \fibF(\covU)& \to& \fibG(\covU) \\
        \mathbf{Objects:}& (\{\xi_i\}, \{\sigma_{ij}\})& \mapsto& (\{F\xi_i\}, \{F\sigma_{ij}\}) \\
        \mathbf{Arrows:}& \{\alpha_i\}& \mapsto& \{F\alpha_i\} \\
    \end{defmap}
    更に二つの射$F, G \colon \fibF \to \fibG$と
    その間のbase-preserving natural transformation :: $\rho \colon F \to G$に対し,
    $\rho_{\covU} \colon F_{\covU} \to G_{\covU}$を次のように定義する.
    \[ (\rho_{\covU})_{(\{\xi_i\}, \{\sigma_{ij}\})}=\{\rho_{\xi_i}\}. \]

    以上から,$F$がequivalenceならば$F_{\covU}$もquivalenceである.
    したがって以下のcommutative diagram of weak $2$-category
    \footnote{ 射の合成の間にnatural isomorphismが存在するという意味で可換. }
    が得られる.
    \[\xymatrix{
            \fibF(U) \ar[r]^-{\epsilon_{\covU}}\ar[d]_-{F}& \fibF(\covU) \ar[d]^-{F_{\covU}}\\
            \fibG(U) \ar[r]_-{\epsilon_{\covU}}& \fibF(\covU)
    }\]
    この可換図式から,主張が得られる.
\end{proof}

\begin{Prop}[\cite{ASS} Exc 4.I]
    $\fibF, \fibF'$ :: stack on $\cat{C}$,
    $f \colon \fibF \to \fibF'$ :: morphism of stacksとする.
    $f$ :: isomorphismは以下の2条件が成立することと同値.
    \begin{enumerate}[label=(\alph*)]
        \item
            任意の$X \in \cat{C}$について,
            fiberの間の射$f_X \colon \fibF(X) \to \fibF'(X)$はfully-faithful.
        \item
            任意の$X \in \cat{C}$と$x \in \fibF'(X)$について,
            covering of $X$ :: $\{\phi_i \colon X_i \to X\} \in \Cov(X)$が存在し,
            全ての$x$のpullback :: $\phi_i^*x \in \fibF'(X_i)$が$\fibF(X_i)$のessential imageに属す.
    \end{enumerate}
\end{Prop}
\begin{proof}
    (TODO)
\end{proof}

\begin{Lemma}
    site :: $\cat{C}$を,
    空集合のcoverとして空集合を持つ($\emptyset \in \Cov(\emptyset)$)ものとする.
    $\pi \colon \fibF \to \cat{C}$ :: stackについて,以下の圏同値が成立する.
    \[ \fibF( \emptyset ) \simeq \cat{1}. \]
    特に,$\fibF(\emptyset)$の任意の二つの対象の間には,ただ一つの同型射が存在する.
\end{Lemma}
\begin{proof}
    category of descent data :: $\fibF(\covU)$の対象を考える.
    これは$\covU$で添字付けられた対象の族の二つ組である.
    なので$\covU=\emptyset$について,
    $\fibF(\emptyset)$の対象は$(\emptyset, \emptyset)$しかない.
    射も$\covU$で添字付けられた族であるから,非自明な射は存在しない.
\end{proof}

この補題の仮定は奇妙に見えるかも知れないが,
以下の通り,このように仮定しても問題はないし,
我々が扱う殆どのsiteはこの仮定を満たす.

\begin{Claim}
    圏$\cat{C}$の任意の対象$U \in \cat{C}$について,
    命題「$\emptyset \in \Cov(U)$」はGrothendieck topologyの公理(定義)と独立である.
    すなわち,$\emptyset \in \Cov(U)$としてもしなくても矛盾は生じない.
\end{Claim}
\begin{proof}
    命題「$\emptyset \in \Cov(U)$」を$P$と書く.
    Grothendieck topologyの定義を見直そう.
    cover of $\emptyset$ :: $\covU \in \Cov(U)$が満たすべき条件を記号で書き下す.
    \begin{enumerate}[label=(\alph*)]
        \item
            $\Forall{[V \to U] \in \cat{C}/U}
            \lbra{ \Forall{[U' \to U] \in \covU} {}^{\exists} U' \times_{U} V }
            \implies \{ U' \times_{U} V \to V \mid [U' \to U] \in \covU \} \in \Cov(V)$.

        \item $\Forall{\covV:=\{ \covU'_{U'} \mid \covU'_{U'} \in \Cov(U')\}_{U' \in \covU}}
                \{ U'' \to U' \to U \mid
                    [U' \to U] \in \covU, [U'' \to U'] \in \covU'_{U'} \} \in \Cov(U)$.
    \end{enumerate}
    クラス$X$と述語$F$について``$\Forall{x \in X} F(x)$"という文は
    ``$\Forall{x} \lbra{x \in X \implies F(x)}$の省略形である.
    したがって,$X=\emptyset$であるとき,
    ``$\Forall{x \in X} F(x)$"という文は任意の$F$について真.
    また,$\{ f(x) \mid x \in \emptyset \}=\emptyset$.

    なので,以上の文を$\covU=\emptyset$の場合に考えると(すなわち$P$を仮定すると),
    いずれも$P$と同値に成る.
    よって$P \implies P$ということになる.
    一方,否定$\lnot P$を仮定しても矛盾が生じないことは明らか.
\end{proof}

\begin{Example}
    圏$\cat{C}$を$\Sch$の部分圏や$\Sch/S$\ ($S$ :: scheme)とする.
    morphism of schemesのクラス$\mathcal{P} \subset \Arr(\cat{C})$をとり,
    以下のように$\cat{C}$上の$\Cov$を定めたとする:
    \begin{align*}
        \Cov(U)
        =&\{
            \covU \mid
            \covU \text{ :: jointly surjective family and }
            \Forall{\phi \in \covU} \phi \in \mathcal{P}
        \} \\
        =&\left\{
            \covU ~\middle|~
            \bigsqcup_{U' \in \covU}U' \to U \text{ :: surjective}
            \mand
            \Forall{\phi} \lbra{ \phi \in \covU \implies \phi \in \mathcal{P} }
        \right\}.
    \end{align*}
    この時,$\bigsqcup_{U' \in \emptyset}U'=\emptyset$なので$\emptyset \in \Cov(\emptyset)$.

    このセミナーで定義したZariski site, etale site, ...などは
    全てこの主張のように定義されている.
\end{Example}

\begin{Lemma}
    圏$\cat{C}$を$\Sch$の部分圏や$\Sch/S$\ ($S$ :: scheme)とする.
    $U \in \cat{C}, \{U_i \to U\} \in \Cov(U)$をとり,
    $V=\bigsqcup_{i} U_i$と置く.
    
    $\left\{ U_i \to V \right\} \in \Cov(V)$と仮定する
    \footnote{ 例えば,Zariski topologyより細かい位相ならばこの仮定は成立する.}と
    $\pi \colon \fibF \to \cat{C}$ :: stackについて,
    圏同値(TODO: strict $2$-equivalence? ここは$\epsilon$と圏同型の合成)
    \[ \fibF \left( \bigsqcup_i U_i \right) \simeq \prod_i \fibF(U_i) \]
    が成立する.
\end{Lemma}
\begin{proof}
    瑣末なことでは有るが: 
    $\{U_i \to V\}$の添字について,
    $i \neq j$ならば$U_i \neq U_j$である,と仮定して一般性を失わない.

    $\covU=\{ \inj_i \colon U_i \to V \}(\in \Cov(V))$と置く.
    次の関手が圏同値であることを示す.
    \begin{defmap}
        E \colon& \im \epsilon_{\covU}& \to& \prod_i \fibF(U_i) \\
        \mathbf{Objects}\colon&
            (\{(\inj_i)^*\xi\}_i, \{\sigma_{ij}\}_{i, j})& \mapsto& ((\inj_i)^*\xi)_i\\
        \mathbf{Arrows}\colon& \{\alpha_i\}_i& \mapsto& (\alpha_i)_i
    \end{defmap}
    $\xi_i:=(\inj_i)^*$とおく.
    これが示せれば,$\fibF$ :: stackなので主張も得られる.

    まず,仮定の状況では,
    injection map (coprojection) :: $U_i \to V$についてのfiber productは
    次のように成る.
    \[
        U_{ij}
        =U_i \times_V U_j =
        \begin{cases}{}
            U_i & (U_i=U_j) \\
            \emptyset & (U_i \neq U_j).
        \end{cases}
    \]
    
    \step{$E$ :: essentially surjective.}
    $i \neq j$の時,
    $\sigma_{ij}$は$\fibF(U_{ij})=\fibF(\emptyset)$の射であるから,
    $\xi_i|_{\emptyset}$と$\xi_j|_{\emptyset}$の間に存在するただ一つの射である.
    一方,$i=j$の時は,
    $(\pr_i)^*(\inj_i)^*\xi$と$(\pr_j)^*(\inj_j)^*\xi$が完全に等しいので,
    $\sigma_{ij}=\id[\xi_i]$.
    以上より,対象同士の次の対応は$\mathrm{Ob}(E)$の逆対応と成る.
    \begin{defmap}
        {} & \mathrm{Ob}\left( \prod_i \fibF(U_i) \right)& \to&
            \mathrm{Ob}\left( \im \epsilon_{\covU} \right) \\
        {}& (\xi_i)_i& \mapsto&
            (\{\xi_i\}_i, \{\sigma_{ij}\}_{i \neq j} \cup \{\id[\xi_i]\}_{i}) \\
    \end{defmap}
    これは$E$ :: essentially surjectiveを意味する.

    \step{$E$ :: fully-faithfull.}
    また,上で述べたことから,
    射$\{\alpha_i\} \colon \epsilon_{\covU}(\xi) \to \epsilon_{\covU}(\eta)$に課された条件が,
    任意の射の組み合わせ$(\alpha_i \colon \xi_i \to \eta_i)_i$について成立することも分かる
    ($i \neq j$と$i=j$の場合で証明は異なる).
    なので全単射
    \[
        \Hom(\epsilon_{\covU}(\xi), \epsilon_{\covU}(\eta))
        =\Hom((\xi_i)_i, (\eta_i)_i)
        =\Hom(E(\epsilon_{\covU}(\xi)), E(\epsilon_{\covU}(\eta)))
    \]
    が得られる.
    これは$E$ :: fully-faithfullを意味する.
\end{proof}

\begin{Thm}[Stackification of category fibered by groupoids.]
    $\cat{C}$ :: site,
    $\fibF$ :: category fibered by groupoids over $\cat{C}$とする.
    この時,
    $\bar{\fibF}$ :: stack in groupoids over $\cat{C}$と
    $\theta \colon \fibF \to \bar{\fibF}$ :: morphism of fibered categoryが存在し,
    \[ (-) \circ \theta \colon \HOM_{\cat{C}}(\bar{\fibF}, -) \to \HOM_{\cat{C}}(\fibF, -) \]
    が圏同値となる.
\end{Thm}
(TODO: これは\cite{ASS} Thm4.6.5からとった.
しかし,\url{https://stacks.math.columbia.edu/tag/02ZM}に一般の場合が書かれている.
出来ればこちらを理解したい)

\begin{Example}
    presheafのstackificationはsheafificationと一致する.
\end{Example}

\begin{Example}[arXiv:math/0305243v1, Prop3.6]
    $S$ :: scheme,
    $\shM$ :: algebraic stack over $\Sch/S$,
    $\fibG$ :: sheaf in groups over $\Sch/S$, acting on $\shM$とする.
    この時,$\shM$の$\fibG$によるcategorical quotient :: $\shM/\fibG$は,
    以下のprestack($2$-functorとして定義する):: $\shP$ のstackificationとして定義される.
    \begin{description}[labelindent=1cm]
        \item[Objects of $\shP(U)$.] $\shM(U)$の対象と同じ.
        \item[Arrows of $\shP(U)$.]  $g \in \fibG(T)$と$\shM(U)$の射$g \ast x \to y$の組.
    \end{description}
    ただし$U \in \Sch/S$は任意.
\end{Example}

\bibliographystyle{jplain}
\bibliography{reference}
\end{document}
