\documentclass[a4paper]{jarticle}
\usepackage{../math_note, braket}

\begin{document}
\section{Definition of regular function}
    \begin{Def}[regular function on quasi-affine variety $Y$] \mbox{} \\
        $f: Y \to k$がregular functiona at a point $P \in Y$とは,以下の論理式が成り立つ事.
        \[ \Exists{U \in \mathcal{O}_Y} (P \in U) \land (\Exists{g,h \in A} (h|_U \neq 0) \land (f|_U=g/h) ) \]
    \end{Def}

    \begin{Def}[regular function on quasi-projective variety $Y$] \mbox{} \\
        $f: Y \to k$がregular functiona at a point $P \in Y$とは,以下の論理式が成り立つ事.
        \[ \Exists{U \in \mathcal{O}_Y} (P \in U) \land (\Exists{g,h \in S^h} (\deg g=\deg h) \land (h|_U \neq 0) \land (f|_U=g/h) ) \]
    \end{Def}

    多様体の間の写像としてregular mapが後に定義される.

\section{About Remark 3.1.1}
    \begin{Lemma}
        $Y (\subset \proj^n)$ :: quasi-projective variety,
        $f: Y \to \proj^1_k$ :: regular function on quasi-projective variety $Y$
        $\implies$
        $f$ :: continus
    \end{Lemma}
    \begin{proof}
        そのためにLemma 3.1と同じ方針で証明をする.
        閉集合が閉集合へ写ることを示す.
        $\proj^1_k$の閉集合は有限集合だから,一点集合が閉集合に写ることだけ見れば良い.
        ある開集合$U$で$f$が
        \[ f|_U=g/h ~~ (g,h \in S^h, \deg g=\deg h, g|_U \neq 0) \]とあらわせたとしよう.
        $a \in f(U) \subseteq \proj^1_k$を任意にとると,
        \[ f^{-1}(a) \cap U=\set{ P \in U | g(P)/h(P)=a }=\zerosp(g-ah) \]
        となる.
        $g-ah$は$g,h$が斉次かつ$\deg g=\deg h$かつ$a$が定数だから斉次である.
        \footnote{この一文だけがaffineの場合と異なる.他の部分はaffineと全く同じ.}
        よって$f^{-1}(a) \cap U$はclosed set.
    \end{proof}

    \begin{Lemma}
        $X$ :: variety, $U$:: open in $X$, $f, g: X \to k$ :: regular on $U$とする.
        このとき$f=g$ on $U$ならば$f=g$ on $X$.
    \end{Lemma}
    \begin{proof}
    今,$U$において$f,g$が$f=f_0/f_1, g=g_0/g_1$とあらわせたとしよう.
    ($U$より大きい集合で$f=f_0/f_1$となっていれば単に制限する.)
    すると$h=f-g=\frac{f_0 g_1-g_0 f_1}{f_1 g_1}$もまた$U$上のregular function.
    したがって$h$は連続である.
    $k=\affine^1_k$はT1空間だから$\{0\}$は閉集合で,$h$は連続.
    よって$A:=h^{-1}(\{0\})=\{P \in X ~|~ f(P)=g(P)\}$は閉集合.
    明らかに$U \subseteq A \subseteq X$となっている.
    $U$は$X$ :: irreducible setの開部分集合なので,Exsercise 1.6より,$U$はdense.
    したがって$\cl_{X}(U)=X$.$A$は閉集合であるから,$X \subseteq A \subseteq X$すなわち$A=X$が得られる.
    \end{proof}


\section{p.16 Definition}
    \subsection{ring of regular functions on $Y$ : $\mathcal{O}(Y)$}
    $\mathcal{O}(Y)$は$Y$全体でregularな関数全体である.
    つまり,任意の点$P \in Y$について,ある$P$の開近傍$U$が存在し,
    $f=g_U/h_U \mbox{ on }U, h_U \neq 0$であるような$g_U,h_U \in S$を見つけられるもの.
    $g_U,h_U$は点$P$と開近傍$U$に依存することに注意.
    $Y$全体で正則な有理関数を全て含むが,それよりも大きな集合になりうる.

    $\mathcal{O}(Y)$がringであることをみる.
    まず$0, 1$は明らかにこの集合に属す.
    $f,g \in \mathcal{O}(Y)$をとると,
    適当な点$P$とその開近傍$U$について$f=\frac{f_0}{f_1}, g=\frac{g_0}{g_1}$という表示がある.
    以下の計算から,$fg, f+g$も$U$でregular.
    \[ fg=\frac{f_0 g_0}{f_1 g_1}, f+g=\frac{f_0 g_1+g_0 f_1}{f_1 g_1} \mbox{ on } U \]
    点$P \in Y$は任意に取ったので,$fg, f+g$は$Y$でregular.

    \subsection{local ring of $P$ : $\mathcal{O}_P$}
    \paragraph{Definition}
    $\mathcal{O}_P$は$P$でregularな関数全体で作られるring.
    中身は$P$の開近傍$U \subset Y$と,quasi-variety $U$の各点でregularな関数$f$の組で,
    \[ \langle U,f \rangle \equiv \langle V,g \rangle \iff f=g \mbox{ on }U \cap V \iff (f-g)|_{U \cap V}=0 \]
    という関係で割っている.
    $f$から$U$が定まるのではなく,その逆である.
    $Q \in U$に対する$f$の有理関数表示が変わるかも知れない.

    \paragraph{check to be equivalence relation.}
    $\equiv$が同値関係であることは次のように分かる.
    まず反射律と対称律は明らか.
    推移律はRemark 3.1.1の
    \[ \langle U,f \rangle \equiv \langle V,g \rangle \implies f=g \mbox{ on } Y  \]
    より直ちに分かる.$U \cap V \neq \emptyset$は$P \in U, V$より直ちに分かる.
    この同値関係で割らなければ,$\langle U,f \rangle$と$\langle V,f \rangle$は異なるものになる.

    \paragraph{check to be ring}
    この集合がringであることを示す.
    \[ a=\langle U,f \rangle, b=\langle V,g \rangle \]
    とすると,$ab, a+b$の計算は次のようになる.
    まず$Q \in U \cap V$を適当にとると,開近傍$(Q \in )W, (Q \in )Z$において
    $f,g$はそれぞれ$f=f_0/f_1, g=g_0/g_1$と書ける.
    $W,Z$はirreducible set $Y$の開部分集合だから$W \cap Z \neq \emptyset$.
    \[ fg=\frac{f_0 g_0}{f_1 g_1}, f+g=\frac{f_0 g_1+g_0 f_1}{f_1 g_1} \mbox{ on } W \cap Z \]
    点$Q \in U \cap V$毎に開近傍$W \cap Z$がとれ,そこで$fg, f+g$がregularなのでringになっている.
    
    \paragraph{check for well-definedness}
    演算がwell-definedであることを示す.
    二つの同値類$A, B$をとる.
    それぞれの代表元を任意に2つずつとり,
    \[ a=\langle U,f \rangle, a'=\langle U',f' \rangle \in A;~ b=\langle V,g \rangle, b'=\langle V',g' \rangle \in B\]
    としよう.
    例えば$ab, a'b'$を計算すると,
    \[ ab=\langle U \cap V,fg \rangle, a'b'=\langle (U' \cap V'),f'g' \rangle \]
    だが,$f=f' \mbox{ on } U \cap U'$, $g=g' \mbox{ on } '$より
    $fg=f'g' \mbox{ on } U \cap U' \cap V \cap V=(U \cap V) \cap (U' \cap V')$.
    よって$ab=a'b'$.
    $a+b \equiv a'+b'$なども同様である.
    
    \paragraph{check to be a local ring}
    $\mathcal{O}_P$は$\I{m}=\{ \langle U,f \rangle ~|~ f(P)=0 \}$を唯一の極大イデアルとする局所環.
    実際,$\mathcal{O}_P$の非単元はすべて$\I{m}$に属す.これは$f(P) \neq 0$ならば適当な$P$の近傍の中で$1/f$が定義出来ること(対偶)から示される..

    \paragraph{check to be a integral domain}
    また,$\mathcal{O}_P$は整域である.
    実際,$fg=0$ on $U \cap V$であるような$\langle U,f \rangle, \langle V,g \rangle \in \mathcal{O}_P$をとる.
    $P \in U, V$より$U \cap V \neq \emptyset$.
    すると,$f=f_0 / f_1, g=g_0 / g_1$ on $U \cap V$となるような$f_0, f_1, g_0, g_1 \in S$が存在する.
    $f_1, g_1$は$U \cap V$で0とならないから,結局$f_0 g_0$ on $U \cap V$となっている.
    \[ \Forall{P \in U \cap V} (f_0 g_0)(P)=0 \in k  \iff \Forall{P \in U \cap V} f_0(P)=0 \lor g_0(P)=0\]
    よって$f=0$ on $U \cap V$または$g=0$ on $U \cap V$が成立.Remark3.1.1より,これは,$f=0$ on $U$または$g=0$ on $V$,と書き換えられる.

    \subsection{function field : $K(Y)$}
    $K(Y)$は$Y$内のいずれかの点でregularな関数全体である.
    中身は$\mathcal{O}_P$のものと同様である.
    これは任意の$\langle U,f \rangle (f|_{U} \neq 0)$に対して逆元を持つ.すなわち体である.
    実際,$U'=U \setminus (U \cap \zeros(f))$
    \footnote{$f$が0にならない$U$の開部分集合である}
    と置けば,($f|_{U} \neq 0$より)これは空でないので,$\langle U',1/f\rangle$と逆元を作れる.

    \subsection{$\mathcal{O}(Y) \subset \mathcal{O}_P \subset K(Y)$}
    この包含関係は殆ど自明である.正確には,injection mapが作れることは自明である.
    $\mathcal{O}(Y)$の元は$Y$全体でregularだから,$\mathcal{O}_P$に属す.
    $\mathcal{O}_P$の元は$P \in Y$でregularだから,必ず$K(Y)$に属す.

\section{Theorem 3.2}
    \subsection{About $\alpha: A(Y) \to \mathcal{O}(Y)$}
    多項式$f \in A$は$Y$全体で定義されるから,
    これは$\langle Y, f|_{Y} \rangle \in \mathcal{O}(Y)$のように$\mathcal{O}(Y)$へ埋め込める.
    $f \mapsto f|_{Y}$という写像によって0となるのは$\mathcal{I}(Y)$の元のみであるから,
    $A \to \mathcal{O}(Y)$から$\alpha: A(Y) \to \mathcal{O}(Y)$が得られる.

    \subsection{About proof of (b)}
    点$P \in Y$に対応する$A(Y)$の極大イデアルは次のように作られる.
    \[ P \mapsto \defs(P) \mapsto \defs(P)/\defs(Y) \]
    さらに$\defs(P)/\defs(Y)$の元$\bar{f}=f+\defs(Y)$を
    $\alpha$によって$Y$上のregular functionとみなすことで
    この極大イデアルを$\I{m}_P:=\{f \in A(Y) | f(P)=0 \}$と書くことが出来る.

    \subsection{About proof of (c)}
    $S=A \setminus \I{m}_P$としよう.
    すると$s \in S$について$\alpha(s)=\langle Y,s|_{Y} \rangle$は$\mathcal{O}_P$の単元である.
    実際$s(P) \neq 0$だから,$s|_{U} \neq 0$であるような$P$の近傍$U$が存在して,
    $\langle U,1/(s|_{Y}) \rangle \in \mathcal{O}_P$.
    このことからAti-MacのProp3.1(商環の普遍性)が使える.
    すなわち,$\beta: A(Y)_{\I{m}_P}=S^{-1}A(Y) \to \mathcal{O}_P$であって
    $\alpha(f)=\beta(f/1)$となるものが存在する.
    $f \mapsto f/1$と$\alpha$は単射だから$\beta$は単射.
    さらにProp3.1の証明から$\beta(a/s)=\alpha(a)/\alpha(s)$.

    この対応$\beta$が全射であることは$\mathcal{O}_P$の定義から直ちに分かる.
    つまり,$\mathcal{O}_P$の元はこのようにして得られるものしか無い.
    よって$A(Y)_{\I{m}_P} \cong \mathcal{O}_P$.

    以上の段落から,
    $\dim \mathcal{O}_P=\dim A(Y)_{\I{m}_P}=\height \I{m}_P$が得られる.
    さらに$A(Y)$の元に点$P$を代入する準同型に準同型定理を用いれば,$A(Y)/\I{m}_P \cong k$が得られる.
    (1.7)と(1.8A)より
    \[ \dim A(Y)/\I{m}_P+\height \I{m}_P=\dim A(Y) \iff 0+\dim \mathcal{O}_P=\dim Y. \]

    \subsection{About proof of (d)}
    \paragraph{$\Quot(A(Y)) \cong \Quot(\mathcal{O}_P)$}
    本文中のquotinet fieldは整域の全商環\footnote{すなわち整域$R$に対する$S=R \setminus (0)$による局所化}の事.
    今,環$R$の全商環を$\Quot(R)$と書くことにしよう.
    すると(c)の結果より,以下が得られる.
    \[ \mathcal{O}_P \cong A(Y)_{\I{m}_P} \implies \Quot(\mathcal{O}_P) \cong \Quot(A(Y)_{\I{m}_P}) \cong \Quot(A(Y)) \]
    $\Quot(A(Y)_{\I{m}_P}) \cong \Quot(A(Y))$は以下のように示される.まず,
    \[ S_0=A(Y) \setminus \I{m}_P, S_1=A(Y) \setminus \{0+\defs(Y) \} \]
    と置くと$\Quot(A(Y))=S_1^{-1}A(Y), \Quot(A(Y)_{\I{m}_P})=\tau(S_1^{-1})(S_0^{-1}A(Y))$である.
    ただし$\tau$は$x \mapsto x/1$という単射写像である.
    Ati-Mac Ch.3 Ex3より,これは$(S_1 S_0)^{-1}A(Y)$と同型.
    \footnote
    {
        Ati-Mac Ch.3 Ex3の証明は,$(a/st) \mapsto ((a/s)/(t/1))$が同型でもあることを示せば良い.
        $st$を$(s,t)$に因子分解するところに任意性があるので,どのように分解しても良いことまで示す必要が有る.
    }
    さらに
    \[ S_1 S_0=\set{ xy | x,y \in A(Y) \land \lnot(x \in \I{m}_P \lor y \in (\bar{0})) } \]
    であるので,$\I{m}_P, (\bar{0})$がイデアルであることから
    \[ S_1 S_0=A(Y) \setminus (\I{m}_P \times (\bar{0}))=A(Y) \setminus (\bar{0})=S_1 \]
    よって$\Quot(A(Y)_{\I{m}_P}) \cong S_1^{-1}A(Y)=\Quot(A(Y))$.

    \paragraph{$\Quot(A(Y)) \cong K(Y)$}
    $\langle U, f \rangle \in \Quot(\mathcal{O}_P) \cong \Quot(A(Y))$を任意にとる.
    すると$\langle U, f \rangle$は$f|_U \neq 0$を満たすから,
    $K(Y)$の元として逆元$\langle \tilde{U}, 1/f \rangle$を持つ.
    したがって
    \[ \langle U, f \rangle^{-1} \mapsto \langle \tilde{U}, 1/f \rangle \]
    という対応が作れた.
    逆写像も作れる.それは単純に以下のように定まる.
    \[ (0 \neq )\langle \tilde{U}, 1/f \rangle \mapsto \langle \tilde{U}, f \rangle \]
    $\tilde{U} \subset U$より$\langle U, f \rangle \equiv \langle \tilde{U}, f \rangle$なのでこれは全単射.
    この写像が同型写像であることを見よう.つまり演算を保つことを見る.
    \begin{align*}
        \frac{\langle U,f \rangle}{\langle V,g \rangle} \frac{\langle U',f' \rangle}{\langle V',g' \rangle}
        &\mapsto \quad
        \langle U \cap \tilde{V},f/g \rangle \langle U' \cap \tilde{V'},f'/g' \rangle
        &=
        \langle U \cap U' \cap \tilde{V} \cap \tilde{V'},ff'/gg' \rangle
        \\ \\
        %%%%%
        \frac{\langle U, f \rangle \langle U', f' \rangle}{\langle V, g \rangle \langle V', g' \rangle}
        &= \qquad\qquad
        \frac{\langle U \cap U', ff' \rangle}{\langle V \cap V', gg' \rangle}
        &\mapsto
        \langle U \cap U' \cap \widetilde{V \cap V'}, ff'/gg' \rangle
    \end{align*}
    あとは$\tilde{V} \cap \tilde{V'}=\widetilde{V \cap V'}$を見れば良い.
    \begin{align*}
        \tilde{V} \cap \tilde{V'}
        &=\set{P \in V | g(P) \neq 0} \cap \set{P \in V' | g'(P) \neq 0} \\
        &=\set{P \in V \cap V' | g(P) \neq 0 \land g'(P) \neq 0} \\
        &=\set{P \in V \cap V' | \lnot(g(P)=0 \lor g'(P)=0)} \\
        &=\set{P \in V \cap V' | g(P)g'(P) \neq 0} \\
        &=\widetilde{V \cap V'}
    \end{align*}
    $+$についても同様である.
    (有理関数の積と和では分子のみ異なるが,以上の議論では分子の零点は関係なかった.)

    \subsection{About proof of (a)}
    $\mathcal{O}_P$は点$P$で定義されるregular function全体だから,
    $\mathcal{O}(Y) \subseteq \bigcap_{P \in Y} \mathcal{O}_P$は自明.
    (b), (c)を用いると$\mathcal{O}_P \cong A(Y)_{\I{m}_P} = A(Y)_{\I{m}}$となる.
    単射$\alpha: A(Y) \to \mathcal{O}(Y)$の存在から$A(Y) \subseteq \mathcal{O}(Y)$とみなせる.
    よって
    \[ A(Y) \subseteq \mathcal{O}(Y) \subseteq \bigcap_{\I{m} \in \Max(A(Y))} A(Y)_{\I{m}}. \]
    この最左辺と最右辺が一致することを示そう.
    すると直ちに3つが全て等しいことが分かる.

    整域$B$について,その極大イデアルを2つとり,$\I{m}, \I{m}'$とする.
    すると$B_{\I{m}} \cap B_{\I{m}'}$の元は,
    分母が$\I{m}$にも$\I{m}'$にも属さない元であるような分数である.
    よって$B_{\I{m}} \cap B_{\I{m}'}=B_{\I{m} \cup \I{m}'}$.
    このように考えることで,以下が得られる.
    \[ \bigcap_{\I{m} \in \Max(B)} B_{\I{m}} = B_{M} ~~ \left( M:=\bigcup_{\I{m} \in \Max(B)} \I{m} \right) \]
    $B \setminus M$に非単元があればそれを含む極大イデアルが存在し,
    したがってその非単元は$M$に属す.よって$B \setminus M$は単元のみの集合(単元群)である.
    $B_M$の元は分母が単元であるような分数だから,$a/s \mapsto a \cdot s^{-1}$という対応で直ちに$B$と同型になる.
    以上より$B \cong B_M$.

\section{Proposition 3.3}
    \begin{Prop}
        $U_i = \proj^n \setminus \zerosp(x_i)$と置く.
        この時(2.2)の写像$\phi_i: U_i \to \affine^n$はisomorphism of varietyとなっている.
    \end{Prop}
    \begin{proof}
        isomorphism of varietyの定義からまず$\phi_i$が連続であることが必要だが,
        すでに同相写像であることがわかっている.
        あとはregular function $f/g$について$(f/g) \circ \phi_i$もregularであることを示せば良い.

        regular functionの定義より,$f,g \in k[x_0, \dots, x_n]$は斉次である.
        これらを(2.2)の$\alpha_i, \beta_i$で写し合うことで証明を行う.
        $\alpha, \beta$の添字$i$は以降省略する.
        \begin{align*}
            {}& \left( \frac{f}{g} \circ \phi_i \right)(x_0, \dots, x_n) \\
            =&  \frac{f}{g} \left( \frac{x_0}{x_i}, \dots, \frac{x_n}{x_i} \right) \\
            =&  \frac{\alpha(f)}{\alpha(g)} (x_0, \dots, x_n)
        \end{align*}
        $\alpha(g)(P)=0$ならば$\beta \alpha(g)(P)=g(P)=0$が得られる.
        よって$g(P) \neq 0 \implies \alpha(g)(P) \neq 0$なので,
        $\frac{\alpha(f)}{\alpha(g)}$はregular.
    \end{proof}

\section{About $S_{(\I{p})}, S_{(f)}$}
    $S_{(\I{p})}$がlocal ringで,その極大イデアルが
    $\I{m}:=(\I{p} \cdot T^{-1}S) \cap S_{(\I{p})}$であることを見よう.
    $S_{(\I{p})}$がringであることは明らか.
    まず$\I{m}$がイデアルとなっていることだが,これは実際に計算して分子分母の次数を見れば良い.
    あとは$S_{(\I{p})} \setminus \I{m}$が単元のみからなることを見れば良いが,
    これは$E:=T^{-1}S \setminus \I{p} \cdot T^{-1}S$が単元のみからなることを元に直ちに分かる.
    $E$の元は$f/g ~(g \not \in \I{p})$と表されるような$S_{\I{p}}$の単元全体であり,
    $S_{(\I{p})} \setminus \I{m}$は$E$の内次数0のものである.
    $S_{(f)}$がringであることも容易.

\section{Theorem 3.4}
    \subsection{About $\phi^{\ast}_i$}
        多項式$f(y_1, \dots, y_n) \in A(Y)$を取る.
        これを代入写像$\Phi$
        \[ \Phi: y_1 \mapsto x_0/x_i, \dots, y_n \mapsto x_n/x_i \]($x_i/x_i$はommited)
        で写して通分すると,$\bar{f}(x_0,\dots,x_n)/x_i^{\deg f}$という$S(Y)_{(x_i)}$の元が出来る.
        例えば$i=1$とすると
        \[ y_1^3+y_1 y_2+1 \mapsto \frac{x_0^3}{x_1^3}+\frac{x_0}{x_1}\frac{x_2}{x_1}+1=\frac{x_0^3+x_0^2 x_1+x_1^3}{x_1^3} \]
        逆に任意に$S(Y)_{(x_i)}$の元を取るとそれは$F(x_0,\dots,x_n)/x_i^{\deg F}$という形をしているが,
        \[ x_0 \mapsto y_1, \dots, x_i \mapsto 1, \dots, x_n \mapsto y_n \]
        とすれば,この写像$\Phi$で$F$に写ってくるようなものが作れる.
        よって$\Phi$は全射である.

        構成した$\Phi$の逆写像を観察すると,
        これは$Y$上で消える斉次多項式$F(x_0,\dots,x_n)$を$F(x_0,\dots,1,\dots,x_n)$に写す.
        このようにして出来る多項式はまさしく$\defsa(Y_i)$の元である.
        よって$\ker \Phi=\defsa(Y_i)$.正確には$\ker \Phi=\defsa(Y_i)/\defsa(Y)$である.
        ($Y_i \subset Y$より$\defsa(Y_i) \supset \defsa(Y)$が成り立つことに注意.)
        $(A/\defsa(Y))/(\defsa(Y_i)/\defsa(Y)) \cong A/\defsa(Y_i)$だから,
        $\Phi: A(Y) \to S(Y)_{(x_i)}$から同型写像$\phi^{\ast}_i: A(Y_i) \to S(Y)_{(x_i)}$が作れた.

    \subsection{About (b)}
    affineの$\mathcal{O}_P$は$\mathcal{O}_P^a$と書くことにする.projectiveも同様.

    \paragraph{$\mathcal{O}_P^p \cong A(Y_i)_{\I{m}'_P}$}
    最初に$\mathcal{O}_P^p \cong A(Y_i)_{\I{m}'_P}$とある.
    Theorem 3.2は$\mathcal{O}_P^a \cong A(Y)_{\I{m}_P}$しか言っていないので,
    これを導く.
    まず$\affine^n \cong \sqcup_i$を用いると,
    
    \paragraph{$\phi^{\ast}_i(\I{m}'_P)=\I{m}_P \cdot S(Y)_{(x_i)}$}
    pass

    \paragraph{$S(Y)_{(x_i)} \cong A(Y_i)_{\I{m}'_P}$}
    pass

    \subsection{About (c)}
    \paragraph{$\Forall{i} K^p(Y)=K(Y_i)$}
    pass

    \paragraph{$K^p(Y) \cong S(Y)_{((0))}$}
    pass

    \subsection{About (a)}
    \paragraph{$f \in A(Y_i), x_0^q f^q \in S(Y)$}
    $f \in A(Y_i)$をとると,$A(Y_i) \cong S(Y)_{(x_i)}$より
    $f$は$g_i/x_i^{N_i}~(g_i \in S(Y)_{N_i})$と書ける.
    したがって各$i$について$g_i=f \cdot x_i^{N_i} \in S(Y)_{N_i}$.

    今,$N \geq \sum{N_i}$とすると,
    $N$次単項式$M=\prod_{i=1}^{n}{x_i^{e_i}}$について少なくともひとつの$e_i$は$N_i$より大きい.
    なので$f \cdot M$は$f \cdot x_i^{N_i} (\in S(Y)_{N_i})$の倍数である.
    $S(Y)_N$はこのような単項式で張られる$k$-vector spaceだから
    $S(Y)_N \cdot f \subset S(Y)_N$が分かる.

    \[ g_i^q=f^q \cdot x_i^{qN_i}=(f \cdot x_i^{N_i})(f^{q-1} \cdot x_i^{(q-1)N_i}) \in S(Y)_{N_i} \]
    だから,$g_i^q \in S(Y)_{N_i}$.あとは$q=1$の時と同様にして$S(Y)_N \cdot f^q \subset S(Y)_N$.

    \paragraph{``we can replace the $a_i$ by ..."}
    式の左辺を観る.$a_j \in S(Y)$である.$a_j$を斉次分解してみる.
    \[ f^m+a_1 f^{m-1}+\dots+a_m=f^m+(a^{(1)}_0+a^{(1)}_1+\dots) f^{m-1}+\dots+(a^{(m)}_0+a^{(m)}_1+\dots) \]
    すると$f$は0次式だから,この多項式の0次斉次部分は以下のようになる.
    \[ f^m+a^{(1)}_0 f^{m-1}+\dots+a^{(m)}_0 \]
    これは斉次多項式だからprojective varietyの点を入れた時に値が0かどうかということは意味を持つ.

\section{Proposition 3.5}
    \paragraph{A morphism $\mathcal{O}(Y) \to \mathcal{O}(X)$ which induced by $\phi$ is a homomorphism of $k$-algebras.}
    $\phi:X \to Y$から誘導される写像$\phi^{\ast}: \mathcal{O}(Y) \to \mathcal{O}(X)$は次のようなものである.
    \[ \phi^{\ast}: (f:Y \to k) \mapsto (f \circ \phi: X \to k). \]
    これは$(af+bg) \circ \phi=a(f \circ \phi)+b(g \circ \phi)$より,$k$-加群としての写像でもある.
    よってこれは$k$-代数の写像.
    また,$f \circ \phi \circ \phi^{-1}=f$より,以下が$\phi^{\ast -1}$.
    \[ \phi^{\ast -1}: (g:X \to k) \mapsto (g \circ \phi^{-1}: Y \to k). \]

    \paragraph{Naturallity of $\alpha$}
    多分.単に「標準的な」という意味でnaturalと言っている.

\section{Lemma 3.6}
    $x_i \circ \psi$とあるが,これは$x_i$を$(x_0,\dots,x_n) \mapsto x_i$という関数だと考えれば
    単に$\pr_i \circ \psi$のことである.
    この解釈は$f(x_0,\dots,x_n) \circ \psi=f(x_0 \circ \psi,\dots,x_n \circ \psi)$を考えれば自然である.

    \paragraph{If $x_i \circ \psi$ is regular, then for all $f \in k[x_0,\dots,x_n]$, $f \circ \psi$ is also regular on $X$.}
    \[ f \circ \psi=f(x_0,\dots,x_n) \circ \phi=f(x_0 \circ \phi,\dots,x_n \circ \phi) \]
    $\mathcal{O}(X)$は任意のvarietyについて環を成すから,
    $\mathcal{O}(X)$の元の積と和で表示される$f(x_0 \circ \phi,\dots,x_n \circ \phi)$は$\mathcal{O}(X)$の元.
    よって$f \circ \psi$は$X$でregular.
    
    \paragraph{$\psi$ is continuous.}
    $Y$の閉集合$\zerosa(E)~~(E \subset k[x_0,\dots,x_n])$をとると,
    \[ \psi^{-1}(\zerosa(E))=\{\psi^{-1}(P) \in X ~|~ \Forall{f \in E} P \in f^{-1}(0) \}=\bigcap_{f \in E} \psi^{-1} \circ f^{-1}(0) \]
    任意の多項式$f$について$f \circ \psi:X \to k$がregular.
    regularならcontinuousなので$(f \circ \psi)^{-1}=\psi^{-1} \circ f^{-1}$は閉集合を閉集合に写す.
    よってこの最右辺は閉集合.

\section{Corollary 3.8}
    $k$上のaffine varietyとmorphism of varietiesが成す圏を$\mathbf{Aff}$と書くことにし,
    $k$上の有限生成整域とその間の準同型写像が成す圏を$\mathbf{k-FinDom}$と書くことにしよう.
    関手$A(-):\mathbf{Aff} \to \mathbf{k-FinDom}$は対象$X$を$A(X)$に写し,射$\phi$を$\alpha(\phi)$にうつす.
    これが圏同値をつくることを示そう.

    $A(-)$が圏同値をつくることと$A(-)$が忠実充満関手かつ本質的全射であることは同値である.
    忠実充満関手であることはProp 3.5で示されている.
    この系では一般のvarietyとしている$X$をaffineとしているので$\mathcal{O}(X) \cong A(X)$が使えることに注意せよ.

    本質的全射であることは,Ex1.5から得られる.
    Ex1.5ではべき零元を持たない$k$上の有限生成代数は必ず何らかの代数的集合のaffine coordinate ringと同型であることを示す.
    しかしその証明から,任意の$k$上の有限生成整域$B$は素イデアル$\I{p}$を用いて$k[x_0,\dots,x_n]/\I{p}$と表せることがわかるから,
    任意の$k$上の有限生成整域$B$はあるaffine variety$X$を用いて$A(X)$と表せる.

\end{document}
