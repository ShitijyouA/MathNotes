\documentclass[a4paper, dvipdfmx]{jsarticle}
\usepackage{macros}

\newcommand{\cons}{\mathrm{cons}}
\newcommand{\centerpb}{\ar[lu, phantom, "p.b."]}
\newcommand{\xto}[1]{\xrightarrow{#1}}

\begin{document}
\title{ゼミノート \#12 \\ Quotients of Algebraic Spaces}
\author{七条彰紀}
\maketitle
\tableofcontents
\vspace{10pt}

\section{Notes on Topology}
\subsection{Constructible Topology}
以下を参考にした.
\begin{itemize}
    \item \cite{Rydh10} \S1
    \item \url{http://virtualmath1.stanford.edu/~conrad/Perfseminar/Notes/L3.pdf} by B.Conrad
    \item \cite{SP} 08YF \url{https://stacks.math.columbia.edu/tag/08YF}
\end{itemize}

\begin{Def}
    $X$ :: topological spaceとする.
    \begin{enumerate}
    \item 
        $X$のlocally closed subsetとは,
        closed subsetとopen subsetの共通部分で表せるsubsetである.

    \item
        $X$のconstructible setとは,
        $X$の有限個のlocally closed subsetの和集合で表せるsubsetのことである.

    \item
        $U \subseteq X$が$X$のlocally constructible setであるとは,
        $U$のある開被覆$\{U_i\}$について,
        各$U \cap U_i$がconstructible setである,ということ.

    \item
        $X$のconstructible topologyとは,
        $X$のconstructible setを開基とする位相のことである.
        $X$のunderlying setに$X$のconstructible topologyを
        与えた位相空間を$X_{\cons}$と書く.

    \item
        有限個とは限らない$X$のconstructible setの,
        和集合をind-constructible subsetと呼び,
        共通部分をpro-constructible subsetと呼ぶ
        \footnote{ ``ind-"はinductive limitを意味し,``pro-"はprojective limitを意味する. }.

    \item
        map of topological spaces :: $f \colon X \to Y$について,
        $f^{\cons}$をconstructible topologyでのmapとする.
        (map of setsとしては$f=f^{\cons}$である.)
    \end{enumerate}
\end{Def}

\begin{Prop}
    $X$ :: topological spaceとする.
    \begin{enumerate}
    \item
        $X$のopen subsetとclosed subsetはconstructible setである.

    \item
        有限個のconstructible setの和,共通部分はconstructible setである.
        constructible setの補集合もconstructible setである.

    \item
        $X$のconstructible topologyに於けるopen subsetはind-constructible subsetに限る.
        同様に,closed subsetはpro-constructible subsetに限る.

    \item
        map of topological spaces :: $f \colon X \to Y$について,
        $f^{\cons}$ :: continuous.
    \end{enumerate}
\end{Prop}
\begin{proof}
    自明.   
\end{proof}

\begin{Prop} \label{nontrivial_constop}
    \begin{enumerate}
    \item 
        qcqs($=$quasi-compact and quasi-separated) schemeのpro-constructible subsetは,
        affine schemeからの射の像に限る.

    \item
        locally of finite presentation morphismはconstructible topologyにおいてopen.

    \item
         quasi-compact morphismはconstructible topologyにおいてclosed.

    \item
        $f$ :: surjective morphismで
        $f$ :: locally of finite presentation or quasi-compactならば
        $f^{\cons}$ :: submersive.
    \end{enumerate}
\end{Prop}
\begin{proof}
    (i)はRydh10のProp1.1である.
    (ii)はChevalley's theoremからの帰結.
    (iii)はlocallyに調べれば容易に分かる.
    (iv)は(ii), (iii)からの帰結である.
\end{proof}

\begin{Remark}
    constructible topologyはspectral space
    \footnote
    {
        spectral spaceとは,以下の性質をもつ位相空間:
        sober, quasi-compact,
        the intersection of two quasi-compact opens is quasi-compact,
        and the collection of quasi-compact opens forms a basis for the topology
        (\cite{SP} 08FG).
    }
    と共に扱われることが多い.
    例えばqcqs schemeのunderlying spaceはspectralである.
\end{Remark}

\begin{Prop}{\cite{Rydh10} Prop1.7} \label{refrecting_property_of_pullback}
    morphism of schemes :: $f \colon X \to Y, g \colon Y' \to Y$を考え,
    $f$の$g$によるpullbackを$f'$と書く.
    \begin{enumerate}
    \item 
        $P$をopen, closed, submersiveのいずれかとする.
        $g$がsubmersiveならば,
        $f'$ :: Pと$f$ :: Pは同値.
    \item 
        $P$をuniversally open, universally closed, universally submersive, separatedのいずれかとする.
        $g$がuniversally submersiveならば,
        $f'$ :: Pと$f$ :: Pは同値.
    \item 
        $g^{\cons}$がuniversally submersiveならば,
        $f'$ :: quasi-compactと$f$ :: quasi-compactは同値.
    \end{enumerate}
\end{Prop}
\begin{proof}
    (TODO)
    (iii)だけ証明を与える.
\end{proof}

\begin{Remark}
    おそらく,\cite{Rydh13}はこの命題を利用するために,
    topological quotientに「$q^{\cons}$ :: universal submersive」を要求している.
    より詳しく言うと以下の命題で使われている.
    \begin{Prop}[\cite{Rydh13} Prop2.12 (ii)]
        $R \parto{s}{t} X$ :: groupoidとし,
        $q \colon X \to Y$をtopological quotientとする.
        $j$ :: quasi-compactと,
        $Y$ :: quasi-separatedかつ$j_{/Y}$ :: quasi-compactは同値.
    \end{Prop}
    これを経由して,
    $X \to S$ :: quasi-separatedならば
    GC quotient :: $Y \to S$がquasi-separatedであることなどを示している(Prop4.7).
\end{Remark}

\subsection{Equivalence Relation on Topological Space Induced by Groupoid}
$S$ :: algebraic spaceとし,
groupoid in algebraic $S$-space :: $R \parto{s}{t} X$を考える.

topological space :: $|U|$に,次のようにして同値関係$\sim_{R}$を定義する.
\begin{Def}
    点$x_1, x_2 \in |X|$について,
    \[ x_1 \sim_R x_2 \iff \Exists{r \in |R|} |s|(r)=x_1, |t|(r)=x_2 \]
    と定義する.
\end{Def}
$|R \times_{x} R| \to |R| \times_{|X|} |R|$が全射であることを用いると,
groupoidの定義から,$\sim_R$が同値関係であることが分かる.

\begin{Def}
    点$x \in |X|$の同値類をorbitと呼び,$R(x)$と書く.
    $R(x)$は$|t|(|s|^{-1}(x))$と等しい.

    また,$W \subseteq |X|$が$R$-stableであるとは,
    $W$が$\sim_{R}$についてstableであること.
    すなわち,
    \[ \{ x \in |X| \mid \Exists{w \in W} w \sim_R x \}=R \]
    となること.
    これは$|s|^{-1}(W)=|t|^{-1}(W)$ in $|R|$とも同値.
\end{Def}

\begin{Remark}
    \cite{SP} 04XJには,
    $S=|X| \times_{|[X/R]|} |X|$とすると位相空間として$|[X/R]|=|X|/S$,
    という命題が有る.
\end{Remark}

\section{Quotients}

以降は引き続き$S$ :: algebraic spaceとし,
groupoid in algebraic $S$-space :: $R \parto{s}{t} X$を考える.

\subsection{Definitions}
\begin{Def}[equivariant morphism]
    morphism :: $q \colon X \to Y$について,
    $q \circ s=q \circ t$であるとき,
    $q$をequivariant morphismという.
\end{Def}

\begin{Def}[$j, j_{Y}$]
    $s,t \colon R \to X$から$X \times_{S} X$の普遍性により得られる
    射 :: $R \to X \times_{S} X$を$j$と書く.

    また,equivariant morphism :: $q \colon X \to Y$について,
    $s,t$から$X \times_{Y} X$の普遍性により得られる
    射 :: $R \to X \times_{Y} X$を$j_{/Y}$と書く.
\end{Def}

stabilizerはまたの機会に定義する.

\begin{Remark}
    fiber productの普遍性から,
    $j_{/Y}$に$X \times_{Y} X \to X \times_{S} X$を合成すると$j$に一致する.
\end{Remark}

\begin{Remark}
    equivariant morphism :: $R \parto{s}{t} X \to Y$は,
    quotient stackからの射$[X/R] \to Y$に一対一に対応する.
    (TODO: proof)
\end{Remark}

\begin{Def}[universal, uniform quotient]
    $q \colon X \to Y$が性質$P$をもつとする.

    \begin{itemize}
    \item 
        任意の射$Y' \to Y$によるpullback :: $q' \colon X \times_Y Y' \to Y'$も
        性質$P$をもつ時,$P$はuniversalであると言う.
    \item
        任意のflat射$Y' \to Y$によるpullback :: $q' \colon X \times_Y Y' \to Y'$も
        性質$P$をもつ時,$P$はuniformであると言う.
    \end{itemize}
\end{Def}

\begin{Def}
    equivariant morphism :: $q \colon X \to Y$を考える.

\begin{description}[labelindent=3ex, leftmargin=7ex, style=nextline, font=\textbf]
    \item[Categorical quotient         ]
        任意のequivariant morphism :: $r \colon X \to Z$が$q$を介して一意に分解する時,
        すなわち$\bar{r} \circ q=r$を満たす
        射 :: $\bar{r} \colon Y \to Z$が一意に存在するとき,
        $q$をcategorical quotientと呼ぶ.

    \item[Zariski quotient             ]
        $|q| \colon |X| \to |Y|$がtopological spaceの圏における
        $|R| \parto{|s|}{|t|} |X|$のcoequalizerである時,
        $q$をZariski quotientと呼ぶ.
        同値な言い換えとして,
        任意の点の$|q|$による逆像が丁度一つのorbitから成り,
        かつ$|q|$ :: submersiveである,というものが有る.

    \item[Constructible quotient       ]
        $|q|^{\cons} \colon |X|^{\cons} \to |Y|^{\cons}$がtopological spaceの圏における
        $|R|^{\cons} \parto{|s|^{\cons}}{|t|^{\cons}} |X|^{\cons}$のcoequalizerである時,
        $q$をconstructible quotientと呼ぶ.
        言い換えについてはZariski quotientと同様である.

    \item[Topological quotient         ]
        $q$ :: universal Zariski \& universal constructible quotientである時,
        $q$をtopological quotientと呼ぶ.

    \item[Strongly topological quotient]
        $q$ :: topological quotient かつ $j_{/Y}$ :: universally submersiveである時,
        $q$をstrongly topological quotientと呼ぶ.

    \item[Geometric quotient           ]
        $q$がtopological quotientであり,
        かつ$\shO_{Y}$が$Y_{\ET}$ (category of etale sheaves on $Y$)における$s^*, t^*$のequalizerである時,
        $q$をgeometric quotientと呼ぶ.
        \[
        \begin{tikzcd}
            \shO_{Y} \ar[r]& q_*\shO_{X}
                \ar[r, shift left, "s^*"] \ar[r, shift right, "t^*"']& (q \circ s)_* \shO_R
        \end{tikzcd}
        \]

    \item[Strongly geometric quotient  ]
        $q$ :: geometric quotient \& strongly topological quotientであるとき,
        すなわち$q$ :: geometric quotientかつ$j_{/Y}$ :: universally submersive
        であるとき,
        $q$をstrongly geometric quotientと呼ぶ.
\end{description}
\end{Def}

\begin{Remark}
    strongly topological quotientでは
    $j_{/Y} \colon R \to X \times_Y X$がuniv. submersiveであるから,
    $X \times_{Y} X \to X \times_{S} X$もuniv. submersive.
    このことは$X \times_{Y} X$に「適切な」位相が入っていることを意味する.
\end{Remark}

\begin{Remark}
    geometric quotient in \cite{GIT}
    \begin{itemize}
        \item $q$ :: surjective and equivariant.
        \item $\shO_{Y}=(q_* \shO_X)^{R}$.
        \item 任意の点$y \in Y$について,$q^{-1}(y)$はただ一つのorbitからなる.
        \item
            $W_1, W_2 \subseteq X$ :: disjoint closed subsetについて
            $\cl_Y(q(W_1)), \cl_Y(q(W_2))$ :: disjoint.
    \end{itemize}

    以下のように言い換えても良い.
    \begin{itemize}
        \item $q$ :: Zariski quotient.
        \item $\shO_{Y}=(q_* \shO_X)^{R}$.
        \item $q$のopen immersionによるpullbackも上記を満たす.
    \end{itemize}
    ref. E.Viehweg ``D. Mumford's Geometric Invariant Theory".
    なお,\cite{GIT}の初版では$q$ :: universally submersiveを仮定している.
\end{Remark}

\subsection{Propositions : Paraphase}

\begin{Def}
    $X$ :: algebraic space over a scheme $S$とする.
    点$x \in |X|$のresidue fieldとは,
    $x$を代表するmonomorphism :: $\Spec k \to X$が存在するような体$k$のことである.
\end{Def}

\begin{Remark}
    residue fieldは常に存在するとは限らない.
    ``descent algebraic space"と呼ばれる重要な種類のalgebraic spaceでは,
    任意の点がresidue fieldをもつ.
\end{Remark}

\begin{Lemma}
    $X$ :: algebraic space over a scheme $S$とする.
    点$x \in |X|$をとり,
    \begin{itemize}
        \item $x$を代表するmonomorphism :: $\phi \colon \Spec k \to X$と
        \item $x$を代表する任意の射 :: $\psi \colon \Spec l \to X$
    \end{itemize}
    をとる.
    この時$\psi$は$\phi$を通じて一意に分解する.
\end{Lemma}
\begin{proof}
    fiber product :: $Y=(\Spec k) \times_{\phi, X, \psi} (\Spec l)$をとる.
    $\phi, \psi$が同値であるから,$Y$は空でない.
    monoはpullbackで保たれるから$Y \to \Spec l$もmono.
    よって\cite{SP} 03DP
    \footnote
    {
        証明を簡単にまとめると次のように成る.
        (1) 可換代数の命題「体から代数への全射準同型$\phi \colon k \to R$は同型(特に単射)」に帰着させる.
        (2) $R \to R \otimes_{k} R; r \mapsto r \otimes 1$は,$\tilde{r} \in \phi^{-1}(r)$について$r \otimes 1=\tilde{r} (1 \otimes 1)$なので単射.
        (3) $R$は体上の代数なのでfree,特にfaithfully flat $k$-module.
    }
    から$Y \iso \Spec l$.
    \[
    \begin{tikzcd}
        Y       \ar[r]\ar[d, tail, "\text{\rotatebox{90}{$\iso$}}"']&   \Spec k \ar[d, tail, "\phi"]\\
        \Spec l \ar[r, "\psi"']&                                         X
    \end{tikzcd}
    \]
    こうして$\psi$の$\phi$を通じた分解が存在する.
    $\phi$がmonoなのでこの分解は一意.
\end{proof}

\begin{Cor}
    $X$ :: algebraic space over a scheme $S$とする.
    点$x \in |X|$を代表するmonomorphismは高々一つ.
\end{Cor}

\begin{Prop}[\cite{Rydh13}, Prop2.3]
    $R$-equivariant morphism :: $q \colon X \to Y$を考える.
    以下の$3 \times 3=9$個の命題を考える.
    {\samepage
    \begin{enumerate}
    \item
        任意の体$k$と射$y \colon k \to Y$ \footnote{$\Spec k$を$k$と略した.}
            について$|X \times_{Y} k|$は,\mnewline
        少なくとも1つの / 多くとも一つの / 丁度一つの,$(R \times_{Y} k)$-orbitを含む.
    \item
        $q$ :: surjective / $j_{/Y}$ :: surjective / $q, j_{/Y}$:: surjective.
    \item
        任意の代数閉体$K$について,$\bar{q}_K \colon X(K)/R(K) \to Y(K)$
        \footnote
            {
                $X(K)/R(K)$は$R(K) \parto{s_K}{t_K} X(K)$のcoequalizerで,
                $\bar{q}_K$はcoequalizerによる$q_K \colon X(K) \to Y(K)$の一意な分解である.
            }は \mnewline
        surjective / injective / bijective.
    \end{enumerate}
    }

    この時,(i)$ \iff$ (ii) $\impliedby$ (iii)がそれぞれ成り立つ.
    さらに$q, j_{/Y}$ :: locally of finite type or integralならば,
    (ii) $\implies$ (iii)も成り立つ.
\end{Prop}
\begin{proof}
    以下で計$6$つの命題の証明を行う.
    そのために,取り扱う$6$つの命題を以下のようにまとめる.
    \begin{center}
    \begin{tabular}{|l|l|l|}
    \hline
          & a                       & b                             \\ \hline
        1 & (i) 少なくとも1つの     & (i) 多くとも一つの            \\ \hline
        2 & (ii) $q$ :: surjective  & (ii) $j_{/Y}$ :: surjective   \\ \hline
        3 & (iii) surjective        & (iii) injective               \\ \hline
    \end{tabular}
    \end{center}
    例えば``(1a)"という記号は,(i)に含まれる
    「任意の体$k$と射$y \colon k \to Y$について$|X \times_{Y} k|$は
        少なくとも1つの$(R \times_{Y} k)$-orbitを含む.」
    という命題を意味する.

    体 :: $k$, morphism :: $y \colon \Spec k \to Y$をとる.
    $\Spec k$を$k$と略し,
    $X \times_{Y} \Spec k$や$R \times_{Y} k$をそれぞれ$X_y, R_y$と略す.

    \paragraph{(1a) $\implies$ (2a)}
    仮定より$|X_y| \neq \emptyset$である.
    この集合から$|q|^{-1}(y)$への写像が存在するので$|q|^{-1}(y) \neq \emptyset$.
    よって$q$は全射.
    \[
    \begin{tikzcd}[]
        \vert X_y \vert \ar[r]&
        \vert X \vert \times_{\vert Y \vert} \vert k \vert \ar[r, "\pr_{\vert X \vert}"]&
        \vert q \vert^{-1}(y)
    \end{tikzcd}
    \]

    \paragraph{(2a) $\implies$(1a)}
    %% {{{
    仮定から直ちに次がわかる.
    \[
                 q \colon X \to Y \text{ :: surj}
        \implies y^*q \colon X_y \to k \text{ :: surj}
        \iff     |y^*q| \text{ :: surj}
        \iff     \Forall{t \in |\Spec k|}|y^*q|^{-1}(t) \neq \emptyset
    \]

    $R_y$-equiv. morphismによる一点の逆像は$R_y$-orbitを含む
        \footnote{ 点$t \in |\Spec k|$について$q_y(t)=q_y(R_y(t))$だから. }から,
    $|X_y|$は$R_y$-orbit :: $|y^*q|^{-1}(t) \subseteq |X_y|$を含む.
    %% }}}

    \paragraph{(3a) $\implies$ (2a)}
    %% {{{
    $k$の代数閉包を$K$と書く.
    この時,$y$と$k \inclmap K$から誘導される射を合成すると,
    $y$と同値な点$\Spec K \to \Spec k \to X$が得られる.
    これを改めて$y \colon \Spec K \to X$と命名する.
    以下,$\Spec K$を$K$と略す.
    すると仮定($X(K)/R(K) \to Y(K)$ :: surj)と米田の補題により,
    $q \circ z=y$を満たす$z \colon K \to X$が存在する.
    よって$|q|$ :: surj.
    %% }}}

    \paragraph{(2b) $\implies$(1b)}
    %% {{{
    まず,$(X \times_{Y} X) \times_{Y} k \iso X_y \times_{k} X_y$に注意する.
    以下のpullback図式の$(j_{/Y})_y$を考える.
    仮定からこれは全射.
    \[
    \begin{tikzcd}[column sep=30pt]
        R_y \ar[r, "(j_{/Y})_y"]\ar[d]& X_y \times_k X_y \ar[r]\ar[d]& k \ar[d, "y"]\\
        R \ar[r, "j_{/Y}"']& X \times_Y X \ar[r]& Y 
    \end{tikzcd}
    \]
    こうして全射$|R_y| \to |X_y \times_k X_y| \to |X_y| \times_{|k|} |X_y|$が得られる.
    $j_{/Y}$の定義から,
    これらと$\pr_{i} \colon |X_y| \times_{|k|} |X_y| \to |X_y|$ ($i=1,2$)を合成すれば
    それぞれ$|s|, |t|$となる.
    よって任意の$u,v \in |X_y|$について$|s|(r)=u, |t|(r)=v$となる$r \in |R_y|$が存在する.
    すなわち,任意の$u,v \in |X_y|$について$u \sim_{R_y} v$.
    %% }}}

    \paragraph{(3b) $\implies$ (2b)}
    %% {{{
    点$z \colon \Spec k \to X \times_Y X$を任意に取ると,
    上述のとおり,$k$をその代数閉包$K$に取り替えることが出来る.
    したがってここでは$z \colon \Spec K \to X \times_{Y} X$を扱う.
    $z_i:=\pr_i \circ z, y:=q \circ \pr_i \circ z$とする.
    $q \circ \pr_1=q \circ \pr_2$に注意.
    以下,$\Spec K$を$K$と略し,
    米田の補題によって対応するもの(例えば射$z \colon K \to X$と$X(K)$の元)を同じ記号で書く.

    $q_K(z_1)=q_K(z_2)=y$なので,
    仮定から$s_K(r)=z_1, t_K(r)=z_2$を満たす元$r \in R(K)$が存在する.
    $s=\pr_1 \circ j_{/Y}, t=\pr_2 \circ j_{/Y}$なので,
    さらに以下の図式が可換に成る.
    \[
    \begin{tikzcd}[row sep=35pt]
        K \ar[r, "r"] \ar[rrd, "z_2"']
            \ar[rrr, "z_1", rounded corners, to path={(\tikztostart) -- ([yshift=15pt] \tikztostart.north) -- ([yshift=15pt] \tikztotarget.north)\tikztonodes -- (\tikztotarget)}]&
        R \ar[r, "j_{/Y}"]& X \times_Y X \ar[r, "\pr_1"]\ar[d, "\pr_2"']& X \ar[d, "q"]\\
        {} & {} & X \ar[r, "q"']& Y \centerpb
    \end{tikzcd}
    \]
    $X \times_Y X$の普遍性から$r \circ j_{/Y}=z$が分かる.
    すなわち,$|j_{/Y}|$は,したがって$j_{/Y}$は全射である.
    %% }}}

    \paragraph{(1b) $\implies$(2b)}
    %% {{{
    一つ前の段落と同じく$z, z_i\ (i=1,2), y$をとる.
    この$y$で$X \times_Y X \to X \to Y$をpullbackする.
    そこで$z_{i,y} \colon K \to X_y$を$z_i$と$\id[K]$から誘導される射とする.
    さらに点$z_y \colon K \to X_y \times_K X_y$を$z_{1,y}, z_{2,y}$から誘導される射とする
    \footnote
    {
        定義の仕方から$q_y \circ \pr_{i,y} \circ z_{y}=\id[K]$が成立する.
        これは重要な等式で,$z_{i,y}$の後に$z_{y}$を定義したのもこのためである.
        ここで$z_y$から先に定義すると,
        $i=1,2$の両方でこの等式が成立することを明言できない.
    }.
    明らかに$z_{i,y}=\pr_{i,y} \circ z_y$が成立する.
    \[
    \begin{tikzcd}[row sep=30pt]
        K \ar[d, "\id"']
            \ar[rrrr, "\id", rounded corners, to path={(\tikztostart) -- ([yshift=30pt] \tikztostart.north) -- ([yshift=30pt] \tikztotarget.north)\tikztonodes -- (\tikztotarget)}]
            \ar[rr, "z_y", rounded corners, to path={(\tikztostart) -- ([yshift=15pt] \tikztostart.north) -- ([yshift=15pt] \tikztotarget.north)\tikztonodes -- (\tikztotarget)}]&
        R_y \ar[r, "(j_{/Y})_y"]\ar[d]& X_y \times_K X_y \ar[r, "\pr_{i,y}"]\ar[d]& X_y \ar[r,"q_y"]\ar[d] & K \ar[d, "y"]\\
        K {}
            \ar[rr, "z"', rounded corners, to path={(\tikztostart) -- ([yshift=-15pt] \tikztostart.south) -- ([yshift=-15pt] \tikztotarget.south)\tikztonodes -- (\tikztotarget)}]
            \ar[rrrr, "y"', rounded corners, to path={(\tikztostart) -- ([yshift=-30pt] \tikztostart.south) -- ([yshift=-30pt] \tikztotarget.south)\tikztonodes -- (\tikztotarget)}]&
        R \ar[r, "j_{/Y}"']& X \times_Y X \ar[r, "\pr_i"']\centerpb& X \ar[r, "q"']\centerpb& Y \centerpb
    \end{tikzcd}
    \]

    今,仮定(1b)から,
    $r \in |R_y|$が存在し,$|s_y|(r)=[z_{1,y}], |t_y|(r)=[z_{2,y}]$を満たす.
    この$r$の代表元として$\tilde{r} \colon \Spec L \to R_y$をとる.

    \begin{Claim}\label{claim:z_yequiv}
        $z_y \circ \phi=(j_{/Y})_y \circ \tilde{r}$を満たす
        射$\phi \colon L \to K$が存在する.
    \end{Claim}

    この主張から,
    任意の点$z$について$|j_{/Y}|$による像が$z$である点$L \to R_y \to R$の存在が言える.
    \[
        \begin{tikzcd}[column sep=40pt]
        L \ar[r]\ar[d, "\tilde{r}"']& K \ar[d, "z_y"] \ar[dd, bend left=90, "z"]\\
        R_y \ar[r, "(j_{/Y})_y"]\ar[d]& X_y \times_{K} X_y \ar[d]\\
        R \ar[r, "j_{/Y}"']& X \times_{Y} X \centerpb
    \end{tikzcd}
    \]
    よって$j_{/Y}$は全射.

    \begin{proof}[(主張\ref{claim:z_yequiv}の証明)]
        $\tilde{r}':=(j_{/Y})_y \circ \tilde{r}$とおく.

        \[
        \begin{tikzcd}[row sep=50pt, column sep=40pt]
            {} & K \ar[d, "z_y" description] \ar[rd, tail, red, "z_{1,y}" description] \ar[dd, tail, "z_{2,y}"' description, near start, bend right=35, blue] & \\
            L \ar[r, "\tilde{r}'"']
                \ar[rr, bend left, red]\ar[rd, blue]\ar[ru, red, dashed, shift left, bend left, "\phi_1"]\ar[ru, blue, dashed, shift right, bend left,"\phi_2"']&
                X_y \times_{K} X_y \ar[r, "\pr_1"] \ar[d, "\pr_2"']& X_y \ar[d, "q_y"] \\
            {} & X_y \ar[r, "q_y"'] & K                 
        \end{tikzcd}
        \]
        仮定$|s_y|(r)=[z_{1,y}]$から赤実線で示した$2$つの射は同値である.
        $q_y \circ \,z_{1,y}=q_y \circ \pr_1 \circ \,z_y=\id[K]$,とくにこれがmonoだから$z_{1,y}$ :: mono.
        したがって補題より$\phi_1 \colon L \to K$が存在し,
        $\pr_1 \circ \,z_y \circ \phi_1=\pr_1 \circ \,\tilde{r}'$となっている.
        青実線で示した$2$つの射についても,
        同じく$\phi_2 \colon L \to K$が存在し,
        $\pr_2 \circ \,z_y \circ \phi_2=\pr_2 \circ \,\tilde{r}'$.

        それぞれ$q_y$を合成して,
        \[
            q_y \circ \pr_1 \circ \,z_y \circ \phi_1=q_y \circ \pr_1 \circ \,\tilde{r}', \quad
            q_y \circ \pr_2 \circ \,\tilde{r}'=q_y \circ \pr_2 \circ \,z_y \circ \phi_2.
        \]
        $q_y \circ \pr_1=q_y \circ \pr_2$と$q_y \circ \pr_1 \circ \,z_y=q_y \circ \pr_2 \circ \,z_y=\id[K]$から,
        \[ \phi_1=q_y \circ \pr_1 \circ \,\tilde{r}'=q_y \circ \pr_2 \circ \,\tilde{r}'=\phi_2. \]
        したがって元の等式は次のように成る.
        \[
            \pr_1 \circ\, (z_y \circ \phi_1)=\pr_1 \circ \,\tilde{r}', \qquad
            \pr_2 \circ\, (z_y \circ \phi_1)=\pr_2 \circ \,\tilde{r}'.
        \]
        なので$X_y \times X_y$の普遍性から$z_y \circ \phi_1=\tilde{r}'=(j_{/Y})_y \circ \tilde{r}$.
    \end{proof}
    %% }}}
\end{proof}

\begin{Remark}
    したがってquotientの定義の幾つかは次のように書き換えられる.
    
    \begin{tabular}{rcl}
        $q$ :: univ. Zariski & $\iff$ & $q$ :: univ. submersive and $j_{/Y}$ :: surjective. \\
        $q$ :: topological & $\iff$ & $q, q^{\cons}$ :: univ. submersive and $j_{/Y}$ :: surjective. \\
        $q$ :: strongly topological & $\iff$ & $q, q^{\cons},j_{/Y}$ :: univ. submersive
    \end{tabular}
\end{Remark}

\begin{Lemma}[\cite{Rydh13} Prop2.4, Remark2.5]
    $q \colon X \to Y$ :: universal Zariski quotientとする.
    以下の時,$q$ :: topological quotient.
    \begin{enumerate}
        \item $q$ :: quasi-compact,
        \item $q$ :: locally of finite presentation,
        \item $q$ :: universally open/closed,
        \item $s$ :: universally open/closed.
    \end{enumerate}
\end{Lemma}
\begin{proof}
    $q^{\cons}$ :: univ. submersiveを示す.
    これには$q^{\cons}$ :: univ. open or univ. closedを示せば十分である.
    なので(i),(ii)については命題(\ref{nontrivial_constop})から分かる.
    (iii)について$q^{\cons}$ :: univ. openは自明.
    (iv)を証明する.

    $s$ :: univ. open/closedと仮定する.
    $j_{/Y}$の定義から,$s$は以下のように分解できる.
    \[
    \begin{tikzcd}
        R \ar[r, "j_{/Y}"]& X \times_Y X \ar[r, "\pr"]& X
    \end{tikzcd}
    \]
    一つ前の命題から,今$j_{/Y}$ :: surjectiveとなっている.
    $U \subseteq X \times_Y X$をopen/closedとすると,
    $\pr(U)=s(j_{/Y}^{-1}(U))$もopen/closed.
    よって$\pr$ :: open/closed map.
    univ. open/closedやsurj.はpullbackで保たれるので,特に$\pr$ :: univ. open/closed.
    $q$ :: univ. submersiveなので,
    命題(\ref{refrecting_property_of_pullback})と合わせて$q$ :: univ. open/closedを得る.
\end{proof}

\begin{Prop}[\cite{Rydh13} Prop2.10]
    $q \colon X \to Y$ :: equivariantとし,
    $f \colon Y' \to Y$による$q$のpullbackを$q' \colon X \times_{Y} Y' \to Y'$とする.

    \begin{enumerate}
        \item $q$ :: topological quotient,ならば$q'$ :: topological quotient.
        \item $f$ :: flatかつ$q$ :: geometric quotient,ならば$q'$ :: geometric quotient.
        \item $f$ :: fpqc or fppf 
            \footnote{faithfully flat and quasi-compactまたはfaithfully flat and locally of finite presentation}
                かつ$q'$ :: topological / geometric / universal geometric quotientならば,
                $q$もそうである.
    \end{enumerate}
    いずれも``topological"を``strongly topological"に,
    ``geometric"を``strongly geometric"に置き換えても成立する.
\end{Prop}

\bibliographystyle{jplain}
\bibliography{../references/stacks_reference}
\end{document}
