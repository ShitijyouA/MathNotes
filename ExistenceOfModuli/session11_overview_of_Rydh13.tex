\documentclass[a4paper, dvipdfmx]{jsarticle}

\usepackage[]{../math_note, enumitem}
\usepackage{xcolor}
\usepackage{graphics}
\usepackage[all, pdf, 2cell, cmtip]{xy}
\usepackage{tikz}
\usetikzlibrary{cd, positioning, arrows}
%% for hyperref {{{
\usepackage[dvipdfmx, colorlinks=true, linkcolor=black]{hyperref}
\usepackage{pxjahyper}
%% }}}

%% environment: question/problem {{{
\usepackage{chngcntr}
\makeatletter
    \newcounter{c@question}
    \counterwithin{c@question}{section}
    \newenvironment{question}[0]%
    {\stepcounter{c@question}\begin{itembox}[l]{問\arabic{section}.\arabic{c@question}}}%
    {\end{itembox}}%
    \newenvironment{question*}[0]%
    {\stepcounter{c@question}\begin{itembox}[l]{問}}% 
    {\end{itembox}}%
\makeatother

\makeatletter
    \newcounter{c@problem}
    \counterwithin*{c@problem}{section}
    \newenvironment{problem}[0]%
    {\stepcounter{c@problem}\begin{itembox}[l]{問題\arabic{section}.\arabic{c@problem}}}%
    {\end{itembox}}%
    \newenvironment{problem*}[0]%
    {\stepcounter{c@problem}\begin{itembox}[l]{問題}}% 
    {\end{itembox}}%
\makeatother
%% }}}

\newenvironment{myenum}[1][\roman*]
{\hfill \vspace{-0.8cm}\begin{enumerate}[label=(#1), labelindent=1cm]}
{\end{enumerate}}

\newcommand{\step}[1]{\paragraph{\bf #1}}

\setenumerate{label=(\roman*),itemsep=3pt,topsep=7pt}

%% category
\newcommand{\Sch}{\mathbf{Sch}}
\newcommand{\Sets}{\mathbf{Sets}}
\newcommand{\Ring}{\mathbf{Ring}}
\newcommand{\Alg}{\mathbf{Alg}}
\newcommand{\Cat}{\mathbf{Cat}}
\newcommand{\Sh}{\mathbf{Sh}}
\newcommand{\PSh}{\mathbf{PSh}}

\newcommand{\Fib}[1]{\cat{Fib}(\cat{#1})}
\newcommand{\cFib}[1]{\cat{cFib}(\cat{#1})}
\newcommand{\sFib}[1]{\cat{sFib}(\cat{#1})}
\newcommand{\FibBP}[1]{\cat{Fib}^{\mathrm{bp}}(\cat{#1})}
\newcommand{\CFG}[1]{\cat{CFG}(\cat{#1})}
\newcommand{\Shv}[1]{\cat{Shv}(\cat{#1})}

\newcommand{\Comma}[2]{#1{\downarrow}#2}
\newcommand{\Lim}{\operatorname{lim}}
\newcommand{\Colim}{\operatorname{colim}}

\newcommand{\kiso}[1][{}]{\overset{#1}{\iso}}
\newcommand{\kequiv}[1][{}]{\overset{#1}{\simeq}}

%% derivation
\newcommand{\shDer}{\Omega}
\newcommand{\modDer}{\Omega}
\newcommand{\Der}{\mathrm{Der}}

%% functor
\newcommand{\ftor}[1]{\underline{#1}}
\newcommand{\ftorSh}{\mathit{Shff}}
\newcommand{\ftorFgt}{\mathit{Fgt}}

%% sites
\newcommand{\Cov}{\operatorname{Cov}}
\newcommand{\et}{\mathrm{et}}
\newcommand{\Et}{\mathrm{Et}}
\newcommand{\ET}{\mathrm{ET}}

%% covers
\newcommand{\covU}{\mathcal{U}}
\newcommand{\covV}{\mathcal{V}}
\newcommand{\covW}{\mathcal{W}}

%% utility
\newcommand{\mnewline}{\mbox{}\newline}
\newcommand{\tp}[2]{\texorpdfstring{#1}{#2}}
\newcommand{\parto}[2]{\mathrel{\mathop{\rightrightarrows}^{#1}_{#2}}}

%% {{{ fibered categories
\newcommand{\fib}[1]{\mathscr{#1}}
\newcommand{\fibA}{\fib{A}}
\newcommand{\fibB}{\fib{B}}
\newcommand{\fibC}{\fib{C}}
\newcommand{\fibD}{\fib{D}}
\newcommand{\fibE}{\fib{E}}
\newcommand{\fibF}{\fib{F}}
\newcommand{\fibG}{\fib{G}}
\newcommand{\fibH}{\fib{H}}
\newcommand{\fibI}{\fib{I}}
\newcommand{\fibJ}{\fib{J}}
\newcommand{\fibK}{\fib{K}}
\newcommand{\fibL}{\fib{L}}
\newcommand{\fibM}{\fib{M}}
\newcommand{\fibN}{\fib{N}}
\newcommand{\fibO}{\fib{O}}
\newcommand{\fibP}{\fib{P}}
\newcommand{\fibQ}{\fib{Q}}
\newcommand{\fibR}{\fib{R}}
\newcommand{\fibS}{\fib{S}}
\newcommand{\fibT}{\fib{T}}
\newcommand{\fibU}{\fib{U}}
\newcommand{\fibV}{\fib{V}}
\newcommand{\fibW}{\fib{W}}
\newcommand{\fibX}{\fib{X}}
\newcommand{\fibY}{\fib{Y}}
\newcommand{\fibZ}{\fib{Y}}
%% }}}

%% {{{ stacks 
\newcommand{\st}[1]{\mathcal{#1}}
\newcommand{\stA}{\st{A}}
\newcommand{\stB}{\st{B}}
\newcommand{\stC}{\st{C}}
\newcommand{\stD}{\st{D}}
\newcommand{\stE}{\st{E}}
\newcommand{\stF}{\st{F}}
\newcommand{\stG}{\st{G}}
\newcommand{\stH}{\st{H}}
\newcommand{\stI}{\st{I}}
\newcommand{\stJ}{\st{J}}
\newcommand{\stK}{\st{K}}
\newcommand{\stL}{\st{L}}
\newcommand{\stM}{\st{M}}
\newcommand{\stN}{\st{N}}
\newcommand{\stO}{\st{O}}
\newcommand{\stP}{\st{P}}
\newcommand{\stQ}{\st{Q}}
\newcommand{\stR}{\st{R}}
\newcommand{\stS}{\st{S}}
\newcommand{\stT}{\st{T}}
\newcommand{\stU}{\st{U}}
\newcommand{\stV}{\st{V}}
\newcommand{\stW}{\st{W}}
\newcommand{\stX}{\st{X}}
\newcommand{\stY}{\st{Y}}
\newcommand{\stZ}{\st{Z}}
%% }}}


\begin{document}
\title{ゼミノート \#11 \\ Overview of \\ ``Existence and properties of geometric quotients"(D.Rydh, 2013)}
\author{七条彰紀}
\maketitle
\tableofcontents
\vspace{10pt}

このノートは,
\cite{Rydh13}を理解することを目的とするゼミのためのノートである.
Algebraic stackについては既に私のノート\cite{SAAlgSt}の内容程度のことを分かっているものとする.

\begin{Remark}
    著者``Rydh"はスウェーデン語で,大体「リード」と発音する.
    参考: \url{https://forvo.com/word/annika_rydh/}
\end{Remark}

\section*{Conventions and Nootations}
    \begin{itemize}[leftmargin=*]
    \item 
        定義は私のノート\cite{SAAlgSt} ch.5 ``Algebraic Stacks and Spaces"にあるものとする.
        特に,diagonal mapがquasi-compact,quasi-separatedであることを仮定しない.
        これは\cite{Rydh13}と同じである.
        algebraic stackという語は基本的に使わず,artin stackを中心的に扱う.

    \item
        artin stackのmoduli spaceというとき,
        特別なalgebraic spaceのことを意味することも有るし,
        artin stackからalgebraic spaceへの特別な射のことを意味することもある.
        (このノートでは後者の意味であることが多い.)

    \item
        unramifiedという語はlocally of finite typeかつformally unramifiedを意味する.
        私のノートではlocally of finite presentationまで要求するので注意せよ.

    \item
        fppf atlas of an artin stackとは,
        surjective, flat
            \footnote{ surjective$+$flat$=$falthfully flatに注意. },
        locally of finite presentationであるような
        algebraic spaceからartin stackへの射を言う.
        取り扱う論文\cite{Rydh13}ではこれをpresentationと呼んでいるが,
        我々はpresentationをquotient stackによるartin stackの表示のことを言う.

    \item
        geometric pointとは
        $\Spec k$ ($k$ :: algebraically closed field)からの射のことである.
    \end{itemize}

\section{事実のまとめ}
    \subsection{stackからschemeまで}
    moduli問題を含む多くの問題では,まずstack(in groupoids)が得られる.
    得られたstackがschemeであればとても取り扱いやすいが,そうなることはめったに無い.
    
    一方でstackとschemeの間には,
    artin stack, Deligne-Mumford(DM) stack, algebraic spaceといった中間的概念がある.
    いつどれになるのか,という条件は既によく研究されていて,
    同値条件も得られている.
    以下でそれらを列挙する.

    まず,得られたmoduli stackがartin stackであるか否かは,
    ``Criteria for Representability"として\cite{SP} ch.91にまとめられている.
    また,artin stack over a scheme :: $\stX$がDeligne-Mumford stackであるかどうかは,
    例えば任意のgeometric pointの自己同型群がreduced finite group schemeであることと同値である
    (\cite{ASS} Thm8.3.3).
    さらに,$\stX$がalgebraic spaceであることは,
    例えば任意のgeometric pointの自己同型群が自明であることと同値である
    (\cite{Con07} Thm2.2.5, \cite{SP} tag 04SZ).
    最後に,algebraic space :: $X$がschemeであるためには,
    例えばrepresentable sheafによる(Zariski) open coveringを持てば十分である(\cite{SP} 01JJ).

    \subsection{coarse moduli spaceが存在するための十分条件}
    しかし,得られたartin stackがalgebraic spaceであることもまためったに無い.
    筆者の感覚では,DM stackで既に綺麗すぎる(too neat)対象である.
    そこで,artin stackをalgebraic spaceやschemeで近似出来ないか,という問題が生まれる.
    この近似をcoarse moduli spaceと呼ぶ.
    なお,これは歴史的な経緯から来た命名であり,
    一般には必ずしもmoduli問題と関係が有るわけではない.

    artin stackがcoarse moduli spaceをもつための条件も90年代から考えられているが,
    必要十分条件を得るには程遠い.
    十分条件として有名なのはKeel-Moriの定理(\cite{KM97})が提示した
    「inertia stackが$\stX$-finite」である.
    当初は多くの追加条件付きで証明されたが,
    \cite{Con05}でbase schemeに関する条件が取り外され,
    最終的に\cite{Rydh13}でartin stackに関する条件が全て取り外された.
    
    他に,gerbeは必ずcoarse moduli spaceを持つ.
    artin stackがgerbeであることは,
    inertia stack :: $\stI_{\stX} \to \stX$が
    flat and locally of finite presentationであることと同値である.

    \subsection{coarse moduli spaceが存在するための必要条件}
    一方で,coarse moduli spaceが存在するための必要条件については
    ほとんど知られていない.
    (\cite{Con05} Cor5.2)では様々な前提条件付きで
    「separated coarse moduli spaceが存在する」と
    「inertia stackが$\stX$-finite」が同値であることを示している.
    一方で\cite{Rydh13}では反例を構成し,
    「inertia stackが$\stX$-proper」さえ必要条件ではないことを示している.

    \subsection{その他のmoduli space}
    また,「inertia stackが$\stX$-finite」より強い条件を課したものとして,
    「quasi-coherent sheafのpushforwardがexact」を追加したtame Artin stackがある.
    これはcoarse moduli spaceがetale localに綺麗なものとなっている.

    違う方向性では,J.Alperが提案したadiquate moduli spaceとgood moduli spaceがある.
    good moduli spaceはquotient mapにはなっていないが,
    GIT quotientに似た優れた性質を持つ(\cite{Alp13}).
    さらに,artin stackがgood moduli spaceを持つための必要十分条件が分かっている
    (\cite{AHLH18}).

\section{論文``Existence and properties of geometric quotients"\tp{\cite{Rydh13}}{}の構成}
    今回取り扱う\cite{Rydh13}で述べられている命題のうち,
    次のものを特に研究する.
    \begin{Thm}
        $\stX$をartin stackとする.
        $\stX$が$\stX$-finite inertia stackを持つならば,
        $\stX$がcoarse moduli spaceを持つ.
    \end{Thm}
    論文は基本的にalgebraic spaceのgroupoidによる商を扱っており,
    最後の\S 6でそれらがstackの言葉に翻訳される.
    この方針は$\cite{KM97}$と同じである.

    \subsection{証明の構成}
    証明の手順を説明する.
    未定義の用語がかなり多くなるが,適宜無視して欲しい.
    \[
    \begin{tikzcd}[arrows=Rightarrow]
        \text{$\stX$は$\stX$-finite inertia stackを持つ.}
            \ar[d, "\text{Thm6.11}"]\\
        \text{$\stX$はfpr etale cover with finite locally free presentation :: $\stW$を持つ.}
            \ar[d, "\text{Thm5.3}"]\\
        \text{$\stW$はstrongly geometric quotientを持つ.}
            \ar[d, "\text{Thm3.19}"]\\
        \text{$\stX$はstrongly geometric quotientを持つ.}
            \ar[d, "\text{Thm3.8}"]\\
            \text{$\stX$はcoarse moduli spaceを持つ.(Thm 6.12)}
    \end{tikzcd}
    \]
    命題番号を見ると,結論に近い部分から述べられていることが分かる.

    また,ここに挙げられていないがcoarse moduli spaceを考える上で重要な命題として次が有る.
    \begin{Thm}[Part of \cite{Rydh13} Thm 3.16]
        $R \rightrightarrows X$をgroupoidとし,
        $q \colon X \to Z$をそのstrongly geometric quotientとする.
        この時,$q$がuniversally openであり,かつ$q$がproperまたはintegralであれば,
        $q$はcategorical quotientでもある.
    \end{Thm}

    全体として,この論文によるD.Rydhの貢献は,
    geometric quotient等の概念とcategorical quotientの概念,
    またdescent condtionなどの概念の関係性を解明した点に有ると思う.

    \subsection{中心的概念}
    上で述べたなかに頻出したとおり,strongly geometric quotientという概念が頻出である.
    この定義を述べよう.
    \begin{Def}[strongly geometric quotient, stack version, \cite{Rydh13} Def 6.1]
        $\stX$をartin stackとし,
        $q$をalgebraic spaceへの射$q \colon \stX \to Z$とする.
        次の条件を満たす時,$q$はstrongly geometric quotientと呼ばれる.
        \begin{itemize}
            \item $q$ :: universal homeomorphism,
            \item the diagonal $\Delta_q$ :: universally submersive,
            \item $q^{\#} \colon \shO_{Z} \to q_*\shO_{\stX}$ :: isomorphism.
        \end{itemize}
        最初の二つが成立する時,$q$はstrongly topological quotientと呼ばれる.
    \end{Def}
    \cite{Rydh13} Prop3.8(とDef 6.1の上の段落)より,これはcategorical quotientでもある.
    見ての通り,これはかなり強い条件である.
    なお,最後の条件が成立するためには,
    $q$のsmooth射によるpullbackがまたcategorical quotientであることが十分である
    \footnote
    {
        特に$q$がuniform categorical quotientであれば十分.
        証明には,\cite{SAAlgSt}で証明した
        $(q_*\shO_{\stX})(U)=\Gamma(U \times_{Z} \stX, \pr_2^{-1}\shO_{\stX})$と,
        関手$\Gamma(-, \shO_{\stX})$がaffine line :: $\affine_{\Z}^1$で表現可能であることを用いる
    }.

\bibliographystyle{jplain}
\bibliography{../references/stacks_reference}
\end{document}
