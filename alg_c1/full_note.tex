\documentclass[a4paper]{jsarticle}
\usepackage{../math_note}
\usepackage[all]{xypic}
\newcommand{\Res}{\operatorname{Res}}

\begin{document}
\section{アフィン曲線}
    \subsection{アフィン空間}
    \begin{Def}[アフィン空間(大雑把な定義)]
        体kについて、
        \[ \affine^n=\affine^n_{k}=k^n \]
        をk上のn次元アフィン空間(Affine space)と呼ぶ。
    \end{Def}

    \subsection{アフィン曲線}
    \begin{Def}
        $f \in k[x, y]-{0}$に対して、その零点集合
        \[ C=\zeros(f)=\{ p \in \affine^2 | f(p)=0 \} \]
        をアフィン曲線と呼ぶ。
        この曲線$C$を他には
        \[ C:f=0 ~\mbox{in}~ \affine^2 \]
        と書く。
    \end{Def}

    他に次の用語を導入する。
    \begin{description}
        \item[Cの定義多項式:] f
        \item[Cの定義方程式:] f=0
    \end{description}

    $f$の$k[x,y]$に於ける既約分解を以下のようにする。
    \begin{gather*}
        f=cf_1^{e_1} \dotsm f_i^{e_i} \dotsm f_n^{e_n} \\
        (c \in k,~ f_i : k[x, y]\mbox{の既約元},~ e_i \geq 1)
    \end{gather*}

    このとき、$C=\cup_{i=1}^{n}{\zeros(f_i)}$となる。

    \begin{proof}
    \begin{eqnarray*}
        &{}&    p \in C \\
        &\iff&  f(p) =0 \\
        &\iff&  c \prod{f_i(p)}=0
    \end{eqnarray*}
        $k$は整域なので、
    \begin{eqnarray*}
        &\iff& \exists i, f_i(p)=0 \\
        &\iff& \exists i, p \in \zeros(f_i) \\
        &\iff& p \in \cup_{i=1}^{n}{\zeros(f_i)}
    \end{eqnarray*}
    
    \end{proof}

    このようにして得られた$C_i=\zeros(f_i)$達を$C$の既約成分、
    $C=\cup {C_i}$を$C$の既約分解と呼ぶ。
    また、$f$が既約多項式の時は$C$を既約曲線と呼ぶ。
    $\deg C=\deg f$とし、$d=\deg f$の時には$C$を$d$次曲線と呼ぶ。

    \subsubsection{重複度}
    以下、$C:f=0 ~\mbox{in}~ \affine^2$とする。
    今、
    \[ f=\sum_{i,j}{a_{ij} x^i y^j} ~ (a_{ij} \in k)\]
    とする。$p_0=(a,b) \in \affine^2$について、
    \[ x=(x-a)+a, y=(y-b)+b\]
    を代入して$(x-a), (y-b)$についてまとめると、fは次のように変形できる。
    \begin{eqnarray*}
        f_k&=&\sum_{i+j=k}{c_{ij}(x-a)^i(y-b)^j} \\
        f&=&\sum_{k}{f_k}
    \end{eqnarray*}
    この表示を$f$の$p_0$におけるテイラー展開と呼ぶ。

        \subsubsection{偏微分}
        一般の体$k$について偏微分を定義できる。ここでは$\affine^n$を考える。
        \begin{Def}[偏微分]
        \[ f=\sum_{i_1, \dots, i_n} {a_{ i_1 \dots i_n } x^{i_1}_1 \dotsm x^{i_n}_n} \in k[x_1, \dots, x_n]\]
        に対して、$f$の偏微分を、
        \[ \frac{\partial f}{\partial x_{j}}=\sum{i_j \cdot a_{ i_1 \dots i_n } (x^{i_1}_1 \dotsm x^{i_j-1} \dotsm x^{i_n}_n) } \]
        と定義する。
        \end{Def}

        標数0の体については、次が成り立つ。
        \[
            c_{i_1 \dots i_n}
            =\frac{1}{i_1! \dots i_n!} f_{x_{i_1}^{i_1} \dots x_{i_n}^{i_n}}(p_0)
            =\frac{1}{i_1! \dots i_n!} \frac{\partial^{i_1+\dots+i_n} f}{\partial x_{i_1}^{i_1} \dots \partial x_{i_n}^{i_n}}(p_0)
        \]
        ただし$c_{i_1 \dots i_n}$は$f$を$(x_1-p_0^{(1)}), (x_2-p_0^{(2)}), \dots$の多項式として表した時の係数であることに注意。
        特に$n=2$の時は次のよう。
        \[
            c_{i_1 \dots i_n}
            =\frac{1}{i! j!} f_{x^{i} y^j}(p_0)
            =\frac{1}{i! j!} \frac{\partial^{i+j} f}{\partial x^i \partial y^j}(p_0)
        \]

    \subsection{接線と特異点}
    点$p_0:=(a,b)$とおく。他の点は$p$で表す。
    \begin{Def}[$C$の$p$に於ける重複度]
        $m_p(C)=\min\{k : f_k(p) \neq 0\}$
    \end{Def}
    $m=m_p(C)$のとき、$p$を$C$の$m$重点と呼ぶ。

    再び2次元アフィン空間を考える。
    $f_0=c_{00}=f(p)$から、
    \[ m_p(C)>0 \iff f(p)=0 \iff p \in C \]
    が成り立つ。

    $p=(x,y) \in C$とする。
    \[f_1(p)=f_x(p_0)(x-a)+f_y(p_0)(y-b)=c_{01}(x-a)+c_{10}(y-b)\]
    よって、
    \[ m_p(C)=1 \iff f_1(p) \neq 0 \iff f_x(p) \neq 0~\mathrm{or}~f_y(p) \neq 0 \]。

    \begin{Def}[単純点と特異点]
        $m_p(C)=1$の時$p$を$C$の単純点、$m_p(C) > 1$の時$p$を$C$の特異点と呼ぶ。
    \end{Def}
    単純点$p$における$C$の接線は定義方程式$f_1=0$で定められる。これを
    \[ T_p(C)=\zeros(f_1) \subset \affine^2 \]
    と書く。

    $p$が特異点の時はどうだろうか。
    以下では$m=m_p(C) \geq 2$とする。
    このとき、
    \[ f=\underbrace{f_0+\dots +f_{m-1}}_{=0}+f_m+f_{m+1}+\dots \]
    となっている。
    実は$k$が代数閉体ならば、$f_m$は次のように$x,y$の一次式の積に分解される(後に示す)。
    つまり、
    \begin{gather*}
        f_m(x)=\prod^{e}_{i=1} (\alpha_{i}(x-a)+\beta_{i}(y-b))^{m_{i}} \\
        \alpha_{i},\beta_{i} \in k,~ m_i \geq 1,~ \sum^e_{i=1}{m_i}=m
    \end{gather*}
    と表すことが出来る。
    $\alpha_i, \beta_i$は単数倍で等しいものをまとめられるので、
    \[
         \left|
        \begin{array}{cc}
            \alpha_i & \beta_i \\
            \alpha_j & \beta_j \\
        \end{array}
        \right|
        \neq 0
        ~(i \neq j)
    \]
    として良い(?)。

    この時、$e$本の直線$\alpha_{i}(x-a)+\beta_{i}(y-b)=0$を$p$における$C$の接線とする。
    また、$m_i$をその重複度と呼ぶ。

    \begin{Def}
        $m=e$ (i.e. $\forall i, m_i=1$)の時、
        $p$を$C$の通常特異点(ordinary singular point)と呼ぶ。
        通常2重点を結節点と呼ぶ。
    \end{Def}

    \subsection{斉次多項式}
    $k$を体とする。
    $f \in k[X_1, \dots, X_n]=k[\mathbb{X}]$は、
    \[
        f=\sum{c_{i_0 \dots i_n} X_0^{i_0} \dots X_n^{i_n}}
        =\sum{c_{\mathbb{I}} \mathbb{X}^{\mathbb{I}}} ~(\mathbb{I}=(i_0, \dots, i_n))
    \]
    と表される。
    $\mathbb{X}^{\mathbb{I}}$を単項式、
    $|\mathbb{I}|=i_0+\dots+i_n$をその次数と呼ぶ。
    $f(\neq 0)$のに現れる次数が全て等しい時、$f$を斉次多項式と呼ぶ。

    \[
        f=\sum_{d \geq 0}{ \left( \sum_{|\mathbb{I}|=d}{c_{\mathbb{I}} \mathbb{X}^{\mathbb{I}}} \right)}
    \]
    ()内を$f_d$と置けば$f=\sum_{d \geq 0}{f_d}$となる。
    $f_d$はそれぞれ$d$次の斉次多項式。そこで、この表示を$f$の斉次分解と呼ぶ。

    次の補題は2次斉次多項式と1変数多項式が同型であることを言っている。
    \begin{Lemma}
        $F(x, y) \in k[x,y]$を$d$次の斉次多項式とする。
        $f(t)=F(1, t) \in k[t]$とおくと、以下が成り立つ。
        \[ F(x, y)= x^d f(\frac{y}{x})\]
    \end{Lemma}
    \begin{proof}
        \begin{eqnarray*}
            F(x, y) &=& \sum{a_{ij} x^{i} y^{j}} \\
            f(t)    &=& \sum{a_{ij} t^{j}} \\
            f\left( \frac{y}{x} \right)&=& \sum{a_{ij} x^{-j} y^{j}} \\
        \end{eqnarray*}
        $F(x, y)$は$d$次の斉次多項式だから$i+j=d$。よって、
        \begin{eqnarray*}
            x^d f\left( \frac{y}{x} \right)=\sum{a_{ij} x^{i} y^{j}}
        \end{eqnarray*}
        
    \end{proof}

    \begin{Prop}
        $F(x, y) \in k[x,y]$を$d$次の斉次多項式とする。
        $k$が代数的閉包の時、$F(x, y)$は次の形に分解される。
        \begin{gather*}
            F(x, y)=\prod^{e}_{i=1} (\alpha_{i}(x-a)+\beta_{i}(y-b))^{d_{i}} \\
            ( \alpha_{i},\beta_{i} \in k,~ d_i \geq 1,~ \sum^e_{i=1}{d_i}=d )
        \end{gather*}
    \end{Prop}
    \begin{proof}
        $f(t)=F(1, t)$とおく。$\bar{k}=k$だから、$f(t)$は一次式に分解される。
        \begin{gather*}
            f(t) = c \prod_{i=1}^{l}{(t-\gamma_{i})} \\
            (\gamma_{i} \in k, c \in k^{\times})
        \end{gather*}
        ただし$l=\deg f$。先ほどの補題より、以下の様にして命題が成り立つ。
        \begin{eqnarray*}
            F(x, y) \\
            &=& x^{d} f\left( \frac{y}{x} \right) \\
            &=& c x^{d} \prod_{i=1}^{l}{ \left( \frac{y}{x} - \gamma_{i} \right)} \\
            &=& c x^{d-l} \prod_{i=1}^{l}{ \left( y - \gamma_{i}x \right)} \\
            &=& (1 \cdot x + 0 \cdot y)^{d-l} \prod_{i=1}^{l}{ \left( c^{\frac{1}{l}}y - c^{\frac{1}{l}}\gamma_{i}x \right)} \\
        \end{eqnarray*}
        
    \end{proof}

    \begin{Prop}
        $F(x, y) \in k[x,y]$を$d$次の斉次多項式とする。
        $(\lambda, \mu) \in k^2, (\lambda, \mu) \neq (0, 0)$に対して、
        \[
            F(\lambda, \mu)=0 \iff (\lambda y -\mu x) | F(x, y)
        \]
    \end{Prop}
    \begin{proof}
        ($\impliedby$)は自明なので($\implies$)を示す。

        $\lambda, \mu$の両方が同時に0になることは無いので、
        $\lambda \neq 0$とする。
        $\mu \neq 0$としても以降の文字をただ置き換えれば証明が出来る。

        \begin{eqnarray*}
            &{}& F\left(\lambda, \mu\right)=0 \\
            &\iff& \lambda^{d} F\left(1, \frac{\mu}{\lambda}\right)=0 \\
            &\iff& f\left(\frac{\mu}{\lambda}\right)=0 \\
            &\iff& \exists g \in k[t] ~s.t.~
                f\left(\frac{\mu}{\lambda}\right) = \left(t-\frac{\mu}{\lambda}\right) g\left(\frac{\mu}{\lambda}\right) \\
            &{}&\mbox{以下、行頭には$\exists g$があると思え。補題から次が成り立つ。} \\
            &\iff& F(x, y)=x^d f(t)=x^d \left(\frac{y}{x}-\frac{\mu}{\lambda}\right) g\left(\frac{y}{x}\right) \\
            &\iff& F(x, y)=\frac{1}{\lambda} (\lambda y - \mu x)\cdot x^{d-1}g\left(\frac{y}{x}\right) \\
        \end{eqnarray*}
        ここで、$\deg g = \deg f -1 \leq \deg F -1 =d-1$。
        よって$x^{d-1}g\left(\frac{y}{x}\right) \in k[x, y]$($x$の指数は全て0以上)。

        
    \end{proof}

    \subsection{直線との交点数}
    $C=\zeros(f), f \in k[x, y] \setminus \{0\}, p=(a, b) \in C$とする。
    $p$を通る直線$L$に対して、$C$と$L$との$p$における交点数$i(C, L; p)$を以下のとおり定める。
    
    \begin{Def}
        直線Lのパラメータ表示を以下のようにおく。
        \begin{gather*}
            L: (x, y)=(a+ \lambda t, b+ \mu t) \\
            (\lambda, \mu \in k, (\lambda, \mu) \neq (0, 0))
        \end{gather*}
        このとき、交点数$i(C, L; p)$は、次のよう。
        \[
            i(C, L; p)
              :=\operatorname{ord}_{t} f(a+ \lambda t, b+ \mu t)
              :=\max \{d : t^d | f(a+ \lambda t, b+ \mu t) \}
        \]
    \end{Def}
    
    $p \in L$なので$i(C, L; p) \geq 1$。
    さらに、この定義は$L$のパラメータ表示によらないことが示せる。
    
    \begin{Prop}
          $C$を曲線、$L$を点$p \in C$を通る直線とする。
        \[
            i(C, L; p) \geq m_p(C)
        \]
        特に、次が成り立つ。
        \[
            i(C, L; p) > m_p(C) \iff \mbox{Lはpに於けるCの接線の一つ}
        \]
    \end{Prop}
    
    \begin{proof}
          座標全体を平行移動して$p=(0, 0)$とする。
          このとき$L$のパラメータ表示は、
          \begin{gather*}
              L: (x, y)=(\lambda t, \mu t) \\
              (\lambda, \mu \in k, (\lambda, \mu) \neq (0, 0))
          \end{gather*}
          となる。$m:=m_p(C)$とおくと、$p$における$f$のテイラー展開は以下の様。
          \[
              f=\sum_{k \geq m}{ \left( \sum_{i+j=k}{c_{ij} x^{i} y^{j}} \right) }
          \]
          $L$上の点では、
          \begin{eqnarray*}
              f
              &=&\sum_{k \geq m}{ x^k \left( \sum_{i+j=k}{c_{ij} \lambda^{i} \mu^{j}} \right) } \\
              &=&\sum_{k \geq m}{ x^k f_k(\lambda, \mu) }
          \end{eqnarray*}
          $k \geq m$から、$i(C, L; p) \geq m$。
          さらに、$i(C, L; p) > m \iff f_m(\lambda, \mu)=0$だから、
          斉次因数定理より次が成り立つ。
          \begin{eqnarray*}
              &{}& f_m(\lambda, \mu)=0 \\ 
              &\iff& (\lambda y - \mu x) | f_m(\lambda, \mu) \\
              &\iff& \zeros(\lambda y - \mu x)\mbox{は$f$の接線の一つ(接線の定義を見よ)} \\
              &\iff& L\mbox{は$f$の接線の一つ}
          \end{eqnarray*}
          
    \end{proof}
        
\section{射影曲線}
    \subsection{射影空間}
    $\affine^{n+1} \setminus \{ 0 \}$において、
    $\mathbf{a}=(a_0, \dots, a_n)$, $\mathbf{b}=(b_0, \dots, b_n)$に対し、
    以下のように同値関係を入れる(同値関係であることは自明)。
    \[
    \mathbf{a} \sim \mathbf{b} \iff \exists \lambda \in k^{\times} ~s.t.~ \lambda \mathbf{a}=\mathbf{b}
    \]
    そこで、$n$次元射影空間を以下で定める。
    \[
    \mathbb{P}^n := (\affine^{n+1} \setminus \{ 0 \}) / \sim
    \]
    
    点$A \in \proj^n$に対し、
    その代表元として$\mathbf{a}=(a_0, \dots, a_n)$をとる。
    このとき、$A=(a_0 : \dots : a_n)$と表し、
    $a_i$達を$A$の斉次座標と呼ぶ。
    全ての$a_i$が0になる点は無い。
    
    各$i=0, 1, \dots, n$に対し、
    \[
        \sqcup_i:=\{(a_0: a_1: \dots: a_n) | a_i \neq 0\} \subset \proj^n
    \]
    このとき、$\proj^n = \bigcup_{i=0}^{n}{\sqcup_i}$となる。
    この$\{\sqcup_i \}_{i=0}^{n}$をアフィン開被覆と呼ぶ。
    
    \begin{Lemma}
        各$i$に対し、$\phi_i$を
        \begin{eqnarray*}
            \phi_{i} :
            \sqcup_i &\to& \affine^n \\
            (a_0: \dots: a_i: \dots: a_n) &\mapsto& (a_0/a_i, \dots, a_n/a_i)
        \end{eqnarray*}
        とおく。これは全単射で、
        \begin{eqnarray*}
            \psi_{i} :
            \affine^n &\to& \sqcup_i \\
            (a_0, \dots, a_n) &\mapsto& (a_0: \dots: \underset{\text{$i$番目の要素}}{1}: \dots: a_n)
        \end{eqnarray*}
        がその逆写像である。
    \end{Lemma}
    
    \begin{proof}
        全単射の定義にしたがって調べれば良い。 
    \end{proof}
    
    零点集合$\zeros(X_i)=\{(a_0: a_1: \dots: a_n) | a_i=0\}$を$\sqcup_i$の
    無限遠超平面と言う。これはそれぞれ$\sqcup_i$の補集合で、
    $\affine^n \simeq \sqcup_i = \proj^n \setminus \zeros(X_i)$が成り立つ。
    零点集合はぞれぞれ$\sqcup_i$と無限遠で交わる。

    \subsection{射影曲線}
    \begin{Def}[射影曲線]
        体$k$上の斉次多項式$F \in k[X, Y, Z]$について、
        以下で定まる集合を射影曲線と呼ぶ。
        \[ C:=\zeros(F)=\{p \in \proj^2 : F(p)=0 \}\]
        $\deg C:=\deg F=d$を$C$の次数と呼び、
        また、$C$を$d$次曲線と呼ぶ。
    \end{Def}

    これはwell-definedである。
    なぜなら任意の点$p \in \proj^2$と、任意の$\lambda \in k^{\times}$について、
    $F(\lambda p)=(\lambda^{\deg F}) F(p)$だからである。
    したがって$F(p)=0$の解集合は点$p$の斉次座標のとり方によらない。
    これは$F$が斉次多項式であることから成り立つ。
    逆に、このように射影曲線$C$がwell-definedであるためには
    $F$は斉次多項式でなくてはならない。

    \begin{Prop}
        2点$\mathbf{a}, \mathbf{b} \in \proj^2$
        (ただし$\mathbf{a} \neq \mathbf{b}$)を通る直線はただ一つであり、
        \footnote{$\lambda \mathbf{a}$,$\mu \mathbf{b}$も同じ2点を表すということを考えれば、これは自明ではない。}
        その定義多項式$\bar{F}$は以下で与えられる。
        \[
        \bar{F}=
          \left|
              \begin{array}{ccc}
                  X & Y & Z \\
                  a_0 & a_1 & a_2 \\
                  b_0 & b_1 & b_2
              \end{array}
          \right|
        \]
        ここで$\mathbf{a}=(a_0 : a_1 : a_2), \mathbf{b}=(b_0 : b_1 : b_2)$とした。
    \end{Prop}
    \begin{proof}
        主張に有る$\bar{F}$は$\mathbf{a}$か$\mathbf{b}$を代入すると
        同じ行を2つもつ行列式になるから、0になる。
        また、$\bar{F}$は一次斉次多項式である。
        したがって$\bar{F}$は$\mathbf{a}, \mathbf{b}$を通る直線の定義多項式の一つである。
        以下、このような直線がただ一つであることを示す。

        直線の定義多項式$F$は1次斉次多項式だから、
        $F(X, Y, Z)=\alpha X + \beta Y + \gamma Z$のように表される。
        写像$\varepsilon$を以下で定義する。
        \begin{eqnarray*}
            \varepsilon : \{ \mbox{k上の1次斉次多項式全体} \} &\to& k^2 \\
            F &\mapsto& 
                    \left[
                      \begin{array}{c}
                          F(\mathbf{a})\\
                          F(\mathbf{b})
                      \end{array}
                    \right]
        \end{eqnarray*}
        $\varepsilon$で$F$を送った先が$\mathbf{0}$であれば
        $F$は2点$\mathbf{a}, \mathbf{b}$を通る。
        すなわち、定義多項式は$\ker \varepsilon$の元である。
        すでに述べたように、$\bar{F} \in \ker \varepsilon$となっている。

        $\varepsilon$で$F$を送った先をもう少し考えると、次のようになる。
        \[
            F \mapsto 
                    \left[
                      \begin{array}{c}
                          F(\mathbf{a}) \\
                          F(\mathbf{b})
                      \end{array}
                    \right]
                    =
                    \left[
                      \begin{array}{ccc}
                          a_0 & a_1 & a_2 \\
                          b_0 & b_1 & b_2
                      \end{array}
                    \right]
                    \left[
                      \begin{array}{c}
                          \alpha \\ \beta \\ \gamma
                      \end{array}
                    \right]
        \]
        
        $\mathbf{a} \neq \mathbf{b}$から、
        $\operatorname{rank} \left[ \begin{array}{ccc} a_0 & a_1 & a_2 \\ b_0 & b_1 & b_2 \end{array} \right]=2$である。
        したがって$\dim \ker \varepsilon=3-2=1$となる。
        つまり、$\ker \varepsilon$は$\bar{F}$を有る一つのパラメータで変化させたもの全体。
        実際、$\bar{F}$を$k^{\times}$倍したものも$\ker \varepsilon$の元である
        \footnote{$\bar{F}$の$X$の係数だけ変化させても$\ker \varepsilon$の元になる、といった可能性を排除するための議論だった。}。
        以上の議論から、$\ker \varepsilon$の元は$k^{\times}$倍を除いて一意。
        したがって、2点$\mathbf{a}, \mathbf{b}$を結ぶ直線$\zeros(F)$は一意。
        
    \end{proof}

    \begin{Def}
        $\proj^2$内の直線全体のなす集合を$\check{\proj^2}$と書く。
        \[ \check{\proj}^2 := \{\zeros(\alpha X+ \beta Y+ \gamma Z) : (\alpha, \beta, \gamma) \in k^3 \setminus\{0\} \}\]
        これを双対射影空間と呼ぶ。
    \end{Def}
    実際、
    $\check{\proj}^2 \ni \zeros(\alpha X+ \beta Y+ \gamma Z) \mapsto (\alpha : \beta : \gamma) \in \proj^2$
    は全単射。
    
    問。素数$p$について$\proj^2_{\mathbb{F}_p}$に含まれる直線は何本か。

    \subsection{多項式の斉次化・非斉次化}
    以下では$k$を体、
    $\mathbb{X}=(X_0, \dots, X_n)$\footnote{$(n+1)$個の不定元。},
    $\mathbb{Y}=(Y_1, \dots, Y_n)$\footnote{$n$個の不定元。}とおく。

    \begin{Def}[非斉次化]
    \begin{eqnarray*}
        \alpha :
            k[\mathbb{X}] &\to& k[\mathbb{Y}]\\
            F(\mathbb{X}) &\mapsto& F(1, Y_1, \dots, Y_n)
    \end{eqnarray*}
    これを$X_0$に関する非斉次化と呼ぶ。
    \end{Def}
    これは代入なので環の準同型写像である。

    \begin{Def}[斉次化]
    \begin{eqnarray*}
        \beta :
            k[\mathbb{Y}] &\to& k[\mathbb{X}] \\
            f(\mathbb{Y}) &\mapsto& X_0^{\deg f} f \left( \frac{X_1}{X_0}, \dots, \frac{X_n}{X_0} \right)
    \end{eqnarray*}
    これを$X_0$に関する斉次化と呼ぶ。
    \end{Def}
    これは次に示すように準同型写像でない。

    \begin{Prop}
        $f \in k[\mathbb{Y}]$に対して$\beta (f)$は斉次多項式。
        さらに、$f$の斉次分解を$f=\sum^{d}_{k=0}{f_k(\mathbb{X})}$とした時
        \[ \beta(f)(\mathbb{X})=\sum^{d}_{k=0}{X_0^{d-k} f_k(\mathbb{X}')}\]
        となる。ただし$d:=\deg f$, $\mathbb{X}'=(X_1, \dots, X_n)$\footnote{$X_0$を$\mathbb{X}$から消した。}とした。
    \end{Prop}
    \begin{proof}
        「さらに、」以降の主張から前半の主張は明らか。
        $f(\mathbb{Y})=\sum_{0 \leq k \leq d}{ \sum_{|\mathbb{I}|=k}{c_\mathbb{I} \mathbb{Y}^{\mathbb{I}}} }$
        とする。$\beta(f)$は次のようになる。
        \begin{eqnarray*}
            \beta(f)(\mathbb{X})
            &=& X_0^d \sum^{d}_{k=0}{
                \left( \sum_{|\mathbb{I}|=k}{c_\mathbb{I}
                \left(\frac{X_1}{X_0}\right)^{i_1} \cdots \left(\frac{X_n}{X_0}\right)^{i_n}} \right)} \\
            &=& \sum^{d}_{k=0}{
                \left( \sum_{|\mathbb{I}|=k}{X_0^{d-|\mathbb{I}|} \cdot c_\mathbb{I} {\mathbb{X}'}^{\mathbb{I}}} \right)} \\
            &=& \sum^{d}_{k=0}{X_0^{d-k} f_k(\mathbb{X}')}
        \end{eqnarray*}
        
    \end{proof}

    \subsubsection{$\alpha, \beta$の関係}
    証明は略すが、$\alpha(\beta(f))=f$が成り立つ。
    しかし$\beta(\alpha(F))=F$は一般に成立しない。

    \begin{Lemma}
        斉次多項式$F \in k[\mathbb{X}]$に対し、
        ある$e \geq 0$が存在して次式が成り立つ。
        \[ F(\mathbb{X}) = X_0^e \cdot \beta(\alpha(F(\mathbb{X})))\]
    \end{Lemma}
    \begin{proof}
        $d:=\deg F$として、
        \[ F(\mathbb{X}) = \sum_{|\mathbb{I}|=d}{c_{\mathbb{I}} X_0^{i_0} \cdots X_n^{i_n}}\]
        と表せる。この時、
        \[ \alpha(F)(\mathbb{X}) = \sum_{|\mathbb{I}|=d}{c_{\mathbb{I}} 1^{i_0} X_1^{i_1} \cdots X_n^{i_n}}\]
        明らかに$\deg \alpha(F) \leq d$なので、この差を$e$と置く。
        つまり$\deg \alpha(F)=d-e$とする。
        すると$d-\sum_{1 \leq j \leq n}{i_j}=i_0$より、以下のようになる。
        \begin{eqnarray*}
            X_0^e \cdot \beta(\alpha(F(\mathbb{X}))) \\
            &=& X_0^e
            \left(
                X_0^{d-e}
                \sum_{|\mathbb{I}|=d}{c_{\mathbb{I}} \left(\frac{X_1}{X_0}\right)^{i_1} \cdots \left(\frac{X_n}{X_0}\right)^{i_n}}
            \right) \\
            &=&
            \left(
                \sum_{|\mathbb{I}|=d}{c_{\mathbb{I}} X_0^{i_0} X_1^{i_1} \cdots X_n^{i_n}}
            \right) \\
            &=& F(\mathbb{X})
        \end{eqnarray*}
        等号が成立するのは$i_0=0 \implies c_{\mathbb{I}}=0$の時。
        
    \end{proof}

    \subsection{アフィン曲線の射影化}
    \begin{Def}
        多項式$f \in k[x,y]$によって定まるアフィン曲線$C:=\zerosa(f) \subset \affine^2$に対し、
        $\zerosp(\beta(f)) \subset \proj^2$を$C$の射影化と呼ぶ。
        ただし、$\beta$は$Z$に関する斉次化、
        すなわち$\beta : f(x, y) \mapsto Z^{\deg f}f \left( \frac{X}{Z},\frac{Y}{Z}\right)$である。
    \end{Def}

    \begin{Def}
        斉次多項式$F \in k[X, Y, Z]$により定まる射影曲線$C:=zerosp(F)$に対し、
        $zerosa(\alpha(F)) \subset \affine^2$をその$Z \neq 0$のアフィン部分と呼ぶ。
        ただし$\alpha$は$Z$に関する非斉次化、
        すなわち$\alpha : F(X, Y, Z) \mapsto F(x, y, 1)$である。
    \end{Def}
    アフィン部分には他に$X$に関するもの、$Y$に関するものがある。

    \begin{Prop}
        斉次多項式$F \in k[X, Y, Z]$に対して、
        \[ zerosp(F) \cap \sqcup_c = \psi_c(zerosa(\alpha(F)))\]
        ただし$\sqcup_c$と$\psi_c$は補題1で定義したものである。
    \end{Prop}
    \begin{proof}
        $\sqcup_c \ni p=(a : b : 1)$をとる。
        \begin{eqnarray*}
            &{}&    p \in zerosp(F) \\
            &\iff&  F(a,b,1)=0 \\
            &\iff&  \alpha(F)(a,b)=0 \\
            &\iff&  \phi_c(p) \in zerosa(\alpha(F)) \\
            &\iff&  p \in \psi_c(zerosa(\alpha(F))) \\
        \end{eqnarray*}
        
    \end{proof}

    \begin{Lemma}
        $f \in k[x, y]$に対して、
        \[ \overline{\psi(zerosa(f))}=zerosp(\beta(f)) \]
        ただし、左辺はZariski位相での閉包である。
    \end{Lemma}
    この補題は利用しないので証明もしない。

\subsection{特異点}
    \begin{Def}[射影曲線の特異点]
        斉次多項式$F$により定まる射影曲線$C:=zerosp(F)$において、
        $p \in C$が$C$の特異点であるとは、
        $p$を含むアフィン開被覆における$C$のアフィン部分が$p$に於いて特異点を持つこと
        と定める。

        したがって、$p \in C$が$C$の特異点であるとは、
        $p \in \sqcup_i$のとき、
        アフィン部分$zerosa(\alpha(F))$が$\phi_i(p)$に於いて特異点を持つことである。
    \end{Def}

    \begin{Lemma}
        $p$が$zerosp(F)$の特異点である。
        $\iff$ $F_X(p)=F_Y(p)=F_Z(p)=0$
        \footnote{$F_X$は斉次多項式$F$を$X$について偏微分したものである。$F_Y$なども同様。}
    \end{Lemma}
    \begin{proof}
        $\sqcup_c \ni p=(a:b:1)$をとり、$f:=\alpha(F)=F(x, y, 1)$とおく。
        $C:=zerosa(f)$が特異点$p$を持つとは、$f$の斉次分解$\{ f_k \}$について
        $m_p(C)=\min\{k : f_k(p)=0\}>1$ということ。
        したがって、
        \[ f_x(a,b)=f_y(a,b)=0 \]
        ここで$F_X(x, y, 1)=f_x(x,y)$, $F_Y(x, y, 1)=f_y(x,y)$だったから、
        \[ F_X(a, b, 1)=F_Y(a, b, 1)=0 \]
        が成り立つ。これは特異点の定義と同値。

        さらにここでオイラーの公式\[ XF_X+YF_Y+ZF_Z=(\deg F)F \]を用いると、
        \[ a \cdot F_X(p)+b \cdot F_Y(p)+1 \cdot F_Z(p)=(\deg F)F(p)=0 \]
        だから、$F_Z(a, b, 1)=0$も出る。
        逆に、$F_X(p)=F_Y(p)=F_Z(p)=0$は明らかに$F_X(p)=F_Y(p)=0$を含む。
        
    \end{proof}

    \subsection{接線}
    \begin{Def}[射影曲線の接線]
        射影曲線$C$の点$p \in C$における接線を、
        $p$を含むアフィン開被覆の$\phi(p)$における接線の射影化として定める。
    \end{Def}

    \begin{Lemma}
        斉次多項式$F$について、$p \in C:=zerosp(F)$がCの非特異点(単純点)であるとき、
        $p$における$C$の接線は次式で定まる。
        \[ F_X(p)X+F_Y(p)Y+F_Z(p)Z=0 \]
    \end{Lemma}
    \begin{proof}
        $\sqcup_c \ni p=(a:b:1)$をとり、$f:=\alpha(F)$とする。
        $C$の$\sqcup_c$におけるアフィン部分$zerosa(f)$への$\phi_c(p)$に於ける接線は
        次で定まる。
        \[ f_x(a,b)(x-a)+f_y(a,b)(y-b)=0 \]
        fの定義より、
        \[ F_X(a,b,1)(x-a)+F_Y(a,b,1)(y-b)=0 \]
        これを斉次化すれば
        \[ F_X(a,b,1)(X-aZ)+F_Y(a,b,1)(Y-bZ)=0 \]
        オイラーの公式を用いれば、結論が得られる。
    \end{proof}

    例として$F=XZ-Y^2$を取ると、
    これの点$p=(a:b:c)$における接線は$cX-2bY+aZ=0$となる。
    標数2の体に於いては、$cX+aZ=0$となり、これは点$(0:1:0)$を常に通る。
    接線が定点を通る曲線をstrange曲線と呼ぶが、
    これは以下の定理の通り、かなり限られた状況のものしか無い。
    \begin{Them}[Samuel]
        非特異射影曲線でstrangeのものは、
        直線(自明な場合)か標数2の2次曲線に限る。
    \end{Them}
    証明はHartshorn, IV, Theorem 3.9にある。

    \subsection{直線との交点数}
    $A=(a_0:a_1:a_2), B=(b_0:b_1:b_2) \in \proj^2$とおく。
    $A, B$を通る直線$L$のパラメータ表示として、
    \[ L: (X:Y:Z)=sA+tB=(s a_0+t b_0:s a_1+t b_1:s a_2+t b_2) \]
    をとる。
    斉次多項式$F$に$L$のパラメータ表示を代入して得られる多項式を
    \[ \Phi(s,t)=F(s a_0+t b_0, s a_1+t b_1, s a_2+t b_2) \]
    と置く。

    $L$上の点$P$は$(s_0, t_0) \neq (0, 0)$によって$P=s_0 A+t_0 B$と表される。
    このとき、$C \cap L$に於ける$C$と$L$の交点数を
    \[ I(C, L; P)=\max \{ m : (s_0 t-t_0 s)^m | \Phi(s,t) \} \]
    で定義する。これはwell-definedである。

    \paragraph{問}
    射影曲線$C=\zeros(F)$と直線$L$に対して、$L \not \subset C$とする。
    体$k$が代数的閉包ならば、
    \[ \sum_{P \in C \cap L}{I(C, L; P)}=\deg F \]
    となる。これを示せ。ヒントはテイラー展開。

    \begin{Prop}
        $P \in \proj^2$を含むアフィン開被覆での、$C$と$L$のアフィン部分を$C_0, L_0$とすれば
        \[ I(C, L; P)=i(C_0, L_0; \phi(P)) \]
        が成立する。
    \end{Prop}
    \begin{proof}
        \textbf{定義の確認}~~
        適当に座標変換して$L=zerosp(Y), P=(0:0:1)$とする。
        $f(x, y)=\alpha(F)=F(x, y, 1)$と置けば、
        $C:=zerosp(F)$と$L$のアフィン部分は
        \[ C_0:=zerosa(f), L_0:=zerosa(y) \]
        である。
        $L_0$のパラメータ表示は$(x, y)=(t, 0)$とすれば、
        $P=(0:0:1)$に対応する点は$t=0$で与えられる。
        アフィン曲線の交点数の定義より、
        $i(C_0, L_0; \phi(P))=\max \{ k : t^k|f(t,0) \}$
        
        \paragraph{$F$の分解}
        ここで、$F$を$Y$の多項式として整理する。
        つまり、$F$を多項式環$k[X, Z][Y]$の元として見る。
        $d:=\deg F$とおき、$F_i \in k[X, Z]$をi次斉次多項式とする。
        \[ F=F_d(X, Z)+F_{d-1}(X, Z)Y + \dots + F_0(X, Z) Y^d \]
        すると、
        \[ f(t, 0)=F(t, 0, 1)=F_d(t, 1) \]
        となるから、
        \[ i(C_0, L_0; \phi(P))=\max \{ k : t^k|F_d(t,1) \} \]
        となる。

        \paragraph{$\Phi$の表示を見る}
        一方$L$のパラメータ表示として$(X, Y, Z)=(t:0:s)$をとれば、
        $P(=(0:0:1))$に対応するのは$(s_0,t_0)=(1,0)$が与える点。
        したがって$\Phi(s, t)=F(t, 0, s)=F_d(t, s)$となり、
        あとは単なる計算で結論が得られる。
        \begin{eqnarray*}
            &{}&    I(C, L; P) \\
            &=&     \max \{ m : (s_0 t-t_0 s)^m | \Phi(s,t) \} \\
            &=&     \max \{ k : t^k|F_d(t,s) \} \\
            &=&     \max \{ k : t^k|F_d(t,1) \} \\
            &=&     i(C_0, L_0; \phi(P)) \\
        \end{eqnarray*}
        
    \end{proof}

\subsection{射影変換}
正則行列$A \in GL(3, k)$による線形写像
\begin{eqnarray*}
    A : \affine^3 &\to& \affine^3 \\
    \begin{bmatrix}
        x \\ y \\ z
    \end{bmatrix}
    &\mapsto&
    A
    \begin{bmatrix}
        x \\ y \\ z
    \end{bmatrix}
\end{eqnarray*}
が定まる。
任意の$\lambda  \in k$に対し、
\[
    \lambda
    \begin{bmatrix}
        x \\ y \\ z
    \end{bmatrix}
    \mapsto
    A
    \begin{bmatrix}
        \lambda x \\ \lambda y \\ \lambda z
    \end{bmatrix}
\]
となるので、正則行列$A$によって
\begin{eqnarray*}
    \phi_{A} : \proj^2 &\to& \proj^2 \\
    (a:b:c) &\mapsto& \psi_{Z}(A \cdot {}^t[a~b~c])
\end{eqnarray*}
が定まる。これはwell-definedである。

明らかに以下が成り立つ。
\begin{eqnarray*}
    \phi_{E} &=& id_{\proj^2} \\
    \phi_{AB} &=& \phi_{A} \circ \phi_{B} ~(\forall A, B \in GL(3,k))
\end{eqnarray*}
下の式から正則行列Aについて$\phi_A$は全単射となり、
\[ (\phi_A)^{-1}=\phi_{A^{-1}} \]となる。
特に射影変換全体
\[ PGL(2, k):=\{ \phi_A : A \in GL(3,k) \} \]
は群を成す。これを射影変換群と呼ぶ。

\begin{Lemma}
    正則行列$A$が定める射影変換$\phi_A$を考える。
    3点$P_1, P_2, P_3 \in \proj^2$に対して、
    $P_1, P_2, P_3$が同一直線上に有ることと
    $\phi_A(P_1), \phi_A(P_2), \phi_A(P_2)$が同一直線上に有ることは同値。
\end{Lemma}
\begin{proof}
    $P_i=(p_{i0}:p_{i1}:p_{i2}) \in \proj^2$に対して
    $\mathbf{p}_i={}^t[ p_{i0}, p_{i1}, p_{i2} ]$とおく。
    この時、アフィン空間に於いて2点を通る直線は行列式で書ける、
    という命題から、以下のように証明が出来る。
    \begin{eqnarray*}
        &{}&    \mbox{$P_1, P_2, P_3$が同一直線上に有る} \\
        &\iff&  \det[\mathbf{p}_1~\mathbf{p}_2~\mathbf{p}_3]=0 \\
        &\iff&  (\det A) (\det[\mathbf{p}_1~\mathbf{p}_2~\mathbf{p}_3])=0 \\
        &\iff&  \det[A\mathbf{p}_1~A\mathbf{p}_2~A\mathbf{p}_3]=0 \\
        &\iff&  \mbox{$\phi_A(P_1), \phi_A(P_2), \phi_A(P_2)$が同一直線上に有る} \\
    \end{eqnarray*}
    
\end{proof}

\begin{Prop}[Four Points Lemma]
    4点$P_1, P_2, P_3, P_4 \in \proj^2$はどの3点も同一直線上にないとする。
    $O_1=(1:0:0), O_2=(0:1:0), O_3=(0:0:1), O_4=(1:1:1)$
    とするとき、
    \[ \phi(P_i)=O_i ~(i=1,2,3,4)\]
    となる射影変換はただ一つ存在する。
\end{Prop}
\begin{proof}
    $P_i$と$\mathbf{p}_i$と前のように定める。
    $B'=[\mathbf{p}_1~\mathbf{p}_2~\mathbf{p}_3]$
    と置けば、$P_i$はどの3つも同一直線上にないので$B'$は正則。
    $B=(B')^{-1}$と置くと、
    \[ B[\mathbf{p}_1~\mathbf{p}_2~\mathbf{p}_3]=B B'=E \]
    なので、$i=1,2,3$について$\phi_B(P_i)=O_i$となる。

    $\phi(P_4)$を考える。そのために
    \begin{gather}
    B \mathbf{p}_4=
    \begin{bmatrix}
        \lambda_1 \\ \lambda_2 \\ \lambda_3
    \end{bmatrix}
    =\sum_{i=1}^{4}{(\lambda_i \cdot B \mathbf{p}_i)} \label{astarisk}
    \end{gather}
    とする。
    この時、$\lambda_i \neq 0$である。
    実際、例えば$\lambda_1$とすると
    \[ \phi_B(P_2)=O_2,~ \phi_B(P_3)=O_3,~ \phi_B(P_4)=(0:\lambda_2:\lambda_3) \]となり、
    これらは直線$X=0$上にある。
    補題よりこれは3点$P_2, P_3,P_4$が同一直線上に有ることと同値であり、
    したがって仮定に反する。
    そこで正則行列Aを
    \[
    A=
    \begin{bmatrix}
        1/\lambda_1& {}& {} \\
        {}& 1/\lambda_2& {} \\
        {}& {}& 1/\lambda_3 \\
    \end{bmatrix}
    B
    \]
    と置けば、$\phi_A$が求める射影変換。
    実際に計算してみると、
    \begin{eqnarray*}
        &\phi_A(P_1)=\psi_{c}(A \mathbf{p}_1)=(\lambda_1:0:0)&=O_1 \\
        &\phi_A(P_2)=\psi_{c}(A \mathbf{p}_2)=(0:\lambda_2:0)&=O_2 \\
        &\phi_A(P_3)=\psi_{c}(A \mathbf{p}_3)=(0:0:\lambda_3)&=O_3 \\
        &\phi_A(P_4)=\psi_{c}(A \mathbf{p}_4)=(1 : 1: 1)    ~&=O_4
    \end{eqnarray*}

    もしも$A' \in GL(3,k)$によって$\phi_{A'}(P_i)=O_i$が成立したとする。
    この時ある定数$\alpha \in k^{\times}$によって$A=\alpha A'$となることを示す。
    この時、0でない定数$\mu_i$によって、
    \[
        A'[\mathbf{p}_1~\mathbf{p}_2~\mathbf{p}_3~\mathbf{p}_4]
        =
        \begin{bmatrix}
            \mu_1& 0& 0& \mu_4 \\
            0& \mu_2& 0& \mu_4 \\
            0& 0& \mu_3& \mu_4
        \end{bmatrix}
    \]
    と書ける。
    \[ \frac{1}{\mu_4}A'\mathbf{p}_4=\frac{1}{\mu_1}A'\mathbf{p}_1+\frac{1}{\mu_2}A'\mathbf{p}_2+\frac{1}{\mu_3}A'\mathbf{p}_3 \]
    また、式(\ref{astarisk})の左に$\frac{1}{\mu_4} A'B'$を掛けると、
    \[ \frac{1}{\mu_4}A'\mathbf{p}_4=\frac{\lambda_1}{\mu_4}A'\mathbf{p}_1+\frac{\lambda_2}{\mu_4}A'\mathbf{p}_2+\frac{\lambda_3}{\mu_4}A'\mathbf{p}_3 \]
    仮定より、$A'\mathbf{p}_i$は基底になっているから、係数が一致して
    \[ \frac{\lambda_i}{\mu_4}=\frac{1}{\mu_i} ~~(i=1,2,3,4) \]
    が成立する。したがって、
    \[ \mu_4 A [\mathbf{p}_1~\mathbf{p}_2~\mathbf{p}_3]=A'[\mathbf{p}_1~\mathbf{p}_2~\mathbf{p}_3] \]
    と、$B'=[\mathbf{p}_1~\mathbf{p}_2~\mathbf{p}_3]$が正則であることから、
    \[ \mu_4 A=A' \]
    が成立する。よって$\phi_A=\phi_{A'}$である。
    これで一意性が言えた。
    
\end{proof}

\begin{Lemma}
    斉次多項式$F \in k[X, Y, Z]$により定まる射影曲線$C:=zerosp(F)$を$\phi_A$で写した像は
    \[\phi_A(C)=\zeros(F \circ A^{-1}) \]
    さらに$\deg(\phi_A(C))=\deg C$である。
\end{Lemma}
\begin{proof}
    任意の$P \in \proj^2$に対して、以下のようになる。
    \[ P \in \phi_A(C) \iff \phi^{-1}_A(P) \in C \iff F(A^{-1} P)=0 \iff P \in \zeros(F \circ A^{-1}) \]
    さらに、一般に行列$M$について$\deg F \geq \deg (F \circ M)$
    であることを用いて後半を証明する。
    \[
        \underbrace{\deg (F \circ A^{-1}) \leq \deg F}_{M=A^{-1}}
        =
        \underbrace{\deg (F \circ A^{-1} \circ A) \leq \deg(F \circ A^{-1})}_{M=A}
    \]
    
\end{proof}

\begin{Def}
    $F, G$を斉次多項式とし、$C:=\zeros(F), D:=\zeros(G)$とおく。$C,D$が射影同値であるとは、
    ある$A \in GL(3,k), \lambda \in k^{\times}$によって
    \[ G=\lambda F \circ A^{-1} \]
    となることである。
\end{Def}
\paragraph{注意}
$k$が代数的閉包であるときは$\lambda'=\lambda^{1/\deg F} \in k$となるので、
$F \circ (\lambda' A)^{-1}=\lambda F \circ A^{-1}$が成り立つ。
つまり、定数$\lambda$を行列$A$に纏めることが出来る。

\paragraph{例: 平行移動}
アフィン空間における平行移動$(x,y) \mapsto (x-a, y-b)$を、
射影化$\psi_{Z}$によって射影変換にする。
\[ \phi_{A} : (X:Y:Z) \mapsto (X-aZ:Y-bZ:Z)\]
このような射影変換$\phi_{A}$を与える正則行列$A$を求めよう。

Four Points Lemmaより、射影変換は4点の写った先が決まれば一意に定まる。
4点として$(0,0),(0,1),(1,0),(1,1)$をとり、これを射影化してから$\phi_A$で写す。
するとその値から、
\[
    A=
    \begin{bmatrix}
        1& {}& -a \\
        {}& 1& -b \\
        {}& {}& 1
    \end{bmatrix}
\]
と定まる。

\section{ベズーの定理}
    \subsection{終結式}
        \begin{Lemma}
            UFD $R$上の多項式$f, g \in R[x] \setminus R$に対して、
            以下は同値。
            \begin{enumerate}
                \renewcommand{\labelenumi}{(\roman{enumi})}
                \item $\exists h \in R[x] \setminus R ~s.t.~ h|f ~\mbox{and}~ h|g$
                \item $\exists A,B \in R[x] ~s.t.~ \deg A<\deg g,~ \deg B<\deg f ~\mbox{and}~ Af+Bg=0$
            \end{enumerate}
        \end{Lemma}
        \begin{proof}
            \textbf{(i) $\implies$ (ii)}~~
                仮定より$f=hB, g=-hA$を満たす$A, B \in  R[x]$が存在する。
                すると明らかに$Af+Bg=0$となる。
                多項式の次数に関する部分も、
                $\deg h \neq 0$から$\deg A=\deg g - \deg h<\deg g$のようにして導かれる。

            \textbf{(ii) $\implies$ (i)}~~
                仮定から、$Af=-Bg$となる$A, B \in R[x]$が存在する。
                $g$の全ての既約因子は$Af$を割り切る。
                このとき$\deg A < \deg g$から、
                $g$の1次以上の既約因子であって$f$を割り切るものが有る。
                それを$h$とすれば(i)の条件を満たす。
            
        \end{proof}

        \begin{Def}
            多項式$f, g \in k[t]$を
            \[ f(t)=\sum_{0 \leq i \leq p}{a_{i}t^{i}}, g(t)=\sum_{0 \leq j \leq q}{b_{j}t^{j}} \]
            とおく。これに対して以下のように$(p+q)$次正方行列\footnote{シルベスター行列と呼ばれる。}を定める。
            \[
                M(f,g; t)=
                \begin{bmatrix}
                    a_0&    a_1&    \cdots&     a_q \\
                    {}&     a_0&    a_1&        \cdots&     a_q \\
                    {}&     {}&     \ddots&     \ddots&     {}&     \ddots& \\
                    {}&     {}&     {}&         a_0&        a_1&    \cdots&    a_q \\
                    b_0&    b_1&    \cdots&     b_p \\
                    {}&     b_0&    b_1&        \cdots&     b_p \\
                    {}&     {}&     \ddots&     \ddots&     {}&     \ddots& \\
                    {}&     {}&     {}&         b_0&        b_1&    \cdots&    b_p
                \end{bmatrix}
            \]
            この時、
            \[ \Res(f,g; t)=\det M(f,g; t) \]
            を$f,g$の$t$に関する終結式と呼ぶ。
        \end{Def}

    \begin{Them}
        $f,g \in R[t] \setminus R$に対して、以下は同値。
        \begin{enumerate}
            \item $\exists h \in R[t]~s.t.~ h|f~\mbox{and}~h|g$
            \item $\Res(f,g; t)=0$
        \end{enumerate}
    \end{Them}
    \begin{proof}
        補題より、以下のような$A,B$があって$Af+Bg=0$を満たす。
        \[ A(t)=\sum_{0 \leq j \leq q-1}{A_{j}t^{j}},B(t)=\sum_{0 \leq i \leq p-1}{B_{i}t^{i}} \]
        さて、$Af+Bg=0$を計算してみると、以下のようになる。
        \[ \sum_{0 \leq d \leq p+q-1}{ \left\{ \sum_{i+j=d}{(a_i A_j + b_j B_i)} \right\}}=0 \]
        各項の係数を見ると以下が分かる。
        \begin{eqnarray*}
        (1.) \iff
        \exists [A_j]_{0 \leq j \leq q-1}, [B_i]_{0 \leq i \leq p-1} ~s.t.~ \\
        a_0 A_0 + b_0 B_0 &=&0 \\
        a_1 A_0 + b_1 B_0+a_0 A_1 + b_0 B_1&=&0 \\
        \vdots \\
        a_{p+q-1} A_0 + \cdots + b_0 B_{p+q-1}&=&0
        \end{eqnarray*}
        まとめて表せば次のようになる。
        \begin{gather*}
            (1.) \iff \\
            \exists [A_0, \dots, A_{q-1}, B_0, \dots, B_{p-1}] \in R^{p+q} \setminus \{ \mathbf{0} \} ~s.t.~
            [A_0, \dots, A_{q-1}, B_0, \dots, B_{p-1}] M(f,g;t)=\mathbf{0}
        \end{gather*}
        次元定理より$\Res(f,g;t)=0$と$\operatorname{ker}M(f,g;t) \neq \{ \mathbf{0} \}$は同値である
        \footnote{次元定理より、$M(f,g;t)$が正則ならばkerの次元は0であるから、kerの元は自明な物に限る}。
        したがって$(1.) \iff (2.)$が示された。
        
    \end{proof}

    \begin{Prop}
        \[
            \forall f,g \in R[t] \setminus R,~\exists A, B \in R[t]~s.t.~
            \deg A < \deg g, \deg B < \deg f ~\mbox{and}~ Af+Bg=\Res(f,g;t)
        \]
    \end{Prop}
    \begin{proof}
        各$i=2,3,\dots,p+q$に対して、$M(f,g;t)$の各$i$列目の$t^i$倍を1列目に加える。
        すると次のようになる。
        \[
            M'=
            \begin{bmatrix}
                f&          a_1&    \cdots&     a_q \\
                tf&         a_0&    a_1&        \cdots&     a_q \\
                {}&         {}&     \ddots&     \ddots&     {}&     \ddots& \\
                t^{q-1}f&   {}&     {}&         a_0&        a_1&    \cdots&    a_q \\
                g&          b_1&    \cdots&     b_p \\
                tg&         b_0&    b_1&        \cdots&     b_p \\
                {}&         {}&     \ddots&     \ddots&     {}&     \ddots& \\
                t^{p-1}g&   {}&     {}&         b_0&        b_1&    \cdots&    b_p
            \end{bmatrix}
        \]
        この操作は基本操作であるから、行列式を変えない。$M'$を第1列で余因子展開する。
        \begin{eqnarray*}
            \Res(f,g;t) &=& \det M' \\
            &=& (f A_0+tf A_1+\dots+t^{q-1}f A_{q-1})
            + (g B_0+tg B_1+\dots+t^{p-1}g B_{p-1}) ~~ (A_i, B_j \in R)\\
            &=& (A_0+t A_1+\dots+t^{q-1} A_{q-1})f+(B_0+t B_1+\dots+t^{p-1} B_{p-1})g \\
            &=& Af+Bg
        \end{eqnarray*}
        
    \end{proof}

    \paragraph{注意}
    $f,g \in k[x_1, \dots, x_n, t]$に対して、$\Res(f,g; t)$は$f,g$が成すイデアルに属す。

    \subsubsection{例}
    $F=X^3-YZ^2, G=X^2-YZ \in k[X, Y, Z]$を考える。
    $C:=\mathcal{Z}_p(F), D:=\mathcal{Z}_p(G) \in \proj^2$とおき、
    $C$と$D$の交点を求める。
    \[
        R(X,Z):=\Res(F,G; Y) \footnote{Yについての射影化と解釈できる?}=
        \begin{bmatrix}
            -Z^3& X^3 \\
            -Z& Z^2
        \end{bmatrix}
        =X^2 Z (X-Z)
    \]
    したがって$C,D$の交点は$X=0, Z=0, X-Z=0$の上に有る。

    例えば$\mathcal{Z}(X) \cap C \cap D$の属す交点を$P=(a:b:c)$とする。
    $0=F(a,b,c)=-bc^2, 0=G(a,b,c)=-bc$なので$bc=0$となるが、$(a:b:c)\neq(0:0:0)$なので、
    $P=(0:0:1)$または$P=(0:1:0)$となる。
    同様に$Z=0, X-Z=0$についても計算して、$C \cap D=\{ (0:0:1),(0:1:0),(0:1:0) \}$となる。

    \subsection{弱ベズーの定理}
    \begin{Lemma}
        kを無限体、斉次多項式$F,G \in k[X,Y,Z]$とし、
        $m:=\deg F, n:=\deg G$と置く。
        この時、$R(X,Y):=\Res(F,G; Z)$は$mn$次斉次多項式
    \end{Lemma}
    \begin{proof}
        主張は$R(tX,tY)=t^{mn} \cdot R(X,Y)$と同値なので、これを考える。
        計算のため、
        \begin{eqnarray*}
            F&=&\sum_{0 \leq i \leq m}{a_{i} Z^{m-i}} \\
            G&=&\sum_{0 \leq j \leq n}{b_{j} Z^{n-j}}
        \end{eqnarray*}
        とおく。
        この時、$a_{i}, b_{j}$はそれぞれ$k[X,Y]$に属す$i$次斉次多項式と$j$次斉次多項式である。
        したがって、
        \begin{eqnarray*}
            a_{i}(tX,tY)&=&t^{i} \cdot a_{i}(X,Y) \\
            b_{j}(tX,tY)&=&t^{j} \cdot b_{j}(X,Y)
        \end{eqnarray*}
        が成り立つ。

        このことを使うと、$R(tX,tY)$は次のようになっている。
        \[
        R(tX,tY)=
        \begin{vmatrix}
            a_0&    ta_1&   \cdots&     t^m a_m \\
            {}&     a_0&    t a_1&      \cdots&     t^m a_m \\
            {}&     {}&     \ddots&     \ddots&     {}&     \ddots& \\
            {}&     {}&     {}&         a_0&        t a_1&  \cdots&    t^m a_m \\
            b_0&    t b_1&  \cdots&     t^n b_n \\
            {}&     b_0&    t b_1&      \cdots&     t^n b_n \\
            {}&     {}&     \ddots&     \ddots&     {}&     \ddots& \\
            {}&     {}&     {}&         b_0&        t b_1&  \cdots&    t^n b_n
        \end{vmatrix}
        \]
        この右辺の各行にそれぞれ
        $t^{0}=1, t^{1}=t, \dots, t^{n-1}, t^0, t^1, \dots, t^{m-1}$
        を掛け、左辺にもこれらをまとめて掛ける。
        \[
        R(tX,tY)\cdot t^{0+1+\dots+(m-1)+0+1+\dots+(n-1)}=
        \begin{vmatrix}
            a_0&    t a_1&  \cdots&     t^m a_m \\
            {}&     t a_0&  t^2 a_1&    \cdots&         t^{m+1} a_m \\
            {}&     {}&     \ddots&     \ddots&         {}&             \ddots& \\
            {}&     {}&     {}&         t^m a_0&        t^{m+1} a_1&    \cdots&    t^{m+n-1} a_m \\
            b_0&    t b_1&  \cdots&     t^n b_n \\
            {}&     t b_0&  t^2 b_1&    \cdots&         t^{n+1} b_n \\
            {}&     {}&     \ddots&     \ddots&         {}&             \ddots& \\
            {}&     {}&     {}&         t^{m-1} b_0&    t^m b_1&        \cdots&    t^{m+n-1} b_n
        \end{vmatrix}
        \]
        右辺の各列はそれぞれ$t^{0}=1, t^{1}=t, \dots, t^{m+n-1}$
        でくくり出し、$t^{\dots} \cdot R(X,Y)$の形にすることが出来る。
        \[
            R(tX,tY) \cdot t^{\frac{m(m-1)}{2}+\frac{n(n-1)}{2}}=R(X,Y) \cdot t^{\frac{(m+n)(m+n-1)}{2}}
        \]
        そして、
        \[
            \frac{(m+n)(m+n-1)}{2}-\left( \frac{m(m-1)}{2}+\frac{n(n-1)}{2} \right)=mn
        \]
        より、$R(tX,tY)=t^{mn} \cdot R(X,Y)$が成り立つ。

        
    \end{proof}

    \begin{Lemma}
        $k$が無限体であるとき、
        斉次多項式$F \in k[X,Y,Z] \setminus \{0\}$に対して
        $\proj^2_{k} \setminus \mathcal{Z}(F)$は空でない。
    \end{Lemma}
    \begin{proof}
        対偶を示す。
        \[ \forall P \in \proj^2,~ F(P)=0 \implies F=0 \]
        $F \in (k[X,Y])[Z]$と見て、
        \[ F=\sum_{i=0}^{d}{G_i Z^{d-i}} \]
        と書く。
        ただし$d:=\deg F$で、$G_i$は$k[X,Y]$に属す$i$次斉次多項式である。

        任意の$P=(a:b:c) \in \proj^2_{k}$に対して、$F(a,b,c)=0$であるとする。
        任意の$a,b \in k$に対し、
        \[f(Z):=F(a,b,Z)=\sum_{i=0}^{d}{G_i(a,b) Z^{d-i}}\]
        は$k[Z]$の関する多項式である。

        任意の$c \in k \setminus \{0\}$に対して$f(c)=0$となるから、
        \[ \#(\mathcal{Z}(f(Z)))=\#(k \setminus \{0\})=\infty \]
        となる。
        ここで$f(Z) \neq 0$と仮定すると、
        \[ \#(\mathcal{Z}(f(Z))) \leq \deg f(Z) < \infty \]
        となってしまうので$f(Z)=0$が分かる。
        \[ \forall i,~ \forall a, b \in k,~ G_i(a,b)=0 \]

        さて、$G_i(X,Y)$の$\proj^1_{k}$における零点集合$\mathcal{Z}(G_i)$を考える。
        $G_i$は任意の$a,b \in k$に対して$G_i(a,b)=0$となるから、
        \[ \#(\mathcal{Z}_p(G_i))=\# \proj^1=\infty \]
        ここで$G_i \neq 0$と仮定すると、斉次因数定理より
        \[ \#\mathcal{Z}_p(G_i) \leq \deg G_i=i< \infty \]
        となってしまうので$G_i=0$

        合わせて、$F=0$が示された。
        
    \end{proof}
    なお、$k$が無限体でない時はこれは成り立たない。
    例えば$k=\mathbb{F}_2$の時、
    $F(X,Y,Z)=(X-Y)(Y-Z)(Z-X)$とおくと$F \neq 0$にも関わらず
    $\mathcal{Z}(F)=\proj^2_{\mathbb{F}_2}$となる。

    \begin{Prop}(弱ベズーの定理)
        kを無限体、斉次多項式$F,G \in k[X,Y,Z]$により定まる曲線を
        $C:=\mathcal{Z}(F), D:=\mathcal{Z}(G) \in \proj^2$とし、
        $m:=\deg F, n:=\deg G$とする。
        もし$F,G$に共通因子がないならば、
        \[ |C \cap D| \leq mn \]
        が成り立つ。
    \end{Prop}
    \begin{proof}
        $|C \cap D|>mn$であると仮定し、矛盾を導く。
        $C \cap D \supset \{ P_1, P_2, \dots, P_{mn+1} \}(i \neq j \implies P_i \neq P_j)$とおく。
        さらに、$P_i$と$P_j$を通る直線を$L_{ij}$とする。

        $k$は無限体であるから、点$O$として
        \[ O \not \in C \cup D \cup \bigsqcup_{i \neq j}{L_ij} \]
        となるものが取れる。
        この点$O$が$(0:0:1)$になるように$\proj^2$全体を射影変換し、
        各$P_i$も射影変換したものにラベルを貼り直しておく。

        $F,G \in (k[X, Y])[Z]$とみて、
        \[ F=\sum_{0 \leq i \leq m}{a_{i} Z^{m-i}}, G=\sum_{0 \leq j \leq n}{b_{j} Z^{n-j}} \]
        とおく。ただし$a_i, b_j \in k[X, Y]$であって、$\deg a_i=i, \deg b_j=j$である。
        すると、$C,D \not \ni 0$より$0 \neq F(O)=a_0, 0 \neq G(O)=b_0$が成り立つ。
        したがって$R(X, Y):=\Res(F,G; Z)$とおくと、$R(X,Y) \neq 0$となる。

        準備をする。$(a,b) \in k^2 \setminus \{ (0,0) \}$を取る。
        この時、以下が成り立つ。
        \begin{eqnarray*}
            &{}& \exists c \in k,~ (a:b:c) \in C \cap D \\
            &\iff& \exists c \in k,~ F(a,b,c)=G(a,b,c)=0 \\
            &\implies& F(a,b,Z), G(a,b,Z)\mbox{は共通因子をもつ。} \\
            &\iff& R(a,b)=\Res ( F(a,b,Z), G(a,b,Z); Z )=0 \\
            &\iff& (aY-bX) | R(X,Y)
        \end{eqnarray*}
        ただし、3行目の$\implies$は$k$が代数的閉包の時には逆も成り立つ。
        また、4行目は前の定理を、そして5行目は斉次因数定理を用いている。
        
        $P_i=(a_i:b_i:c_i)$とおくと、$P_i \in C \cap D$だから、すでに示したとおり、
        \[ (a_i Y - b_i X) | R(X,Y) \]
        ここで、$L_{ij}$の定義式は
        \[
            \begin{vmatrix}
                a_i& b_i& c_i \\
                a_j& b_j& c_j \\
                X  & Y  & Z
            \end{vmatrix}
            =0
        \]
        $O \not \in L_{ij}$なので、
        \[
            0 \neq
            \begin{vmatrix}
                a_i& b_i& c_i \\
                a_j& b_j& c_j \\
                0  & 0  & 1
            \end{vmatrix}
            =
            \begin{vmatrix}
                a_i& b_i \\
                a_j& b_j \\
            \end{vmatrix}
        \]
        したがって$\{ (a_i Y - b_i X) \}$の各元は単数倍で一致しない。
        つまり$\{ (a_i Y - b_i X) \}$のそれぞれが異なる$R(X,Y)$の1次因子である。
        ゆえに
        \[ \deg R(X,Y) \geq |\{ (a_i Y - b_i X) \}| = mn+1\]
        となり、補題に反する。
        
    \end{proof}

    \subsubsection{注意}
        体の拡大を考える。
        $\mathcal{Z}_{k}(F):={(a:b:c) \in k^3 : F(a,b,c)=0}$
        とおくと、一般に斉次多項式$F \in k[X,Y,Z]$について
        \[ K/k::\mbox{体の拡大} \implies \mathcal{Z}_K(F) \supset \mathcal{Z}_k(F) \]
        例として$F=X^2+Y^2+Z^2$と$\bar{\mathbb{Q}}/\mathbb{Q}$を考えよ。

    \subsection{ベズーの定理}
    \begin{Them}(ベズーの定理)
    $F,G,C,D,m,n$の定義は今までと同じようにする。
    $C \cap D=\{P_{1},P_{2}, \dots, P_{r} \}$とする。
    基礎体$k$が代数的閉包であるとき、
    各$P_{i}$について交点数$I_{R}(C,D; P_i)$が定義され、
    以下が成立する。
    \[ \sum_{i=1}^{r}{I_{R}(C,D; P_i)}=mn \]
    \end{Them}
    \begin{proof}
        まず、$P_{i}=(a_i:b_i:c_i) \in \proj_{k}^2$とおく。
        基礎体$k$が代数的閉包であるから、
        $R(X,Y)=\Res(F,G; Z)$は次のように一次式の積に分解される。
        \begin{eqnarray*}
            R(X,Y)&=&\lambda \prod_{i=1}^{r}(a_i Y - b_i X)^{m_i} \\
            mn&=&\sum_{i=1}^{r}{m_i} \\
        \end{eqnarray*}
        ただし$\lambda \in k^{\times}, m_i \geq 1$としている。
        点$P_i$における交点数は$I_{R}(C,D; P_i)=m_i$と定義される。
        定理の成立は明らか。
        
    \end{proof}

    \subsubsection{例}
    $F=X^3-YZ^2,G=X^2-YZ$をとり、交点数を求めてみる。
    $R(X,Y)=X^3 Y (Y-X)$となるので、計算すると
    \[ L_{12}=\mathcal{Z}(X-Z), L_{23}=\mathcal{Z}(X-Y),L_{31}=\mathcal{Z}(X) \]
    となる。
    取りうる点$O \not \in C \cap D \cap \bigcup{L_{ij}}$として$O=(1:0:1)$がある。
    これを$(0:0:1)$に写す射影変換は、例えば
    \[
        A=
        \begin{bmatrix}
            0& 0& 1 \\
            0& 1& 0 \\
            1& 0& 0
        \end{bmatrix}
    \]
    で定まる$\phi_A$である。
    この射影変換で各交点と曲線を変換する。
    \begin{eqnarray*}
        F'&=&F \circ A^{-1} =Z^3-YX^2 \\
        G'&=&G \circ A^{-1} =Z^2-YX \\
        C'&=&\mathcal{Z}(F') \\
        D'&=&\mathcal{Z}(G') \\
        P'_{1}&=&(0:1:0) \\
        P'_{2}&=&(1:1:1) \\
        P'_{3}&=&(1:0:0) \\
    \end{eqnarray*}
    改めて$R(X,Y)$を計算すると、
    $R(X,Y)=\underbrace{X^3}_{P'_{1}} \underbrace{(X-Y)}_{P'_{2}} \underbrace{Y^2}_{P'_{3}}$となる。
    よって、
    \begin{eqnarray*}
        I_R(C',D'; P'_{1})&=&3 \\
        I_R(C',D'; P'_{2})&=&1 \\
        I_R(C',D'; P'_{3})&=&2
    \end{eqnarray*}
    と計算できる。

    別のやり方としては$\Res(F,G; X)$を計算しても良い。

    $\Res(F,G; Z)$なら、$\Res(F,G; Z)=0$は$C \cap D$を$Z$軸上に射影した時の
    $C \cap D$の各元が満たす方程式。これは終結式を計算する際に選ぶ変数の幾何学的意味。

    \subsubsection{(弱)ベズーの定理の応用}
    \begin{Prop}
        $F,G \in k[X,Y,Z]$を斉次多項式とし、
        $C:=\mathcal{Z}_p(F),D:=\mathcal{Z}_p(G),
        m:=\deg F, n:=\deg G$とおく。
        $G$が既約多項式とすると、
        \[ \#(C \cap D) > mn \]
        ならば$C \subset D$である。
        さらに$m=n$ならば$C=D$で、
        $m>n$ならば$C=C' \cup D$を満たす$(m-n)$次曲線$C'$が存在する。
    \end{Prop}
    \begin{proof}
        弱ベズーの定理から、$F,G$は共通因子を持つ。
        しかも$G$が既約なので、
        ある$F' \in k[X,Y,Z]$が存在して$F=F'G$が成立する。
        したがって$m \geq n$となる。
        さらに詳しく、$m=n$ならば$F' \in k^{\times}$なので$C=D$が成り立つ。
        $m>n$なら$C'=\mathcal{Z}(F')$とおけば$C=C' \cup D$となる。
        
    \end{proof}

    \subsection{線形系}
    自然数$d$に対して、
    \[ \Lambda_d=\{ F \in k[X,Y,Z] : \mbox{$F$は斉次多項式であり$\deg F=d$} \} \cup \{ 0 \} \]
    これは$k$上の線形空間となる。
    この$\Lambda_d$(または、付随する射影空間$(\Lambda_d \setminus \{0\})/k^{\times}$)を
    次数$d$の完備線形系と呼ぶ。
    また、これの部分空間は、単に線形系と呼ぶ。
    この時、$\dim \Lambda_d=\frac{1}{2}(d+1)(d+2)$である。

    次数$d$の線形系$\Lambda(\subset \Lambda_d)$と
    $S \subset \proj^2$に対して、
    \[ \Lambda(S):=\{ F \in \Lambda : \forall P,~F(P)=0 \} \]
    と定義する。

    \begin{Lemma}
        $\Lambda(S)$は線形空間で、
        \[ \dim \Lambda(S) \geq \dim \Lambda - \#S \]
        さらに``="の時、
        \[ S' \subset S \implies \dim \Lambda(S) = \dim \Lambda - \#S \]
    \end{Lemma}
    \begin{proof}
        もし$s:=\#S=\infty$なら$\Lambda(S)=\{0\}$となるので$\#S<\infty$とする。
        この時、選択公理を仮定せずとも$S$の元を整列できるので、
        $S=\{ P_1, \dots, P_s \}, P_i=(p_{i0}: p_{i1}: p_{i2})$とおく。
        この設定の上で、次のように線形写像$\phi_S$を定義する。
        \begin{eqnarray*}
            \phi_S : \Lambda &\to& k^{\oplus s} \\
            F &\mapsto& (F(p_{i0}, p_{i1}, p_{i2}))_{1 \leq i \leq s}
        \end{eqnarray*}
        すると、この時$\operatorname{ker}\phi_S=\Lambda(S)$である。
        よって$\Lambda(S)$は$S$の部分空間で、しかも
        \[ \dim \Lambda(S) \geq \dim \Lambda - s \]
        が成り立つ。

        さらに、$S' \subset S, s':=\#S'$とするとき、
        以下の図式が可換となる。
        \[
            \begin{xy}
                (0,20) *{\Lambda}="la",
                (20,20) *{k^{\oplus s}}="ks",
                (20,0) *{k^{\oplus s'}}="ksd",
                \ar@{->}^{\phi_S} "la";"ks"
                \ar@{->}^{\pi} "ks";"ksd"
                \ar@{->}^{\phi_{S'}} "la";"ksd"
            \end{xy}
        \]
        そして、以下の様になる。
        \begin{eqnarray*}
            &{}& \dim \Lambda(S) = \dim \Lambda - s \\
            &\iff& \phi_S :: \mbox{全射} \\
            &\implies& \phi_{S'} :: \mbox{全射} \\
            &\iff& \dim \Lambda(S') = \dim \Lambda - s' \\
        \end{eqnarray*}
        
    \end{proof}

    \begin{Prop}
        与えられた$\frac{1}{2}d(d+3)$個の点を通る$d$次の射影曲線が存在する。
    \end{Prop}
    \begin{proof}
        与えられた点の集合を$S$とする。仮定より$\#S=\frac{1}{2}d(d+3)$である。
        補題より
        \[ \dim \Lambda_d(S) \geq \dim \Lambda_d - \#S=\frac{1}{2}(d+1)(d+2)-\frac{1}{2}d(d+3)=1 \]
        よって$F \Lambda_d(S), F \neq 0$となるものが存在する
        \footnote{$\dim \Lambda_d=0$ならば$\Lambda_d$が1点、すなわち0のみからなることを意味する。}。
        
    \end{proof}

    \begin{Prop}
        相異なる5点に対し、どの3点も一直線上に無いとする。
        この時、この5点を通る既約2次曲線がただ一つ存在する。
    \end{Prop}
    \begin{proof}
        すでに示した命題より、そのような二次曲線$C:=\mathcal{Z}(F)$が存在する。
        この$F$が既約であることと、ただ一つであることを示す。

        $F$が既約でないとすると、$F$は1次式の積に分解される。
        すると$C$は2直線の和集合ということになる。
        しかしこれは与えられた5点の内どの3点も一直線上に無いという仮定に反する。
        よって$F$は既約。

        与えられた5点を$S=\{P_1, \dots, P_5\}$とおく。
        $\dim \Lambda_d (S) \geq 2$とすると、
        $F$と一次独立な$F' \in \Lambda_d (S)$が取れる。
        F,F'は既約で一次独立だから、共通因子を持たない。
        したがって弱ベズーの定理が適用できて、
        \[ \# (\mathcal{Z}(F) \cap \mathcal{Z}(F')) \leq 2 \cdot 2=4 \]
        となる。
        しかし$\mathcal{Z}(F) \cap \mathcal{Z}(F') \supset S$なので、矛盾する。
        よってこのような$F'$は存在せず、$\dim \Lambda_d (S) =1$である。
        
    \end{proof}

\section{平面3次曲線}
    \subsection{結合律のための準備}
        平面3次曲線の点にアーベル群の構造が入ることを示す中で、
        特に結合律が成り立つことは自明でない。
        その証明のために準備する。

        \begin{Lemma} \label{lemma401:1}
            相異なる5点$P_1, \dots, P_5 \in \proj^2$が、
            どの4点も一直線上にないならば、
            その5点を通る2次曲線は高々1つ。
        \end{Lemma}
        \begin{proof}
            相異なる2次曲線$C, C'$で$P_1, \dots, P_5$を通るものがあると仮定する。
            このとき$F, F' \in \Lambda_2(\{P_1, \dots, P_5\})$によって
            $C=\mathcal{Z}(F), C'=\mathcal{Z}(F')$と表せる。
            $C \cap C' \supset \{P_1, \dots, P_5\}$が成り立つから
            弱ベズーの定理より、$F,F'$は共通因子を持つ。
            $C \neq C'$から、共通因子は1次式。

            $F=GH, F'=GH'$が成り立つように$G,H,H' \in \Lambda_1$を取る。
            $L:=\mathcal{Z}(G), M:=\mathcal{Z}(H), M':=\mathcal{Z}(H')$とおくと、
            $C\neq C'$より$M \neq M'$。
            \[ C=L \cup M, C'=L \cup M' \]
            となるから、
            \begin{eqnarray*}
                C \cap C'
                &=&(L \cup M) \cap (L \cup M') \\ 
                &=& L \cup (M \cap M') \\
                &\in& \{P_1, \dots, P_5\}
            \end{eqnarray*}
            $M, M'$は直線で$M \neq M'$だから$M \cap M'$は高々1点。
            よって直線$L$上に4点があり、仮定に矛盾する。
        \end{proof}

        さらに、定理の証明には以下が必要である。
        証明はこの二つの補題の証明は演習問題。
        \begin{Lemma}
            $k$を代数的閉体だとする。$F \in k[X,Y,Z] \setminus \{0\}$によって
            $C:=\mathcal{Z}(F)$とおくと、
            \[ |C|=\infty \]
        \end{Lemma}

        \begin{Lemma}
            $k$を無限体とする。$L \subset \proj^2$が直線なら、
            \[ |L|=\infty \]
        \end{Lemma}

        こちらは証明が難しい。
        \begin{Prop}
            $k$を無限体とする。
            $F \in \Lambda_2[X,Y,Z], C:=\mathcal{Z}(F)$とおく。
            このとき、
            \[ |C| \neq 0 \implies |C|=\infty \]
        \end{Prop}

        \begin{Them}
            $k$を無限体とする。
            相異なる8点$P_1, \dots, P_8 \in \proj^2_k$は
            どの4点も一直線上に無く、どの7点も既約2次曲線上に無いとする。
            この時、\[ \dim\Lambda_3(\{P_1, \dots, P_8 \})=2 \]となる。
        \end{Them}
        \begin{proof}
            前章の補題から\[ \dim\Lambda_3(\{P_1, \dots, P_8 \}) \geq 10-8=2 \]が分かる。
            以下では$\leq$も成り立つことを示す。
            そのために与えられた8点の分布の仕方によって場合分けをする。

            \begin{itembox}[l]{場合I}
                8点$P_1, \dots, P_8$が以下を満たす場合。
                \begin{enumerate}
                    \item どの3点も1つの直線上にない
                    \item どの6点も1つの2次曲線上にない
                \end{enumerate}
            \end{itembox}
            もしある6点が可約な1つの2次曲線上にあるとすると、
            それらの点は2本の直線に載っている。
            したがって条件2.と条件1.とを満たす8点は
            「どの6点も1つの既約2次曲線上にない」も満たす。

            $\dim\Lambda_3(\{P_1, \dots, P_8 \}) \geq 3$として矛盾を導く。
            直線$L$を2点$P_1, P_2$を結ぶものとする($L$の定義式も$L$と表す)。
            このとき条件1.より$P_1, \dots, P_8 \not \in L$となる。
            互いに異なる2点$P_{9}, P_{10}$を$L \setminus \{P_1, P_2\}$から取る。
            前章の補題より、
            \[ \dim \Lambda_3(\{P_1, \dots, P_8\} \cup \{ P_9, P_{10} \}) \geq 3-2=1 \]
            となるので、$F \in \Lambda_3(\{P_1, \dots, P_{10} \}) \setminus \{0\}$が取れる。

            曲線$C:=\mathcal{Z}(F)$を考える。
            $C \cap L \subset \{P_1, P_2, P_9, P_{10}\}$となるから、
            \[ |C \cap L| \geq 4 > \deg C \cdot \deg L=3 \]
            したがって弱ベズーの定理より、$F$と$L$は共通因子を持つ。
            $L$は既約なので、ある$G \in \Lambda_2$が存在して$F=L \cdot G$となる。
            さらに$P_3, \dots, P_8 \not \in L$から$P_3, \dots, P_8 \not \in \mathcal{Z}(G)$が分かる。
            これは条件2.に反する。

            条件1., 2.と$\dim \Lambda_3(\{P_1, \dots, P_8 \}) \geq 3$を仮定して条件2.と矛盾したが、
            条件1., 2.を満たす点の分布は存在する。
            よって$\dim \Lambda_3(\{P_1, \dots, P_8 \}) \geq 3$は否定され、
            \[ \mbox{条件1., 2.} \implies \dim\Lambda_3(\{P_1, \dots, P_8 \})=2 \]
            が成立する。

            \begin{itembox}[l]{場合II}
                3点$P_1, \dots, P_3$が直線$L$上にある場合。
            \end{itembox}
            $P_9 \in L \setminus \{P_1, P_2, P_3\}$を取る。
            前章の補題より、
            \[ \dim \Lambda_3(\{P_1, \dots, P_8, P_9 \}) \geq 10-9=1 \]
            となるので、$F \in \Lambda_3(\{P_1, \dots, P_9 \}) \setminus \{0\}$が取れる。
            そして場合Iと同様に$G \in \Lambda_2$が存在して$F=L \cdot G$となる。
            ここで定義の前提より$P_4, \dots, P_8 \not \in L$であった。
            したがって$P_4, \dots, P_8 \in \mathcal{Z}(G)$。つまり
            \[ G \in \Lambda_3(\{P_4, \dots, P_8 \}) \]

            $F$は$\Lambda_3(\{P_1, \dots, P_9 \}) \setminus \{0\}$から任意に取り、
            また$\{P_1, P_2, P_3, P_9\} \subset L$だから
            \[ \Lambda_3(\{P_1, \dots, P_9 \})=L \cdot \Lambda_3(\{P_4, \dots, P_8 \}) \subset \Lambda_3 \]
            補題 \ref{lemma401:1}より$\dim \Lambda_3(\{P_4, \dots, P_8 \})=1$。
            ゆえに
            \[ \Lambda_3(\{P_1, \dots, P_9 \})=1 \]
            よって
            \[ \dim \Lambda_3(\{P_1, \dots, P_8 \}) \leq 2 \]

            \begin{itembox}[l]{場合III}
                6点$P_1, \dots, P_6$が既約2次曲線$D:=\mathcal{Z}(G)$上にある場合。
            \end{itembox}
            $P_9 \in D \setminus \{P_1, \dots, P_6\}$を取る。
            前章の補題より、
            \[ \dim \Lambda_3(\{P_1, \dots, P_8, P_9 \}) \geq 10-9=1 \]
            となるので、$F \in \Lambda_3(\{P_1, \dots, P_9 \}) \setminus \{0\}$が取れる。
            そして場合Iと同様に$L \in \Lambda_1$が存在して$F=L \cdot G$となる。
            ここで定義の前提より$P_7, P_8 \not \in D$であった。
            したがって$P_7, P_8 \in L$。つまり
            \[ L \in \Lambda_1(\{P_7, P_8 \}) \]

            $F$は$\Lambda_3(\{P_1, \dots, P_9 \}) \setminus \{0\}$から任意に取り、
            また$\{P_1, P_2, P_3, P_9\} \subset L$だから
            \[ \Lambda_3(\{P_1, \dots, P_9 \})=G \cdot \Lambda_3(\{P_7, P_8 \}) \subset \Lambda_3 \]
            $\dim \Lambda_1(\{P_7, P_8 \})=1$であるから、
            \[ \Lambda_3(\{P_1, \dots, P_9 \})=1 \]
            よって
            \[ \dim \Lambda_3(\{P_1, \dots, P_8 \}) \leq 2 \]
        \end{proof}

        \begin{Cor} \label{cor401}
            $k$を無限体とする。$C_1, C_2 \subset \proj^2$を
            共通成分を持たない3次曲線とし、
             \[ C_1 \cap C_2 = \{ P_1, \dots, P_9 \} \]
             とおく。この時、任意の3次曲線$C \subset \proj^2$について
             \[ P_1, \dots, P_8 \in C \implies P_9 \in C \]
             が成り立つ。
        \end{Cor}
        \begin{proof}
        $\{ P_1, \dots, P_8 \}$はどの4点も一直線上に無い。
        実際、ある4点は直線L上に会ったとすると、
        $|C_i \cap L| \geq 4 > 3 \cdot 1=3$$(i=1,2)$となり、
        弱ベズーの定理から$L \subset C_i$。よって$L \subset (C_1 \cap C_2)$
        となり、$C_1$と$C_2$が共通因子を持たないことに反する。
        同様にして、どの7点も1つの既約2次曲線上に無い。
        ゆえに$\{ P_1, \dots, P_8\}$は定理の仮定を満たす。
        したがって$\dim \Lambda_3(\{ P_1, \dots, P_8\})=2$。

        $C_i=\mathcal{Z}(F_i)$とすると、$C_1 \neq C_2$より、
        $F_1, F_2$は一次独立である。
        さらに
        \[ F_1, F_2 \in \Lambda_3(\{ P_1, \dots, P_8\}), \dim \Lambda_3(\{ P_1, \dots, P_8\})=2 \]
        であるから、$F_1, F_2$は$\Lambda_3(\{ P_1, \dots, P_8\})$の基底となっている。
        よってある斉次多項式$F \in \Lambda_3(\{ P_1, \dots, P_8\})$によって
        3次曲線$C=\mathcal{Z}(F)$と置くと、\[ F=aF_1+bF_2(a,b \in k) \]の様になる。
        このことから直ちに\[ C \supset C_1 \cap C_2 \]が分かる。
        \end{proof}

    \subsection{平面3次曲線にはアーベル群の構造が入る}
        \begin{Def}
        3次斉次多項式$F \in k[X,Y,Z] \setminus \{0\}$によって$C:=\mathcal{Z}(F)$とおく。
        この$C$に対し、以下のように二項演算$\ast : C \times C \to C$を定める。
        2点$P, Q \in C$を取る。
            \begin{itemize}
            \item $P \neq Q$の時
                \begin{itemize}
                    \item $\#(\overline{PQ} \cap C)=3$の時、$P \ast Q=(\overline{PQ} \cap C) \setminus \{P, Q\}$
                    \item $\#(\overline{PQ} \cap C)=2$の時、
                        \begin{itemize}
                            \item $\overline{PQ}$が$P$に於いて$C$に接する時、$P \ast Q=P$
                            \item $\overline{PQ}$が$Q$に於いて$C$に接する時、$P \ast Q=Q$
                        \end{itemize}
                \end{itemize}
            \item $P = Q$の時、$L$を$P$に於ける$C$の接線として、
                \begin{itemize}
                    \item $\#(L \cap C)=2$の時、$P \ast Q=(L \cap C) \setminus \{P\}$
                    \item $\#(L \cap C)=1$の時、$P \ast Q=P$
                \end{itemize}
            \end{itemize}
        \end{Def}
        場合分けが上の定義で尽くされることと$P \ast Q$が存在することはベズーの定理による。

        \begin{Remark} \label{remark401}
            定義から明らかに$P \ast Q=Q \ast P$。
            更に$R=P \ast Q$の時、$P \ast R=R \ast Q=Q$が成立する。$Q \ast R$でも同様。
        \end{Remark}

        さて、点$O \in C$を1つ取って固定する。
        その上で二項演算$+$を以下のように定める。
        \begin{eqnarray*}
            + : C \times C &\to& C \\
            (P, Q) &\mapsto& (P \ast Q) \ast O
        \end{eqnarray*}
        これがアーベル群を作る。

        \begin{Them}
            $(C,+)$はアーベル群を成す。
        \end{Them}
        \begin{proof}
            以下を順に示す。ただし$P, Q, R \in C$とする。
            \begin{enumerate}
                \item $P+Q=Q+P$(可換律の成立)
                \item $O$が単位元(単位元の存在)
                \item $P$の逆元は$P \ast (O \ast O)$(逆元の存在)
                \item $(P+Q)+R=P+(Q+R)$(結合律の成立)
            \end{enumerate}

            (1.)
            注意 \ref{remark401}より、
            \[ P+Q=(P \ast Q) \ast O=(Q \ast P) \ast O=Q+P \]

            (2.)
            $R:=P \ast O$と置くと、注意 \ref{remark401}より$R \ast O=P$。よって
            \[ P+O=(P \ast O) \ast O=R \ast O=P \]

            (3.)
            $O:=O \ast O$とおく。さらに$Q:=P \ast O'=P \ast (O \ast O)$とすれば、
            \[ P \ast Q = O', O' \ast O=O \]
            ゆえに
            \[ P+Q=(P \ast Q) \ast O=O' \ast O=O \]
            すなわち$-P=Q=P \ast (O \ast O)$。

            (4.)
            まず$P \neq Q$として証明する。
            \begin{eqnarray*}
                (P+Q)+R &=& (((P \ast Q) \ast O) \ast R) \ast O \\
                P+(Q+R) &=& (P \ast ((Q \ast R) \ast O)) \ast O
            \end{eqnarray*}
            なので示したいことは$(P+Q) \ast R=P \ast (Q+R)$と同値。

            直線$L_1, L_2, L_3$と$M_1, M_2, M_3$を以下のように定義する。
            \begin{eqnarray*}
                &{}&L_1=\overline{P,Q},~ L_2=\overline{Q+R,O},~ L_3=\overline{P+Q,R} \\
                &{}&M_1=\overline{Q,R},~ M_2=\overline{O,P+Q},~ M_3=\overline{P,Q+R}
            \end{eqnarray*}
            そしてこれらを用いて3次曲線$L, M$を
            \[ L:=L_1 \cup L_2 \cup L_3, M:=M_1 \cup M_2 \cup M_3 \]
            定義し、これらの交点を考える。
            \begin{eqnarray*}
                \mathcal{J} &:=& \{P,Q,R,O, P \ast Q, Q \ast R, P+Q, Q+R \} \\
                T &:=& L_3 \cap M_3
            \end{eqnarray*}
            と定義すると明らかに$L \cap M = \mathcal{J} \cup \{T\}$である。
            この時、$J \subset C$なので、系 \ref{cor401}より$T \in C$が成り立つ。
            ここで$C \cap L= \{(P+Q) \ast R \} \cup \mathcal{J}$であるから、$T=(P+Q) \ast R$。
            同様に$C \cap M$を考えて、$T=P \ast (Q+R)$。

            次に$P=Q$として証明する。
            これは二項演算子$+$の連続性を用い、$P \to Q$の極限として結合律を証明する。
            まず写像$\phi_1, \phi_2 : C^3 \to C$を以下で定義する。
            \begin{eqnarray}
                \phi_1(P, Q, R)=(P+Q)+R \\
                \phi_2(P, Q, R)=P+(Q+R)
            \end{eqnarray}
            さらに
            \[ E=\{ (P, Q, R) \in C^3 : \phi_1(P, Q, R)=\phi_2(P, Q, R) \} \]
            これはZariski位相で閉。示したいことは$E=C^3$と表現できる。
            一方、
            \[ \sqcup=\{ (P, Q, R) \in C^3 : \#(\mathcal{J} \cup \{T\})=9 \} \]
            (ただし$\mathcal{J}$は上で定めたもの)とおくと、
            $\sqcup$は$C^3$の空ではない開集合となる。
            $C^3$(既約)の中で$\sqcup$が稠密であることは前半の証明から分かる。
            \[ \sqcup \subset E \subset C^3 \]
            なので、閉包$\bar{\sqcup}$が$\sqcup$を含む最小の閉集合であることより、
            \[ C^3=\bar{\sqcup} \subset E \subset C^3 \]
            すなわち$C^3=E$。
        \end{proof}

        \begin{Example}
            $y^2+y=x^3-x \in \affine^2$を考え、
            \[ F(X,Y,Z)=Y^2Z+YZ^2-X^3+XZ^2 \]として$C:=\mathcal{Z}(F)$を調べる。
            \begin{Lemma}
                体$k$に於いて$C \subset \proj^2$が特異点を持つ $\iff p:=\operatorname{char}(k)=37$
            \end{Lemma}
            \begin{proof}
            $P \in C$が特異点である必要十分条件は$F_X(P)=F_Y(P)=F_Z(P)=0$である。
            \begin{eqnarray*}
                F_X&=&-3X^2+Z^2 \\
                F_Y&=&2YZ+Z^2 \\
                F_Z&=&Y^2+2YZ+2XZ
            \end{eqnarray*}
            $P=(a:b:c) \in C$が特異点だとする。
            $p=37$である時に$P=(5:18:1),(32:18:1)$が特異点であることを示す。
            \end{proof}

            $O=(0:1:0) \in C$として群構造を調べる。

            \begin{Lemma}
            $O \ast O=O$が成り立つ。
            特に任意の$Q \in C$に対し$-Q=Q \ast O, Q=(-Q) \ast O$が成立し、
            さらに$P \ast Q=-(P + Q)$。
            \end{Lemma}
            \begin{proof}
                点$O$における$C$の接線は$Z=0$であり、これを満たす$C$上の点は$O$しかない。
                したがって$O \ast O=O$が成り立つ。
                任意の楕円曲線と$Z=0$と$O=(0:1:0)$の交点は$O$だけであり、
                しかも$O$における接線は必ず$Z=0$となるから、
                これは任意の楕円曲線で成り立つ。
                
                また、$R:=Q \ast O$とおくと
                \begin{align*}
                    {}& R=Q \ast O \\
                    \iff& Q \ast R=O \\
                    \iff& (Q \ast R) \ast O=O \ast O \\
                    \iff& Q+R=O \\
                    \iff& R=-Q=Q \ast O
                \end{align*}
                となる。
                このことから更に$(P \ast Q) \ast O=P+Q=-(P \ast Q)$が分かる。
            \end{proof}

            \begin{Lemma}
                $Q=(a:b:1)$に対して$-Q=Q \ast O=(a:-b-1:1)$
            \end{Lemma}

        \end{Example}

\end{document}
