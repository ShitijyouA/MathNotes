\documentclass[a4]{article}
\usepackage{../math_note}
\usepackage{amsmath}
\usepackage[all]{xypic}
\newcommand{\affine}{\mathbb{A}}
\newcommand{\proj}{\mathbb{P}}
\newcommand{\Res}{\operatorname{Res}}

\begin{document}
    \section{平面3次曲線}
    \subsection{結合律のための準備}
        平面3次曲線の点にアーベル群の構造が入ることを示す中で、
        特に結合律が成り立つことは自明でない。
        その証明のために準備する。

        \begin{Lemma} \label{lemma401:1}
            相異なる5点$P_1, \dots, P_5 \in \proj^2$が、
            どの4点も一直線上にないならば、
            その5点を通る2次曲線は高々1つ。
        \end{Lemma}
        \begin{proof}
            相異なる2次曲線$C, C'$で$P_1, \dots, P_5$を通るものがあると仮定する。
            このとき$F, F' \in \Lambda_2(\{P_1, \dots, P_5\})$によって
            $C=\mathcal{Z}(F), C'=\mathcal{Z}(F')$と表せる。
            $C \cap C' \supset \{P_1, \dots, P_5\}$が成り立つから
            弱ベズーの定理より、$F,F'$は共通因子を持つ。
            $C \neq C'$から、共通因子は1次式。

            $F=GH, F'=GH'$が成り立つように$G,H,H' \in \Lambda_1$を取る。
            $L:=\mathcal{Z}(G), M:=\mathcal{Z}(H), M':=\mathcal{Z}(H')$とおくと、
            $C\neq C'$より$M \neq M'$。
            \[ C=L \cup M, C'=L \cup M' \]
            となるから、
            \begin{eqnarray*}
                C \cap C'
                &=&(L \cup M) \cap (L \cup M') \\ 
                &=& L \cup (M \cap M') \\
                &\in& \{P_1, \dots, P_5\}
            \end{eqnarray*}
            $M, M'$は直線で$M \neq M'$だから$M \cap M'$は高々1点。
            よって直線$L$上に4点があり、仮定に矛盾する。
        \end{proof}

        さらに、定理の証明には以下が必要である。
        証明はこの二つの補題の証明は演習問題。
        \begin{Lemma}
            $k$を代数的閉体だとする。$F \in k[X,Y,Z] \setminus \{0\}$によって
            $C:=\mathcal{Z}(F)$とおくと、
            \[ |C|=\infty \]
        \end{Lemma}

        \begin{Lemma}
            $k$を無限体とする。$L \subset \proj^2$が直線なら、
            \[ |L|=\infty \]
        \end{Lemma}

        こちらは証明が難しい。
        \begin{Prop}
            $k$を無限体とする。
            $F \in \Lambda_2[X,Y,Z], C:=\mathcal{Z}(F)$とおく。
            このとき、
            \[ |C| \neq 0 \implies |C|=\infty \]
        \end{Prop}

        \begin{Them}
            $k$を無限体とする。
            相異なる8点$P_1, \dots, P_8 \in \proj^2_k$は
            どの4点も一直線上に無く、どの7点も既約2次曲線上に無いとする。
            この時、\[ \dim\Lambda_3(\{P_1, \dots, P_8 \})=2 \]となる。
        \end{Them}
        \begin{proof}
            前章の補題から\[ \dim\Lambda_3(\{P_1, \dots, P_8 \}) \geq 10-8=2 \]が分かる。
            以下では$\leq$も成り立つことを示す。
            そのために与えられた8点の分布の仕方によって場合分けをする。

            \begin{itembox}[l]{場合I}
                8点$P_1, \dots, P_8$が以下を満たす場合。
                \begin{enumerate}
                    \item どの3点も1つの直線上にない
                    \item どの6点も1つの2次曲線上にない
                \end{enumerate}
            \end{itembox}
            もしある6点が可約な1つの2次曲線上にあるとすると、
            それらの点は2本の直線に載っている。
            したがって条件2.と条件1.とを満たす8点は
            「どの6点も1つの既約2次曲線上にない」も満たす。

            $\dim\Lambda_3(\{P_1, \dots, P_8 \}) \geq 3$として矛盾を導く。
            直線$L$を2点$P_1, P_2$を結ぶものとする($L$の定義式も$L$と表す)。
            このとき条件1.より$P_1, \dots, P_8 \not \in L$となる。
            互いに異なる2点$P_{9}, P_{10}$を$L \setminus \{P_1, P_2\}$から取る。
            前章の補題より、
            \[ \dim \Lambda_3(\{P_1, \dots, P_8\} \cup \{ P_9, P_{10} \}) \geq 3-2=1 \]
            となるので、$F \in \Lambda_3(\{P_1, \dots, P_{10} \}) \setminus \{0\}$が取れる。

            曲線$C:=\mathcal{Z}(F)$を考える。
            $C \cap L \subset \{P_1, P_2, P_9, P_{10}\}$となるから、
            \[ |C \cap L| \geq 4 > \deg C \cdot \deg L=3 \]
            したがって弱ベズーの定理より、$F$と$L$は共通因子を持つ。
            $L$は既約なので、ある$G \in \Lambda_2$が存在して$F=L \cdot G$となる。
            さらに$P_3, \dots, P_8 \not \in L$から$P_3, \dots, P_8 \not \in \mathcal{Z}(G)$が分かる。
            これは条件2.に反する。

            条件1., 2.と$\dim \Lambda_3(\{P_1, \dots, P_8 \}) \geq 3$を仮定して条件2.と矛盾したが、
            条件1., 2.を満たす点の分布は存在する。
            よって$\dim \Lambda_3(\{P_1, \dots, P_8 \}) \geq 3$は否定され、
            \[ \mbox{条件1., 2.} \implies \dim\Lambda_3(\{P_1, \dots, P_8 \})=2 \]
            が成立する。

            \begin{itembox}[l]{場合II}
                3点$P_1, \dots, P_3$が直線$L$上にある場合。
            \end{itembox}
            $P_9 \in L \setminus \{P_1, P_2, P_3\}$を取る。
            前章の補題より、
            \[ \dim \Lambda_3(\{P_1, \dots, P_8, P_9 \}) \geq 10-9=1 \]
            となるので、$F \in \Lambda_3(\{P_1, \dots, P_9 \}) \setminus \{0\}$が取れる。
            そして場合Iと同様に$G \in \Lambda_2$が存在して$F=L \cdot G$となる。
            ここで定義の前提より$P_4, \dots, P_8 \not \in L$であった。
            したがって$P_4, \dots, P_8 \in \mathcal{Z}(G)$。つまり
            \[ G \in \Lambda_3(\{P_4, \dots, P_8 \}) \]

            $F$は$\Lambda_3(\{P_1, \dots, P_9 \}) \setminus \{0\}$から任意に取り、
            また$\{P_1, P_2, P_3, P_9\} \subset L$だから
            \[ \Lambda_3(\{P_1, \dots, P_9 \})=L \cdot \Lambda_3(\{P_4, \dots, P_8 \}) \subset \Lambda_3 \]
            補題 \ref{lemma401:1}より$\dim \Lambda_3(\{P_4, \dots, P_8 \})=1$。
            ゆえに
            \[ \Lambda_3(\{P_1, \dots, P_9 \})=1 \]
            よって
            \[ \dim \Lambda_3(\{P_1, \dots, P_8 \}) \leq 2 \]

            \begin{itembox}[l]{場合III}
                6点$P_1, \dots, P_6$が既約2次曲線$D:=\mathcal{Z}(G)$上にある場合。
            \end{itembox}
            $P_9 \in D \setminus \{P_1, \dots, P_6\}$を取る。
            前章の補題より、
            \[ \dim \Lambda_3(\{P_1, \dots, P_8, P_9 \}) \geq 10-9=1 \]
            となるので、$F \in \Lambda_3(\{P_1, \dots, P_9 \}) \setminus \{0\}$が取れる。
            そして場合Iと同様に$L \in \Lambda_1$が存在して$F=L \cdot G$となる。
            ここで定義の前提より$P_7, P_8 \not \in D$であった。
            したがって$P_7, P_8 \in L$。つまり
            \[ L \in \Lambda_1(\{P_7, P_8 \}) \]

            $F$は$\Lambda_3(\{P_1, \dots, P_9 \}) \setminus \{0\}$から任意に取り、
            また$\{P_1, P_2, P_3, P_9\} \subset L$だから
            \[ \Lambda_3(\{P_1, \dots, P_9 \})=G \cdot \Lambda_3(\{P_7, P_8 \}) \subset \Lambda_3 \]
            $\dim \Lambda_1(\{P_7, P_8 \})=1$であるから、
            \[ \Lambda_3(\{P_1, \dots, P_9 \})=1 \]
            よって
            \[ \dim \Lambda_3(\{P_1, \dots, P_8 \}) \leq 2 \]
        \end{proof}

        \begin{Coll} \label{coll401}
            $k$を無限体とする。$C_1, C_2 \subset \proj^2$を
            共通成分を持たない3次曲線とし、
             \[ C_1 \cap C_2 = \{ P_1, \dots, P_9 \} \]
             とおく。この時、任意の3次曲線$C \subset \proj^2$について
             \[ P_1, \dots, P_8 \in C \implies P_9 \in C \]
             が成り立つ。
        \end{Coll}
        \begin{proof}
        $\{ P_1, \dots, P_8 \}$はどの4点も一直線上に無い。
        実際、ある4点は直線L上に会ったとすると、
        $|C_i \cap L| \geq 4 > 3 \cdot 1=3$$(i=1,2)$となり、
        弱ベズーの定理から$L \subset C_i$。よって$L \subset (C_1 \cap C_2)$
        となり、$C_1$と$C_2$が共通因子を持たないことに反する。
        同様にして、どの7点も1つの既約2次曲線上に無い。
        ゆえに$\{ P_1, \dots, P_8\}$は定理の仮定を満たす。
        したがって$\dim \Lambda_3(\{ P_1, \dots, P_8\})=2$。

        $C_i=\mathcal{Z}(F_i)$とすると、$C_1 \neq C_2$より、
        $F_1, F_2$は一次独立である。
        さらに
        \[ F_1, F_2 \in \Lambda_3(\{ P_1, \dots, P_8\}), \dim \Lambda_3(\{ P_1, \dots, P_8\})=2 \]
        であるから、$F_1, F_2$は$\Lambda_3(\{ P_1, \dots, P_8\})$の基底となっている。
        よってある斉次多項式$F \in \Lambda_3(\{ P_1, \dots, P_8\})$によって
        3次曲線$C=\mathcal{Z}(F)$と置くと、\[ F=aF_1+bF_2(a,b \in k) \]の様になる。
        このことから直ちに\[ C \supset C_1 \cap C_2 \]が分かる。
        \end{proof}

    \subsection{平面3次曲線にはアーベル群の構造が入る}
        \begin{Def}
        3次斉次多項式$F \in k[X,Y,Z] \setminus \{0\}$によって$C:=\mathcal{Z}(F)$とおく。
        この$C$に対し、以下のように二項演算$\ast : C \times C \to C$を定める。
        2点$P, Q \in C$を取る。
            \begin{itemize}
            \item $P \neq Q$の時
                \begin{itemize}
                    \item $\#(\overline{PQ} \cap C)=3$の時、$P \ast Q=(\overline{PQ} \cap C) \setminus \{P, Q\}$
                    \item $\#(\overline{PQ} \cap C)=2$の時、
                        \begin{itemize}
                            \item $\overline{PQ}$が$P$に於いて$C$に接する時、$P \ast Q=P$
                            \item $\overline{PQ}$が$Q$に於いて$C$に接する時、$P \ast Q=Q$
                        \end{itemize}
                \end{itemize}
            \item $P = Q$の時、$L$を$P$に於ける$C$の接線として、
                \begin{itemize}
                    \item $\#(L \cap C)=2$の時、$P \ast Q=(L \cap C) \setminus \{P\}$
                    \item $\#(L \cap C)=1$の時、$P \ast Q=P$
                \end{itemize}
            \end{itemize}
        \end{Def}
        場合分けが上の定義で尽くされることと$P \ast Q$が存在することはベズーの定理による。

        \begin{Remark} \label{remark401}
            定義から明らかに$P \ast Q=Q \ast P$。
            更に$R=P \ast Q$の時、$P \ast R=R \ast Q=Q$が成立する。$Q \ast R$でも同様。
        \end{Remark}

        さて、点$O \in C$を1つ取って固定する。
        その上で二項演算$+$を以下のように定める。
        \begin{eqnarray*}
            + : C \times C &\to& C \\
            (P, Q) &\mapsto& (P \ast Q) \ast O
        \end{eqnarray*}
        これがアーベル群を作る。

        \begin{Them}
            $(C,+)$はアーベル群を成す。
        \end{Them}
        \begin{proof}
            以下を順に示す。ただし$P, Q, R \in C$とする。
            \begin{enumerate}
                \item $P+Q=Q+P$(可換律の成立)
                \item $O$が単位元(単位元の存在)
                \item $P$の逆元は$P \ast (O \ast O)$(逆元の存在)
                \item $(P+Q)+R=P+(Q+R)$(結合律の成立)
            \end{enumerate}

            (1.)
            注意 \ref{remark401}より、
            \[ P+Q=(P \ast Q) \ast O=(Q \ast P) \ast O=Q+P \]

            (2.)
            $R:=P \ast O$と置くと、注意 \ref{remark401}より$R \ast O=P$。よって
            \[ P+O=(P \ast O) \ast O=R \ast O=P \]

            (3.)
            $O:=O \ast O$とおく。さらに$Q:=P \ast O'=P \ast (O \ast O)$とすれば、
            \[ P \ast Q = O', O' \ast O=O \]
            ゆえに
            \[ P+Q=(P \ast Q) \ast O=O' \ast O=O \]
            すなわち$-P=Q=P \ast (O \ast O)$。

            (4.)
            まず$P \neq Q$として証明する。
            \begin{eqnarray*}
                (P+Q)+R &=& (((P \ast Q) \ast O) \ast R) \ast O \\
                P+(Q+R) &=& (P \ast ((Q \ast R) \ast O)) \ast O
            \end{eqnarray*}
            なので示したいことは$(P+Q) \ast R=P \ast (Q+R)$と同値。

            直線$L_1, L_2, L_3$と$M_1, M_2, M_3$を以下のように定義する。
            \begin{eqnarray*}
                &{}&L_1=\overline{P,Q},~ L_2=\overline{Q+R,O},~ L_3=\overline{P+Q,R} \\
                &{}&M_1=\overline{Q,R},~ M_2=\overline{O,P+Q},~ M_3=\overline{P,Q+R}
            \end{eqnarray*}
            そしてこれらを用いて3次曲線$L, M$を
            \[ L:=L_1 \cup L_2 \cup L_3, M:=M_1 \cup M_2 \cup M_3 \]
            定義し、これらの交点を考える。
            \begin{eqnarray*}
                \mathcal{J} &:=& \{P,Q,R,O, P \ast Q, Q \ast R, P+Q, Q+R \} \\
                T &:=& L_3 \cap M_3
            \end{eqnarray*}
            と定義すると明らかに$L \cap M = \mathcal{J} \cup \{T\}$である。
            この時、$J \subset C$なので、系 \ref{coll401}より$T \in C$が成り立つ。
            ここで$C \cap L= \{(P+Q) \ast R \} \cup \mathcal{J}$であるから、$T=(P+Q) \ast R$。
            同様に$C \cap M$を考えて、$T=P \ast (Q+R)$。

            次に$P=Q$として証明する。
            これは二項演算子$+$の連続性を用い、$P \to Q$の極限として結合律を証明する。
            まず写像$\phi_1, \phi_2 : C^3 \to C$を以下で定義する。
            \begin{eqnarray}
                \phi_1(P, Q, R)=(P+Q)+R \\
                \phi_2(P, Q, R)=P+(Q+R)
            \end{eqnarray}
            さらに
            \[ E=\{ (P, Q, R) \in C^3 : \phi_1(P, Q, R)=\phi_2(P, Q, R) \} \]
            これはZariski位相で閉。示したいことは$E=C^3$と表現できる。
            一方、
            \[ \sqcup=\{ (P, Q, R) \in C^3 : \#(\mathcal{J} \cup \{T\})=9 \} \]
            (ただし$\mathcal{J}$は上で定めたもの)とおくと、
            $\sqcup$は$C^3$の空ではない開集合となる。
            $C^3$(既約)の中で$\sqcup$が稠密であることは前半の証明から分かる。
            \[ \sqcup \subset E \subset C^3 \]
            なので、閉包$\bar{\sqcup}$が$\sqcup$を含む最小の閉集合であることより、
            \[ C^3=\bar{\sqcup} \subset E \subset C^3 \]
            すなわち$C^3=E$。
        \end{proof}

        \begin{Example}
            $y^2+y=x^3-x \in \affine^2$を考え、
            \[ F(X,Y,Z)=Y^2Z+YZ^2-X^3+XZ^2 \]として$C:=\mathcal{Z}(F)$を調べる。
            \begin{Lemma}
                体$k$に於いて$C \subset \proj^2$が特異点を持つ $\iff p:=\operatorname{char}(k)=37$
            \end{Lemma}
            \begin{proof}
            $P \in C$が特異点である必要十分条件は$F_X(P)=F_Y(P)=F_Z(P)=0$である。
            \begin{eqnarray*}
                F_X&=&-3X^2+Z^2 \\
                F_Y&=&2YZ+Z^2 \\
                F_Z&=&Y^2+2YZ+2XZ
            \end{eqnarray*}
            $P=(a:b:c) \in C$が特異点だとする。
            $p=37$である時に$P=(5:18:1),(32:18:1)$が特異点であることを示す。
            \end{proof}

            $O=(0:1:0) \in C$として群構造を調べる。

            \begin{Lemma}
            $O \ast O=O$が成り立つ。
            特に任意の$Q \in C$に対し$-Q=Q \ast O, Q=(-Q) \ast O$が成立し、
            さらに$P \ast Q=-(P + Q)$。
            \end{Lemma}
            \begin{proof}
                点$O$における$C$の接線は$Z=0$であり、これを満たす$C$上の点は$O$しかない。
                したがって$O \ast O=O$が成り立つ。
                任意の楕円曲線と$Z=0$と$O=(0:1:0)$の交点は$O$だけであり、
                しかも$O$における接線は必ず$Z=0$となるから、
                これは任意の楕円曲線で成り立つ。
                
                また、$R:=Q \ast O$とおくと
                \begin{align*}
                    {}& R=Q \ast O \\
                    \iff& Q \ast R=O \\
                    \iff& (Q \ast R) \ast O=O \ast O \\
                    \iff& Q+R=O \\
                    \iff& R=-Q=Q \ast O
                \end{align*}
                となる。
                このことから更に$(P \ast Q) \ast O=P+Q=-(P \ast Q)$が分かる。
            \end{proof}

            \begin{Lemma}
                $Q=(a:b:1)$に対して$-Q=Q \ast O=(a:-b-1:1)$
            \end{Lemma}

        \end{Example}

\end{document}
