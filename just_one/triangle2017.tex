\documentclass{jarticle}
\usepackage{../math_note}

\begin{document}
\begin{screen}
少なくとも1辺の長さが2017であって
各辺の長さが自然数であるような三角形を全て求めよ.
ただし辺を交換して一致する三角形同士は同じものとみなす.
\end{screen}

\subsection{特殊な場合}
1辺2017の正三角形,
及び,2辺は2017でもう1辺が1以上の整数であるような二等辺三角形は
自明に条件を満たす.
以降は1辺だけ2017であるようなものを探す.

\subsection{直角三角形の場合}
まず,直角三角形を探す.2017は素数であることに注意する.
したがってそのような3辺は原始ピタゴラス数となる.
原始ピタゴラス数は以下を満たす自然数の組$(m,n)$によって
\[ (a, b, c) = (m^2−n^2, 2mn, m^2+n^2) \mbox{~or~} (2mn, m^2−n^2, m^2+n^2) \]
と表される.
\begin{enumerate}[i)]
    \item $m,n$は互いに素
    \item $m>n$
    \item $m-n$は奇数
\end{enumerate}
したがって$2017=m^2-n^2$または$2017=m^2+n^2$が必要.
$2017=m^2+n^2=(m+in)(m-in)$となるような$(m,n)$は$(44,9)$しかない.
また,$2017=(m-n)(m-n)$となるような$(m,n)$は$(1009, 1008)$しかない.
なので条件を満たす直角三角形は
\begin{enumerate}[i)]
    \item (1855, 792, 2017)
    \item (2017, 2034144, 2034145)
\end{enumerate}
の2個だけ.

\subsection{一般の場合}
他には有るかというと,
三角形の三辺$(a,b,c)$が三角不等式を満たせば良いので,
無数に有る.

\end{document}
