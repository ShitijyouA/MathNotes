\documentclass[a4j]{jarticle}
\usepackage{../math_note}

\begin{document}
\begin{Them}[Eisenstein's criterion]
\[ f(x)=\sum_{0 \leq k \leq n}{a_k x^k} \in \Z[x] \]
について、ある素数$p$が存在して、整数$a_0, a_1, \dots, a_n$が
    \begin{enumerate}
    \item $i \neq n$の場合は$a_i$は$p$で割り切れる
    \item $a_n$は$p$で割り切れない
    \item $a_0$は$p^2$で割り切れない
    \end{enumerate}
を満たすならば、$f(x)$は有理数体上既約である。
\end{Them}
\begin{proof}
多項式$g,h$を$f(x)=g(x) h(x)$を満たすものとおき、
多項式$f, g, h$の各係数を
\[ g(x)=\sum_{0 \leq i \leq n}{g_i x^i}, h(x)=\sum_{0 \leq j \leq n}{h_j x^j} \]
と置く。

この時、単純な計算で
\[ a_{k}=\sum_{i+j=k}{g_i h_j} \]
が成り立つと分かる。
記法を簡単にするため、$P=(p) \subset \Z$とおく。
これが素イデアルであることを何度も使う。

$a_0$を考える。
\[ a_0=g_0 h_0 \]
前提条件1.より$a_0$は$p$の倍数である。
さらに前提条件3.から、$a_0$には素因数として$p$がただ一つ含まれる。
その$p$は$g_0$か$h_0$のどちらか一方に含まれている。
そこで仮定(*)として$g_0 \in P, h_0 \not \in P$とする。
この議論全体で$g$と$h$を単純に入れ替えても議論は破綻しない。

帰納法で$g_0, g_1, \dots, g_{n-1} \in P$を示す。
まず、$k=1$で示す。
\[ a_1=g_0 h_1 + g_1 h_0 \in P \]
$P$はイデアルだから$g_0 h_1 \in P, h_0 \not \in P$。
特に$P$は素イデアルだから$g_1 \in P$。

次に、$0 \leq N+1 < n$を満たす自然数$N$について
$g_0, g_1, \dots, g_{N} \in P$が成り立つとする。
\[ a_{N+1}=g_{N+1} h_0+g_{N} h_1+\sum_{1 \leq j \leq N+1}{g_{N+1-j} h_j } \]
そして前提条件1.より$a_{N+1} \in P$が成り立つ。
帰納法の仮定より、$g_{N} h_1, \sum_{2 \leq j \leq N+1}{g_{N+1-j} h_j } \in P$。
仮定(*)より$h_0 \not \in P$だから$g_{N+1} \in P$。

さて、最後に$a_n$を考える。
\[ a_{n}=g_{n} h_0+\sum_{1 \leq j \leq n}{g_{n-j} h_j } \]
前提条件2.より$a_{n} \not \in P$。すでに示したとおり、$g_0, g_1, \dots, g_{n-1} \in P$が成り立つ。
したがって、仮定(*)と合わせて$g_n \not \in P$が成立する。

$0 \in P$だから、このことから$g_n \neq 0$。
よって$\deg g=n, \deg h=n-n=0$。これで$f$の既約性が示された。
\end{proof}

これと以下の命題を組み合わせると、多くの多項式の既約性が示せる。
\begin{Prop}
多項式$f(x) \in \Z[x]$について、
「任意の定数$a \in \Z$について$f(x+a)$が既約」と「$f(x)$も既約」は同値。
\end{Prop}
\begin{proof}
$f(x)$が既約だとする。
定数$a$に対し、1次以上の多項式$g,h$(これは$a$によって変化する)が
存在して$f(x+a)=g(x)h(x)$が成り立つ($f(x+a)$が既約でない)ならば、
$f(x)=g(x-a)h(x-a)$となり、$g(x-a), h(x-a)$は一次以上の多項式。これは前提に矛盾。
よって$f(x+a)$も既約。

$f(x)$が既約でないとする。
すると1次以上の多項式$g,h$が存在して$f(x)=g(x)h(x)$が成り立つが、
$f(x+a)=g(x+a) h(x+a)$となり、$g(x+a), h(x+a)$は一次以上の多項式。
よって$f(x+a)$も既約でない。
\end{proof}

\end{document}
