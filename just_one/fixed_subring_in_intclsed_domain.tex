\documentclass[a4paper]{jsarticle}
\usepackage{../math_note}

\begin{document}
\begin{Lemma}
    環$A$をintegrally closed domainとし,
    $G$をその自己準同型写像が成すある有限群とする.
    この時,固定環$A^G (\subseteq A)$はintegrally closed domainである.
\end{Lemma}
\begin{proof}
    $\Quot(A^G)$の元$f/g$を任意にとる.
    以下の式を満たす$\{a_i\}_i \subset A^G$が存在したとしよう.
    \[ \left( \frac{f}{g} \right)^n+ a_1 \left( \frac{f}{g} \right)^{n-1}+\dots+a_n=0  \]
    $A^G \subseteq A, \Quot(A^G) \subseteq \Quot(A)$だから,
    $f/g \in \Quot(A), \{a_i\}_i \subset A$とみなすことが出来る.
    $A$がintegrally closed domainであることから$f/g \in A$.
    まとめて$f/g \in \Quot(A^G) \cap A$が得られる.
    
    さて,$f/g=h$と置くと$h \in A$かつ$gh=f$.
    $f/g \in \Quot(A^G)$だから$f, g$はどちらも$A^G$の元である.
    なので,
    \[ \Forall{\sigma \in G} g \cdot h=\sigma(f)=f=g \cdot \sigma(h) \iff \Forall{\sigma \in G} h=\sigma(h) \]
    よって$h=f/g \in A^G$が得られた.
\end{proof}
\end{document}
