\documentclass[a4j]{jarticle}
\usepackage{../math_note}

\begin{document}
任意の収束列$\{ x_i \}_{i=1}^{\infty}$をとり、これがCauchy列であることを示す。
$\{ x_i \}_{i=1}^{\infty}$は$x$へ収束するものとする。
このとき、定義から次が成り立つ。

\[
    \Forall{\epsilon>0} \Exists{N \in \N} (\Forall{n>N}, | x_n -x | < \epsilon)
\]

    $s.t.$以降において、$m, n > N$であるような任意の$m, n$を取る。
    三角不等式から次が成り立つ。
\[
    |x_m -x_n| \leq |x_m-x|+|x_n-x| < 2 \epsilon
\]
    したがって次が成り立つ。
\[
    \Forall{\epsilon/2>0} \Exists{N \in \N} (\Forall{m,n>N}, | x_m -x_n | < \epsilon)
\]
    これはCauchy列の定義である。
\end{document}

