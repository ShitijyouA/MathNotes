\documentclass[a4paper]{jsarticle}
\usepackage[]{../math_note}
\usepackage[all]{xy}

\newcommand{\CaCl}{\operatorname{CaCl}}
\newcommand{\Pic}{\operatorname{Pic}}
\newcommand{\lsd}{\mathfrak{d}}
\newcommand{\gProj}{\mathbf{Proj}\,}
\newcommand{\inclmap}{\hookrightarrow}

\begin{document}
\begin{Def}
    教科書及びこのメモではschemeについての仮定($*$)を次で定める.
    この条件($*$)を満たすschemeではWeil divisorが定義できる.
    \begin{itemize}
        \item integral,
        \item separated,
        \item noetherian, and
        \item regular in codimention one.
    \end{itemize}

    最後の条件の定義は次のよう.
    $X$ :: schemeに対し,
    $X^{(1)}$を$\codim(\cl_X(\{z\},X))=1$なる点$z \in X$の全体とする.
    (Weil divisor全体はfree abelian group $\Z^{(X^{(1)})}$と表現できる.)
    $X$が``regular in codimention one"であるとは,
    任意の$z \in X^{(1)}$に対し
    $\shO_{X,z}$ :: integrally closed domainであるということ.

    さらに強く,
    codimension one とは限らない点$x \in X$において
    $\shO_{X,x}$ :: UFDであるとき,
    $X$はlocally factorialであるという(Prop6.11).
    locally fatorialになる十分条件のひとつは$X$ :: regularである(Remark 6.11.1A).
\end{Def}

\begin{Def}
    $X$ :: scheme上の$\shF$ :: sheafについて定義する.
    
    $\shF$がinvertibleであるとは,
    $X$の開被覆$\{ U_{\lambda} \}_{\lambda \in \Lambda}$が存在し,
    $\shF|_{U_{\lambda}} \iso \shO_X$が成り立つということである.

    $\shF$がgenerated by global sectionsであるとは,
    $G \subseteq \Gamma(X,\shF)$が存在し,
    任意の点$x \in X$について$\shF_x$ :: $\shO_{X,x}$-moduleが
    $G_x=\{ g_x \mid g \in G \}$で生成されるということである.
    $G$のことをglobal generator of $\shF$と呼ぶ.

    $\shF$がlocally generated by $\{\sect{U}{g_U}\}_U (\subseteq \shF)$とは,
    $\shF|_V$ :: sheaf on $V$が
    generated by global sections $\{\sect{U \cap V}{g_U|_{U \cap V}}\}_U$,
    になるということ.
    $\{\sect{U}{g_U}\}$のことをlocal generator of $\shF$と呼ぶ.
\end{Def}

    Weil/Carier divisorの定義は難しくないので改めて書かない.


\section{Cartier Divisor $\leftrightarrow$ Weil Divisor.}

    \paragraph{Weil divisor $\leftarrow$ Cartier divisor.}
    Prop6.11の前半から.
    \textbf{この対応は($*$)を満たすならば成立する.}
    $\{\sect{U_i}{f_i}\}_i$をCartier Divisorとすると,
    $X$ :: integralから$\shK$は$K$のconstant sheaf.
    なので$z \in X^{(1)}$について
    $v_z(f_i|_{\cl_{U_i}(x)})$が定まる.
    したがって
    $\sum_{z \in X^{(1)}} \sum_{i} v_z(f_i|_{\cl_{U_i}(x)}) \cl_X(z)$
    とすればよい.
    (この対応をEGA IVでは$\operatorname{cyc}$と呼んでいる.)

    \paragraph{Weil divisor $\rightarrow$ Cartier divisor.}
    Prop6.11の後半から.
    \textbf{この対応は$X$が($*$)を満たし,かつlocally factorialであるならば成立する.}
    $D$ :: Weil divisorとし,
    詳しく$D=\sum_{z \in X^{(1)}} n_z \cl_X(z)$とする.
    点$x \in X$をとり,$T_x=\Spec \shO_{X,x}$とする.
    \[ D_x=D|_{T_x}=\sum_{z \in X^{(1)} \cap T_x} n_z \cl_{T_x}(z) \]
    (記法はEGA ch.IV, p.274より借用した.)は$T_x$上のWeil divisorであり,
    $X$ :: nonsingular projective curveより(Remark 6.11.1A) $\shO_x$ :: UFD.
    なのでProp6.2より,
    $D_x=(f_x)$なる$f_x \in K$が存在する(つまりprincipal divisor).
    よって$D \cap U_x=f_x|_{U_x}$となる$x$の近傍$U_x$が存在する.
    これをまとめて,$\{\sect{U_x}{f_x}\}_x$がCartier divisorになる.
    (ここはU.G\"ortz, T.Wedhorn ``Algebraic Geometry I" pp.306-309も参考になる.
    基本的にEGA ch.IVの翻訳ではあるが,Hartshorneの記法に近い.)

\section{Cartier Divisor $\leftrightarrow$ Invertible Subsheaf of $\shK$.}
    $D$ :: Cartier divisorは
    $X$の被覆$\{U_i\}_i$と$f_i \in \Gamma(U_i, \shK_X)$で表現される.
    ただし,$f_i/f_j \in \Gamma(U_i \cap U_j, \shO_X^*)$となっている.

    対応はProp6.13で証明されている.
\paragraph{Cartier Divisor $\rightarrow$ Invertible Subsheaf of $\shK$.}
    $\{\sect{U_i}{f_i}\}_i$で表現される$D$ :: Cartier divisorに対して,
    $\shL(D)$ :: subsheaf of $\shK$を,
    $\shO_{U_i} \ni 1 \mapsto f_i^{-1}$で定まる準同型の像として定める.
    ($\shL$のlocal generatorは$\{\sect{U_i}{f_i}\}_i$である.)

\paragraph{Cartier Divisor $\leftarrow$ Invertible Subsheaf of $\shK$.}
    $X$の開被覆$\{U_i\}$が存在し,
    $\shL$のlocal generatorが$\{\sect{U_i}{g_i}\}_i$だとする.
    この時,$\{\sect{U_i}{g_i^{-1}}\}_i$がCartier divisorを定める.

\section{Morphism to $\proj^n$ $\leftrightarrow$ Invertible Sheaf \& Global Generators.}
    対応が存在することはThm7.1による.
    また,$\shO(1)=\shO_{\proj_A^n}(1)=(A[x_0,\dots,x_n](1))\sidetilde$とする.
    これのgeneratorは$x_0,\dots,x_n$である.

    \paragraph{Morphism to $\proj^n$ $\rightarrow$ Invertible Sheaf \& Global Generators.}
    $\phi: X \to \proj_A^n$を$A$-morphismとする.
    $\phi$に対し,$\phi^*(\shO(1))$はinvertible sheafであり,
    $\phi^*(x_0), \dots, \phi^*(x_n)$がそのglobal generatorである.

    \paragraph{Morphism to $\proj^n$ $\leftarrow$ Invertible Sheaf \& Global Generators.}
    上の対応の逆も成り立つ.
    $\shL$ :: invertible sheaf,
    $G \subseteq \Gamma(X, \shL)$ :: global generator ($\#G=n$)
    に対し,
    $\shL \iso \phi^*(\shO(1)), G=\{\phi^*(x_i)\}_{i=1}^n$
    であるような$\phi: X \to \proj_A^n$が存在する.

\section{Ample Sheaf.}
    \begin{Def}
        $\shL$ :: invertible sheaf on noetherian scheme $X$が
        ample sheafであるとは,
        任意の$\shF$ :: coherent sheaf on $X$に対して
        十分大きいすべての$n$で$\shF \otimes \shL^{\otimes n}$が
        generated by global sectionになるということである.
    \end{Def}

\section{Linear System.}
    $k$ :: algebraically closed field,
    $X$ :: nonsingular projective variety over $k$,
    $K$ :: functor field of $X$とする.
    この時は,次の対応関係が成り立つ.

    \paragraph{Cartier Divisor $\leftrightarrow$ Weil Divisor.}
    Prop6.11に証明がある.
    詳細は上に書いたとおり.

    \paragraph{$\CaCl X \iso \Pic X$.}
    Prop6.15にある.

    また,この状況では次の事実が成り立つ(5.19):
    任意の$\shL$ :: invertible sheaf on $X$について,
    $\Gamma(X,\shL)$ :: finite dimentional $k$-vector space.

    \paragraph{Linear System $\rightarrow$ Invertible Sheaf.}
    $\lsd$ :: linear systemの元はすべて
    ある$D_0$ :: Cartier divisor on $X$とlinear equivalentである.
    なので上の段落にある$\CaCl X \iso \Pic X$より,
    $\lsd$の任意の元は$\shL \in \Pic X$ :: invertible sheafに対応する.

    \paragraph{Nonzero Global Section of $\shL$ $\rightarrow$ Effective Cartier Divisor.}
    $\shL$ :: invertible sheaf on $X$,
    $s \in \Gamma(X,\shL)$ :: nonzero global section of $\shL$とする.
    $\shL|_U \iso \shO_U$となる任意の開集合$U$について,
    この同型写像を$\phi^{(U)}: \shL|_U \isomap \shO_U$とする.
    すると明らかに$\phi_U(s|_U) \in \Gamma(U,\shO_U)$.
    なのでこうして得られる$\{U, \phi^{(U)}_U(s|_U)\}$は
    effective Cartier divisorを定めている.
    $\phi^{(U)}$のとり方には$\Gamma(U,\shO_U^*)$の分だけ自由度があるが,
    これは結局同じCartier divisorを定めている.

    \paragraph{Complete Linear System $|D_0| $$\approx$ Nonzero Global Sections of $\shL(D_0)$.}
    Complete Linear System $|D_0|$は,
    ある与えられたdivisor $D_0$と線形同値なすべてのeffective divisorの集合である.
    Prop7.7より,$|D_0|$の任意の元は$s \in \Gamma(X,\shL(D_0))-\{0\}$を用いて$(s)_0$と書ける.
    $k^*=\Gamma(X,\shO_X^*)$の分だけ自由度があるから,
    結局$|D_0|$は$(\Gamma(X,\shL(D_0))-\{0\})/k^*$と同型である.
    ($X$の様子を表すdivisorから線形空間という比較的わかりやすいものが取り出せた.)
    Thm5.19より,$\Gamma(X,\shL(D_0)$はfinite dimensional $k$-vector spaceであることに注意.

    \begin{Def}
        $D$ :: Weil divisor on $X$が$D=\sum_{x \in X^{(1)}} n_x \cl_X(x)$と書けたとする.
        この時,\[ \Supp D=\bigcup_{x \in X^{(1)}, n_x \neq 0} \cl_X(x) \subseteq X \]とする.
        これは閉集合の有限和なので閉集合.
        $\lsd$ :: linear system on $X$について,
        $\bigcap_{D \in \lsd} \Supp D$の点をbase pointと呼び,
        この集合が空ならば$\lsd$はbase point freeと呼ばれる.
    \end{Def}
 
\section{Global (or Relative) Proj $\gProj$ (or $\underline{\Proj})$.}
    \paragraph{Assumption ($\dagger$).}
    $X$ :: noetherian scheme, 
    $\shS$ :: graded $\shO_X$-algebra
    とする.
    また,$d \in \Z, d \geq 0$について,
    $\shS_d$ :: homogeneous part of $\shS$を$U \mapsto \shS(U)_d$で定める.
    次のように仮定する.
    \begin{itemize}
        \item $\shS$ :: quasi-coherent.
        \item $\shS=\bigoplus_{d \geq 0} \shS_d$.
        \item $\shS_0=\shO_X$.
        \item $\shS_1$ :: coherent $\shO_X$-module.
        \item $\shS$ :: locally generated by $\shS_1$ as $\shO_X$-algebra.
    \end{itemize}
    下の二つから,
    任意の$d \geq 0$についても$\shS_d$ :: coherentであることが分かる.

    \newpage %% !!!!!!!!!!!!!

    \paragraph{Construction of $\gProj$.}
    $X, \shS$を($\dagger$)を満たすscheme, graded $\shO_X$-algebraとする.
    任意のaffine open subset $U=\Spec A \subseteq X$をとる.
    $\shS_0(U)=\shO_X(U)=A$に注意する.
    $\shS$ :: quasi-coherentなので,
    $\shS|_U=\tilde{M}$となる$M$ :: $A$-algebraが存在する.
    $D_+(g) ~(g \in M_+ \otimes A_f=(M \otimes A_f)_+)$でのsectionを見ると,
    次式の最後の等号が分かる.
    \[ \Proj \shS(U)=\Proj M=\Proj M \otimes_A A=\Proj M \times_{\Spec A} \Spec A. \]
    なので,次のfiber productの可換図式が得られる.
    \[
        \xymatrix
        {
            (\Proj \shS(U)) \times_U U \ar[r]^-{\pi_U} \ar[d]_-{\pr}& U \ar@{=}[d] \\
            \Proj \shS(U) \ar[r]& U.
        }
    \]
    このように,
    $\Proj \shS(U)=(\Proj \shS(U)) \times_U U$からのprojection mapとして$\pi_U$を定める.
    $f \in A, U_f=\Spec A_f \subseteq U$とすると,
    Thm3.3の証明より次が得られる.
    \[ \pi_U^{-1}(U_f)=(\Proj \shS(U)) \times_U U_f=\Proj M \otimes A_f=\Proj M_f=\Proj \shS(U_f). \]
    $\shS(U_f)=M_f=\shS(U)_f$に注意.
    これら(と私が書いたEx3.1の解答にある道具たち)を使うと,
    $U,V$ :: open affine subset in $X$について
    $\pi_U^{-1}(U \cap V) \iso \pi_V^{-1}(U \cap V)$となることがわかる.
    したがって$\Proj \shS(U)$と$\pi_U$の張り合わせが可能であり,
    こうして$\gProj \shS, \pi: \gProj \shS \to X$が構成できる.
    $\shO_{\Proj \shS(U)}(1)$達も貼りあわせて,$\shO(1)$を得る.

\end{document}
