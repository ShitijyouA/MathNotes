\documentclass[a4paper]{jsarticle}
\usepackage{../math_note, exercise}
\usepackage[all]{xy}
\renewcommand{\thesection}{Ex8.\arabic{section}}

\begin{document}
\section{Strengthen Some Results in the Text.} %% Ex8.1 

\section{$0 \to \shO_X \to \shE \to \shE' \to 0$.} %% Ex8.2 
    $X$ :: variety of dimension $n$ over $k$,
    $\shE$ :: locally free sheaf of rank $>n$,
    $V^{\#} \subset \Gamma(X,\shE)$ ::
    $k$-vector space of global sections which generate $\shE$
    とする.
    $X$ :: varietyより$X$ :: connectedなので$\shE$のrankは$X$全体で一定である.
    $\rank \shE=r (>n)$としておこう.

    \begin{Claim}
        ある$s \in V$について次が成立する.
        \[ \Forall{x \in X} s_x \not \in \I{m}_x \shE_x. \]
    \end{Claim}

    \paragraph{Convensions and Notations.}
    $X$のclosed point全体を$X^+$と書く.
    Ex3.14より,これはdense in $X$.
    また,$d=\dim_k V^{\#}, V=\proj_k^{d-1}$とし,
    $V^+=(V^{\#}-\{0\})/k^*$を$V$のclosed pointsと同一視する.
    この同一視の仕方はProp7.7やdual projective spaceと同じである.
    $\dim_k V^{\#}-1=\dim V$に注意.
    $V^{\#}$のsubspaceも同様に$V$のsubspaceとみなす.

    \paragraph{Definition of $B, B^+$.}
    $B \subseteq X \times_k V$を次のように置く.
    \[ B=\bigcap_{s \in V^{\#}} \pr_1^{-1}(\{ x \in X \mid s_x \in \I{m}_x\shE_x \}). \]
    $B$は$X \times V$のclosed subschemeである.
    ($\{\}$部分がclosedであることはEx2.16を参照.)
    $B$にはreduced structureを与えておく.
    $\pr_1|_{B}: B \to X$を$p_1$と略す.
    $B$のclosed points :: $B^+$は次のような集合である.
    \[ B^+=\{ (x, s) \in X^+ \oplus V^+ \mid s_x \in \I{m}_x\shE_x \}. \]

    \paragraph{Plot.}
    主張は,$\pr_2(B) \not \supseteq V^+$と言い換えられる.
    (詳細は後ほど.)
    これには$B$の次元が$V$の次元より小さいことを言えば良い.
    $B$の次元はEx3.22の結果を用いればそのfiber :: $B_x$から計算できる.
    全ての$x \in X$について$\dim B_x$を計算することは難しい.
    しかし少し妥協して,$x \in X^+$についての$\dim B_x$を計算することは出来る.
    この場合でもEx3.22cの結果を用いて$\dim B_x$が計算できる.

    \paragraph{Definition of $\phi_x$.}
    $x \in X$について次の写像を考える.
    \begin{defmap}
        \phi_x:& V^{\#}& \to& \shE_x \otimes_k k(x) \\
        {}& s& \mapsto& s_x \otimes 1
    \end{defmap}
    これが$k$-linear mapであることは明らか.
    $k(x):=\shO_x/\I{m}_x$より$\shE_x \otimes_k k(x) \iso \shE_x/\I{m}_x \shE_x$.
    このことと$\phi_x$の定義の仕方から,
    $\ker \phi_x=\{ s \in V^{\#} \mid s_x \in \I{m}_x\shE_x \}.$
    
    \paragraph{$\phi_x$ for $x \in X^+$.}
    この段落では$x \in X^+$とする.
    すると$k(x)=k$
    \footnote
    {
        $X$ :: varietyより,
        $k$ :: algebraically closed fieldかつ$X$ :: finite type / $k$.
        $A=k[x_1,\dots,x_n], \I{a} \subseteq A$とし,
        $\I{m}/\I{a} \in \Spec A/\I{a} \subseteq X$が$x$に対応する極大イデアルだとする.
        ここで$\I{m}$は$A$の極大イデアル.
        $S=A-\I{m}$とすると
        \[
            k(x)
            =\frac{S^{-1}(A/\I{a})}{S^{-1}(\I{m}/\I{a})}
            \iso S^{-1}\left( \frac{A/\I{a}}{\I{m}/\I{a}} \right)
            \iso S^{-1}(A/\I{m}).
        \]
        $A/\I{m} \iso k$は体だから,これは$k(x) \iso k$.
    }
    なので$\shE_x \otimes_k k(x) \iso \shE_x$.
    また$\shE_x \otimes_k k(x) \iso \shE_x/\I{m}_x\shE_x$.
    さらに$V^{\#}$ :: global generators of $\shE$であるから,
    $\phi_x$はsurjective.
    なので$x \in X^+$について$\dim \ker \phi_x$が分かる.
    \[
        \dim_k \ker \phi_x
        =\dim_k V^{\#} \otimes_k k(x)-\dim_k \shE_x
        =\dim_k V^{\#}-r.
    \]

    \paragraph{Dimension of fiber :: $\dim B_x$.}
    $p_1$についての$x \in X^+$のfiber :: $B_x$のbase spaceは,
    Ex3.10 より,$\basesp B_x \homeo p_1^{-1}(x)$.
    したがって次が分かる.
    \[ \basesp B_x \cap \basesp B^+ \homeo p_1^{-1}(x) \cap \basesp B^+=\{x\} \times \ker \phi_x. \]
    ここで$\times$は集合としての直積を表す.
    よって$B_x$の次元が分かる
    \footnote
    {
        closed subscheme of $B$ :: $C$について$\dim C=\dim C \cap B^+$を示す.
        $C \cap B^+ \subset C$より$\dim C \geq \dim C \cap B^+$は明らか.
        $d=\dim C$とし,
        $C$のirreducible closed subsetが成す真の極大上昇鎖をとる:
        $Z_0 \subsetneq \dots \subsetneq Z_d.$
        closed immersion $\implies$ finite typeに注意すると,
        $Z_i$ :: finite type/$k$.
        なのでEx3.14より$Z_i \cap B^+$ :: dense in $Z_i$.
        したがって$Z_i \cap B^+=Z_j \cap B^+ \implies Z_i=Z_j$となり,
        $Z_0 \cap B^+ \subsetneq \dots \subsetneq Z_d \cap B^+$
        は$B^+$のirreducible closed subsetが成す真の上昇鎖.
        以上から$\dim C \leq \dim C \cap B^+$も成り立つ.
    }.
    \[ \dim B_x=\dim_k \ker \phi_x-1=\dim_k V^{\#}-r-1=\dim V-r. \]

    \paragraph{$p_1$ :: closed map.}
    $V \to \Spec k$はprojectiveであり,
    $V, \Spec k$共にnoetherianであるからこの射はproper.
    よってuniversally closedである.
    \[
    \xymatrix
    {
        X \times_k V \ar[r]\ar[d]_-{\pr_1}& V
            \ar[d]^-{\substack{\text{universally} \\ \text{closed}}}\\
        X \ar[r]& \Spec k
    }
    \]
    $B$ :: closedなので$B$のclosed subsetは$X$でもclosed.
    したがって
    $p_1=\pr_1|_B$ :: closed map.

    \paragraph{$p_1(B)=X$ or $B=\emptyset$.}
    $p_1(B) \supseteq X^+$とする.
    すると$p_1(B)$ :: closedより$p_1(B) \supseteq \cl_X(X^+)=X$.
    次に$p_1(B) \not \supseteq X^+$とする.
    すると上で述べたこと
    (全ての$x \in X^+$について$\dim p_1^{-1}(x)$が等しいこと)
    から,結局$p_1(B) \cap X^+=\emptyset$が分かる.
    $p_1(B)$が空でないと仮定しよう.
    すると$p_1$ :: closed mapより,
    $x \in p_1(B)$なら$\cl_X(\{x\}) \subseteq p_1(B)$.
    $\cl_X(\{x\})$はclosed pointを含むので矛盾が生じる.
    よって$p_1(B) \not \supseteq X^+$ならば$p_1(B)=\emptyset$.
    これは$B=\emptyset$を意味し,
    さらにこれは$0$を除く全ての$V^{\#}$の元がclaimの条件を満たすことを意味する.
    \underline{以下,$B \neq \emptyset$と仮定する.}

    \paragraph{$B$ :: irreducible.}
    以上から$B$ :: irreducibleが分かる.
    実際,$B$が二つの閉集合$C_1, C_2 (\neq \emptyset, X)$の和であったとすると,
    $p_1(B)=X$より,
    \[ X=p_1(C_1) \cup p_1(C_2). \]
    一方,$X$ :: irreducible.
    よって矛盾が生じ,$B$ :: irreducibleが示される.

    \paragraph{Dimension of $B$.}
    $B$ :: integral \& finite type/$k$ ($\implies$ variety/$k$)なので,
    Ex3.22cから次が成り立つ:
    $x \in U$ならば$\dim B_x=\dim B-\dim X$,
    となる$U$ :: open dense subset in $X$が存在する.
    $U$ :: non-empty open subsetと$X^+$ :: denseから,$U \cap X^+ \neq \emptyset$.
    $x \in X^+$であるときの及び開集合$\dim B_x$が既に分かっているから,
    $\dim B$も分かる.
    \[ \dim B=\dim B_x+\dim X=\dim V-r+n. \]
    $r>n$なので,$\dim B<\dim V$.

    \paragraph{$\pr_2(B) \supseteq V^+ \implies \dim B \geq \dim V$.}
    $\pr_2(B) \supseteq V^+$としよう.
    $B^+$の場合と同様に$\dim V^+=\dim V$.
    ch I, Ex1.10より,$\dim U=\dim V$を満たす
    affine open subset of $V$ :: $U$がとれる.
    適当に$\pr_1(B)$からもaffine open subset ::  $U'$をとると,
    $X, V$共にfinite type /$k$だから,
    ch I, Ex3.15 (Products of Affine Varieties)が使える.
    よって$\dim U \times U'=\dim U+\dim U' \geq \dim U=\dim V$.
    $U \times_k U' \subset B$だから$\dim B \geq \dim V$
    
    \paragraph{Complete proof of the claim.}
    今はこれの対偶が成立する.
    すなわち,$s \in V^+-\pr_2(B)$が存在する.
    この$s$と任意の$x \in X$について$s_x \not\in \I{m}_x\shE_x$が成り立つ.

    \paragraph{An exact sequence.}
    $\Phi$を以下で定める.
    \begin{defmap}
        \Phi:& \shO_X& \to& \shE  \\
        {}& \sect{U}{\sigma}& \mapsto& \sect{U}{(s|_U) \cdot \sigma}
    \end{defmap}
    これの$x \in X$におけるstalkを見ると,
    $\Phi_x: \sigma_x \mapsto s_x \cdot \sigma_x$と成っている.
    $\shE_x \iso \shO_x^{\oplus r}$かつ$\shO_x$ :: domainより,
    $\Ann(\shE_x)=0$.
    そして$s_x \not\in \I{m}_x\shE_x$から,$s_x \neq 0$.
    なので$\Phi_x$は,したがって$\Phi$はinjective.
    よって$\shE'=\coker \Phi$とおくと以下はexact sequence.
    \[ 0 \to \shO_X \to \shE \to \shE' \to 0. \]
    
    \paragraph{$\shE'$ :: locally free.}
    $\shE'$がlocally freeであることを示そう.
    Ex5.7bから,任意の点におけるstalkがfreeであることを示せば十分.
    以下,$\shE_x=\shO_x^{\oplus r}$($\iso$でなく$=$)とする.
    点$x \in X$について
    \[ s_x=(s_x^{(i)})_i \in \shO_x^{\oplus r}=\shE_x \]とする.
    $s_x \not\in \I{m}_x\shE_x=\I{m}_x^{\oplus r}$から,
    ある$i$について$s_x^{(i)} \not \in \I{m}_x$.
    すなわち$s_x^{(i)}$ :: unit.
    ここでは$i=0$とし,
    \[
        u
        =(s_x^{(0)})^{-1}s_x
        =\left( 1, s_x^{(2)}(s_x^{(0)})^{-1}, \dots, s_x^{(r)}(s_x^{(0)})^{-1} \right) \in s_x\shO_x
    \]
    と置く.
    すると$\shE'_x \iso \shE_x/\im \Phi_x=\shO_x^{\oplus r}/s_x\shO_x$は
    次の写像で$\shO_x^{\oplus r-1}$と同型.
    \begin{defmap}
        {}& \shO_x^{\oplus r}/s_x\shO_x& \to& 0 \oplus \shO_x^{\oplus r-1} \\
        {}& (t^{(j)})_j \bmod s_x\shO_x& \mapsto& (t^{(j)})_j-t^{(0)}u
    \end{defmap}
    well-definedであることは明らか.
    逆写像は次のもの.
    \begin{defmap}
        {}& \shO_x^{\oplus r-1}& \to& \shO_x^{\oplus r}/s_x\shO_x \\
        {}& t & \mapsto& (0 \oplus t) \bmod s_x\shO_x
    \end{defmap}

\section{Product Schemes.} %% Ex8.3 

\section{Complete Intersections in $\proj^n$.} %% Ex8.4 

\section{Blowing Up a Nonsingular Subvariety.} %% Ex8.5 

\section{The Infinitesimal Lifting Property.} %% Ex8.6 

\section{Classifying Infinitesimal Extension: One Case.} %% Ex8.7 

\section{Plurigenera and Hodge Numbers are Birational Invariants.} %% Ex8.8 

\end{document}
