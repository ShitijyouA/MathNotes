\documentclass[a4paper]{jsarticle}
\usepackage{../math_note, exercise}
\usepackage[all]{xy}
\renewcommand{\thesection}{Ex8.\arabic{section}}

\newcommand{\der}[1][\!]{\mathrm{d}_{#1}\,}
\newcommand{\Der}{\Omega}
\newcommand{\shCano}{\omega}
\newcommand{\pbundle}{\mathbb{P}}

\begin{document}
\section{Strengthen Some Results in the Text.} %% Ex8.1 
    \subsection{}
%        liftingの存在を示せば,
%        あとはMatsumuraにあるSecond Fundamental Exact Sequenceの証明と同じ.
%        $B'=M/\I{m}^2, \I{m}'=\I{m/\I{m}^2}$とおくと,
%        $(B', \I{m}')$はlocal ring.
%        また$(\I{m}')^2=0$だから,$B'$の$\I{m}'$によるcomplitionは$B'$に等しい.
%        すなわち,$B'$はlocal complete ring.
%        また$k(B') \iso k(B)$であり,
%        仮定からこれはseparably generated extension of $k$である.
%        Thm8.25より,
%        $k \subseteq K \subseteq B'$と
%        $K \iso k(B')(\iso k(B))$ by natural projectionを満たす体$K$が存在する.
%        よって以下が同型写像に成る.
%        \[ k(B) \isomap K \inclmap B' \to k(B') \isomap k(B)  \]
%        $B'$より前半が求めるliftingである.

    \subsection{}

    \subsection{}

    \subsection{}

\section{$0 \to \shO_X \to \shE \to \shE' \to 0$.} %% Ex8.2 
    $X$ :: variety of dimension $n$ over $k$,
    $\shE$ :: locally free sheaf of rank $>n$,
    $V^{\#} \subset \Gamma(X,\shE)$ ::
    $k$-vector space of global sections which generate $\shE$
    とする.
    $X$ :: varietyより$X$ :: connectedなので$\shE$のrankは$X$全体で一定である.
    $\rank \shE=r (>n)$としておこう.

    \begin{Claim}
        ある$s \in V$について次が成立する.
        \[ \Forall{x \in X} s_x \not \in \I{m}_x \shE_x. \]
    \end{Claim}

    \paragraph{Convensions and Notations.}
    $X$のclosed point全体を$X^+$と書く.
    Ex3.14より,これはdense in $X$.
    また,$d=\dim_k V^{\#}, V=\proj_k^{d-1}$とし,
    $V^+=(V^{\#}-\{0\})/k^*$を$V$のclosed pointsと同一視する.
    この同一視の仕方はProp7.7やdual projective spaceと同じである.
    $\dim_k V^{\#}-1=\dim V$に注意.
    $V^{\#}$のsubspaceも同様に$V$のsubspaceとみなす.

    \paragraph{Definition of $B, B^+$.}
    $B \subseteq X \times_k V$を次のように置く.
    \[ B=\bigcap_{s \in V^{\#}} \pr_1^{-1}(\{ x \in X \mid s_x \in \I{m}_x\shE_x \}). \]
    $B$は$X \times V$のclosed subschemeである.
    ($\{\}$部分がclosedであることはEx2.16を参照.)
    $B$にはreduced structureを与えておく.
    $\pr_1|_{B}: B \to X$を$p_1$と略す.
    $B$のclosed points :: $B^+$は次のような集合である.
    \[ B^+=\{ (x, s) \in X^+ \oplus V^+ \mid s_x \in \I{m}_x\shE_x \}. \]

    \paragraph{Plot.}
    主張は,$\pr_2(B) \not \supseteq V^+$と言い換えられる.
    (詳細は後ほど.)
    これには$B$の次元が$V$の次元より小さいことを言えば良い.
    $B$の次元はEx3.22の結果を用いればそのfiber :: $B_x$から計算できる.
    全ての$x \in X$について$\dim B_x$を計算することは難しい.
    しかし少し妥協して,$x \in X^+$についての$\dim B_x$を計算することは出来る.
    この場合でもEx3.22cの結果を用いて$\dim B_x$が計算できる.

    \paragraph{Definition of $\phi_x$.}
    $x \in X$について次の写像を考える.
    \begin{defmap}
        \phi_x:& V^{\#}& \to& \shE_x \otimes_k k(x) \\
        {}& s& \mapsto& s_x \otimes 1
    \end{defmap}
    これが$k$-linear mapであることは明らか.
    $k(x):=\shO_x/\I{m}_x$より$\shE_x \otimes_k k(x) \iso \shE_x/\I{m}_x \shE_x$.
    このことと$\phi_x$の定義の仕方から,
    $\ker \phi_x=\{ s \in V^{\#} \mid s_x \in \I{m}_x\shE_x \}.$
    
    \paragraph{$\phi_x$ for $x \in X^+$.}
    この段落では$x \in X^+$とする.
    すると$k(x)=k$
    \footnote
    {
        $X$ :: varietyより,
        $k$ :: algebraically closed fieldかつ$X$ :: finite type / $k$.
        $A=k[x_1,\dots,x_n], \I{a} \subseteq A$とし,
        $\I{m}/\I{a} \in \Spec A/\I{a} \subseteq X$が$x$に対応する極大イデアルだとする.
        ここで$\I{m}$は$A$の極大イデアル.
        $S=A-\I{m}$とすると
        \[
            k(x)
            =\frac{S^{-1}(A/\I{a})}{S^{-1}(\I{m}/\I{a})}
            \iso S^{-1}\left( \frac{A/\I{a}}{\I{m}/\I{a}} \right)
            \iso S^{-1}(A/\I{m}).
        \]
        $A/\I{m} \iso k$は体だから,これは$k(x) \iso k$.
    }
    なので$\shE_x \otimes_k k(x) \iso \shE_x$.
    また$\shE_x \otimes_k k(x) \iso \shE_x/\I{m}_x\shE_x$.
    さらに$V^{\#}$ :: global generators of $\shE$であるから,
    $\phi_x$はsurjective.
    なので$x \in X^+$について$\dim \ker \phi_x$が分かる.
    \[
        \dim_k \ker \phi_x
        =\dim_k V^{\#} \otimes_k k(x)-\dim_k \shE_x
        =\dim_k V^{\#}-r.
    \]

    \paragraph{Dimension of fiber :: $\dim B_x$.}
    $p_1$についての$x \in X^+$のfiber :: $B_x$のbase spaceは,
    Ex3.10 より,$\basesp B_x \homeo p_1^{-1}(x)$.
    したがって次が分かる.
    \[ \basesp B_x \cap \basesp B^+ \homeo p_1^{-1}(x) \cap \basesp B^+=\{x\} \times \ker \phi_x. \]
    ここで$\times$は集合としての直積を表す.
    よって$B_x$の次元が分かる
    \footnote
    {
        closed subscheme of $B$ :: $C$について$\dim C=\dim C \cap B^+$を示す.
        $C \cap B^+ \subset C$より$\dim C \geq \dim C \cap B^+$は明らか.
        $d=\dim C$とし,
        $C$のirreducible closed subsetが成す真の極大上昇鎖をとる:
        $Z_0 \subsetneq \dots \subsetneq Z_d.$
        closed immersion $\implies$ finite typeに注意すると,
        $Z_i$ :: finite type/$k$.
        なのでEx3.14より$Z_i \cap B^+$ :: dense in $Z_i$.
        したがって$Z_i \cap B^+=Z_j \cap B^+ \implies Z_i=Z_j$となり,
        $Z_0 \cap B^+ \subsetneq \dots \subsetneq Z_d \cap B^+$
        は$B^+$のirreducible closed subsetが成す真の上昇鎖.
        以上から$\dim C \leq \dim C \cap B^+$も成り立つ.
    }.
    \[ \dim B_x=\dim_k \ker \phi_x-1=\dim_k V^{\#}-r-1=\dim V-r. \]

    \paragraph{$p_1$ :: closed map.}
    $V \to \Spec k$はprojectiveであり,
    $V, \Spec k$共にnoetherianであるからこの射はproper.
    よってuniversally closedである.
    \[
    \xymatrix
    {
        X \times_k V \ar[r]\ar[d]_-{\pr_1}& V
            \ar[d]^-{\substack{\text{universally} \\ \text{closed}}}\\
        X \ar[r]& \Spec k
    }
    \]
    $B$ :: closedなので$B$のclosed subsetは$X$でもclosed.
    したがって
    $p_1=\pr_1|_B$ :: closed map.

    \paragraph{$p_1(B)=X$ or $B=\emptyset$.}
    $p_1(B) \supseteq X^+$とする.
    すると$p_1(B)$ :: closedより$p_1(B) \supseteq \cl_X(X^+)=X$.
    次に$p_1(B) \not \supseteq X^+$とする.
    すると上で述べたこと
    (全ての$x \in X^+$について$\dim p_1^{-1}(x)$が等しいこと)
    から,結局$p_1(B) \cap X^+=\emptyset$が分かる.
    $p_1(B)$が空でないと仮定しよう.
    すると$p_1$ :: closed mapより,
    $x \in p_1(B)$なら$\cl_X(\{x\}) \subseteq p_1(B)$.
    $\cl_X(\{x\})$はclosed pointを含むので矛盾が生じる.
    よって$p_1(B) \not \supseteq X^+$ならば$p_1(B)=\emptyset$.
    これは$B=\emptyset$を意味し,
    さらにこれは$0$を除く全ての$V^{\#}$の元がclaimの条件を満たすことを意味する.
    \underline{以下,$B \neq \emptyset$と仮定する.}

    \paragraph{$p_1^{-1}(x)$ :: irreducible.}
    任意のclosed point :: $x \in X^+$について$p_1^{-1}=(\ker \phi_x-\{0\})/k^*$.
    これはprojective linear spaceだからirreducible.

    \paragraph{$B$ :: irreducible.}
    以上から$B$ :: irreducibleが分かる.
    $B$が二つの閉集合$C_1, C_2$の和であったとすると,
    $x \in X^+$について$p_1^{-1}(x)$は次のように書ける.
    \[ p_1^{-1}(x)=(C_1 \cap \pr_1^{-1}(x)) \cup (C_2 \cap \pr_1^{-1}(x)). \]
    これはirreducibleだから,
    $C_1 \cap \pr_1^{-1}(x)$か$C_2 \cap \pr_1^{-1}(x)$に一致する.
    $x_1, x_2 \in X^+$について次のようになっていたと仮定しよう.
    \[
        p_1^{-1}(x_1)=C_1 \cap \pr_1^{-1}(x),~~~
        p_1^{-1}(x_2)=C_2 \cap \pr_1^{-1}(x).
    \]
    すると,$x_1 \in, x_2 \not \in p_1(C_1)$となる.
    $p_1(C_2)$も同様.
    すなわち$p_1(C_1), p_1(C_2)$は$p_1(B)(=X)$空でないの真の閉集合である.
    しかし$X=p_1(B)=p_1(C_1) \cup p_1(C_2)$であり$X$ :: irreducibleであるから,
    これはありえない.
    よって任意の$x \in X^+$について
    $p_1^{-1}(x)=C_1 \cap \pr_1^{-1}(x)$(あるいは$=C_2 \cap$...)となる.
    両辺で$\bigcup_{x \in X^+}$として
    \[ p_1^{-1}(X^+)=C_1 \cap p_1^{-1}(X^+). \]
    $p_1^{-1}(X^+)=(X^+ \times V) \cap B \supset B^+$であり,
    $B^+$ :: dense in $B$.
    $B^+ \cap C_1$ :: dense in $C_1$もEx3.14から得られるので,
    両辺の$B$での閉包を取って$B=C_1$.
    したがって$B$ :: irreducible.

    \paragraph{Dimension of $B$.}
    $B$ :: integral \& finite type/$k$ ($\implies$ variety/$k$)なので,
    Ex3.22cから次が成り立つ:
    $x \in U$ならば$\dim B_x=\dim B-\dim X$,
    となる$U$ :: open dense subset in $X$が存在する.
    $U$ :: non-empty open subsetと$X^+$ :: denseから,$U \cap X^+ \neq \emptyset$.
    $x \in X^+$であるときの及び開集合$\dim B_x$が既に分かっているから,
    $\dim B$も分かる.
    \[ \dim B=\dim B_x+\dim X=\dim V-r+n. \]
    $r>n$なので,$\dim B<\dim V$.

    \paragraph{$\pr_2(B) \supseteq V^+ \implies \dim B \geq \dim V$.}
    $\pr_2(B) \supseteq V^+$としよう.
    $B^+$の場合と同様に$\dim V^+=\dim V$.
    ch I, Ex1.10より,$\dim U=\dim V$を満たす
    affine open subset of $V$ :: $U$がとれる.
    適当に$\pr_1(B)$からもaffine open subset ::  $U'$をとると,
    $X, V$共にfinite type /$k$だから,
    ch I, Ex3.15 (Products of Affine Varieties)が使える.
    よって$\dim U \times U'=\dim U+\dim U' \geq \dim U=\dim V$.
    $U \times_k U' \subset B$だから$\dim B \geq \dim V$
    
    \paragraph{Complete proof of the claim.}
    今はこれの対偶が成立する.
    すなわち,$s \in V^+-\pr_2(B)$が存在する.
    この$s$と任意の$x \in X$について$s_x \not\in \I{m}_x\shE_x$が成り立つ.

    \paragraph{An exact sequence.}
    $\Phi$を以下で定める.
    \begin{defmap}
        \Phi:& \shO_X& \to& \shE  \\
        {}& \sect{U}{\sigma}& \mapsto& \sect{U}{(s|_U) \cdot \sigma}
    \end{defmap}
    これの$x \in X$におけるstalkを見ると,
    $\Phi_x: \sigma_x \mapsto s_x \cdot \sigma_x$と成っている.
    $\shE_x \iso \shO_x^{\oplus r}$かつ$\shO_x$ :: domainより,
    $\Ann(\shE_x)=0$.
    そして$s_x \not\in \I{m}_x\shE_x$から,$s_x \neq 0$.
    なので$\Phi_x$は,したがって$\Phi$はinjective.
    よって$\shE'=\coker \Phi$とおくと以下はexact sequence.
    \[ 0 \to \shO_X \to \shE \to \shE' \to 0. \]
    
    \paragraph{$\shE'$ :: locally free.}
    $\shE'$がlocally freeであることを示そう.
    Ex5.7bから,任意の点におけるstalkがfreeであることを示せば十分.
    以下,$\shE_x=\shO_x^{\oplus r}$($\iso$でなく$=$)とする.
    点$x \in X$について
    \[ s_x=(s_x^{(i)})_i \in \shO_x^{\oplus r}=\shE_x \]とする.
    $s_x \not\in \I{m}_x\shE_x=\I{m}_x^{\oplus r}$から,
    ある$i$について$s_x^{(i)} \not \in \I{m}_x$.
    すなわち$s_x^{(i)}$ :: unit.
    ここでは$i=0$とし,
    \[
        u
        =(s_x^{(0)})^{-1}s_x
        =\left( 1, s_x^{(2)}(s_x^{(0)})^{-1}, \dots, s_x^{(r)}(s_x^{(0)})^{-1} \right) \in s_x\shO_x
    \]
    と置く.
    すると$\shE'_x \iso \shE_x/\im \Phi_x=\shO_x^{\oplus r}/s_x\shO_x$は
    次の写像で$\shO_x^{\oplus r-1}$と同型.
    \begin{defmap}
        {}& \shO_x^{\oplus r}/s_x\shO_x& \to& 0 \oplus \shO_x^{\oplus r-1} \\
        {}& (t^{(j)})_j \bmod s_x\shO_x& \mapsto& (t^{(j)})_j-t^{(0)}u
    \end{defmap}
    well-definedであることは明らか.
    逆写像は次のもの.
    \begin{defmap}
        {}& \shO_x^{\oplus r-1}& \to& \shO_x^{\oplus r}/s_x\shO_x \\
        {}& t & \mapsto& (0 \oplus t) \bmod s_x\shO_x
    \end{defmap}

    \subsubsection{$B$の別構成.}
    $d+1=\dim_k V^{\#}$とし,$\shV=(V^{\#})\sidetilde$とする.
    $V^{\#} \iso k^{\oplus d+1}$から
    $\shV$は$\rank \shV=d+1$のlocally free sheafとなる.
    そして全射$\shV \otimes_k \shO_X \to \shE$が
    $\sect{U}{s} \otimes \sect{U}{a} \mapsto \sect{U}{sa}$の様に構成できる
    \footnote
    {
        $\shO_X$が$k$-moduleであることは次のように分かる.
        今,$f: X \to \Spec k$が存在するので
        $\shO_{\Spec k} \to f^*\shO_X$が存在する.
        これのadjoint :: $f^{-1}\shO_{\Spec k} \to \shO_X$を考えれば,
        開集合$U \subseteq X$について$\shO_X(U)$が$k$-moduleであることが分かる.
        また,ここで書いた$\shV \otimes_k \shO_X \to \shE$の定義は
        presheaf :: $U \mapsto \shV(U) \otimes_k \shO_X(U)$からの
        morphismなのでsheafificationが必要である.
    }.
    これの$\ker$を$\shB$とおく.
    \[ \xymatrix{ 0 \ar[r]& \shB \ar[r]& \shV \otimes \shO_X \ar[r]& \shE \ar[r]& 0 } \]
    構成から$\shB$ :: locally freeと$\rank \shB=d+1-r$が分かる(?).
    双対をとる.(すなわち$\shHom(-, \shO_X)$で写す.)
    \[ \xymatrix{ 0 \ar[r]& \check{\shE} \ar[r]& \check{\shV} \otimes \shO_X \ar[r]& \check{\shB} \ar[r]& 0 } \]
    全射$\check{\shV} \otimes \shO_X \to \check{\shB}$から,
    injective $X$-morphism :: $\pbundle(\check{\shB}) \to \proj_k^{d} \times X$が誘導される(?).
    ここでの$\pbundle(\check{\shB})$が$B$である(?).
    構成の仕方から,$\dim B=\rank \check{\shB}-1$.

    \subsubsection{$\shE'$ :: locally freeの別証明.}
    任意の点$x \in X$におけるstalkを考える.
    \[ \xymatrix{ 0 \ar[r]& \shO_x \ar[r]^-{\times s_x}& \shE_x \ar[r]& \shE'_x \ar[r]& 0 } \]
    これを$\otimes_{\shO_x} k(x)$で写し,
    $k(x)$-moduleのexact sequenceにする.
    \[ \xymatrix{ \shO_x \otimes k(x) \ar[r]^-{\times (s_x \otimes 1)}& \shE_x \otimes k(x) \ar[r]& \shE'_x \otimes k(x) \ar[r]& 0 } \]
    同型で書き換える.
    \[ \xymatrix{ k(x) \ar[r]^-{\times (s_x)\sidebar}& \shE_x/\I{m}_x\shE_x \ar[r]& \shE'_x \otimes k(x) \ar[r]& 0 } \]
    ただし$(s_x)\sidebar=s_x \bmod \I{m}_x\shE_x$.
    これは$s_x \not \in \I{m}_x\shE_x$から,$0$でない.
    したがって左の写像は$1 \in k(x)$を非ゼロ元に写す.
    このexact sequenceは$k(x)$-moduleのものだったから,
    左の写像はinjective.
    よって次が分かる.
    \[ \dim_{k(x)} \shE'_x \otimes k(x)=\dim_{k(x)} \shE_x \otimes k(x)-\dim_{k(x)} k(x)=r-1. \]
    すなわち$\dim_{k(x)} \shE'_x \otimes k(x)$は$x \in X$について定数関数.
    Ex5.8より,$\shE'$ :: locally freeと分かる.

\section{Product Schemes.} %% Ex8.3 
    \subsection{$\Der_{X \times_S Y/S} \iso \pr_X^* \Der_{X/S} \oplus \pr_Y^* \Der_{Y/S}$.}
    $S$ :: scheme, $X,Y$ :: scheme /$S$とする.
    Thm8.10より,$\Der_{X \times Y/Y} \iso \pr_X^*\Der_{X/S}$が分かる.
    これとThm8.11を合わせて次の完全列が得られる.
    \[
    \xymatrix
    {
        \pr_Y^* \Der_{Y/S} \ar[r]&
        \Der_{X \times Y/S} \ar[r]&
        \pr_X^*\Der_{X/S} \ar[r]&
        0.
    }
    \eqno{(*)}
    \]
    $X$と$Y$を交換したものと合わせて次の図式を得る.
    これは$\shO_{X \times Y}$-moduleの図式である.
    \[
    \xymatrix
    {
        {} &
        \pr_X^* \Der_{X/S} \ar[r] \ar[d]^-{\bar{\gamma}}&
        \Der_{X \times Y/S} \ar[r] \ar@{=}[d]^-{\id}&
        \pr_Y^*\Der_{Y/S} \ar[r]&
        0 \\
        0 &
        \pr_X^* \Der_{X/S} \ar[l]&
        \Der_{X \times Y/S} \ar[l]&
        \pr_Y^*\Der_{Y/S} \ar[l]&
        {}
    }
    \]
    この図式において$\bar{\gamma}$は
    $\Der_{X \times Y/S}$を経由する射の合成である.
    $\gamma=\id[\pr_Y^* \Der_{Y/S}]$が示せれば,
    $\alpha$ :: inj \& splitが得られる.
    これは$X \times Y$のopen affine coverをとって
    localに調べれば良い.
    $\Spec R \subseteq S, \Spec A \subseteq X, \Spec B \subseteq Y$を任意にとり,
    $C=A \otimes_R B$とする.
    図式全体を$\Gamma(\Spec C,-)$で写す.
    $\Der$の構成から,これは次のように成る.
    これは$C$-moduleの図式である.
    \[
    \xymatrix
    {
        {} &
        \Der_{A/S} \otimes_A C \ar[r]^-{\alpha} \ar[d]^-{\gamma}&
        \Der_{C/S} \ar[r] \ar@{=}[d]^-{\id}&
        \Der_{B/S} \otimes_B C \ar[r]&
        0 \\
        0 &
        \Der_{A/S} \otimes_A C \ar[l]&
        \Der_{C/S} \ar[ld]^-{\beta}&
        \Der_{B/S} \otimes_B C \ar[l]&
        {} \\
        {} & \Der_{C/A} \ar[u]^-{\iso}
    }
    \]
    それぞれの写像は次のように定義される
    (Matsumura, p.193 \& Eisenbud, Prop16.4).
    \begin{defmap}
        \alpha:& [\der[A/S] a] \otimes c& \mapsto& [\der[C/S] (a \otimes 1_B)] \cdot c \\
        \beta:& \der[C/S] c& \mapsto& \der[C/A] c \\
        \iso:& \der[C/A] (a \otimes b)& \mapsto& [\der[A/S] a] \otimes (1_A \otimes b)  \\
    \end{defmap}
    よって$\gamma$は次のように成る.
    \[
        [\der[A/S] a] \otimes c
        \mapsto
        [\der[C/S] (a \otimes 1_B)] \cdot c
        \mapsto
        [\der[C/A] (a \otimes 1_B)] \cdot c
        \mapsto
        ([\der[A/S] a] \otimes 1_C) \cdot c
        =
        [\der[A/S] a] \otimes c.
    \]
    以上より$\gamma=\id$が示された.

    \subsection{$\shCano_{X \times Y} \iso \pr_X^*\shCano_{X} \otimes \pr_Y^*\shCano_{Y}$.}
    $X, Y$ :: nonsingular varieties over a field $k$とする.
    $d_X=\dim X, d_Y=\dim Y$とする.
    この時Thm8.15より,
    $\Der_{X/k}, \Der_{Y/k}$は
    それぞれ$\rank=d_X, d_Y$のlocally free sheafである.
    また(a)の完全列$(*)$より,
    $\rank \Der_{X \times_k Y/k}=d_X+d_Y$
    \footnote
    {
        各点でのstalkをとって$\rank$がadditiveであることを使えば分かる.
    }.

    Ex5.16dを(a)の完全列$(*)$に用いれば,
    \[
        \shCano_{X \times Y}
        =\bigwedge^{d_X+d_Y}\Der_{X \times Y/k}
        \iso \left( \bigwedge^{d_X}\pr_X^* \Der_{X/k} \right)
            \otimes \left( \bigwedge^{d_X}\pr_Y^* \Der_{Y/k} \right).
    \]
    Ex5.16eより$\pr_X^*, \pr_Y^*$はそれぞれ$\bigwedge$と交換できる.
    よって$\shCano_{X \times Y} \iso \pr_X^*\shCano_{X} \otimes \pr_Y^*\shCano_{Y}$.

    \subsection{An Example that Gives $p_g \neq p_a$.}
    $Y \subset \proj^2_k$をnon-singluar cubic curveとする.
    さらに$Y \times_k Y$をSegre embeddingで
    $\proj^8$に埋め込んだものを$X$とする.

    Example8.20.3より,$\shCano_Y \iso \shO_Y(0)=\shO_Y$.
    したがって(b)より$p_g(X)$が計算できる.
    \[
        p_g(X)
        =\dim_k \Gamma(X, \pr_1^*\shO_Y \otimes \pr_2^* \shO_Y)
        =\dim_k \Gamma(X, \shO_X)
        =\dim_k k
        =1.
    \]
    ここでEx5.11:
    $\shO_X(1) \iso \pr_1^*\shO_Y(1) \otimes \pr_2^* \shO_Y(1)$
    (両辺に逆元をテンソルすれば利用した同型が得られる)と
    Ex4.5dを順に用いた.

    I, Ex7.2bより$p_a(Y)=\frac{1}{2}(3-1)(3-2)=1$.
    同じくI, Ex7.2eより$p_a(X)$が計算できる.
    \[ p_a(X)=(p_a(Y))^2-2p_a(Y)=-1. \]

    \subsubsection{Direct Calc of $\shCano_Y$.}
    体$k$の標数は$0$としておく.
    $U_z=\zerosp(z)^c \iso \affine^2$とし,
    $Y \cap U_z \subset \affine^2$の定義多項式を$y^2-f(x) \in k[x,y]$とする.
    ただし$\deg f=3$.
    \[ B=k[x,y], \quad I=(y^2-f(x))B, \quad C=B/I \]
    と置いて加群$\Der_{C/k}$を求めよう.
    $\shCano_Y=\bigwedge^1 \Der_{Y/k} \iso \Der_{Y/k}$だから,
    $\shCano_Y$も以下の計算から分かる.
    second exact sequence (Thm8.4)を用いると$\Der_{C/k}$が計算できる.
    \begin{align*}
                \Der_{C/k}
        \iso&   \frac{\Der_{B/k} \otimes_B C}{\delta(I/I^2)} \\
        \iso&   \frac{(B \der x \oplus B \der y) \otimes C}
                {\langle \der \alpha \otimes 1 \mid \alpha \in I \rangle} \\
        \iso&   \frac{C \der \bar{x} \oplus C \der \bar{y}}
                {\langle 2 \bar{y} \cdot \der \bar{y} -(\partial_x f) \der \bar{x} \rangle}.
    \end{align*}
    ここで$\bar{x}=x \bmod I, \bar{y}=y \bmod I$とした.
    以下,これらの$\bar{\square}$は省略する.
    
    点$\I{p} \in C$における$\Der_{C/k}$の局所化を計算する.
    $Y$ :: non-singularから,
    $2y, \partial_x f$の両方が同時に$0$になることはない.
    なので任意の点において
    $\der x=(*) \der y$あるいは$\der y=(*) \der x$の形になる.
    より詳細には次の通り.
    \[
        (\Der_{C/k})_{\I{p}} \iso
        \begin{cases}{}
            C_{\I{p}} \der x & \text{if} \quad \I{p} \in D(y) \\
            C_{\I{p}} \der y & \text{if} \quad \I{p} \in D(\partial_x f). \\
        \end{cases}
    \]
    よって$\rank \Der_{C/k}=1$.
    (TODO: $\Der_{Y/k} \iso \shO_Y$は示せるか?)

\section{Complete Intersections in $\proj^n$.} %% Ex8.4 
    \begin{Def}
        closed subscheme of $\proj^n_k$ :: $Y$は,
        $Y$の定義イデアル$I \subseteq S=k[x_0,\dots,x_n]$が
        $r=\codim (Y, \proj^n)$個の元で生成される時,
        (strict, global) complete intersetionと呼ばれる.
    \end{Def}

    \subsection{$Y$ :: complete intersetion $\iff$ $Y=H_1\cap\dots H_r$.}
    $Y$ :: closed subscheme of $\proj^n$が
    $\codim=r$のcomplete intersectionであることと,
    hypersurfaces :: $H_1, H_r \subseteq \proj^n$が存在して
    schemeとして$Y=H_1\cap\dots H_r$,
    すなわち$\shI_Y=\shI_{H_1}+\dots+\shI_{H_r}$となることは同値.

    \paragraph{$\impliedby$.}
    Ex5.?から,$I$はradical idealと考えて良い.
    $\shI_{H_1}+\dots+\shI_{H_r}$は高さ$r$かつ生成元は$r$個であるか?

    \paragraph{$\implies$.}
    Matsumura, Thm17.6と$S$ :: Cohen-Macaulay ringより,
    $I$はunmixed,すなわち,
    $I$に属す極小素イデアルの高さは全て等しい.
    $I \subseteq \I{p}$を$I$に属す極小素イデアルとする.
    $\height \I{p}=1$か?

%    $I$の生成元を$g_1,\dots,g_r$としたとき,
%    $H_i=\Proj S/{g_i}$とすればよい.
%    ここから$I=\sum_i (g_i)$.
%    $\shI=\tilde{I}, \shI_{H_i}=(g_i)\sidetilde$なので
%    $\shI_Y=\hI_{H_1}+\dots+\shI_{H_r}$は明らか.

    \subsection{$Y$ :: normal complete intersection of dim $ \geq 1$ in $\proj^n$ is projectively normal.}
    $Y=\Proj S/I$の代わりに
    affine cone :: $Y^{*}=\Spec S/I \subset \affine^{n+1}_k$を考える.
    これがnormalであるとき,
    I, Ex3.17dから$S/I$ :: integrally closed.
    すなわち$Y$ :: projectively normal.

    $Y \subseteq Y^*$とみなすと,$\codim(Y, Y^*)=1$(TODO).
    $Y$ :: normalより,$Y^*$ :: regular in codim $1$.
    よってThm8.23bより$S/I$ :: normal.

    \subsection{With same hypotheses as (b), 
        $\Gamma(\proj^n, \shO_{\proj^n}(l)) \to \Gamma(Y, \shO_Y(l))$ is surj.}
    (b)とEx5.14から,
    \[ \Gamma(\proj^n, \shO_{\proj^n}(l)) \to \Gamma(Y, \shO_Y(l)) \]
    は任意の$l \geq 0$についてsurjective.
    Ex4.5より$\Gamma(\proj^n, \shO_{\proj^n})=k$だから,
    これは$k$-algebra homomorphismである.

    さらに$l=0$とすると,$Y$ :: connectedが示せる.
    Ex5.14aの証明から$Y$はintegral schemeのdisjoint unionである.
    なので$Y$のconnected componentの個数を$m$とすると,
    再びEx4.5を用いて,
    \[ \Gamma(Y, \shO_Y)=k^{\oplus m} \]
    となる.
    $\Gamma(U \sqcup V, \shO_Y)=\Gamma(U, \shO_Y) \oplus \Gamma(V, \shO_Y)$に注意.
    今,$k \to k^{\oplus m}$が全射なのだから$m=1$.
    すなわち$Y$はconnected.

    \subsection{For given integers $r<n$ and $d_1, \dots, d_r \geq 1$,
        there exists complete intersection of codim$=r$ in $\proj^n$.}
    次の条件を満たす schemes :: $H_1,\dots,H_r \subseteq \proj^n$の存在を示す.
    \begin{itemize}
        \item $H_1,\dots,H_r$ :: hypersurface in $\proj^n$,
        \item and nonsingular.
        \item $\deg H_i=d_i$.
        \item $Y=H_1 \cap \dots \cap H_r$ :: irreducible,
        \item nonsingular,
        \item and $\codim(Y, \proj^n)=r$.
    \end{itemize}

    $r$についての帰納法で示そう.
    $r=1$についてはI, Ex5.5で存在を示した
    \footnote
    {
        私の解答では,
        $x_0 x_1^{d-1}+x_1 x_2^{d-1}+x_2 x_0^{d-1}$
        で定まるhypersurfaceを取っている.
        これはKlein quarticの自然な拡張である.
    }.
    $H_1,\dots,H_r$と$Y$は条件を満たしていると仮定して,
    条件を満たす$r+1$個目のhypersurface :: $H \ (\deg H=d)$の存在を示す.
    (当然$r+1<n$とする.)

    まず$Y$と$X=\proj^n$を
    $d_{r+1}$-uple embedding :: $\rho$で$\proj^N$へ埋め込む.
    ここで$N$は$n, d_{r+1}$で定まる整数である.
    すると$\proj^N$のhyperplaneは
    $\proj^n$の$d_{r+1}$次のhypersurfaceに対応する.

    Example 7.8.3より,
    $\proj^N$のhyperplane全体の集合はcomplete linear systemを成す.
    これを$L$としよう.
    Thm8.18より,
    $\rho(X)$との交わりがnonsingularであるようなhyperplane全体は
    $L$のopen dense subsetである.
    $\rho(Y)$についても同様に$L$のopen dense subsetが存在する.
    どちらものopen dense subsetであるから,
    その交わりが存在する.
    これを$\bar{H} \in L$とし,
    $\rho^{-1}(\bar{H})=H$としよう.
    すると先程述べたように$H$はdegree $d$のhypersurfaceであり,
    $\rho$ :: isomorphismから$X \cap H(=H), Y \cap H$はnonsingular.
    残るは$Y \cap H$ :: irreducibleと$\codim(Y \cap H, X)=r+1$であるが,
    前者は$r+1<n (\iff \dim Y=n-r>1)$とThm8.18から,
    後者は(a)から分かる.

    \subsection{$Y$ as in (d), $\shCano \iso \shO_Y(\sum d_i-n-1).$}
    これも$r$についての帰納法で示す.
    $r=1$についてはExample8.20.3の通り.
    $Y'=Y \cap H_{r+1}$とおくと$\codim(Y', Y)=1$だから,
    Prop8.20より次が成立する.
    \[
        \shCano_{Y'}
        \iso \shCano_Y \otimes \shO_Y(d_{r+1}) \otimes \shO_Y
        \iso \shO_Y \left(\sum_{i=1}^r d_i-n-1 \right) \otimes \shO_Y(d_{r+1}) \otimes \shO_Y
        \iso \shO_Y \left(\sum_{i=1}^{r+1} d_i-n-1 \right).
    \]

    \subsection{Calc geometric genus of nonsingular hyper surface of degree $d$ in $\proj^n$.}
    (e)より$\shCano_Y \iso \shO_Y(d-n-1)$.
    (c)より次のsurjective $k$-algebra homomorphismが存在する.
    \[ \Gamma(\proj^n, \shO_{\proj^n}(d-n-1)) \to \Gamma(Y, \shCano_Y). \]
    両辺のring構造を忘却すれば,
    これはsurjective linear map.
    次元等式から両辺の$\dim_k$は等しい.
    よって$p_g(Y)$が得られる.
    \[
        p_g(Y)
        =\dim_k \Gamma(\proj^n, \shO_{\proj^n}(d-n-1))
        =\binom{(d-n-1)+n}{n}
        =\binom{d-1}{n}.
    \]

    以上とI, Ex7.2の結果を比較すると,$p_g(Y)=p_a(Y)$.
    特に$Y \subseteq \proj^2$
    ($Y$ :: nonsingular plane curve of degree $d$)ならば
    $p_g(Y)=\frac{1}{2}(d-1)(d-2)$.

    \subsection{Calc geometric genus of 
        complete intersection of nonsingular surfaces of degree $d,e$ in $\proj^3$.}
    (f)と同様に計算する.
    \[
        p_g(Y)
        =\dim_k \Gamma(\proj^n, \shO_{\proj^3}(d+e-n-1))
        =\binom{(d+e-3-1)+3}{3}
        =\binom{d+e-1}{3}.
    \]

\section{Blowing Up a Nonsingular Subvariety.} %% Ex8.5 

\section{The Infinitesimal Lifting Property.} %% Ex8.6 
    $k$ :: algebraically closed field,
    $A$ :: finitely generated $k$-algebraとし,
    $\Spec A$ :: nonsingular variety/$k$と仮定する.
    さらに$0 \to I \to B' \to B \to 0$を$k$-algebraの完全列とし,
    $I^2=0$とする.

    以下を示す.
    \begin{Thm}
        $A$はInfinitesimal Lifting Propertyを持つ.
        すなわち,
        任意の$k$-algebra homomorphism :: $f: A \to B$に対し,
        次の図式を可換にする$g: A \to B'$がただひとつ存在する.
        \[
        \xymatrix@R=15pt@C=30pt
        {
            {} & 0 \ar[d]\\
            {} & I \ar[d]\\
            {} & B'\ar[d]\\
            A \ar[r]_-{{}^{\forall} f} \ar@{-->}[ru]^{{}^{\exists_1} g}& B \ar[d]\\
            {} & 0 \\
        }
        \]
    \end{Thm}
    $g$は$f$のliftingと呼ばれる.
    
    \subsection{$Der_k(A,I)=\{ g-g' \mid g,g' \text{ :: lifting of }f\}$.}
    Matsumura, p.191と同じ議論をする.
    $\pi: B' \to B$を与えられた完全列中の全射としよう.

    \paragraph{Consider $I$ as $A$-module.}
    $a \in A$に対し,$\pi^{-1}(f(a))$の元をひとつ取って
    $\tilde{a} \in B'$とする.
    これをもちいて$i \in I$に対し$a \cdot i=\tilde{a}i$と置く.
    これがwell-definedであることは$I^2=0$から従う.
    実際,
    $\tilde{a}_1, \tilde{a}_2 \in \pi^{-1}(f(a))$をとると,
    $\tilde{a}_1-\tilde{a}_2 \in I$なので
    \[ \tilde{a}_1 i-\tilde{a}_2 i \in I i=0. \]

    \paragraph{$\subseteq$.}
    $g, g'$を$f$のliftingとし,$\theta=g-g'$とする.
    まず$\im \theta \subseteq I$を確かめよう.
    図式の可換性から次が得られる.
    \[ \pi \circ \theta=\pi \circ g- \pi \circ g'=f-f=0. \]
    よって$\im \theta \subseteq \ker \pi=I$.
    次に$x, y \in A$をとり,
    $\theta$がLeibniz ruleを満たすことを確かめる.
    \begin{align*}
        {}& \theta(xy) \\
        =&  g(x)g(y)-g'(x)g'(y) \\
        =&  g(x)g(y)-g'(x)g(y)+g'(x)g(y)-g'(x)g'(y) \\
        =&  (g(x)-g'(x)) g(y)+g'(x) (g(y)-g'(y)) \\
        =&  \theta(x) g(y)+g'(x) \theta(y) \\
        =&  \theta(x) \cdot y+x \cdot \theta(y)
    \end{align*}
    ここで$\pi \circ g(y)=f(y)$より
    $g(y) \in \pi^{-1}(f(a))$であることに注意せよ.
    $g'(x)$も同様.
    $\theta(x+y)=\theta(x)+\theta(y)$は自明である.

    \paragraph{$\supseteq$.}
    $g$を$f$のliftingとし,$\theta \in Der_k(A,I)$をとる.
    $g'=g+\theta$が$f$のliftingであることを示そう.
    まず準同型であることを示す.
    和を保つことは明らかなので積を保つことを見る.
    $x, y \in A$をとる.
    \begin{align*}
        {}& g'(xy) \\
        =&  g(x)g(y)-\theta(x) \cdot y-x \cdot \theta(y) \\
        =&  g(x)g(y)-\theta(x) g(y)-g(x) \theta(y)+\theta(x)\theta(y) \\
        =&  (g(x)-\theta(x))(g(y)-\theta(y)) \\
        =&  g'(x)g'(y)
    \end{align*}
    ここで$\theta(x)\theta(y) \in I^2=0$と
    $g(x)\in \pi^{-1}(f(x)), g(y) \in \pi^{-1}(f(y))$を用いた.
    以上から$g'$も代数の準同型.
    さらに
    $\im \theta \subseteq \ker \pi=I$だから,
    \[ \pi \circ g'=f+\pi \circ \theta=f. \]
    よって$g'$はlifting.

    \subsection{$P=k[x_1,\dots,x_n] \to B'$.}
    $P=k[x_1,\dots,x_n]$とし,$A=P/J$とする.
    次の図式を可換にする写像$h$の存在を示す.
    \[
    \xymatrix@R=15pt@C=30pt
    {
        0 \ar[d]& 0 \ar[d]\\
        J \ar[d]& I \ar[d]\\
        P \ar[d]_-{\rho}\ar@{-->}[r]^-{h}& B'\ar[d]^-{\pi}\\
        A \ar[d]\ar[r]_-{f}& B \ar[d]\\
        0 & 0 \\
    }
    \]
    $h$は$x_i$の像で決定されるから,
    $b_i \in \pi^{-1}(f \circ \rho(x_i)))$を選び,
    $h: x_i \mapsto b_i$で写像を定めれば良い.
    $b_i$のとり方から図式が可換になることは明らか.

    可換性から,$\pi(h(J))=f(\rho(J))=0$.
    したがって$h(J) \subseteq \pi^{-1}(0)=I$.
    また$h(J^2)=(h(J))^2 \subseteq I^2=0$.
    このことから,次のように$A$-module homomorphismが定まる.
    \begin{defmap}
        \bar{h}:& J/J^2& \to& I \\
        {}& j \bmod J^2& \mapsto& h(j)
    \end{defmap}
    
    \subsection{Complete the proof.}
    $\Spec A \subseteq \Spec P=\affine^n$がnon-singularであることから,
    Thm8.17が使える.
    $\Gamma(\Spec A, -)$がleft-exactであったこととThm8.4から,
    次はexact.
    \[
    \xymatrix
    {
        0 \ar[r]& J/J^2 \ar[r]^-{\delta}& \Der_{P/k} \otimes_P A \ar[r]& \Der_{A/k} \ar[r]& 0.
    }
    \eqno(*)
    \]
    これを$\Hom_A(-, I)$で写して次を得る.
    \[
    \xymatrix
    {
        0 \ar[r]& Der_k(A,I) \ar[r]& Der_k(P,I) \ar[r]^-{\delta^*}& \Hom_A(J/J^2,I).
    }
    \eqno(**)
    \]

    ここで$\delta^*$は3つの写像の合成である.
    \[
    \xymatrix@R=10pt
    {
        Der_k(P,I) \ar[r]^-{\iso}& \Hom_P(\Der_{P/k}, I) \ar[r]^-{\iso} &
        \Hom_A(\Der_{P/k} \otimes_P A, I) \ar[r]^-{(-) \circ \delta}& \Hom_A(J/J^2,I) \\
        D \ar@{|->}[r]& \phi \ar@{|->}[r]&
        \iota \circ (\phi \otimes_P \id[A]) \ar@{|->}[r]& (\iota \circ (\phi \otimes_P \id[A])) \circ \delta.
    }
    \]
    $\iota: I \otimes A \to I$は標準的同型写像である.
    $\delta^*(D)=(\iota \circ (\phi \otimes_P \id[A])) \circ \delta$が
    どのようなものか計算しておこう.
    $x \in J$をとり$\bar{x}=x \bmod J^2$とおく.
    \[
        \delta^*(D)(\bar{x})
        =((\iota \circ (\phi \otimes_P \id[A])) \circ \delta)(\bar{x})
        =\iota((\phi \otimes_P \id[A])([\der[P/k] x] \otimes 1_A))
        =\iota(Dx \otimes 1_{A})
        =Dx.
    \]

    \begin{Claim}
        $\delta^*: Der_k(P,I)(\iso \Hom_P(\Der_{P/k}, I)) \to \Hom_A(J/J^2,I)$は全射である.
    \end{Claim}

    この主張を仮定すると,
    $\delta^*(\theta)=\bar{h}$を満たす
    $\theta \in Der_k(P,I) \subset \Hom_k(P, B')$が存在する.
    $x \in J$とすると次のよう.
    \[ \delta^*(\theta)(x \bmod J^2)=\bar{h}(x \bmod J^2)=h(x). \]
    一方,上で述べた$\delta^*(\theta)$の計算から,
    $\delta^*(\theta)(x \bmod J^2)=\theta(x)$.
    よって$x \in J$について$h(x)=\theta(x)$が成立する.
    すなわち$h'=h-\theta$と置くと$h'(J)=0$.
    なので$h': P \to B'$から$g: A \to B'$が誘導される.
    \begin{defmap}
        g:& A& \to& B' \\
        {}& x \bmod J& \mapsto& h(x)-\theta(x)
    \end{defmap}
    これが求めていた写像である.
    実際,$x \in P$について,
    \[ \pi \circ g(x \bmod J)=\pi(h(x)-\theta(x))=f(x \bmod J)-\pi(\theta(x))=f(x \bmod J). \]
    $\im \theta \subseteq I=\ker \pi$に注意.

    \begin{proof}
        完全列
        \[
        \xymatrix
        {
            0 \ar[r]& J/J^2 \ar[r]^-{\delta}& \Der_{P/k} \otimes_k A \ar[r]& \Der_{A/k} \ar[r]& 0.
        }
        \eqno(*)
        \]
        に現れる加群$\Der_{A/k}$に対応するsheaf of modules on $X$は,
        Thm8.17よりlocally freeである.
        そのため$\Der_{A/k}$ :: projective $A$-module
        (by Eisenbud, Ex4.11b).
        このことから,完全列(*)がsplitすることがわかる.
        \[
        \xymatrix
        {
            {} & \Der_{A/k} \ar[ld] \ar[d]^-{\id} \\
            \Der_{P/k} \otimes_k A \ar@{->>}[r]& \Der_{A/k}
        }
        \]
        $\delta$のretructを
        $r: \Der_{P/k} \otimes_k A \to J/J^2$とおく.
        すると任意の$\phi \in \Hom_A(J/J^2,I)$に対して,
        \[
            ((-) \circ \delta)(\phi \circ r)
            =(\phi \circ r) \circ \delta
            =\phi \circ \id[J/J^2]
            =\phi.
        \]
        すなわち,$(-) \circ \delta(=\Hom_A(\delta, I))$は全射である.
        よってこれに二つの同型を合成した$\delta^*$も全射.
    \end{proof}

    \subsection*{Notes}
    証明した図式で$\Spec$をとると,
    次のように成る.
    \[
    \xymatrix@R=15pt@C=30pt
    {
        {} & \ar@{-->}[ld]_-{{}^{\exists_1} g} Y'\\
        X  & \ar[l]^-{{}^{\forall} f} Y \ar[u]
    }
    \]
    $Y \to Y'$がclosed immersionであるとき,
    $Y \subseteq Y'$をinfinitesimal thickening of $Y$と呼ぶ.
    Hartshorne, ``Deformation Theory"を参照せよ.

\section{Classifying Infinitesimal Extension: One Case.} %% Ex8.7 
    \paragraph{Infinitesimal Extension.}
    $X$ :: scheme of finite type /$k$,
    $\shF$ :: coherent sheaf on $X$とする.
    この時,次の条件を満たすsheaf of ideal :: $\shI$をもつ
    $X'$ :: scheme/$k$の分類を考える:
    \begin{enumerate}
        \item $\shI^2=0$,
        \item $(X', \shO_{X'}/\shI) \iso (X, \shO_X)$,
        \item $\shI \iso \shF$ as $\shO_X$-module.
    \end{enumerate}
    $X', \shI$の組をinfinitesimal extension of $X$ by $\shF$と呼ぶ.
    
    \paragraph{Trivial One.}
    trivialなものは次のように構成される.
    すなわち,$\basesp X'=\basesp X$とし,
    structure sheafを$\shO_{X'}=\shO_X \ast \shF$とする.
    これはMatsumura, p.191にあるtrivial extensionとほぼ同じ構成方法である.

    \paragraph{Setting.}
    次の場合のinfinitesimal extensionを考える:
    $X$ :: non-singular affine scheme of finite type /$k$.
    coherent sheaf :: $\shF$と,
    infinitesimal extension of $X$ by $\shF$ :: $(X', \shI)$を任意にとる.

    \paragraph{About Global Sections.}
    $B'=\Gamma(X', \shO_{X'}), I=\Gamma(X', \shI), A=\Gamma(X, \shO_X)$とする.
    $\Gamma(X',-)$は単射を保つ関手なので
    包含写像$\shI \inclmap \shO_X$から
    $I \inclmap B'$が得られる.
    また$\shI^2$は$U \mapsto (\Gamma(U, \shI))^2$で定まるsheafだから
    \footnote
    {
        このpresheafがsheafであることを示せば良い.
        問題はgluability axiomであるが,
        これは$(t|_{U_i})^2=(t^2)|_{U_i}$なので成立する.
        この等式自体はgermを見れば分かる.
    },
    $I^2=(\Gamma(X', \shI))^2=0$となる.
    $B=B'/I=\Gamma(X', \shO_{X'}/\shI)$とおこう.

    \paragraph{Lifting of $\id: A \to A$.}
    同型$(X', \shO_{X'}/\shI) \iso (X, \shO_X)$から
    環同型$u: A \isomap B'/I=B$が得られる.
    仮定より$X=\Spec A$はnonsingularなので,
    Ex8.6から,$A \to A \iso B$のliftingが存在する.
    ここで$\pi'$は
    標準的全射$B' \to B'/I=B$と
    環同型$u^{-1}: B \isomap A$の合成である.
    \[
    \xymatrix@R=15pt@C=30pt
    {
        {}   & 0 \ar[d]\\
        {}   & I \ar[d]\\
        {}   & B' \ar[d]^-{\pi'}\\
        A \ar[r]_-{\id}\ar@{-->}[ru]^-{\sigma}& A \ar[d]\\
        {} & 0 \\
    }
    \]
    よって$\pi': B' \to B$はsplitする.
    また,$I, B', B$を$A$-moduleとみなすことが出来る.
    $M=\Gamma(X, \shF)$とすると
    条件$\shI \iso \shF$ as $A$-moduleから
    $I \iso M$ as $A$-module.
    以上をまとめて,以下を示すことが出来る.
    \begin{Claim}
        \[ B' \iso A \ast M \text{ as $A$-module}. \]
    \end{Claim}
    よってこの設定では,
    infinitesimal extension of $X$ by $\shF$は
    trivialなものしかない.
    \begin{proof}
        five lemmaを用いる.
        以下の$A$-moduleの可換図式を見よ.
        \[
        \xymatrix
        {
            0 \ar[r]\ar@{=}[d]& M \ar[r]\ar[d]^-{v}&
            A \ast M \ar[r]\ar[d]& A \ar[r]\ar[d]^-{u}\ar[ld]_-{\sigma}& 0 \ar@{=}[d]\\
            0 \ar[r]& I \ar[r]_-{\iota}& B' \ar@<-1.5pt>[r]_-{\pi}& B \ar[r]\ar@<-1.5pt>[l]_-{s}& 0
        }
        \]
        ここで$M \to A \ast M, A \ast M \to A$は標準的な入射と射影である
        (集合としては$A \ast M=A \oplus M$であったことに注意せよ).
        また,$u,v$は既に述べた同型写像である.
        $A \ast M \to B'$を
        $(a,m) \mapsto \iota \circ v(m)+s \circ u(a)$で定めると,
        図式が可換に成ることが分かる.
        よってfive lemmaより$A \ast M \iso B'$.
    \end{proof}
    
\section{Plurigenera and Hodge Numbers are Birational Invariants.} %% Ex8.8 
    $X$ ::projective ninsingular variety/$k$とする.
    正整数$n(>0)$に対して,$n$-th plurigenus of $X$を
    \[ P_n=\dim_k \Gamma(X, \shCano_X^{\otimes n}) \]
    と定める.
    また$0 \leq q \leq \dim X$についてHodge numbersを
    \[ h^{q,0}=\dim_k \Gamma(X, \Der_{X/k}^{\wedge q}) \]
    と定める.
    plurigenusとHodge numbersがbirational invariantであることを示す.

    category of sheaves of modules on $X$の自己関手$M$を,
    $f^*$ (inverse image functor)と可換であるものとする.
    $M$は例えば$\square^{\otimes n}$や$\square^{\wedge q}$である.
    Thm8.19 (geometric genus is birational invarational invariant)
    の証明を見ると,この証明方法は,
    $\dim_k \Gamma(X, M \Der_{X/k})$が
    birational invariantであることの証明に使える.

\end{document}
