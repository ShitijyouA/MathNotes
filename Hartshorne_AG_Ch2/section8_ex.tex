\documentclass[a4paper]{jsarticle}
\usepackage{../math_note, exercise}
\usepackage[all]{xy}
\renewcommand{\thesection}{Ex8.\arabic{section}}

\begin{document}
\section{Strengthen Some Results in the Text.} %% Ex8.1 

\section{$0 \to \shO_X \to \shE \to \shE' \to 0$.} %% Ex8.2 
    $X$ :: variety of dimension $n$ over $k$,
    $\shE$ :: locally free sheaf of rank $>n$,
    $V \subset \Gamma(X,\shE)$ :: $k$-vector space of global sections which generate $\shE$
    とする.
    $X$ :: varietyより$X$ :: connectedなので$\shE$のrankは$X$全体で一定である.
    $\rank \shE=r (>n)$としておこう.

    $x \in X$について次の写像を考える.
    \begin{defmap}
        \phi_x:& V \otimes_{k} k(x)& \to& \shE_x \otimes_k k(x) \\
        {}& s \otimes \alpha& \mapsto& s_x \otimes \alpha
    \end{defmap}
    $k(x)$-linear mapであることは明らか.
    $\shE_x \otimes_k k(x) \iso \shE_x/\I{m}_x \shE_x$と
    $V$ :: global generators of $\shE$から,これはsurjective.
    なので
    \[
        \dim_{k(x)} \ker \phi_x
        =\dim_{k(x)} V \otimes_{k} k(x)-\dim_{k(x)} \shE_x \otimes_k k(x)
        =\dim_k V-r.
    \]
    $\shE_x \otimes_k k(x) \iso \shE_x/\I{m}_x \shE_x$と$\phi_x$の定義の仕方から,
    $\ker \phi_x=\{ s \otimes \alpha \mid s_x \alpha \in \I{m}_x\shE_x \}.$

    $V$をprojective linear spaceとみなして,
    $B \subseteq X \times V$を次のように置く.
    \[ B=\{ (x, s) \mid x \in X, s \in V, s_x \in \I{m}_x\shE_x \}. \]
    $B$は$X \times V$のclosed subschemeである.
    (以下がclosedであることはEx2.16を参照.)
    \[ B=\bigcap_{s \in V} \pr_1^{-1}(\{ x \in X \mid s_x \in \I{m}_x\shE_x \}). \]
    $\pr_1|_{B}: B \to X$を$p_1$と略す.
    $p_1$についての$x \in X^+$のfiberは,Ex3.10aより,次のような点から成る.
    \[ \basesp B_x=p_1^{-1}(x)=\{x\} \times \{ s \in V \mid s_x \in \I{m}_x\shE_x \}. \]
    これは$\ker \phi_x$と同型である.
    したがって$\dim B_x=\dim_k V-r$.
    $B$ :: irreducibleだから,
    \[ \dim B=\dim_k V-r+n. \]
    $r>n$なので,$\dim B<\dim_k V=\dim V-1$.
    ($\dim V$はprojective spaceとしての次元.)
    したがって$\pr_2|_B$ :: not surjective.
    よって$s \in V-\pr_2(B)$が存在し,
    この$s$と任意の$x \in X$について$s_x \not\in \I{m}_x\shE_x$が成り立つ.

    $\psi$を以下で定める.
    \begin{defmap}
        \psi:& \shO_X& \to& \shE  \\
        {}& \sect{U}{\sigma}& \mapsto& \sect{U}{(s|_U) \cdot \sigma}
    \end{defmap}
    $s_x \not\in \I{m}_x\shE_x$から,($(s|_U)^{-1}$が存在し,)これはinjective.
    Prop5.7から$\coker \psi$もcoherent sheaf.
    すなわち,以下はcoherent sheafのexact sequence.
    \[ 0 \to \shO_X \to \shE \to \coker \psi \to 0. \]
    Ex5.7bから$\coker \psi$はlocally free.

\section{Product Schemes.} %% Ex8.3 

\section{Complete Intersections in $\proj^n$.} %% Ex8.4 

\section{Blowing Up a Nonsingular Subvariety.} %% Ex8.5 

\section{The Infinitesimal Lifting Property.} %% Ex8.6 

\section{Classifying Infinitesimal Extension: One Case.} %% Ex8.7 

\section{Plurigenera and Hodge Numbers are Birational Invariants.} %% Ex8.8 

\end{document}
