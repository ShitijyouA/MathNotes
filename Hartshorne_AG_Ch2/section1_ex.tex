\documentclass[a4paper]{jsarticle}
\usepackage{../math_note, exercise}
\usepackage[all]{xy}
\usepackage{nicefrac}
\renewcommand{\thesection}{Ex1.\arabic{section}}

\newcommand{\shA}{\mathcal{A}}
\newcommand{\shB}{\mathcal{B}}
\newcommand{\shF}{\mathcal{F}}
\newcommand{\shG}{\mathcal{G}}
\newcommand{\shH}{\mathcal{H}}
\newcommand{\shI}{\mathcal{I}}
\newcommand{\shJ}{\mathcal{J}}
\newcommand{\shL}{\mathcal{L}}

\newcommand{\Sh}{\mathbf{Sh}}
\newcommand{\PSh}{\mathbf{PSh}}

\usepackage[scr=boondoxo,scrscaled=1.0]{mathalfa}
\newcommand{\shHom}{\mathscr{H\!o\!m}}

\newcommand{\ftorSh}{\mathit{Sh}}
\newcommand{\ftorFgt}{\mathit{Fgt}}
\newcommand{\ftorSec}{\mathit{Sec}}
\newcommand{\ftorEt}{\text{\textit{\'Et}}}

\newcommand{\res}{\operatorname{res}}
\newcommand{\emb}{\operatorname{emb}}
\newcommand{\Spe}{\text{Sp\'e}}

\newcommand{\OpenIn}{\text{ :: open in }}

\begin{document}
    このnoteではHartshorne ``Algebraic Geometry'' p.61にある
    sheaf property (3)をIdentity Axiomと呼び,
    同じく(4)をGluability Axiomと呼ぶ.
    これらの名称はVakil ``Foundations of Algebraic Geometry''にあるものである.

\section{Constant Sheaf is Associated to Constant Presheaf.} %% Ex1.1 
    $A$ :: abelian group, $X$ :: topological spaceとする.
    任意の空でない開集合$U \subseteq X$について$\shA(U)=A$とし,
    restriction map $\rho_{UV}: \shF(U) \to \shF(V)$は$\id{A}$とする.
    この$\shA$をconstant presheafと呼ぶ.
    $\shA$に対応するsheafを$\shA^+$としよう.
    また,開集合$U \subseteq X$に対し,
    $\hat{\shA}=\{ f:U \to A ~|~ f :: \text{continus}. \}$とおく.
    $\shA^+=\hat{\shA}$を示そう.
    
    $\shA$のgermを考える.
    明らかに$\varinjlim_{P \in U} \shA(U)=\varinjlim_{P \in U} A \cong A$.
    よって$\shA$のgermは$A$の元と同一視出来る.
    すると,$\shA^+(U)$の元$s$は,以下の条件を満たすものである.
    \[
        \Forall{P \in U} P \in {}^{\exists}V \subseteq U,~~ \Exists{a \in \shF(V)}
        \Forall{Q \in V} s(Q)=a_Q=a.
    \]
    これは$\shA^+(U)$の元$s$がlocally constantな写像であることを言っている.
    locally constantであれば連続であることは自明
    ($\shA^+(U) \supseteq \hat{\shA}(U)$).
    逆に連続なsectionは$s^{-1}(\{a\})$が開集合になるのでlocally constantとなる
    ($\shA^+(U) \subseteq \hat{\shA}(U)$).
    よって$\shA^+=\hat{\shA}$.

\section{The Image/Kernel in a Sheaf/Stalk.} %% Ex1.2 
    \subsection{ASSERTION.}
    $\shF'$を$\shF$のsubsheafだとする.
    この時,以下の写像$\iota_{\shF'_P}^{\shF_P}: \shF'_P \to \shF_P$に依って
    $\shF'_P$は$\shF_P$のsubgroupとみなせる.
    なお,germは$\sim_P$についての同値類(点ではなく集合)とみなす.
    $\sim_P$は「点$P$の開近傍において二つのsectionが一致する.」という同値関係である.
    \[
    \iota_{\shF'_P}^{\shF_P}(s_P)=
        \left\{ \germ{U}{\sigma} ~\middle|~ 
            \begin{array}{@{}c@{}}
                P \in U, \sigma \in \shF(U), \\ 
                P \in {}^{\exists}{V} \subseteq U,~ \germ{V}{\sigma} \in s_P.
            \end{array}
        \right\}
        \Big/ \sim_P
    \]
    $\germ{U}{\sigma}$は本文p.62の記号である.
    以下,$\iota_{\shF'_P}^{\shF_P}$は適宜$\iota$と略す.
    $s_P=t_P$であるとき$\iota(s_P)=\iota(t_P)$であることは
    定義の``$\germ{V}{\sigma} \in s_P$''の部分から明らか.
    この写像が単射であることは以下のように示される.
    まず互いに異なる$s_P, t_P \in \shF'_P$をとる.
    すると$\germ{U}{\sigma} \in s_P \setminus t_P$が取れる.
    明らかに$\germ{U}{\sigma} \in \iota(s_P)$.
    この$\germ{U}{\sigma}$について,開集合$U$をより小さい$U'$に取り替えても
    $\germ{U}{\sigma} \in s_P \setminus t_P$となる.
    これは$s_P, t_P$が$\sim_P$についての同値類だからである.
    したがって$\germ{*}{\sigma}$は$\iota(t_P)$に属さない.
    以上から$\iota(s_P) \neq \iota(t_P)$.

    $\phi: \shF \to \shG$をmorphisms of sheavesとする.
    以下の例ではsubsheafのstalk $(\ker \phi)_P$と$\ker \phi_P \subset \shF_P$が一致しない.
    まず$\shF, \shG$をどちらも実直線$\R$上の連続な関数がなす層(変数は$x$)とし,
    $\phi(f)=f-x$とする.
    この$\phi$でramp function $\mathit{ramp}(x)=[x \geq 0]x$を写したものは
    $\phi(\mathit{ramp})(x)=[x<0](-x)$となる.
    これは明らかに$x=1$の近傍$(0,2)$で0になるから,$\germ{(0,2)}{\mathit{ramp}} \in \ker \phi_P$.
    また,近傍を$(-2,2)$としても,
    $\germ{(0,2)}{\mathit{ramp}} \sim_P \germ{(-2,2)}{\mathit{ramp}} \in \ker \phi_P$.
    しかし,$\mathit{ramp}|_{(-2,2)} \neq 0$だから
    $\mathit{ramp} \not \in (\ker \phi)((-2,2))$となる.
    なので$(\ker \phi)_P$に$\germ{(-2,2)}{\mathit{ramp}}$は入っていない.
    よって$(\ker \phi)_P$と$\ker \phi_P$は上で定義した$\iota$を介さなければ一致しない.
    しかし,この二つを$\shF_P$のsubgroupとみなせば,
    一致していると言うことも出来る.
    Hartshroneはこの意味で$(\ker \phi)_P=\ker \phi_P$と主張している.

    \subsection{Preparing.}
    Ati-Mac Ex2.19から,加群のdirect limitはexact functorである
    これの証明は\url{https://math.stackexchange.com/questions/121122}などにある.
%    以下の図式でdiagram chasingを行い,
%    $\im (\psi \circ \phi)_P=0~(\implies \im \phi \subseteq \ker \psi)$と
%    $\im \phi \supseteq \ker \psi$を示せばよい.
%    ただし$U$は点$P$を含む十分小さな開集合である.
    このことをsheafのexact sequenceに用いたいが,使えることは自明ではない.
    実際,$\xymatrix{ \shF \ar[r]^-{\phi}& \shG \ar[r]^-{\psi}& \shH }$
    がexactであっても,加群の列
    $\xymatrix{ \shF(U) \ar[r]^-{\phi_U}& \shG(U) \ar[r]^-{\psi_U}& \shH(U) }$が
    完全であるとは限らないからである.

    点$P$を任意の点とし,
    $\ker \psi_P \subseteq \im \phi_P$を示す.
    まず,$\ker \psi_P$からgerm $s_P$をとる.
    \[
    \xymatrix@R=5pt
    {
    \shF_P \ar[r]^-{\phi_P}& \shG_P \ar[r]^-{\psi_P}& \shH_P \\
    {} & s_P \ar@{|->}[r]_-{\psi_P} & 0_P
    }
    \]
    すると点$P$の開近傍$U$と,section $\sigma \in \ker \psi_U=(\ker \psi)(U)$が取れて,$\sigma_P=s_P$となる.
    \[
    \xymatrix@R=5pt
    {
    \shF_P \ar[r]^-{\phi_P}& \shG_P \ar[r]^-{\psi_P}& \shH_P \\
    {} & s_P \ar@{|->}[r] & 0_P \\
    {} & {} & {} \\
    {} & {} & {} \\
    {} & \sigma \ar@{|->}[uuu] \ar@{|->}[r] & 0 \ar@{|->}[uuu]\\
    \shF(U) \ar[r]_-{\phi_U}& \shG(U) \ar[r]_-{\psi_U}& \shH(U) \\
    }
    \]
    仮定より,$\sigma \in (\ker \psi)(U)=(\im \psi)(U)$.
    なので,
    \[ (\im^{pre} \psi)_P=(\im \phi)_P \ni \sigma_P=s_P \in \ker \phi_P. \]
    よって以下が得られる.
    \[ P \in {}^{\exists}V \subseteq U,~~ \sigma|_V \in (\im^{pre} \psi)(V)=\im \psi_V. \]
    以上より,$\sigma_P=s_P$かつ$\im \psi_V \ni \sigma|_V \in \ker \psi_V$.
    あとは$\phi_V(\tau)=\sigma|_V$となる$\tau \in \shF(V)$をとり,
    図式の可換性を用いれば良い.

    \subsection{Prooves.}
    \begin{enumerate}[(a)]
        \item $\Forall{P \in X} (\ker \phi)_P=\ker \phi_P, (\im \phi)_P=\im \phi_P$.
        \item $\phi$ :: inj/surj $\iff$ $\Forall{P \in X} \phi_P$ :: inj/surj.
        \item $\shF \to \shG \to \shH$ :: exact $\iff$ $\Forall{P \in X} \shF_P \to \shG_P \to \shH_P$ :: exact
    \end{enumerate}
    \paragraph{Proof of Half of (c).}
    以下が成り立つ.
    \[
        \shF \xrightarrow{\phi} \shG \xrightarrow{\psi} \shH \text{ :: exact}
        \implies
        \Forall{P \in X}
        \shF_P \xrightarrow{\phi_P} \shG_P \xrightarrow{\psi_P} \shH_P \text{ :: exact}.
    \]
    ただし$\shF,\shG,\shH$は位相空間$X$上のsheafである.
    この命題は(c)の半分である.

    \paragraph{Proof of (a).}
    $\phi: \shF \to \shG$に対し,
    $0 \to \ker \phi \xrightarrow{i} \shF \xrightarrow{\phi} \shG$はexact.
    このことから
    $0 \to (\ker \phi)_P \xrightarrow{i_P} \shF_P \xrightarrow{\phi_P} \shG_P$はexact.
    よって$\im i_P=\ker \phi_P$が得られる.
    明らかに$i_P$はinjectiveだから,$(\ker \phi)_P \cong \im i_P=\ker \phi_P$となる.
    また,$\shF \xrightarrow{\phi} \shG \to \im \phi \to 0$がexactであることから
    $(\im \phi)_P \cong \im \phi_P$も得られる.

    \paragraph{Proof of Remained Part of (c).}
    任意の点$P \in X$について,
    $\shF_P \xrightarrow{\phi_P} \shG_P \xrightarrow{\psi_P} \shH_P$がexactであったとする.
    そこで任意の開集合$U \subset X$と,任意のsection $s \in \shF(U)$を取る.
    以下のように$\im \phi \subseteq \ker \phi$が示される.
    \begin{align*}
        {}&         s \in (\im \phi)(U) \\
        \implies&   \Forall{P \in U} s_P \in (\im \phi)_P=\im \phi_P \\
        \iff&       \Forall{P \in U} s_P \in \ker \phi_P \\
        \iff&       \Forall{P \in U} P \in {}^{\exists} V_P \subseteq U,~~ s|_{V_P} \in (\ker \phi)(V_P) \\
        \implies&   s \in (\ker \phi)(U)
    \end{align*}
    最後の行でGluability Axiomを用いた.
    この証明で$\ker$と$\im$を交換すれば$\im \phi \supseteq \ker \phi$も示され,
    よって$\im \phi=\ker \psi$が得られる.

    \paragraph{Proof of (b).}
    $0 \to \shF \to \shG$と$\shF \to \shG \to 0$に(c)を用いれば良い.

    \section{Surjectivity of Morphism is (Not) Local Property.} %% Ex1.3 
    \subsection{Paraphrase of Surjectivity.}
    $\shF, \shG: X \to A$, $\phi: \shF \to \shG$について
    $\phi$ :: surjが以下の命題と同値であることを示す.
    \[
        (*)~~~
        \Forall{U \text{ :: open in } X}
        \Forall{s \in \shG(U)}
        \bigcup {}^\exists U_i=U,~~
        \Exists{t_i \in \shF(U_i)}
        \Forall{i} \phi(t_i)=s|_{U_i}.
    \]
    $\phi$ :: surjならばcovering $\{U_i\}$として$U$をとり,
    $\phi(t)=s$となる$t$を$t_i$とすれば良い.

    逆を示す.
    Ex1.2bより,任意の$P \in U$について$\phi_P$ :: surjであることを示せば良い.
    仮定より$P \in V \subseteq U$となる$V$($(*)$中の$U_i$)が存在し,
    $\phi_P(t_P)=s|_V=s_P$を満たす$t_P \in \shF(V) \subseteq \shF_P$が存在する.
    よって$\phi_P$ :: surj.

    \subsection{Give an Counterexample.}
%    $X=\R, Y=\R-\{0\}$とし,$d(x)=[x \neq 0]$とする.
%    さらに$U \subseteq X$ :: openに対し$\phi_U(s)(x)=s(x) ~ (x \neq 0)$とおく.
%    すると$\phi$は
%    $X$上の実連続関数の層$\shF$から
%    $Y$上の実連続関数の層$\shG$へのmorphismである.

%    点$P \in X$について$\phi_P$が全射であることを見よう.
%    $P \in U, s \in \shF(U)$とすると,$\phi_U(s)$は$U-\{0\}$で$s$と全く同じ値を取る.
%    なので$\phi_U(s)$の$U - \{0\}$の各点における値から,
%    $\phi_U(s)$を$Y=\R-\{0\}$全体に拡張することは容易.
%    (開区間$U$の両端への$s$の片側極限をとる.)
%    よって$\phi_P$は全射である.

%    しかし,$U=(-1,0) \cup (0,1)$とすると,以下のような連続関数は$\phi_U$の値域に入らない.
%    \[ s(x)=[-1 < x < 0](x-1)+[0 < x < 1](x+1) \]
%    このことを確かめよう.
%    $s(x)=\phi_U(t)(x)$なる関数$t$が存在したとすると,
%    $t$は$s$の値を$x=0$において補完したものである.
%    しかし$s(x)$のグラフを書けば明らかな通り,
%    $t(0)$をどのような値にとっても$t$は連続にならない.

\section{Induced Injective Sheaf Morphism.} %% Ex1.4 
    \subsection{Injective Presheaf Morphism Induces Injective Sheaf Morphism.}
    以下は可換図式である.
    \[
    \xymatrix
    {
        \shF^+ \ar@{-->}[r]^{\phi^+} & \shG^+ \\
        \shF \ar[u] \ar[r]_-{\phi} \ar[ur]& \shG \ar[u]
    }
    \]
    これをstalkをとる関手$\lim_{\to_{P \in U}}$で写すと,
    Prop-Def1.2の直後に言及されている$\shF_P=\shF_P^+$から,以下が得られる.
    \[
    \xymatrix
    {
        \shF_P^+ \ar[r]^{\phi_P^+} & \shG_P^+ \\
        \shF_P \ar@{=}[u] \ar[r]_-{\phi_P} \ar[ur]& \shG_P \ar@{=}[u]
    }
    \]
    この可換図式から$\phi_P=\phi_P^+$.
    よってEx1.2bから$\phi$ :: inj $\iff$ $\phi^+$ :: inj.

    \subsection{Natural Induced Map $\im \phi \to \shG$ is Injective.}
    埋め込み写像$\im^{pre} \phi \hookrightarrow \shG$はinjectiveなので,
    ここから誘導される$\im \phi \to \shG$もinjective.

\section{For Morphism of Shaves, iso=inj+surj.} %% Ex1.5 
    $\phi: \shF \to \shG$を考える.
    $\phi$がisoであることと,任意の点$P$で$\phi_P$がisoであることは同値.
    また,$\phi$がinj+surjであることと,任意の点$P$で$\phi_P$がinj+surjであることは同値である.
    これらはそれぞれProp1.1とEx1.2から理解る.
    よって$\phi_P$についてiso=inj+surjを確かめれば必要十分.
    
    \paragraph{$\phi_P$ :: iso $\implies$ $\phi_P$ :: inj+surj.}
    $\phi_P$ :: iso ならば,
    \[
        \Forall{x_1, x_2 \in \shF_P}
        \phi_P(x_1)=\phi_P(x_2)
        \implies
        \phi_P^{-1} \circ \phi_P(x_1)=x_1=x_2=\phi_P^{-1} \circ \phi_P(x_2)
        \label{prop:ex15-inj}
    \]
    すなわち$\phi_P$ :: inj.
    同時に
    \[ \Forall{y \in \shG_P} \phi_P \big( \phi_P^{-1}(y) \big)=y \]
    すなわち$\phi_P$ :: surj.

    \paragraph{$\phi_P$ :: iso $\impliedby$ $\phi_P$ :: inj+surj.}
    まず$\phi_P$ :: surjから以下が成り立つ.
    \[ \Forall{y \in \shG_P} \Exists{x \in \shF_P} \phi_P(x)=y. \]
    この命題を満たす$x \in \shF_P$は$\phi_P$ :: injからただひとつである.
    \[ \Forall{y \in \shG_P} {}^{\exists_1} x \in \shF_P,~~ \phi_P(x)=y. \]
    なので$\phi_P^{-1}(y)=x$と定めればこれは写像になる.
    なお,$\phi_P$でなく$\phi$で議論をすると,構成した$\phi$のnaturalityを示す必要がある.

\section{Short Exact Sequence of Sheaves.} %% Ex1.6 
    \subsection{Natural Map $q:\shF \to \shF/\shF'$ Has $\im q=\shF/\shF'$ and $\ker q=\shF'$.}
    quotient sheafの定義(p.65)より,
    任意の点$P$について$(\shF/\shF')_P=\shF_P/\shF'_P$
    \footnote
    {
        これはsheafification functor $sh_X: \mathbf{PSh}(X, \mathfrak{C}) \to \mathbf{Sh}(X, \mathfrak{C})$が
        forgetful functorのleft adjoint functorであること,
        及びleft adjoint functorはcolimitを保つことからも得られる.
    }.
    よって$q$から誘導される$q_P$は$\shF_P \to \shF_P/\shF'_P$の自然な写像である.
    $\im q_P=\shF_P/\shF'_P, \ker q_P=\shF'$となるから,Ex1.2aより主張が得られる.

    \subsection{If $0 \to \shF' \xrightarrow{f} \shF \xrightarrow{g} \shF'' \to 0$ is Exact, ...}
    仮定より,$0=\ker f, \im f=\ker g, \im g=\shF''$.
    よって$f$はinjで,$f|^{\im f}: \shF' \to \im f$はsurj+inj.
    なのでEx1.5よりこれはisoであり,$\shF'$は$\im f \subset \shF$と同型である.
    続けて$g:\shF \to \shF''$から誘導される$g_P: \shF_P \to \shF''_P$を考える.
    定義より$\shF_P, \shF''_P$はabelian group(abelian groupの圏でのcolimit)で,$g_P$はそのmorphism.
    だからabelian groupの準同型定理からの帰結として$\shF_P/\ker g_P=(\shF/\ker g)_P \cong \shF''_P$が得られる.
    Prop1.1より$\shF'' \cong \shF/\ker g=\shF/\im f \cong \shF/\shF'$.

\section{$\im \phi \cong \shF/\ker \phi,$ and $\coker \phi \cong \shG/\im \phi$.} %% Ex1.7 
    $\phi: \shF \to \shG$について考える.
    $\im \phi \cong \shF/\ker \phi$は以下の完全列にEx1.6bを用いて得られる.
    \[ 0 \to \ker \phi \xrightarrow{i} \shF \xrightarrow{\phi} \im \phi \to 0. \]
    ただし$i$は埋め込み写像である.
    $\coker \phi \cong \shG/\im \phi$は同様に以下の完全列から得られる.
    \[ 0 \to \im \phi \xrightarrow{\phi} \shG \xrightarrow{q} \coker \phi \to 0. \]
    ただし$q$は$q^{pre}: \shG \to \coker \phi=\shG/\im \phi$から誘導される写像.
    これが完全列であることは次のように示される.
    まずEx1.6aを用いてstalkの完全列を得る.
    \[ 0 \to \im \phi_P \xrightarrow{\phi_P} \shG_P \xrightarrow{q_P} \coker \phi_P=\shG_P/\im \phi_P \to 0. \]
    Ex1.2a,cを用いて元の列が完全であることが示される.

\section{$\Forall{U \subset X} \Gamma(U,-)$ :: left exact functor} %% Ex1.8 
    以下を$X \to A$のsheavesがなす完全列とする.
    \[ 0 \to \shF' \xrightarrow{f} \shF \xrightarrow{g} \shF''. \]
    完全列なので$0=\ker f, \im f=\ker g$.
    $\shA \mapsto \shA(U)$で定義されるfunctor $\Gamma(U,-)$により,
    この完全列は以下の列になる.
    \[ 0 \to \shF'(U) \xrightarrow{f_U} \shF(U) \xrightarrow{g_U} \shF''(U). \]
    これが完全列であることは$0=\ker f_U, \im f_U=\ker g_U$と同値.

    まず$\ker f$を考えると,定義より$0=(\ker f)(U)=\ker f_U$.
    よって$f_U$ :: inj.
    また,$\Gamma(U,-)$はfunctorだから
    \[ 0=\Gamma(U,g \circ f)=\Gamma(U,g) \circ \Gamma(U,f)=0. \]
    すなわち$g_U \circ f_U=0$,$\im f_U \subseteq \ker g_U$.

    残るは逆の包含関係である.
    まず$s \in \ker g_U \subseteq \shF(U)$を取る.
    Ex1.2aより,任意の$P \in U$について$\im f_P=\ker g_P$.
    なので任意の点$P$について$s_P \in \im f_P=\ker g_P$であり,
    $f_P(t_P)=s_P$となる$t_P \in \shF'_P$が存在する.
    そこで$s_P,t_P$の代表元$\germ{V_P}{s|_{V_P}}, \germ{V_P}{t^P|_{V_P}}$をとると
    $f_{V_P}(t^P|_{V_P})=s|_{V_P}$となる.
    同様に別の点$Q \in U, t_Q=\germ{V_Q}{t^Q|_{V_Q}}$をとると,
    $W_{PQ}:=V_P \cap V_Q$について
    \[ f_{W_{PQ}}(t^P|_{W_{PQ}})=s|_{W_{PQ}}=f_{W_{PQ}}(t^Q|_{W_{PQ}}). \]
    $0=(\ker f)(W_{PQ})=\ker f_{W_PQ}$より$f_{W_{PQ}}$はinj.
    したがって$t^P|_{W_{PQ}}=t^Q|_{W_{PQ}}$が得られる.
    $(P \in)~W_{PQ}$は$U$を被覆するから,Gluability Axiomより,
    $t|_{W_{PQ}}=t^P|_{W_{PQ}}=t^Q|_{W_{PQ}}$なる$t \in \shF'(U)$が存在する.
    morphismとrestrictionのnaturalityにより,
    \[ f_{U}(t)|_{W_{PQ}}=f_{W_PQ}(t|_{W_{PQ}})=f_{W_{PQ}}(t^P|_{W_{PQ}})=s|_{W_{PQ}} \]となるから,
    Identity Axiomより$f_{U}(t)=s$.
    以上より$\im f_U \supseteq \ker g_U$.

\section{Direct Sum.} %% Ex1.9 
    sheaves $\shF,\shG: X \to \mathfrak{C}$について,$\shF \oplus \shG$を以下で定める.
    \[ \shF \oplus \shG: U \mapsto \shF(U) \oplus \shG(U). \]
    ただし$U$ :: open in $X$.
    これがpresheafであることは自明なので,sheafであることを示す.
    以下,$U$ :: open in $X$とその開被覆$\{U_i\}$を固定する.

    \paragraph{$\shF \oplus \shG$ Satisfies Identity Axiom.}
    $s \oplus t \in \shF(U) \oplus \shG(U)$が$(s \oplus t)|_{U_i}=0 \oplus 0=0$を満たすとする.
    この仮定を論理式で書下すと,
    \[ \Forall{P \in U_i} (s \oplus t)(P)=s(P) \oplus t(P)=0 \oplus 0. \]
    abelian groupのcoproductはproductと同型だから,これは以下のように書き換えられる.
    \[ \Forall{P \in U_i} s(P)=0 \land t(P)=0. \]
    これは$s|_{U_i}=t|_{U_i}=0$と同値.
    なので$\shF,\shG$はsheafであることから$s=t=0$.
    すなわち$s \oplus t=0$.

    \paragraph{$\shF \oplus \shG$ Satisfies Gluability Axiom.}
    $s_i \oplus t_i \in \shF(U_i) \oplus \shG(U_i)$が存在し,以下を満たすとする.
    \[ \Forall{i,j} (s_i \oplus t_i)|_{U_i \cap U_j}=(s_j \oplus t_j)|_{U_i \cap U_j}. \]
    前段落と同様に書き換えて,以下が得られる.
    \[ \Forall{i,j} s_i|_{U_i \cap U_j}=s_j|_{U_i \cap U_j} \land t_i|_{U_i \cap U_j}=t_j|_{U_i \cap U_j}. \]
    $\shF,\shG$はsheafであることから,以下を満たす$s \in \shF(U_i),t \in \shG(U_i)$が存在する.
    \[ \Forall{i} s|_{U_i}=s_i \land t|_{U_i}=t_i. \]
    この$s,t$について$(s \oplus t)|_{U_i}=(s|_{U_i}) \oplus (t|_{U_i})=s_i \oplus t_i$.

    \paragraph{$\shF \oplus \shG$ is Coproduct in $\Sh(X)$.}
    以下の図式を考える.
    \[
    \xymatrix
    {
        {} & \shF \oplus \shG \ar@{-->}[dd]^-{{}^{\exists_1} [f,g]}& {} \\
        \shF \ar[rd]_-{{}^\forall f} \ar[ru]^-{i}& {} & \shG \ar[ld]^-{{}^\forall g} \ar[lu]_-{j}\\
        {} & {}^\forall \mathcal{Z} & {}
    }
    \]
    ただし$Z,f,g$は任意で,$i, j$はそれぞれ$s \mapsto s \oplus 0, t \mapsto 0 \oplus t$とする.
    すると$\shF, \shG$から$\mathcal{Z}$へ至る二つのパスをたどることで,
    この図式を可換にする$[f,g]$は以下のものしか無い事が理解る.
    \[ [f,g]: s \oplus t \mapsto f(s)+g(t). \]
    $f,g$はmorphism of abelian groupで$f(s),g(t)$はelement of abelian group.
    だから,例えば$\shF \to \mathcal{Z}$の二つのパスは次の計算の通り可換になる.
    \[ [f,g] \circ i : s \mapsto s \oplus 0 \mapsto f(s)+g(0)=f(s) ~\mapedfrom~ s: f \]
    よって$\shF \oplus \shG$はcoproduct.

\section{Direct Limit.} %% Ex1.10 
    Ex1.8のfunctor $\Gamma(-,-)$,sheafification functor $sh_X$と
    abelian categoryのdirect limit $\lim_{\to i}$を用いて,
    $\lim_{\to i}\shF_i$を以下で定める.
    \[ \Gamma(-,\lim_{\to i}\shF_i)=sh_X \lim_{\to i} \Gamma(-,\shF_i). \]
    ただし$\{\shF_i\}_{i \in I}$はdirect systemである.
    これが$\Sh(X)$のdirect limitであることを示す.

    まず,$\shL: U \mapsto \lim_{\to i} \shF_i(U)$とおく.
    これは明らかに$\PSh(X)$におけるdirect limitで
    \footnote{$\PSh(X)$がdirect limitを持つことはabelian category $\mathfrak{C}$がdirect limitを持つことによる.},
    $\shL^+=\lim_{\to i}\shF_i$を満たす.
    よってsheafification functor $sh_X$がdirect limitを保つことを見れば良い.
    次の可換図式は$\shL$のUMPを表す.
    \[
    \xymatrix
    {
        \shL \ar@{-->}[rrd]^-{\bar{f}_i}& {} & {} & {} \\
        {} & \shF_i \ar[r]_-{f_i} \ar[lu] & \shG & {}
    }
    \]
    ただし$\shG, f_i$は任意.
    sheafificationのUMPを$\bar{f}_i: \shL \to \shG$に用いて,
    次の可換図式が得られる.
    \[
    \xymatrix
    {
    \shL \ar@{-->}[rrd]^-{\bar{f}_i} \ar[rrr]^-{\theta}& {} & {} & \shL^+ \ar[lld] \ar@{-->}[ld]^-{\bar{\bar{f}}_i}\\
        {} & \shF_i \ar[r]_-{f_i} \ar[lu] & \shG & {}
    }
    \]
    よって$f_i:\shF_i \to \shG$に対して一意に$\bar{\bar{f}}_i: \shL^+ \to \shG$が存在する.
    これで$\shL^+=\lim_{\to i}\shF_i$のUMPが示せた.
    $\shF_i \to \shF_j$との可換性はmorphismを結合すれば容易に分かる.

    \subsubsection{Another Proof.} \label{sssec:1-10-1}
    sheafification functor $sh_X: \PSh(X) \to \Sh(X)$が
    Forgetful Functor $F: \Sh(X) \to \PSh(X)$の
    left adjoint functorであることを用いる.
    これはR.Vakil ``Foundations of Algebraic Geometry" Part I, 2.4.Lなどにある事実である.
    direct limitがcolimitであることと,
    ``Left Adjoint Preserves Colimits"より,
    \[ sh_X \lim_{\to i} \shF_i \cong \lim_{\to i} sh_X \shF_i \cong \lim_{\to i} \shF_i. \]

\section{Pre-Direct Limit on Noetherian Top.Sp. is Already a Sheaf.} %% Ex1.11 
sheaves $\{\shF^i\}_{i \in I}$ with morphisms $f^{ij}: \shF^i \to \shF^j$ :: direct systemとし,
    $\PSh(X)$におけるdirect limitを$\shL$で書く.
    $X$ :: noetherian topological spaceであるとき,$\shL$が予めsheafであることを示す.
    以下,$U$ :: open in $X$と開被覆$\{U_{\lambda}\}_{\lambda \in \Lambda}$を任意にとり,固定する.

    $X$ :: noetherianより,$X$ :: quasi-compact.
    なので集合$\{U_{\lambda}\}$から有限被覆$\{ U_j \}_{j \in J}$が出来る.

\section{Inverse Limit.} %% Ex1.12 
    sheaves $\{\shF^i\}$ with morphisms $f^{ij}: \shF^i \to \shF^j$ :: inverse systemとし,
    $\PSh(X)$におけるinverse limit $U \mapsto \lim_{i \leftarrow} \shF^i(U)$ を $\shL$ で書く.
    このとき$\shL$は$\Sh(X)$においてもinverse limitであることを示す.
    
    sheafification functorを$\ftorSh: \PSh(X) \to \Sh(X)$, 
    forgetful functorを$\ftorFgt: \Sh(X) \to \PSh(X)$で書く.
    $\ftorFgt$は$\ftorSh$のright adjoint functor($\ftorSh \dashv \ftorFgt$.)なので,$\lim_{i \leftarrow}$と可換
    \footnote{``Right Adjoints Preserves Limits."}.
    inverse limitはlimitなので以下が得られる.
    \[ \lim_{i \leftarrow} \ftorFgt \shF^i \cong \ftorFgt \lim_{i \leftarrow} \shF^i \cong \lim_{i \leftarrow} \shF^i. \]
    最後の$\cong$は$F$がforgetful functor,すなわちobjectを変化させないことによる.
    したがって$\PSh(X)$におけるinverse limitは$\Sh(X)$におけるinverse limitと一致する.
    まったく同様の議論で$\PSh(X)$におけるlimitは$\Sh(X)$におけるlimitに一致する.

    \subsubsection{Proof of $\ftorSh \dashv \ftorFgt$.}
    adjointの定義にはいくつか同値なものがあるが,
    ここではSteve Awodey ``Category Theory" p.214にあるCor9.5を用いる.

    $F$はobjectを変えない埋め込み写像なので,
    直ちに全単射$\tilde{\eta}_{(-)}: (-) \leftrightarrow F(-):\tilde{\epsilon}_{(-)}$がとれる.
    これにsheafificationのUMPを用いると以下の可換図式が得られる.
    \[
    \xymatrix
    {
    \ftorSh \ar@{->}[r]^-{\tilde{\eta}_{}}& \ftorFgt \ftorSh\\
    \id{\PSh(X)} \ar[ru]_-{\eta} \ar[u]^-{\theta}& {}
    }
    \hspace{15mm}
    \xymatrix
    {
    \ftorSh \ftorFgt \ar@{-->}[r]^-{\epsilon_{}}& \id{\Sh(X)}\\
    \ftorFgt \ar[ru]_-{\tilde{\epsilon}_{}} \ar[u]^-{\theta_{\ftorFgt}}& {}
    }
    \]
    こうして
    unit $\eta: \id{\PSh(X)} \to \ftorFgt \ftorSh$と
    counit $\epsilon: \ftorSh \ftorFgt \to \id{\Sh(X)}$が得られる.
    さらに,この二つの可換図式を組み合わせて,以下の可換図式が作れる.
    \[
    %% @C : 列間隔, @R : 行間隔
    \xymatrix@C=5pt@R=35pt
    {
    \id{\PSh(X)} \ar[rr]^{\theta_{}} & {} & \ftorSh  \\
    \ftorFgt \ar@{<->}[rr]^-{\tilde{\epsilon}_{}} \ar[rd]_-{\theta_{\ftorFgt}} & {} & \id{\Sh(X)} \\
    {} & \ftorSh \ftorFgt \ar[ru]_-{\epsilon_{}} & {}
    }
    \]

    さて,$\shF \in \PSh(X), \shG \in \Sh(X)$と$g: \ftorSh \shF \to \shG$を任意に取る.
    この時の可換図式は以下の(1)である.
    \[
    (1)
    \xymatrix@C=2pt@R=35pt
    {
    \shF \ar[rr]^{\theta_{\shF}} & {} & \ftorSh \shF \ar[d]^{g} \\
    \ftorFgt \shG \ar@{<->}[rr]^-{\tilde{\epsilon}_{\shG}} \ar[rd]_-{\theta_{\ftorFgt \shG}} & {} & \shG \\
    {} & \ftorSh \ftorFgt \shG \ar[ru]_-{\epsilon_{\shG}} & {}
    }
    \hspace{20pt}
    (2)
    \xymatrix@C=2pt@R=35pt
    {
    \shF \ar[rr]^{\theta_{\shF}} \ar[d]^{\bar{g}} \ar@<-2mm>@/_12mm/@{~>}[rdd] & {} & \ftorSh \shF \ar[d]^{g} \\
    \ftorFgt \shG \ar@{<->}[rr]^-{\tilde{\epsilon}_{\shG}} \ar[rd]_-{\theta_{\ftorFgt \shG}} & {} & \shG \\
    {} & \ftorSh \ftorFgt \shG \ar[ru]_-{\epsilon_{\shG}} & {}
    }
    \hspace{20pt}
    (3)
    \xymatrix@C=2pt@R=35pt
    {
    \shF \ar[rr]^{\theta_{\shF}} \ar[d]^{\bar{g}} & {} & \ftorSh \shF \ar[d]^{g} \ar@<2mm>@/^12mm/@{-->}[ldd]^{\ftorSh(\bar{g})} \\
    \ftorFgt \shG \ar@{<->}[rr]^-{\tilde{\epsilon}_{\shG}} \ar[rd]_-{\theta_{\ftorFgt \shG}} & {} & \shG \\
    {} & \ftorSh \ftorFgt \shG \ar[ru]_-{\epsilon_{\shG}} & {}
    }
    \]
    コの字型の部分をたどることで,(2)の$\bar{g}: \shF \to \ftorFgt \shG$が得られる.
    Ex1.4における$\phi^+$の作り方をなぞると,
    $\ftorSh(\bar{g})$は(2)の波矢印$\theta_{\ftorFgt \shG} \circ \bar{g}$から
    sheafificationのUMPで得られるものである.
    sheafificationをしたあとの可換図式が(3)である.
    UMPから,$\theta_{\ftorFgt \shG}$および$\theta_{\ftorFgt \shG} \circ \bar{g}$と共に
    可換な三角形をなす射は$\ftorSh(\bar{g})$に等しい.
    よって$\ftorSh(\bar{g})=\theta_{\ftorFgt \shG} \circ \tilde{\epsilon}_{\shG}^{-1} \circ g$.
    こうして$g=\epsilon_{\shG} \circ \ftorSh(\bar{g})$が得られる.

\section{Espace \'Etal\'e of a Presheaf.} %% Ex1.13 
    \subsubsection{Definition of Espace \'Etal\'e.}
    $\shF \in \PSh(X)$に対し,espace \'etal\'e of $\shF$ $\Spe(\shF)$を以下のように定義する.
    まず,集合として$\Spe(\shF)=\bigsqcup_{P \in X} \shF_P$とおく.
    projection map $\pi$とその``section" $\bar{s}$を以下で定める.
    まず,$\pi$は以下のもの.
    \begin{defmap}
        \pi:& \Spe(\shF)& \to& X \\ 
        {}& s \in \shF_P& \mapsto& P.
    \end{defmap}
    任意の$U$ :: open in $X$と$s \in \shF(U)$に対して
    $\bar{s}: U \to \Spe(\shF)$を以下で定める.
    \begin{defmap}
        \bar{s}:& U& \to& \Spe(\shF) \\ 
        {}& P& \mapsto& s_P.
    \end{defmap}
    この時,$\pi \circ \bar{s}=\id{U}$.
    すなわち,$\bar{s}$は$U$上で$\pi$の``section"である.
    そして$\Spe(\shF)$に以下のような位相を入れる: 
    任意の$U$と任意の$s$について$\bar{s}$が連続であるような最強の位相.
    これはつまり$\{\bar{s}\}$についての終位相である.

    \subsubsection{More References for Espace \'Etal\'e.}
    WikipediaのSheafのページ
    \url{https://www.wikiwand.com/en/Sheaf_(mathematics)#/The_.C3.A9tal.C3.A9_space_of_a_sheaf}
    (2017年3月30日参照)
    に概略が書かれている.
    詳細についての資料は以下の通り.
    まず,一般のespace \'etal\'e(\'etal\'e space)のcategoricalな定義が
    \url{https://ncatlab.org/nlab/show/etale+space}にある.
    \'Etal\'e spaceの圏とsheafの圏が圏同値であることの証明は
    Saunders Mac Lane, Ieke Moerdijk ``Sheaves in Geometry and Logic''の\S5-6, pp.83-90にある.
    (この命題はこの本のp.90 Cor3である.)
    同様のことが
    ``Etale cohomology course notes''
    \url{http://math.colorado.edu/~jonathan.wise/teaching/math8174-spring-2014/notes.pdf}
    の7 Etale spaces and sheaves (p.24)にあるが,
    このnoteはミスが多いしわかりにくいのでおすすめしない.

    \subsubsection{Proposition and Proof.}
    $X$上の\'etal\'e spaceをとって,
    その連続なsection全体をとる関手を$\ftorSec: \mathbf{Et}(X) \to \Sh(X)$とする.
    逆にpresheafから\'etal\'e spaceを作る関手を$\ftorEt: \PSh(X) \to \mathbf{Et}(X)$とする.
    sheafification functorが$\ftorSh=\ftorSec \ftorEt$で定義できることを示す.

    \paragraph{Plan of Proof.}
    二つの写像を定める.
    \begin{defmap}
        \alpha:& \id{\Sh(X)}& \to& \ftorSec \ftorEt \\ 
        \vspace{2ex}
        {}& s \in \shF(U)& \mapsto& [\bar{s}: P \mapsto s_P] \\
        \beta:& \ftorEt \ftorSec& \to& \id{\mathbf{Et}(X)} \\ 
        {}& P \times [\sigma:U \to \id{\mathbf{Et}(X)}]& \mapsto& \sigma(P) \\
    \end{defmap}
    ただし$U$は任意の$X$の開集合で,$P$は$U$の任意の点である.
    この$\alpha, \beta$がnatural mapかつisomorphismであることが証明できるので,
    圏同値$\mathbf{Et}(X) \simeq \Sh(X)$が示せる.
    しかし我々の目的はsheafificationのUMPであり,
    これには$\alpha$についてさえ示せれば十分である.
    この証明は Saunders Mac Lane, Ieke Moerdijk ``Sheaves in Geometry and Logic'' pp.85-86にある
    \footnote
    {
        この本では$\alpha$は$\eta$と書かれている.
        また,この本でいうcross-sectionは$\pi \circ \sigma=\id{U}$なるsectionのこと.
        $\dot{s}$は$\bar{s}$のことである.
        その他,germの記法などがだいぶ違うので注意.
    }.

    \paragraph{$\alpha$ :: natural.}
    $\shF, \shG \in \PSh(X)$とする.
    \[
    \xymatrix
    {
        \shF \ar[d]^{\phi} \ar[r]^{\alpha_{\shF}}& \ftorSec \ftorEt \shF \ar[d]^{\ftorSec \ftorEt \phi}\\
        \shG \ar[r]_{\alpha_{\shG}}& \ftorSec \ftorEt \shG
    }
    \]
    $\ftorSec \ftorEt \phi$は次のような,sectionをsectionへ写す写像である.
    \[ \ftorSec \ftorEt \phi: [P \mapsto *_P] \mapsto [P \mapsto *_P \mapsto \phi_P(*_P)]. \]
    したがって$\shF \to \ftorSec \ftorEt \shG$のどちらのパスでも
    $s \mapsto [P \mapsto \phi_P(s_P)=(\phi(s))_P]$とsectionをsectionへ写す写像になる.
    ただし$P$は$X$の点である.
    これで$\alpha$ :: naturalが示せた.

    \paragraph{$\alpha$ :: iso.}
    まず$\alpha$ :: inj はIndentity Axiomから容易に示されるので略す.
    $\alpha$ :: surjの証明は長い.
    まず$U \OpenIn X, \sigma \in (\ftorSec \ftorEt \shF)(U)$を任意に取る.
    すると$\ftorSec \ftorEt$の定義から,以下が成り立つ.
    \[ \Forall{P \in U} P \in {}^{\exists} V \subseteq U \text{ :: open},~~ \Exists{s \in \shF(V)} \sigma(P)=s_P. \]
    $\ftorEt \shF$の位相は像位相であり,かつ$\alpha(s)=\bar{s}$は明らかに単射.
    なので$\alpha(s)(V)=\bar{s}(V)$はopenである
    \footnote
    {
        $\bar{s}$の像位相において,
        開集合$V$の像が開集合であることは
        $\bar{s}^{-1} \circ \bar{s}(V)$が開集合であることと同値だが,
        単射性から,この集合は$V$に等しい.
    }.
    しかも$\sigma$ :: continuousだから,
    $\sigma(S) \subseteq \alpha(\sigma)(V)$なる$P \in S \subseteq \sigma^{-1}(\alpha(\sigma)(V))$ :: openが存在する
    \footnote
    {
        これは$\epsilon$-$\delta$論法に似ている.
        $\sigma^{-1}(\alpha(\sigma)(V))$が開集合であるから,
        任意の点,特に$P$はこの集合の内点である.
        このことから開集合$S$が存在することは自明である.
    }.
    直ちに以下が成り立つ.
    \[ \Forall{Q \in S} \Exists{Q' \in V} \shF_Q \ni \sigma(Q)=s_{Q'}. \]
    明らかに$Q=Q'$,すなわち$\sigma|_S=\alpha(s)|_S$が成り立つ.
    点$P$を様々に取ることで,$S$で$U$を被覆できることがわかる.
    $s \& S$と$t \& T$の二組について
    \[ \alpha(s)|_{S \cap T}=\sigma|_{S \cap T}=\alpha(t)|_{S \cap T}. \]
    したがって$\alpha$ :: injから$s|_{S \cap T}=t|_{S \cap T}$.
    こうしてGluability Axiomから,
    $\alpha(s)|_S=\alpha(\int)|_S=\sigma|_S$なる$\int \in \shF(U)$の存在が示せる.
    最後にIdentity Axiomを用いて$\alpha(\int)=s$.
    これで$\alpha$ :: isoが示せた.

    \paragraph{UMP of Sheafification.}
    $\ftorSh=\ftorSec \ftorEt$とすると,これがsheafification functorとなる.
    そのUMPを見よう.
    $\shF \in \PSh(X), \shG \in \Sh(X)$とする.
    $\alpha: \id{\Sh(X)} \to \ftorSh$のnaturalityから,次の可換図式が得られる.
    \[
    \xymatrix
    {
    \shF \ar[r] \ar[d]& \ftorSh \shF \ar[d]\\
    \shG \ar[r]^-{\sim}& \ftorSh \shG
    }
    \]
    $\alpha_{\shG}: \shG \to \ftorSh \shG$ :: isoだから,
    $\shF \to \shG$から$\ftorSh \shF \to \shG$が得られた.
    次に,以下で示す可換図式(1)が与えられたとしよう.
    全体を$\ftorSh$で写し,$\ftorSh|_{\Sh(X)} \cong \id{\Sh(X)}$を用いて可換図式(2)が得られる.
    \[
    (1)
    \xymatrix
    {
        \ftorSh \shF \ar@<0.5mm>[r]^-{f} \ar@<-0.5mm>[r]_-{g}& \shG \\
        \shF \ar[u]^-{\alpha_{\shF}} \ar[ur]_-{\phi} & {}
    }
    \hspace{5em}
    (2)
    \xymatrix
    {
        \ftorSh \shF \ar@<0.5mm>[r]^-{f} \ar@<-0.5mm>[r]_-{g}& \shG \\
        \ftorSh \shF \ar@{=}[u] \ar[ur]_-{\ftorSh \phi} & {}
    }
    \]
    したがって$f=g$.
    以上でexistence \& uniquenessが示せた.

\section{Support.} %% Ex1.14 
    $\shF \in \Sh(X), U \OpenIn X, s \in \shF(U)$をとる.
    $\Supp s=\{ P \in U ~|~ s_P \neq 0 \}$としたとき,
    これがclosed in $U$であることを示そう.
    そのために$T=(\Supp s)^c=\{ P \in U ~|~ s_P=0 \}$として,
    これがopenであることを示す.

    $P \in T$を任意に取る.
    すると$s_P$の代表元として$\germ{V_P}{s}~~(P \in V_P \subset U)$が取れる.
    今$s_P=0$なので,$s|_{V_P}=0$.
    したがって$V_P \subset T$となる.
    任意の$P \in T$についてこのように$V_P$が取れるので,
    $T$はopen covering $\{V_P\}_{P \in T}$を持つ.
    よって$T=\bigcup_{P \in T} V_P$ :: open in $U$.

    $\Supp \shF$は$\{P \in X ~|~ \shF_P \neq 0 \}$と定義される.
    これはclosedとは限らない.
    実際,$\shF$の元を,なめらかな実関数に$\mathit{bump}(x)=[x>0]e^{-1/x}$
    \footnote
    {
        $[True]=1, [False]=0$とした.Iversonの記法である.
        $\mathit{bump}(x)$がなめらかであることは次のPDFを参照せよ: \url{https://andromeda.rutgers.edu/~loftin/difffal03/bump.pdf}.
    }
    をかけたものとすると,
    $\Supp \mathit{bump}(x)=[0,\infty), \Supp \shF=(0,\infty)$となる.
    後者は明らかに閉集合でない.

\section{Sheaf $\shHom$.} %% Ex1.15 
    $\shF, \shG \in \Sh(X), U \OpenIn X$とし,
    $\shF$の$U$へのrestrction(p.65)を$\shF|_U$で書く.
    $U \mapsto \Hom(\shF|_U, \shG|_U)$で定まる
    presehaf $\shHom(\shF, \shG)$がsheafであることを示そう.
    以下では$U$とその開被覆$\{U_{\lambda}\}_{\lambda \in \Lambda}$を任意にとって固定する.

    \paragraph{$\shHom(\shF, \shG)(U)$ :: Abelian Group.}
    $s,t \in \shHom(\shF, \shG)(U)$について,$s+t$を以下で定める.
    \[ (s+t)(\sigma)=s(\sigma)+t(\sigma) \mwhere V \OpenIn U,~ \sigma \in (\shF|_U)(V). \]
    単位元は$\im \shF|_U$の単位元を返す定値写像である.
    単位元を以下では0と書く.

    \paragraph{$\shHom(\shF, \shG)$ :: Presheaf.}
    $U,V$ :: openかつ$V \subseteq U$とする.
    $\overline{\res}_U^V: \shHom(\shF, \shG)(U) \to \shHom(\shF, \shG)(V)$を以下のように定める.
    \[ \{ \shF|_U \ni \sigma|_U \mapsto \tau|_U \in \shG|_U \} \mapsto \{ \shF|_V \ni \sigma|_V \mapsto \tau|_V \in \shG|_V \} \]
    これは$\res(\shF)_U^V: \shF(U) \to \shF(V)$と$\res(\shG)_U^V: \shG(U) \to \shG(V)$の自然性から誘導される.

    \paragraph{Identity Axiom.}
    $s \in \shHom(\shF, \shG)(U)=\Hom(\shF|_U, \shG|_U)$をとる.
    この$s$が任意の$\lambda$について$s|_{U_{\lambda}}=0$を満たすとする.
    さて,$V$ :: open in $U$と$\sigma \in \shF(V)$を任意に取る.
    $\{V_{\lambda}\}$を$V_{\lambda}=V \cap U_{\lambda}$で定めると,
    これは$V$の開被覆になる.
    仮定より,$s|_{V_{\lambda}}(\sigma)=s(\sigma)|_{V_{\lambda}}=0$.
    よって$s(\sigma) \in \shG(V)$にIndentity Axiomを用いることで$s(\sigma)|_V=0$が示される.
    $V, \sigma$は任意なので,結局以下が得られた.
    \[ \Forall{V \OpenIn U} \Forall{\sigma \in \shF(V)} s(\sigma)=0. \]
    すなわち,$s$は定値写像$0$である.
    以上でIndentity Axiomの成立が示された.

    \paragraph{Gluability Axiom.}
    sections $s_{\lambda} \in \shHom(\shF, \shG)(U_{\lambda})$をとる.
    これが任意の$\lambda, \mu \in \Lambda$について
    $s_{\lambda}|_{U_{\lambda} \cap U_{\mu}}=s_{\mu}|_{U_{\lambda} \cap U_{\mu}}$を満たすとしよう.
    この仮定は以下のように書ける.
    \[
        \Forall{\lambda, \mu \in \Lambda} \Forall{\sigma \in \shF(U_{\lambda} \cap U_{\mu})}
        s_{\lambda}(\sigma)=s_{\mu}(\sigma).
    \]
    そこで$\lambda$をひとつ取って固定し,$\sigma \in \shF(U_{\lambda})$とする.
    さらに$\{ V_{\mu} \}_{\mu \in \Lambda}$を$V_{\mu}=U_{\lambda} \cap U_{\mu}$で定める.
    この$\{ V_{\mu} \}$は$U_{\lambda}$の開被覆である.
    すると最初の仮定と$V_{\mu} \cap V_{\nu}=U_{\lambda} \cap (U_{\mu} \cap U_{\nu}) \subseteq U_{\mu} \cap U_{\nu}$から以下が成り立つ.
    \[
        \Forall{\mu, \nu \in \Lambda}
        s_{\mu}(\sigma)|_{V_{\mu} \cap V_{\nu}}=s_{\nu}(\sigma)|_{V_{\mu} \cap V_{\nu}}.
    \]
    sections $s_{\mu}(\sigma) \in \shG(U_{\lambda})$に対してGluability Axiomを用いて,
    $s(\sigma)|_{V_{\mu}}=s_{\mu}(\sigma)|_{V_{\mu}}$なる$s(\sigma)$の存在が言える.
    Indentity Axiomから$s(\sigma)|_{U_{\lambda}}=s_{\mu}(\sigma)|_{U_{\lambda}}$.
    こうして,以下を満たす$s \in \shHom(\shF, \shG)(U)$の像が
    各点$\sigma \in \shF(U_{\lambda})$ごとに定義できる.
    \[
        \Forall{\lambda \in \Lambda} \Forall{\sigma \in \shF(U_{\lambda})}
        s(\sigma)|_{U_{\lambda}}=s_{\lambda}(\sigma)|_{U_{\lambda}}.
    \]
    簡潔にかけば,$s|_{U_{\lambda}}=s_{\lambda}|_{U_{\lambda}}$.
    よってGluability Axiomの成立が示せた.

\section{Flasque Sheaves.} %% Ex1.16 
    $U,V$ :: open in $X$,$V \subseteq U$とする.
    restriction map $\res_U^V$がsurjectiveであるような$\shF \in \Sh(X)$を
    flasque\footnote{フランス語.フラスコのこと.軟弱という意味.発音は \url{https://ja.forvo.com/word/flasque/}.} sheafと呼ぶ.

    \subsection{Constant Sheaf on Irreducible Top.Sp is Flasque.}
    $X$ :: irreducible,$A$ :: abelian group, $U,V$ :: open in $X$,$V \subseteq U$とする.
    $\shA$を$X$から$A$へのconstant sheafとすると,
    定義より$\shA(V)=\{ s: V \to A ~|~ s \text{ :: continuous. }\}$.
    そこで$s \in \shA(V)$を一つとって固定する.
    $s$ :: continuousという条件は次と同値
    \[ \Forall{a \subseteq A} s^{-1}(a) \OpenIn V. \]
    $X$ :: irreducibleであるとき,$s \in \shF(V)$がどのようなものか考えよう.
    
    \paragraph{Case: $\#A=1$.}
    まず$\#A=1$,すなわち$A$が自明なabelian group $\{e\}$であったとする.
    この時,明らかに$\shF(V)$は定値写像$x \mapsto e$のみからなる.
    $\shF(U)$も同じ定値写像からなるので,この時constant sheafはflasque.

    \paragraph{Case $\#A>1$.}
    $a \neq b$が成り立つような$a,b \in A$を任意に取る.
    すると以下が成り立つ.
    \[ s^{-1}(\{a\}) \cap s^{-1}(\{b\})=s^{-1}(\{a\} \cap \{b\})=s^{-1}(\emptyset)=\emptyset. \]
    したがって$X$ :: irreducibleから$s^{-1}(\{a\}) \mor s^{-1}(\{b\})=\emptyset$.
    仮に任意の$a \in A$について$s^{-1}(\{a\})=\emptyset$であったとすると$s$が写像にならない.
    したがって$s^{-1}(\{a_s\}) \neq \emptyset$となる$a_s \in A$がただひとつ存在する.
    $s$は写像なので$s^{-1}(A)=V$が成り立ち,
    したがって$s$はこのような$a_s$への定値写像である事が分かる.
    すると容易に$s$は$U$へ拡張できるので,この時もconstant sheafはflasque.

    \subsection{If $0 \to \shF' \to \shF \to \shF'' \to 0$ is Exact and $\shF'$ is Flasque, then... }
    $0 \to \shF' \to \shF \to \shF'' \to 0$がexactかつ$\shF'$がflasqueであったとする.
    この時,任意のopen set $U$について$0 \to \shF'(U) \to \shF(U) \to \shF''(U) \to 0$はexactであることを示す.

    写像に$0 \to \shF' \xrightarrow{f} \shF \xrightarrow{g} \shF'' \to 0$と名前をつけ,
    $U$ :: open in $X$と$s'' \in \shF''(U)$をとる.
    Ex1.8より,$0 \to \shF'(U) \xrightarrow{f_U} \shF(U) \xrightarrow{g_U} \shF''(U)$はexact.
    なのであとは$g_U$がsurjectiveであることを示せば良い.
    元のexact sequenceから$g$ :: surjが言える.
    Ex1.3より,以下が成り立つ.
    \[
        (*)~~~
        \bigcup {}^\exists U_i=U,~~
        \Exists{t_i \in \shF(U_i)}
        \Forall{i} g(t_i)=s''|_{U_i}.
    \]
    任意に$i,j$をとり,以下の可換図式でdiagram chaseをする.
    ただし$U=U_i \cap U_j$とした.
    % {{{ diagram
    \[
    \xymatrix@C=-5pt
    {
        {} & ~~~0~~~ \ar[rrr] & {} & {} & \shF'(U)~ \ar@{->>}[rd] \ar@{->>}[ldd] \ar@{>->}[rrr]^-{f} & {} & {} & \shF(U) \ar[rd] \ar[ldd] \ar[rrr]^-{g} & {} & {} & \shF''(U) \ar[rd] \ar[ldd] & {} \\
        {} & {} & ~~~0~~~ \ar[rrr]|(0.47)\hole & {} & {} & \shF'(U_j)~ \ar@{->>}[ldd]|(0.50)\hole \ar@{>->}[rrr]|(0.51)\hole^(0.3){f} & {} & {} & \shF(U_j) \ar[ldd]|(0.50)\hole \ar[rrr]|(0.48)\hole^(0.3){g} & {} & {} & \shF''(U_j) \ar[ldd]& {} \\
        ~~~0~~~ \ar[rrr] & {} & {} & \shF'(U_i)~ \ar@{->>}[rd]\ar@{>->}[rrr]^(0.3){f} & {} & {} & \shF(U_i) \ar[rd]\ar[rrr]^(0.3){g} & {} & {} & \shF''(U_i) \ar[rd]& {} \\
        {} & ~~~0~~~ \ar[rrr] & {} & {} & \shF'(U_{ij})~ \ar@{>->}[rrr]^-{f} & {} & {} & \shF(U_{ij}) \ar[rrr]^-{g} & {} & {} & \shF''(U_{ij})& {} \\
    }
    \]
    % }}}
    $s'' \in \shF''(U)$と,
    $(*)$から存在が示される$t_i \in \shF(U_i), t_j \in \shF(U_j)$から
    diagram chasingを始める.
    \begin{enumerate}[(1)]
    \setlength{\itemindent}{2em}
        \item naturalityから$g_{U_{ij}}(t_i|_{U_{ij}})=s''|_{U_{ij}}=g(t_j|_{U_{ij}})$.
        \item よって列の完全性から$t_i-t_j \in \ker g_{U_{ij}}=\im f_{U_{ij}}$.
        \item したがって$f_{U_{ij}}(u'_{ij})=t_i-t_j$なる$u'_{ij} \in \shF'(U_{ij})$が存在する.
        \item $\res_U^{U_{ij}}$ :: surjから$s'_{ij}|_{U_{ij}}=u'_{ij}$なる$s'_{ij} \in \shF'(U)$が存在する.
        \item $s_{ij}=f_U(s'_{ij})|_{U_i}+t_j \in \shF(U_i)$とおく.(足すのは$t_j$であることに注意.)
%        \item 構成より$s_{ij_1}|_{U_{ij_1} \cap U_{ij_2}}=s_{ij_2}|_{U_{ij_1} \cap U_{ij_2}}$(TODO: 示す.).
%        \item Gluability Axiomより,$s_i|_{U_{ij}}=s_{ij}$なる$s_i \in \shF(U_i)$の存在が分かる.
%        \item $i$を動かして再びGluability Axiomを用いると,$s|_{U_i}=s_i$となる$s \in \shF(U)$が存在する.
%        \item 構成より,$g_{U}(s)|_{U_{ij}}=g_{U_{ij}}(s_{ij})=g_{U_{ij}}(t_i|_{U_{ij}})=s''|_{U_{ij}}$.
%        \item Identity Axiomから$g_U(s)=s''$.
    \end{enumerate}
    以上から,$g_U(s)=s''$なる$s \in \shF(U)$の存在が示せた.

    \subsubsection{Another Proof}
    次のPDFのLemma2.12(p.10)がこの演習問題と同じ命題である: 
    \url{http://www.math.mcgill.ca/goren/SeminarOnCohomology/Sheaf_Cohomology.pdf}.
    次のPDFのLemma0.3(p.12)も同じ:
    \url{http://www.uio.no/studier/emner/matnat/math/MAT4215/v15/notes1.pdf}.
    どちらの証明でもZorn's Lemmaが用いられている.

    \subsection{If $0 \to \shF' \to \shF \to \shF'' \to 0$ is Exact and $\shF', \shF$ is Flasque, then $\shF''$ also.}
    $U,V$ :: open in $X$, $V \subseteq U$とする.
    (b)より,以下の完全列が得られる.
    \[
    \xymatrix
    {
    0 \ar[r]& \shF'(U)~ \ar@{>->}[r]^-{f} \ar@{->>}[d]& \shF(U) \ar@{->>}[r]^-{g} \ar@{->>}[d]& \shF''(U) \ar[r] \ar[d]& 0 \\
    0 \ar[r]& \shF'(V)~ \ar@{>->}[r]^-{f}& \shF(V) \ar@{->>}[r]^-{g}& \shF''(V) \ar[r]& 0
    }
    \]
    証明はdiagram chasingによる.
    \begin{enumerate}[(1)]
    \setlength{\itemindent}{2em}
        \item   $s'' \in \shF''(V)$を任意に取る.
        \item   $\shF(U) \to \shF(V) \to \shF''(V)$ :: surjから,
                $g(\tilde{s})|_V=s''$なる$\tilde{s} \in \shF(U)$が取れる.
        \item   naturalityから$g(\tilde{s}|_V)=s''=g(\tilde{s})|_V$.
    \end{enumerate}
    $s:=g(\tilde{s}) \in \shF''(U)$とおけば$s|_V=s''$.

    \subsection{If $f: X \to Y$ is Conti. and $\shF$ is Flasque, then $f_* \shF$ is Flasque.}
    $U,V$ :: open in $Y$,$V \subseteq U$とする.
    このとき$f^{-1}(V) \subseteq f^{-1}(U)$.
    なので$\shF(U) \to \shF(V)$ :: surjより$\shF(f^{-1}(U)) \to \shF(f^{-1}(V))$ :: surj.
    $f_* \shF(U)=\shF(f^{-1}(U))$だから,$f_* \shF$ :: flasque.

    \subsection{Discontinuous Sections.}
    $\shF \in \Sh(X)$とする.
    これに対し,discontinuous sections of $\shF$と呼ばれるsheaf $\shG$が以下のように作れる.
    $\pi$はEx1.13の$s_P \mapsto P$なる写像である.
    \[ \shG: U \mapsto \left\{ s:U \to \bigsqcup_{P \in U} \shF_P ~\middle|~ \pi \circ s=\id{U} \right\} \]
    $\shG$がflasque sheafであることと,$\shF \to \shG$の自然な単射が存在することを示す.
    
    \paragraph{$\shG$ :: sheaf.}
    $\shG$ :: presheafは明らか.
    sheafであることを示すため,
    $U$ :: open in $X$とそのopen cover $\{U_i\}_{i \in I}$をとり,固定する.
    任意の$i \in I$について$s|_{U_i}=0$であるような$s \in \shG(U)$が存在したとする.
    $\bigcup U_i=U$より,任意の点$P \in U$に対して$s(P)=0$.
    これはIdentity Axiomの成立を意味する.
    同様に``$\Forall{i,j} \Forall{P \in U_i \cap U_j}$''を``$\Forall{P \in U}$''に書き換えるだけで,
    Gluability Axiomの成立が証明できる.

    \paragraph{$\shG$ :: flasque.}
    $V \subset U$とする.
    $s \in \shG(V)$をとる.これは例えば以下のように拡張できる.
    \[
        \bar{s}(P)=
        \begin{cases}{}
            s(P) & (P \in V) \\
            0 & (P \in U \setminus V)
        \end{cases}
    \]

    \paragraph{$\alpha$ in Ex1.13 is injective.}
    Ex1.13の$\alpha: s \mapsto [P \mapsto s_P]$がinjectiveであることは以下のように示される.
    ある$s,t \in \shF(U)$について$\alpha(s)=\alpha(t)$が成立するとしよう.
    すると十分小さいopen set $(P \in) V_P (\subset U)$が存在して,
    $s|_{V_P}=t|_{V_P}$となる.
    明らかに$\{V_P\}_{P \in U}$は$U$のopen coverなので,
    $s-t \in \shG$にIdentity Axiomを用いて$s=t$が得られる.

\section{Skyscraper Sheaves.} %% Ex1.17 
    $X$ :: topological space, $P \in X$, $A$ :: abelian groupとする.
    sheaf $i_P(A)$を以下で定める.
    \[
        i_P(A)(U)=
        \begin{cases}{}
            A & (P \in U) \\
            0 & (\text{otherwise})
        \end{cases}
    \]
    点$P$を含む最小の閉集合を$\{P\}^-$と書く.

    \subsection{$(i_P(A))_Q=A$ is $A$ if $Q \in \{P\}^-$, otherwise $0$ .}
    $U$を$Q$を含む極小の開集合とした時,
    $(i_P(A))_Q$は集合として$\shF(U)$と一致する.
    したがって以下が成立する.
    \begin{align*}
        {}& (i_P(A))_Q=A \\
        \iff& \Forall{U \subset X} Q \in U \implies P \in U \\
        \iff& \Forall{U \subset X} P \in U^c \implies Q \in U^c.
    \end{align*}
    最後の行は対偶として得られた.
    一方,点$P$を含む最小の閉集合$\{P\}^-$は以下を満たす唯一の集合として特徴づけられる.
    \[ \Forall{U \subset X} P \in U^c \implies \{P\}^- \subseteq U^c \]
    よって$(i_P(A))_Q=A \iff Q \in \{P\}^-$.
    他の場合は明らかに$(i_P(A))_Q=0$となる.
    また,この特徴付けの対偶から$U \cap \{P\}^- \neq \emptyset$ならば$P \in U$.
    $P \in U$ならば$P \in U \cap \{P\}^-$なので逆も成立する.

    \subsection{$i_P(A)$ can be described as direct image.}
    abelian group $A$に伴う$\{P\}^-$上のconstant sheafを$\shA$とする.
    すると$i_P(\shA)$は埋め込み写像$i: \{P\}^- \hookrightarrow X$のdirect image $i_*(\shA)$に等しい.
    実際,開集合$U$について$i_*(\shA)(U)=\shA(i^{-1}(U))=\shA(U \cap \{P\}^-)$であるから以下のようになる.
    \[
        i_*(\shA)(U) \cong
        \begin{cases}{}
            A & (U \cap \{P\}^- \neq \emptyset) \\
            0 & (\text{otherwise})
        \end{cases}.
    \]
    (a)で見たとおり$U \cap \{P\}^- \neq \emptyset$と$P \in U$は同値.
    よって$i_*(\shA)=i_P(A)$.
    特に,$\{P\}^-$はその最小性からirreducibleなので,
    Ex1.16a,dと合わせれば$i_P(A)$はflasqueであることが分かる.

\section{Adjoint Property of $f^{-1}$.} %% Ex1.18 
    $f: X \to Y$ :: continuous mapについて,
    $f^{-1}$が$f_*$のleft adjoint functorであることを示す.
    そのために,当分の間は$f^{-1}=\ftorSh \varinjlim_{V \supseteq f(-)} \ftorFgt$でなく,
    $f^{pre}\shF=\varinjlim_{V \supseteq f(-)} \shF(-)$がleft adjointであることを示す.
    left adjointの定義としては$\Hom$についての定義を用いる.

    最初にunit $\eta: \id{\Sh(Y)} \to f_* f^{pre}$と
    counit $\epsilon: f^{pre} f_* \to \id{\Sh(X)}$を構成する.

    \paragraph{Construction of Unit $\eta$.}
    $\shG \in \Sh(Y)$をとると,$U$ :: open in $Y$について次の等式が成り立つ.
    \[
        (f_* f^{pre} \shG)^{pre}(U)
        =(f^{pre} \shG)^{pre}(f^{pre}(U))
        =\lim_{\substack{\longrightarrow \\ V \supseteq f \circ f^{pre}(U)}} \shG(V).
    \]
    $U \supseteq f \circ f^{pre}(U)$(全射と等号成立は同値)だから,
    coconeの「母線」として
    \[ (\eta_{\shG})_U: \shG(U) \to (f_* f^{pre} \shG)^{pre}(U) \]
    が得られる.
    $\eta$の自然性は容易に示される.
    ($f_* f^{pre}$は母線と可換になるように射を写す.)

    \paragraph{Construction of Counit $\epsilon$.}
    $\shF \in \Sh(X)$をとると,$U$ :: open in $X$について次の等式が成り立つ.
    \[ (f^{pre} f_* \shF)(U)=\lim_{\substack{\longrightarrow \\ V \supseteq f(U)}} \shF(f^{pre}(V)). \]
    $V \supseteq f(U)$であるとき,$f^{pre}(V) \supseteq f^{pre} \circ f(U) \supseteq U$(単射と等号成立は同値).
    したがってcolimitのUMPにより$(\epsilon_{\shF})_U$が得られる.
    \[
    \xymatrix
    {
    (f^{pre} f_* \shF)(U) \ar@{-->}[r]^-{(\epsilon_{\shF})_U}& \shF(U) \\
    \dots \ar[r]_-{\res} \ar[u] \ar@/_10pt/[ur]& \shF(f^{pre}(V)) \ar[u]_-{\res} \ar@/^10pt/[ul]
    }
    \]
    $\epsilon$の自然性はあまり自明ではないのでここで示そう.
    \[
    \begin{xy}
        (20,0) *{\shF(V)}="F1", (50,0)*{\shF'(V)}="F'1",
        (20,20)*{\shF(U)}="F2", (50,20)*{\shF'(U)}="F'2",
        (0,30)*{(f^{pre} f_* \shF)(U)}="F3", (70,30)*{(f^{pre} f_* \shF')(U)}="F'3"
        \ar "F1";"F2" \ar "F1";"F'1" \ar "F'1";"F'2" \ar "F2";"F'2"
        \ar "F1";"F3" \ar"F'1";"F'3"
        \ar "F3";"F'3"
        \ar@{-->}_-{\epsilon} "F3";"F2" \ar@{-->}^-{\epsilon} "F'3";"F'2"
        \ar@{~>} "F3";"F'2" \ar "F1";"F'2"
    \end{xy}
    \]
    最初,この図式を書いた時には,上面の台形が可換になっていることは非自明である.
    direct limitのUMPから$\shF(V), (f^{pre} f_* \shF)(U), \shF'(U)$の三角形が
    可換になるような唯一の射$(f^{pre} f_* \shF)(U) \to \shF'(U)$(波線のもの)が存在する.
    しかし同じ三角形を可換にするような射として,
    $\epsilon$を通る射$(f^{pre} f_* \shF)(U) \to \shF(U) \to \shF'(U)$が既に存在する.
    UMPから,この射は波線の射に等しい.
    また,同様に$(f^{pre} f_* \shF)(U) \to \shF(U) \to \shF'(U)$も波線の射と等しい.
    まとめると,上面の台形が可換であることが判る.
    これは$\epsilon$の自然性を意味する.

    \paragraph{Preparation of Unit-Counit Equations.}
    まず$f \circ f^{-1}(B) \subseteq B$に$B=f(A)$を代入すると$f \circ f^{-1} \circ f(A) \subseteq f(A)$.
    続いて$f^{-1} \circ f(A) \supseteq A$の両辺を$f$で写すと$f \circ f^{-1} \circ f(A) \supseteq f(A)$.
    二つを合わせて$f \circ f^{-1} \circ f=f$が得られる.
    双対的に$f^{-1} \circ f \circ f^{-1}=f^{-1}$が得られる.

    \paragraph{Unit-Counit Equations.}
    まず計算.
    \begin{align*}
        {}& (f_* f^{pre} f_* \shF)(U) \\
        =&  (f_* f^{pre} \shF)(f^{-1}(U)) \\
        =&  \varinjlim_{V \supseteq f \circ f^{-1}(U)} (f_* \shF) (V) \\
        =&  \varinjlim_{V \supseteq f \circ f^{-1}(U)} \shF (f^{-1}(V))
    \end{align*}
    $V \supseteq f \circ f^{-1}(U)$であるとき,既に見たように$f^{-1}(V) \supseteq f^{-1}(U)$.
    以下の可換図式を見よ.
    \[
    \xymatrix
    {
    (f_* f^{pre} f_* \shF)(U) \ar[r]^-{f_* (\epsilon_{\shF})} & (f_* \shF)(U) \\
    \dots \ar[u] \ar[ur] \ar[r]
    & \shF(f^{-1}(U)) \ar[ul]^(0.7){(\eta)_{f_* \shF}} \ar@{=}[u]
    }
    \]
    よって$\id{f_*}=f_* \epsilon \circ \eta_{f_*}$が得られた.
    再び計算する.
    \begin{align*}
        {}& (f^{pre} f_* f^{pre} \shG)(U) \\
        =&  f^{pre} f_* \varinjlim_{V \supseteq f(U)} \shG (V) \\
        =&  f^{pre} \varinjlim_{V \supseteq f(U)} \shG (f^{-1}(V)) \\
        =&  \varinjlim_{W \supseteq f \circ f^{-1}(V)} ~ \varinjlim_{V \supseteq f(U)} \shG(W)
    \end{align*}
    $V \supseteq f(U)$であるとき$W \supseteq f \circ f^{-1}(V) \supseteq f(U)$.
    $V \supseteq f(U)$を満たす$V$全体は,
    明らかに$W \supseteq f \circ f^{-1}(V) \supseteq f(U)$を満たす$W$全体を包含する
    \footnote
    {
        $W$の方がより厳しい条件を満たさなくてはならない.
        $f \circ f^{-1}(V)$は開集合にならないかもしれないが,$W$は開集合である.
    }
    .
    したがって以下の図式が得られる.
    (点線の射は存在するとは限らない.)
    \[
    \xymatrix@C=5em@R=3em
    {
    (f^{pre} \shG)(U) \ar[r]^-{(f^{pre}\eta_{\shG})_U}
    & (f^{pre} f_* f^{-1} \shG)(U) \ar[r]^-{(\epsilon_{f^{pre}\shG})_U}
    & (f^{pre} \shG)(U) \\
    \dots \ar[u] \ar[ur] \ar[r] \ar[urr]
    & \shG(W) \ar[ul] \ar[u] \ar[r] \ar[ur]
    & \shG(V) \ar[ull] \ar[u] \ar@{..>}[ul]
    }
    \]
    $(f^{pre} \eta_{\shG})_U$は$\shG(U) \xrightarrow{\eta} (f_* f^{pre} \shG)(U)$を$f^{pre}$で写せば直ちに得られ,
    $(\epsilon)_{f^{pre}\shG(U)}$は$(f^{pre} f_* f^{pre} \shG)(U)$のUMPから得られる.
    最後に,$(f^{pre} \shG)(U)$のUMPから,
    $\id{f^{pre}}=\epsilon_{f^{pre}\shG} \circ f^{pre} \eta_{\shG}$.

    \paragraph{Hom-set Definition.}
    以下のように写像を定義する.
    \[
        \phi(-)=f_*(-) \circ \eta_{\shG},~~
        \psi(-)=\epsilon_{\shF} \circ f^{pre}(-).
    \]
    するとunit-counit equationsからこれらが互いに逆写像であることが分かる.
    こうして所期の同型$\phi: \Hom(f^{pre}\shF, \shG) \bimap \Hom(\shF, f_* \shG)$が得られる.
    自然性は$\eta, \epsilon$の自然性から誘導される.

    \paragraph{Adjointness of $f^{-1}$.}
    $f^{-1}=\ftorSh f^{pre}$と$\ftorSh \dashv \ftorFgt, f^{pre} \dashv f_*$を用いる.
    \begin{align*}
        {}& \Hom_{\Sh}(f^{-1} \shF, \shG) \\
        =&  \Hom_{\Sh}(\ftorSh f^{pre} \shF, \shG) \\
        =&  \Hom_{\PSh}(f^{pre} \shF, \ftorFgt \shG) \\
        =&  \Hom_{\PSh}(\shF, f_* \ftorFgt \shG) \\
        =&  \Hom_{\Sh}(\shF, f_* \shG)
    \end{align*}
    最後に,$\shF, \shG$が予めsheafであること,
    及び$\ftorFgt$がobjectを変化させない埋め込み関手であることを用いた.

\section{Extending a Sheaf by Zero.} %% Ex1.19 
    $X$ :: topological space, $Z$ :: closed subset in $X$,$U=X \setminus Z$とする.
    さらに$i: Z \hookrightarrow X, j: U \hookrightarrow X$を埋め込み写像とする.

    \subsection{$i_* \shF$ : Extending $\shF \in \Sh(Z)$ by Zero Outside $Z$.}
    $\shF \in \Sh(Z)$とする.
    $i$は埋め込み写像なので,開集合$U$について$(i_* \shF)(U)=\shF(U \cap Z)$.

    点$P$の開近傍を考える.

    \paragraph{Case: $P \in Z^c$.}
    $Z^c$は開集合だから,$P \in Z^c$ならば,開近傍$V$が存在して$P \in V \subseteq Z^c$となる.
    このとき,$\shF(Z \cap V)=\shF(\emptyset)=0$となる.
    しかも$\shF(Z \cap V)=0$は十分小さいすべての$U$について成り立つ.
    したがって$P$の任意の開近傍$V$について次の図式が可換.
    \[
    \xymatrix
    {
        {} & (i_* \shF)_P & {} \\
        \shF(Z \cap V) \ar[rr]_-{i_* \res_{V}^{\emptyset}} \ar[ur]& {} & 0 \ar[ul]\\
    }
    \]
    よって$\shF(Z \cap V) \to (i_* \shF)_P$はゼロ写像しかなく,
    $(i_* \shF)_P$のUMPから$(i_* \shF)_P=0$.

    \paragraph{Case: $P \in Z$.}
    逆に$P \in Z$ならば,点$P$の$X$における開近傍$U$から作られる$Z \cap U$は,
    常に空でない$P$の開近傍.
    いつでも埋め込み射$\shF(V) \to \shF(Z \cap V)$が存在するので,
    結局$\shF(V) ~~(P \in V)$なるabelian group全てから$(i_* \shF)_P$に射がのびている.
    よって$(i_* \shF)_P=\shF_P$.

    \paragraph{Conclusion.}
    まとめると,以下が成り立つ.
    \[
        (i_* \shF)_P=
        \begin{cases}{}
            \shF_P & (P \in Z) \\
            0 & (P \not \in Z)
        \end{cases}
    \]

    \subsection{$j_! \shF$ : Extending $\shF \in \Sh(U)$ by Zero Outside $U$}
    $\shF \in \Sh(U)$とし,
    $j_! \shF$を以下で定まるpresheafのsheafificationとする.
    \[
        (j_! \shF)^{pre}(V)=
        \begin{cases}{}
            \shF(V) & (V \subseteq U) \\
            0 & (\text{otherwise})
        \end{cases}
    \]
    sheafificationでstalkは変わらないから,$(j_! \shF)_P=(j_! \shF)^{pre}_P$.

    点$P$の開近傍を考えよう.

    \paragraph{Case: $P \in U$.}
    $U$ :: openなので,ある$V$ :: openが存在して$P \in V \subseteq U$となる.
    このような$V$について$(j_! \shF)^{pre}(V)=\shF(V)$.
    $U$より小さい任意の開近傍$V$については$(j_! \shF)^{pre}(V)=\shF(V)$となる上,
    $U$より大きい任意の開近傍$V$から射$\res_V^{U \cap V}: \shF(V) \to \shF(U)$が生えている.
    よって$(j_! \shF)^{pre}_P=\shF_P$.

    \paragraph{Case: $P \in U^c$.}
    このとき,どのように$P$の開近傍$V$をとっても,
    $P \in V$かつ$P \not \in U$なので$V \not \subseteq U$.
    したがって$(j_! \shF)^{pre}_P=0$となる.

    \paragraph{Conclusion.}
    まとめると,以下が成り立つ.
    \[
        (j_! \shF)_P=
        \begin{cases}{}
            \shF_P & (P \in U) \\
            0 & (P \not \in U)
        \end{cases}
    \]

    \subsection{$0 \to j_!(\shF|_U) \to \shF \to i_*(\shF|_Z) \to 0$ is Exact.}
    Ex1.2cを応用する.
    $P \in X$を任意の点とする.
    $P \in U \text{ exor } Z$なので,それぞれの場合について考える.

    \paragraph{Case: $P \in Z$.}
    この時,$(j_!(\shF|_U))_P=\shF_P, (i_*(\shF|_Z))_P=0$となる.
    よって$0 \to (j_!(\shF|_U))_P \to \shF_P \to (i_*(\shF|_Z))_P \to 0$は
    $0 \to \shF_P \to \shF_P \to 0 \to 0$に等しく,これは完全列.

    \paragraph{Case: $P \in U$.}
    この時,$(j_!(\shF|_U))_P=0, (i_*(\shF|_Z))_P=\shF_P$となる.
    よって$0 \to (j_!(\shF|_U))_P \to \shF_P \to (i_*(\shF|_Z))_P \to 0$は
    $0 \to 0 \to \shF_P \to \shF_P \to 0$に等しく,これは完全列.

\section{Subsheaf with Supports.} %% Ex1.20 
    $Z$ :: closed in $X$, $\shF \in \Sh(X)$とする.
    $\Gamma_Z(X, \shF)$を以下で定める.
    \[ \Gamma_Z(X, \shF)=\{ s \in \Gamma(X, \shF)=\shF(X) ~|~ \Supp(s) \subseteq Z. \}. \]
    ``$\Supp(s) \subseteq Z$''は``$\Forall{P \in Z^c} s(P)=0$''と同値である.
    また,特に$\Supp(0)=\emptyset$より,$0 \in \Gamma_Z(X, \shF)$.

    \subsection{Presheaf $V \mapsto \Gamma_{V \cap Z}(V, \shF|_V)$ is a Sheaf.}
    Presheaf $\mathscr{H}_Z^0(\shF)$を
    \[ \mathscr{H}_Z^0(\shF): V \mapsto \Gamma_{V \cap Z}(V, \shF|_V) \]
    で定める.
    これがsheafであることを示そう.
    開集合$V$とその開被覆$\{V_i\}_{i \in I}$を任意にとる.

    \paragraph{Identity Axiom.}
    $s \in \mathscr{H}_Z^0(\shF)(V)$をとる.
    任意の$i \in I$について$s|_{V_i}=0$が成り立つとしよう.
    この時,$\mathscr{H}_Z^0(\shF)$の定義から,$s \in \shF(V)$と$\Supp (s|_V) \subseteq V \cap Z$が成り立つ.
    $\shF$のindentity axiomをもちいて,$s|_V=0$が得られる.
%    なお,$\Supp (s|_{V_i})=\emptyset$から$\Supp (s|_V)=\emptyset$が得られる.

    \paragraph{Gluability Axiom.}
    $s_i \in \mathscr{H}_Z^0(\shF)(V_i)$をとる.
    任意の$i,j \in I$について$s_i|_{V_i \cap V_j}=s_j|_{V_i \cap V_j}$が成り立つとしよう.
    するとやはり$s_i \in \shF(V_i)$なので,$\shF$のgluability axiomから,
    $s|_{V_i}=s_i$なる$s \in \shF(V)$が存在する.
    あとは$s \in \mathscr{H}_Z^0(\shF)(V)$,すなわち$\Supp(s) \subseteq V \cap Z$を示せば良い.
    これは
    \[ \Supp(s_i)=\Supp(s|_{V_i})=\Supp(s) \cap V_i \subseteq V_i \cap Z\]
    から$\Supp(s)=\bigcup (\Supp(s) \cap V_i) \subseteq \bigcup (V_i \cap Z)=V \cap Z$と計算できる.

    \subsection{For $U=X \setminus Z$,$0 \to \mathscr{H}_Z^0(\shF) \to \shF \to j_*(\shF|_U)$ is Exact.}
    開集合$U=X \setminus Z$と$j: U \hookrightarrow X$について,
    $0 \to \mathscr{H}_Z^0(\shF) \to \shF \to j_*(\shF|_U)$がexactであることを示す.
    さらに,$\shF$ :: flasqueならば$0 \to \mathscr{H}_Z^0(\shF) \to \shF \to j_*(\shF|_U) \to 0$もexactであることを示す.

    定義より,$\mathscr{H}_Z^0(\shF), j_*(\shF|_U)$は以下のような集合である.
    \[
        \mathscr{H}_Z^0(\shF)=\{ s \in \shF(V) ~|~ \Forall{Q \in U \cap V} s(Q)=0. \},~~
        j_*(\shF|_U)=\shF(U \cap V).
    \]
    そこで写像$\zeta: \shF \to j_*(\shF|_U)$を以下で定義する.
    \[ \zeta(s)(Q)=[Q \in U \cap V]s(Q) \mwhere V \OpenIn X, s \in \shF(V), Q \in V. \]
    ただし$[Q \in U \cap V]$はIversonの記法である.
    (ここは指示関数を用いて$\chi_{U \cap V}(Q)$と書いても良い.)
    すると既に確認した$\mathscr{H}_Z^0(\shF)$の定義から,$\ker \zeta=\mathscr{H}_Z^0(\shF)$.
    よって$0 \to \mathscr{H}_Z^0(\shF) \hookrightarrow \shF \xrightarrow{\zeta} j_*(\shF|_U)$はexact.
    
    さらに$\shF$ :: flasqueだと仮定する.
    すると,$s \in \shF(U \cap V)$に対して$s'|_{U \cap V}=s$なる$s' \in \shF(V)$が存在する.
    明らかに$\zeta(s')=s$となるから,この時$\zeta$は全射.
    したがって$0 \to \mathscr{H}_Z^0(\shF) \to \shF \to j_*(\shF|_U) \to 0$もexactになる.

\section{Some Examples of Sheaves on Varieties.} %% Ex1.21 
    $k$ :: algebraically closed field, $X$ :: variety over $k$とする.
    $\mathcal{O}_X$をthe sheaf of regular functons on $X$ (Example 1.0.1)とする.
    \subsection{The Sheaf of Ideals $\shI_Y$.}
    $Y$ :: closed in $X$とする.
    任意の$U$ :: open in $X$について,$\shI_Y(U)$を以下で定める.
    \[ \shI_Y: U \mapsto \{ f \in \mathcal{O}_X(U) ~|~ \Forall{P \in Y \cap U} f(P)=0. \}. \]
    $\shI_Y(U)$は$\mathcal{O}_X(U)$のイデアルである.
    この時,$\shI_Y (\subseteq \mathcal{O}_X)$がsheafであることを示す.

    \subsection{If $Y$ :: subvariety, then $\mathcal{O}_X \cong i_*(\mathcal{O}_Y)$.}

    \subsection{ }
    \subsection{ }
    \subsection{ }

\section{Glueing Sheaves.} %% Ex1.22 
    $X$ :: topological space, 
    $\mathfrak{U}=\{U_i\}_{i \in I}$ :: open cover of $X$,
    $\shF_i \in \Sh(U_i)$とする.
    この$\{\shF_i\}_{i \in I}$に付随して,
    同型写像$\phi_{ij}: \shF_i|_{U_i \cap U_j} \isomap \shF_j|_{U_i \cap U_j}$が存在し,
    $\{\shF_i\}_{i \in I}$ with $\{\phi_{ij}\}_{i,j \in I}$が
    inverse systemをなすとする.
    この時,inverse limit $\shF$の存在を示す.
    さらに,$\shF|_{U_i} \iso \shF_i$となることを示す.
    この命題はsectionでなくsheafのGluablity Axiomと言える.
 
    Prop1.1を用いて仮定を書き換える.
    $\{\shF_i\}_{i \in I}$について,以下の同型がある.
    \[ \Forall{i,j \in I} \Forall{P \in U_i \cap U_j} (\shF_i)_P \cong (\shF_j)_P. \]
    この時,sheaf $\shF$が存在して,
    \[ \Forall{i \in I} \Forall{P \in U_i} \shF_P=(\shF|_{U_i})_P \cong (\shF_i)_P \]
    となることを示す.
    Ex1.19bの結果が結論によく似ているので,これを参考にする.

    $\shF$を以下のpresheafのsheafificationと定義する.
    \[
        \shF^{pre}(V)=
        \begin{cases}{}
            \shF_i(V) & (\Exists{i \in I} V \subseteq U_i) \\
            0 & (\text{otherwise})
        \end{cases}
    \]
    もし$V \subseteq U_i$なる$i$が複数存在した時には,どれを選んでも構わない.
    その時$V \subset U_i \cap U_j$なる$i,j \in I$が存在し,
    $i,j$どちらを選んでも$\shF^{pre}(V)$が$\phi_{ij}$を介して同型になるからである.
    そしてEx1.19bの証明から分かるように,$(\shF|_{U_i})_P=(\emb^{U_i}_! \shF_i)_P=(\shF_i)_P$.
    ただし$\emb^{U_i}: U_i \hookrightarrow U$は埋め込み写像である.

\end{document}
