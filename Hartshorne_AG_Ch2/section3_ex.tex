\documentclass[a4paper]{jsarticle}
\usepackage[all]{xy}
\usepackage{../math_note, exercise, enumitem}
\renewcommand{\thesection}{Ex3.\arabic{section}}

\newcommand{\shO}{\mathcal{O}}
\newcommand{\Sch}{\mathbf{Sch}}
\newcommand{\Var}{\mathbf{Var}}
\newcommand{\Rings}{\mathbf{Rings}}
\newcommand{\red}[1]{#1_{\text{red}}}
\newcommand{\basesp}{\operatorname{sp}}
\newcommand{\res}{\operatorname{res}}
\newcommand{\Rat}{\operatorname{Rat}} %% rational points
\newcommand{\Proj}{\operatorname{Proj}}
\newcommand{\CoverU}{\mathfrak{U}}
\newcommand{\OpenIn}{\text{ :: open in }}
\newcommand{\ClosedIn}{\text{ :: open in }}
\newcommand{\mnot}{/\hspace{-1.2ex}}

\begin{document}
    schemeやscheme morphismの性質の定義は
    \url{section3_text.pdf}にまとめたので参照すること.
    同じPDFで$B$-fin.gen. schemeなどの独自の用語を定義している.
    \url{http://stacks.math.columbia.edu/tag/01T0}も参照すると良い.

    記法について.$\Spec A_f=D_A(f)$と書く.

\section{Definition(s) of Locally of Finite Type Morphism.} %% Ex3.1 

\begin{Lemma}[Nike's Lemma]
    $X$ :: scheme, $U,V \subseteq X, U=\Spec A, V=\Spec B$かつ
    $U \cap V \neq \emptyset$とする.
    この時,任意の点$P \in U \cap V$に対し,$a \in A, b \in B$であって
    \[ P \in D_A(a)=D_B(b) \subset U \cap V \]となるものがある.
    系としてProp2.2より$A_a \cong B_b$が得られる.
\end{Lemma}
\begin{proof}
    適当に$a \in A, b \in B$をとり,
     \[ P \in D_B(b) \subseteq D_A(a) \subseteq U \cap V \]
    としよう.
    $X=\Spec B, X_f=D_B(b), \bar{b}=b|_{D_A(a)} \in A_a$
    としてEx2.16aを用いると,
    \[ D_B(b)=D_A(a) \cap D_B(b)=\Spec (A_a)_{\bar{b}}. \]
    なので,あとは$(A_a)_{\bar{b}}$を調べれば良い.

    $(A_a)_{\bar{b}}$の元は以下のように書ける.
    \[ \frac{u/a^m}{\bar{b}^n}=\frac{u}{a^m\bar{b}^n} ~~(m,n \in \N; u \in A). \]
    $\bar{b} \in A_a$なので$a^N\bar{b}=a' \in A$となる$N \in \N$が存在する.
    \[ \frac{u a^{nN}}{a^m a^{nN}\bar{b}^n}=\frac{u a^{nN}}{a^m a'^{n}}. \]
    仮に$m \geq n$とすると
    \[ \frac{u a^{nN}}{a^m a'^{n}}=\frac{ua^{m-n+nN}}{(aa')^m} \]
    $m \leq n$でも同様に分子分母に$a'^{n-m}$をかければ,
    $(A_a)_{\bar{b}}$の元は$A_{aa'}$の元として書ける.
    逆に$A_{aa'}$の元を$(A_a)_{\bar{b}}$の元として書くことは直ちに出来る.
    よって$(A_a)_{\bar{b}}=A_{aa'}$.

    以上より,$\alpha=aa' \in A, b \in B$について$D_B(b)=D_A(\alpha)$.
\end{proof}

\begin{Lemma}[Preimage of POS\footnote{Principle Open Set} is POS.]
    $f: X \to Y$ :: scheme morphism.
    $\Spec B \subseteq Y, f^{-1}\Spec B=\bigcup_{i \in I} \Spec C_i$とする.
    この時,以下が成立する.
    \[
        \Forall{b \in B} \Exists{\{c_i(\in C_i)\}}
        f^{-1}D_B(b)=\bigcup_{i \in I}D_{C_i}(c_i).
    \]
\end{Lemma}
\begin{proof}
    $U=\Spec B, V_i=\Spec C_i$とする.
    すると$f$の制限によりscheme morphism $f|_{V_i}: V_i \to U$が得られる.
    これは$V_i \hookrightarrow X \xrightarrow{f} Y$という写像で,
    したがって逆写像は$(f|_{V_i})(S)=f^{-1}(S) \cap V_i$であることに注意.
    structure sheafの間の射を考えると,以下が得られる.
    \[ \phi_i=\left((f|_{V_i})^{\#}\right)_{U}: B=\shO_U(U) \to (f|_{V_i})_* \shO_{V_i}(U)=C_i. \]
    ここでProp2.2を用いた.
    Prop2.3から,$\phi_i$は$f|_{V_i}: V_i \to U$に1-1対応し,
    特にtopological spaceとして
    \[ f|_{V_i}(\I{p})=\phi_i^{-1}(\I{p}) ~~(\I{p} \in \Spec C_i) \]
    が成り立つ.このことから以下が得られる.
    \[ f^{-1}(D_B(b)) \cap V_i=(f|_{V_i})^{-1}D_B(b)=D_{C_i}(\phi_i(b)). \]
    最左辺と最右辺を$\bigcup_{i \in I}$すれば主張が示せる.
\end{proof}

\begin{Lemma}
    $f \in A$とする.
    有限生成$A_{f}$代数は有限生成$A$代数でもある.
\end{Lemma}
\begin{proof}
    変数の数は問題にならないので1変数で証明する.
    (つまり以下で$A_{f}[x]$を多変数にしても構わない.)
    有限生成$A_{f}$代数$B$には$A_{f}[x]$からの全射が存在する.
    $A_{f}[x]$には$A[x,y]$から次のような全射が存在する.
    \[ y \mapsto 1/f \]
    これが全射であることは,
    \[ ay^nx^m \mapsto (a/f^n)x^m \in A_{f}[x] \]
    のように分かる.
    あとはこの写像が$A$準同型(代入写像)であることに注意すれば良い.
    よって$A[x,y] \to A_{f}[x] \to B$という全射が存在する.
\end{proof}

以下の命題を示す.
\begin{align*}
    {}&
    \Exists{\{B_i\}_{i \in I}}
    \lbra{Y=\bigcup_{i \in I} \Spec B_i} \land \lbra{\Forall{i \in I} f^{-1}(\Spec B_i)\text{ :: locally $B_i$-fin.gen. scheme}} \\
    \iff&
    \Forall{\Spec A \subseteq X} f^{-1}(\Spec A)\text{ :: locally $A$-fin.gen. scheme}
\end{align*}

下から上は自明である.上から下を示そう.

$U=\Spec A \subset X, V_i=\Spec B_i$とする.
$U \cap V_i$の各点$P$に対し,
\[ P \in D_{B_i}(b_{ij})=D_A(a_{ij}) \subseteq U \cap V_i \]
であるような$b_{ij} \in B_i, a_{ij} \in A$が取れる.
$P$を動かせば,このようにして$U$が被覆できる.
\[ U=\bigcup_{i,j} D_{B_i}(b_{ij})=\bigcup_{i,j} D_A(a_{ij}). \]
仮定より,各$V_i$は$\{ \Spec C_{ik} \}_{i,k}$で被覆され,
これらの$C_{ik}$は有限生成$B_i$代数
\footnote{$\phi_{ik}=\left((f|_{\Spec C_{ik}})^{\#}\right)_{\Spec B_i}$で代数とみなす.}
であるようにとれる.

Lemma (Preimage of POS is POS)より,$c_{ijk} \in C_{ik}$が存在し,
以下のようになる.
\[ f^{-1}U=\bigcup_{i,j} f^{-1}D_{B_i}(b_{ij})=\bigcup_{i,j} \bigcup_{k} D_{C_{ik}}(c_{ijk}). \]
$D_{C_{ik}}(c_{ijk})=\Spec (C_{ik})_{c_{ijk}}$であり,
$(C_{ik})_{c_{ijk}}$は有限生成$(B_i)_{b_{ij}}$代数.
これは有限生成代数の定義から存在する全射$B[x_1,\dots,x_n] \to C_{ik}$の両辺を局所化
\footnote{$C_{ik}$が$\phi_{ik}$による$B_i$代数であることと$c_{ijk}=\phi_{ik}(b_{ij})$を用いて計算する.}
すれば分かる.
$(B_i)_{b_{ij}} \cong A_{a_{ij}}$(Nike's Lemmaの最後の文)と最後のLemmaより,
$(C_{ik})_{c_{ijk}}$は有限生成$A$代数.

以上より,$f^{-1}\Spec A$は$\Spec (C_{ik})_{c_{ijk}}$で被覆され,
各$(C_{ik})_{c_{ijk}}$は有限生成$A$代数である.

\section{Definition(s) of Quasi-Compact Morphism.} %% Ex3.2 
以下を示す.
\begin{align*}
    {}&
        \Exists{\{B_i\}_{i \in I}}
        \lbra{Y=\bigcup_{i \in I} \Spec B_i} \land \lbra{\Forall{i \in I} f^{-1}(\Spec B_i)\text{ :: quasi-compact.}} \\
    \iff&
        \Forall{\Spec A \subseteq Y} f^{-1}(\Spec A)\text{ :: quasi-compact.}
\end{align*}

まず$\Spec A=\bigcup_{i,j} D_{B_i}(b_{ij})$となるように$b_{ij}$をとる.
Ex2.13bより$\Spec A$はquasi-compactだから$b_{ij}$は有限個でよい.
$f^{-1}\Spec B_i$はopen subschemeだから,
$f^{-1}\Spec B_i=\bigcup_{i,k} \Spec C_{ik}$なる$C_{ik}$がある.
仮定より$f^{-1}\Spec B_i$はquasi-compactであるから$C_{ik}$は有限個.
これにEx3.1の中で示したLemma (Preimage of POS is POS)を用いると以下のようになる.
\[ f^{-1}\Spec A=\bigcup_{i,j} f^{-1}D_{B_i}(b_{ij})=\bigcup_{i,j} \bigcup_{k} D_{C_{ik}}(c_{ijk}). \]
確認したとおり組$(i,j,k)$は高々有限の組み合わせしか無い.
Ex2.13の証明にあるとおり,$D_{C_{ik}}(c_{ijk})$はquasi-compactだから,
$f^{-1}\Spec A$はquasi-compactな開集合の有限和.
よって$f^{-1}\Spec A$もquasi-compact.

\section{Definition(s) of Finite Type Morphism.} %% Ex3.3 
    \subsection{Finite Type = Locally Finite Type+Quasi-Compact.}
    定義より明らか.

    \subsection{Another Definition of Finite Type Morphism.}
    Ex3.1の弱い形である.

    \subsection{If $f$ :: Finite Type and Any $\Spec A \subseteq f^{-1}(\Spec B)$, $A$ :: Fin.Gen $B$-Alg.}

\section{Definition(s) of Finite Morphism.} %% Ex3.4 
    Ex3.1と同様に証明できる.

\section{Finite/Quasi-Finite Morphism.} %% Ex3.5 
    $f: X \to Y$がquasi-finite morphismであるとは,
    任意の点$y \in Y$について$f^{-1}(y)$が有限集合であるという事である.

    \subsection{Finite $\implies$ Quasi-Finite.}
    \subsection{Finite $\implies$ Closed.}
    \subsection{Give an Example of morphism that is Surjective, Finite-Type, Quasi-Finite BUT NOT Finite.}

\section{Function Field.} %% Ex3.6 
    $X$ :: integral schemeとし,
    $\shO_{X, \zeta}$が体であることと,
    任意のaffine open subset $\Spec A$について$\shO_{X, \zeta} \cong \Quot(A)$であることを示す.

    $\zeta \in X$をgeneric pointとしよう.
    $\{\zeta\}$は$X$でdenseな1点集合だから,任意の開集合に含まれる.
    だから$\Spec A$ :: affine open subsetをどのように取ってもよい.
    $\shO_{X,\zeta}=(\shO_X|_{\Spec A})_{\zeta}=A_{\zeta}$であり,
    $A=\shO_X|_{\Spec A}(\Spec A)$がintegralであることから,
    $\zeta=(0) \in \Spec A$.
    以上から
    \[ \shO_{X,\zeta}=(\shO_X|_{\Spec A})_{\zeta}=A_{\zeta}=A_{(0)}=\Quot(A) \]
    が得られる.

\section{Dominant, Generically Finite Morphism of Finite Type of Integral Schemes.} %% Ex3.7 

\section{Normalization.} %% Ex3.8 
    schemeがnormalであるとは,その任意の局所環がintegrally closed domainである,という意味である.
    $X$ :: integral schemeとする.
    $U=\Spec A \subseteq X$に対し,$\tilde{A}$を$A$のintegral closure, $\tilde{U}=\Spec \tilde{A}$とする.

    \subsection{$\{ \tilde{U} \}$ can be glued.}
    Ati-Mac Prop5.1をつかう.

    \subsection{$\tilde{X}$ has a UMP.}

    \subsection{$X$ :: finite type $\implies$ $\tilde{X} \to X$ :: finite.}

\section{The Topological Space of a Product.} %% Ex3.9 
    \subsection{$\affine^1_k \times_{\Spec k} \affine^1_k \cong \affine^2_k$ but $\affine^2_k \neq \affine^1_k \times \affine^1_k$ as sets.}
    $\affine_k^1=\Spec k[x]$として$\affine^1_k \times_{\Spec k} \affine^1_k$を考える.

    \paragraph{$\affine^1_k \times_{\Spec k} \affine^1_k \cong \affine^2_k$.}
    $\affine^1_k \times_{\Spec k} \affine^1_k \cong \Spec k[x] \otimes_k k[y]$かつ,
    $k[x] \otimes_k k[y] \cong k[x,y]$(Ch I, Ex3.18の解答を参照.)なので明らか.

    \paragraph{$\affine^1_k \times_{\Spec k} \affine^1_k \neq \affine^2_k$ as sets.}
    $\Spec k[x,y]$は$(y-x^2)$のような点(generic point of a variety)を含むが,
    $\affine^1_k \times_{\Spec k} \affine^1_k$にこれに対応する点はない.

    \subsection{Describe $\Spec k(s) \otimes_{\Spec k} \Spec k(t)$.}
    $\Spec k(s) \otimes_{\Spec k} \Spec k(t) \cong \Spec k(s) \otimes_k k(t)$である.
    $k(s) \otimes_k k(t)$の元は$0$でなければ単元である.
    実際,$f,g,f',g' \neq 0$であるとき,
    \[ \frac{f(s)}{g(s)} \otimes \frac{f'(t)}{g'(t)} \cdot \frac{g(s)}{f(s)} \otimes \frac{g'(t)}{f'(t)}=1 \otimes 1=1. \]
    よって$k(s) \otimes_k k(t)$は体で,$\Spec k(s) \otimes_{\Spec k} \Spec k(t)$は1点scheme.

\section{Fibres of a Morphism.} %% Ex3.10 
\subsection{$\basesp(X_y) \homeo f^{-1}(y)$.}
    \subsubsection{Affine Case}
    $\phi: B \to A, f: X=\Spec A \to \Spec B=Y$とし,
    $A$を$\phi$で$B$代数とみなす.
    $\I{p} \in \Spec B, S=B-\I{p}$とすると,
    $A \otimes_B k(\I{p})$は以下のようになる.
    なお,以下で$\phi(\I{p})$から生成されるイデアルを
    $I_{\I{p}}=\phi(\I{p})A=\langle \phi(\I{p}) \rangle, T=\overline{\phi(S)}$と置く.
    \begin{align*}
        {}& A \otimes_B \frac{S^{-1}B}{S^{-1}\I{p}} \\
        =& A \otimes \bar{S}^{-1}\left( \frac{B}{\I{p}} \right) \\
        \cong& A \otimes \frac{B}{\I{p}} \otimes S^{-1}B \\
        \cong& \frac{A}{\phi(\I{p})A} \otimes S^{-1}B \\
        \cong& T^{-1}\left( \frac{A}{I_{\I{p}}} \right)
    \end{align*}
    途中でAti-Mac Cor3.4, Prop3.5, Ex2.2を使った.

    Ati-Mac Prop1.1, 3.11より,$T^{-1}\left( \frac{A}{I_{\I{p}}} \right)$の素イデアルは,
    $A$の素イデアルであって,$I_{\I{p}}$を含み,
    $T$と共通部分を持たないものに対応する.
    \[
        \Spec T^{-1}\left( \frac{A}{I_{\I{p}}} \right)
        \homeo
        \{\I{q} \in \Spec A \mid I_{\I{p}} \subseteq \I{q} \land \phi(S) \cap \I{q}=\emptyset. \}
    \]
    同相であることは以下のように一般論から分かる.
    まず,任意のイデアル$\I{a} \subseteq A$について
    $\Spec \frac{A}{\I{a}}$は$\Spec A$の閉集合$V(\I{a})$と同相である
    \footnote{$V(\I{a}) \cap V(I) \leftrightarrow V \left( \frac{\I{a}+I}{\I{a}} \right)$なので同相.}.
    また任意の積閉集合$S \subseteq A$について$\Spec S^{-1}A$は$\Spec A$の部分集合と同相
    \footnote{みなす時の対応は$\I{p}S^{-1}A \leftrightarrow P \cap A$である.}.
    よって$\Spec T^{-1}\left( \frac{A}{I_{\I{p}}} \right)$は
    $V(I_{\I{p}})$の部分集合と同相である.

    一方,$f^{-1}(\I{p})=\{ \I{q} \in \Spec A \mid \phi^{-1}(\I{q})=\I{p} \}$.
    なので$\I{q} \in \Spec A$についての命題
    \[
        I_{\I{p}} \subseteq \I{q} \land T \cap \bar{\I{q}}=\emptyset
        \iff
        f(\I{q})=\phi^{-1}(\I{q})=\I{p}
        ~~~(*)
    \]
    が示されれば証明が完了する.
    ただし$\bar{\I{q}}=\frac{\I{q}}{I_{\I{p}}}$.

    まず$\I{p} \not \supseteq \ker \phi$だとしよう.
    すると$S \cap \ker \phi \neq \emptyset$なので,$\phi(S) \ni 0$.
    任意の$\I{q}$について$\I{q} \ni 0$なので,
    \[ T \cap \bar{\I{q}} \supseteq \overline{\phi(S) \cap \I{q}} \supseteq \overline{\{0\}}. \]
    よって$(*)$の左辺は常に偽.
    同じ条件の下で(*)の右辺が偽になることは明らかなので,
    $\I{p} \not \supseteq \ker \phi \implies (*)$が言えた.

    続いて$\I{p} \supseteq \ker \phi$だとしよう.
    この時$I_{\I{p}}=\phi(\I{p})$となる
    \footnote
    {
    $\phi: B \to A$について$\ker \phi \subseteq \I{b} \subseteq B$としよう.
    $B/\ker \phi \cong \im \phi$の同型射は$b \bmod \ker \phi \mapsto \phi(b)$なので,
    これに$\I{b}$を入れれば$\I{b}/\ker \cong \phi(\I{b})$となる.
    $\ker \phi \subseteq \I{b}$より左辺はイデアルだから右辺もイデアル.
    }.
    $\phi(S)=\phi(B-\I{p})$だから,
    \begin{align*}
        {}&         \phi(\I{p}) \subseteq \I{q} \land T \cap \bar{\I{q}}=\emptyset \\
        \implies&   \phi(\I{p}) \cap \I{q}=\phi(\I{p}) \land \overline{\phi(B-\I{p}) \cap \I{q}}=\emptyset \\
        \implies&   \phi(\I{p}) \cap \I{q}=\phi(\I{p}) \land \phi(B-\I{p}) \cap \I{q})=\emptyset \\
        \implies&   \phi(B) \cap \I{q}
                    =\left( \phi(\I{p}) \cup \phi(B-\I{p}) \right) \cap \I{q}
                    =(\phi(\I{p}) \cap \I{q}) \cup (\phi(B-\I{p}) \cap \I{q})
                    =\phi(\I{p}) \\
        \iff&   \phi^{-1}(\I{q})=\I{p}.
    \end{align*}
    最後の行で準同型定理を用いた.
    逆に$\I{p}=\phi^{-1}(\I{q})$ならば$\phi(\I{p}) \subseteq \I{q}$は明らか.
    同様に$\phi^{-1}(A-\I{q})=B-\phi^{-1}(\I{q})=B-\I{p}$より
    $\phi(B-\I{p}) \subseteq A-\I{q}$も得られる.
    最後に$T \cap \bar{\I{q}}=\emptyset$を示す.
    これは以下と同値である.
    \[ {}^{\not \exists} x \in \I{q}, y \in \phi(B-\I{p}),~~ x-y \in \I{p}. \]
    このような$x,y$が存在すると仮定する.
    $x-y \in \I{p}=\I{q} \cap \phi(B)$なので$x-(x-y)=y \in \I{q}$.
    仮定と合わせて$y \in \I{q} \cap \phi(B-\I{p})$を得るが,
    $\I{q} \cap \phi(B-\I{p}) \subseteq \I{q} \cap (A-\I{q})=\emptyset$なので,
    矛盾が生じた.
    以上より$\I{p} \supseteq \ker \phi \implies (*)$が言えた.

    \subsubsection{General Case.}
    $Y$の$y$を含むaffine open subset $\tilde{Y}$をとる.
    すると$f^{-1}\tilde{Y}$もopen afine coveringをもつので,
    それを$f^{-1}\tilde{Y}=\bigcup X_{i}$とする.
    $f: X \to Y$を制限して
    $f|_{X_{i}}: X_{i} \to Y_i$とする.
    するとThem3.3の証明のStep6,7より,$X_y$は
    \[ (X_y)_{i}:=X_{i} \times_{\tilde{Y}} \Spec (\shO_{\tilde{Y},y}/\I{m}_{\tilde{Y},y}) \]
    の貼り合わせ
    \footnote
    {
    ここで,$y \not \in Y_i$である場合は
    $\Spec (\shO_{Y_i,y}/\I{m}_{Y_i,y}) \to Y_i$が無い.
    これで大丈夫なのか気になる.
    $\shO_{Y_i,y}=\varinjlim_{y \in V \subseteq Y_i} \shO_Y(V)$
    は$y \not \in Y_i$の時$\{0\}$のdirect limitなので$0$(零環)となる.
    したがって
    $\Spec (\shO_{Y_i,y}/\I{m}_{Y_i,y}) \to Y_i$は零写像から誘導される物になり,
    $(X_y)_{ij}=X_{ij} \times_{Y_i} \emptyset=0$となる.
    以上から,$y \in Y_i$かどうか気にせず上のように述べて問題ない.
    }
    .
    $\basesp X_y$は$\basesp (X_y)_{i}$の張り合わせで,
    Affine Caseでの議論により
    $\basesp (X_y)_{i} \homeo (f|_{X_{i}})^{-1}(y)=f^{-1}(y) \cap X_{i}$.
    よって$\basesp X_y=\bigcup_{i} (f^{-1}(y) \cap X_{i})=f^{-1}(y)$.
    位相空間としては
    $(f|_{X_{ij}})^{-1}(y) \homeomap \basesp (X_y)_{ij} \to \basesp X_y$を使って貼り合わせる.

    \subsection{Another Solution of (b).}
    Ch.I Ex3.18(Product of Affine Varieties)で使った補題を少し変形したものと,
    中国剰余定理を用いる.

    \begin{Lemma}
        $I,J$をそれぞれ$k[s][t](=k[s,t]), k[s]$のイデアルとする.
        この時,以下が成り立つ.
        \[ \frac{k[s][t]}{I} \otimes_{k[s]} \frac{k[s]}{J} \cong \frac{k[s][t]}{I+J^e} \]
        ただし,$\frac{k[s][t]}{I}, \frac{k[s]}{J}$はそれぞれ
        $f \mapsto f \bmod I, f \mapsto f \bmod J$で$k[s]$代数とみなす.
    \end{Lemma}
    \begin{proof}
        $\pi_1: k[s][t] \to \frac{k[s][t]}{I}, \pi_2: k[s] \to \frac{k[s]}{I}$を標準的全射とする.
        すると$\pi_1 \otimes_{k[s]} \pi_2$
        も全射である.
        $\kappa: k[s][t] \to k[s][t] \otimes_{k[s]} k[s]$を標準的同型だとすると,
        以下は全射である.
        \[
            \kappa \circ \pi_1 \otimes_{k[s]} \pi_2:
                k[s][t] \to k[s][t] \otimes_{k[s]} k[s]
                \to k[s][t] \otimes k[s]
                \to \frac{k[s][t]}{I} \otimes_{k[s]} \frac{k[s]}{J}
        \]
        これの$\ker$を計算すると$I+J^e$となり,準同型定理により主張が得られる.
    \end{proof}

    これをつかって(b)を計算していく.
    \paragraph{At $y=(s-a) \in Y ~~(a \neq 0)$.}
    $\phi(y)=(\bar{t}^2-a)=(\bar{t}-\sqrt{a}) \cap (\bar{t}+\sqrt{a})$だから,
    (a)から以下が成り立つ.
    \[
        \basesp(X_y) \homeo
        f^{-1}(y)=f^{-1}V(y)=V(\phi(\I{a}))=\{ (\bar{t}-\sqrt{a}), (\bar{t}+\sqrt{a}) \}.
    \]
    $k(y)=B_y/yB_y \cong (B/y)_{\bar{y}}$だが,$B/y \cong k$は体だから$k(y)=k$.
    $X_y=\Spec A \otimes_B B/y$なので補題が使える.
    \begin{align*}
        \frac{k[s,t]}{(s-t^2)} \otimes_{k[s]} \frac{k[s,u]}{(s-a,u)}
        \cong& \frac{k[s,t,u]}{(s-t^2, s-a, u)} \\
        \cong& \frac{k[t]}{(t^2-a)} \\
        =& \frac{k[t]}{(t-\sqrt{a}) \cap (t+\sqrt{a})} \\
        \cong& \frac{k[t]}{(t-\sqrt{a})} \oplus \frac{k[t]}{(t+\sqrt{a})} \\
        \cong& k \times k
    \end{align*}
    途中で中国剰余定理を使った.
    このことから$X_y=\Spec (k \times k)$で,各点での剰余体は$k$.

    \paragraph{At $y=(s) \in Y$.}
    \begin{align*}
        \frac{k[s,t]}{(s-t^2)} \otimes_{k[s]} \frac{k[s,u]}{(s,u)}
        \cong& \frac{k[s,t,u]}{(s-t^2, s, u)} \\
        \cong& \frac{k[t]}{(t^2)} \\
    \end{align*}
    $\frac{k[t]}{(t^2)}$は$(t) \bmod (t^2)$を唯一の極大イデアルとする局所環なので,
    Ex2.3bより$\Spec \frac{k[t]}{(t^2)}$は1点空間.
    また,non-reduced schemeである.

    \paragraph{At $y=(0)=\eta \in Y$.}
    $(B/\eta)_{\eta}=B_{(0)}=k(s)$なので$k(\eta)=k(s)$.
    $S=k[s]-\{0\}$とすると以下のように計算できる.
    \begin{align*}
        \frac{k[s,t]}{(s-t^2)} \otimes_{k[s]} S^{-1}k[s]
        \cong& \bar{S}^{-1} \frac{k[s,t]}{(s-t^2)} \\
        \cong& \frac{S^{-1}k[s,t]}{S^{-1} (s-t^2)} \\
        \cong& \frac{k(s)[t]}{(t^2-s)}
    \end{align*}
    $t^2-s$は$k(s)$係数既約多項式だから,この環は体.
    なので$X_y=\Spec \frac{k(s)[t]}{(t^2-s)}$は1点空間である.
    しかも剰余体は$k(s)=k(y)$の2次拡大体.

\section{Closed Subschemes.} %% Ex3.11 
\subsection{Closed Immersions are Stable under Base Extension.}

\subsection{* Closed Subscheme of Affine Scheme is Determined by a Suitable Ideal.}
$X=\Spec A$とそのclosed subscheme $Y$を考える.
$Y=X$ならば主張は自明なので$Y \subsetneq X$とする.


\subsection{The Smallest Subscheme Structure on a Closed Subset.}

\subsection{The Scheme-Theoretic Image of $f$.}

\section{Closed Subschemes of $\Proj S$.} %% Ex3.12 

\section{Properties of Morphisms of Finite Type.} %% Ex3.13 

\section{The Closed Points of Scheme of Finite Type over a Field.} %% Ex3.14 

\section{Geometrically Irreducible/Reduced/Integral Schemes.} %% Ex3.15 
    $k$ :: field, $X$ :: scheme of finite type over $k$.
    この時$X$は$\Spec \frac{k[x_1,\dots,x_n]}{I}$という形のopen affine subschemeで被覆できる.
    以下ではこの被覆のうちの一つのopen affine subschemeを取って考察をする.
    $R=\frac{k[x_1,\dots,x_n]}{I}$としておく.
    また,$X \times_{\Spec k} \Spec \bar{k}$を$X \times_k \bar{k}$と略す.

    \subsection{Geometrically Irreducible.}
    以下の条件の同値性を示す.
    \begin{enumerate}[label=(\roman*)]
        \item $X \times_k \bar{k}$ is irreducible.
        \item $X \times_k k_s$ is.
        \item $X \times_k K$ is, for every extension field $K$ of $k$.
    \end{enumerate}
    ただし$k_s$は$k$の分離閉包で,$\bar{k}$の部分体である.
    これらのいずれか(したがって全部)が
    成り立つ$X$はgeometrically irreducibleである,という.

    (iii)$\implies$(i),(ii)は明らか.
    また,一般の位相空間$T$について以下が成り立つ.
    (TODO:check)
    \[
        T\text{ :: irreducible}
        \iff
        \Exists{\{U_i\}\text{ :: open cover of }T}
        \Forall{i,j} U_i \cap U_j \neq \emptyset \land U_i\text{ :: irreducible.}
    \]
    よって,(i)$\implies$(ii)$\implies$(iii)をaffine caseで確かめれば十分である.
    Ati-Mac Ex1.19より,これは更に,次の3つの条件が同値であることだと言い換えられる.
    \begin{enumerate}[label=(\roman*)]
        \item $\Nil(R \otimes_k \bar{k})$ is a prime ideal.
        \item $\Nil(R \otimes_k k_s)$ is.
        \item $\Nil(R \otimes_k K)$ is, for every extension field $K$ of $k$.
    \end{enumerate}

    \subsection{Geometrically Reduced.}
    以下の条件の同値性を示す.
    \begin{enumerate}[label=(\roman*)]
        \item $X \times_k \bar{k}$ is reduced.
        \item $X \times_k k_p$ is.
        \item $X \times_k K$ is, for every extension field $K$ of $k$.
    \end{enumerate}
    ただし$k_p$は$k$の完全閉包で,$\bar{k}$の部分体である.
    これらのいずれか(したがって全部)が
    成り立つ$X$はgeometrically reducedである,という.

    (iii)$\implies$(i),(ii)は明らか.
    また,Ex2.3aより,reducedという性質はlocalな性質であるから,
    一般のscheme $S$について以下が成り立つ.
    \[
        S\text{ :: reduced}
        \iff
        \Exists{\{U_i\}\text{ :: open cover of }S}
        \Forall{i} \shO_S(U_i)\text{ :: reduced.}
    \]
    よって,(i)$\implies$(iii)をaffine caseで確かめれば十分である.
    これはaffine caseでは更に言い換えられる.
    \begin{enumerate}[label=(\roman*)]
        \item $\Nil(R \otimes_k \bar{k})=0$
        \item $\Nil(R \otimes_k k_s)=0$.
        \item $\Nil(R \otimes_k K)=0$, for every extension field $K$ of $k$.
    \end{enumerate}

    \subsection{Geometrically Integral.}
    $X \times_k \bar{k}$がintegralであるとき
    $X$はgeometrically integralであるという.
    \textbf{integral} schemeだがgeometrically irreducibleでない,
    またはgeometrically reducedでない例を作る.

\section{Noetherian Induction.} %% Ex3.16 

\section{Zariski Spaces.} %% Ex3.17 
    $X$ :: topological spaceについて,
    $X$がnoetherianかつ$X$の任意のirreducible closed subsetがただひとつのgeneric pointを持つとき,
    $X$はZariski spaceであるという.

    \subsection{$X$ :: Noetherian Scheme $\implies$ $\basesp(X)$ :: Zariski Space.}
    Ex2.9より明らか.

    \subsection{Minimal Nonempty Closed Subset of a Zariski Space $=$ One Point Set.}
    $X$ :: Zariski Space,$M$ :: minimal nonempty closed subset of $X$とする.
    この時,$M$ :: irreducibleである.
    実際,$M=Z_0 \cup Z_1$と空でない閉集合の和へ分解できるならば
    $Z_0,Z_1 \subsetneq M$となり,minimalityに反するからである.
    また,Ch I, Ex1.7より$M$はNoetherian.
    なので$g \in M$ :: generic pointが存在する.
    $M-\{g\}$が空でないと仮定し,$\mnot g \in M-\{g\}$をとる.
    $\cl_X(\{\mnot g\}) \subseteq M$であるが,
    $M$は極小な閉集合だから$\cl_X(\{\mnot g\})=M$.
    これは$M$のgeneric pointとして$g, \mnot g$の二つが取れることを意味し,
    generic pointの唯一性に矛盾する.

    \subsection{Zariski Space is a $\mathrm{T_0}$-Space.}
    互いに異なる2点$x,y \in X$をとる.
    これらのうち一方を含み,もう一方を含まない閉集合が存在することを示す.
    まず,一般の空間における閉包作用素の性質より,以下が成り立つ.
    \[ \cl_X(\{x,y\})=\cl_X(\{x\}) \cup \cl_X(\{y\}). \]
    左から順に$C_{xy}, C_x, C_y$とする.

    \paragraph{$C_{xy}$ :: not irreducible.}
    $\{x\} \subseteq C_y$ならば$C_x \subseteq \cl_X(C_y)=C_y$となる.
    よって$C_{xy}=C_y$が導かれる.
    しかし$C_y$ :: irreducible
    \footnote
    {
        $\cl_X(\{x\})$が$x$を含む最小の閉集合であることから,
        $\cl_X(\{x\})$は$x$を含む真の部分閉集合を持たない.
        よって$\cl_X(\{x\})=Z_0 \cup Z_1$ならば,
        $Z_0, Z_1$のどちらか一方は真の部分閉集合になり得ない.
    }
    だから,これは矛盾.
    したがって$x \not \in C_y$.
    逆に$y \not \in C_x$も得られる.

    \paragraph{$C_{xy}$ :: irreducible.}
    $C_x,C_y$は空でない閉集合だから,
    $C_{xy}$がirreducibleであったとすると,
    $C_x, C_y$のいずれかは$C_{xy}$と一致している.
    なので$x,y$のどちらか一方は$C_{xy}$のgeneric pointである.
    $x$がそのgeneric pointだと仮定しよう.
    $\{x\} \subseteq C_y$であれば$C_{xy}=\cl_X(C_y)=C_y$となるから,
    $\{x\} \subseteq C_y$から$y$が$C_{xy}$のgeneric pointであることが導かれる.
    これはgeneric pointの唯一性に反するから,
    $x \not \in C_y$.

    \subsection{The Generic Point of Irreducible Zariski Space is in Any Open Subset of That.}
    $X$ :: irreducible Zariski space, $g$ :: generic point of $X$とおく.
    $g$を含まない空でない開集合$U$が存在したと仮定する.
    すると$g \in U^c$であり,$U^c$は真の閉部分集合である.
    これは$\cl_X(\{g\}) \subseteq \cl_X(U^c) \subsetneq X$を意味するので,矛盾.

    \subsection{Specialization.}
    $X$ :: Zariski spaceとし,$X$の点に以下のように順序を入れたものを$\Sigma$とする.
    \[ x_1 \geq x_0 \iff x_1 \rightsquigarrow x_0 \iff \cl_X(\{x_1\}) \ni x_0. \]
    これは半順序集合をなす(CHECK).
    $x_1 \rightsquigarrow x_0$であるとき$x_0$は$x_1$のspecializationという.
    逆に$x_1$は$x_0$のgenerizationだという.

    \subsubsection{The Minimal/Maximal Elements of $\Sigma$.}
    $\Sigma$の極小元$x$は以下を満たす点である.
    \[ {}^{\not \exists} y \in \Sigma,~~ x \neq y \land \cl_X(\{x\}) \ni y. \]
    つまり$x$は$\{x\}$が閉集合であるような点である.
    よって$x$はclosed point.

    次に$x$を$\Sigma$の極大元だとする.
    これは以下を満たす.
    \[ {}^{\not \exists} y \in \Sigma,~~ x \neq y \land \cl_X(\{y\}) \ni x. \]
    $x$を含むirreducible componentのgeneric pointを$g$とする.
    $y \neq g$であるとき$\cl_X(\{y\}) \ni g$はgeneric pointの唯一性に反するから,
    $g$は$\Sigma$の極大元である.
    逆に,任意の元$x \neq g$に対し,$\cl_X(\{g\}) \ni x$が成立する.
    結局,$x$がそのgeneric point(すなわち$x=g$)であるときかつその時に限り,
    $x$は$\Sigma$の極大元となる.

    \subsubsection{Closed/Open Subset is Stable under Specialization/Generization.}
    $S \subseteq X$に対し,
    \[
        S_S=\{ y \in X \mid \Exists{x \in S} x \rightsquigarrow y. \},~~~
        S_G=\{ x \in X \mid \Exists{y \in S} x \rightsquigarrow y. \}
    \]
    とおく.
    $x \rightsquigarrow x$なので$S \subseteq S_S, S_G$となる.
    
    \paragraph{$S$ :: closed $\implies$ $S_S=S$.}
    $S \supseteq S_S$を示せば良い.
    これは以下と同値.
    \[
        \Forall{x \in S} \Forall{y \in X} 
        \cl_X(\{x\}) \ni y \implies y \in S
    \]
    これは以下から示せる.
    \[ \{x\} \subseteq S \implies \cl_X(\{x\}) \subseteq \cl_X(S)=S \]

    \paragraph{$S$ :: open $\implies$ $S_G=S$.}
    $S \supseteq S_G$を示せば良い.
    これは以下と同値.
    \[
        \Forall{y \in S} \Forall{x \in X} 
        \cl_X(\{x\}) \ni y \implies x \in S
    \]
    これの対偶は以下のようになる.
    \[
        \Forall{y \in S} \Forall{x \in X} 
        x \in S^c \implies y \not \in \cl_X(\{x\}) \subseteq \cl_X(S^c)=S^c
    \]
    これは明らかに成立する($y \in S$に注意).

    \subsection{$X$ :: Noetherian Topological Space $\implies$ $t(X)$ :: Zariski Space.}
    $t(X)$は$X$のirreducible closed subsetsであり,
    $t(X)$の閉集合は$X$の閉集合$Y$を用いて$t(Y)$と表せる集合である.

    \paragraph{$X$ :: Noetherian $\implies$ Irreducible Subset in $t(X)$ has Unique Generic Point.}
    $Y \subseteq X$がclosedだとし,
    さらに$t(Y) \subseteq t(X)$がirreducible closed subsetだとする.
    $t(Y)$の点は$X$のirreducible closed subsetだから,
    $X$ :: Noetherianより,$t(Y)$は極小元$G \subseteq Y$をもつ.
    $G \in t(Z)$すなわち$G \in t(Y \cap Z)$ならば$Y \cap Z=Y$すなわち$Y \subseteq Z$,
    ということを示せば,$G$ :: generic pointが得られる.

    \paragraph{$X$ :: Noetherian $\implies$ $t(X)$ :: Noetherian.}
    以下を使う.
    \[ \Forall{Y,Z \ClosedIn X} Y \subsetneq Z \iff t(Y) \subsetneq t(Z). \]
    これは次のように示される.
    まず$(\implies)$は,$z \in Z-Y$とすると,$\cl_X(z) \in t(Z)-t(Y)$となることから得られる.
    $\cl_X(z) \in t(Y)$ならば$z \in \cl_X(z) \subseteq Y$だが,
    $z \not \in Y$なのでこれはありえない.
    また$(\impliedby)$は,$t(Z)-t(Y)$の極小元を考えれば得られる.
    その極小元を$M$とし,$m \in M$とすると$\cl_X(m) \subseteq M$.
    $M$の極小性から等号が成り立つ.
    もし$m \in Y$ならば$\cl_X(m) \subseteq Y$かつ$\cl_X(m) \not \in t(Y)$となり,
    これはありえない.
    今示したことから,以下の同値が得られ,$t(X)$ :: Noetherianが示せる.
    \[ X_0 \supsetneq X_1 \supsetneq \dots \iff t(X_0) \supsetneq t(X_1) \supsetneq \dots \]
    ただし$X_0,X_1,\dots$は$X$の閉集合である.

\section{Constructible Sets.} %% Ex3.18 
    $X$ :: Zariski topological spaceの部分集合族$\mathfrak{F}_X$を,以下のように定める.
    \begin{enumerate}[label=(\arabic*),leftmargin=5\parindent]
        \item 任意の開集合は$\mathfrak{F}_X$に属す.
        \item $\mathfrak{F}_X$の有限個の元の共通部分は$\mathfrak{F}_X$に属す.
        \item $\mathfrak{F}_X$の元の補集合は$\mathfrak{F}_X$に属す.
        \item 以上の操作を繰り返して得られる集合のみが$\mathfrak{F}_X$の元である.
    \end{enumerate}
    $\mathfrak{F}_X$の元を$X$のconstructible subsetと呼ぶ.
    ひとつのZariski spaceしか扱わない時は$\mathfrak{F}_X$を$\mathfrak{F}$と略す.
    \subsection{$\mathfrak{F}$=$\{$ Finite Disjoint Union of Locally Closed Subsets.$\}=:\mathfrak{L}$}

    \begin{Lemma}
        $Z \subseteq X$ :: finite union of locally closed
        then
        $Z$ :: finite \textbf{disjoint} union of locally closed.
    \end{Lemma}
    \begin{proof}
        $Z=\bigcup_{i=1}^r C_i \cap O_i$がdisjoint unionであるためには,
        $\bigcup_{i=1} C_i$がlocally closed subsetのdisjoint unionで書ければ十分であることに注意する
        \footnote
        {
        $[D_j \cap (V_j \cap W_j)] \cap [D_k \cap (V_k \cap W_k)]=([D_j \cap V_j] \cap [D_k \cap V_k]) \cap (W_j \cap W_k)=\emptyset$
        }
        .
        実際,$\bigcup_{i=1} C_i=\bigcup_{j=1}^s D_j \cap V_j$となったとする.
        \[ W_j=\bigcup_{i;~ C_i \cap D_j \cap V_j=D_j \cap V_j} O_i \]
        とおくと,
        $C_i \cap D_j \cap V_j=D_j \cap V_j \mor \emptyset$(以下の構成から分かる)から
        $\bigcup_i C_i \cap D_j \cap V_j \cap O_i=D_j \cap V_j \cap W_j$.
        これより以下が得られる.
        \begin{align*}
            Z 
            =&  \bigcup_{i=1}^r \left(\bigsqcup_{j=1}^s D_j \cap V_j \cap C_i \right) \cap O_i \\
            =&  \bigcup_{i=1}^r \bigcup_{j=1}^s D_j \cap V_j \cap C_i \cap O_i \\
            =&  \bigcup_{j=1}^s \bigcup_{i=1}^r D_j \cap V_j \cap C_i \cap O_i \\
            =&  \bigsqcup_{j=1}^s D_j \cap (V_j \cap W_j).
        \end{align*}

        $\bigcup_{i=1} C_i$をlocally closed subsetのdisjoint unionで書く.
        $n=1,\dots,r$に対し,
        \[ \Sigma_n=\{ (i_1,\dots,i_n) \mid i_1<\dots<i_n \land i_1,\dots i_n \in \{1,\dots,r\} \} \]
        とおく.これは要素数$\binom{r}{n-r}$の有限集合である.
        さらに$\sigma=(i_1,\dots,i_n) \in \Sigma_n$に対して
        $C_{\sigma}=C_{i_1} \cap C_{i_n}$とする.
        以下は明らかにlocally closed.
        \[ K_{\sigma}=C_{\sigma} \cap \left( \bigcup_{m=n+1}^r ~\bigcup_{\tau \in \Sigma_m} C_{\tau} \right)^c \]
        これらはdisjointである.
        実際に$\sigma \in \Sigma_{n}, \sigma' \in \Sigma_{n'}$を考えてみる.
        $n < n'$ならば明らかにdisjoint.
        $n=n'$のとき,例えば$\sigma=(i_1,\dots,i_n),\sigma'=(i_1',\dots,i_n), i_1'<i_1$に対して
        $\sigma \cap \sigma'=(i_1',i_1,\dots,i_n)$とすると,
        $C_{\sigma} \cap C_{\sigma'}=C_{\sigma \cap \sigma'}$となる.
        $\sigma \cap \sigma' \in \Sigma_{n+1}$なのでやはりdisjoint.
    \end{proof}

    $\mathfrak{L}$の元は,以下のように書ける.
    \[
        \bigsqcup_{i=1}^r (C_i \cap O_i)
        \mwhere
        \{C_i\}_{i=1}^r, \{O_i\}_{i=0}^r\text{ :: closed,open subsets of }X.
    \]
    $\mathfrak{F} \supseteq \mathfrak{L}$は
    \[
        \bigsqcup_{i=1}^r C_i \cap O_i
        =\left(\bigcap_{i=1}^r (C_i^c)^c \cap O_i\right)^c
    \]
    から明らか.

    $\mathfrak{F} \subseteq \mathfrak{L}$を示すために,
    induction by construct of constructible subsetを行う.
    開集合全体を$\mathfrak{F}_0$とし,
    これらの元に(2),(3)の操作を$n (\geq 0)$回繰り返して得られる集合族を$\mathfrak{F}_{n}$とする.
    任意の$n$について$\mathfrak{F}_{n} \subseteq \mathfrak{L}$であることを示す.
    (1) $\mathfrak{F}_0$の元,すなわち開集合は明らかに$\mathfrak{L}$に属す.
    以下,$\mathfrak{F}_{n} \subseteq \mathfrak{L}$と仮定して,
    数学的帰納法により$\mathfrak{F}_{n+1} \subseteq \mathfrak{L}$を示す.
    (2) $\mathfrak{F}_{n}$の2個の元は$\mathfrak{L}$に属す.
    それらの共通部分は$\mathfrak{F}_{n+1}$に属すが,これは以下のように書ける.
    \[
        \left( \bigsqcup_{i=1}^r (C_i \cap O_i) \right) \cap \left( \bigsqcup_{j=1}^s (D_i \cap P_i) \right)
        =
        \bigsqcup_{\substack{1 \leq i \leq r \\ 1 \leq j \leq s}}
            (C_i \cap D_j) \cap (O_i \cap P_j).
    \]
    よって$\mathfrak{F}_{n}$の有限個の元の共通部分は$\mathfrak{L}$に属す.
    (3) $\mathfrak{F}_{n}$の元は$\mathfrak{L}$に属す.
    その補集合は
    \[
        \bigcap_{i=1}^r (C_i^c \cup O_i^c)
        =
        \bigcup_{1 \leq i,j \leq r} (C_i^c \cap O_j^c)
    \]
    これはlocally closed subsetのunionだから,Lemmaより,$\mathfrak{L}$に属す.
    以上より,任意の$n$について$\mathfrak{F}_{n} \subseteq \mathfrak{L}$である.
    まとめて,$\mathfrak{F} = \bigcup_{n=0}^{\infty} \mathfrak{F}_{n} \subseteq \mathfrak{L}$

    \subsection{Dense Constructible Subset In Irreducible Zariski Space.}
    constructible subsetがdenseなのはそれがgeneric point $\zeta$を含むときに限る.
    これを示そう.
    (a)からconstructible subset $Z$は以下のように書ける.
    \[
        Z=\bigsqcup_{i=1}^r (C_i \cap O_i)
        \mwhere
        \{C_i\}_{i=1}^r, \{O_i\}_{i=0}^r\text{ :: closed,open subsets of }X.
    \]
    Ex3.17dより,各$O_i$は$\zeta$を含む.
    なのですべての$C_i$が$\zeta$を含まない時に限り$Z$は$\zeta$を含まない.
    この時,$\zeta \not \in \bigcup_{i=1}^r C_i$なので
    $\bigcup_{i=1}^r C_i$は真の閉部分集合である.
    $C_i \cap O_i \subseteq C_i$より
    \[ \cl_X(Z) \subseteq \bigcup_{i=1}^r C_i \subsetneq X \]
    よって$\zeta \not \in Z$ならば$Z$はdenseでない.

    また,constructible subset $Z$がdenseならば,
    ある$i$について$\zeta \in C_i \cap O_i$となる.
    しかも$\zeta \in C_i$より$C_i=X$.
    よって$C_i \cap O_i=O_i \subseteq Z$となり,
    $Z$は$X$の開集合を含む.

    \subsection{$S \subseteq X$ :: Closed $\iff$ $S$ :: Constructible And Stable Under Specialization.}
    $(\implies)$はconstructible subsetの定義とEx3.17eから得られる.
    $(\impliedby)$を示す.

    $Z=\bigsqcup_{i \in I_0} C_i \cap O_i$をとり,
    $\bigcup_{i \in I_0} C_i$を($X$の)irreducible componentsに分解して
    $\bigcup_{i \in I_0} C_i=\bigcup_{(i,j) \in I_1} K_{ij}$とする
    ($X$ :: noetherianとCh.I Prop1.5を用いた).
    更に,$K_{ij} \cap O_i \neq \emptyset$となる$(i,j) \in I_1$を選び出して$I_2$とする.
    この時$Z=\bigcup_{(i,j) \in I_2} K_{ij} \cap O_i$となる.
    さて,$K_{ij}$はirreducibleなので
    $K_{ij} \cap O_i$は$K_{ij}$のgeneric point $\zeta_{ij}$を含む.
    $Z$がstable under specializationならば以下が成り立つ.
    \[
        \bigcup_{(i,j) \in I_2} K_{ij}
        =\bigcup_{(i,j) \in I_2} \cl_X(\{\zeta_{ij}\})
        \subseteq Z
        \subseteq \bigcup_{(i,j) \in I_2} K_{ij}
    \]
    よって$Z=\bigcup_{(i,j) \in I_2} K_{ij}$となり,これは閉集合.

    \subsection{If $f: X \to Y$ :: Continuous Map then $f^{-1}(\mathfrak{F}_Y)=\mathfrak{F}_X$.}
    すべて基本的な位相空間の結果である.
    (1) $U$ :: open in $Y$について$f^{-1}(U)$ :: open in $X$.
    (2) $Z,W \in \mathfrak{F}_Y$について$f^{-1}(Z \cap W)=f^{-1}(Z) \cap f^{-1}(W)$.
    (3) $Z \in \mathfrak{F}_Y$について$f^{-1}(Z^c)=(f^{-1}(Z))^c$.

\section{Chevalley's Theorem on Constructible Set.} %% Ex3.19 

\section{Dimension.} %% Ex3.20 

\section{$\Spec$ of D.V. Ring Gives Counterexample for Ex3.20a,d,e.} %% Ex3.21 

\section{Dimension of the Fibres of a Morphism.} %% Ex3.22 

\section{$t(V \times W)=t(V) \times_{\Spec k} t(W)$.} %% Ex3.23 

\end{document}
