\documentclass[a4paper]{jsarticle}
\usepackage[all]{xy}
\usepackage{../math_note, exercise}
\renewcommand{\thesection}{Ex3.\arabic{section}}

\newcommand{\shO}{\mathcal{O}}
\newcommand{\Sch}{\mathbf{Sch}}
\newcommand{\Var}{\mathbf{Var}}
\newcommand{\Rings}{\mathbf{Rings}}
\newcommand{\red}[1]{#1_{\text{red}}}
\newcommand{\basesp}{\operatorname{sp}}
\newcommand{\res}{\operatorname{res}}
\newcommand{\Rat}{\operatorname{Rat}} %% rational points
\newcommand{\Proj}{\operatorname{Proj}}
\newcommand{\CoverU}{\mathfrak{U}}
\newcommand{\OpenIn}{\text{ :: open in }}
\newcommand{\ClosedIn}{\text{ :: open in }}

\begin{document}
    schemeやscheme morphismの性質の定義は
    \url{section3_text.pdf}にまとめたので参照すること.
    同じPDFで$B$-fin.gen. schemeなどの独自の用語を定義している.
    \url{http://stacks.math.columbia.edu/tag/01T0}も参照すると良い.

    記法について.$\Spec A_f=D_A(f)$と書く.

\section{Definition(s) of Locally of Finite Type Morphism.} %% Ex3.1 

\begin{Lemma}[Nike's Lemma]
    $X$ :: scheme, $U,V \subseteq X, U=\Spec A, V=\Spec B$かつ
    $U \cap V \neq \emptyset$とする.
    この時,任意の点$P \in U \cap V$に対し,$a \in A, b \in B$であって
    \[ P \in D_A(a)=D_B(b) \subset U \cap V \]となるものがある.
    系としてProp2.2より$A_a \cong B_b$が得られる.
\end{Lemma}
\begin{proof}
    適当に$a \in A, b \in B$をとり,
     \[ P \in D_B(b) \subseteq D_A(a) \subseteq U \cap V \]
    としよう.
    $X=\Spec B, X_f=D_B(b), \bar{b}=b|_{D_A(a)} \in A_a$
    としてEx2.16aを用いると,
    \[ D_B(b)=D_A(a) \cap D_B(b)=\Spec (A_a)_{\bar{b}}. \]
    なので,あとは$(A_a)_{\bar{b}}$を調べれば良い.

    $(A_a)_{\bar{b}}$の元は以下のように書ける.
    \[ \frac{u/a^m}{\bar{b}^n}=\frac{u}{a^m\bar{b}^n} ~~(m,n \in \N; u \in A). \]
    $\bar{b} \in A_a$なので$a^N\bar{b}=a' \in A$となる$N \in \N$が存在する.
    \[ \frac{u a^{nN}}{a^m a^{nN}\bar{b}^n}=\frac{u a^{nN}}{a^m a'^{n}}. \]
    仮に$m \geq n$とすると
    \[ \frac{u a^{nN}}{a^m a'^{n}}=\frac{ua^{m-n+nN}}{(aa')^m} \]
    $m \leq n$でも同様に分子分母に$a'^{n-m}$をかければ,
    $(A_a)_{\bar{b}}$の元は$A_{aa'}$の元として書ける.
    逆に$A_{aa'}$の元を$(A_a)_{\bar{b}}$の元として書くことは直ちに出来る.
    よって$(A_a)_{\bar{b}}=A_{aa'}$.

    以上より,$\alpha=aa' \in A, b \in B$について$D_B(b)=D_A(\alpha)$.
\end{proof}

\begin{Lemma}[Preimage of POS\footnote{Principle Open Set} is POS.]
    $f: X \to Y$ :: scheme morphism.
    $\Spec B \subseteq Y, f^{-1}\Spec B=\bigcup_{i \in I} \Spec C_i$とする.
    この時,以下が成立する.
    \[
        \Forall{b \in B} \Exists{\{c_i(\in C_i)\}}
        f^{-1}D_B(b)=\bigcup_{i \in I}D_{C_i}(c_i).
    \]
\end{Lemma}
\begin{proof}
    $U=\Spec B, V_i=\Spec C_i$とする.
    すると$f$の制限によりscheme morphism $f|_{V_i}: V_i \to U$が得られる.
    これは$V_i \hookrightarrow X \xrightarrow{f} Y$という写像で,
    したがって逆写像は$(f|_{V_i})(S)=f^{-1}(S) \cap V_i$であることに注意.
    structure sheafの間の射を考えると,以下が得られる.
    \[ \phi_i=\left((f|_{V_i})^{\#}\right)_{U}: B=\shO_U(U) \to (f|_{V_i})_* \shO_{V_i}(U)=C_i. \]
    ここでProp2.2を用いた.
    Prop2.3から,$\phi_i$は$f|_{V_i}: V_i \to U$に1-1対応し,
    特にtopological spaceとして
    \[ f|_{V_i}(\I{p})=\phi_i^{-1}(\I{p}) ~~(\I{p} \in \Spec C_i) \]
    が成り立つ.このことから以下が得られる.
    \[ f^{-1}(D_B(b)) \cap V_i=(f|_{V_i})^{-1}D_B(b)=D_{C_i}(\phi_i(b)). \]
    最左辺と最右辺を$\bigcup_{i \in I}$すれば主張が示せる.
\end{proof}

\begin{Lemma}
    $f \in A$とする.
    有限生成$A_{f}$代数は有限生成$A$代数でもある.
\end{Lemma}
\begin{proof}
    変数の数は問題にならないので1変数で証明する.
    (つまり以下で$A_{f}[x]$を多変数にしても構わない.)
    有限生成$A_{f}$代数$B$には$A_{f}[x]$からの全射が存在する.
    $A_{f}[x]$には$A[x,y]$から次のような全射が存在する.
    \[ y \mapsto 1/f \]
    これが全射であることは,
    \[ ay^nx^m \mapsto (a/f^n)x^m \in A_{f}[x] \]
    のように分かる.
    あとはこの写像が$A$準同型(代入写像)であることに注意すれば良い.
    よって$A[x,y] \to A_{f}[x] \to B$という全射が存在する.
\end{proof}

以下の命題を示す.
\begin{align*}
    {}&
    \Exists{\{B_i\}_{i \in I}}
    \lbra{Y=\bigcup_{i \in I} \Spec B_i} \land \lbra{\Forall{i \in I} f^{-1}(\Spec B_i)\text{ :: locally $B_i$-fin.gen. scheme}} \\
    \iff&
    \Forall{\Spec A \subseteq X} f^{-1}(\Spec A)\text{ :: locally $A$-fin.gen. scheme}
\end{align*}

下から上は自明である.上から下を示そう.

$U=\Spec A \subset X, V_i=\Spec B_i$とする.
$U \cap V_i$の各点$P$に対し,
\[ P \in D_{B_i}(b_{ij})=D_A(a_{ij}) \subseteq U \cap V_i \]
であるような$b_{ij} \in B_i, a_{ij} \in A$が取れる.
$P$を動かせば,このようにして$U$が被覆できる.
\[ U=\bigcup_{i,j} D_{B_i}(b_{ij})=\bigcup_{i,j} D_A(a_{ij}). \]
仮定より,各$V_i$は$\{ \Spec C_{ik} \}_{i,k}$で被覆され,
これらの$C_{ik}$は有限生成$B_i$代数
\footnote{$\phi_{ik}=\left((f|_{\Spec C_{ik}})^{\#}\right)_{\Spec B_i}$で代数とみなす.}
であるようにとれる.

Lemma (Preimage of POS is POS)より,$c_{ijk} \in C_{ik}$が存在し,
以下のようになる.
\[ f^{-1}U=\bigcup_{i,j} f^{-1}D_{B_i}(b_{ij})=\bigcup_{i,j} \bigcup_{k} D_{C_{ik}}(c_{ijk}). \]
$D_{C_{ik}}(c_{ijk})=\Spec (C_{ik})_{c_{ijk}}$であり,
$(C_{ik})_{c_{ijk}}$は有限生成$(B_i)_{b_{ij}}$代数.
これは有限生成代数の定義から存在する全射$B[x_1,\dots,x_n] \to C_{ik}$の両辺を局所化
\footnote{$C_{ik}$が$\phi_{ik}$による$B_i$代数であることと$c_{ijk}=\phi_{ik}(b_{ij})$を用いて計算する.}
すれば分かる.
$(B_i)_{b_{ij}} \cong A_{a_{ij}}$(Nike's Lemmaの最後の文)と最後のLemmaより,
$(C_{ik})_{c_{ijk}}$は有限生成$A$代数.

以上より,$f^{-1}\Spec A$は$\Spec (C_{ik})_{c_{ijk}}$で被覆され,
各$(C_{ik})_{c_{ijk}}$は有限生成$A$代数である.

\section{Definition(s) of Quasi-Compact Morphism.} %% Ex3.2 
以下を示す.
\begin{align*}
    {}&
        \Exists{\{B_i\}_{i \in I}}
        \lbra{Y=\bigcup_{i \in I} \Spec B_i} \land \lbra{\Forall{i \in I} f^{-1}(\Spec B_i)\text{ :: quasi-compact.}} \\
    \iff&
        \Forall{\Spec A \subseteq Y} f^{-1}(\Spec A)\text{ :: quasi-compact.}
\end{align*}

まず$\Spec A=\bigcup_{i,j} D_{B_i}(b_{ij})$となるように$b_{ij}$をとる.
Ex2.13bより$\Spec A$はquasi-compactだから$b_{ij}$は有限個でよい.
$f^{-1}\Spec B_i$はopen subschemeだから,
$f^{-1}\Spec B_i=\bigcup_{i,k} \Spec C_{ik}$なる$C_{ik}$がある.
仮定より$f^{-1}\Spec B_i$はquasi-compactであるから$C_{ik}$は有限個.
これにEx3.1の中で示したLemma (Preimage of POS is POS)を用いると以下のようになる.
\[ f^{-1}\Spec A=\bigcup_{i,j} f^{-1}D_{B_i}(b_{ij})=\bigcup_{i,j} \bigcup_{k} D_{C_{ik}}(c_{ijk}). \]
確認したとおり組$(i,j,k)$は高々有限の組み合わせしか無い.
Ex2.13の証明にあるとおり,$D_{C_{ik}}(c_{ijk})$はquasi-compactだから,
$f^{-1}\Spec A$はquasi-compactな開集合の有限和.
よって$f^{-1}\Spec A$もquasi-compact.

\section{Definition(s) of Finite Type Morphism.} %% Ex3.3 
    \subsection{Finite Type = Locally Finite Type+Quasi-Compact.}
    定義より明らか.

    \subsection{Another Definition of Finite Type Morphism.}
    Ex3.1の弱い形である.

    \subsection{If $f$ :: Finite Type and Any $\Spec A \subseteq f^{-1}(\Spec B)$, $A$ :: Fin.Gen $B$-Alg.}

\section{Definition(s) of Finite Morphism.} %% Ex3.4 
    Ex3.1と同様に証明できる.

\section{Finite/Quasi-Finite Morphism.} %% Ex3.5 
    $f: X \to Y$がquasi-finite morphismであるとは,
    任意の点$y \in Y$について$f^{-1}(y)$が有限集合であるという事である.

    \subsection{Finite $\implies$ Quasi-Finite.}
    \subsection{Finite $\implies$ Closed.}
    \subsection{Give an Example of morphism that is Surjective, Finite-Type, Quasi-Finite BUT NOT Finite.}

\section{Function Field.} %% Ex3.6 
    $X$ :: integral schemeとし,
    $\shO_{X, \zeta}$が体であることと,
    任意のaffine open subset $\Spec A$について$\shO_{X, \zeta} \cong \Quot(A)$であることを示す.

    $\zeta \in X$をgeneric pointとしよう.
    $\{\zeta\}$は$X$でdenseな1点集合だから,任意の開集合に含まれる.
    だから$\Spec A$ :: affine open subsetをどのように取ってもよい.
    $\shO_{X,\zeta}=(\shO_X|_{\Spec A})_{\zeta}=A_{\zeta}$であり,
    $A=\shO_X|_{\Spec A}(\Spec A)$がintegralであることから,
    $\zeta=(0) \in \Spec A$.
    以上から
    \[ \shO_{X,\zeta}=(\shO_X|_{\Spec A})_{\zeta}=A_{\zeta}=A_{(0)}=\Quot(A) \]
    が得られる.

\section{Dominant, Generically Finite Morphism of Finite Type of Integral Schemes.} %% Ex3.7 

\section{Normalization.} %% Ex3.8 
    schemeがnormalであるとは,その任意の局所環がintegrally closed domainである,という意味である.
    $X$ :: integral schemeとする.
    $U=\Spec A \subseteq X$に対し,$\tilde{A}$を$A$のintegral closure, $\tilde{U}=\Spec \tilde{A}$とする.

    \subsection{$\{ \tilde{U} \}$ can be glued.}
    Ati-Mac Prop5.1をつかう.

    \subsection{$\tilde{X}$ has a UMP.}

    \subsection{$X$ :: finite type $\implies$ $\tilde{X} \to X$ :: finite.}

\section{The Topological Space of a Product.} %% Ex3.9 
    \subsection{$\affine^1_k \times_{\Spec k} \affine^1_k \cong \affine^2_k$ but $\affine^2_k \neq \affine^1_k \times \affine^1_k$ as sets.}
    $\affine_k^1=\Spec k[x]$として$\affine^1_k \times_{\Spec k} \affine^1_k$を考える.

    \paragraph{$\affine^1_k \times_{\Spec k} \affine^1_k \cong \affine^2_k$.}
    $\affine^1_k \times_{\Spec k} \affine^1_k \cong \Spec k[x] \otimes_k k[y]$かつ,
    $k[x] \otimes_k k[y] \cong k[x,y]$(Ch I, Ex3.18の解答を参照.)なので明らか.

    \paragraph{$\affine^1_k \times_{\Spec k} \affine^1_k \neq \affine^2_k$ as sets.}
    $\Spec k[x,y]$は$(y-x^2)$のような点(generic point of a variety)を含むが,
    $\affine^1_k \times_{\Spec k} \affine^1_k$にこれに対応する点はない.

    \subsection{Describe $\Spec k(s) \otimes_{\Spec k} \Spec k(t)$.}
    $\Spec k(s) \otimes_{\Spec k} \Spec k(t) \cong \Spec k(s) \otimes_k k(t)$である.
    $k(s) \otimes_k k(t)$の元は$0$でなければ単元である.
    実際,$f,g,f',g' \neq 0$であるとき,
    \[ \frac{f(s)}{g(s)} \otimes \frac{f'(t)}{g'(t)} \cdot \frac{g(s)}{f(s)} \otimes \frac{g'(t)}{f'(t)}=1 \otimes 1=1. \]
    よって$k(s) \otimes_k k(t)$は体で,$\Spec k(s) \otimes_{\Spec k} \Spec k(t)$は1点scheme.

\section{Fibres of a Morphism.} %% Ex3.10 
\subsection{$\basesp(X_y) \homeo f^{-1}(y)$.}
    $X,Y$ :: scheme, $f: X \to Y$, $y \in Y$とし,fiber $X_y$を考える.
    $k(y)$ :: the residue field at $y$は体だから,$\Spec k(y)$は1点空間.
    そこで定値写像$ct_y$を
    \[ ct_y: \Spec k(y)=\{*\} \to Y;~~~ * \mapsto y \]とする.
    $X_y:=X \times_Y \Spec k(y)$は以下の図式を伴う.
    \[
    \xymatrix
    {
    \ar[d] \ar[drr] X_y & {} &\ar@{-->}[ll]_{\exists!} \forall Z \ar[d]^-{\forall} \ar[dll]^(0.25){\forall}\\
    X \ar[r]_-{f} & Y & \ar[l]^-{ct_y} \Spec k(y)
    }
    \]
    以下,schemeはそのbase spaceだけを考える.
    (つまりtopology spaceの圏に落とし込んで考える.)
    普遍性が保たれることは,scheme morphismが$(f,f^{\#})$と言う
    topological spaceとstructure spaceの間の射の組であり,
    $f \neq g$ならば$\shO_X \to f_*\shO_Y$と$\shO_X \to g_*\shO_Y$の写像が
    一致し得ないことから分かる.
    $f^{-1}(y)$が$X_y$の普遍性を満たせば,
    直ちに\footnote{homeomorphismはtopology spaceの圏におけるisomorphismであることに注意.}
    $\basesp(X_y) \homeo f^{-1}(y)$が得られる.

    上の図式で$X_y$の部分に$f^{-1}(y)$を入れた時の図式は次のよう.
    示すべきは$Z \to f^{-1}(y)$が一意に存在することである.
    \[
    \xymatrix
    {
    \ar@{^{(}->}[d]_-{i} \ar[drr]^(0.3){j} f^{-1}(y) & {} & \forall Z \ar[d]^-{\forall q} \ar[dll]_(0.3){\forall p}\\
    X \ar[r]_-{f} & Y & \ar[l]^-{ct_y} \Spec k(y)
    }
    \]
    まず,$f^{-1}(y) \to Y$の射が可換であることを見る.
    つまり
    \[ \Forall{x \in f^{-1}(y)} f \circ i(x)=f(x)=y=ct_y \circ j(x) \]
    が成立することを示す必要があるが,これは$f^{-1}(y)$の定義.
    次に$Z \to Y$の射が可換であることから,以下が成り立つ.
    \[ \Forall{z \in Z} f \circ p(z)=y=ct_y \circ q(z). \]
    よって$Z \subseteq (f \circ p)(y)=p^{-1}(f^{-1}(y))$.
    逆の包含関係は自明だから,$Z=p^{-1}(f^{-1}(y))$が得られる.
    したがって$Z \to f^{-1}(y)$の射として$p$を取ることが出来る.
    \[
    \xymatrix
    {
    \ar@{^{(}->}[d]_-{i} \ar[drr]_(0.2){j} f^{-1}(y) & {} & \ar[ll]_-{p} \forall Z \ar[d]^-{\forall q} \ar[dll]^(0.2){\forall p}\\
    X \ar[r]_-{f} & Y & \ar[l]^-{ct_y} \Spec k(y)
    }
    \]
    $i$が単射であることから,この図式を可換にする$Z \to f^{-1}(y)$の射は$p$に一致する.
    よって$f^{-1}(y)$は$\basesp(X_y)$の普遍性を持つ.

\subsection{Fibers of $f: X=\Spec k[s,t]/(s-t^2) \to \Spec k[s]=Y$.}
    $k$ :: algebraically closed fieldとする.
    $A=k[s,t]/(s-t^2)=k[\bar{t},\bar{s}], B=k[s]$とおき,
    $X=\Spec A, Y=\Spec B$とする.
    $\bar{s}=\bar{t}^2$に注意.
    また,$f$を$\phi: B \to A; s \mapsto \bar{s}$から誘導されるmorphismだとする.
    この設定のもとで各点におけるfiberを調べていく.

    \paragraph{At $y=(s-a) \in Y ~~(a \neq 0)$.}
    まず$k(y)$を調べる.
    $k(y)=B_y/yB_y \cong (B/y)_{\bar{y}}$だが,$B/y \cong k$は体だから$k(y)=k$.
    よって$X_y=\Spec (A \otimes_B k)$となる.
    $\phi(y)=(\bar{t}^2-a)=(\bar{t}-\sqrt{a}) \cap (\bar{t}+\sqrt{a})$だから,
    (a)から以下が成り立つ.
    \[
        \basesp(X_y) \homeo
        f^{-1}(y)=f^{-1}V(y)=V(\phi(\I{a}))=\{ (\bar{t}-\sqrt{a}), (\bar{t}+\sqrt{a}) \}.
    \]
    各点でのresidue fieldを見ていく.
    $X_y$は$\Spec A \otimes_B k$である.
    ここでの$T_y=A \otimes_B k$は,
    $A,k(\cong (B/y)_{\bar{y}})$をそれぞれ$\phi, s \mapsto a$
    \footnote{$s \mapsto a$は$s \mapsto s \bmod (s-a)$という写像を書き換えたものである.}
    で$B$代数とみなしている.
    この時,$T_y$は
    \[ \I{m}_{\pm}=\langle (\bar{t}-\pm \sqrt{a})\otimes_B 1 \rangle \]を極大イデアルにもち,
    それぞれでの剰余体は$k$である.
    これは$F=\sum_{i=0}^d c_i(\bar{t}^i \otimes 1) \in A$について$F \otimes 1$を変形してみると分かる.
    \begin{align*}
        {}& F \otimes 1 \\
        =& \sum_{i=0}^d c_i(\bar{t}^i \otimes 1) \\
        =&  \sum_{0 \leq i \leq d, i \in 2\Z} c_i(\bar{t}^i \otimes 1)+\sum_{0 \leq i \leq d, i \not \in 2\Z} c_i(\bar{t}^i \otimes 1) \\
        =&  \sum_{0 \leq i \leq \lfloor d/2 \rfloor} c_i(\bar{t}^{2i} \otimes 1)+\sum_{0 \leq i \leq d, i \not \in 2\Z} c_i(\bar{t}^i \otimes 1) \\
        =&  \sum_{0 \leq i \leq \lfloor d/2 \rfloor} c_i(a^i \otimes 1)+\sum_{0 \leq i \leq d, i \not \in 2\Z} c_i(\bar{t}^i \otimes 1) \\
        =&  \left(\sum_{0 \leq i \leq \lfloor d/2 \rfloor} c_ia^i \otimes 1\right)+\sum_{0 \leq i \leq d, i \not \in 2\Z} c_i(\bar{t}^i \otimes 1).
    \end{align*}
    途中で
    \[ s(1 \otimes 1)=\bar{s} \otimes 1=\bar{t}^2 \otimes 1=a \otimes 1=1 \otimes a=(1 \otimes 1)s \]
    を使った.
    したがって$(F \otimes 1)(\pm \sqrt{a})=F(\pm \sqrt{a}) \otimes 1$となる.
    よって$\bar{t} \otimes 1 \mapsto \pm \sqrt{a}$の$\ker$は$\I{m}_{\pm}$.
    また,上の計算からこの代入写像は$k$への全射であることが分かる.
    つまり$T_y/\I{m}_{\pm} \cong k$なので,$k(\I{m}_{\pm})=k$.
    $\Spec T_y$が2点のみを持つことから,これ以外に素イデアルはない
    \footnote
    {
        $\alpha^2 \neq a$であるとき
        \[
            (\bar{t}^2-\alpha^2) \otimes 1
            =\bar{t}^2 \otimes 1-\alpha^2 \otimes 1
            =s(1 \otimes 1)-s(1 \otimes \alpha^2/a)
            =(\bar{t}^2 \otimes 1)((1-\alpha^2/a) \otimes 1)
        \]
        だから$\langle (\bar{t}-\alpha)\otimes_B 1 \rangle$は素イデアルでない.
    }
    .

    \paragraph{At $y=(s) \in Y$.}
    $k(y)$はやはり$B/y \cong k$より$k(y)=k$.
    $\phi(y)=(\bar{t}^2)=(\bar{t})^2$となるから
    \[ \basesp(X_y) \homeo f^{-1}(y)=\{ (\bar{t}) \}. \]
    $X_y=\Spec (A \otimes_B k)$のsheafを考えよう.
    ここでの$T_y=A \otimes_B k$は,$A,k$をそれぞれ$\phi, s \mapsto 0$
    \footnote{$s \mapsto 0$は$s \mapsto s \bmod (s-0)$という写像を書き換えたものである.}
    で$B$代数とみなしている.
    この時$A \otimes_B k$はnon-reducedである.
    \begin{align*}
        {}  & (\bar{t} \otimes 1)^2 \in T_y \\
        =   & \bar{t}^2 \otimes 1 \\
        =   & \bar{s} \otimes 1 \\
        =   & s(1 \otimes 1) \\
        =   & 1 \otimes s(0) \\
        =   & 1 \otimes 0 \\
        =   & 0
    \end{align*}

    \paragraph{At $y=(0)=\eta \in Y$.}
    $(B/\eta)_{\eta}=B_{(0)}=k(s)$なので$k(\eta)=k(s)$.
    $\phi(\eta)=(0)=\zeta$ :: generic point of $A$で,
    $\phi^{-1}(\I{a})=\eta=(0)$となる$\I{a}$は他にないから
    $\basesp(X_{\eta}) \homeo \{\zeta\}$となる.
    ($\{\eta\} \neq V(\eta)$に注意.)
    $T_{\eta}=A \otimes_B k(s)$を考える.
    ここでは$A,k$をそれぞれ$\phi, s \mapsto s/1$で$B$代数とみなしている.

    \subsection{Another Solution of (b).}
    Ch.I Ex3.18(Product of Affine Varieties)で使った補題を少し変形した
    \[ \frac{k[s][t]}{I} \otimes_{k[s]} \frac{k[s][u]}{J} \cong \frac{k[s][t,u]}{I^e+J^e} \]
    と,中国剰余定理を用いる.

    \begin{Lemma}
        $I,J$をそれぞれ$k[s][t](=k[s,t]), k[s][u](=k[s,u])$のイデアルとする.
        この時,以下が成り立つ.
        \[ \frac{k[s][t]}{I} \otimes_{k[s]} \frac{k[s][u]}{J} \cong \frac{k[s][t,u]}{I^e+J^e} \]
        ただし,$\frac{k[s][t]}{I}, \frac{k[s][u]}{J}$はそれぞれ
        $f \mapsto f \bmod I, f \mapsto f \bmod J$で$k[s]$代数とみなす.
    \end{Lemma}
    \begin{proof}
        以下の写像を考える.
        \begin{defmap}
            {}& \frac{k[s][t]}{I} \otimes_{k[s]} \frac{k[s][u]}{J}& \to& \frac{k[s][t,u]}{I^e+J^e} \\ 
            {}& (f(s)t^m \bmod I) \otimes (g(s)u^n \bmod J)& \mapsto& f(s)g(s)t^mu^n \bmod I^e+J^e \\
            {}& h(s) \cdot (t^m \bmod I) \otimes (u^n \bmod J)& \mapedfrom& h(s) t^mu^n \bmod I^e+J^e
        \end{defmap}
        (任意の元については加法準同型として拡張する.)
        $\rightarrow$を$\phi$, $\leftarrow$を$\psi$と名付ける.
        $(f(s)t^m \bmod I) \otimes (g(s)u^n \bmod J)=f(s)g(s) \cdot (t^m \bmod I) \otimes (u^n \bmod J)$だから,
        二つがwell-definedならばこれらが互いに逆を与えることをは明らか.

        $\phi, \psi$のwell-defiendnessを占めす.
        $I,J \subset I^e+J^e$だから,$\phi$がwell-definedであることは明らか.
        問題は$\psi$のwell-defiendnessである.
        $I^e+J^e$の元は,
        \[
            \sum_{\text{finite}}(\text{element of }k[s][t,u])\cdot(\text{element of }I)
            +\sum_{\text{finite}}(\text{element of }k[s][t,u])\cdot(\text{element of }J)
        \]
        のように書ける.
        したがって,$w=c(s)t^{c_0}u^{c_1} \cdot i(s)t^{i_0}+d(s)t^{d_0}u^{d_1}\cdot j(s)u^{j_0}$の形の元の和である.
        $\psi$は加法準同型であるように定義されているから,
        $\psi(w \bmod I^e+J^e)=0$さえ示せば十分.
        $\psi(w \bmod I^e+J^e)$を計算する.
        \begin{align*}
            {}& c(s)i(s) \cdot (t^{c_0}t^{i_0} \bmod I) \otimes (u^{c_1} \bmod J)
                +d(s) j(s) \cdot (t^{d_0} \bmod I) \otimes (u^{j_0}u^{d_1} \bmod J) \\
            =&  c(s) \cdot (t^{c_0} \cdot i(s)t^{i_0} \bmod I) \otimes (u^{c_1} \bmod J)
                +d(s)  \cdot (t^{d_0} \bmod I) \otimes (u^{j_0} \cdot j(s)u^{d_1} \bmod J) \\
            =&  c(s) \cdot (0 \otimes (u^{c_1} \bmod J))
                +d(s)  \cdot ((t^{d_0} \bmod I) \otimes 0) \\
            =&  0
        \end{align*}
        よって$\psi$はwell-defined.
    \end{proof}

    これをつかって(b)を計算していく.
    \paragraph{At $y=(s-a) \in Y ~~(a \neq 0)$.}
    $\phi(y)=(\bar{t}^2-a)=(\bar{t}-\sqrt{a}) \cap (\bar{t}+\sqrt{a})$だから,
    (a)から以下が成り立つ.
    \[
        \basesp(X_y) \homeo
        f^{-1}(y)=f^{-1}V(y)=V(\phi(\I{a}))=\{ (\bar{t}-\sqrt{a}), (\bar{t}+\sqrt{a}) \}.
    \]
    $k(y)=B_y/yB_y \cong (B/y)_{\bar{y}}$だが,$B/y \cong k$は体だから$k(y)=k$.
    $X_y=\Spec A \otimes_B B/y$なので補題が使える.
    \begin{align*}
        \frac{k[s,t]}{(s-t^2)} \otimes_{k[s]} \frac{k[s,u]}{(s-a,u)}
        \cong& \frac{k[s,t,u]}{(s-t^2, s-a, u)} \\
        \cong& \frac{k[t]}{(t^2-a)} \\
        =& \frac{k[t]}{(t-\sqrt{a}) \cap (t+\sqrt{a})} \\
        \cong& \frac{k[t]}{(t-\sqrt{a})} \oplus \frac{k[t]}{(t+\sqrt{a})} \\
        \cong& k \times k
    \end{align*}
    途中で中国剰余定理を使った.
    このことから$X_y=\Spec (k \times k)$で,各点での剰余体は$k$.

    \paragraph{At $y=(s) \in Y$.}
    \begin{align*}
        \frac{k[s,t]}{(s-t^2)} \otimes_{k[s]} \frac{k[s,u]}{(s,u)}
        \cong& \frac{k[s,t,u]}{(s-t^2, s, u)} \\
        \cong& \frac{k[t]}{(t^2)} \\
    \end{align*}
    $\frac{k[t]}{(t^2)}$は$(t) \bmod (t^2)$を唯一の極大イデアルとする局所環なので,
    Ex2.3bより$\Spec \frac{k[t]}{(t^2)}$は1点空間.
    また,non-reduced schemeである.

    \paragraph{At $y=(0)=\eta \in Y$.}
    $(B/\eta)_{\eta}=B_{(0)}=k(s)$なので$k(\eta)=k(s)$.
    $k(s)=\frac{k[s,u]}{(su-1)}$より,以下のように計算できる.
    \begin{align*}
        \frac{k[s,t]}{(s-t^2)} \otimes_{k[s]} \frac{k[s,u]}{(su-1)}
        \cong& \frac{k[s,t,u]}{(s-t^2, su-1)} \\
        \cong& \frac{k[s,t,1/s]}{(t^2-s)} \\
        \cong& \frac{k(s)[t]}{(t^2-s)}
    \end{align*}
    $t^2-s$は$k(s)$係数既約多項式だから,この環は体.
    なので$X_y=\Spec \frac{k(s)[t]}{(t^2-s)}$は1点空間である.
    しかも剰余体は$k(s)=k(y)$の2次拡大体.

\section{Closed Subschemes.} %% Ex3.11 
\subsection{Closed Immersions are Stable under Base Extension.}
\subsection{* Affine Closed Subscheme of Affine Scheme is Determined by a Suitable Ideal.}
\subsection{The Smallest Subscheme Structure on a Closed Subset.}
\subsection{The Scheme-Theoretic Image of $f$.}

\section{Closed Subschemes of $\Proj S$.} %% Ex3.12 

\section{Properties of Morphisms of Finite Type.} %% Ex3.13 

\section{The Closed Points of Scheme of Finite Type over a Field.} %% Ex3.14 

\section{Geometrically Irreducible/Reduced/Integral Schemes.} %% Ex3.15 

\section{Noetherian Induction.} %% Ex3.16 

\section{Zariski Spaces.} %% Ex3.17 

\section{Constructible Sets.} %% Ex3.18 

\section{Chevalley's Theorem on Constructible Set.} %% Ex3.19 

\section{Dimension.} %% Ex3.20 

\section{$\Spec$ of D.V. Ring Gives Counterexample for Ex3.20a,d,e.} %% Ex3.21 

\section{Dimension of the Fibres of a Morphism.} %% Ex3.22 

\section{$t(V \times W)=t(V) \times_{\Spec k} t(W)$.} %% Ex3.23 

\end{document}
