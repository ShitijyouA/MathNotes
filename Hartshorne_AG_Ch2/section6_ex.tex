\documentclass[a4paper]{jsarticle}
\usepackage[all]{xy}
\usepackage{../math_note, exercise, enumitem}
\renewcommand{\thesection}{Ex6.\arabic{section}}

%% symbols of sheaves {{{
\newcommand{\shA}{\mathcal{A}}
\newcommand{\shB}{\mathcal{B}}
\newcommand{\shC}{\mathcal{C}}
\newcommand{\shD}{\mathcal{D}}
\newcommand{\shE}{\mathcal{E}}
\newcommand{\shF}{\mathcal{F}}
\newcommand{\shG}{\mathcal{G}}
\newcommand{\shH}{\mathcal{H}}
\newcommand{\shI}{\mathcal{I}}
\newcommand{\shJ}{\mathcal{J}}
\newcommand{\shK}{\mathcal{K}}
\newcommand{\shL}{\mathcal{L}}
\newcommand{\shM}{\mathcal{M}}
\newcommand{\shN}{\mathcal{N}}
\newcommand{\shO}{\mathcal{O}}
\newcommand{\shP}{\mathcal{P}}
\newcommand{\shQ}{\mathcal{Q}}
\newcommand{\shR}{\mathcal{R}}
\newcommand{\shS}{\mathcal{S}}
\newcommand{\shT}{\mathcal{T}}
\newcommand{\shU}{\mathcal{U}}
\newcommand{\shV}{\mathcal{V}}
\newcommand{\shW}{\mathcal{W}}
\newcommand{\shX}{\mathcal{X}}
\newcommand{\shY}{\mathcal{Y}}
\newcommand{\shZ}{\mathcal{Z}}
%% }}}

\newcommand{\res}{\operatorname{res}}
\newcommand{\basesp}{\operatorname{sp}}
\newcommand{\Proj}{\operatorname{Proj}}
\newcommand{\coverU}{\mathfrak{U}}
\newcommand{\OpenIn}{\text{ :: open in }}
\newcommand{\ClosedIn}{\text{ :: open in }}
\newcommand{\divible}{\!\backslash\,}
\newcommand{\ndivible}{\!\!\not\!\backslash\,}

\newcommand{\Div}{\operatorname{Div}}
\newcommand{\Cl}{\operatorname{Cl}}
\newcommand{\CaCl}{\operatorname{CaCl}}
\newcommand{\nullCaCl}{\operatorname{CaCl}^{0}}
\newcommand{\Pic}{\operatorname{Pic}}

\begin{document}
    以下での(*)とは,次のもの:
    $X$ :: integral noetherian separated (over $\Z$) scheme which is regular in codimension one.

\section{If $X$ Satisfies (*), $\Cl(X \times \proj^n) \cong \Cl(X) \times \Z$.} %% Ex6.1 
    $X'=X \times_{\Z} \proj^n_{\Z}=\proj_X^n$とおく.
    また,$S=\Z[t_0,\dots,t_n]$とし,$\proj^n=\Proj S$とみなす.

    \paragraph{$X'$ :: integral noetherian separated.}
    $X$のaffine open coverを$\{\Spec A_i\}_{i=0}^r$とすると,
    $A_i$ :: integral noetherian $\Z$-algebra.
    $\proj^n$のaffine open coverは$\{\Spec S_{(x_j)}\}_{j=0}^n$で与えられる.
    $S_{(x_j)}$もintegral noetherian $\Z$-algebra.
    したがって$R_{ij}=A_i \otimes_{\Z} S_{(x_j)}$とおくと,
    $X'$は$\Spec R_{ij}$の張り合わせであり(Thm3.3),
    $R_{ij}$ :: integral noetherian $\Z$-algebra.
    任意の$(i,j), (i',j')$について$R_{ij}, R_{i'j'}$が交わることから,
    $X'$全体でもirreducible.
    よって$X'$ :: integral noetherian scheme.
    being separated :: stable under base extensionより,
    $X'$ :: separated.

    \paragraph{$X'$ :: regular in codimension one.}
    $x=\tilde{\I{p}} \in \Spec R_{ij}$とする.
    $A_i \otimes \Z[t_0,\dots,t_n]_{(t_j)} \cong A_i[t_0,\dots,t_n]_{(t_j)}$を,
    簡単のため$j=0$とし,$R_0:=A[\{t_j\}_{j=0}^n]$とおく.
    Ati-Mac Prop3.1より,$\tilde{\I{p}} \subset (R_0)_{(t_0)}$に対応する
    $\height=1$の素イデアル$\I{p} \subset R_0$がただひとつ存在し,
    $\tilde{\I{p}}=\I{p}_{(t_0)}$となる.
    これを使って計算すると,以下のようになる.
    \[
        \shO_{X',x}
        =((R_0)_{(t_0)})_{\I{p}}
        =\left\{~ \frac{a/t_0^d}{b/t_0^e} ~\middle|~ d,e \geq 0, a \in (R_0)_d, b \in (R_0-\I{p})_e ~\right\}
        \cong A[\{t_j\}_{j=1}^n]_{\I{p}'}=:R_1.
    \]
    最後の同型は次のように与えられる.
    \begin{defmap}
        {}& (R_0)_{(t_0)}=A[\{t_j\}_{j=0}^n]_{(t_0)}& \to& R_1=A[\{t_j\}_{j=1}^n] \\
        {}& f(t_0, t_1, \dots, t_n)& \mapsto& f(1, t_1, \dots, t_n) \\
        {}& g(t_1/t_0, \dots, t_n/t_0)& \mapedfrom& g(t_1, \dots, t_n)
    \end{defmap}
    $\I{p}'$はこの写像による$\I{p}$の像である.
    $R_1$は$A$と同様にintegral noetherian ring.
    $\I{q}=\I{p}' \cap A$とおく.
    $A \subset R_1$はflat extensionだから,
    going-down theoremが成立し,$\height \I{q} \leq \height \I{p}'=1$.
    また,計算すると
    \[ (R_1)_{\I{p}'} \cong (A_{\I{q}}[\{t_j\}_{j=1}^n])_{\I{p}'}. \]
    $\height \I{q}=1$の時,
    仮定から$A_{\I{q}}=\shO_{X,\I{q}}$ :: regular local ring.
    よって$(R_1)_{\I{p}'}$はD.V.R.
    $\height \I{q}=0$すなわち$\I{q}=0$の時,
    同様に$(R_1)_{\I{p}'}$は
    体$A_{(0)}$上の多項式環の$\I{p}$における局所化だからD.V.R.

    \paragraph{Another Proof: $X'$ :: regular in codimension one.}
    $\proj^n$は$n+1$個の$\affine^n$で被覆出来るから,
    $X \times \proj^n$は$n+1$個の$X \times \affine^n$で被覆できる.
    $X \times \affine^n$はProp6.6のとおり(*)を満たすから,
    $X \times \proj^n$も(*)を満たす.

    \paragraph{Define $\Cl(X \times \proj^n) \cong \Cl(X) \times \Z$.}
    $\I{p}=(t_0) \in \Z[t_0,\dots,t_n]=S$とすると,
    Krulls Hauptidealsatzより$\height \I{p}=1$.
    したがって$Z=\pr_2^{-1}(V(\I{p}))$とおくと,
    $Z$ :: irreducible closed subset of $\codim=1$ in $X \times \proj^n$
    \footnote
        {
            $Z$ :: irreducibleは最初のparagraphと同様に示しても良いし,
            ch I, Ex3.5と同様にtopologicalに示しても良い.
            $\codim=1$はtopologicalに分かる.
            $\pr_1$は連続な全射だから,
            $Z_0 \subsetneq Z_1$が異なる2つのirreducible closed subsetの列ならば,
            $\pr_1^{-1}Z_0 \subsetneq \pr_1^{-1}Z_1$もそうである($\neq$は全射性からくる).
            したがって$\codim(\I{p}, X)=1$より$\codim(\pr_1^{-1}\I{p}, X) \leq 1$.
            $\codim(\pr_1^{-1}\I{p}, X)=0$は$V(\pr_1^{-1}\I{p})=X$と同値である.
        }.
    $U=Z^c=\pr_2^{-1}(V(\I{p})^c) \cong X \times \affine^n$だから,
    Ex3.9aとProp6.6より,$\Cl(U) \cong \Cl(X)$.
    したがってProp6.5の完全列は以下のようになる.
    \[
    \xymatrix
    {
        \Z \ar[r]^-{i}& \Cl(X \times \proj^n) \ar@{->>}[r]^-{j}& \Cl(X) \ar[r]& 0
    }
    \]
    $\Cl(X \times \proj^n) \cong \Cl(X) \times \Z$を示すには,
    $i: 1 \mapsto 1 \cdot Z$が単射であること,
    および$j: Y \mapsto Y \cap U$がsplitすることを示せば十分である.
    後者はすぐに分かる.
    実際,prime divisor $Y$がirreducible closed subschemeであることから,
    $j$は\[ Y \mapsto Y \cap U \mapsto \cl_{X \times \proj^n}(Y \cap U) \]とsplitする.

    \paragraph{$i$ :: inj.}
    $X'=X \times \proj^n, K$ :: function field of $X'$とし,$d \in \Z-\{0\}$をとる.
    示すべきことは,$dZ=(f)$を満たす$f \in K^{\times}$が存在しないこと.
    正次数の斉次元$u \in A[t_0,\dots,t_n]$をとり,
    $V=\Spec A[t_0,\dots,t_n]_{(u)}=D_+(u) \subset X'$
    において$f$がregular(poleを持たない)だとしよう.
    $D_+(u)$は基本開集合を成すから,このようにすることは可能である.
    また,$V \cap Z \neq \emptyset$,したがって$u \not \in (t_0)$とする.
    この時,$f \in A[t_0,\dots,t_n]_{(u)}, V \cap Z=V((t_0))$.
    $V \cap Z$のgeneric pointを$\eta=t_0 \cdot A[t_0,\dots,t_n]_{(u)}$
    \footnote
        {
            $\pr_1$は埋め込み写像$A \to A \otimes \Z[t_0,\dots,t_n]_{(u)}$で誘導されるから,
            $Z=\pr_1^{-1}V((t_0))=V((t_0 \otimes 1))$.
            $t_0 \otimes 1$は$A \otimes \Z[t_0,\dots,t_n]_{(u)} \cong A[t_0,\dots,t_n]_{(u)}$の同型写像で
            $t_0$へ写る.
        }
    とおくと,
    $\eta$に対応するvaluationは
    $v_{V \cap Z}(f)=\sup \{d \mid f \in \eta^d-\eta^{d+1} \}$で定まる.
    なので$v(f)=d$ならば,$f$は次のようになる.
    \[
        f=\left(\frac{t_0^m}{u}\right)^d \frac{g}{u^e}
        \mwhere
        m:=\deg u,~~ e \geq 0,~~ g \in A[t_0,\dots,t_n]_{em},~~ u, t_0^m \ndivible g
    \]
    $e=\deg g=0$の時,
    $u$の既約因数によって定まるprime divisor $U$上で$v_U(f)<0$となる.
    $e>0$の時,
    $g$の既約因数によって定まるprime divisor $G$上で$v_G(f)>0$となる.
    以上から,$dZ=(f)$となることはない.

\section{Varieties in Projective Space.} %% Ex6.2 

\section{Cones.} %% Ex6.3 

\section{$A=k[x_1,\dots,x_n,z]/(z^2-f)$ :: integrally closed.} %% Ex6.4 
    $\fchar k \neq 2$とする.
    $x_1,\dots,x_n$を$\vec{x}$と略す.
    $f \in k[\vec{x}]$ :: square-freeとし,
    $A=k[\vec{x},z]/(z^2-f)$とおく.
    また,$\bar{z}=z+(z^2-f)$とする.
    ($\bar{z}=\sqrt{f}, A=k[\vec{x}, \sqrt{f}]$と考えて良い.)
    $f$ :: square-freeより$z^2-f$ :: irreducible, $A$ :: integral domain.

    \paragraph{$K$の同定.}
    この時,$K=\Quot(A)$は$k(\vec{x})[z]/(z^2-f)$である.
    実際,$K$の元は$g,h \in A$の元によって$g/h$と表されるが,
    $z^2=f$なので,$g/h$は分母の「有理化」によって$k(\vec{x})[z]/(z^2-f)$に属すことが分かる.
    したがって$k(\vec{x})[z]/(z^2-f) \subseteq K$であり,逆の包含関係は明らか.

    \paragraph{$K/k(\vec{x})$.}
    $K$は$k(\vec{x})$上の2次式$\bar{z}^2-f$の最小分解体だから,
    $K/k(\vec{x})$は2次のガロア拡大である.
    $\Gal(K/k(\vec{x}))$は,
    $\sigma: \bar{z} \mapsto -\bar{z}$で生成される位数$2$の群.

    \paragraph{$A$ :: integral closure of $k[\vec{x}]$ in $K$.}
    $\alpha \in K$をとると,
    これは$g,h \in k(\vec{x})$を用いて$g+h\bar{z}$と書ける.
    $\alpha$の最小多項式は,
    \[ (X-\alpha)(X-\sigma(\alpha))=X^2-2gX + (g^2 - h^2 f). \]
    この多項式の各係数が$k[\vec{x}]$に属しているとしよう.
    すると,まず明らかに$g \in k[\vec{x}]$である.
    また$f$ :: square-freeより,
    $h \not \in k[\vec{x}]$ならば$h^2$の分母は$f$の因子で打ち消されず,
    $h^2 f, g^2 - h^2 f\not \in k[\vec{x}]$となる.
    よって$\alpha$ :: integral / $k[\vec{x}]$ならば$\alpha \in k[\vec{x}]$.
    逆に$\alpha \in k[\vec{x}]$ならば$g,h \in k[\vec{x}]$だから
    $\alpha$の最小多項式は$k[\vec{x}]$係数多項式になる.
    以上をまとめて$A$ :: integrally closedが分かる.

    \paragraph{系.}
    以上から,$z^2-f=0$で定まるhypersurfaceはaffine varietyとしてnormalである.
    特に,$f(x) \in k[x]$が重根を持たない$3$次多項式であるとき,
    楕円曲線$y^2=f(x)$はnormal curveである.

\section{Quadric Hypersurfaces.} %% Ex6.5 
    $k$ :: field, $\fchar k \neq 2$とし,
    \[
        f=x_0^2+\dots+x_r^2 \in k[x_0, \dots, x_n],~~~
        A(X)=k[x_0,\dots,x_n]/(f),~~~
        X=\Spec A(X)
    \]
    とおく.
    ch I, Ex3.12より,$\affine^{n+1}$の任意の
    $r$変数quadric hypersurfacesは$X$と同型である.

    \subsection{$X$ :: normal if $r \geq 2$.}
    $f=x_0^2-(-x_1^2-\dots-x_n^2)$なので,Ex6.4より$A(X)$ :: integrally closed.
    よって任意の点における$A(X)$の局所化もintegrally closedである.
    すなわち$X$ :: nornal

\section{Consider $X=\zerosp(y^2z-x^3+xz^2)$.} %% Ex6.6 

\section{For $X=\zerosp(y^2z-x^3-x^2z)$, $\nullCaCl(X) \iso \mathbf{G}_m$.} %% Ex6.7 
    $k$ :: algebraically closed field, $\fchar k \neq 2$とし,$\proj^2_k$内の曲線を考えていく.
    $f=y^2z-x^3-x^2z, X=\Proj k[x,y,z]/(f) \subset \proj_k^2$とする.
    $S(X)=k[x,y,z]/(f)$と書く.
    $X$のcodimension 1の点は,$\dim X=1$より,closed pointに他ならない.
    $X$は$Z=(0:0:1)$にnodeをもつ.

    \paragraph{$\nullCaCl(X) \cong \Cl(X-Z)$.}
    $X$のsingular pointは$Z$しかない.
    これはch I, Ex5.8をつかって確認できる.
    $X=\Proj S(X)$がnoetherian schemeであることから,
    Thm4.9より$X-Z$ :: nonsingular \& separated \& finite type.
    明らかにintegralであることと合わせれば,$X-Z$が(*)を満たすことが分かる.
    $X$全体でもintegralだから,
    $\shK_X$ :: sheaf of total quotient rings of $\shO_X$はfunction field $K$である.
    $P \in X-Z$に対するCartier Divisor $D_P$の定め方,
    $\nullCaCl(X)$の任意の元に対して,それが$D_P$と線形同値になる
    closed point $X-Z$が存在することの議論はExample6.11.4と全く同様である.

    \paragraph{$X-Z \cong \affine^1-\{0\}$.}
    $(s:t:0) \in V(z) \cong \proj^1$をとり,
    $(s:t:0)$と$Z=(0:0:1)$を結ぶ直線$sy-tx=0$と$X$の交点を計算する.
    すると$\proj^1 \to X$の写像が得られる.
    \[ (s:t) \mapsto (x:y:z)=(s(t^2-s^2):t(t^2-s^2):s^3) \]
    $(1:1), (1:-1)$はこの写像で$Z$へうつる.
    そこで以下のように置くと,isomorphismになる.
    \begin{defmap}
        {}& \proj^1-\{(1:1),(1:-1)\}& \to& X-Z \\
        {}& (s:t)& \mapsto& (s(t^2-s^2):t(t^2-s^2):s^3) \\
        {}& (x:y)& \mapedfrom& (x:y:z) 
    \end{defmap}
    $\proj^1-\{(1:1),(1:-1)\}$は
    $(s:t) \mapsto \frac{-s+t}{s+t}=u \mapsto (1-u:1+u)$
    によって$\affine^1-\{0\}$と同型である.
    したがって,結局次の同型が出来る.
    \begin{defmap}
        \phi:& \affine^1-\{0\}& \to& X-Z \\
        {}& t& \mapsto& (4(1-t)t: 4(1+t)t:(1-t)^3) \\
        {}& \frac{-x+y}{x+y}& \mapedfrom& (x:y:z) 
    \end{defmap}

    \paragraph{$\Cl(X)$の特徴.}
    $\phi(1)=P_1=(0:1:0)$とおく.
    計算すると$(x:y:z) \in X$について$P_1, (x:y:z), (x:-y:z)$が一直線上にある.
    つまり,$P_1, (x:y:z), (x:-y:z)$を零点に持つ一次式$l$が存在する.
    よって$P_1+(x:y:z)+(x:-y:z) \sim 0$が得られる.
    (TODO: Example6.10.2の$P+Q+R \sim 3P_1$は更に$3P_1=(z) \sim 0$ということで良いのか?)

    \paragraph{$\nullCaCl(X) \cong \Cl(X-Z) \cong \mathbf{G}_m$.}
    $\phi(1)=P_1$に注意する.
    計算すると,
    $\phi(t), \phi(u)$と
    $\phi(tu)$の$y$成分の符号を反転させたものが一直線上にある.
    \[ \phi(t)+\phi(u)-(\phi(tu)+P_1) \sim 0. \]
    変形して,
    \begin{align*}
        \phi(t)+\phi(u)-(\phi(tu)+P_1) &\sim 0 \\
        \phi(t)+\phi(u)-P_1 &\sim \phi(tu) \\
        (\phi(s)-P_1)+(\phi(t)-P_1) &\sim \phi(st)-P_1.
    \end{align*}
    よって,$P_1$を単位元とすれば,
    $\nullCaCl(X) \cong \Cl(X-Z) \cong \mathbf{G}_m$.

\section{Morphism of Schemes Induces Homomorphism of $\Pic$ / $\Cl$.} %% Ex6.8 

\section{(Culating the Picard Groups of ) Singular Curves.} %% Ex6.9 
    $X$ :: projective curve /$k$,
    $\tilde{X}$ :: normalization of $X$ (Ex3.8), 
    $\pi: \tilde{X} \to X$ :: projection,
    $\widetilde{\shO}_P$ :: integral closure of $\shO_P$ ($P \in X$)とする.
    p.136にあるcurve /$k$の定義から,
    $X$ :: integral, separated, finite type/$k$.
    このこととEx3.8より,$\pi$ :: finite mmorphism.

    \subsection{Show there is an exact sequence.}
    次の完全列を示す.
    \[
    \xymatrix
    {
        0 \ar[r]&
        \bigoplus_{P \in X} \tilde{\shO}_P^*/\shO_P^* \ar[r]&
        \Pic X \ar[r]^{\pi^*}& \Pic \tilde{X} \ar[r]& 0.
    }
    \]
    Prop6.15から,$\Pic X, \Pic \tilde{X}$は
    それぞれ$\CaCl X, \CaCl \tilde{X}$と同型である.
    次の写像を考える.
    \begin{defmap}
        \phi:& (\pi_*\shO_{\tilde{X}})^*/\shO_X^*& \to& \shK^*/\shO_X^* \\
        \phi_U& s+\shO_X(U)^*& \mapsto& s/1+\shO_X(U)^*
    \end{defmap}
    単元を単元に写す写像だから,これは単射.
    したがって次の完全列が得られる.
    \[
    \xymatrix
    {
        0 \ar[r]&
            (\pi_*\shO_{\tilde{X}})^*/\shO_X^* \ar[r]&
            \shK^*/\shO_X^* \ar[r]&
            \shK^*/(\pi_*\shO_{\tilde{X}})^* \ar[r]& 0
    }
    \]
    (これは$0 \to \ker \to M \to N \to \coker \to 0$という形の完全列である.)
    (TODO: global sectionをとる? $\shK^*$ :: quasi-coherentかどうか怪しい.)

\section{The Grothendieck Group $K(X)$.} %% Ex6.10 
TODO

\section{The Grothendieck Group of a Nonsingular Curve.} %% Ex6.11 
    $k$ :: algebraically closed field, $X$ :: nonsingular curve / $k$
    とする.$K(X) \cong \Pic X \oplus \Z$を示そう.

\section{The Degree of Coherent Sheaf.} %% Ex6.12 
    Ex6.11の続きと言える.
    $X$ :: complete nonsingular curveとする.
    Ex6.11より$K(X) \iso \Pic X \oplus \Z$.
    またnonsingular $\implies$ regular $\implies$ locally factorialなので
    Cor6.16より$\Pic X \iso \Cl X$.
    そこで,$\shF$ :: coherent sheaf on $X$に対する$\deg \shF$を,
    \[ \gamma(\shF) \in K(X) \isomap \Pic X \oplus \Z \to \Pic X \isomap \Cl X \xrightarrow{\deg} \Z \]
    で定める.右端の$\deg$はdegree of Weil divisorである.
    $D$ :: Weil divisorに対し,$\gamma(\shL(D))$は上の写像で$D$へ写る.
    なので,The Grothendieck Groupの定義と合わせて,以下が成立する.

    \begin{enumerate}[label=(\arabic*)]
        \item If $D$ :: divisor, $\deg \shL(D)=\deg D$.
        \item If $0 \to \shF' \to \shF \to \shF'' \to 0$ :: exact sequence, 
            then $\deg \shF=\deg \shF'+\deg \shF''$.
    \end{enumerate}
    次を示す:
    If $\shT$ is a torsion sheaf, then $\deg \shT=\sum_{P \in X} \length \shT_P$.

    $U=\Spec A \subseteq X$を任意にとり,$T$ :: torsion $A$-moduleについて
    $\shT|_U \iso \tilde{T}$であるとする.
    $\I{p} \in U$に対し,$T_{\I{p}}$は$A-\I{p}$が$\I{a}=\Ann(T)$の元を含む時$0$になる.
    したがって$\I{a} \subseteq \I{p}$の時のみ$T_{\I{p}} \neq 0$.
    そこで$V=V(\I{a})$とする.
    $\I{a} \neq (0)$かつ$X$は$1$次元だから,$V$は有限個の点のみからなる.

%    $\dim A/\I{a}=\dim V=0$と$A$ :: noetherianなのでAti-Mac Thm8.5より
%    $A/\I{a}$ :: artin ring.
%    なので$A_{\I{p}}$ :: artin local ring(?).
%    Ati-Mac Prop6.5から$T$ :: finite length module.
%    Ati-Mac Prop6.9より$\length$は加法的関数なので,
%    $0 \to T' \to T \to T'' \to 0$なる完全列について
%    $\length T_P=\lenght T'_P+\length T'_P'$.

    $\tilde{T} \in K(U)$に対応する$D_T \in \Cl U$を考える.
    Ex6.11aの構成によると,
    $D_T$のthe structure sheaf of the associated subschemeが$\tilde{T}$である.
    したがって$D_T$は以下のようになる.
    \[ D_T=\sum_{P \in V} v_P(f_P) \{P\}. \]
    ただし$f_P \in A$は$V(f_P)=\{P\} \subseteq U$を満たす.
    したがって$v_P(f_P)=\length T_P$を示せば十分.

\end{document}
