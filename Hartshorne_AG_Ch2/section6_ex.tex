\documentclass[a4paper]{jsarticle}
\usepackage[all]{xy}
\usepackage{../math_note, exercise, enumitem}
\renewcommand{\thesection}{Ex6.\arabic{section}}

\newcommand{\shA}{\mathcal{A}}
\newcommand{\shE}{\mathcal{E}}
\newcommand{\shF}{\mathcal{F}}
\newcommand{\shG}{\mathcal{G}}
\newcommand{\shH}{\mathcal{H}}
\newcommand{\shI}{\mathcal{I}}
\newcommand{\shJ}{\mathcal{J}}
\newcommand{\shO}{\mathcal{O}}
\newcommand{\res}{\operatorname{res}}
\newcommand{\basesp}{\operatorname{sp}}
\newcommand{\Proj}{\operatorname{Proj}}
\newcommand{\coverU}{\mathfrak{U}}
\newcommand{\OpenIn}{\text{ :: open in }}
\newcommand{\ClosedIn}{\text{ :: open in }}

\newcommand{\Cl}{\operatorname{Cl}}
\newcommand{\CaCl}{\operatorname{CaCl}}
\newcommand{\nullCaCl}{\operatorname{CaCl}^{\circ}}
\newcommand{\Pic}{\operatorname{Pic}}

\begin{document}
    以下での(*)とは,次のもの:
    $X$ :: integral noetherian separated (over $\Z$) scheme which is regular in codimension one.

\section{If $X$ Satisfies (*), $\Cl(X \times \proj^n) \cong \Cl(X) \times \Z$.} %% Ex6.1 
    $X'=X \times_{\Z} \proj^n_{\Z}=\proj_X^n$とおく.
    また,$S=\Z[t_0,\dots,t_n]$とし,$\proj^n=\Proj S$とみなす.

    \paragraph{$X'$ :: integral noetherian separated.}
    $X$のaffine open coverを$\{\Spec A_i\}_{i=0}^r$とすると,
    $A_i$ :: integral noetherian $\Z$-algebra.
    $\proj^n$のaffine open coverは$\{\Spec S_{(x_j)}\}_{j=0}^n$で与えられる.
    $S_{(x_j)}$もintegral noetherian $\Z$-algebra.
    したがって$R_{ij}=A_i \otimes_{\Z} S_{(x_j)}$とおくと,
    $X'$は$\Spec R_{ij}$の張り合わせであり(Thm3.3),
    $R_{ij}$ :: integral noetherian $\Z$-algebra.
    任意の$(i,j), (i',j')$について$R_{ij}, R_{i'j'}$が交わることから,
    $X'$全体でもirreducible.
    よって$X'$ :: integral noetherian scheme.
    being separated :: stable under base extensionより,
    $X'$ :: separated.

    \paragraph{$X'$ :: regular in codimension one.}
%    Prop6.6の証明を参考にする.
%    $\pr_1,\pr_2$を$X'=X \times_{\Z} \proj^n_{\Z}$から
%    それぞれ$X, \proj^n_{\Z}$への射影とする.
%    $d=\dim X$とおくと$\dim X'=d+n$(?).
%    $d+n-1=(d-1)+n=d+(n-1)$だから,
%    $x \in X'$がcodimension oneならば,
%    \begin{enumerate}[label=(\roman*)]
%        \item $I \subset X$ :: irreducible closed subset of codimension oneに対する$\pr^{-1}_1(I)$のgeneric point.
%        \item $\pr_1(x)$ :: point of codimension one in $X$ \& $\pr_2(x)$ :: generic point of $\proj^n$.
%        \item $J \subset \proj^n$ :: irreducible closed subset of codimension oneに対する$\pr^{-1}_2(J)$のgeneric point.
%        \item $\pr_1(x)$ :: generic point of $X$ \& $\pr_2(x)$ :: point of codimension one in $\proj^n$.
%    \end{enumerate}
%    のどちらかが成り立つ.
%    $X, \proj^n$ :: integralより,$\codim=0$の点はgeneric pointに限ることに注意.
%    (i)の場合を考えよう.
    $x=\I{p} \in \Spec R_{ij}$とする.
    $A_i \otimes \Z[t_0,\dots,t_n]_{(t_j)} \cong A_i[t_0,\dots,t_n]_{(t_j)}$を,
    簡単のため$A[T]_{(t)}$を書くことにする.
    Ati-Mac Prop3.1より,$\I{p} \subset A[T]_{(t)}$に対応する
    $\height=1$の素イデアル$\tilde{\I{p}} \subset A[T]$がただひとつ存在し,
    $\I{p}=\tilde{\I{p}}_{(t)}$となる.
    これを使って計算すると,以下のようになる.
    \[ \shO_{X',x}=(A[T]_{(t)})_{\I{p}} \cong A[T]_{(\tilde{\I{p}})}. \]
    さて,$A$の素イデアル$\I{q}$であって$\height \I{q}=1$を満たすものについて,
    局所化$A_{\I{q}}$は$\dim A_{\I{q}}=1$を満たすregular local ringである.
    したがってch I, Thm6.3より$A_{\I{q}}$はintegrally closedであり,
    Thm6.

\section{Varieties in Projective Space.} %% Ex6.2 

\section{Cones.} %% Ex6.3 

\section{$A=k[x_1,\dots,x_n,z]/(z^2-f)$ :: integrally closed.} %% Ex6.4 
    $\fchar k \neq 2$とする.
    $x_1,\dots,x_n$を$\vec{x}$と略す.
    $f \in k[\vec{x}]$ :: square-freeとし,
    $A=k[\vec{x},z]/(z^2-f)$とおく.
    また,$\bar{z}=z+(z^2-f)$とする.
    ($\bar{z}=\sqrt{f}, A=k[\vec{x}, \sqrt{f}]$と考えて良い.)
    $f$ :: square-freeより$z^2-f$ :: irreducible, $A$ :: integral domain.

    \paragraph{$K$の同定.}
    この時,$K=\Quot(A)$は$k(\vec{x})[z]/(z^2-f)$である.
    実際,$K$の元は$g,h \in A$の元によって$g/h$と表されるが,
    $z^2=f$なので,$g/h$は分母の「有理化」によって$k(\vec{x})[z]/(z^2-f)$に属すことが分かる.
    したがって$k(\vec{x})[z]/(z^2-f) \subseteq K$であり,逆の包含関係は明らか.

    \paragraph{$K/k(\vec{x})$.}
    $K$は$k(\vec{x})$上の2次式$\bar{z}^2-f$の最小分解体だから,
    $K/k(\vec{x})$は2次のガロア拡大である.
    $\Gal(K/k(\vec{x}))$は,
    $\sigma: \bar{z} \mapsto -\bar{z}$で生成される位数$2$の群.

    \paragraph{$A$ :: integral closure of $k[\vec{x}]$ in $K$.}
    $\alpha \in K$をとると,
    これは$g,h \in k(\vec{x})$を用いて$g+h\bar{z}$と書ける.
    $\alpha$の最小多項式は,
    \[ (X-\alpha)(X-\sigma(\alpha))=X^2-2gX + (g^2 - h^2 f). \]
    この多項式の各係数が$k[\vec{x}]$に属しているとしよう.
    すると,まず明らかに$g \in k[\vec{x}]$である.
    また$f$ :: square-freeより,
    $h \not \in k[\vec{x}]$ならば$h^2$の分母は$f$の因子で打ち消されず,
    $h^2 f, g^2 - h^2 f\not \in k[\vec{x}]$となる.
    よって$\alpha$ :: integral / $k[\vec{x}]$ならば$\alpha \in k[\vec{x}]$.
    逆に$\alpha \in k[\vec{x}]$ならば$g,h \in k[\vec{x}]$だから
    $\alpha$の最小多項式は$k[\vec{x}]$係数多項式になる.
    以上をまとめて$A$ :: integrally closedが分かる.

    \paragraph{系.}
    以上から,$z^2-f=0$で定まるhypersurfaceはaffine varietyとしてnormalである.
    特に,$f(x) \in k[x]$が重根を持たない$3$次多項式であるとき,
    楕円曲線$y^2=f(x)$はnormal curveである.

\section{Quadric Hypersurfaces.} %% Ex6.5 

\section{Consider $X=\zerosp(y^2z-x^3+xz^2)$.} %% Ex6.6 
TODO

\section{For $X=\zerosp(y^2z-x^3-x^2z)$, $\nullCaCl(X) \iso \mathbf{G}_m$.} %% Ex6.7 

\section{Morphism of Schemes Induces Homomorphism of $\Pic$ / $\Cl$.} %% Ex6.8 
TODO

\section{(Culating the Picard Groups of ) Singular Curves.} %% Ex6.9 
TODO

\section{The Grothendieck Group $K(X)$.} %% Ex6.10 
TODO

\section{The Grothendieck Group of a Nonsingular Curve.} %% Ex6.11 

\section{The Degree of Coherent Sheaf.} %% Ex6.12 
TODO

\end{document}
