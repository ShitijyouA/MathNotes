\documentclass[a4paper]{jsarticle}
\usepackage[all]{xy}
\usepackage{../math_note, exercise, enumitem}
\renewcommand{\thesection}{Ex5.\arabic{section}}

\newcommand{\shA}{\mathcal{A}}
\newcommand{\shE}{\mathcal{E}}
\newcommand{\shF}{\mathcal{F}}
\newcommand{\shG}{\mathcal{G}}
\newcommand{\shH}{\mathcal{H}}
\newcommand{\shI}{\mathcal{I}}
\newcommand{\shJ}{\mathcal{J}}
\newcommand{\shO}{\mathcal{O}}
\newcommand{\shM}{\mathcal{M}}
\newcommand{\ord}{\operatorname{ord}}
\newcommand{\res}{\operatorname{res}}
\newcommand{\basesp}{\operatorname{sp}}
\newcommand{\Proj}{\operatorname{Proj}}
\newcommand{\bfSpec}{\mathbf{Spec}}
\newcommand{\coverU}{\mathfrak{U}}
\newcommand{\OpenIn}{\text{ :: open in }}
\newcommand{\ClosedIn}{\text{ :: open in }}
\newcommand{\sidetilde}{\,\tilde{}}
\newcommand{\sidecheck}{\,\check{}}
\newcommand{\sidebar}{\,\bar{}}
\newcommand{\ccheck}[1]{#1\sidecheck\,\sidecheck\,}
\DeclareMathOperator{\sgn}{sgn}

\usepackage[scr=boondoxo,scrscaled=1.0]{mathalfa}
\newcommand{\shHom}{\mathscr{H\!o\!m}}


\begin{document}

\section{The Dual of Locally Free Module Sheaf.} %% Ex5.1 
    $(X,\shO_X)$をringed spaceとし,
    $\shE$を有限階数のlocally free $\shO_X$-moduleとする.
    また,$\shE$の双対を$\check{\shE}=\shHom_{\shO_X}(\shE, \shO_X)$で定める.
    ($\shHom$はEx1.15で定義されている.)
    $\ccheck{\shE}$も同様である.

    \begin{Lemma}
        $\shF$ :: $\shO_X$-module, $x \in X$とする.
        このとき,$x$に対して$n>0$が存在して
        \[ (\shHom_{\shO_X}(\shE, \shF))_x \iso (\shF_x)^{\oplus n} \iso \Hom_{\shO_{X,x}}(\shE_x, \shF_x). \]
    \end{Lemma}
    \begin{proof}
        Ex5.7の内容は使う.
        $U \OpenIn X$を十分小さく取れば$\shE|_U$はfree moduleになる.
        したがって以下が成り立つ.
        \begin{align*}
            {}&     (\shHom_{\shO_X}(\shE, \shF))(U) \\
            =&      \Hom_{\shO_U}(\shE|_U, \shF|_U) \\
            \iso& \Hom_{\shO_U}(\shO_U^{\oplus n}, \shF|_U) \\
            \iso& \Hom_{\shO_U}(\shO_U, \shF|_U)^{\oplus n} \\
            \iso& (\shF|_U)^{\oplus n} \\
            =& \varinjlim_{W \supseteq U} (\shF(W))^{\oplus n}
        \end{align*}
        (TODO: 4行目が怪しい)
        最後で$\bigoplus$と$\varinjlim$が可換であることを用いた.
        このことから以下を得る.
        \[
            \varinjlim_{U \ni x} \varinjlim_{W \supseteq U} (\shF(W))^{\oplus n}
            =\varinjlim_{W \ni x}(\shF(W))^{\oplus n}
            =(\shF_x)^{\oplus n}.
        \]
        あとは$\shO_{X,x}$-moduleの同型から最後の同型を得る.
        \[
            (\shF_x)^{\oplus n}
            \iso \Hom_{\shO_{X,x}}((\shO_{X,x})^{\oplus n}, \shF_x)
            \iso \Hom_{\shO_{X,x}}(\shE_x, \shF_x).
        \]
    \end{proof}
    $\shO_X$-homomorphismを構成し,
    それがstalkでmoduleの射としてisomorphismになっていることを確認する.

\subsection{$\ccheck{\shE} \iso \shE$.}
    写像$\Phi: \shE \to \ccheck{\shE}$を以下のように定める.
    \[
        (\Phi_U(s))_V(\phi)=\phi(s|_V)
        ~\mwhere~
        U,V \OpenIn X, V \subseteq U, s \in \shE(U), \phi \in \check{\shE}(V).
    \]
    これが$\shO_X$-homomorphismであることは明らか.
    $x \in X$を任意に取ると,$\Phi_x$は以下のようになる.
    \[ \Phi_x(s_x)(\phi_x)=\phi_x(s_x) \mwhere s_x \in \shE_x, \phi_x \in \check{\shE}_x \]
    補題より,$\check{\shE}_x, (\ccheck{\shE})_x$が計算できる.
    \[
        \check{\shE}_x \iso \shHom(\shE, \shO_X)_x \iso (\shE_x)^*,~~
        (\ccheck{\shE})_x
        \iso \Hom(\check{\shE}_x, \shO_{X,x})
        = (\shE_x)^{**}.
    \]
    ただし,$(\shE_x)^*$は$\shE_x$の$\shO_{X,x}$-moduleとしての双対である.
    $(\shE_x)^*$はfree moduleだから,上記の同型が成り立つ.
    上記の$\Phi_x: \shE_x \to (\ccheck{\shE})_x \iso (\shE_x)^{**}$が
    同型写像であることはよく知られている.
    よってProp1.1より,$\Phi$も同型.

\subsection{For any $\shO_X$-module $\shF$,
    $\shHom_{\shO_X}(\shE,\shF) \iso \check{\shE} \otimes_{\shO_X} \shF.$}
    写像$\Psi$を以下で定める.
    \begin{defmap}
        \Psi_U:& \check{\shE}(U) \otimes \shF(U)& \to& \shHom(\shE,\shF)(U) \\
        {}& \phi_U \otimes f& \mapsto& \big[ \shE(V) \ni s \mapsto \phi_V(s) \cdot f|_V \in \shF(V) \big]
    \end{defmap}
    ただし$U,V$は$X$の開集合で$V \subseteq U$を満たし,
    $\big[ \big]$内は$\shE(V) \to \shF(V)$の写像の定義である.
    これの$x \in X$におけるstalkは以下のようになる.
    \begin{defmap}
        \Psi_x:& \Hom(\shE_x, \shO_{X,x}) \otimes \shF_x& \to& \Hom(\shE_x,\shF_x) \\
        {}& \phi_x \otimes f_x& \mapsto& \big[ \shE_x \ni s_x \mapsto \phi_x(s_x) \cdot f_x \in \shF_x \big]
    \end{defmap}

\subsection{For any $\shO_X$-module $\shF,\shG$,
    $\Hom_{\shO_X}(\shE \otimes \shF,\shG) \iso \Hom_{\shO_X}(\shF, \shHom_{\shO_X}(\shE,\shG))$}
    $U \OpenIn X$を任意に取る.
    テンソル積の定義(普遍性)より,以下が成り立つ.
    \[ \Hom(\shE(U) \otimes \shF(U),\shG(U)) \iso \Hom(\shF(U), \Hom(\shE(U),\shG(U))) \]
    テンソル積と$\Hom$は$\shO_X(U)$-moduleとしてのものである.
    あとは
    \[ \Hom_{\shO_X(U)}(\shE(U),\shG(U)) \iso \shHom_{\shO_X}(\shE,\shG)(U) \]
    を示せば良い.
    (TODO: 主張自体が怪しい.)
    これは以下の写像で得られる.
    \begin{defmap}
        \Theta_U:& \shHom_{\shO_X}(\shE,\shG)(U) & \to& \Hom_{\shO_X(U)}(\shE(U),\shG(U))   \\
        {}& \phi:\shE|_U \to \shG|_U& \mapsto& \phi_U:\shE(U) \to \shG(U) \\
    \end{defmap}
    これは$x \in U$について$\Theta_x: \phi_x \mapsto \phi_x$を与えるから,
    同型写像.

\subsection{Projection Formula.}
    $f: (X,\shO_X) \to (Y,\shO_Y)$をringed spaceのmorphismとし,
    $\shF$を$\shO_X$-module, $\shE$を
    finite rank locally free $\shO_Y$-moduleとする.
    すると
    $f_*(\shF \otimes_{\shO_X} f^* \shE) \iso f_*(\shF) \otimes_{\shO_X} \shE$
    という自然同型がある.
    これを示す.

    米田の補題を用いて証明する.
    $\shG$を任意の$\shO_Y$-moduleとする.
    \begin{align*}
        {}&     \Hom_{\shO_Y}(\shG, f_*(\shF \otimes f^* \shE)) \\
        \iso&  \Hom_{\shO_X}(f^* \shG, \shF \otimes f^* \shE)                          & (\text{Ex1.18})\\
        \iso&  \Hom_{\shO_X}(f^* \shG, \shF \otimes f^* \ccheck{\shE})                 & (\text{a})\\
        \iso&  \Hom_{\shO_X}(f^* \shG, \shF \otimes (f^* \check{\shE})\,\check{}\,)    & (\text{?})\\
        \iso&  \Hom_{\shO_X}(f^* \shG, \shHom_{\shO_X}(f^* \check{\shE},\shF))         & (\text{b})\\
        \iso&  \Hom_{\shO_X}(f^* \shG \otimes f^* \check{\shE}, \shF)                  & (\text{c})\\
        \iso&  \Hom_{\shO_X}(f^* (\shG \otimes \check{\shE}), \shF)                    & (\text{?})\\
        \iso&  \Hom_{\shO_X}(\shG \otimes \check{\shE}, f_* \shF)                      & (\text{Ex1.18})\\
        \iso&  \Hom_{\shO_X}(\shG, \shHom(\check{\shE}, f_* \shF))                     & (\text{c})\\
        \iso&  \Hom_{\shO_X}(\shG, \ccheck{\shE} \otimes f_* \shF)                     & (\text{b})\\
        \iso&  \Hom_{\shO_X}(\shG, f_* \shF \otimes \shE)                              & (\text{a})\\
    \end{align*}
    $\check{\shE}, \ccheck{\shE}, f^* \shE$が
    finite rank locally free moduleであることは容易に分かる.
    残すは以下の2つの主張である.

    \begin{Claim}
        \[ f^* \ccheck{\shE} \iso (f^* \check{\shE})\check{}. \]
    \end{Claim}
    \begin{proof}
        $U \OpenIn X$をとる.
        \begin{align*}
            (f^* \check{\shE})\check{}
            =&\shHom(f^{-1}\check{\shE}, \shO_X) \\
            =&f^{-1}\shHom(\check{\shE}, \shO_X) \\
            =&f^{-1}\ccheck{\shE} \otimes \shO_X
        \end{align*}
        ここで$\shHom(f^{-1}*,*) \iso f^{-1}\shHom(*,*)$とした.
        TODO:どう示す.
    \end{proof}

    \begin{Claim}
        \[ f^* \shG \otimes f^* \check{\shE} \iso f^* (\shG \otimes \check{\shE}). \]
    \end{Claim}
    \begin{proof}
        \begin{align*}
            {}&     f^* \shG \otimes f^* \check{\shE} \\
            =&      (f^{-1}\shG \otimes \shO_X) \otimes (f^{-1}\check{\shE} \otimes \shO_X) \\
            \iso&  (f^{-1}\shG \otimes f^{-1}\check{\shE}) \otimes \shO_X \\
            \iso&  f^{-1}(\shG \otimes f^{-1}\check{\shE}) \otimes \shO_X \\
            \iso&  f^{-1}f^{-1}(\shG \otimes \check{\shE}) \otimes \shO_X \\
            =&      f^{-1}(\shG \otimes \check{\shE}) \otimes \shO_X \\
            =&      f^*(\shG \otimes \check{\shE})
        \end{align*}
        ここで$f^{-1}(\shF \otimes \shG) \iso (f^{-1}\shF \otimes \shG)$を用いた.
    \end{proof}

\section{Module Sheaves over the $\Spec$ of a D.V.R.} %% Ex5.2 
    $R$ :: D.V.R., $X=\Spec R, K=\Quot(R)$とおく.
    $X$は2点空間$\{ \zeta, \I{m} \}~~(\zeta=(0))$であり,
    開集合系は$\{ \emptyset, \{\zeta\}, X \}$である.

    \subsection{$\shO_X$-module $\shF$ $\leftrightarrow$ $\rho: M \otimes_R K \to L$.}
    $\shF$ :: $\shO_X$-moduleをとる.
    $\shO_X(\{\zeta\})=K,\shO_X(X)=R$
    \footnote
    {
        $\shO_X(\{\zeta\})$はaffine schemeのsheafの定義から分かる.
        つまり,$\shO_X(\{\zeta\})$の元は
        $\{\zeta\} \to \shO_{X,\zeta}=K$の写像であって
        localには$R$の元の分数で書けるものである.
        $\{\zeta\}$は1点集合だから,
        これは$\zeta \mapsto f/g \in K$なる写像全体を取れば良い.
        $\shO_X(\{\zeta\})$と$K$に集合としての全単射だけでなく
        同型もあることは自明であろう.
        一般に,点$x \in X$について$\{x\}$が開集合ならば
        $\shF(\{x\}) \iso \shF_{x}$.
    }
    だから,
    $\shF$は$K$-module $L=\shF(\{\zeta\})$と$R$-module $M=\shF(X)$と,
    以下の図式を可換にするrestriction map $\tilde{\rho}$で与えられる.
    \[
    \xymatrix
    {
        M \ar[r]_{\tilde{\rho}}& L \\
        R \ar[u]\ar[r]_{l}& K \ar[u]
    }
    \]
    ここで$l: R \to K=R_{(0)}$は標準的な局所化写像である.
    したがって$L$も$R$-moduleとみなせて,
    以下の図式にある$\rho: m \otimes x \mapsto \tilde{\rho}(m) \cdot x$が得られる.
    \[
    \xymatrix
    {
        M \otimes_R K \ar[rd]^{\rho} & {} \\
        M \ar[r]_{\tilde{\rho}} \ar[u]& L \\
        R \ar[u]\ar[r]_{l}& K \ar[u]
    }
    \hspace{2cm}
    \]
    逆に$\rho: M \otimes_R K \to L$があるとき,
    $\tilde{\rho}=\rho|_{M \otimes R}$とすれば$\tilde{\rho}: M \to L$が得られる.

    \subsection{$\shF$ :: quasi-coherent $\iff$ $\rho$ :: isomorphism.}
    \paragraph{$\implies$.}
    $\shF$ :: quasi-coherentのとき,$M:=\Gamma(X,\shF)$とすると,
    Prop5.1aから$\shF=\tilde{M}$.
    $\tilde{M}$の定義から,restriction map $\tilde{\rho}$は次のようなものである.
    \begin{defmap}
        \tilde{\rho}:& M& \to& \tilde{M}(\{\zeta\}) \\
        {}& m& \mapsto& [\zeta \mapsto m/1 \in M_{\zeta}]
    \end{defmap}
    こうして$\rho$が定まる.
    \begin{defmap}
        \rho:& M \otimes K& \to& \tilde{M}(\{\zeta\}) \\
        {}& m \otimes x& \mapsto& [\zeta \mapsto x(m/1) \in M_{\zeta}]
    \end{defmap}
    逆写像が$[\zeta \mapsto m/s] \mapsto m \otimes (1/s)$
    (この逆写像が加群準同型かつwell-defined mapであることは容易に分かる)
    で定まるからこれは同型である.

    \paragraph{$\impliedby$.}
    仮定より,
    \[
        \shF(X)=M,~~
        \shF(\{\zeta\})=L \overset{\rho}{\iso} M \otimes K \iso M_{\zeta} \iso \tilde{M}(\{\zeta\})
    \]
    なので$\shF=\tilde{M}$.

    \section{$\tilde{\square}$ and $\Gamma(X,\square)$ are Adjoint Pair.} %% Ex5.3 
    $A$ :: ring, $X=\Spec A$とする.
    この時,$\tilde{\square}$と$\Gamma(X,\square)$はadjointである.
    すなわち,任意の$M$ :: $A$-module, $\shF$ :: $\shO_X$-moduleについて,
    \[ \Hom_A(M,\Gamma(X,\shF)) \iso \Hom_{\shO_X}(\tilde{M}, \shF). \]
    
    写像$\Phi: \Hom_A(M,\Gamma(X,\shF)) \to \Hom_{\shO_X}(\tilde{M}, \shF)$を次のように定める.
    $U \OpenIn X$に対し,$\tilde{M}(U)$の元$s$は
    \begin{defmap}
        \sigma:& U& \to& M_u \\
        {}& u& \mapsto& m/s
    \end{defmap}
    のような写像である.
    この元$\sigma$を$\alpha \in \Hom_A(M,\Gamma(X,\shF))$によって以下のように写す.
    \begin{defmap}
        (\Phi(\alpha))_U:& \tilde{M}(U)& \to& \shF(U) \\
        {}& \sigma& \mapsto& \bar{\alpha}(\sigma)|_U
    \end{defmap}
    ただし$A_u$-module homomorphism $\alpha_{(u)}$を
    $m/s \mapsto \alpha(m)/s$で定めた.

    TODO:
    $A$-module homomorphism $\alpha \in \Hom_A(M,\Gamma(X,\shF))$に対し,
    $A_x$-module homomorphism $\tilde{M}_x \to \shF_x$は$m/s \mapsto \alpha(m)_x/s$で定まる.
    これを貼り合わせれば良い.

\section{The Original Definition of (Quasi-)Coherent Sheaves.} %% Ex5.4 
    \paragraph{quasi-coherent $\iff$ cokernel of free sheaves locally.}
    $\shF$ :: sheaf on $X$がquasi-coherentならば,
    任意のopen affine subset $U=\Spec A$について$\shF|_U \iso \tilde{M}$となる
    $A$-module $M$が存在する(Prop5.4).
    $M$はcokernel of free moduleとして表現できる(Ati-Mac Prop2.3の証明を参照せよ)から,
    完全列$0 \to A^m \to A^n \to M \to 0$に対してfunctor $\tilde{\square}$を用いれば
    $\shF|_U \iso \tilde{M}$はcokernel of free sheavesで表現できる事が得られる
    \footnote{Ex5.3から$\tilde{\square}$はleft adjointなのでdirect sum (colimit)と可換であることに注意.}.
    逆はfree sheafがquasi-coherentであることとProp5.7より従う.

    \paragraph{coherent $\iff$ cokernel of finite rank free sheaves locally.}
    $X$がNoetherianで$\shF$がcoherentならば,
    任意のopen affine subset $U=\Spec A$について$\shF|_U \iso \tilde{M}$となる
    finitely generated $A$-module $M$が存在する(Prop5.4).
    $\tilde{M}$がfinite rank free sheafであることはquasi-coherentの場合と同様.
    逆はfinite rank free sheafがcoherentであることとProp5.7より従う.

\section{Is $f_* \shF$ Coherent?} %% Ex5.5 
    \subsection{An Example that $\shF$ is Coherent but $f_*\shF$ is NOT Coherent.}
    材料は次の通り.
    \[ A=\C,~~ B=\C[x],~~ X=\Spec A=\Spec \C,~~ \shF=\tilde{\C}=\shO_X. \]
%    材料は次の通り.
%    \[ A=\Z,~~ B=\Z[t],~~ X=\Spec A=\Spec \Z,~~ \shF=\tilde{\Z[i]}. \]
%    ただし$t$は不定元で$i$は虚数単位である.
    明らかに$\shF$はcoherent $\shO_X$-moduleである(Example 5.2.1).
    $f: \Spec B \to \Spec A$を
    標準的埋込み$A \hookrightarrow B$から誘導されるものとすると,
    Prop5.1eより
    \[ f_* \shF \iso (\C \otimes_{\C} \C[x])~\tilde{} \iso (\C[x])~\tilde{}=\tilde{B} \]
    となる.
    $B=\C[x]$は明らかにfinitely generated $A$-moduleでない.

    \subsection{Closed Immersion is a Finite Morphism.}
    $f: X \to Y$ :: closed immersionをとる.
    $Y$のopen affine subset $U=\Spec A$をとり,
    $V=f^{-1}(U)$とすると,
    $f|_V: V \to \Spec A$はclosed immersionになっている.
    Cor5.10(あるいはEx3.11b)より
    $V \iso \Spec A/\I{a}$となるイデアル$\I{a}$が存在する.

    以上から任意のopen affine subset $U=\Spec A \subseteq Y$に対し
    $V=f^{-1}(U)=\Spec A/\I{a}$かつ$B$ :: finitely generated $A$-module
    ($1+\I{a}$で生成される)なので$f$ :: finite.

    \subsection{If $X,Y$ :: noetherian schemes, $f: X \to Y$ :: finite and $\shF$ :: coherent on $X$, then $f_*\shF$ :: coherent.}
    $f$ :: finiteから,
    任意のaffine open subset $\Spec B \subset Y$に対して
    $f^{-1}\Spec B$はaffine scheme $\Spec A$であり,
    かつ$A$ :: finitely generated $B$-moduleである.
    $U=\Spec A, V=\Spec B$としておこう.

    Prop5.4より,$\shF|_{\Spec A}=\tilde{M}$なる
    finitely generated $A$-module $M$が存在する.
    $f|_{U}: \Spec A \to \Spec B$だから
    Prop5.1から
    \[ (f_* \shF)|_V \iso f_* (\shF|_U) \iso f_* \tilde{M} \iso (M \otimes_{B} A) \check{}. \]
    (一番左の同型は任意の$V$の開集合上でのsectionを見れば良い.)
    今$M,B$は共にfinitely generated $A$-moduleだから,
    $f_* \tilde{M}$もfinitely generated $\shO_V$-module.
    $V$は任意のaffine open subsetとしていたから,
    $f_* \shF$ :: coherent.

\section{Support.} %% Ex5.6 
    \subsection{$\Supp m=V(\Ann(m))$.}
    $A$ :: ring, $M$ :: $A$-module, $X=\Spec A$とおく.
    さらに$\shF=\tilde{M}$とする.
    $m \in M=\Gamma(X,\shF)$について,
    $\Supp m=\{ \I{p} \in X \mid m_{\I{p}} \neq 0 \}$を考える.
    $\Ann m=\{a \in A \mid am=0 \}$とする.
    \[ m_{\I{p}}=0 \iff \Exists{a \in A-\I{p}} am=0 \]
    だから,$m_{\I{p}} \neq 0$となるのは
    $(A-\I{p}) \cap \Ann m=\emptyset$であるとき,
    すなわち$\Ann m \subseteq \I{p}$となっているときである.
    よって$\Supp m=V(\Ann m)$.

    \subsection{If $X$ :: Noetherian and $M$ :: Finitely Generated, then $\Supp \shF=V(\Ann M)$.}
    Prop5.1から$\shF_{\I{p}} \iso M_{\I{p}}$.
    $M$の生成元全体を$G$とすると,以下のようになる.
    \[
        M_{\I{p}} \neq 0
        \iff \Exists{g \in G} g_{\I{p}} \neq 0
        \iff \Exists{g \in G} \I{p} \supseteq \Ann g
        \iff \I{p} \supseteq \bigcap_{g \in G} \Ann g.
    \]
    $\Ann M=\bigcap_{g \in G} \Ann g$は明らか.
    よって$\Supp \shF=V(\Ann M)$.

    \subsection{The Support of a Coherent Sheaf on a Noetherian Scheme is Closed.}
    $\shF|_{U}$がfinitely generated moduleならば$\Supp \shF|_{U}$ :: closed.
    このような開集合$U$は有限個で十分だから
    $\Supp \shF$が閉集合の有限和となり,したがって閉集合である.

    \subsection{$\Gamma_{\I{a}}(M)\sidetilde \iso \mathcal{H}_Z^0(\shF)$.}

    \subsection{$\shF$ :: (Quasi-)Coherent $\implies$ $\mathcal{H}_Z^0(\shF)$ :: (Quasi-)Coherent.}

\section{A Sheaf is Locally Free $\iff$ Its Stalks are Free.} %% Ex5.7 
    $(X,\shO_X)$をnoetherian ringed spaceとし,$\shF$をcoherent sheafとする.

    \subsection{$\shF_x$ :: free $\implies$ $\shF|_U$ is free for a $x \in U \OpenIn X$.}
    $\shF$ :: coherentより,$\shF|_U=\tilde{M}$となる
    $U=\Spec A \OpenIn X$と$M$ :: finitely generated $A$-moduleが存在する.
    $M$のgeneratorが$n$個あるとすると,
    finite rank free module $A^{\oplus n}$のgeneratorを$M$のgeneratorに写す
    surjective module homomorphism $f$によってexact sequenceが出来る.
    \[
        \xymatrix
        {
            0 \ar[r]& \ker f \ar[r]& A^{\oplus n} \ar@{>->}[r]^{f}& M \ar[r]& \coker f=0 \ar[r]& 0
        }
    \]
    このexact sequenceをfunctor $\tilde{\square}$で写し$x$でのstalkをとると,
    Prop5.2, 5.1から,再びexact sequenceが得られる.
    \[
        \xymatrix
        {
            0 \ar[r]& \ker f_x \ar[r]& (A_x)^{\oplus n} \ar@{>->}[r]^{f_x}& M_x \ar[r]& \coker f_x=0 \ar[r]& 0
        }
    \]
    今$(\shF|_U)_x=\tilde{M}_x=M_x$だから,$f$の作り方から$f_x$ :: iso.
    よって$\ker f_x=0$.
    したがって十分小さな開集合$(x \in)~ V ~(\subseteq U)$をとれば$\ker f|_V=0$となる.
    $f$ :: iso on $V$ということになるから,$\Gamma(V,\shF|_V) \iso A^{\oplus n}$,
    $\shF|_V \iso (A^{\oplus n})\sidetilde$.

    \subsection{$\shF$ :: locally free $\iff$ $\shF_x$ are free $\shO_{X,x}$-modules for all $x \in X$.}
    $\implies$は定義から,$\impliedby$は(a)から明らか.

    \subsection{$\shF$ :: invertible $\iff$ $\Exists{\shG\text{ :: coherent}} \shF \otimes \shG \iso \shO_X$.}
    \paragraph{$\implies$.}
    $\shG=\check{\shF}$とおくと,Ex5.1bから
    \[ \shF \otimes \shG \iso \shHom_{\shO_X}(\shF, \shF) \iso \shO_X. \]

    \paragraph{$\impliedby$.}
    $x \in X$を任意にとり,stalkをみる.
    $\shF, \shG$ :: coherentだから,
    $\shF_x, \shG_x$はfinitely generated $\shO_{X,x}$-moduleと同型である.
    \[ M_x \otimes N_x \iso \shO_{X,x}. \]
    $\{\id{\shO_{X,x}}\}=\Hom(M_x \otimes N_x, *) \iso \Hom(M_x, \Hom(N_x, *))$だから
    $M_x \iso \shO_{X,x}$(?).
    (a)から$\tilde{M} \iso \shO_X$が得られる.
    よって$\shF$はinvertible.

\section{$\phi(x)=\dim_{k(x)} \shF_x \otimes_{\shO_x} k(x)$.} %% Ex5.8 
    $X$ :: noetherian scheme, $\shF$ :: coherent sheaf on $X$とする.
    \[ \phi(x)=\dim_{k(x)} \shF_x \otimes_{\shO_x} k(x),~~~ k(x)=\shO_x/\I{p}_x \]
    という関数を考えよう.
    Ati-Mac Ex2.1から$\shF_x \otimes_{\shO_x} k(x) \iso \shF_x/\I{p}_x \shF_x$.

    \subsection{$\phi$ :: upper semi-continuous.}
    任意の$n \in \Z$について$\phi^{-1}(\Z_{\geq n})$ :: closed in $X$を示す.
    $\shF|_U$が$A=\shO_X(U)$と置いた時finitely generated $A$-module $M$を用いて
    $\tilde{M}$と書けるような$U \OpenIn X$をとる.
    このような$U$で$X$を被覆できるから,
    $U \cap \phi^{-1}(\Z_{\geq n})$ :: closed in $U$
    を示せば十分である.

    Prop5.1から$x \in U$について$\shF_x \iso M_x$.
    完全列$0 \to \I{m}_x M_x \to M_x \to M_x/\I{m}_x M_x \to 0$と,
    $\I{m}_x=0$ in $k(x)$を考えれば,
    $\phi(x)=\dim_{k(x)} M_x$と分かる.
    $M$の最小の生成元集合を$G$とおくと,
    $\dim_{k(x)} M_x$は$g_x \neq 0$であるような$g \in G$の個数に等しい.
    そこで次の集合をとる.
    \[ \bigcup_{G_n \subseteq G} \bigcap_{g \in G_n} \Supp g \subseteq U \]
    この集合の点では$n$個以上の$g \in G$が$0$にならず,$\phi(x) \geq n$となる.
    ただし$\bigcup_{G_n \subseteq G}$は丁度$n$個の元を持つ$G$の部分集合$G_n$すべてを渡る.
    Ex5.6aより,$g \in G$について$\Supp g$ :: closedで,
    $G$は有限だから,これは閉集合である.

    \subsection{If $\shF$ :: locally free and $X$ :: connected then $\phi$ :: constant.}
    $\shF$ :: locally freeから,$U \OpenIn X$について,
    $A_U=\shO_X(U)$とおくと$\shF|_U=(A_U^{\oplus n_U})\sidetilde$となる$n_U$が存在する.
    したがって$U$においては$\phi$の値は常に$n_U$である.
    この$n_U$を$\rank \shF|_U$と書くことにする.

    さて,$U \subseteq V$ならば,
    $\shF|_U=(\shF|_V)|_U=((A_V^{\oplus n_V})\sidetilde)|_U$なので$\rank \shF|_U=\rank \shF|_V$.
    ここから一般に,
    $U \cap V \neq \emptyset$ならば$\rank \shF|_U=\rank \shF|_V$だと分かる.
    このことを元に次の同値関係を考える.
    \[ U \frown V \iff \Exists{W_1,\dots,W_s \OpenIn X} U \cap W_1, W_1 \cap W_2, \dots, W_s \cap V \neq \emptyset. \]
    $U \frown V$ならば$\rank \shF|_U=\rank \shF|_V$であることは今や明らか.
    $X$ :: connctedより,この同値関係による$X$の開集合の同値類はただひとつ.
    よって$\phi$ :: constant.

    \subsection{If $X$ :: reduced and $\phi$ :: constant then $\shF$ :: locally free.}
    $\shF|_U=\tilde{M}$となるような
    $U$ :: affine open in $X$, 
    $M$ :: finitely generated $A$-module ($A:=\shO_X(U)$)をとる.
    $X$ :: reduced noetherian schemeより$A$ :: reduced noetherian ring.
    
    点$\I{p} \in \Spec A=U$をとり,$n=\phi(\I{p})$とする.
    次の完全列を考える.
    \[
        \xymatrix
        {
            0 \ar[r]& \ker \iota_{\I{p}} \ar[r]& A_{\I{p}}^{\oplus n} \ar[r]^-{\iota_{\I{p}}}& M_{\I{p}} \ar[r]& 0.
        }
    \]
    $A^{\oplus n}$の標準的基底を$\{e_i\}_{i=1}^n$とすると,
    $\iota$は$e_i \mapsto g_i$なるものである.
    $k(\I{p})$ :: fieldは平坦だから,$\otimes_{A_{\I{p}}} k(\I{p})$は完全列を保つ.
    \[
        \xymatrix
        {
            0 \ar[r]
            & (\ker \iota_{\I{p}}) \otimes k(\I{p}) \ar[r]
            & A_{\I{p}}^{\oplus n}\otimes k(\I{p}) \ar[r]^-{\iota_{\I{p}} \otimes 1}
            & M_{\I{p}} \otimes k(\I{p}) \ar[r]& 0.
        }
    \]
    $n=\dim_{k(\I{p})} A_{\I{p}}^{\oplus n}\otimes k(\I{p})=\dim_{k(\I{p})} M_{\I{p}} \otimes k(\I{p})$と
    $\dim_{k(\I{p})}$の加法性から,
    $(\ker \iota_{\I{p}}) \otimes k(\I{p})=0$.
    変形すると$\ker \iota_{\I{p}}/\I{p}\ker \iota_{\I{p}}=0$.
    $\ker \iota_{\I{p}}$は$A_{\I{p}}$-moduleだったから,
    中山の補題より$\ker \iota_{\I{p}}=0$.

    $\iota: A^{\oplus n} \to M$を,生成元を生成元へ写す標準的全射とする.
    もし$\ker \iota \neq 0$ならば,
    $\ker \iota_{\I{p}} \neq 0$であるような$\I{p}$がとれる(by $A$ :: reduced?).
    ((a)での議論と$\phi$ :: constantから,$M$の生成元の任意の点での局所化は常にnon-zero.)
    しかし上での議論から常に$\ker \iota_{\I{p}}=0$だから,
    $\ker \iota=0$.
    よって$M$ :: free module.

\section{Quasi-Finitely Generated Graded $S$-Modules.} %% Ex5.9 

\section{Saturated Ideals and Closed Sub-Schemes.} %% Ex5.10 
    $S=A[x_0,\dots,x_n], X=\Proj S$とおく.
    既にhomogeneous ideal $I \subset S$が$X$のclosed subschemeを定めること(Ex3.12),
    逆に$X$のclosed subsechemeはこのように定まることを示した(5.16).
    homogeneous ideal $I \subset S$に対し,そのsaturationを以下で定める.
    \[ \bar{I}=\{s \in S \mid \Forall{i=0,\dots,n} \Exists{r \geq 0} x_i^r s \in I. \} \]
    $I=\bar{I}$の時,$I$はsaturated idealであると言われる.
    $S$のsaturated idealと$X$のclosed subschemeの間に一対一対応があることを示そう.

    \begin{Remark}
        $x_i^r s \in I$は$s \in (I:x_i^r), x_i^r \in (I:s)$と同値である.
        なので$I$ :: saturated idealについて次が成り立つ.
        \[ \Forall{s \in S} \lbra{\Forall{i=0,\dots,n} x_i \in \sqrt{(I:s)}}  \implies s \in I. \]
        $\implies$の左辺は$S_1 \subseteq \sqrt{(I:s)}$とも表現できる.
        あるいは次のようにも表現できる.
        \[ \bigcap_{i=0,\dots,n} \bigcup_{r>0} (I:x_i^r) \subseteq I. \]
    \end{Remark}

    \subsection{$I$ :: Homogeneous Ideal $\implies$ $\bar{I}$ :: Also.}
    $s \in \bar{I}$をとり,$s=s_u+\dots+s_v$と斉次分解する.
    今,$i=0,\dots,n$を任意に取る.
    $i$に対して次を満たす$r$が存在する: $x_i^r s \in I$.
    したがって
    \[ (x_i^r s_u)+\dots+(x_i^r s_v) \in I \]
    が成立している.
    $I$ :: homogeneousなので$x_i^r s_u,\dots,x_i^r s_v \in I$.
    よって$s_u,\dots,s_v \in \bar{I}$となる.

    \subsection{$\bar{I}_1 \iso \bar{I}_2 \iff \Proj S/I_1 \iso \Proj S/I_2$.}
    次を示す:
    \begin{enumerate}
        \item $I$ :: homogeneous idealについて$\Proj S/I \iso \Proj S/\bar{I}$.
        \item $I_1, I_2$ :: saturated homogeneous idealについて
              $\Proj S/I_1 \iso \Proj S/I_2$ならば$I_1 \iso I_2$.
    \end{enumerate}

    \begin{Remark}
        $S=k[x,y], I_1=(x), I_2=(y)$の時,
        $I_1, I_2$はsaturated idealかつ$I_1 \neq I_2$.
        しかし$\Proj S/I_1, \Proj S/I_2$はどちらもhyperplaneで同型である.
        なので,主張を$\bar{I}_1=\bar{I}_2 \iff \Proj S/I_2 \cong \Proj S/I_2$と
        理解してはいけない.
    \end{Remark}

    \paragraph{$\Proj S/I \iso \Proj S/\bar{I}$.}
    $I \subseteq \bar{I}$なので,次の全射準同型がある.
    \begin{defmap}
        \iota:& S/I& \to& S/\bar{I} \\
        {}& s+I& \mapsto& s+\bar{I}
    \end{defmap}
    $\iota$が誘導する
    $(S/I)_{(x_i+I)} \to (S/\bar{I})_{(x_i+\bar{I})}$の写像は
    明らかに全射であり,以下で示すように単射でもある.
    これはisomorphism of affine schemes
    $\Spec (S/\bar{I})_{(x_i+\bar{I})} \isomap \Spec (S/I)_{(x_i+I)}$
    を誘導し,これで被覆される$\Proj S/\bar{I} \isomap \Proj S/I$も同型である.

    \begin{Claim}
        $\iota$から誘導される
        $\phi_i: (S/I)_{(x_i+I)} \to (S/\bar{I})_{(x_i+\bar{I})}$
        は単射.
    \end{Claim}
    \begin{proof}
        $i=0,\dots,n$を一つ取る.
        \begin{defmap}
            \phi_i:& (S/I)_{(x_i+I)}& \to& (S/\bar{I})_{(x_i+\bar{I})} \\
            {}& \frac{s+I}{x_i^{\deg s}+I}& \mapsto& \frac{s+\bar{I}}{x_i^{\deg s}+\bar{I}}
        \end{defmap}
        $\frac{s+I}{x_i^{d}+I}$の像が$0$となるのは
        次が成立する時.
        \[ \Exists{r \geq 0} x_i^{r}s \in \bar{I}. \]
        $\bar{I}$の定義より,$R \geq 0$を十分大きくすると$x_i^{r+R}s \in I$となる.
        これは$\frac{s+I}{x_i^{\deg s}+I}=0$を意味する.
        よって$\ker \phi_i=0$.
    \end{proof}

    \paragraph{$\Proj S/I_1 \iso \Proj S/I_2 \implies I_1 \iso I_2$.}
    $S/I_1, S/I_2$が$S_0$-moduleとして同型であることを示す.
    $f: \Proj S/I_2 \isomap \Proj S/I_1, \phi_i=(f|_{D_+(x_i+I_1)})^{\#}$とする.
    この時,$\phi_i: (S/I_1)_{(x_i+I_1)} \isomap (S/I_2)_{(y_i+I_2)}$であり,
    $y_i+I_2 \in (S/I_2)_1$である
    \footnote
    {
        $D_+(x_i)$が$\Proj S/I_1$を被覆するから
        $\phi D_+(x_i)=D_+(y_i)$が$\Proj S/I_2$を被覆する.
        したがって$\{y_i\}$が$S/I_2$の生成元であり,それは$(S/I_2)_1$の元である.
    }.
    \begin{Claim}
        $\rho_i: S/I_1 \to (S/I_1)_{(x_i+I_1)}$を
        $s+I_1 \mapsto \frac{s+I_1}{x_i^{\deg (s+I_1)}+I_1}$で定める.
        $d \geq 0,s+I_1, t+I_1 \in (S/I_1)_d$をとる.
        \[
            \lbra{ \Forall{i=0,\dots,n} \rho_i(s+I_1)=\rho_i(t+I_2) }
                \implies s+I_1=t+I_1.
        \]
        すなわち,$\bigoplus_{0 \leq i \leq n} \rho_i$は単射である.
    \end{Claim}
    \begin{proof}
        任意の$i$について
        $\frac{s+I_1}{x_i^d+I_1}=\frac{t+I_1}{x_i^d+I_1}$となることは次と同値.
        \[ \Forall{i=0,\dots,n} \Exists{r \geq 0} x_i^r(s x_i^d-x_i^d t)=x_i^{r+d}(s-t) \in I_1. \]
        saturated idealの定義から,$s-t \in I_1$.
    \end{proof}
    $I_1$を$I_2$に,$x_i$を$y_i$に変えても同様である.
    ($s-t$の斉次分解を経由する.)
    これを$\sigma_i: S/I_2 \to (S/I_2)_{(y_i+I_2)}$としておこう.
    $\rho_i, \sigma_i$は定義の仕方から全射である.
    したがって主張と合わせて次の全単射が構成できる.
    \[
        \xymatrix
        {
            (S/I_1)_d \ar[r]^-{\bigoplus_i \rho_i}&
            \bigoplus_{0 \leq i \leq n} \{ s/x_i^d \in (S/I_1)_{(x_i+I_1)} \} \ar[r]^-{\bigoplus_i \phi_i}&
            \bigoplus_{0 \leq i \leq n} \{ t/y_i^d \in (S/I_2)_{(y_i+I_2)} \} \ar[r]^-{(\bigoplus_i \sigma_i)^{-1}}&
            (S/I_2)_d.
        }
    \]
    すなわち$S/I_1$と$S/I_2$は$S_0$-moduleとして同型である.
    $I_1, I_2$は同じ$S$の部分加群だから,$I_1 \iso I_2$.

    \subsection{The Ideal $\Gamma_*(\shI_Y)$ is saturated.}
    $i: Y \to X=\Proj S$をclosed immsersionとすると,
    $\shI_Y=\ker i^{\#} \subseteq \shO_X$.
    $\shO_X(n) \otimes \shJ_Y(d) \iso \shJ_Y(n+d)$(pp.115-116)に注意.

    Prop5.13より,$S=\Gamma_*(X,\shO_X)$,
    そこで$s \in \Gamma(X,\shO_X(n))$をとり,次が成り立つとする.
    \[ \Forall{i=0,\dots,n} \Exists{r_i \geq 0} x_i^{r_i}s \in \Gamma(X,\shJ_Y(n+r_i)). \]
    この時$s \in \Gamma(X,\shJ_Y(n))$となることを示せば良い.
    $x_i^{-r_i}$は$\Gamma(D_+(x_i),\shO_X(-r_i))=(S(-r_i))_{(x_i)}$の元だから,
    \[  x_i^{-r_i}(x_i^{r_i}s)=s \in \Gamma(D_+(x_i),\shO_X(-r_i) \otimes \shJ_Y(n+r_i)) \iso \Gamma(D_+(x_i),\shJ_Y(n)). \]
    よって$s \in \bigcap_i \Gamma(D_+(x_i),\shJ_Y(n))$.
    $\shJ_Y$のGluability Axiomを用いて,主張が得られる.

    \subsection{Saturated Homogeneous Ideals $\leftrightarrow$ Closed Subschemes of $X=\Proj S$.}
    $\rightarrow$は(b)から,
    $\leftarrow$は(c)からわかる.

\section{The Segre Embedding.} %% Ex5.11 
    $S,T$を$S_0=T_0=A$であるようなgraded ringとする.
    ($S,T$は$A$-moduleである.)
    これらのCartesian product $S \times_A T$を
    $\bigoplus_{d \geq 0} S_d \otimes_A T_d$とする.
    $X=\Proj S, Y=\Proj T$の時,$\Proj (S \times_A T)=X \times_A Y$であること,
    加えて$\shO_{\Proj (S \times_A T)}(1) \iso (\pr_1^*\shO_{X}(1)) \otimes (\pr_2^*\shO_Y(1))$
    となることを示す.

\section{Very Ample Invertible Sheaves.} %% Ex5.12 

\section{The $d$-uple Embedding.} %% Ex5.13 
    $S$ :: graded ringとし,$S_0$-algebra として$S_1$で生成されているとする.
    この$S$と正整数$d>0$に対して
    \[ S_n^{(d)}:=S_{nd},~~~ S^{(d)}:=\bigoplus_{n \geq 0} S_n^{(d)} \]
    とおく.
    $\Proj S^{(d)} \iso \Proj S$を示そう.

    仮定より$S_{1}$が$S$を$S_0$-algebraとして生成する.
    また,明らかに$S_{1}^{(d)}=S_d$が$S^{(d)}$を$S_0$-algebraとして生成する.
    そこで$g_0 \in S_1$を適当にとる.
    すると$f \in S_1$について$S_{(f)}=S_{(f f_0^{d-1})}^{(d)}$が簡単に分かる.
    \[ \frac{a}{f^n}=\frac{a \cdot f_0^{n(d-1)}}{f^n \cdot f_0^{n(d-1)}}. \]
    ここで$a, f^n \in S_n, a f_0^{n(d-1)}, f^n f_0^{n(d-1)} \in S_n^{(d)}$に注意する.
    逆に$f' \in S_1^{(d)}$をとると,
    $f \backslash f'$であるような$f \in S_1$について$S_{(f)}=S_{(f')}^{(d)}$となる.
    したがって次が分かる.
    \[ \Forall{f \in S_1, f' \in S_1^{(d)}} f \backslash f' \implies \Spec S_{(f)}=\Spec S_{(f')}^{(d)}. \]
    $S$の生成元$f \in S_1$(resp. $S^{(d)}$の生成元$f' \in S_1^{d}$)を様々に取れば
    $\Spec S_{(f)}$(resp. $\Spec S_{(f')}^{(d)}$)で
    $\Proj S$(resp. $\Spec S^{(d)}$)を被覆できる
    ($S_1$の$n$個の元の積全体で$S_1^{(d)}$は生成されるから$f'$に対応する$f$は常に存在すると考えて良い).
    よって$\Proj S \iso \Proj S^{(d)}$.

    $S_1$の$n$個の元の積$f'$をとり,$f \backslash f'$となる$f \in S_1$をとる.
    $X_{(f')}=\Spec S_{(f')}^{(d)}$上の$\shO(1)$の元
    \[ h \cdot \frac{a'}{f'^n} ~~~(a' \in S_n^{(d)}, h \in S_1^{(d)}) \]は,
    分子分母に$(f/f')^n$をかければ直ちに
    \[ h \cdot \frac{a}{f^{n}} ~~~ (a \in S_{n}, h \in S_d) \]と読み替えられる.
    よって$\shO_{\Spec S_{(f')}^{(d)}}(1)=\shO_{\Spec S_{(f)}}(d)$,
    $\shO_{\Spec S^{(d)}}(1)=\shO_{\Spec S}(d)$.

\section{The $d$-uple Embedding is Projectively Normal.} %% Ex5.14 
    これはch I, Ex3.17bで私が考察したことのSchemeにおける一般化である.

    $A$ :: ring, $S_A^r=A[x_0,\dots,x_r]$,
    $X$ :: closed subscheme of $\proj^r_A=\Proj S_A^r$とおく.
    $i: X \to \proj^r_A$を埋め込みとし,$\shI_X=\ker i^{\#}$とおく.
    さらに$n \in \Z$に対して以下のように定義する(p.50, p.117, p.118).
    \[
        S_A^r(n)=\bigoplus_{d \in \Z} (S_A^r)_{d+n},~~~
        \shO_X(n)=(S_A^r(n))\sidetilde,~~~
        \shI_X(n)=\shI_X \otimes_{\shO_X} \shO_X(n),~~~
        \Gamma_*(\shI_X)=\bigoplus_{d \in \Z} \Gamma(X,\shI_X(d))
    \]
    Ex3.12, Cor5.12より,
    $I=\Gamma_*(\shI_X), S(X)=S_A^r/I$とおくと$X \iso \Proj S(X)$となる.
    また,$X$がnormalであるとは,
    任意の点で$X$のlocal ringがintegrally closedであることで,
    $X$がprojectively normalであるとは,
    $S(X)$がintegrally closedであることである.

    以下
    $k$ :: integrally closed field,
    $X$ :: connected normal closed subscheme of $\proj_k^r$とし,
    $S=S(X)$とする.
    $X$の$d$-uple embedding(Ex5.13)が
    十分大きな$d>0$についてprojectively normalであることを示す.

    \subsection{$S$ :: domain and $S':=\bigoplus_{n \geq 0} \Gamma(X,\shO_X(n))=\bar{S}$.}
    \subsubsection{$S$ :: domain.}
    \paragraph{$X$ :: integral projective scheme.}
    $X$ :: integral schemeならば,
    任意の$f \in S_+$について$\shO_X(D_+(f))=S_{(f)}$ :: domain.
    したがって前段落より$S$ :: domainとわかる.
    なので$X$ :: integral schemeを示す.
    まず$X$ :: normalより,任意の$x \in X$について$\shO_{X,x}$ :: integral.
    なのでEx2.3aより$X$ :: reduced.
    次の段落で証明するとおり$X$ :: irreducibleもわかる.
    Prop1.1から$X$ :: integral scheme.
    (以上の証明から,
    $X$ :: normal scheme $\implies$ $X$ :: disjoint union of integral schemesが分かる.)

    \paragraph{$X$ :: irreducible.}
    $X$が二つ以上の異なるirreducible componentを持っていたとして,
    それぞれ$C_1, C_2$とする.
    $X$ :: connectedより,$C_1 \cap C_2 \neq \emptyset$であるようにとれる.
    そこで$x \in C_1 \cap C_2$をとると,
    $\shO_{X,x}$はintegrally closedになり得ない.
    実際,
    $x \in \Spec R=U$をaffine open subsetとすると,
    $U$は二つの異なるirreducible component $U \cap C_1, U \cap C_2$をもつから,
    $R$はこれらに対応する2つの極小素イデアル$\I{p}_1, \I{p}_2$をもつ.
    $x$に対応する素イデアル$\I{q} \in V(\I{p}_1) \cap V(\I{p}_2)$での
    局所化によって$R$の極小素イデアルが消えることはないから,
    結局$R_{\I{q}} \iso \shO_{X,x}$は二つの極小素イデアルをもつ.
    これは$\shO_{X,x}$ :: integral domainに反する.
    よって$X$ :: irreducible.

    \paragraph{$X$ :: integral projective scheme $\implies$ $S$ :: domain.}
    $k$ :: algebraically closed fieldなので,
    Prop4.10から,$X$はprojective variety $V$に対応する.
    $V$ :: irreducibleに注意.
    Ex2.14dから$V$のhomogeneous coordinate ringは$S$だから,
    $S$ :: domain (cf. ch I, Ex2.4).

    \subsubsection{$S'$ :: integrally closed.}

    \paragraph{$\shJ$の定義.$\shJ_x$の元.}
    $\shJ=\bigoplus_{n \geq 0} \shO_X(n)$とおくと$S'=\Gamma(X,\shJ)$とみなせる.
    点$x \in X$をとり,$\shJ_x$を考えよう.
    direct sumとstalk(どちらもdirect limit)は可換だから,
    \[ \shJ_x=\bigoplus_{n \geq 0} (\shO_X(n))_x=\bigoplus_{n \geq 0} (S(n))_x. \]
    各$(S(n))_x$は次のような集合である.
    \[
        (S(n))_x=
        (S(n))_{(\I{p}_x)}=
        \left\{ \frac{a}{f} ~\middle|~ d \geq 0, f \in (S-\I{p}_x)_d,~ a \in (S(n))_{d}=S_{d+n} \right\}.
    \]
    ただし$\I{p}_x$は点$x$に対応する$S$の斉次素イデアルである.
    $n \geq 0$だから
    \[ \shJ_x=\left\{ \frac{a}{f} \in S_{\I{p}_x}~\middle|~ f\text{ :: homogeneous},~ \ord a \geq \deg f \right\}. \]
    ここでの$\deg,\ord$は$S$に付与されているものである.
    ただし$\ord a$は$a$が持つ斉次成分の次数で最低のものである.
    $\Quot(\shJ_x)=\Quot(S)$も分かる.

    \paragraph{$\shJ_x$ :: integrally closed / Introduction.}
    $\shJ_x$ :: integrally closedであることを示すため,
    次の等式を考える.
    \[
        \left(\frac{a}{f}\right)^r+c_{r-1}\left(\frac{a}{f}\right)^{r-1}+\dots+c_0=0
        \mwhere
        a/f \in \Quot(S),~~ \{ c_i \}_{i=0}^{r-1} \subset \shJ_x.
        \leqno{(*)}
    \]
    $d=\deg a/f$とおく.
    $\shO_{X,x}=S_{(\I{p}_x)}$ :: integrally closedを利用するため,
    操作を加えて$a/f, \{c_i\}$を$0$次斉次元にしたい.
    操作は$d$の符号に依って変わってくる.

    \paragraph{$d<0$はありえない.}
    最初に$d < 0$と仮定する.
    すると$\ord c_i \geq 0$より,
    $i=r-1,\dots,0$では$\ord c_i+di>dr$が成り立つ.
    したがって左辺の$dr$次斉次成分を取り出すと
    先頭項($i=r$の項)のみになり,これは明らかに$0$でない.

    \paragraph{$d=0$の場合.}
    $a/f$が負の斉次成分を持つことはないので,$a/f$は$0$次斉次元である.
    なので係数から$0$次斉次成分を取り出しても正しい等式となる.
    (ch I, Thm3.4で同様の操作を行った.)

    \paragraph{$d>0$の場合.}
    (TODO:)
    その上で$s \in (S-\I{p}_x)_1$をとり
    \footnote
    {
        $\I{p}_x$が$S_1$全体を含むことはない.
        そうなれば$S_+ \subseteq \I{p}_x$となるが,これは$\Proj S$の定義からありえない.
        なので$s \in (S-\I{p}_x)_1$がとれる.
    },
    等式($*$)の両辺に$s^{-rd}$を両辺にかけて,
    $0$次斉次成分を取り出しても正しい等式となる.
    以上から,次の等式が得られる.
    \[
        \left(\frac{a_d}{s^df}\right)^r+
        \left(\frac{c_{r-1}}{s^d}\right)_0\left(\frac{a_d}{s^df}\right)^{r-1}+
        \dots+
        \left(\frac{c_0}{s^{rd}}\right)_0
        =0
    \]
    各項の$0$次斉次成分が$\frac{(c_i a^i)_0}{s^{rd}f^{r-1}}$でなく
    上のように取れるのは,
    $\frac{c_i}{s^{d(r-i)}}$と$\frac{a}{a^df}$の斉次成分が
    どちらも$0$以下のものしか無いことに拠る.
    $0$以下の整数同士を加えて$0$になるのは$0$同士の和のみ,
    ということである.
    
    \paragraph{$\shJ_x$ :: integrally closed / Conclusion.}
    以上の操作により,$a/f$及び係数$\{c_i\}$が$0$次斉次元になる.
    このことから$f$が斉次だと分かる.
    等式($*$)が操作後の条件を満たすとして,
    等式の両辺に$f^r$を掛けて$a^r$を移項する:
    \[ c_{r-1}a^{r-1} \cdot f+\dots+c_1a^1 \cdot f^{r-1}+c_0 \cdot f^r=-a^r. \]
    $c_i, a$は斉次元だから,左辺は斉次元である.
    加えて左辺の各項は次数が同じだから,各項が斉次元.
    したがって$f$は斉次でなくてはならない.
    以上から,$a/f \in S_{((0))}, \{c_i\} \subset S_{(\I{p}_x)}$が分かり,
    $\shO_{X,x}=S_{(\I{p}_x)}$ :: integrally closedより$f \in S-\I{p}_x$.
    まとめて,$a/f \in \shJ_x$.
    よって$\shJ_x$ :: integrally closed.

    \paragraph{$\Forall{x \in X} \shJ_x$ :: integrally closed $\implies$ $S'$ :: integrally closed.}
    Thm5.19の証明後半から,$S' \subseteq \bar{S}$.
    そこで$a/f \in \bar{S} \subset \Quot S$をとって考える.
    この時,次のような等式が成立する.
    \[
        \left(\frac{a}{f}\right)^r+c_{r-1}\left(\frac{a}{f}\right)^{r-1}+\dots+c_0=0
        \mwhere
        \{ c_i \}_{i=0}^{r-1} \subset S \subseteq S'.
    \]
    $S'=\Gamma(X,\shJ)$だから,任意の$x \in X$について$\{c_i\} \subset \shJ_x$.
    したがって$\shJ_x$ :: integrally closedより$a/f \in \shJ_x \subset S_{\I{p}_x}$.
    特に,$f$は斉次元である.
    $f$が$S$の単元でないと仮定すると,
    $f$が斉次元であることから,$f \in \I{p}_x$なる$x \in X$がとれる.
    この$x$について$a/f \not \in \shJ_x$となり,矛盾が生じる.
    よって$f$は単元でしかありえない.
    すなわち$a/f \in S \subseteq S'$.
    以上から$S' \supseteq \bar{S}$が示された.

    \subsection{$S_d=S'_d$ for all sufficiently large $d$.}
    $S$がEx5.9の仮定を満たすことは明らか.
    $\tilde{S}=\shO_X$なので,Ex5.9cから$S \approx S'=\Gamma_*(\shO_X)$.

    \subsection{$S^{(d)}$ is integrally closed for sufficiently large $d$.}
    (b)より,$d$を十分大きく取れば,$S_{nd}=S'_{nd} ~(n \geq 0)$となる
    ($S_0=k=\bar{k}=S'_0$に注意).
    なので,この$d$について,
    斉次環$S^{(d)}=\bigoplus_{n \geq 0} S_{nd}$は$S'$の部分環である.
    $S^{(d)}$の元を係数に持つ
    多項式が$\Quot(S^{(d)})$に根を持っていたとする.
    \[
        \left(\frac{f}{g}\right)^r+c_{r-1}\left(\frac{f}{g}\right)^{r-1}+\dots+c_0=0
        \mwhere
        f,g \in S^{(d)}, \{c_i\}_{i=0}^{r-1} \subset S^{(d)}.
    \]
%    斉次分解を行えば,$f, \{c_i\}_{i=0}^{r-1}$は斉次元として考えても良いことが分かる.
    $S^{(d)} \subset S'$かつ(a)から$S'$はintegrally closedなので,$h:=f/g \in S'$.
    上の等式で$(f/g)^r=h^r$を移項してみると,
    $\deg c_i$が$d$の倍数であることから,
    $\deg h^r=r \deg h$も$d$の倍数だと分かる.
    $r$はいくらでも大きくできるから,
    $r$と$d$が互いに素であるようにすれば,
    $\deg h$が$d$の倍数であることが得られる.
    以上から$h=f/g \in S^{(d)}$であり,
    $S^{(d)}$ :: integrally closed.

    \subsection{$X$ :: projectively normal $\iff$ $X$ :: normal and
        $\Gamma(\proj^r_A, \shO_{\proj^r_A}(n)) \to \Gamma(X,\shO_X(n))$ is surj.}

    $X \subseteq \proj^r_A$ :: connected closed subschemeとする.
    $X$ :: projectively normalと,
    $X$ :: normalかつ
    自然な写像$\Gamma(\proj^r, \shO_{\proj^r}(n)) \to \Gamma(X,\shO_X(n))$が
    $n \geq 0$の時全射であること,
    が同値であることを示す.

    \paragraph{$\implies$.}
    $X$ :: projectively normalの時,
    定義から$S=\bar{S}$なので,
    (a)後半から$S \iso S'=\Gamma(X,\shJ)$.
    $x \in X$とすると,$\shO_{X,x}=S_{(\I{p}_x)}=\shJ_x$であり,
    (a)後半の証明から$\shJ_x$ :: integrally closed.
    またclosed immersionの定義から全射$\shO_{\proj^r} \to \shO_{X}$が存在する.
    homomorphism of graded ringsの定義を考えれば,
    全射が$\Gamma(\proj^r, \shO_{\proj^r}(n)) \to \Gamma(X,\shO_X(n))$に遺伝することが分かる.

    \paragraph{$\impliedby$.}
    $S_A^r(n)=\Gamma(\proj^r, \shO_{\proj^r}(n)) \to \Gamma(X,\shO_X(n))=S(n)$が
    全射かつ$X$ :: normalとする.
    すると直ちに全射$\Gamma_*(\shO_{\proj^r}) \to \Gamma_*(\shO_X)$が得られる.
    Ex5.9より,全射$(\shO_{\proj^r}) \to \shO_X$が存在し,
    $\shI_X$の定義から,これの$\ker$が$I$である.
    Prop5.13より$\Gamma_*(\shO_{\proj^r}) \iso S_A^r$なので,
    $S=S_A^r/I \iso \Gamma_*(\shO_X)=S'$.
    (a)後半から,これはintegrally closed.
    ($k$ :: integrally closed fieldは(c)でのみ使われている.)

\section{Extension of Coherent Sheaves.} %% Ex5.15 
    $X$ :: noetherian scheme, $U$ :: open in $X$, 
    $\shF$ :: coherent sheaf on $U$とする.
    この時,$\shF'$ :: coherent sheaf on $X$であって
    $\shF'|_U=\shF$となるものが存在する.
    つまりNoetherian Schemeの開集合上のcoherent sheafは拡張できる.
    これをいくつかの段階に分けて証明する.

\section{Tensor Operations on Sheaves.} %% Ex5.16 
    $(X, \shO_X)$ :: ringed space, $\shF$ :: $\shO_X$-moduleとする.

    \subsection{If $\shF$ :: locally free $\shO_X$-module then 
        $T^r(\shF), S^r(\shF), \bigwedge^r(\shF)$ :: locally free.}
    Prop5.1,5.2(まとめたものがCor5.5)と,
    $M$ :: free $A$-moduleに対して
    $T^r(M), S^r(M), \bigwedge^r(M)$ :: free modulesとなることから,
    $T^r(\shF), S^r(\shF), \bigwedge^r(\shF)$ :: locally freeが分かる.

    また,それぞれの$\rank$も計算できる.
    $\rank M=n ~~(M \iso A^{\oplus n})$とする.
    \[ \rank T^r(M)=\rank \left(\bigotimes_{i=1}^r A^{\oplus n} \right)=n^r. \]
    $\rank S^r(M)$は$r$個の一次独立な元を
    $n$個の基底($A^{\oplus n}$の基底)からとる重複組み合わせの総数に等しい.
    \[ \rank S^r(M)=H_r^n=\binom{n+r-1}{r}=\binom{n+r-1}{n-1}. \]
    $\rank \bigwedge^r(M)$は$r$個の一次独立な元を
    $n$個の基底($A^{\oplus n}$の基底)からとる(重複なし)組み合わせの総数に等しい.
    \[ \rank \bigwedge^r(M)=C_r^n=\binom{n}{r}. \]

    \subsection{the multiplication map induces
        $\bigwedge^r \shF \iso (\bigwedge^{n-r} \shF)\sidecheck \otimes \bigwedge^n \shF$.}
    $M$ :: locally free $A$-moduleとし,
    $M$の基底を$x_1,\dots,x_n$とする.
    
    \paragraph{Notation.}
    $S=\{1,\dots,n\}$とし,$I \subseteq S$に対して$x_I=\bigwedge_{i \in I} x_i$とする.
    ただし,$i \in I$は小さいものから取る.
    例えば$I=\{3,2,5\}$なら$x_I=x_2 \wedge x_3 \wedge x_5$である.
    $x_{\emptyset}=1 \in A$としておく.
    また$\sgn I=\pm 1$を$x_I \wedge x_{I^c}=(\sgn I) x_S$で定める.

    \paragraph{$\to$.}
    $\bigwedge^n M=Ax_S \iso A$が分かる.
    $\bigwedge M$のmultiplication mapは次のものである.
    \begin{defmap}
        \mu:& \bigwedge^r M \otimes \bigwedge^{n-r} M& \to& \bigwedge^n M \iso A \\
        {}& x_I \otimes x_{J}& \mapsto& x_I \wedge x_{J}.
    \end{defmap}
    ただし$I,J \subset S$は$\#I=r, \#J=n-r$を満たす.
    定義より$I \cap J \neq \emptyset$ならば$\mu(x_I, x_J)=0$,
    $I \cap J = \emptyset$ならば$\mu(x_I, x_J)$は$\sgn I=\pm 1 \in A$へ写る.
    そこで,$I \subseteq S, x_I \in \bigwedge^r M$に対し,
    $\mu(x_I,*) \otimes (x_{I^c} \wedge x_I)$を$x_I$の像とする.
    以上で$\bigwedge^r M$全体からの写像が出来た.

    \paragraph{$\leftarrow$.}
    $\phi \otimes x_S \in (\bigwedge^{n-r} M)\sidecheck \otimes \bigwedge^n M$の像を次のように定める.
    \[ \sum_{I \subseteq S, \#I=n-r} (\sgn \sigma_I) \phi(x_I)x_{I^c}. \]
    ただし$S_n$は$n$次対称群である.

    \paragraph{isomorphismであること.}
    $\mu(x_I, x_J) \neq 0$となるのは$J=I^c$の時のみ.
    なので$x_I$の像$\mu(x_I,*) \otimes (x_{I^c} \wedge x_I)$の像は$(\sgn I)^2 x_I=x_I$.
    逆に$\phi \otimes x_S$の像
    $\sum_I (\sgn I) \phi(x_I)x_{I^c}$の像は,
    $\phi=\sum_I (\sgn I) \phi(x_I)\mu(x_{I^c}, *)$ゆえに$\phi \otimes x_S$.

\section{Affine Morphisms.} %% Ex5.17
    Scheme morphism $f: X \to Y$がaffine morphismであるとは,
    $\Spec A \in \coverU$ならば$f^{-1}(\Spec A)$もaffineであるような
    $Y$のaffine cover $\coverU$が存在する,
    ということである.

    \subsection{$f: X \to Y$ :: affine $\iff$ for any $\Spec A \subseteq Y$ $f^{-1}\Spec A$ :: affine.}
    $\impliedby$は明らか.$\implies$を示す.
    $\Spec A \subseteq Y$をとり,$U=\Spec A, V=f^{-1} \Spec A$とおく.
    $f|_V: V \to \Spec A$だけを考えれば十分なので$f: X \to \Spec A=Y$とする.
    Ex3.1の解答で証明した``Nike’s Lemma"(とEx2.13b; $\basesp(Y)$ :: quasi-compact)を使うと,
    $\bigcup_{i=1}^r D_A(a_i)=Y$かつ$f^{-1} D_A(a_i)$ :: affineとなる
    $\{a_i\}_{i=1}^r \subset A$が存在することが分かる.

    $f^{-1} D_A(a_i)=\Spec B_i$としよう.
    さらに$\phi=f^{\#}_Y: A=\shO_Y(Y) \to \shO_X(X)=B$とする.
    Ex3.1で証明した別の補題``Preimage of POS is POS"を使うと,
    $f^{-1} D_A(a_i)=D_{B_i}(b_i)$となる$b_i \in B_i$が存在する事が分かる.
    $\bigcup_{i=1}^r D_A(a_i)=Y$より$(a_1,\dots,a_r)=(1)=A$だから,
    $(\phi(a_1), \dots,\phi(a_r))=(1)=B$.
    ``Preimage of POS is POS"の証明($b_i$の定め方)から,
    $X_{\phi(a_1)}=D_{B_i}(b_i)$ :: affine.
    以上からEx2.17bより$f^{-1}\Spec A$ :: affine.

    \subsection{An affine morphism is quasi-compact and separated. finite $\implies$ affine}
    finite morphismがaffine morphismであることは定義から明らか.
    affine morphismならばquasi-compact(定義はEx3.2)
    であることはEx2.13bから分かる.
    affine morphismならばseparatedであることは,
    Cor4.6fとProp4.1から.
    
    \subsection{$\bfSpec \shA$.}
    $Y$ :: scheme, $\shA$ :: quasi-coherent sheaf of $\shO_Y$-algebraとする.
    この時,以下のような$X$ :: scheme, $f : X \to Y$が一意に存在する:
    任意のaffine open subset $U \subseteq V \subseteq Y$について,
    $f^{-1}U \iso \Spec \shA(U)$であり,
    $f^{-1}U \hookrightarrow f^{-1}V$が
    restrction map $\shA(V) \to \shA(U)$に対応する.
    この$X$を$\bfSpec \shA$で表す.

    \paragraph{Construct $X$.}
    Gluing Lemma(Ex2.12)を用いて$X$を構成する.
    貼り合わせるのは$\Spec \shA(\Spec R)$である.
    $V, W$ :: affine open subset of $Y$に対し,
    $U_{V}=\Spec \shA(V), U_{W}=\Spec \shA(W)$とする.
    2つのrestriction map $\res_V, \res_W: \shA(V), \shA(W) \to \shA(V \cap W)$から
    誘導される写像$i_V, i_W: \Spec \shA(V \cap W) \to U_V, U_W$をとる.
    $U_{V,W}=\im i_V, U_{W,V}=\im i_W$とおくと,
    open immersionの定義に沿って$U_{V,W}, U_{W,V}$ :: open in $U_V, U_W$が確かめられる.
    ($\shO_{U_V}|_{U_{V,W}} \iso \shO_{\Spec \shA(V \cap W)}$となる.)
    以上の設定でgluingが出来ることは明らか.
    $f$はEx2.12にあるisomorphism $(\psi_V)^{-1}: \psi_V(U_V) \to U_V$を貼り合わせれば良い.

    \paragraph{$X$ satisfies the additional condition.}
    $V, W$ :: affine open subset of $Y$をとる.
    $V \subseteq W$の時,
    $f^{-1}V \to f^{-1}W$は$i_W: \Spec \shA(V \cap W)=\Spec \shA(V) \to \Spec \shA(W)$に対応し,
    $i_W$は定義から$\res_W$に対応する.

    \paragraph{Uniqueness.}

    \subsection{$f: X \to Y$ :: affine $\iff$
        $\shA \iso f_* \shO_X$ :: quasi-coherent $\shO_Y$-algebra and $X \iso \bfSpec \shA$.}

    \paragraph{$\implies$.}
    $\shA=f_* \shO_X$とおく.$U=\Spec A \subseteq Y$とすると,
    $f^{-1}U$ :: affineだから$f^{-1}U=\Spec \shO_X(f^{-1}U)=\Spec \shA(U)$.
    また,$\Spec \shA(U)=f^{-1}U \hookrightarrow f^{-1}V=\Spec \shA(V)$は
    直ちに$\res_V^U: \shA(V) \to \shA(U)$を誘導する.
    したがって(c)より$X \iso \bfSpec \shA$.
    また,任意の$U$ :: affine open subset in $Y$について,
    $\shA(U)$は$f^{\#}_U: \shO_Y(U) \to f_*\shO_X(U)=\shA(U)$によって
    $\shO_Y(U)$-algebraとみなすことが出来る.
    よって$\shA$ :: quasi-coherent $\shO_Y$-algebra.

    \paragraph{$\impliedby$.}
    $\shA$ :: quasi-coherent $\shO_Y$-algebraならば,
    (c)から$\bfSpec \shA$が存在する.
    $\bfSpec \shA$の定義から$f: \bfSpec \shA \to Y$はaffine.
    $f^{\#}: \shO_Y \to f_*\shO_X$を考えると,
    $U=\Spec A \subseteq Y$について$f^{-1}U \iso \Spec \shA(U)$だから
    $\shO_X(f^{-1}U)=(f_*\shO_X)(U) \iso \shA(U)$.
    このような$U$で$Y$を被覆できるから,$\shA \iso f_*\shO_X$.

    \subsection{$\{$ quasi-coherent $\shO_X$-modules $\} \leftrightarrow$ $\{$ quasi-coherent $\shA$-modules $\}$.}
    $f: X \to Y$をaffine morphismとし,$\shA=f_*\shO_X$とおく.
    (b)とProp5.8cより,
    $\shF$ :: quasi-coherent $\shO_X$-moduleについて
    $f_* \shF$ :: quasi-coherent $\shA$-moduleが得られる.

    逆に,$\shM$ :: quasi-coherent $\shA$-moduleをとる.
    $U=\Spec A$ :: open in $Y$をとると,
    $f^{-1}U$ :: affineなので$f^{-1}U=V=\Spec B$とする.
    この時,$\shM|_U \iso \tilde{M}$となる$B$-module($B=\shA(U)$)が存在する.
    $\phi=f^{\#}_U: A \to B$によって$M$を$A$-moduleとみなしたものを${}_A M$と書くことにして,
    $\tilde{M}|_U=({}_A M)\sidetilde$とおく.
    こうして$\tilde{\shM}$を構成する
    \footnote
    {
        つまるところ$\tilde{\shM}=f^* \shM$であるが,
        上の構成は$\shM$が$\shO_X$-moduleでないという点でProp5.2の内容と異なる.
    }.
    $({}_A M) \otimes_A B \iso M$が容易にわかるから,
    Prop5.2から$f_*(\tilde{\shM}|_U) \iso \shM|_U$.
    ${}_A(M \otimes_A B) \iso M$も同様にわかるから,
    $\widetilde{f_*(\shM|_U)} \iso \shM|_U$.
    $\tilde{\square}$と$f_*$がfunctorialであることはProp5.2で述べられているとおりである.
    以上で主張が示せた.

\section{Vector Bundles.} %% Ex5.18 

\end{document}
