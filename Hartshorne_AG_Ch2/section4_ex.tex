\documentclass[a4paper]{jsarticle}
\usepackage[all]{xy}
\usepackage{../math_note, exercise, enumitem}
\renewcommand{\thesection}{Ex3.\arabic{section}}

\newcommand{\shO}{\mathcal{O}}
\newcommand{\Sch}{\mathbf{Sch}}
\newcommand{\Var}{\mathbf{Var}}
\newcommand{\Rings}{\mathbf{Rings}}
\newcommand{\red}[1]{#1_{\text{red}}}
\newcommand{\basesp}{\operatorname{sp}}
\newcommand{\res}{\operatorname{res}}
\newcommand{\Rat}{\operatorname{Rat}} %% rational points
\newcommand{\Proj}{\operatorname{Proj}}
\newcommand{\CoverU}{\mathfrak{U}}
\newcommand{\OpenIn}{\text{ :: open in }}
\newcommand{\ClosedIn}{\text{ :: open in }}
\newcommand{\mnot}{/\hspace{-1.2ex}}

\begin{document}
\section{Finite Morphism is Proper.} %% Ex4.1 
    $f: X \to Y$をfiniteだとする.
    Cor4.8fとEx3.4より,$X,Y$がaffine schemeである場合について調べれば十分である.

    $X=\Spec A, Y=\Spec B$とすると,$A$ :: finitely generated $B$-module.
    なので特に$X$ :: Noetherian scheme.
    任意の$R$ :: valuation ringをとり,$K=\Quot R$とする.
    今,以下の可換図式が成り立っているとする.
    \[
    \xymatrix
    {
        \Spec K \ar[d] \ar[r]& \Spec A \ar[d]^-{f} \\
        \Spec R \ar[r] & \Spec B
    }
    \]
    これに対応して,以下の環の可換図式が成り立つ.
    Prop2.3より,二つの可換図式は一対一に対応している.
    \[
    \xymatrix
    {
        K  & A \ar[l]^-{u} \\
        R  \ar@{^{(}->}[u]& B \ar[u]_-{\phi} \ar[l]_-{v}
    }
    \]
    $A$ :: integral / $B$から$v(B) \subseteq u(A)$ :: integral ring extensionが得られる
    \footnote
    {
        $a \in A$をとると,$a^n+\phi(b_{n-1})a^{n-1}+\dots+\phi(b_0)=0$となる$n>0$と$b_i \in B$が存在する.
        両辺を$u$で写すと,$u \circ \phi=v$より$u(a)^n+v(b_{n-1})u(a)^{n-1}+\dots+v(b_0)=0$.
    }.
    $v(B) \subseteq R$と合わせて,$u(A)(\subseteq K)$ :: initegral / $R$.
    $R$が付値環であることから,$R$は$K$上整閉.
    よって$u(A) \subseteq R$.このことから$u: A \to R$の存在が得られる.
    さらに,$R \to K$が単射であることからこのような射はただひとつ.
    図式の一対一対応から,$\Spec R \to \Spec A$の射がただひとつ存在することがわかった.

\section{ } %% Ex4.2 
    $U$ :: dense in $X$とし,以下の可換図式で$f,g$ :: $S$-morphismは$f|_U=g|_U$を満たすとする.
    \[
    \xymatrix
    {
        U \ar@{^{(}->}[d]& {} & {} \\
        X \ar[r]^-{h} \ar[d]_-{f}\ar[rrd]_(0.15){g}& \ar[ld] Y \times_S Y \ar[rd] &\ar@{=}[lld]\ar@{=}[d] \ar[l]_-{\Delta} Y \\
        Y \ar[r]& S & \ar[l]Y
    }
    \]
    $U \to X \to Y=Y \to Y \times_S Y$とたどると,$\Delta(f(U))=h(U)$が得られる.
    $f(U) \subseteq Y$から$h(U) \subseteq \Delta(Y)$.
    さらに次の計算から$h(X) \subseteq \Delta(Y)$が得られる.
    \[ h(X)=h(\cl_X(U)) \subseteq \cl_Y(h(U)) \subseteq \cl_Y(\Delta(Y))=\Delta(Y). \]
    最後の等号で$Y$ :: separated / $S$を用いた.
    可換図式にある$Y \times_S Y \to Y$の射を$\pr_1, \pr_2$とする.
    $h(X) \subseteq \Delta(Y)$から以下が得られる.
    \[
        \Forall{x \in X} \Exists{y \in Y}
        f(x)=\pr_1 \circ h(x)=\pr_1 \circ \Delta(y)=y=\pr_2 \circ \Delta(y)=\pr_2 \circ h(x)=g(x).
    \]
    よってtopological spaceの射として$f=g$.

    さらにschemeの射として$f=g$であることを示す.
    \begin{Claim}
        $V$ :: open in $Y$を任意に取る.
        $\bar{V}=f^{-1}V \cap U=g^{-1}V \cap U(\neq \emptyset)$とする.
        任意の$s \in \shO_Y(V)$に対し$s|_{\bar{V}}=0$ならば$s=0$.
    \end{Claim}
    $f|_U=g|_U$から$(f^{\#}(s)-g^{\#}(s))|_{\bar{V}}=0$が直ちに得られる.
    なので,この主張が示されれば$f^{\#}(s)-g^{\#}(s)=0$すなわち$f^{\#}=g^{\#}$が得られる.
    \begin{proof}
        $V$ :: affineの場合に調べれば十分なので$V=\Spec A$とする.
        $\I{p} \in \bar{V}$を任意に取ると,$s|_{\bar{V}}=0$より$s_{\I{p}}=0$.
        これは$s=0$ in $A_{\I{p}}$を意味する.
        したがって次が成り立つ.
        \[ \Exists{t \not \in \I{p}} st=0 \in \I{p}. \]
        よって$s \in \I{p}$,$\I{p} \in V(s)$となる.
        $\I{p} \in \bar{V}$は任意にとっていたので$\bar{V} \subseteq V(s)$.
        $\bar{V}$は$V=\Spec A$でdenseだから,両辺の閉包をとって$V=V(s)$.
        すなわち$s=0$.
    \end{proof}

\section{$X$ :: Separated over an Affine Scheme $S$.} %% Ex4.3 
    $S$ :: affine scheme, $X$ :: separated scheme /$S$,
    $U, V$ :: affine open subscheme of $X$とする.
    以下がfiber productであれば,主張が示せる.
    \[
    \xymatrix
    {
        U \cap V \ar[d]\ar[r]& X \ar[d]^-{\Delta} \\
    U \times_S V \ar[r]& X \times_S X
    }
    \]
    実際,$\Delta$ :: closed immersionとEx3.11aより
    $U \cap V \to U \times_S V$はclosed immersion.
    $U \times_S V$ :: affineとEx3.11bより$U \cap V$ :: affine.

\section{"The Image of a Proper Scheme is Proper."} %% Ex4.4 

\section{ } %% Ex4.5 

\section{$f$ :: proper morphism of affine varieties/$k$. Then $f$ :: finite.} %% Ex4.6 
    $f: X \to Y$を考える.
    $X,Y$ :: affine variety / $k$より,
    $X,Y$は$A, B$ :: affine domain / $k$を用いて$X=\Spec A, Y=\Spec B$と書ける.
    $f$から誘導される環準同型を$\phi: B \to A$とする.
    $A, B$ :: affine domain / $k$から特に$X, Y$はNoetherianである.
    また,$f$ :: finite typeより$A$ :: finitely generated $B$-algebra.
    よって,$f$ :: finiteであるためには$\phi$ :: integral
    すなわち$A$ :: integral / $\phi(B)$を示せば十分である.

    $K=\Quot(\phi(B))$とし,
    $R$を$\phi(B) \subseteq R \subset K$であるような任意のvaluation ringとする.
    Thm4.7から以下の可換図式が得られる.
    \[
    \xymatrix
    {
        K  & A \ar@{_{(}->}[l] \ar[ld]\\
        R  \ar@{^{(}->}[u]& B \ar[u]_-{\phi} \ar[l]
    }
    \]
    $A \to R \to K=A \to K$かつ右辺が埋め込みであることから$A \to R$は埋め込みである.
    すなわち$(\phi(B) \subseteq )A \subseteq R$.
    $R$のとり方とThm4.11より,$A \subseteq \overline{\phi(B)}$.
    よって$A$ :: integral / $\phi(B)$.

\section{Schemes Over $\R$} %% Ex4.7 

\section{Let $\wp$ :: Property of Morphisms of Schemes} %% Ex4.8 

\section{Composition of Projective Morphisms is Projective} %% Ex4.9 

\section{Chow's Lemma.} %% Ex4.10 

\section{ } %% Ex4.11 

\section{Examples of Valuation Rings.} %% Ex4.12 

\end{document}
