\documentclass[a4paper]{jsarticle}
\usepackage[all]{xy}
\usepackage{../math_note, exercise, enumitem}
\renewcommand{\thesection}{Ex4.\arabic{section}}

\newcommand{\shO}{\mathcal{O}}
\newcommand{\Sch}{\mathbf{Sch}}
\newcommand{\Var}{\mathbf{Var}}
\newcommand{\Rings}{\mathbf{Rings}}
\newcommand{\red}[1]{#1_{\text{red}}}
\newcommand{\basesp}{\operatorname{sp}}
\newcommand{\res}{\operatorname{res}}
\newcommand{\Rat}{\operatorname{Rat}} %% rational points
\newcommand{\Proj}{\operatorname{Proj}}
\newcommand{\CoverU}{\mathfrak{U}}
\newcommand{\OpenIn}{\text{ :: open in }}
\newcommand{\ClosedIn}{\text{ :: open in }}
\newcommand{\mnot}{/\hspace{-1.2ex}}

\begin{document}
\section{Finite Morphism is Proper.} %% Ex4.1 
    $f: X \to Y$をfiniteだとする.
    Cor4.8fとEx3.4より,$X,Y$がaffine schemeである場合について調べれば十分である.

    $X=\Spec A, Y=\Spec B$とすると,$A$ :: finitely generated $B$-module.
    なので特に$X$ :: Noetherian scheme.
    任意の$R$ :: valuation ringをとり,$K=\Quot R$とする.
    今,以下の可換図式が成り立っているとする.
    \[
    \xymatrix
    {
        \Spec K \ar[d] \ar[r]& \Spec A \ar[d]^-{f} \\
        \Spec R \ar[r] & \Spec B
    }
    \]
    これに対応して,以下の環の可換図式が成り立つ.
    Prop2.3より,二つの可換図式は一対一に対応している.
    \[
    \xymatrix
    {
        K  & A \ar[l]^-{u} \\
        R  \ar@{^{(}->}[u]& B \ar[u]_-{\phi} \ar[l]_-{v}
    }
    \]
    $A$ :: integral / $B$から$v(B) \subseteq u(A)$ :: integral ring extensionが得られる
    \footnote
    {
        $a \in A$をとると,$a^n+\phi(b_{n-1})a^{n-1}+\dots+\phi(b_0)=0$となる$n>0$と$b_i \in B$が存在する.
        両辺を$u$で写すと,$u \circ \phi=v$より$u(a)^n+v(b_{n-1})u(a)^{n-1}+\dots+v(b_0)=0$.
    }.
    $v(B) \subseteq R$と合わせて,$u(A)(\subseteq K)$ :: initegral / $R$.
    $R$が付値環であることから,$R$は$K$上整閉.
    よって$u(A) \subseteq R$.このことから$u: A \to R$の存在が得られる.
    さらに,$R \to K$が単射であることからこのような射はただひとつ.
    図式の一対一対応から,$\Spec R \to \Spec A$の射がただひとつ存在することがわかった.

\section{ } %% Ex4.2 
    $U$ :: dense in $X$とし,以下の可換図式で$f,g$ :: $S$-morphismは$f|_U=g|_U$を満たすとする.
    \[
    \xymatrix
    {
        U \ar@{^{(}->}[d]& {} & {} \\
        X \ar[r]^-{h} \ar[d]_-{f}\ar[rrd]_(0.15){g}& \ar[ld] Y \times_S Y \ar[rd] &\ar@{=}[lld]\ar@{=}[d] \ar[l]_-{\Delta} Y \\
        Y \ar[r]& S & \ar[l]Y
    }
    \]
    $U \to X \to Y=Y \to Y \times_S Y$とたどると,$\Delta(f(U))=h(U)$が得られる.
    $f(U) \subseteq Y$から$h(U) \subseteq \Delta(Y)$.
    さらに次の計算から$h(X) \subseteq \Delta(Y)$が得られる.
    \[ h(X)=h(\cl_X(U)) \subseteq \cl_Y(h(U)) \subseteq \cl_Y(\Delta(Y))=\Delta(Y). \]
    最後の等号で$Y$ :: separated / $S$を用いた.
    可換図式にある$Y \times_S Y \to Y$の射を$\pr_1, \pr_2$とする.
    $h(X) \subseteq \Delta(Y)$から以下が得られる.
    \[
        \Forall{x \in X} \Exists{y \in Y}
        f(x)=\pr_1 \circ h(x)=\pr_1 \circ \Delta(y)=y=\pr_2 \circ \Delta(y)=\pr_2 \circ h(x)=g(x).
    \]
    よってtopological spaceの射として$f=g$.

    さらにschemeの射として$f=g$であることを示す.
    \begin{Claim}
        $V$ :: open in $Y$を任意に取る.
        $\bar{V}=f^{-1}V \cap U=g^{-1}V \cap U(\neq \emptyset)$とする.
        任意の$s \in \shO_Y(V)$に対し$s|_{\bar{V}}=0$ならば$s=0$.
    \end{Claim}
    $f|_U=g|_U$から$(f^{\#}(s)-g^{\#}(s))|_{\bar{V}}=0$が直ちに得られる.
    なので,この主張が示されれば$f^{\#}(s)-g^{\#}(s)=0$すなわち$f^{\#}=g^{\#}$が得られる.
    \begin{proof}
        $V$ :: affineの場合に調べれば十分なので$V=\Spec A$とする.
        $\I{p} \in \bar{V}$を任意に取ると,$s|_{\bar{V}}=0$より$s_{\I{p}}=0$.
        これは$s=0$ in $A_{\I{p}}$を意味する.
        したがって次が成り立つ.
        \[ \Exists{t \not \in \I{p}} st=0 \in \I{p}. \]
        よって$s \in \I{p}$,$\I{p} \in V(s)$となる.
        $\I{p} \in \bar{V}$は任意にとっていたので$\bar{V} \subseteq V(s)$.
        $\bar{V}$は$V=\Spec A$でdenseだから,両辺の閉包をとって$V=V(s)$.
        すなわち$s=0$.
    \end{proof}

\section{$X$ :: Separated over an Affine Scheme $S$.} %% Ex4.3 
    $S$ :: affine scheme, $X$ :: separated scheme /$S$,
    $U, V$ :: affine open subscheme of $X$とする.
    以下がfiber productであれば,主張が示せる.
    \[
    \xymatrix
    {
        U \cap V \ar[d]\ar[r]& X \ar[d]^-{\Delta} \\
    U \times_S V \ar[r]& X \times_S X
    }
    \]
    実際,$\Delta$ :: closed immersionとEx3.11aより
    $U \cap V \to U \times_S V$はclosed immersion.
    $U \times_S V$ :: affineとEx3.11bより$U \cap V$ :: affine.

\section{"The Image of a Proper Scheme is Proper."} %% Ex4.4 

\section{Center of Valuation Ring of $K/k$.} %% Ex4.5 
    $X$ :: integral scheme of finite type over a field $k$,
    $K$ :: function field of $K$とする.
    特に,$f: X \to \Spec k$がfinite typeであるとしておく.
    また,$X$はfinitely generated $k$-algebraの
    $\Spec$で被覆されるからNoetherianである.

    $K/k$のvaluation ring $R$が$x \in X$をcenterに持つとは,
    $R$が$\shO_{X,x}$をdominateするということである.
    言い換えれば,injection $\shO_{X,x} \hookrightarrow R$が存在し,
    これがlocal ring homomorphismであるということである.

    \subsection{If $X$ :: separated/$k$ then any valuation of $K/k$ has at most one center.}
    $R$を$K/k$の任意のvaluation ringとする.
    以下の可換図式を考える.
    \[
    \xymatrix
    {
        \Spec K \ar[d] \ar[r]& X \ar[d]^-{f} \\
        \Spec R \ar[r] & \Spec k
    }
    \]
    まず,$f$は与えられたものである.
    $\Spec K \to \Spec R$は包含写像$R \subseteq K$(特に単射)
    から得られるから,
    \[ \Spec K \ni (0) \mapsto (0) \in R. \]
    $\Spec R \to \Spec k$も同様.
    $\Spec K \to X$は$(0) \in \Spec K$を
    $X$のgeneric point $\zeta$に写すものとする.
    実際にこのような射が存在することはLemma4.4による.

    $X$ :: Noetherian schemeだから,
    この時,図式を可換にする$\Spec R \to X$の射$i$が高々一つある(Thm4.3).
    \[
        \xymatrix@R=5pt
        {
            \Spec K \ar[r]& \Spec R \ar[r]^-{i}& X \\
            (0) \ar@{|->}[r]& (0) \ar@{|->}[r]& \zeta \\
            (0) \ar@{=}[u] \ar@{|->}[rr]& & \zeta \ar@{=}[u]
        }
    \]
    可換性から,$i$は$(0) \subseteq R$を$\zeta$へ写す.
    Lemma4.4より,
    $R$のもうひとつの点(極大イデアル)$x$について,
    $R$は$\cl_X(\{\zeta\})=X$上の点$i(x)$における
    local ring $\shO_{X,i(x)}$をdominateする.
    再びLemma4.4から,
    逆に$R$が$\shO_{X,x'}$をdominateするならば,
    $x$を$x'$に写すことで$i$が得られる.
    よって$R$のcenterと$i$は一対一に対応し,
    たかだか一つしか無い.

    \subsection{If $X$ :: proper/$k$ then any valuation of $K/k$ has just one center.}
    この場合,$f$ :: finite typeかつ$X$ :: Noetherianなので,Thm4.7が成立する.
    そしてThm4.7から$i$は丁度一つある.
    ほかは(a)と全く同じ.

    \subsection{The Converses of (a) and (b).}
    Ex2.7とLemma4.4を使うのだとは思う.

    \subsection{Generalization of ch I, 3.4a.}
    $X$ :: proper/$k$かつ$k$ :: algebraically closedとする.
    この時$G:=\Gamma(X, \shO_X)=k$であることを示す.
    $a \in k \implies a \in G$
    ($k \subseteq G$)は明らかなので,
    $a \not \in k \implies a \not \in G$を示せば十分である.
    そのために,$a \not \in k$かつ$a \in G$となる元が存在すると仮定しよう.

    $k[a^{-1}] \subseteq R \subsetneq K$なるvaluation ring $R$すべての共通部分は,
    Thm4.11より$K$における整閉包$\overline{k[a^{-1}]}$である.
    下で示すように,これは$a$を含まない.
    したがって$a^{-1} \in R, a \not \in R$であるようなvaluation ring $R$が存在する.
    $a^{-1}$は$R$の単元でないから$a^{-1} \in \I{m}_R$.
    一方,$a \in \Gamma(X, \shO_X) \subseteq K$から,
    任意の$x \in X$について$a \in \shO_{X,x}$.
    よって$(a \in )\shO_{X,x} \subseteq R$ならば
    $a \cdot a^{-1}=1 \in \I{m}_R$となり,不合理.
    したがって$R$はいかなる点もcenterに持たない.
    これは(b)に反する.

    \begin{Claim}
     \[ a \not \in \overline{k[a^{-1}]}. \] 
    \end{Claim}
    \begin{proof}
        $a \in K$が$k[a^{-1}]$上整ならば,
        以下のような零でない多項式$f$が存在する.
        \[ f \in k[a^{-1}][X] ~,~ \deg f=d>0 ~,~ f(a)=a^d+c_{d-1}a^{d-1}+\dots+c_0=0. \]
        最後の等式の両辺に$a^{-d} \in k[a^{-1}]$をかける.
        \[ a^{-d}f(a)=1+c_{d-1}a^{-1}+\dots+c_0a^{-d}=0. \]
        $c_i \in k[a^{-1}]$だから,
        この等式は
        $f'(a^{-1})=0$を満たす零でない多項式$f' \in k[X]$が存在すること,
        すなわち$a^{-1}$が$k$上代数的であることを意味する
        (等式において,$a^{-1}$についての最高次係数は$k$の元であることに注意).
        しかし$a \not \in k=\bar{k}$だからこれはありえない.
    \end{proof}

%    \begin{Claim}
%        $k$は$K$上整閉.
%    \end{Claim}
%    \begin{proof}
%        $A_i$から一つ適当に選び,$A$とする.
%        $A$ :: integral finitely generated $k$-algebraより,
%        $A=\frac{k[x_1,\dots,x_n]}{\I{p}}$となる$n>0$と素イデアル$\I{p}$が存在する.
%        Ex3.6より,$K=\Quot(A)$だから,$K$の元は$f,g \in A$を用いて$f/g$と書ける.
%        $f/g$が$k$上整だとしよう.
%        するとmonicな多項式$E(X) \in k[X]$が存在し,$E(f/g)=0$となる.
%        分母を払ったり$A=\frac{k[x_1,\dots,x_n]}{\I{p}}$を使ったりすると,
%        結局$E(f/g)=0$は$x_1,\dots,x_n$がある$k$係数多項式の根であることと同値になる.
%        $k$ :: algebraically closedなので,この$k$係数多項式は根をもつ.
%        その値を$f/g$に代入すれば$f/g \in k$が得られる.
%    \end{proof}

\section{$f$ :: proper morphism of affine varieties/$k$. Then $f$ :: finite.} %% Ex4.6 
    $f: X \to Y$を考える.
    $X,Y$ :: affine variety / $k$より,
    $X,Y$は$A, B$ :: affine domain / $k$を用いて$X=\Spec A, Y=\Spec B$と書ける.
    $f$から誘導される環準同型を$\phi: B \to A$とする.
    $A, B$ :: affine domain / $k$から特に$X, Y$はNoetherianである.
    また,$f$ :: finite typeより$A$ :: finitely generated $B$-algebra.
    よって,$f$ :: finiteであるためには$\phi$ :: integral
    すなわち$A$ :: integral / $\phi(B)$を示せば十分である.

    $K=\Quot(\phi(B))$とし,
    $R$を$\phi(B) \subseteq R \subset K$であるような任意のvaluation ringとする.
    Thm4.7から以下の可換図式が得られる.
    \[
    \xymatrix
    {
        K  & A \ar@{_{(}->}[l] \ar[ld]\\
        R  \ar@{^{(}->}[u]& B \ar[u]_-{\phi} \ar[l]
    }
    \]
    $A \to R \to K=A \to K$かつ右辺が埋め込みであることから$A \to R$は埋め込みである.
    すなわち$(\phi(B) \subseteq )A \subseteq R$.
    $R$のとり方とThm4.11より,$A \subseteq \overline{\phi(B)}$.
    よって$A$ :: integral / $\phi(B)$.

\section{Schemes Over $\R$} %% Ex4.7 

\section{Let $\wp$ :: Property of Morphisms of Schemes} %% Ex4.8 

\section{Composition of Projective Morphisms is Projective} %% Ex4.9 

\section{Chow's Lemma.} %% Ex4.10 

\section{Descrite Valuative Criteria of Separatedness and Properness.} %% Ex4.11 

\section{Examples of Valuation Rings.} %% Ex4.12 
    \subsection{If $K/k$ :: function field of dim$=1$, then every valuation ring of $K/k$ is discrete.}
    $K$に対応するnonsingular projective curveを
    $\tilde{C}$(cf. ch I, Cor6.12)とし,
    $C=t(\tilde{C})$を$C$に対応するschemeとする(Prop 2.6).
    これはprojective integral scheme/$k$,
    特にproper \& integral scheme of finite type /$k$(Prop4.10).
    $C$上のirreducible closed subsetは
    $C$とclosed pointしか無いから,
    $C$の点は$\zeta$ :: generic point とclosed pointしかない.

    \begin{Claim}[claim 1]
        $K=\shO_{C,\zeta}=K(\tilde{C})$.
    \end{Claim}
    \begin{proof}
        Prop2.6の証明を見ると,
        $\shO_C$はsheaf of regular functions on $\tilde{C}$である.
        なので$K(\tilde{C}), \shO_{C,\zeta}$の定義から両者は一致する.
    \end{proof}

    \begin{Claim}[claim 2]
        $x \in C$ :: closed pointについて$\shO_{C,x}$ :: DVR.
    \end{Claim}
    \begin{proof}
        $\shO_C$ :: sheaf of regular functions on $\tilde{C}$から,
        $\shO_{C,x}=\shO_{\tilde{C},x}$.
        これがDVRであることはch I, Prop6.7の証明中程にもあるし,
        ch I, Thm6.2とDedekind domainの定義の下の文からも分かる.
    \end{proof}

    $K/k$の($K$ではない)任意のvaluation ring $R$を考える.
    $C$ :: proper \& integral scheme of finite type /$k$,Ex4.5bとclaim 1より,
    $R$が$\shO_{C,x}$をdominateするような点$x \in C$が一つあることが言える.
    $\shO_{C,x} \subseteq R \neq K$から,$x$ :: closed point.
    さらにclaim 2から,$R$は少なくとも一つのDVRをdominateすると言える.
    valuation ringの極大性から$\shO_{C,x}=R$ :: DVR.

\end{document}
