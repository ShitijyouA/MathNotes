%% CatThe 1st session
\documentclass[a4j]{jarticle}
\usepackage{url}
\usepackage{amssymb}
\usepackage{amsthm}
\newcommand {\cat}[1]{%
\mathbf{#1}%
}
\newcommand {\dom}[1]{%
\mathrm{dom}(#1)%
}
\newcommand {\cod}[1]{%
\mathrm{cod}(#1)%
}
\newcommand {\idarrow}[1]{%
1_{#1}%
}

\theoremstyle{definition}
\newtheorem{theorem}{定理}
\newtheorem{definition}[theorem]{定義}
\newtheorem{lemma}[theorem]{定義}

\title{Category Theory 輪読会 \S 1.1-1.9}
\author{shitijyou}
\date{2016年2月25日}

\begin{document}
    \maketitle

    %% TODO セクションのタイトルに元のセクション番号を入れる
    \section{圏の定義 / \S 1.2 Definition of a category}
        \begin{definition}[圏]  %% TODO (3.1)に書き直す
            クラス$A$に二項演算$\ast$($A$の要素2つから$A$の要素1つへの対応)が入っていたとする.
            さらにクラス$A, O$の間に対応 ${\rm dom, cod}: A \rightarrow O$があるものとする.
            これらが次を満たすとき,$\cat{C}=(A, O, {\rm dom ,cod}, \ast)$を圏という.
            \begin{description}
%                \item[C1 (dom, codは\it{A}から\it{O}への対応である)] \mbox{} \\
%                    $^\forall a \in A,~ ^\exists \dom{a},~ \cod{a} \in O$

                \item[C1 ($\ast$とdom, cod)]\mbox{} \\
                $^\forall a, b \in A, \cod{a}=\dom{b} \Rightarrow
                ~^\exists a \ast b \in A ~s.t.~ \dom{a}=\dom{a \ast b} ~\wedge~ \cod{a \ast b}=\cod{b}$

                \item[C2(単位元の存在)]\mbox{} \\
                    $^\forall {z} \in O,~ ^\exists \idarrow{x} \in A ~s.t.~
                        \dom{\idarrow{x}}=\cod{\idarrow{x}}=x$

                \item[C3(単位元の公理)]\mbox{} \\
                    $^\forall a \in A,~ a \ast \idarrow{\dom{a}} = \idarrow{\cod{a}} \ast a$

                \item[C4($\ast$の結合則)]\mbox{} \\
                    $^\forall a, b, c \in A,~ a \ast (b \ast c)=(a \ast b) \ast c$
            \end{description}
        \end{definition}

        用語「クラス」を未定義のまま用いた.これは集合の拡張概念である.詳しくは後述する.
        また,ここでは後の理解のため,$A$の要素が出来るだけ「矢印」「有向線分」に見えないように表現した.

        $A$の要素は圏$\cat{C}$の「射」と呼ばれ,$O$の要素は「対象」と呼ばれる.
        $a \in A$について$x=\dom{a}, y=\cod{a}$なら写像のように$a:x \rightarrow y$と表現される.
        この表現を用いると,上での(C1), (C2), (C3)は次のようになる.``$\in A$"は省略する. %% TODO ここの番号も修正
        \begin{description}
            \item[C1]
            $^\forall a: x \rightarrow y,~ b: y \rightarrow z,~ ^\exists a \ast b: x \rightarrow z$

            \item[C2]
                $^\forall {x} \in O,~ ^\exists \idarrow{x}: x \rightarrow x$

            \item[C3]
                $ ^\forall a: x \rightarrow y,~ a \ast \idarrow{x} = \idarrow{y} \ast a $
        \end{description}

        これらを可換図式(次の節で定義する)として書くと直感的な理解が深まるだろう.
        これらの定義を満たす{\it あらゆるもの}が圏である.
        例えば,圏論ではある圏の射は別の圏の対象になるし,$\cat{Mon}$では集合の要素が圏の射になる.
        私は可換図式による理解のみでは後述のスライス圏の理解が難しかった.
        適切なときにイメージに頼らない表現へ立ち返って欲しい.

    \section{圏の周り}
        \begin{definition}[可換図式 / commutative diagram]
            可換図式とは,頂点とそれを結ぶ向き付きの道(directive path)からなり,
            始点と終点が同じ全ての向きつきの道が,合成によって等しくなるものである.
        \end{definition}

        これは代数学の式のような役割を果たす.
        また,これは後に添字圏などを用いて再定義される. %% TODO 添字圏だったっけ

        \begin{definition}[関手 / Functor]
            関手とは可換図式から可換図式への変換であって,頂点のつながり方を保つものである.
        \end{definition}

    \section{同型 / Isomorphisms}
        同型の定義はご存知の通り.
        ここでは群が対称群で表現できること(Cayley Theorem)を示す.
        これの類似として局所小圏(locally small category)が集合と関数からなる圏で表現できることをみる.
        表現は圏論では(多分)重要なテーマであり,米田の補題も圏の表現についての定理である.
        詳しくは『圏論の歩き方』なんかを参照.

    \section{圏の例 / Examples of categories}
        よく知っている代数的構造を圏として捉えてみる.
        他の例は教科書を参照.

        \begin{definition}[モノイド / monoid $M$]
            対象: ただ一つの何か,
            射:$ M$の元,
            射の合成: $M$の元の積
        \end{definition}
        対象がただ一つに限られるのは,
        各元の積を射とみなすには,各元の$\mathrm{dom}, \mathrm{cod}$が全て等しい必要があるから.

        これは次のようにして$Sets$の部分圏とみなせる.これはTheorem 1.6の基本的アイデアである.

        \begin{definition}[generated category from $M$ $\cat{\bar{M}}$]
            対象: $M$,
            射: Mの元を乗ずる関数$(m \ast)$,
            射の合成: 通常の関数合成
        \end{definition}
        $\cat{\bar{M}}$の射である関数を単位元に適用すれば直ちに射と対応する$M$の元が得られる.

        \begin{definition}[前順序集合 / proset $P$]
            対象: $P$の元,
            射: $a \rightarrow b \Leftrightarrow a \leq b$,
            射の合成: 順序の推移律より自然に定まるもの
        \end{definition}

        なお,前順序(preordred)に反対称律\footnote{$ a \leq b and a \geq b \Leftrightarrow a=b$}を
        加えたものは半順序(partially ordered)であり,半順序集合はposetと略される.
        posetも同様に圏とみなせる.

    \newpage

    \section{圏の構築 / Constructions of categoriesかつ}
        ここはひたすら教科書を読むだけでも良いが,
        それぞれが圏になっていることを確かめるのは大切.
        また,双対圏は圏論全体で重要.

        \begin{definition}[圏の積 / product of category $\cat{C} \times \cat{D}$]
        \end{definition}

        \begin{definition}[双対圏 / opposite category $\cat{C^{op}}$]
        \end{definition}
        圏論で現れるco- と名前につく概念は
        圏$\cat{C}$で得られた概念を$\cat{C^{op}}$に移して得たものである.
        圏の定義はそれ自体が双対(自己双対/self dual)になっているため,
        圏$\cat{C}$で証明できる定理はそれを$\cat{C^{op}}$に書き換えても成り立つ.

        \begin{definition}[射圏 / arrow category $\cat{C^{\rightarrow}}$]
        \end{definition}

        \begin{definition}[スライス圏 / slice category $\cat{C}/C$]
        \end{definition}
        この2つは共にコンマ圏(comma category)の特殊な場合である.コンマ圏はこの教科書で扱われない.

    \section{自由圏 / Free Categories}
        \subsection{自由モノイド / Free Monoid}
            \begin{definition}[クリーネ閉包(Kleene closure)]
                $A^{*}=\{ - \} \cup \{ Aの元の積 \}$
            \end{definition}
            ただし$ - $は空文字列のようなもの.
            これはオートマトンの研究の中で生まれたもので,正規表現で表せる.
            たとえば$A=\{ab, c\}$から$A^{*}$を生成する正規表現は$/((ab)^{*}(c)^{*})^{*}/$.
            この文脈から,$A$の元は$alphabet$,$A$の元の積は$word$と呼ばれる.
            また,例えば$\mathbf{N}=M(\{1\})$である.
            
            これとは別に,$A$から自由生成されたモノイドが定義される.
            以降は自由生成されたモノイドは単に「A上の自由モノイド」と呼ぶ.
            \begin{definition}[free monoid on $A$]
                次が成り立つとき,A上の自由モノイドである.
                \begin{enumerate}
                    \item $A \subset M$
                    \item $M$の元は$A$の元の積で表される\footnote{では$A$の元の積は$M$の元か?}.
                    \item モノイド$M$はnon-trivialな(すなわち公理にない)等式を持たない.
                \end{enumerate}
            \end{definition}

            この定義が次にように書き換えられる.

            \begin{definition}[free monoid on $A$ (rewritten using UMP)]
                A上の自由モノイドとは,($A$をその定義に含む)UMPを持つモノイドのことである.
            \end{definition}

            このことから$A$上の自由モノイドは全て同型であることが示される.
            $A^{*}$が$A$上の自由モノイドであることは明らかだから,
            結局$A$上の自由モノイドは全て$A^{*}$と同型である.

        \subsection{自由圏 / Free Category}
            別名: path category.
            有向グラフ,あるいは箙\footnote{箙は矢を入れる筒を表す漢字}(えびら/quiver)から自由生成される圏のこと.

            この節では忘却関手$U: \cat{Cat} \rightarrow \cat{Graph}$の別の表現を見ていく.
            そのためにp.21の終盤からp.22にかけて$\cat{Graph}$の準同型について別の表現をみる.
            その結果として忘却関手を黒板でデモンストレーションする.

            最後にfree categoryのUMPを得て,free categoryがfree monoidの拡張であること,
            Awodeyによれば``generalized free"であることを観察する.

    \section{大きい圏,小さい圏, 局所小圏 /Large, small, and locally small}
        別プリント.数学基礎論に足を突っ込む.
        別プリントにはlocally smallについて書いていない.

    \section{演習}
        輪読会後,各自で.

\begin{thebibliography}{99}
    \bibitem{ike13} 池渕未来(2013)
        「1. 圏論とプログラミング、プロダクト」
        \url{http://www.iij-ii.co.jp/lab/techdoc/category/category1.html}

    \bibitem{ct4s} David I. Spivak(2013)
        ``Category Theory for Scientists"
        \url{http://math.mit.edu/~dspivak/teaching/sp13/CT4S.pdf} %% dynamic version

    \bibitem{morita} 森田真生
        「哲学者のための圏論入門」
        \url{http://choreographlife.jp/pdf/intro.pdf}

    \bibitem{kubota} 久木田水生(2011)
        「米田埋め込みと米田の補題」
        \url{http://www.info.human.nagoya-u.ac.jp/lab/phil/kukita/others/Yonedas_lemma.pdf}

    \bibitem{kimaira} 
        ブログ「檜山正幸のキマイラ飼育記」で「圏」を検索したもの
        \url{http://d.hatena.ne.jp/m-hiyama/searchdiary?word=%B7%F7}

\end{thebibliography}
\end{document}
