\documentclass[a4paper]{jarticle}
\usepackage{../math_note}

\title{射影空間内の超曲面の間にある同型}
\begin{document}
\maketitle

参考: \url{https://math.stackexchange.com/questions/2657197}.
\begin{Thm}
    $n,d > 1$とし,$k$を任意の体とする.
    $H, H' \subseteq \proj^n$をsmooth hypersurface of degree $d$とすると,
    $2$つの例外$(n,d)=(2,3), (3,4)$の場合を除き,
    isomorphism $H \to H'$は$\proj^n$全体の自己同型($=PGL(n, k)$)の制限から得られる.
\end{Thm}
別の言い方をすれば,
isomorphism $H \to H'$は
$n,d>1, (n,d) \neq (2,3), (3,4)$ならば
isomorphism $\proj^n \to \proj^n$に持ち上げられる.

$f^* \shO_{X'}(1)=\shO_X(1)$が成立するためには,
$\omega_X=\shO_{X}(d-(n+1)) \neq \shO_X$または
$n>3$が十分.
そして$(n,d)=(2,3), (3,4)$で反例が作れる.

$\shO_X(1)$は$X$のhyperplane sectionに対応する.
$f^*:X \cap H \mapsto X' \cap H'$なら$H \mapsto H'$として$f$を拡張できる.
($H, X'$の法線ベクトルを写す.)

$\shO_X(1) \iso \shO_{\proj^n}(1)$は$d$-uple embeddingで$X$をhyperplaneとすれば分かる.

最後に,
$X, X'$ :: smooth hypersurface in projective spaceのcanonical divisorに関して
$K_X \sim dH$となっている.
$K_{X'} \sim f^* K_{X}+R$という関係があって,
$f$がisomorphismなら$R \sim 0$.
なので$|H \cap X|$の生成元が$|H \cap X'|$に写る.
\url{https://mathoverflow.net/questions/80288/pullback-of-the-canonical-divisor-between-smooth-varieties}
\end{document}
