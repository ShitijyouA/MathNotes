\documentclass[a4j]{jsarticle}
\usepackage{../math_note}

\title{多項式の既約性判定法}
\author{七条 彰紀}

\begin{document}
\maketitle

\begin{Thm}[Eisenstein's criterion]
    \[ f(x)=\sum_{0 \leq k \leq n}{f_k x^k} \in \Z[x] \]
    について,ある素数$p$が存在して,
    整数$f_0, f_1, \dots, f_n$が以下を満たすならば,
    $f(x)$は$\Q[x]$の既約元である.
    \begin{enumerate}
        \item $i \neq n$の場合は$f_i$は$p$で割り切れる.
        \item $f_n$は$p$で割り切れない.
        \item $f_0$は$p^2$で割り切れない.
    \end{enumerate}
\end{Thm}
\begin{proof}
    多項式$g,h$を$f(x)=g(x) h(x)$を満たすものとおき,
    多項式$f, g, h$の各係数を
    \[ g(x)=\sum_{0 \leq i \leq n}{g_i x^i}, h(x)=\sum_{0 \leq j \leq n}{h_j x^j} \]
    と置く.
    この時,単純な計算で
    \[ f_{k}=\sum_{i+j=k}{g_i h_j} \]
    が成り立つと分かる.
    記法を簡単にするため,$\I{p}=(p) \subset \Z$とおく.
    これが素イデアルであることを何度も使う.

    \paragraph{$g_0 \in \I{p}, h_0 \not \in \I{p}$.}
    $f_0$を考える.
    \[ f_0=g_0 h_0 \]
    前提条件1.より$f_0$は$p$の倍数である.
    さらに前提条件3.から,$f_0$には素因数として$p$がただ一つ含まれる.
    その$p$は$g_0$か$h_0$のどちらか一方に含まれている.
    そこで前提条件に加えて$(*)~~g_0 \in \I{p}, h_0 \not \in \I{p}$を仮定する.

    \paragraph{$g_0, g_1, \dots, g_{n-1} \in \I{p}$.}
    帰納法で$g_0, g_1, \dots, g_{n-1} \in \I{p}$を示す.
    まず,$k=1$で示す.
    \[ f_1=g_0 h_1 + g_1 h_0 \in \I{p} \]
    $\I{p}$はイデアルだから$g_0 h_1 \in \I{p}, h_0 \not \in \I{p}$.
    特に$\I{p}$は素イデアルだから$g_1 \in \I{p}$.
    次に,$0 \leq N+1 < n$を満たす自然数$N$について
    $g_0, g_1, \dots, g_{N} \in \I{p}$が成り立つとする.
    \[ f_{N+1}=g_{N+1} h_0+g_{N} h_1+\sum_{1 \leq j \leq N+1}{g_{N+1-j} h_j } \]
    そして前提条件1.より$f_{N+1} \in \I{p}$が成り立つ.
    帰納法の仮定より,$g_{N} h_1, \sum_{2 \leq j \leq N+1}{g_{N+1-j} h_j } \in \I{p}$.
    仮定(*)より$h_0 \not \in \I{p}$だから$g_{N+1} \in \I{p}$.

    \paragraph{$g_n \not \in \I{p}$.}
    さて,最後に$f_n$を考える.
    \[ f_{n}=g_{n} h_0+\sum_{1 \leq j \leq n}{g_{n-j} h_j } \]
    前提条件2.より$f_{n} \not \in \I{p}$.すでに示したとおり,$g_0, g_1, \dots, g_{n-1} \in \I{p}$が成り立つ.
    したがって,仮定(*)と合わせて
    $g_n \not \in \I{p}$が成立する.

    \paragraph{結論: $\deg g=n$.}
    $0 \in \I{p}$だから,このことから$g_n \neq 0$.
    よって$\deg g=n, \deg h=n-n=0$.これで$f$の既約性が示された.
\end{proof}

これと命題を組み合わせると,多くの多項式の既約性が示せる.
\begin{Prop}
    多項式$f(x) \in \Z[x]$と任意の定数$a$について,
    「$f(x+a)$が既約」と「$f(x)$が既約」は同値.
\end{Prop}
\begin{proof}
    $f(x)$が既約だとする.
    定数$a$に対し,1次以上の多項式$g,h$(これは$a$によって変化する)が
    存在して$f(x+a)=g(x)h(x)$が成り立つ($f(x+a)$が既約でない)ならば,
    $f(x)=g(x-a)h(x-a)$となり,$g(x-a), h(x-a)$は一次以上の多項式.これは前提に矛盾.
    よって$f(x+a)$も既約.

    $f(x)$が既約でないとする.
    すると1次以上の多項式$g,h$が存在して$f(x)=g(x)h(x)$が成り立つが,
    $f(x+a)=g(x+a) h(x+a)$となり,$g(x+a), h(x+a)$は一次以上の多項式.
    よって$f(x+a)$も既約でない.
\end{proof}

次は有限体への還元を用いた判定法である.
\begin{Thm}[Reduction Criterion in S.Lang ``Algebra'']
    $A,B$を整域とし,$\phi: A \to B$を準同型とする.
    さらに$B$の商体を$L$としておく.
    $f \in A[x]$について以下が成り立つとき,
    $f$は$A[x]$の既約元
    \footnote{すなわち,$f=gh$かつ$\deg g, \deg h >1$であるような$g,h \in A[x]$が存在しない.}
    である.
    \begin{enumerate}
        \item $\phi(f) \neq 0$.
        \item $\deg \phi(f)=\deg f$.
        \item $\phi(f)$は$L[x]$の既約多項式.
    \end{enumerate}
\end{Thm}
\begin{proof}
    $f=gh ~~(g,h \in A[x])$と分解できたとすると,
    $\phi(f)=\phi(g) \phi(h)$となる.
    前提条件3. より$\deg \phi(g) \mor \deg \phi(h)=\deg \phi(f)$であり,
    かつ$\deg \phi(g) \leq \deg g, \deg \phi(h) \leq \deg h$.
    これらと前提条件2. より$\deg g \mor \deg h=\deg \phi(f)=\deg f$.
    以上で主張が示せた.
\end{proof}

\begin{Cor}
    $\mathbb{F}_q$を位数$q$の有限体とし,以下の準同型を定める.
    \[ \rho_q: \Z[x] \to \mathbb{F}_q[x];~~ ax^n \mapsto (a \bmod q)x^n. \]
    $f \in \Q[x]$に適当に$d \in \Z \setminus \{0\}$を掛けて
    $df \in \Z[x]$とする.
    ある$q$について,$\rho_q(df)$が既約ならば$f$は既約である.
\end{Cor}

\begin{Example}
    $n \in \Z \setminus \{0\}, f=x^3-nx^2+(n-3)x+1 \in \Z[x]$とする.
    $\rho_2(f)=x^3+nx^2+(n+1)x+1$となる.
    $\rho_2(f)$は$3$次多項式だから,$1$次以上の因子を持つならば,
    そのうち少なくとも一つは$1$次式である.
    体$\mathbb{F}_2$上の$1$次式は丁度一つの零点を持つから,
    $\rho_2(f)$も少なくとも一つ零点を持つ.
    しかし$\rho_2(f)(0)=1, \rho_2(f)(1)=1$だから$\rho_2(f)$は零点を持たない.
    これは矛盾であるから,$\rho_2(f)$は$\mathbb{F}_2[x]$の既約多項式である.
    そして系から,$f$は$\Q[x]$の既約多項式である.
\end{Example}

次もまた別の判定法である.
\begin{Thm}[Cohn's Criterion]
    $b \in \Z_{\geq 2}$と$p(x)=a_{k}x^{k}+a_{k-1}x^{k-1}+\cdots +a_{1}x+a_{0}$は
    $0 \leq a_{i}\leq b-1$を満たすとする.
    $p(b)$が素数ならば,$p(x)$は$\Z[x]$の既約元である.
\end{Thm}
証明は難しい.
詳細は\url{https://www.wikiwand.com/en/Cohn's_irreducibility_criterion} を参照のこと.
なお,「$0 \leq a_{i} \leq b-1$」という条件は必ずしも最良ではない.
例えば「$p(10)$が素数かつ$0 \leq a_{i} \leq N$ならば$p(x)$は既約」が成り立つ最大の$N$は
\[ N=49598666989151226098104244512918 \]
である(Michael Filaseta and Samuel Gross, \url{ https://doi.org/10.1016/j.jnt.2013.11.001 }).

\end{document}
