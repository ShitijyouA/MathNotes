\documentclass[]{jsarticle}
\usepackage{../math_note, enumitem}

\begin{document}
    \title{層(sheaf)の概観 \\ {\normalsize 代数幾何学に於いて}}
    \author{七条彰紀}
    \maketitle

    代数幾何学を始める人のために,またこれから学ぶ人のために書きます.ノートです.当然ですが.

    \section{代数幾何学に於けるsheafという概念}
    まず代数幾何学でのsheafの立ち位置について書いておきます.
    sheaf,より一般にstackは代数幾何学の中心的概念です.

    \paragraph{scheme theoryに於いて}
    scheme theoryでは可換環からschemeという幾何学的な対象を構築します.
    この構築の際に,可換環論という代数学を記憶しているのがsheafです.
    可換環が「切り刻まれ貼り合わされて」structure sheafとなっているのです.
    他にscheme theoryで現れるsheafとしては,scheme上の(quasi-)coherent sheafが重要です.
    (quasi-)coherent sheafを調べることは,schemeについての情報を得るための基本的な手段です.
    またscheme上のsheaf cohomologyは,
    schemeに「変な感じの部分」がどれだけあるかを調べるための重要な道具となっています.
    (特にetale cohomologyは数論においても重要な位置を占めています.)

    \paragraph{scheme theoryをはみ出す}
    一方で,schemeだけでは用に足りないことがあります.
    例えばschemeの群による商を考えることがあります
    (これは普通の位相空間の群作用による商のようなものです).
    定義は圏論的に,普遍性を用いて定義されるのですが,
    条件を満たすschemeが無い,ということはしょっちゅうです.
    これに対する一つの解決方法として,
    schemeの概念を拡張するということが考えられます.
    圏論的に性質の良い,都合の良い対象まで研究対象に収めようというわけです.

    \paragraph{圏論ちょっとわかる,という人向け}
    schemeの概念を拡張するには,どのような方策を取るべきでしょうか.
    当座の目標は「schemeの圏を包含する圏を探す」ということです.
    全ての極限をschemeの圏に付け加える(pro-scheme),
    基礎を担う可換環論を非可換環論やモノイド論まで拡大する,
    などの手段があります.
    ですがまた別に,米田の補題を手がかりにする事が出来ます.
    米田の補題は,
    米田関手がschemeの圏からscheme上のpresheafの圏への忠実充満関手
    となることをいっています.
    schemeの圏を包含する圏として「schemeの圏からscheme上のpresheafの圏」
    が使える,ということです.

    \paragraph{schemeの一般化に於けるsheaf}
    この,schemeの一般化(generalized scheme)を考える方向では,
    sheafが中心概念です.
    実際にgeneralized schemeの代表であるalgebraic spaceはsheafです.
    そしてalgebraic spaceの定義に
    algebraic spaceの位相空間の定義は含まれていません.
    
    ちなみに極端なことを言うと,
    schemeでさえ最初から位相空間無しに定義することが可能です.
    これは"functorial scheme"などと呼ばれます.
    もちろんこれは普通の意味のschemeではありませんが,
    "functorial scheme"から普通のschemeの体裁を整えることも,
    この逆も可能です.

    \paragraph{さらなる一般化}
    そしてさらにsheafはstackへ,
    algebraic spaceはalgebraic stack (Artin/DM stack)へと一般化されます.
    恐ろしいことにalgebraic spaceにもalgebraic stackに関しても
    (quasi-)coherent sheafやsheaf cohomologyといった理論が構築されています.

\section{sheafの思想}
    \paragraph{「局所的に調べ,大域的に知る」}
    位相空間上のsheafの定義の仕方はいくつか存在しますが,
    意味が分かりやすいのは``identity axiom"と``gluability axiom"を
    満たすpreaheafとして定義することだと思います.
%    (特定の完全列を満たすpresheafとして定義するのは圏論的な取扱いに向いていますし,
%    etale mapのsectionが成すpresheafとして定義するのはsheafの一歩進んだ理解を促します.)

\end{document}
