%% {{{
\RequirePackage[]{silence}
\WarningFilter{latexfont}{Font shape}
\WarningFilter{latexfont}{Some font shapes were not available}
%% }}}
\documentclass[lualatex, ja=standard, a4paper]{bxjsarticle}

\usepackage{../math_note}
\usepackage[backend=biber, style=alphabetic]{biblatex}
\addbibresource{reference.bib}

\newcommand{\introword}[2]{{\bfseries #1} (#2) }
\newcommand{\ftor}[1]{\underline{#1}}

\newcommand{\ZAR}   {\opcat{\Ring}_{\mathrm{ZAR}}}
\newcommand{\ET}    {\opcat{\Ring}_{\mathrm{ET}}}
\newcommand{\SM}    {\opcat{\Ring}_{\mathrm{SM}}}
\newcommand{\FPPF}  {\opcat{\Ring}_{\mathrm{FPPF}}}
\newcommand{\FPQC}  {\opcat{\Ring}_{\mathrm{FPQC}}}

\newcommand{\stX}{\mathcal{X}}
\newcommand{\stY}{\mathcal{Y}}
\newcommand{\stZ}{\mathcal{Z}}

\newcommand{\Aff}{\cat{Aff}}
\newcommand{\catOb}{\operatorname{Ob}}
\newcommand{\CAS}{\cat{CAS}_{\cat{C}, E}}
\newcommand{\property}{\mathcal{P}}

%% \allowbreak {{{
\makeatletter
\def\old@comma{,}
\catcode`\,=13
\def,{%
  \ifmmode%
    \old@comma\discretionary{}{}{}%
  \else%
    \old@comma%
  \fi%
}
\makeatother
%% }}}

\begin{document}
\title{スキーム・代数的空間・代数的スタックの別定義}
\author{七条 彰紀}
\maketitle

\begin{abstract}
    可換環の圏から出発して,
    スキーム,代数的空間,代数的スタックをチャート付き層あるいはチャート付きスタックとして定義する.
    「チャート付き対象」は多様体やスキームと言った対象の定義の様式を一般化したものである.
    それぞれチャート付き対象の様式での定義を述べた後,
    それらが通常使われる定義と同値であることを述べる.
    読者はすでにスキーム,代数的空間,代数的スタックの定義を熟知しているものとする.
\end{abstract}

\section{思想:チャート付き幾何的対象}
多様体は通常,以下のように定義される.

\begin{Def}[\introword{多様体}{manifold}の開被覆]
    位相空間$X$を考える.
    \begin{enumerate}
        \item
            位相空間の射$f \colon X \to Y$が\introword{開埋め込み}{open immersion}であるとは,
            \begin{itemize}
                \item $f(X)$が位相空間$Y$の開部分集合であり,
                \item かつ$f$から誘導される射$X \to f(X)$が同相射である,
            \end{itemize}
            ということ.

        \item
            位相空間の射の族$\{ u_i \colon U_i \to X \}_{i \in I}$が
            開被覆であるとは,
            任意の点$x \in X$に対して$x \in u_i(U_i)$なる$i \in I$が存在するということ.
    \end{enumerate}
\end{Def}

\begin{Def}[ユークリッド空間によるチャート付き位相空間としての実多様体]
    位相空間$X$が$m$次元実多様体であるとは,
    ユークリッド空間$\R^m$の開部分集合からの射からなる
    開被覆$\{ U_i \to X \}_{i \in I}$が存在するということ.
\end{Def}

一方,スキームはアフィンスキームによる開被覆を持つ局所環付き空間として定義されるのであった.

\begin{Def}[\introword{局所環付き空間}{locally ringed space}の開被覆]
    局所環付き空間$(X, \shO_{X})$を考える.
    \begin{enumerate}
        \item
            局所環付き空間の射$(f, f^{\#}) \colon (X, \shO_{X}) \to (Y, \shO_{Y})$が
            \introword{開埋め込み}{open immersion}であるとは,
            \begin{itemize}
                \item $f(X)$が位相空間$Y$の開部分集合であり,
                \item かつ$f$から誘導される射$X \to f(X)$が同相射であり,
                \item かつ層の射$f^{\#}|_{f(X)} \colon \shO_{Y}|_{f(X)}=\shO_{f(X)} \to \shO_{X}$が同型射である,
            \end{itemize}
            ということ.

        \item
            局所環付き空間の射の族$\{ (u_i, u_i^{\#}) \colon (U_i, \shO_{U_i}) \to (X, \shO_{X}) \}_{i \in I}$が
            開被覆であるとは,
            任意の点$x \in X$に対して$x \in u_i(U_i)$なる$i \in I$が存在するということ.
    \end{enumerate}
\end{Def}

\begin{Def}[アフィンスキームによるチャート付き局所環付き空間としてのスキーム]
    局所環付き空間$(X, \shO_{X})$がスキームであるとは,
    アフィンスキームからの射からなる開被覆$\{ \Spec R_i \to (X, \shO_{X}) \}_{i \in I}$が
    存在するということ.
\end{Def}

ここに共通して見られるのは,多様体とスキームはどちらも
「既によく知られている幾何的対象で被覆できる種類の幾何的対象」である,ということである.
このような幾何的対象は\cite{Lin16}で圏論的に取り扱われていて,
\introword{チャート付き対象}{charted object}と呼ばれている.

多様体は特別な位相空間,スキームは特別な局所環付き空間として定義されている.
一方,代数的空間と代数的スタックはそれぞれ特別な層,特別なスタックとして定義される.
スキームも特別な層として,可換環の圏から構成することが出来る.
この際に古典的な意味の位相空間は必要でない.
そして代数的空間と代数的スタックもチャート付き対象の形で定義することが出来る.

\section{代数幾何的な空間的対象の構成方法}
    この節では代数的空間や代数的スタックの文脈における「表現可能な射」や「表現可能な射による被覆」を一般化する.
    通常のスキームはアフィンスキームからの開埋め込み射による被覆(Zariski 被覆)をもつ局所環付き空間である.
    これを
    \begin{itemize}
        \item アフィンスキームの圏を一般の圏$\cat{C}$へ,
        \item 局所環付き空間を$\cat{C}$を含む圏$\cat{S}$へ,
        \item 開埋め込みを$\cat{C}$の射のクラス$E$へ
    \end{itemize}
    一般化する.
    するとスキームは$\cat{C}$の対象からの$E$に属す射からなる被覆を持つ$\cat{S}$の対象と考えることが出来る.
    また代数的空間や代数的スタックもこのような枠組の中で考えることが出来るように成る.
    
    このような枠組を考えた上で$\cat{C}, \cat{S}, E$や被覆に必要最小限かつ自然な要請を課して,
    スキームや代数的空間などが持つ(これらを詳しく調べる上で重要な)性質を再現する.
    こうすることで代数幾何学的な空間的対象を作り上げる一般論を作る.

    単位的可換環(以下,環)の圏を$\Ring$と書く.
    もちろんこの圏には零環が属す.
    非可換環は扱わない.

\subsection{環の幾何的被覆}
    \begin{Def}[幾何的被覆]\label{def:geo_cov}
        $\Ring$の射の族$\{\phi_i \colon R \to S_i\}_{i \in I}$が環$R$の幾何的被覆であるとは,
        任意の体$k$と任意の射$S_i \to k$に対して,
        添字$i \in I$,体$k$と射$R \to k', k' \to k$が存在し,
        これらが以下のように可換図式を成すということ.
        \[
        \begin{tikzcd}
            R \ar[r, "\exists"]\ar[d, "\phi_i"']& k' \ar[d, "\exists"]\\
            S_i \ar[r, "\forall"']& k
        \end{tikzcd}
        \]

        一つの射からなる族$\{f \colon S \to R\}$が幾何的被覆であるとき,
        射$f$は幾何的全射であるという.
    \end{Def}

    \begin{Remark}
        この定義に有る図式は双対圏$\opcat{\Ring}$で描いたほうが分かりやすいかも知れない.
        \[
        \begin{tikzcd}
            k' \ar[r, "\exists"]\ar[d, "\exists"']& S_i \ar[d, "\phi"]\\
            k \ar[r, "\forall"']& R
        \end{tikzcd}
        \]
        この図式では$\Spec$が省略されていると考えれば,
        射$k \to R$は$R$の$k$有理点だと解釈できる.
    \end{Remark}

    \begin{Lemma}\label{lemma:geo_cov_is_stable_under_base_change}
        環$R$の幾何的被覆$\{ R \to S_i \}$と環の射$R \to R'$を任意にとる.
        これらから得られる射の族$\{ R' \to R' \otimes_{R} S_i \}$は
        $R'$の幾何的全射である.
    \end{Lemma}
    \begin{proof}
        テンソル積の普遍性を用いれば,圏論的な議論だけで証明できる.
    \end{proof}

\subsection{表現可能な射}
    次のような二つの圏$\cat{C}, \cat{S}$を考える.
    \[ \opcat{\Ring} \subseteq \cat{C} \subseteq \cat{S} \]   
    この包含関係は対象と射について単射的な関手によって与えるものとする.
    例えば米田関手によって$\opcat{\Ring} \subseteq \PShv(\FPPF)$などが考えられる.

    \begin{Def}[$\cat{C}$で表現可能; $\cat{C}$-representable]
        包含関係$\opcat{\Ring} \subseteq \cat{C} \subseteq \cat{S}$を持つ圏$\cat{C}, \cat{S}$をとる.
        $\cat{S}$は任意の(小さい)ファイバー積を持つものとする.

        \begin{enumerate}
        \item 
            圏$\cat{S}$の対象が$\cat{C}$表現可能とは,
            その対象が$\cat{C}$(の適当な単射的関手による像)の対象と同型であるということ.

        \item
            圏$\cat{S}$の射$x \to y$が$\cat{C}$表現可能であるとは,
            $\cat{C}$の対象から$y$への任意の射$c \to y \ (c \in \catOb \cat{C})$について
            ファイバー積$x \times_{y} c$が$\cat{C}$表現可能である,
            ということ.
        \end{enumerate}
    \end{Def}

    \begin{Example}
        次の場合に「$\cat{C}$で表現可能」の定義を書き下してみる.
        \[ \cat{C}=\opcat{\Ring} \xrightarrow{\text{米田関手}} \PShv(\FPPF)=\cat{S} \]
        $\PShv(\FPPF)$の射$F \to G$が$\opcat{\Ring}$表現可能であるとは,
        任意の環からの射$\Hom_{\Ring}(-,A)=\ftor{A} \to G$について,
        ファイバー積$F \times_{G} \ftor{A}$が環で表現可能である
        (ある環$R$から得られる前層$\ftor{R}$と同型である)ということ.
    \end{Example}

    \begin{Def}[表現可能な対象/射の性質]
        包含関係$\opcat{\Ring} \subseteq \cat{C} \subseteq \cat{S}$を持つ圏$\cat{C}, \cat{S}$をとる.
        $\cat{S}$は任意の(小さい)ファイバー積を持つものとする.

        \begin{enumerate}
        \item 
            $\property$を$\cat{C}$の対象に定義された性質とする.
            圏$\cat{S}$の対象$x$が$\cat{C}$の対象$\tilde{x}$で表現出来る時
            \tablefootnote{ すなわち$\cat{S}$内で$x \iso \tilde{x}$であるとき. },
            $x$が性質$\property$を持つとは,$\tilde{x}$が性質$\property$を持つということ.

        \item
            $\property$を$\cat{C}$の射に定義された性質とする.
            圏$\cat{S}$の$\cat{C}$で表現可能な射$f: \colon x \to y$をとる.
            $f$が性質$\property$を持つとは,
            $\cat{C}$の対象から$y$への任意の射$c \to y$について
            引き戻し$x \times_{y} c \to c$が性質$\property$を持つということ.
        \end{enumerate}
    \end{Def}

    \begin{Lemma}
        $\property$を$\cat{C}$の射に定義された性質とする.
        圏$\cat{S}$の$\cat{C}$で表現可能な射$f: \colon x \to y$をとる.
        次は同値.
        \begin{enumerate}
        \item 
            $\cat{C}$の対象から$y$への\underline{任意の}射$c \to y$について
            引き戻し$x \times_{y} c \to c$が性質$\property$を持つ.
        \item 
            $\cat{C}$の対象から$y$への\underline{ある}射$c \to y$について
            引き戻し$x \times_{y} c \to c$が性質$\property$を持つ.
        \end{enumerate}
    \end{Lemma}
    \begin{proof}
        %% TODO 性質するかどうかも含めて確認.証明.
    \end{proof}

    \begin{Lemma}
        包含関係$\opcat{\Ring} \subseteq \cat{C} \subseteq \cat{S}$を持つ圏$\cat{C}, \cat{S}$をとる.
        $\cat{S}$は任意の(小さい)ファイバー積を持つものとする.
        $\property$を$\cat{C}$の射に定義された性質とする.
        \begin{enumerate}
        \item
            性質$\property$が$\cat{C}$の射の性質として
            合成のもとで安定 (stable under composition) ならば,
            $\property$は$\cat{C}$で表現可能な$\cat{S}$の射の性質としても
            合成のもとで安定である.
        
        \item
            性質$\property$が$\cat{C}$の射の性質として
            基底変換のもとで安定 (stable under base change) ならば,
            $\property$は$\cat{C}$で表現可能な$\cat{S}$の射の性質としても
            基底変換のもとで安定である.
        \end{enumerate}
    \end{Lemma}
    \begin{proof}
        pullback lemmaから明らか.
    \end{proof}

\subsection{許容可能かつ表現可能な被覆}
    \begin{Def}[$\cat{S}$に於ける幾何的被覆]
        圏$\cat{S}$の射の族$\{ f_i \colon x_i \to x \}_{i \in I}$が
        $x \in \catOb \cat{S}$の幾何学的被覆であるとは,
        体$k \in \opcat{\Ring} \subset \cat{S}$から$x$への任意の射$k \to x$について,
        次の可換図式を成す添字$i \in I$,体$k'$と射$k' \to x_i, k' \to k$が存在する.
        \[
        \begin{tikzcd}
            k' \ar[r, "\exists"]\ar[d, "\exists"']& x_i \ar[d, "f_i"]\\
            k \ar[r, "\forall"']& x
        \end{tikzcd}
        \]

        一つの射からなる族$\{f \colon y \to x \}$が$x \in \catOb \cat{S}$の幾何的被覆であるとき,
        射$f$は幾何的全射であるという.
    \end{Def}

    \begin{Def}[射の許容可能なクラス]\label{def:admissible-class}
        $\opcat{\Ring} \subseteq \cat{C}$の包含関係を持つ圏$\cat{C}$と
        $\cat{C}$の射のクラス$E$をとる.
        $\cat{C}$は任意のファイバー積(引き戻し)を持つものとする.
        
        $E$が次の条件を満たす時,$E$は許容可能であるという.
        \begin{enumerate}
            \item 圏$\cat{C}$の同型射は全て$E$に属す.
            \item $E$は合成について閉じている.
            \item $E$の元を$\cat{C}$の任意の射で引き戻して得られる射は再び$E$に属す.
            \item $E$に属す幾何的全射はeffective epimorphismである.
        \end{enumerate}
        最後の条件は
    \end{Def}

    \begin{Example}
        スキームの圏においては,以下のようなクラスが許容可能なクラスを成す.
        \begin{itemize}
            \item 全ての同型射が成すクラス.
            \item 平坦射が成すクラス.
            \item 分離的な射/固有な射が成すクラス
            \item 滑らかな/不分岐な/エタールな射が成すクラス.
        \end{itemize}
    \end{Example}

    \begin{Remark}[圏$\cat{C}, \cat{S}$と射のクラス$E$について]\label{remark:catC_S_class_E}
        以下,
        $\opcat{\Ring} \subseteq \cat{C}$の包含関係を持つ圏$\cat{C}$と
        $\cat{C}$の射のクラス$E$をとる.
        $\cat{C}$は任意のファイバー積(引き戻し)を持つものとする.
    \end{Remark}

    \begin{Def}[$\cat{C}, E$-representable cover]
        圏$\cat{C}, \cat{S}$は上記(\ref{remark:catC_S_class_E})のとおりとする.
        
        対象$x \in \catOb \cat{S}$の$\cat{C}, E$-representable coverとは,
        次の条件を満たす$\cat{S}$の射の族$\{ f_i \colon x_i \to x \}_{i \in I}$である.
        \begin{itemize}
            \item $\{ f_i \colon x_i \to x \}_{i \in I}$は幾何的被覆である.
            \item 任意の$i \in I$について$x_i$は$\cat{C}$の対象である.
            \item 任意の$i \in I$について$\cat{C}$で表現可能である.
            \item
                任意の$i \in I$と任意の$\cat{C}$の対象からの射${\color{blue}c} \to x$について,
                引き戻しで得られる$\cat{C}$の射
                ${\color{blue}x_i \times_{x} c} {\color{red}\ \to\ } {\color{blue}c}$は$E$に属す.
        \end{itemize}
    \end{Def}

    \begin{Def}
        圏$\cat{C}, \cat{S}$は上記(\ref{remark:catC_S_class_E})のとおりとする.
        
        $\cat{C}, E$-representable coverを持つ$\cat{S}$の対象を
        $\cat{S}$の$\cat{C}, E$チャート付き代数的空間対象 ($\cat{C}, E$-charted algebraically space object) という.
        $\cat{C}, E$チャート付き代数的空間対象の射は$\cat{S}$の対象としての射である.

        $\cat{C}, E$チャート付き代数的空間対象が成す圏を$\CAS (\subset \cat{S})$と書く.
    \end{Def}

    \begin{Remark}
        本当は「チャート付き代数的空間」と名付けたいが,
        この名前は特殊な代数的空間を指しているように思えてしまうため,
        以上のように名付けた.
    \end{Remark}

    \begin{Remark}
        以上の定義を$\Ring$を使わずに行う場合,問題と成るのは幾何的被覆の定義であろう.
        圏$\cat{S}$が終対象を持つならば,
        幾何的被覆の定義に現れる体を全て終対象に取り替えることで全射を定義できる.
        他に圏論的にエピ射を用いて定義することも出来るだろうが,有意義なことは出来ないだろう.
    \end{Remark}

\subsection{性質の拡張}
    チャート付き代数的空間対象

    \subsubsection{局所的な性質}
    \begin{Def}[$E$-local properties]
        $E$を圏$\cat{C}$の射が成す許容可能なクラス(定義(\ref{def:admissible-class}))とする.
        \begin{enumerate}
        \item
            $\cat{C}$の対象の性質$\property$をとる.
            性質$\property$が$E$-localであるとは,
            任意の$\cat{C}$の対象$x$について次の二つが同値ということ.
            \begin{itemize}
                \item $x$が性質$\property$を持つ.
                \item $E$に属す射からなる$x$のある幾何的被覆$\{x_i \to x\}$について,
                    全ての$x_i$が性質$\property$を持つ.
            \end{itemize}

        \item
            $\cat{C}$の射の性質$\property$をとる.
            性質$\property$が$E$-local on the sourceであるとは,
            任意の$\cat{C}$の射$y \to x$について次の二つが同値ということ.
            \begin{itemize}
                \item 射$y \to x$が性質$\property$を持つ.
                \item $E$に属す$y$へのある射$z \to y$について,$z \to y \to x$が性質$\property$を持つ.
            \end{itemize}

        \item
            $\cat{C}$の射の性質$\property$をとる.
            性質$\property$が$E$-local on the targetであるとは,
            任意の$\cat{C}$の射$y \to x$について次の二つが同値ということ.
            \begin{itemize}
                \item 射$y \to x$が性質$\property$を持つ.
                \item $E$に属す$x$へのある射$z \to x$について,$z \times_{x} y \to z$が性質$\property$を持つ.
            \end{itemize}
        \end{enumerate}
    \end{Def}

    \begin{Def}[$E$-local on the source and target]
        $E$を圏$\cat{C}$の射が成す許容可能なクラス(定義(\ref{def:admissible-class}))とする.
        $\cat{C}$の射の性質$\property$をとる.

        性質$\property$が$E$-local on the source-and-targetであるとは,
        任意の$\cat{C}$の射$y \to x$について次の二つが同値ということ.
        \begin{itemize}
            \item 射$y \to x$が性質$\property$を持つ.
            \item
                $E$に属すある射$z \to x, w \to z \times_{x} y$について
                $w \to z \times_{x} y \to z$が性質$\property$を持つ.
                \[
                \begin{tikzcd}
                    w \ar[r]\ar[rd]& z \times_{x} y \ar[r]\ar[d]& y \ar[d]\\
                    {} & z \ar[r]& x
                    \ar[ul, phantom, near end, "\ulcorner"]
                \end{tikzcd}
                \]
        \end{itemize}
    \end{Def}

    \begin{Lemma}
        $E$-local on the source-and-targetである性質は
        $E$-local on the sourceかつ$E$-local on the targetである.
    \end{Lemma}

    \begin{Lemma}
        $E$を圏$\cat{C}$の射が成す許容可能なクラス(定義(\ref{def:admissible-class}))とする.
        圏$\cat{C}$の射の性質$\property$が
        $E$-local on the source-and-targetならば次が成り立つ:

        $E$に属す\underline{任意の}射$z \to x, w \to z \times_{x} y$について
        $w \to z \times_{x} y \to z$が性質$\property$を持つ.
        \[
        \begin{tikzcd}
            w \ar[r]\ar[rd]& z \times_{x} y \ar[r]\ar[d]& y \ar[d]\\
            {} & z \ar[r]& x
            \ar[ul, phantom, near end, "\ulcorner"]
        \end{tikzcd}
        \]
    \end{Lemma}
    \begin{proof}
        \[
        \begin{tikzcd}
                    &                                                                  &  & v \arrow[rr] \arrow[rrd]                             &                                   & x \times u \arrow[r] \arrow[d] & x \arrow[d] \\
        w \arrow[r] & x \times u \times u' \times v \arrow[rr] \arrow[rru] \arrow[rrd] &  & x \times u \times u' \arrow[d] \arrow[rru] \arrow[r] & x \times u' \arrow[d] \arrow[rru] & u \arrow[r]                    & y           \\
                    &                                                                  &  & u \times u' \arrow[r] \arrow[rru]                    & u' \arrow[rru]                    &                                &            
        \end{tikzcd}
        \]
    \end{proof}

    \begin{Remark}
        圏$\cat{C}$を(通常の意味の)スキームの圏,
        $E$を開埋め込み射全体が成すクラスとすれば,
        「$E$-localな性質」は通常の意味での「localな性質」と同義である.
    \end{Remark}

    \begin{Remark}
        %% admissible classの定義の理由
    \end{Remark}

    \subsubsection{チャート付き代数的空間対象(の射)の性質}
    \begin{Def}[チャート付き代数的空間対象(の射)の性質]
        圏$\cat{C}, \cat{S}$は上記(\ref{remark:catC_S_class_E})のとおりとする.

        \begin{enumerate}
        \item 
            $\cat{S}$の$\cat{C},E$チャート付き代数的空間対象$x$と,
            圏$\cat{C}$の対象に定義された性質$\property$を考える.
            対象$x$が性質$\property$を持つとは,
            $x$の任意の$\cat{C}, E$-representable cover $\{x_i \to x\}$について
            各$x_i$が性質$\property$を持つということ.

        \item
            $\cat{S}$の$\cat{C},E$チャート付き代数的空間対象の射$f \colon x \to y$と,
            圏$\cat{C}$の射に定義された$E$ local on source and targetな性質$\property$を考える.
            射$f$が性質$\property$を持つとは,
            以下の可換図式において
            $\cat{C}$の射$v \to u$が性質$\property$を持つということ.
            ただし$u, v \in \catOb \cat{C}$かつ
            $u \to y$と$v \to u \times_{y} x$は$E$に属す幾何的全射とする.
            \[
            \begin{tikzcd}
                {\color{blue}v} \ar[r, red]&
                    u \times_{y} x \ar[r]\ar[d]& {\color{blue}u} \ar[d, red]\\
                {} & x \ar[r, "f"']& y
                \ar[ul, phantom, near end, "\ulcorner"]
            \end{tikzcd}
            \]
        \end{enumerate}
    \end{Def}

    \begin{Lemma}
        圏$\cat{C}, \cat{S}$は上記(\ref{remark:catC_S_class_E})のとおりとする.
        圏$\cat{C}$の射の性質$\property$を考える.
        $\cat{S}$の表現可能な射が表現可能な射として性質$\property$を持つならば,

        性質$\property$を持つ射全体が成すクラスが$E$-local on sourceならば,
        $\cat{C}, E$チャート付き代数的空間対象の射として性質$\property$を持つことは同値.
    \end{Lemma}
    \begin{proof}
        以下の圏$\cat{S}$における図式を見よ.
        青で示した対象は$\cat{C}$の対象であり,赤で示した射はクラス$E$に属す.
        射$x \to y$を表現可能な射とする.
        \[
        \begin{tikzcd}
            {\color{blue}v} \ar[r, red]&
                {\color{blue}u \times_{y} x} \ar[r]\ar[d]& {\color{blue}u} \ar[d, red]\\
                {} & x \ar[r, "\text{``$\property$"}"']& y
            \ar[ul, phantom, near end, "\ulcorner"]
        \end{tikzcd}
        \]
        
        $x \to y$が表現可能な射として性質$\property$を持つならば,
        $\cat{C}$の射$u \times x \to u$が性質$\property$を持つ.
        $\property$が$E$-local on sourceという仮定から,
        $v \to u$も性質$\property$を持つ.

        $x \to y$が$\cat{C}, E$チャート付き代数的空間対象の射として性質$\property$を持つならば,
        $v \to u$は性質$\property$を持つ.
        $\property$が$E$-local on sourceという仮定から,
        圏$\cat{C}$の射$u \times x \to u$も性質$\property$を持つ.
    \end{proof}

    \begin{Lemma}
        圏$\cat{C}, \cat{S}$は上記(\ref{remark:catC_S_class_E})のとおりとする.
        $\property$を$\cat{C}$の射に定義された性質とする.
        \begin{enumerate}
        \item
            性質$\property$が$\cat{C}$の射の性質として
            合成のもとで安定 (stable under composition) ならば,
            $\property$は$\cat{C}, E$チャート付き代数的空間対象の射の性質としても
            合成のもとで安定である.
        
        \item
            性質$\property$が$\cat{C}$の射の性質として
            基底変換のもとで安定 (stable under base change) ならば,
            $\property$は$\cat{C}, E$チャート付き代数的空間対象の射の性質としても
            基底変換のもとで安定である.
        \end{enumerate}
    \end{Lemma}
    \begin{proof}
        
    \end{proof}

\section{チャート付き代数的空間対象の性質}
%% Eの被覆がsubcanonical topologyを与える.
%% diagonal mor. が表現可能

%% 貼り合わせ可能性 
%% %% https://stacks.math.columbia.edu/tag/0ADT
%% %% https://stacks.math.columbia.edu/tag/02W5

\section{スキーム,代数的空間,代数的スタック}
\subsection{環の景}
    圏$\opcat{\Ring}$の射の性質として
    開埋め込み,fppf 射,fpqc 射を定義する.
    環と加群の定義,
    モノ射,エピ射,
    (忠実)平坦,(形式的に)滑らかな・不分岐・エタールな射,
    有限表示射の定義は既知とする.

    \begin{Def}[環の開埋め込み射]\label{def:open_imm_ring}
        環の射(準同型)$\phi \colon R' \to R$を考える.
        環の射$\phi \colon R' \to R$が平坦,モノ,有限表示であるとき
        $\phi$は開埋め込みであるという.
    \end{Def}
    通常の意味の開埋め込みとこの定義の関係は \cite{SP} 025G を参照せよ.

    この定義を用いて環の圏を台圏 (underlying category) とする景を定義する.

    \begin{Def}[環のZariski / 平滑 / エタール / fppf 景]
        $\opcat{\Ring}$に次のように Grothendieck 位相を定義する.

        記号$\mu$を表(\ref{table:top_tau_and_mu})にあるいずれかの組とする.
        圏$\opcat{\Ring}$の対象$R$に対して,
        $\mu$である$\opcat{\Ring}$の射の集合$\{S_i \to R\}_i$であって
        合併的に全射であるものを全体のクラスを$\Cov(R)$とする.

        以上で定まる景の名前と記号は表(\ref{table:top_tau_and_mu})のとおりとする.
        $\Cov(R)$の元はこの景における$R$の被覆と呼ばれる.
    \end{Def}

    \begin{table}[htb]
    \centering
    \caption{環の景の名前,記号,対象の種類,被覆の種類}
    \label{table:top_tau_and_mu}
    \begin{tabular}{@{}llll@{}}
        \toprule
        名前 & 記号 & $\mu$ \\ \midrule
        Zariski 大景 & $\ZAR$ & 開埋め込み射 \\
        平滑 大景 & $\SM$ & 平滑 (smooth) 射 \\
        エタール大景 & $\ET$ & エタール射 \\
        fppf 大景 & $\FPPF$ & 平坦かつ局所有限表示な射 \\
        fpqc 大景 & $\FPQC$ & 平坦射 \\ \bottomrule
    \end{tabular}
    \end{table}

    \begin{Remark}
        平坦な環の射$\phi \colon S \to R$について次が同値であることに注意.
        \begin{itemize}
            \item $\phi$は忠実平坦である.
            \item $\phi$から誘導される射$\Spec R \to \Spec S$は全射である($\Spec$は標準的な意味のもの).
            \item $\phi$は幾何的全射(定義\ref{def:geo_cov})である.
        \end{itemize}
        したがって環$R \in \Ring$の$\FPQC$における被覆$\{S_i \to R\}_{i \in I}$について,
        ここから誘導される射$\prod_{i \in I} S_i \to R$は忠実平坦である.
        上で定義した景の被覆はいずれも平坦な射から成るので,
        いずれの景でも同様にして忠実平坦射が得られる.
    \end{Remark}

    \begin{Def}
        $\Ring$上の前層の圏を$\PShv(\opcat{\Ring})=\Set^{\opcat{\Ring}}$と書く.
        景$\mathcal{S}$上の層の圏を$\Shv(\mathcal{S})$と書く.
    \end{Def}

    \begin{Lemma}
        記号$\tau$を Zariski, ET, SM, FPPF, FPQC のいずれかとする.
        $\PShv(\opcat{\Ring}_{\tau})$と$\Shv(\opcat{\Ring}_{\tau})$は
        完備 (complete) かつ余完備 (cocomplete) である.
    \end{Lemma}
    \begin{proof}
        完備性は引き戻し (pullback) と終対象 (terminal object) の存在と同値であり,
        余完備性は押し出し (pushout) と始対象 (initinal object) の存在と同値である.
        %% TODO
    \end{proof}

\subsection{アフィンスキーム}
    アフィンスキームの圏は$\opcat{\Ring}$から米田関手を用いて構成される.

    \begin{Def}
        $\Ring$上の前層の圏を$\PShv(\opcat{\Ring})=\Set^{\opcat{\Ring}}$と書く.
        景$\mathcal{S}$上の層の圏を$\Shv(\mathcal{S})$と書く.
    \end{Def}

    \begin{Def}[表現可能関手]
        環$R \in \opcat{\Ring}$について,関手$\ftor{R}$を次のように定義する.
        \begin{defmap}
            \ftor{R}\colon & \opcat{\Ring}& \to& \Set \\
            \textbf{\underline{対象}:}& S& \mapsto& \Hom_{\opcat{\Ring}}(S,R) \\
            \textbf{\underline{射}:}& \psi& \mapsto& (\circ \psi)
        \end{defmap}
        この関手$\ftor{R}$を環$R$で表現される関手という.
    \end{Def}

    \begin{Lemma}
        任意の環$A \in \opcat{\Ring}$について,
        関手$\ftor{A} \colon \opcat{\Ring} \to \Set$は
        景$\ZAR, \ET, \SM, \FPPF$上の層である.
    \end{Lemma}
    \begin{proof}
        \cite{SP} 023Pを参照せよ.
    \end{proof}

    \begin{Def}
        環で表現可能な fppf 大景$\FPPF$上の層をアフィンスキームと呼ぶ.
        アフィンスキームの射は層としての射とする.
        アフィンスキームの圏を$\Aff (\subset \Shv(\FPPF))$と書く.
    \end{Def}

\subsection{スキーム}
    スキームは,アフィンスキームからの表現可能な開埋め込み射による幾何的被覆を持つ層である.

    \begin{Def}[チャート付き層としてのスキーム]
        $\opcat{\Ring} \subseteq \Aff \subseteq \Shv(\FPPF)$を考える.
        ただし最初の包含関係は米田関手によって与えられる.
        $\mathbf{OpImm}$をアフィンスキームの開埋め込み射が成すクラスとする.
        層の圏$\Shv(\FPPF)$の$\Aff, \mathbf{OpImm}$チャート付き代数的空間対象をスキームという.
        スキームの圏を$\cat{Sch}$と書く.
    \end{Def}

    \begin{Prop}
        スキームの景は$\ZAR$と同型
    \end{Prop}

\subsection{代数的空間}
    \begin{Def}[チャート付き層としての代数的空間]
        $\opcat{\Ring} \subseteq \cat{Sch} \subseteq \Shv(\FPPF)$を考える.
        ただし最初の包含関係は米田関手によって与えられる.
        $\mathbf{Et}$をスキームのエタール射が成すクラスとする.
    \end{Def}

    \begin{Prop}
        代数的空間のdiagonal mapは表現可能
    \end{Prop}

\subsection{代数的スタック}
    \begin{Def}[チャート付きスタックとしての代数的スタック]
        $\FPPF$上の亜群のスタック$\stX$が代数的スタックであるとは,
        代数的空間からの表現可能な射がなす被覆
        $\{ \phi_i \colon \ftor{S}_i \to \shX \}_{i \in I}$であって,
        全ての$i \in I$について$\phi_i$が\underline{エタール射}であるものが存在する,
        ということ.
    %art. st. はschemeによるamooth coverを持つ$\FPPF(R)$上のスタック
    \end{Def}

    \begin{Prop}
        代数的スタックのdiagonal mapは代数的空間で表現可能.
    \end{Prop}

\section{考えられる変種}
% stack charted by rep. et. mor. from scheme

\printbibliography[title=参考文献]
\end{document}
