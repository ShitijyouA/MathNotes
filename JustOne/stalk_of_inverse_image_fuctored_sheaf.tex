\documentclass[a4paper]{jsarticle}
\usepackage{../math_note}

\newcommand{\ftorSh}{\mathit{Sh}}

\title{$(f^{-1}\shF)_x=\shF_{f(x)}$の証明}
\author{七条 彰紀}
\begin{document}
\maketitle

\begin{Def}
    連続写像$f: X \to Y$と$Y$上のsheaf $\shF$に対して,
    $f^{-1} \shF$を$U \mapsto \varinjlim_{f(U) \subseteq V} \shF(V)$で定まるpresheafの
    associated sheafとする.
\end{Def}

$X$上のsheaf $\shG$に対して$U \mapsto \shG(f^{-1}(U))$はsheaf $f_*\shG$を定める.
これに対応して$U \mapsto \shF(f(U))$でpresheafを定めることは,一般には出来ない.
そこで代わりに$f(U)$を含む開集合達で$f(U)$を近似しよう,というのが$f^{-1}$である.
(「それに近いもの全体」で「それ」を表現しよう,という思考は数学の他の場所にも現れる.)

このノートの目的は次の主張に$2$つの証明を与えることである.
\begin{Prop}[$*$]
    $f: X \to Y$を連続写像とし,$\shF$を$Y$上のsheafとする.
    この時,$x \in X$について$(f^{-1}\shF)_x=\shF_{f(x)}$.
\end{Prop}

直接の証明は次の通り.
\begin{proof}
    sheafificationとtaking stalk at $x$が可換であることは既知とする
    \footnote
    {
        これはsheafification $\ftorSh$がleft adjoint functorであり,
        taking stalk at $x$ $\varinjlim_{x \in U}$がcolimitであることによる.
        $\ftorSh$がleft adjoint functorであることは,
        次のpdfファイルに証明を書いた:
        \url{https://github.com/ShitijyouA/MathNotes/blob/master/Hartshorne_AG_Ch2/section1_ex.pdf}
    }
    したがって我々は次を示せば良い.
    \[ \varinjlim_{V \in \mathcal{D}} \shF(V)=\varinjlim_{V \in \mathcal{S}} \shF(V) \]
    ただし$\mathcal{D}, \mathcal{S}$は以下のようなdirect systemである.
    (TODO:
        $(f^{-1}\shF)_x$は厳密にはdirect limit $(f^{-1}\shF)(U)$が成す
        direct systemのdirect limitなので,ここの翻訳は証明が必要.
    )
    \[
        \mathcal{D}=\{ V \supseteq V' \mid \Exists{U \subseteq X} x \in U, f(U) \subseteq V' \subseteq V \},~~
        \mathcal{S}=\{ V \supseteq V' \mid f(x) \in V' \subseteq V \}.
    \]
    ここに現れる$X,Y$の部分集合はすべて開集合である.
    $\mathcal{D} \subseteq \mathcal{S}$は明らか.
    一方,$f(x) \in V$ならば$x \in f^{-1}(V)$である.
    $f$は連続だから$f^{-1}(V)$は開集合であり,
    $x \in U \subseteq f^{-1}(V)$すなわち$f(x) \in f(U) \subseteq V$なる開集合$U \subseteq X$が存在する.
    よって$\mathcal{D} \supseteq \mathcal{S}$も得られる.
    direct systemが同じものであるから,2つのdirect limitも同じである.
\end{proof}

上記の主張($*$)は次の主張の系としても得られる.
\begin{Claim}
    2つの写像$X \xrightarrow{f} Y \xrightarrow{g} Z$を連続写像とし,
    $\shF$を$Z$上のsheafとする.
    この時,$f^{-1}g^{-1}\shF=(g \circ f)^{-1}\shF$
\end{Claim}
\begin{proof}
    functor $f^{-1}, g^{-1}$はそれぞれ$f_*,g_*$のleft adjoint functorである.
    \footnote
    {
        次のpdfファイルの``Ex1.18 Adjoint Property of $f^{-1}$."に証明を書いた:
        \url{https://github.com/ShitijyouA/MathNotes/blob/master/Hartshorne_AG_Ch2/section1_ex.pdf}
    }.
    一方,$g_* f_*$は定義から明らかに$(g \circ f)_*$に等しい.
    なので次が成り立つ.
    \[
        \Hom(f^{-1}g^{-1}\shF, -)
        \iso \Hom(\shF, g_* f_* -)
        =\Hom(\shF, (g \circ f)_* -)
        \iso \Hom((g \circ f)^{-1} \shF,  -).
    \]
    すなわち,$f^{-1}g^{-1}, (g \circ f)^{-1}$は
    どちらも$g_* f_*(=(g \circ f)_*)$のleft adjoint functorである.
    adjoint functorの一意性から,
    $f^{-1}g^{-1}\shF=(g \circ f)^{-1}\shF$.
\end{proof}

この主張の系として($*$)の証明を与える.
\begin{proof}
    $i: \{x\} \hookrightarrow X$を包含写像とする.
    証明は$(i^{-1}f^{-1} \shF)(\{x\})$を二通りの方法で計算することによる.
    
    最初に,$i^{-1}f^{-1} \iso (f \circ i)^{-1}$を用いる.
    $f \circ i$は$\{x\} \to \{f(x)\} \subseteq Y$なる写像であるから,
    これはpresheaf $\{x\} \supseteq U  \mapsto \varinjlim_{f \circ i(U) \subseteq V} \shF(V)$,
    すなわち次のpresheafのassociated sheafである.
    1点空間上のpresheafはsheafだから,実際にはsheafificationは不要である.
    \[
        ((f \circ i)^{-1}\shF)(U)=
        \begin{cases}{}
            0 & (U=\emptyset) \\
            \shF_{f(x)} & (U=\{x\}).
        \end{cases}
    \]
    よって,この方針では$(i^{-1}f^{-1} \shF)(\{x\})=\shF_{f(x)}$.

    一方,$(i^{-1}f^{-1} \shF)=(i^{-1}(f^{-1} \shF))$と
    考えて計算すると,次のように成る.
    \[
        (i^{-1}f^{-1} \shF)(\{x\})
        =(i^{-1}(f^{-1} \shF))(\{x\})
        =\varinjlim_{\{x\} \subseteq V \subseteq X}(f^{-1} \shF)(V)
        =(f^{-1} \shF)_{x}.
    \]
    ここでもsheafificationが不要であることを用いている
    \footnote
    {
        $(i^{-1}(f^{-1} \shF))(\{x\})$は,
        $V \mapsto \varinjlim_{i(U) \subseteq V \subseteq X}(f^{-1} \shF)$
        の\textbf{sheafificationの}
        $U=\{x\}$におけるsection,である.
    }.
\end{proof}
\end{document}
