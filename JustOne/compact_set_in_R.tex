\documentclass[]{jsarticle}
\usepackage{../math_note}

\newcommand{\U}{\mathfrak{U}}

\title{実数集合$\R$のコンパクト集合}
\author{七条 彰紀}

\begin{document}
\maketitle

\section{準備}
\begin{Lemma}\label{lemma:1}
    コンパクト空間の閉部分集合はコンパクト.
\end{Lemma}
\begin{proof}
    $C \subseteq X$で$X$はコンパクトだとする.
    $C$の開被覆 $\{U_{\lambda}\}_{\lambda \in \Lambda}$をとると,
    $\{C^c \cup U_{\lambda}\}_{\lambda \in \Lambda}$は$X$の開被覆になる.
    $X$がコンパクトであることから,
    以下を満たす有限部分集合$\Lambda_f \subseteq \Lambda$が存在する.
    \[ C \subseteq X=\bigcup_{\lambda \in \Lambda_f} (C^c \cup U_{\lambda}). \]
    よって$C \subseteq \bigcup_{\lambda \in \Lambda_f} U_{\lambda}$.
\end{proof}

\begin{Lemma}\label{lemma:2}
    ハウスドルフ空間のコンパクト部分集合は閉集合.
\end{Lemma}
\begin{proof}
    $X$がハウスドルフ空間で,$C \subseteq X$はコンパクトだとする.
    $C^c$の任意の点が内点であること,
    すなわち$C^c$の任意の点が$C^c$に含まれる開近傍を持つことを示そう.

    $x \in C^c$と$y \in C$を任意に取ると,
    $X$がハウスドルフであることから,
    以下を満たす開集合$U_y, V_y \subset X$がとれる.
    \[ x \in U_y,~~~ y \in V_y,~~~ U_y \cap V_y = \emptyset. \]
    この時$C=\bigcup_{y \in C} V_y$.
    $C$はコンパクトだから,$y_1,\dots,y_r \in C$を適当に選ぶことで
    $C \subseteq \bigcup_{i=1}^r V_{y_i}$とできる.
    $U_{y_i} \subseteq (V_{y_i})^c$から,
    \[ U=\bigcap_{i=1}^r U_{y_i} \]
    とおけば$x \in U$かつ$U \subseteq C^c$.
\end{proof}

\begin{Remark}\label{remark:1}
    このノートでは以下を公理として認める. \\
    (\textbf{カントールの公理}あるいは\textbf{区間縮小法の原理})
    閉区間の減少列 $\R \supset I_1 \supsetneq I_2 \supsetneq \cdots$が任意に与えられた時,
    $\bigcap_{i \in \N} I_i \neq \emptyset$.
\end{Remark}

\section{主定理}
\begin{Thm}
    $\R$のコンパクト部分集合は有界閉集合.
\end{Thm}
\begin{proof}
    $\R$のコンパクト部分集合 $C$を考える.
    \[ C \subset \R=\bigcup_{n \in \N} (-n,+n) \]
    という開被覆を考えると,
    $C$がコンパクトであることから,
    この内の有限個で$C$は被覆できる.
    \[ C \subset (-n_1,+n_1) \cup (-n_2,+n_2) \cup \dots \cup (-n_r,+n_r). \]
    $N=\max_{1 \leq i \leq r} n_i$とすれば
    $C \subset (-N, +N)$.
\end{proof}

\begin{Thm}
    $\R$の有界閉集合はコンパクト.
\end{Thm}
証明を二つ述べる.
\begin{proof}
    補題\ref{lemma:1}から,
    有界閉区間$[a,b] ~~(a,b \in \R, a < b)$が
    コンパクトであることを示せば十分である.
    $[a,b]$がコンパクトでないとすると,
    有限部分被覆を持たない$[a,b]$の開被覆 $\U$が存在する.

    これは$[a_0,b_0]=[a,b]$からはじめて$\{[a_k, b_k]\}_{k \geq 0}$を次のように作る.
    すなわち,
    $[a_k,b_k]$が構成されている時,
    $[a_{k+1},b_{k+1}]$は
    \[
        \left[ a_k, \frac{a_k+b_k}{2} \right],
        \left[ \frac{a_k+b_k}{2}, b_k \right]
    \]
    のうちで,
    $\U$の有限部分で被覆できないものである.
    こうして出来る$[a,b]$の閉部分集合鎖
    \[ [a,b]=[a_0,b_0] \supsetneq [a_1,b_1] \supsetneq \dots \]
    は無限に伸ばすことが出来る
    (特に任意の$k \geq 0$について$[a_k,b_k]$は空でない).
    実際,
    $\left[ a_k, \frac{a_k+b_k}{2} \right],\left[ \frac{a_k+b_k}{2}, b_k \right]$
    の両方が$\U$の有限部分で被覆できるのであれば,
    前者,後者を覆い尽くす有限部分被覆を合わせて,
    $[a_k,b_k]$が$\U$の有限部分で被覆できる.
    これを繰り返すと,
    結局$\U$は$[a,b]$の有限部分被覆をもつということになってしまう.

    こうして出来た閉区間列の幅は
    $0 < |a_k-b_k| \leq \frac{|a-b|}{2^k}$の様に縮小していく.
    したがって注意\ref{remark:1}で述べたカントールの公理から,
    $c \in \bigcap_{k \geq 0} [a_k,b_k] \subseteq [a,b]$がとれる.
    $\U$は$[a,b]$の開被覆だから,$c \in U$なる$U \in \U$が存在する.
    $U$は開集合だから,十分小さい$\varepsilon>0$について
    \[ (-\varepsilon+c, c+\varepsilon) \subseteq U \]
    とできる.
    $\varepsilon$に対して,
    整数$N$を$\frac{|a-b|}{2^N}<\frac{\varepsilon}{2}$が成り立つものとすれば,
    次のようになる.
    \[ [a_N, b_N] \subseteq (-\varepsilon+c, c+\varepsilon) \subseteq U. \]
    これは$[a_k, b_k]$のとり方
    ($[a_k, b_k]$は$U$の有限部分で被覆できない)に反する.
\end{proof}

\begin{proof}
    閉区間$[a,b] ~(a<b)$を考えれば十分であることは
    1つめの証明と変わらない.
    $[a,b]$の\kenten{任意}の開被覆$\U$をとる.
    明らかに,$\U$は$[a,x] ~(x \in [a,b])$の開被覆でもある.
    そこで$I$を以下のように取る.
    \[ I=\{ x \in [a,b] \mid \text{$\U$は$[a,x]$の有限部分被覆をもつ}\}. \]
    $b \in I$が我々の目標である.

    $c \in I$を任意に取ると,
    $\U$は$[a,b]$の被覆であることから$c \in U \in \U$なる$U$が存在する.
    $U$が開集合であることから,
    十分小さい$2\varepsilon>0$について
    \[ [-\varepsilon+c, c+\varepsilon] \subsetneq (-2\varepsilon+c, c+2\varepsilon) \subseteq U. \]
    $\U$がもつ$[a,c]$の有限開被覆に$U$を付け加えると,
    $[-\varepsilon+c, c+\varepsilon] \cap [a,b] \subset I$が分かる.
    すなわち,$I$の任意の点は$[a,b]$の位相で閉近傍をもつ.

    以上から直ちに$I$が$[a,b]$の開集合であることが分かる.
    また,$I$が閉集合であることも,次の様に考えれば分かる.
    $l$を$I$の集積点としよう.
    集積点の定義から,$l$の開近傍$U \in \U$は$I$と交わる.
    開近傍の閉包を取れば,これは任意の$l$の閉近傍と言い換えても良いことが分かる.
    この閉近傍を十分小さく取れば$[c,l] \subset I$の様に出来る.
    これを$I$の点$c$の閉近傍と見れば,前段落から$l \in I$が得られる.

    よって$I$は$[a,b]$の閉かつ開な部分集合.
    $a \in I$から$I \neq \emptyset$で,
    しかも$[a,b]$は連結だから,$I=[a,b]$.
\end{proof}

\end{document}
