\documentclass[a4j]{jarticle}
\usepackage{../math_note}

\title{可測集合が可測集合に写る事}

\begin{document}
\maketitle

\begin{Thm}
    集合$X$上の全単射連続写像$T$と0でない定数$\tau$があって,
    集合$X$上の外測度$\mu$に対して,
    \[ \mu \circ T=\tau \cdot \mu \]
    が成立するものとする.
    このとき,$E$が$\mu$-可測集合ならば$T^{-1}(E)$も$\mu$-可測集合.
    特に$T$が同相写像ならば$T(E)$も$\mu$-可測集合.
\end{Thm}
\begin{proof}
    $T,\tau$の関係と,$T$が全単射であることから
    $\tau \cdot (\mu \circ T^{-1})=\mu \circ (T \circ T^{-1})=\mu$が成立する.

    任意の$A \subset X$を取る.
    \begin{align*}
        \mu(A)
        &=\mu(A \cap E)                              &&+&&   \mu(A \cap E^c) \\
        &=\tau \cdot \mu(T^{-1}(A \cap E))           &&+&&   \tau \cdot \mu(T^{-1}(A \cap E^c)) \\
        &=\tau \cdot \mu(T^{-1}(A) \cap T^{-1}(E))   &&+&&   \tau \cdot \mu(T^{-1}(A) \cap T^{-1}(E^c)) \\
        &=\tau \cdot \mu(T^{-1}(A) \cap T^{-1}(E))   &&+&&   \tau \cdot \mu(T^{-1}(A) \cap T^{-1}(E)^c)
    \end{align*}
    $\mu(A)=\tau \cdot \mu \circ T^{-1}(A)$を用いて
    \begin{align*}
        \tau \cdot \mu \circ T^{-1}(A)&=\tau \cdot (\mu(T^{-1}(A) \cap T^{-1}(E))+\mu(T^{-1}(A) \cap T^{-1}(E)^c)) \\
        \mu(T^{-1}(A))&=\mu(T^{-1}(A) \cap T^{-1}(E))+\mu(T^{-1}(A) \cap T^{-1}(E)^c)
    \end{align*}
    任意の集合$A$は$T(A')$と表現できる($T$の全射性)から,
    \[ \mu(A')=\mu(A' \cap T^{-1}(E))+\mu(A' \cap T^{-1}(E)^c) \]
    よって$T^{-1}(E)$は$\mu$-可測.
\end{proof}
実際の所,定数$\tau$が存在するという条件は「加法準同型な全単射写像$U$があって$\mu \circ T=U \circ \mu$」
と書けるのだが,このような$U$であって更に連続なものに限ると,$U(x)=\tau x$の形になるしか無い.

\begin{Example}
    $\mathbb{R}^n$上のルベーグ測度を考えることにする.
    定数$c$に対して$T(E)=E-c, \tau=1$とすればこれは定理の仮定を満たす.
    したがって平行移動に対して可測性は不変.

    また,0でない定数$a$を取ると,$T(E)=aE$に対して$\tau=|a|^n$とすればこれも定理の仮定を満たすから,
    定数倍に対しても可測性は不変.
\end{Example}
\end{document}

%連続関数fはボレル可測関数であることの証明.
%開集合Eについて,f^{-1}(E)は開集合.
%したがってf^{-1}(E)はボレル可測集合.
%よってfはボレル可測関数.

