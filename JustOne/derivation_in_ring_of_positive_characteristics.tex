\documentclass[a4paper]{jsarticle}
\usepackage[]{../math_note, enumitem}

\newcommand{\Der}{\mathrm{Der}}

\title{正標数の環における導分}
\author{七条 彰紀}

\begin{document}
\maketitle

\begin{Def}[Derivation, $k$-Derivation.]
    $A$ :: ring, $M$ :: moduleとする.
    任意の$a,b \in A$に対して
    次を満たす写像$D: A \to M$をderivationとよぶ.
    \begin{enumerate}[label=(\roman*)]
        \item $D(a+b)=D(a)+D(b)$.
        \item $D(ab)=D(a) \cdot b+a \cdot D(b)$.
    \end{enumerate}
    (ii)はLeibniz Formula (or Rule)と呼ばれる.
    以下,必要に応じて$D(a)$を$Da$と表記する.

    準同型$f: k \to A$によって$A$を$k$-moduleとみなせる時,
    $D \circ f=0$を満たすderivation $D$を$k$-derivationと呼ぶ.

    $\Der(A,M)$で$A \to M$のderivation全体を表す.
    $\Der(A,A)$は$\Der(A)$と略す.
    $\Der_k(A,M)$で$A \to M$の$k$-derivation全体を表す.
    $\Der_k(A,A)$は$\Der_k(A)$と略す.
\end{Def}
$\Der(A,M), \Der_k(A,M)$が$A$-moduleになることは明らか.
$a \in A, n \geq 0$について$Da^n=n a^{n-1} Da$が
成り立つことは帰納法を用いて簡単に示せる.

次が成り立つ.
\begin{Prop}
    $A$ :: ring, $D \in \Der(A)$, $a,b \in A$, $n \in \Z_{\geq 0}$とする.
    \[ D^n(ab)=\sum_{i=0}^n \binom{n}{i} (D^i a)(D^{n-i} b). \]
    ただし$D^0=\id{A}$とする.
\end{Prop}
証明は$n$についての帰納法に拠る.

今,$A$ :: ringが正標数$c>0$
\footnote
{
    $f: \Z \to A$を唯一の写像$1_{\Z} \mapsto 1_A$とすると,
    $f^{-1}((0))=\ker f \subseteq \Z$は$\Z$のイデアルであり,
    したがって$\ker f=(c)$となる$c \in \Z_{\geq 0}$が存在する.
    この$c$を$A$の標数と呼ぶ.
}
を持つとしよう.
$c$が素数ならば,
$\binom{c}{i}$は$i=1,\dots,c-1$について$c$の倍数であるから,
次が成り立つ.
\[ D^c(ab)=(D^c a) \cdot b+a \cdot (D^c b). \eqno{(*)}\]
すなわち,$D^c \in \Der(A)$となる.

一方,$c$が素数でない,
すなわち合成数でない時には($*$)が成り立たないことがある.

\begin{Example}
    $A=(\Z/4\Z)[x], D=x \frac{d}{dx}$とする.
    この場合,$A$の標数は$4$.
    ただし$\frac{d}{dx}$は$x$についての通常の微分であり,
    明示すれば$\frac{d}{dx} x=1, \frac{d}{dx} 1=0$を満たす.
    $\frac{d}{dx} \in \Der(A)$と$\Der(A)$ :: $A$-moduleより$D \in \Der(A)$.
    $Dx=x \cdot 1=x$だから,$D^4(x^2)$は次のように成る.
    \[ D^4(x^2)=D^3(D(x^2))=D^3(2x^{2-1} (Dx))=D^3(2x^2)=\dots=2^4 x^2=0. \]
    一方,$D^4(x^2)=D^4(x \cdot x)$と考えて($*$)の右辺を計算すると,次のよう.
    \[ (D^4 x) \cdot x+x \cdot (D^4 x)=2x^2 \neq 0. \]
    なので($*$)は成立しない.
\end{Example}

\begin{Example}
    文字$c$を導入して一般化を試みる.
    $A=(\Z/c\Z)[x], D=x \frac{d}{dx}$とする.
    \[ D^c(x^2)=2^c x^2,~~~ 2x^2=D^c(x) \cdot x+x \cdot D^c(x). \]
    この二つはほとんどの$c$で異なり,そのとき($*$)の反例と成る.
    しかし,よく知られている通り,
    Fermatの小定理の逆には反例が存在する.
    なので,ここで与えた$A,D$は,
    例えば$c$がCarmichael数($561, 1105, 1729, 2465, 2821, \dots$)
    である場合についての($*$)の反例にならない.
    (他に例がないか探してみると,$c=341(=11 \cdot 31), 645, 1387, 1905, 2047$でも反例にならない.)
\end{Example}

\begin{Example}
    $c>0$を\textbf{square-freeでない}合成数とし,
    $c$の互いに異なる素因数の積を$r$とする.
    ($r$は$c$のradicalと呼ばれる.)
    $c > r$に注意せよ.
    $A=(\Z/c\Z)[x], D=x^{r+1} \frac{d}{dx}$とする.
    この時,$Dx=x^{r+1}$.
    $t \geq 1$とすると,$D^c(x^t)$は次のように成る.
    \[ D^c(x^t)=\delta_t x^{cr+t},~~~ \delta_t=\prod_{i=0}^{c-1}(ir+t). \]
    したがって$D^c(x^r)$は$\delta_r x^{(c+1)r}=c! r^c \cdot x^{(c+1)r}$.
    一方,($*$)の右辺は次のように成る.
    \[ D^c(x \cdot x^{r-1})=D^c(x) \cdot x^{r-1}+x \cdot D(x^{r-1})=(\delta_1+\delta_{r-1}) x^{(c+1)r}. \]
    $c>r$より$\delta_1+\delta_{r-1}$が$c$の倍数でないことが示される.
    (この証明は難しいと思われる.計算機で$c<100$の範囲で正しいことを確かめた.)
    よって($*$)が成立しない.
\end{Example}
残念ながら,Carmichael数はsquare-freeである
\footnote{\url{http://mathworld.wolfram.com/CarmichaelNumber.html}などを参照せよ.}.
上で挙げたその他の$c$もsquare-freeである.
標数がCarmichael数ならば常に($*$)が成り立つ可能性もあるが,
それはFuture Workとしよう.

標数$c$が合成数であっても($*$)が成り立つのはどんな場合か,
という問に対しては次がひとつの答えを与える.
\begin{Prop}
    $A, k$ :: ring, $D \in \Der_k(A)$とする.
    $A$の標数$c$は合成数であるとする.
    $A$は次を満たすとする.
    \begin{enumerate}[label=(\arabic*)]
        \item $A$の任意の元が$G \subseteq A$の元の積の$k$線型結合として書ける.
        \item $G$の任意の元$g$について$D^2g=0$.
    \end{enumerate}
    この時,$D^c=0$.
    したがって任意の$a,b \in A$について($*$)の等号が成り立つ.
\end{Prop}
この命題の仮定のうち,条件(2)以外は次のような環で成り立つ:
$k$上の多項式環・形式的ベキ級数環,及びその剰余環,$k$の元による局所化,テンソル積,直積.

これは次の補題から得られる.
\begin{Lemma}
    $A, k$ :: ring, $D \in \Der_k(A)$とする.
    $x \in A$と$n,k \in \Z_{\geq 0}$について,次が成り立つ.
    \[ D^k x^n=\sum_{i=0}^{k-1} \binom{k-1}{i} n^{\underline{i+1}} x^{n-(i+1)} (Dx)^i (D^{k-i} x). \]
    ここで$n^{\underline{i+1}}=n(n-1)\cdots(n-(i+1)+1)$は降下階乗べきである.
\end{Lemma}
$D \in \Der_k(A)$は$k$線形写像であること,
及び$n^{\underline{i+1}}$が$i=0,\dots,k-1$で$(i+1)!$の倍数に成ることに気をつければ,
この補題から上の命題はすぐに出る.

\end{document}
