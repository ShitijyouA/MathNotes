\documentclass[a4paper]{jsarticle}
\usepackage[]{../math_note, enumitem}

\newcommand{\Der}{\mathrm{Der}}

\title{正標数の環における導分}
\author{七条 彰紀}

\begin{document}
\maketitle

\section{準備}
\begin{Def}[Derivation, $k$-Derivation.]
    $A$ :: ring, $M$ :: moduleとする.
    任意の$a,b \in A$に対して
    次を満たす写像$D: A \to M$をderivationとよぶ.
    \begin{enumerate}[label=(\roman*)]
        \item $D(a+b)=D(a)+D(b)$.
        \item $D(ab)=D(a) \cdot b+a \cdot D(b)$.
    \end{enumerate}
    (ii)はLeibniz Formula (or Rule)と呼ばれる.
    以下,必要に応じて$D(a)$を$Da$と表記する.

    準同型$f: k \to A$によって$A$を$k$-moduleとみなせる時,
    $D \circ f=0$を満たすderivation $D$を$k$-derivationと呼ぶ.

    $\Der(A,M)$で$A \to M$のderivation全体を表す.
    $\Der(A,A)$は$\Der(A)$と略す.
    $\Der_k(A,M)$で$A \to M$の$k$-derivation全体を表す.
    $\Der_k(A,A)$は$\Der_k(A)$と略す.
\end{Def}
$\Der(A,M), \Der_k(A,M)$が$A$-moduleになることは明らか.
$a \in A, n \geq 0$について$Da^n=n a^{n-1} Da$が
成り立つことは帰納法を用いて簡単に示せる.

次が成り立つ.
\begin{Prop}
    $A$ :: ring, $D \in \Der(A)$, $a,b \in A$, $n \in \Z_{\geq 0}$とする.
    \[ D^n(ab)=\sum_{i=0}^n \binom{n}{i} (D^i a)(D^{n-i} b). \]
    ただし$D^0=\id{A}$とする.
\end{Prop}
証明は$n$についての帰納法に拠る.

今,$A$ :: ringが正標数$n>0$
\footnote
{
    $f: \Z \to A$を唯一の写像$1_{\Z} \mapsto 1_A$とすると,
    $f^{-1}((0))=\ker f \subseteq \Z$は$\Z$のイデアルであり,
    したがって$\ker f=(n)$となる$n \in \Z_{\geq 0}$が存在する.
    この$n$を$A$の標数と呼ぶ.
    $A$が整域,すなわち$(0) \subseteq A$が素イデアルならば,
    $\ker f$も素イデアルになり(可換環論の基本的命題),
    したがって標数$n$は素数になる.
}
を持つとしよう.
$n$が素数ならば,
$\binom{n}{i}$は$i=1,\dots,n-1$について$n$の倍数であるから,
次が成り立つ.
\[ D^n(ab)=(D^n a) \cdot b+a \cdot (D^n b). \eqno{(*)}\]
すなわち,$D^n \in \Der(A)$となる.

\section{($*$)の反例}
一方,$n$が素数でない,
すなわち合成数でない時には($*$)が成り立たないことがある.

\begin{Example}
    $A=(\Z/4\Z)[x], D=x \frac{d}{dx}$とする.
    この場合,$A$の標数は$4$.
    ただし$\frac{d}{dx}$は$x$についての通常の微分であり,
    明示すれば$\frac{d}{dx} x=1, \frac{d}{dx} 1=0$を満たす.
    $\frac{d}{dx} \in \Der(A)$と$\Der(A)$ :: $A$-moduleより$D \in \Der(A)$.
    $Dx=x \cdot 1=x$だから,$D^4(x^2)$は次のように成る.
    \[ D^4(x^2)=D^3(D(x^2))=D^3(2x^{2-1} (Dx))=D^3(2x^2)=\dots=2^4 x^2=0. \]
    一方,$D^4(x^2)=D^4(x \cdot x)$と考えて($*$)の右辺を計算すると,次のよう.
    \[ (D^4 x) \cdot x+x \cdot (D^4 x)=2x^2 \neq 0. \]
    なので($*$)は成立しない.
\end{Example}

\begin{Example}
    $n$に加えて文字$a,b$を加えて更に一般化する.
    $A=(\Z/n\Z)[x], D=x \frac{d}{dx}$とする.
    ある$a,b>0$について$(a+b)^n \neq a^n+b^n$であるとしよう.
    この時,($*$)の反例がある:
    \[ D^n(x^a \cdot x^b)=(a+b)^n x^{a+b} \neq (a^n+b^n)x^{a+b}=D^n(x^a) \cdot x^b+x^a \cdot D^n(x^b). \]
\end{Example}

一方,次の命題が成立する.
\begin{Prop}
    $n$を正整数とする.次は同値
    \footnote
    {
        Pratibha Ghatage and Brian Scott(2005),
        \textit{Exactly When Is $(a + b)^{n} \equiv a^{n} + b^{n} ~(\bmod~ n)$ ?},
        \url{http://www.jstor.org/stable/30044877}.
    }.
    \begin{enumerate}[label=(\arabic*)]
        \item $\Forall{a,b \in \Z/n\Z} (a+b)^n=a^n+b^n$.
        \item $\Forall{x \in \Z/n\Z} x^n=x$.
        \item $n$は素数またはCarmichael数.
    \end{enumerate}
\end{Prop}
\begin{proof}
    $(1) \implies (2)$の証明は$x=1+1+\dots+1$とすれば出来るし,
    $(2) \implies (1)$の証明は$x=a+b$とすれば出来る.
    $(2) \iff (3)$はFermatの小定理とCarmichael数の定義である.
\end{proof}

したがって,以上の方法では$n$がCarmichael数
($561, 1105, 1729, 2465, 2821, \dots$)であるときの($*$)の反例が作れない.
しかし,Carmichael数は常に奇数である
\footnote{ $n$が偶数の合成数の時$(-1)^{n} \bmod n=1 \neq -1$. }
から,
環$A$を多変数にすれば容易に($*$)の反例が作れる.
というよりも,一般の設定を具体的な環で再現できる.

\begin{Example}
    $n$を正の\textbf{奇数}とする.
    \[
        A=(\Z/n\Z)[x_0,\dots,x_n],
        \hspace{20pt}
        D=\sum_{i=0}^{n} x_{i+1} \frac{\partial}{\partial x_{i}}.
    \]
    ただし,$x_{n+1}=1$とする.
    このようにすると,$i=0,\dots,n$について$D^i x_0=x_{i}$となる.
    したがって$D^n(x_0 \cdot x_0)$は次のようになる.
    $n$は奇数であることに注意せよ.
    \[
        D^n(x_0 \cdot x_0)
        =x_0 x_n + \sum_{i=1}^{n-1} \binom{n}{i} x_i x_{n-i} + x_n x_0
        =2 x_0 x_n+\sum_{i=1}^{\frac{n-1}{2}} 2 \binom{n}{i} x_i x_{n-i}.
    \]
    当然,$\{x_i x_{n-i}\}_{i=1}^{\frac{n-1}{2}}$は$\Z/n\Z$上線形独立である.
    したがって$\sum$の部分が$0$になるのは,
    $i=1,\dots,n-1$について$2\binom{n}{i} \bmod n=0$となる時のみである.
    $n$は奇数であるから,
    特に$\binom{n}{i} \bmod n=0$ならば$\sum$の部分が$0$.
    このことは,次の主張の通り,$n$が合成数であるときはありえない.
\end{Example}

\begin{Prop}
    正整数$n$を考える.
    $i=1,\dots,n-1$について次式が成り立つことと,
    $n$が素数であることは同値である.
    \[ \binom{n}{i} \bmod n=\frac{n!}{i! (n-i)!} \bmod n=0. \]
\end{Prop}
\begin{proof}
    \paragraph{($\implies$).}
    $n$を合成数とし,$p$をその素因数
    \footnote
    {
        \url{https://www.anothermathblog.com/?p=72}
        では$p$を特に最小のものとしているが,
        以下の通り,この仮定は不要である
    }
    とする.
    また$m=n/p$とする.
    \[
        \binom{n}{p}
        =\frac{n(n-1)\cdots(n-p+1)}{p!}
        =m \cdot \frac{(n-1)\cdots(n-p+1)}{(p-1)!}.
    \]
    これが$n$の倍数であると仮定しよう.
    $n=m \cdot p$なので,仮定により,
    $\frac{(n-1)\cdots(n-p+1)}{(p-1)!}$は$p$の倍数である.
    特に$(n-1)\cdots(n-p+1)$が$p$の倍数.
    しかし$p-1$個の整数$n-1, \dots, n-(p-1)$はいずれも$p$と互いに素である
    \footnote
    {
        $1, \dots, p-1 \bmod n \neq 0$と$n \bmod n=0$から
        $n-1, \dots, n-(p-1) \bmod \neq 0$が得られる.
    }
    から,
    これはありえない.
    特に,$\binom{n}{i}$は$n/p$の倍数であって$n$の倍数でない.

    \paragraph{($\impliedby$).}
    $n$が素数であるとする.
    すると$1 \leq i \leq n-1$より,$i!$は$n$の倍数ではない.
    $1 \leq i \leq n-1$ならば$1 \leq n-i \leq n-1$だから,$(n-i)!$も同様.
    したがって$i! (n-i)!$は$n$の倍数ではなく,
    $\binom{n}{i}=\frac{n!}{i! (n-i)!}$は$n$の倍数.
\end{proof}

\section{($*$)の成立}
標数$n$が合成数であっても($*$)が成り立つのはどんな場合か,
という問に対しては次がひとつの答えを与える.
\begin{Prop}
    $A, k$ :: ring, $D \in \Der_k(A)$とする.
    $A$の標数$n$は合成数であるとする.
    $A$は次を満たすとする.
    \begin{enumerate}[label=(\arabic*)]
        \item $A$の任意の元が$G \subseteq A$の元の積の$k$線型結合として書ける.
        \item $G$の任意の元$g$について$D^2g=0$.
    \end{enumerate}
    この時,$D^n=0$.
    したがって任意の$a,b \in A$について($*$)の等号が成り立つ.
\end{Prop}
この命題の仮定のうち,条件(2)以外は次のような環で成り立つ:
$k$上の多項式環・形式的ベキ級数環,及びその剰余環,$k$の元による局所化,テンソル積,直積.

%これは次の補題から得られる.
%\begin{Lemma}
%    $A, k$ :: ring, $D \in \Der_k(A)$とする.
%    $x \in A$と$n,k \in \Z_{\geq 0}$について,次が成り立つ.
%    \[ D^k x^n=\sum_{i=0}^{k-1} \binom{k-1}{i} n^{\underline{i+1}} x^{n-(i+1)} (Dx)^i (D^{k-i} x). \]
%    ここで$n^{\underline{i+1}}=n(n-1)\cdots(n-(i+1)+1)$は降下階乗べきである.
%\end{Lemma}
%$D \in \Der_k(A)$は$k$線形写像であること,
%及び$n^{\underline{i+1}}$が$i=0,\dots,k-1$で$(i+1)!$の倍数に成ることに気をつければ,
%この補題から上の命題はすぐに出る.

\end{document}
