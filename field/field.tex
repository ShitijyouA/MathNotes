\documentclass[a4j]{jarticle}

\usepackage[all]{xy}
\usepackage{../math_note}
\newcommand{\Aut}{\operatorname{Aut}}
\newcommand{\Gal}{\operatorname{Gal}}
\newcommand{\Irr}{\operatorname{Irr}}

\begin{document}
\section{$\mKer(\bar{\phi})$は極大イデアル(p.182, l.5)}
    簡単のため、$I=\mKer(\phi)$とおく。
    定義から、$I$の元である多項式$f$に$\theta$を代入すれば0となる。
    そのため次の$\bar{\phi}$が$\phi$から自然に導かれる。

    \begin{eqnarray*}
        \bar{\phi}: \quad K[x]/I &\to& K(\theta) \\
                    f(x)+I &\mapsto& f(\theta) 
    \end{eqnarray*}

    補題4.5より、$\operatorname{Irr}(\theta, K, x)$が既約元であることを示せば
    $I$は極大イデアルであることが分かる。
    そのことから次のように論理がつながる。
    \[
        \mIm(\bar{\phi})が K と \theta を含む体 \footnote{$x+I \in K[x]/I$が$\theta$に行く} \implies
        \mIm(\bar{\phi}) = K(\theta) \implies
        K[x]/Iも体なので\bar{\phi}はK \mdash 同型
    \]

    まず、一変数多項式環$K[x]$はPIDである。これは$K[x]$がユークリッド環であること、
    さらにユークリッド環はPIDであることから分かる。詳細はp.87の例4.6とp.88下。

    $I$は$K[x]$のイデアルで、しかも$K[x]$はPIDだから、$I$は$f(x) \in K[x]$から生成される。
    この生成元$f(x)$は$I$の元の内で次数が最小のものである。
    なぜなら$I$は$\theta$を代入すると0になる多項式全体で、
    $I$の元であって$f(x)$より次数が大きい多項式\footnote{例えば$x \cdot f(x)$など}が生成するイデアルは$f(x)$を含まず、
    $I$とは一致しないからである。
    p.88下から、これは$f(x)$が既約元であることを言っている。

    \subsection{注意: 最小多項式は唯一つ}
        上のような$f(x)$を適当にとって単数倍しても同じイデアルを作るから、
        最高次の係数が1である最小多項式$\operatorname{Irr}(\theta, K, x)$をこれらの代表に取れる。
        まとめると、最小多項式の定義は次のようになる。
        \begin{Def}
            最小多項式とは、次の条件を満たす$f(x) \in K[x]$のことである。
            \begin{enumerate}
                \item $f(x)$は$\theta$を根に持つ。
                \item $f(x)$は$\theta$を根に持つ$K[x]$の元の中で、次数が最小。
                \item 最高次の係数が1
            \end{enumerate}
        \end{Def}

        仮に$f,g$が最小多項式であったとすると、新たな多項式$h=f-g$も条件1を満たす。
        しかし条件2, 3から$\operatorname{deg}(h)<\operatorname{deg}(f)-1=\operatorname{deg}(g)-1$
        となり、$f,g$の条件2に反する。$K(\theta)$をベクトル空間と見ると、これは次のように表現できる。

        \begin{Prop}
            $\theta$が次数$n$の代数的数だとすると、
            $0 \in K(\theta)$は$K(\theta)$の元達
            $\{ 1, \theta, \theta^2, \dots, \theta^{n-1}\}$の線型結合によって
            ただ一通りに書かれる。
        \end{Prop}

        p.182の命題1.5はさらに$K(\theta)$の全ての元がこのように表せることを言っている。
        \begin{Prop}[命題1.5]
            $\theta$がKの代数的数であるとき、
            $K$上のベクトル空間$K(\theta)$の基底として
            $\{ 1, \theta, \theta^2, \dots, \theta^{n-1}\}$が取れる。
        \end{Prop}

    \subsection{命題1.5について}
        証明の概略を述べる。
        すでに$K[x]/I \overset{\bar{\phi}}{\simeq} K(\theta)$を述べたのでこれを使おう。
        $K[x]$を$K[x]/I$に射影して0に潰れない元は、
        最小多項式$\operatorname{Irr}(\theta, K, x)$よりも次数が小さいK上の多項式、
        すなわち$K+Kx+\dots+Kx^{n-1}$の元である。
        $K[x]/I=\{f+I | f \in K+Kx+\dots+Kx^{n-1} \}$と表せるから、
        \[ K+Kx+\dots+Kx^{n-1} \overset{(\bar{\psi})^{-1}}{\simeq} K[x]/I \overset{\bar{\phi}}{\simeq} K(\theta) \]
        左から右へ元がどのように写されるかを思い出すと、$f(x) \mapsto f(x)+I \mapsto f(\theta)$となっていた。
        $f(x)$は$K+Kx+\dots+Kx^{n-1}$の任意の元だから、命題が成り立つ。

        $1 \in K(\theta) $がただ一つの表現を持つことを示す
        という方針も思いつくが、最小多項式の「最高次の係数が1」という条件は「0は(単)数倍しても0」というところから
        来ていたことを考えれば、これは使えないと分かる。

        命題1.5を命題1.3との関連で考えると、拡大$L/K$が単拡大の代数拡大であった時には
        命題1.3で出てくる分母$g(\theta)$は(0でない)定数に変換でき、
        しかも分子$f(\theta)$は次数が$n$未満であるものに限る、と言っている。

%        他には、$K+Kx+ \dots +Kx^{n-1} \simeq K[x]-I \cup {0}$(?)でIは素イデアルだから右辺は乗法的閉集合、
%        よって命題1.3で示されるような$K(\theta)$の元$f(\theta)/g(\theta)$は$K+Kx+ \dots +Kx^{n-1}の元$で表せる、
%        とみることも出来る。

    \section{定理2.8(1)について}
    \subsection{代数的閉包の存在}
    最初の方針は任意の定数でない多項式$f(x) \in  K[x] \setminus K$が
    少なくとも1根を持つような、$K$の拡大体$L_1$を作る、というものである。
    $K$と$L_1$はどちらも体だから、拡大を繰り返せるだろう。

    さてどうやって拡大させるかが問題である。
    $L_1$が持つべき条件を整理して再度述べる。
    ただし$K[x]^{\ast}=K[x] \setminus K$である。
    \begin{enumerate}
        \renewcommand{\labelenumi}{(\Alph{enumi})}
        \item $L_1$は$K$の拡大体
        \item $\forall f \in K[x]^{\ast},~ \exists \theta_{f} \in L_1 \setminus \{ 0 \}, ~s.t.~ f(\theta)=0$
    \end{enumerate}
    (B)で$\theta_{f}$と添字をつけたのは、
    $f$によってその根が決まる、という気持ちからである。

    やりたいことは、以上の条件を持つ拡大体を作ることだ。
    似たようなことをやった例として根体の存在を示すことをした。
    この時は多項式環の既約多項式で生成される極大イデアルによる剰余体が根体になっていた。
    同様に、$K$に不定元を入れて極大イデアルで割りたい。

    ではどのような不定元を入れて、どのような極大イデアルで割るべきか?
    後半から考えていこう。
    極大イデアルで割った後は$f(\theta)=0$となってほしいが、
    これは極大イデアルに$f(\theta)$が入っていれば、ちょうどそのようになる!~
    そうなると$\theta$は$f$の根に対応するものだから、
    $K[x]$の既約多項式(しかも最高次の係数を1だとかに特定したもの)と一対一対応させるとよい。

    まとめると次のようになる。
    簡単のために、$K[x]$の既約多項式最高次の係数が1であるものの全体を$\Im$とする。
    $K$には$f \in \Im$に対応して不定元を入れ、$K[\{R_f\}_{f \in \Im}]$とする。
    これを$\{f(R_f)\}_{f \in \Im}$を含む極大イデアルで割れば良い。
    こうしてできた体は、$K[x]$の既約多項式の根を少なくとも一つ持つ。
    操作を繰り返せば、$K[x]$の既約多項式は全て1次式へ分解できる。

    \subsection{根体を作る時との違い}
    根体を作る時と違うのは、
    $\{f(R_f)\}_{f \in \Im}$が生成するイデアルが極大イデアルかわからない、ということである。
    根体の時は不定元を1つだけ入れて$K[X]$とし、
    さらに$K[X]$がPIDであることから既約多項式が生成するイデアル$(f(X))$が極大イデアルとなることを導いた。
    しかし今は多数の不定元を入れたので、$K[\{R_f\}_{f \in \Im}]$はPIDではなく、
    既約多項式達が作るイデアル$(\{f(R_f)\}_{f \in \Im})$が極大かどうかはわからない。
    しかし$(\{f(R_f)\}_{f \in \Im})$は真のイデアル
        \footnote{$(\{f(R_f)\}_{f \in \Im}) \neq K[\{R_f\}_{f \in \Im}]$}
    であることがわかるので、これを含む極大イデアルの存在が保証される。

    \section{定理2.8(2)について}
    圏論的に見てみる。
    まず、$K_1$と$\bar{K}_1$の中間体達を以下のように並べる。
    ここにある射は全て埋め込み(insection)だ。
    \[
    \begin{xy}
        (00, 30)    *{L_1}="L1",
        (15, 30)    *{L_2}="L2",
        (30, 30)    *{L_3}="L3",
        (45, 30)    *{L_4}="L4",
        (70, 30)    *{~\dots}="Li",
        (5, 60)     *{\bar{K}_1}="cK1",
        (5, 00)     *{K_1}="K1",

        \ar@{->}            "L1"+<0.8em,0em>;"L2"
        \ar@{->}            "L2"+<0.8em,0em>;"L3"
%        \ar@{->}@/_5mm/     "L2"+<0.8em,-0.7em>;"L4"
        \ar@{->}            "L3"+<0.8em,0em>;"L4"
        \ar@{->}            "L4"+<0.8em,0em>;"Li"

        \ar@{->}@/_1mm/ "K1"+<0.0em,0.5em>;"L1"+<-0.0em,-0.5em>
        \ar@{->}@/^3mm/ "K1"+<0.0em,0.5em>;"L2"+<-0.6em,-0.5em>
        \ar@{->}@/^7mm/ "K1"+<0.0em,0.5em>;"L3"+<-0.6em,-0.5em>
        \ar@{->}@/^7mm/ "K1"+<0.0em,0.5em>;"L4"+<-0.8em,-0.5em>
        \ar@{->}@/^7mm/ "K1"+<0.0em,0.5em>;"Li"+<-0.8em,-0.5em>
        \ar@{->}@/_1mm/ "L1"+<-0.0em,0.5em>;"cK1"
        \ar@{->}@/^2mm/ "L2"+<-0.6em,0.5em>;"cK1"
        \ar@{->}@/^5mm/ "L3"+<-0.6em,0.5em>;"cK1"
        \ar@{->}@/^5mm/ "L4"+<-0.8em,0.5em>;"cK1"
        \ar@{->}@/^5mm/ "Li"+<-0.8em,0.5em>;"cK1"
    \end{xy}
    \]
    中間体達の間ではこのような挿入射だけを考える。
    ここはposetになる。

    さらに$K_2$と$\bar{K}_2$が現れる。
    この図は可換になっていることに注意。
    \[
    \begin{xy}
        (00, 30)    *{L_1}="L1",
        (15, 30)    *{L_2}="L2",
        (30, 30)    *{L_3}="L3",
        (45, 30)    *{L_4}="L4",
        (70, 30)    *{~\dots}="Li",
        (5, 60)     *{\bar{K}_1}="cK1",
        (5, 00)     *{K_1}="K1",
        (55, 60)    *{\bar{K}_2}="cK2",
        (55, 00)    *{K_2}="K2",

        \ar@{->}            "L1"+<0.8em,0em>;"L2"
        \ar@{->}            "L2"+<0.8em,0em>;"L3"
%        \ar@{->}@/_5mm/     "L2"+<0.8em,-0.7em>;"L4"
        \ar@{->}            "L3"+<0.8em,0em>;"L4"
        \ar@{->}            "L4"+<0.8em,0em>;"Li"

        \ar@{->}@/_1mm/ "K1"+<0.0em,0.5em>;"L1"+<-0.0em,-0.5em>
        \ar@{->}@/^3mm/ "K1"+<0.0em,0.5em>;"L2"+<-0.6em,-0.5em>
        \ar@{->}@/^7mm/ "K1"+<0.0em,0.5em>;"L3"+<-0.6em,-0.5em>
        \ar@{->}@/^7mm/ "K1"+<0.0em,0.5em>;"L4"+<-0.8em,-0.5em>
        \ar@{->}@/^7mm/ "K1"+<0.0em,0.5em>;"Li"+<-0.8em,-0.5em>
        \ar@{->}@/_1mm/ "L1"+<-0.0em,0.5em>;"cK1"
        \ar@{->}@/^2mm/ "L2"+<-0.6em,0.5em>;"cK1"
        \ar@{->}@/^5mm/ "L3"+<-0.6em,0.5em>;"cK1"
        \ar@{->}@/^5mm/ "L4"+<-0.8em,0.5em>;"cK1"
        \ar@{->}@/^5mm/ "Li"+<-0.8em,0.5em>;"cK1"

        \ar@{.>}@/_7mm/ "K2"+<0.0em,0.5em>;"L1"+<0.0em,-0.4em>
        \ar@{.>}@/_6mm/ "K2"+<0.0em,0.5em>;"L2"+<0.4em,-0.4em>
        \ar@{.>}@/_3mm/ "K2"+<0.0em,0.5em>;"L3"+<0.4em,-0.4em>
        \ar@{.>}@/_1mm/ "K2"+<0.0em,0.5em>;"L4"+<0.4em,-0.4em>
        \ar@{.>}@/^5mm/ "K2"+<0.0em,0.5em>;"Li"+<0.0em,-0.4em>
        \ar@{.>}@/_7mm/ "L1"+<0.0em,0.4em>;"cK2"
        \ar@{.>}@/_5mm/ "L2"+<0.4em,0.4em>;"cK2"
        \ar@{.>}@/_3mm/ "L3"+<0.4em,0.4em>;"cK2"
        \ar@{.>}@/_1mm/ "L4"+<0.4em,0.4em>;"cK2"
        \ar@{.>}@/^5mm/ "Li"+<0.0em,0.4em>;"cK2"

        \ar@{->}@/^3mm/^{\sigma} "K1";"K2"
        \ar@{->}@/^3mm/^{\sigma^{-1}} "K2";"K1"
    \end{xy}
    \]
    煩雑であったため、挿入射を表す矢印の根本のカールは取り除いた。
    この図にある点線は準同型を表していて、
    具体的な内容は$\sigma$によって決まる。
    しかし$\sigma$と点線の合成は$K_1$から中間体への挿入射に等しいので、
    ほとんど挿入射に近い。
    また、この時点で射を追っても、同型射の拡張は言えない。
    中間体への射は全て準同型だからである。

    さて、中間体達はposetとなっているから、
    この構造を反映した関手$L:\mathbf{Pos} \to \mathbf{Field}$をとり、余極限$\varinjlim L$を見る。
    これは中間体のposetが帰納的順序集合であることからその存在が出る中間体の極大元である。
    すると詳細な議論から$\varinjlim L=\bar{K}_1$が分かる。
    つまり$\bar{K}_1$は$\mathrm{Cocone}(D)$の終対象としてのUMPを持つのである。

    \[
    \begin{xy}
        (00, 30)    *{L_1}="L1",
        (15, 30)    *{L_2}="L2",
        (30, 30)    *{L_3}="L3",
        (45, 30)    *{L_4}="L4",
        (70, 30)    *{~\dots}="Li",
        (5, 60)     *{\bar{K}_1}="cK1",
        (5, 00)     *{K_1}="K1",
        (55, 60)    *{\bar{K}_2}="cK2",
        (55, 00)    *{K_2}="K2",
        (5, 68)     *{\varinjlim L}="lim",

        \ar@{->}            "L1"+<0.8em,0em>;"L2"
        \ar@{->}            "L2"+<0.8em,0em>;"L3"
%        \ar@{->}@/_5mm/     "L2"+<0.8em,-0.7em>;"L4"
        \ar@{->}            "L3"+<0.8em,0em>;"L4"
        \ar@{->}            "L4"+<0.8em,0em>;"Li"

        \ar@{->}@/_1mm/ "K1"+<0.0em,0.5em>;"L1"+<-0.0em,-0.5em>
        \ar@{->}@/^3mm/ "K1"+<0.0em,0.5em>;"L2"+<-0.6em,-0.5em>
        \ar@{->}@/^7mm/ "K1"+<0.0em,0.5em>;"L3"+<-0.6em,-0.5em>
        \ar@{->}@/^7mm/ "K1"+<0.0em,0.5em>;"L4"+<-0.8em,-0.5em>
        \ar@{->}@/^7mm/ "K1"+<0.0em,0.5em>;"Li"+<-0.8em,-0.5em>
        \ar@{->}@/_1mm/ "L1"+<-0.0em,0.5em>;"cK1"
        \ar@{->}@/^2mm/ "L2"+<-0.6em,0.5em>;"cK1"
        \ar@{->}@/^5mm/ "L3"+<-0.6em,0.5em>;"cK1"
        \ar@{->}@/^5mm/ "L4"+<-0.8em,0.5em>;"cK1"
        \ar@{->}@/^5mm/ "Li"+<-0.8em,0.5em>;"cK1"

        \ar@{.>}@/_7mm/ "K2"+<0.0em,0.5em>;"L1"+<0.0em,-0.4em>
        \ar@{.>}@/_6mm/ "K2"+<0.0em,0.5em>;"L2"+<0.4em,-0.4em>
        \ar@{.>}@/_3mm/ "K2"+<0.0em,0.5em>;"L3"+<0.4em,-0.4em>
        \ar@{.>}@/_1mm/ "K2"+<0.0em,0.5em>;"L4"+<0.4em,-0.4em>
        \ar@{.>}@/^5mm/ "K2"+<0.0em,0.5em>;"Li"+<0.0em,-0.4em>
        \ar@{.>}@/_7mm/ "L1"+<0.0em,0.4em>;"cK2"
        \ar@{.>}@/_5mm/ "L2"+<0.4em,0.4em>;"cK2"
        \ar@{.>}@/_3mm/ "L3"+<0.4em,0.4em>;"cK2"
        \ar@{.>}@/_1mm/ "L4"+<0.4em,0.4em>;"cK2"
        \ar@{.>}@/^5mm/ "Li"+<0.0em,0.4em>;"cK2"

        \ar@{->}@/^3mm/^{\sigma} "K1";"K2"
        \ar@{->}@/^3mm/^{\sigma^{-1}} "K2";"K1"

        \ar@{=} "cK1"+<0mm,3mm>;"lim"+<0mm,-3mm>
    \end{xy}
    \]

    $\varinjlim L$のUMPから、$\varinjlim L$から$\bar{K}_2$へ、
    すでに有る射を可換にする射がただ一つ存在する。
    中間体達から$\bar{K}_2$への射(母線)は$\sigma$によって定まるから、
    この射も$\sigma$によって定まる。

    \[
    \begin{xy}
        (00, 30)    *{L_1}="L1",
        (15, 30)    *{L_2}="L2",
        (30, 30)    *{L_3}="L3",
        (45, 30)    *{L_4}="L4",
        (70, 30)    *{~\dots}="Li",
        (5, 60)     *{\bar{K}_1}="cK1",
        (5, 00)     *{K_1}="K1",
        (55, 60)    *{\bar{K}_2}="cK2",
        (55, 00)    *{K_2}="K2",
        (5, 68)     *{\varinjlim L}="lim",

        \ar@{->}            "L1"+<0.8em,0em>;"L2"
        \ar@{->}            "L2"+<0.8em,0em>;"L3"
%        \ar@{->}@/_5mm/     "L2"+<0.8em,-0.7em>;"L4"
        \ar@{->}            "L3"+<0.8em,0em>;"L4"
        \ar@{->}            "L4"+<0.8em,0em>;"Li"

        \ar@{->}@/_1mm/ "K1"+<0.0em,0.5em>;"L1"+<-0.0em,-0.5em>
        \ar@{->}@/^3mm/ "K1"+<0.0em,0.5em>;"L2"+<-0.6em,-0.5em>
        \ar@{->}@/^7mm/ "K1"+<0.0em,0.5em>;"L3"+<-0.6em,-0.5em>
        \ar@{->}@/^7mm/ "K1"+<0.0em,0.5em>;"L4"+<-0.8em,-0.5em>
        \ar@{->}@/^7mm/ "K1"+<0.0em,0.5em>;"Li"+<-0.8em,-0.5em>
        \ar@{->}@/_1mm/ "L1"+<-0.0em,0.5em>;"cK1"
        \ar@{->}@/^2mm/ "L2"+<-0.6em,0.5em>;"cK1"
        \ar@{->}@/^5mm/ "L3"+<-0.6em,0.5em>;"cK1"
        \ar@{->}@/^5mm/ "L4"+<-0.8em,0.5em>;"cK1"
        \ar@{->}@/^5mm/ "Li"+<-0.8em,0.5em>;"cK1"

        \ar@{.>}@/_7mm/ "K2"+<0.0em,0.5em>;"L1"+<0.0em,-0.4em>
        \ar@{.>}@/_6mm/ "K2"+<0.0em,0.5em>;"L2"+<0.4em,-0.4em>
        \ar@{.>}@/_3mm/ "K2"+<0.0em,0.5em>;"L3"+<0.4em,-0.4em>
        \ar@{.>}@/_1mm/ "K2"+<0.0em,0.5em>;"L4"+<0.4em,-0.4em>
        \ar@{.>}@/^5mm/ "K2"+<0.0em,0.5em>;"Li"+<0.0em,-0.4em>
        \ar@{.>}@/_7mm/ "L1"+<0.0em,0.4em>;"cK2"
        \ar@{.>}@/_5mm/ "L2"+<0.4em,0.4em>;"cK2"
        \ar@{.>}@/_3mm/ "L3"+<0.4em,0.4em>;"cK2"
        \ar@{.>}@/_1mm/ "L4"+<0.4em,0.4em>;"cK2"
        \ar@{.>}@/^5mm/ "Li"+<0.0em,0.4em>;"cK2"

        \ar@{->}@/^3mm/^{\sigma} "K1";"K2"
        \ar@{->}@/^3mm/^{\sigma^{-1}} "K2";"K1"

        \ar@{=} "cK1"+<0mm,3mm>;"lim"+<0mm,-3mm>
        \ar@{~>}^{\bar{\sigma}} "lim";"cK2"
    \end{xy}
    \]

    最後に、$\bar{K}_1$と$\bar{K}_2$が共に代数的閉包であることを用いる。
    まず、$\bar{\sigma}(\varinjlim L) \simeq \bar{K}_1$から$\bar{\sigma}(\varinjlim D)$は代数的閉包。
    図式の射は全て準同型だから$\bar{\sigma}(\varinjlim L) \subset \bar{K}_2$となる。
    この二つと$\bar{\sigma}(\varinjlim D)$,$\bar{K}_2$が特に代数閉体であることから、
    $\bar{\sigma}(\varinjlim L)=\bar{K}_2$が分かる
    \footnote{定理2.8の直前に有る「Kが代数閉体で$K \subset L$なら$\bar{K}_L=K$」という記述はこのことを言っている。
    表現をこれに合わせると、「$\bar{K}_2$の任意の代数拡大体$L$について$\overline{(\bar{\sigma}(\varinjlim L))}_L=\bar{K}_2=\overline{(\bar{K}_2)}_L$」}。
    よって$\bar{\sigma}$は全射。体同士の準同型は単射であったから、とくに同型射。

    \section{定理5.1(1)について}
        \subsection{「明らかに$\sigma \in G$はこれらを全体として保つ」}
        \[ \sigma \in G = \Aut(L/K)=\{ \sigma : L \to L \} \]
        より明らかに$\sigma(\theta) \in L$である。
        さらに$\theta \in L$の($K$上の)共役元は群$\Aut(L/K)$による$\theta$の軌道$\theta \Aut(L/K)$の元。
        なので$\theta \in L$の共役元$\theta_i \in L$は$\sigma \in \Aut(L/K)$によってLに属すような$\theta$の共役元へ写る。

        \subsection{$f \in K[x] \implies f = \Irr(\theta, K, x)$}
        「$f(x)$の係数は$K=F(G)$に入る」から「$\theta$の任意の$K$上の共役元は$f(x) \in K[x]$の根となり」の導出を
        まとめるとこのようになる。
        しかしこれは$\deg f \leq \deg \Irr(\theta, K, x)$と、
        $\Irr(\theta, K, x)$が$\theta$を根に持つ$K$係数の多項式として次数が最小のものであることから自明である。
        $\deg f \leq \deg \Irr(\theta, K, x)$は、$f(x)$の根が$\theta$のK上の共役元であり、
        したがって$\Irr(\theta, K, x)$よりも根が少ないか等しいことから分かる。

        正規拡大であることの証明として以下のような方針もありうる。
        まず$\Aut(\bar{K}/K)$の元の定義域を$L$に制限し、$\sigma: L \to \bar{L}=\bar{K}$とする。
        証明することは命題3.1の(2)、すなわち任意の$L$の元を$\sigma$で写した時に$L$からはみ出さないことである。
        $\alpha \in L$の$K$上の共役元全体を$\Theta_{\alpha}$とおく。
        すでに述べたとおり、\[ \sigma(\Theta_{\alpha} \cap L)=\Theta_{\alpha} \cap L \]である。
        このことから、\[ \bigcup_{\alpha \in L}{\Theta_{\alpha}}=L \]が示せれば十分。

    \section{定理5.2(2)について}
        \subsection{「$\tilde{r}$は全射」}
        シュタイニッツの定理より、
        $\Gal(L/K)$の元$\sigma:L \to L(K-\mbox{同型})$は
        $\Aut(\bar{K}/K)$の元$\bar{\sigma}:\bar{K} \to \bar{K}(K-\mbox{同型})$に拡張される。
        よって$\tilde{r}$は全射。
        
        \subsection{$M/K$が正規拡大$\iff \dots$}
        \begin{eqnarray*}
            &{}& M/K\mbox{が正規拡大} \\
            &\iff& \forall \sigma \in Aut(\bar{K}/K), \sigma(M)=M \\
            &\iff& \forall \sigma \in Aut(\bar{K}/K), \phi(\sigma(M))=\phi(M)
        \end{eqnarray*}
        二行目はシュタイニッツの定理より。
        三行目は$\phi$の単射性より。

        \subsection{系3.3について}
        シュタイニッツの定理より、
        $\sigma : L \to \bar{K}$は
        $\bar{\sigma}:\bar{L}=\bar{K} \to \bar{K}$に拡張される。
        しかも命題3.1(2)より$\bar{\sigma}(L)=L$である。
        
        \subsection{「$r: Aut(L/K) \to Gal(M/K)$は全射」}
        $\sigma \in \Gal(M/K)$を任意にとる。
        $\sigma$$はM \to M$の$K$-同型だから、
        シュタイニッツの定理よりこれは$K$-同型$\bar{\sigma}: \bar{M} \to \bar{M}$に拡張される。
        これをLに制限すると、これがさらに$K$-同型$\bar{\sigma}|_{L}:L \to L$となる。
        これが$K$-同型であることは、$K$-同型の定義$\bar{\sigma}|_{K}=id_K$と$K \subset L$より。
        以上より、$\Gal(M/K)$の各元に対して$\Aut(L/K)$の元$\bar{\sigma}|_{L}$が存在する。
        よって$r$は全射。
        
        \subsection{$\sigma|_L:L \to L$}
        写像$\sigma : \bar{K} \to \bar{K}$は制限で$\sigma|_L:L \to L$になるか?
        つまり、$\sigma(L)=L$か?
        これは$L/K$が正規拡大であることと命題3.1(2)より真。

    \section{命題5.4について}
        \subsection{$L/K$が有限かつ分離的拡大なら$L'/K'$もそうか?}
        $L'/L/(L \cap K')$と$L'/K'/(L \cap K')$とを考える。
        まず、仮定と命題4.5(1)より$L/L \cap K'$は分離的拡大。
        系4.6を分離的拡大$L/(L \cap K')$と適当な拡大$K'/(L \cap K')$について用いて、
        $(L \cdot K'=)L'/K'$が分離的拡大であることが出る。

\end{document}
