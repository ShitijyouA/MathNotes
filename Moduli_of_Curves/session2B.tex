\documentclass[a4paper]{jsarticle}
\usepackage[]{../math_note}
\usepackage[all]{xy}

\usepackage[dvipdfmx, colorlinks=true, linkcolor=black]{hyperref}
\usepackage{pxjahyper}

\newcommand{\tp}[2]{\texorpdfstring{#1}{#2}}
\newcommand{\modM}{\mathcal{M}}
\newcommand{\modC}{\mathcal{C}}
\newcommand{\modP}{\mathcal{P}}

\begin{document}
\title{ゼミノート \#2}
\author{七条彰紀}
\maketitle
    以降は,curveと言えば
    \begin{center}
        smooth, complete, reduced and connected scheme of dimension $1$ over $\C$
    \end{center}
    のことである.
    \cite{HarAG} II, 6.7より,以上の意味でのcurveはprojectiveである.

    \cite{HaMo}では``curve"の定義に``curve"を用いているので
    些か定義を定め難い.
    このノートでは,通常要求されるirreducibilityは要求しないことにした.
    これは\cite{HaMo} Exercise 1.7に現れる$xy=0$を除外しないためである.

    また,(geometric) genus of curveは通常$g$で表す.

\section{ Moduli spaces we'll be concerned with }
    以降で考えていくmoduli spaceを簡単に紹介する.

    \subsection{\tp{$\modM_g$}{Mg}
        :: the coarse moduli space of curves of genus \tp{$g$}{g}. }
    これまで議論してきた.
    まだ存在は示されていない.
    trivial automorphismしか持たないcurveに対応する$\modM_g$の点全体を
    $\modM_g^0$と書くことにする.
    これは$\modM_g$の開集合であることが知られている.

    \subsection{\tp{$\modM_{g,n}$}{Mgn}
        :: the coarse moduli space of pairs of curve of genus \tp{$g$}{g} and \tp{$n$}{n} distinct points. }
    $\modC_g=\modM_{g,1}$もここで述べる.

    corve of genus $g$ :: $C$と
    $C$の$n$個の\kenten{互いに異なる}点 :: $p_1,\dots, p_n$を合わせた\kenten{順序}組
    $(C, p_1,\dots, p_n)$のmoduli spaceを$\modM_{g,n}$と呼ぶ.

    \cite{HaMo}によれば,
    圏点をつけた条件(互いに異なる点の順序組)は,
    $\modM_{d,g}$のcompactificationを考える上で必要である.
    また,curve :: $C$と,互いに異なるとは限らない点の順序無し組の組
    $(C, \{p_1,\dots,p_n\})$のcoarse moduli spaceを構成することも出来る.

    $(C, p_1,\dots,p_n)$から$n$点$p_1,\dots,p_n$の情報を忘れると,
    標準的な射$\modM_{g,n} \to \modM_g$が得られる.

    \subsection{\tp{$\modP_{d,g}$}{Pdg}
        :: the coarse moduli space of pairs of curve of genus \tp{$g$}{g} and line bundle of degree \tp{$d$}{d}. }
    $\modP_{d,g}$は,
    curve of genus $g$とその上のline bundle of degree $d$の組$(C, \shL)$から
    $\shL$の情報を忘れれば,
    標準的な射$\phi: \modP_{d,g} \to \modM_g$が得られる.

    \begin{enumerate}[leftmargin=*]
        \item
            $[C] \in \modM_g^0$上の$\phi$のfiberは
            $\Pic^d(C)$のconnected componentになっている.

        \item
            $\modP_{0,g}$は時々Jacobian bundle over $\modM_g$と呼ばれる.
    \end{enumerate}

\bibliographystyle{jplain}
\bibliography{reference}
\end{document}
