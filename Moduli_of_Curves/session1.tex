\documentclass[a4paper]{jsarticle}
\usepackage[]{../math_note}
\usepackage[dvipdfmx, colorlinks=true]{hyperref}
\usepackage{pxjahyper}
\usepackage[]{enumitem}
\usepackage[all]{xy}

\usepackage{chngcntr}
\makeatletter
    \newcounter{c@question}
    \counterwithin{c@question}{section}
    \newenvironment{question}[0]%
    {\stepcounter{c@question}\begin{itembox}[l]{問\arabic{section}.\arabic{c@question}}}%
    {\end{itembox}}%
    \newenvironment{question*}[0]%
    {\stepcounter{c@question}\begin{itembox}[l]{問}}% 
    {\end{itembox}}%
\makeatother

\newcommand{\Sch}{\mathbf{Sch}}
\newcommand{\Set}{\mathbf{Set}}
\newcommand{\Ring}{\mathbf{Ring}}
\newcommand{\Alg}{\mathbf{Alg}}
\newcommand{\Open}{\mathrm{Open}}
\newcommand{\OpenSubSch}{\mathbf{OpenSubSch}}
\newcommand{\GrpSch}{\mathbf{GrpSch}}
\newcommand{\famF}{\mathcal{F}}
\newcommand{\famG}{\mathcal{G}}
\newcommand{\famU}{\mathcal{U}}

\newcommand{\func}[1]{\underline{#1}}
\newcommand{\ftorM}{\mathcal{M}}
\newcommand{\ftorGL}{\mathcal{GL}}

\begin{document}
\title{ゼミノート \#1 \\ Fine/Coarse Moduli Spaceの非存在}
\author{七条彰紀}
\maketitle

\begin{question}
    Fine/Coarse moduli spaceとは何か?
\end{question}
moduli spaceは``moduli functor"の情報を
可能な限り精密に写したschemeのことである.
その理解のためにはまず``functor of points"の概念が必要である.
以下,私が過去に書いたノート``Group Scheme"を引用・加筆する.

\section{Functor of Points}
    圏論で言う``generalized point"の概念を,
    名前を変えて用いる.

    \begin{Def}
    \enumfix
    \begin{enumerate}[label=(\roman*),leftmargin=*]
    \item 
    $X, T \in \Sch/S$に対し,
    $\func{X}(T)=\Hom_{\Sch/S}(T,X)$を\textbf{$X$の$T$-valued points}と呼ぶ.
    $T=\Spec R$と書けるときは$\func{X}(T)$を$\func{X}(R)$と書く.
    したがって$\func{X}$は$\Sch/S$からのcovariant functorと見ることも,
    $k$-algebraの圏からのcontravariant functorと見ることも出来る.
    この関手$\func{X}$はfunctor of pointsと呼ばれる.

    \item
    体$k$上のscheme :: $X$ ($S=\Spec k, X \in \Sch/S$)と
    field extension :: $k \subseteq K$について,
    $\func{X}(K)$を$X$の\textbf{$K$-rational points}と呼ぶ.

    \item
    morphism :: $h: X \to Y$について
    自然変換$\func{h}: \func{X} \to \func{Y}$は
    $\phi \mapsto h \circ \phi$のように射を写す.
    \end{enumerate}
    \end{Def}

    \begin{Remark}
        $\Sch$はlocally small categoryである.
        すなわち,任意の$X, T \in \Sch$について$\func{X}(T)$は集合である.
        これを確かめるために,
        $X, Y \in \Sch$を任意にとり,
        $\Hom(X,Y)$の濃度がある濃度で抑えられることを見よう.
        射$X \to Y$の作られ方に沿って考える.
        \begin{enumerate}[label=(\arabic*), leftmargin=*]
        \item
            base spaceの間の写像$f: \basesp X \to \basesp Y$をとる.
            このような写像全体の濃度は高々$|\basesp Y|^{|\basesp X|}$.
        \item
            $|Y|$の開集合$U$をとる.
            開集合全体の濃度は高々$2^{|\basesp Y|}$.
        \item
            写像$f^{\#}_U: \shO_Y(U) \to (f_* \shO_X)(U)$を定める.
            このような写像全体の濃度は高々$|(f_* \shO_X)(U)|^{|\shO_Y(U)|}$.
        \end{enumerate}
        したがって$\Hom(X,Y)$の濃度は高々
        \[|\basesp Y|^{|\basesp X|} \times \prod_{U \in 2^{\basesp Y}} |(f_* \shO_X)(U)|^{|\shO_Y(U)|} \]
        となる.
        濃度の上限が存在する(すなわち,ある集合への単射を持つ)から,
        $\Hom(X,Y)$は集合である.
%        presheaf on $Y$の圏は
%        $Y$の開集合が成す圏(small)から
%        集合の圏(locally small)の圏への
%        関手圏である.
%        これがlocally smallであることを示しても良いと思う.
    \end{Remark}

    \begin{Remark}
        上の注意から,Yoneda Lemmaが成立する.
        したがって自然変換$\func{G} \to \func{H}$と
        射$G \to H$が一対一対応する.
        このため,
        schemeの間の射についての議論と
        functor of pointsの間の射の議論は
        (ある程度)互いに翻訳することが出来る.
    \end{Remark}

    \begin{Remark}
        $K$-rational pointについては,
        $\func{X}(K)=\{ x \in X \mid k(x) \subseteq K \}$とおく定義もある.
        ここで$k(x)$は$x$でのresidue fieldである.
        しかし\cite{HarAG} Chapter.2 Ex2.7から分かる通り,
        この二つの定義は翻訳が出来る.
        すなわち,
        $k(x) \subseteq K$を満たす$x \in X$と,
        $\Spec k$-morpsihm :: $\Spec K \to X$は一対一に対応する.

        また$X$ :: finite type /$k$であるとき,
        closed point :: $x \in X$について,
        $k(x)$は$k$の有限次代数拡大体である.
        これはZariski's Lemmaの帰結である.
        したがって$\func{X}(\bar{k})$は$X$のclosed point全体に対応する.
        ただし$\bar{k}$は$k$の代数閉包である.
    \end{Remark}

    \begin{Example}

        $\R$上のaffine scheme $X=\Spec \R[x,y]/(x^2+y^2)$の
        $\R$-rational pointと$\C$-rational pointを考えよう.

        $\Spec \R \to X$の射は環準同型 $\R[x,y]/(x^2+y^2) \to \R$と一対一に対応する.
        しかし直ちに分かる通り,
        このような環準同型は
        \[ (\bar{x}, \bar{y}) \mapsto (0, 0) \]
        で定まるものしか存在し得ない.
        ここで$\bar{x}=x \bmod (x^2+y^2), \bar{y}=y \bmod (x^2+y^2)$と置いた.
        よって$\func{X}(\R)$は1元集合.
        また,この環準同型が誘導する$\Spec R \to X$の射は
        1点空間$\Spec \R$を原点へ写す.

        一方,環準同型 $\R[x]/(x^2+1) \to \C$は
        \[ (\bar{x}, \bar{y}) \mapsto (a, \pm ia) \]
        (ここで$i=\sqrt{-1}, a \in \R$)で定まることが分かる.
        すなわち,$\zerosa(x^2+y^2) \subseteq \affine^2_{\C}$の点に対応して,
        $\R[x]/(x^2+1) \to \C$の環準同型が定まる.
        逆の対応も明らか.
        よって$\func{X}(\C)$の元は
        $\zerosa(x^2+y^2) \subseteq \affine^2_{\C}$の点に
        対応している.
    \end{Example}

    \begin{Example}
        体$k$上のaffine variety :: 
        $X \subseteq \affine^n_k$を
        多項式系 :: $F_1,\dots,F_n \in k[x_1,\dots,x_n]$で定まるものとする.
        すると$k$上の環$R$に対して,次の集合が考えられる.
        \[ V_R=\left\{ p=(r_1,\dots,r_n) \in R^{\oplus n} ~\middle|~ F_1(p)=\dots=F_n(p)=0 \right\}. \]
        この集合の元も$R$-value pointと呼ばれる.
        (\cite{Muk1}ではこちらのみを$R$-value pointと呼んでいる.
        実際,こちらのほうが字句``value point"の意味が分かりやすいだろう.)
        $V_R$の点が$\func{X}(R)$の元と一対一に対応することを見よう.

        $X$のaffine coordinate ringを
        $A=k[x_1,\dots,x_n]/(F_1,\dots,F_n)$とし,
        $\bar{x}_i=x_i \bmod (F_1,\dots,F_n) ~(i=1,\dots,n)$とおく.
        $\phi: A \to R$を考えてみると,
        これは次のようにして定まる.
        \[ (\bar{x}_1,\dots,\bar{x}_n) \mapsto (r_1,\dots,r_n) \in V_R. \]
        すなわち,$V_R$の点に対して$\Hom_{\Ring/k}(A,R)$の元が定まる.
        逆の対応は明らか.
        そして,$\Hom_{\Ring/k}(A,R)$が
        $\Hom_{\Sch/\Spec k}(\Spec R, X)=\func{X}(R)$と一対一対応することはよく知られている.
    \end{Example}

\section{Moduli Functor and Fine/Corse Moduli Space}
    $A$を代数幾何学的対象の集合とし,
    $\sim$を$Z$の中の同値関係とする.
    ``naive moduli problem"は,
    $M$の点と$A/\sim$の元(同値類)が一対一対応するような
    scheme :: $M$を見つけよ,という問題である.
    更に$A/\sim$の元が「連続的に変化」する様子も
    「エンコード」しているような$M$を見つけよ,
    という問題を``extended moduli problem"と呼ぶ
    (正確な定義は\cite{Hos} \S 2.2).
    ``extended moduli problem"を定式化するには,
    「連続的に変化」と「エンコード」を定式化しなくてはならない.
    前者の為に``family"が定義され,
    後者の為に``moduli functor"が定義される.
    すると「エンコード」は関手の表現であると理解できる.

    \subsection{Families}
    \begin{Def}
        $\mathcal{P}$を集合のクラス
        \footnote
        {
            集合$X$を変数とする
            述語$X \in \mathcal{C}$の意味を
            「$X$はある条件を満たす対象である」と定義した,
            と考えて良い.
            「属す」の意味は集合と同様に定める.
        }
        とする.
        集合$B$について,
        $B$の構造と整合的な構造を持った集合$\famF$と
        全射写像$\pi: \famF \to B$の組が
        $\mathcal{P}$の$B$上の\textbf{family}であるとは,
        各$b \in B$について集合$\pi^{-1}(b) \subseteq \famF$が
        $\mathcal{P}$に属すということ.
%        $B$はfamily $\pi: \mathcal{F} \to B$のbaseと呼ばれる.
    \end{Def}
    「$B$の構造と整合的な構造」というのは,
    例えば,
    $S$が位相空間であって
    写像$\famF \to S$を連続にするような位相が$\famF$に入っている,
    ということである.
    familyの構造は場合毎に明示されなくてはならない.

    用語``family"を厳密に定義しているものは全くと言っていいほど無いが,
    ここではRenzoのノート
    \footnote{ \url{http://www.math.colostate.edu/~renzo/teaching/Topics10/Notes.pdf} }
    の定義を参考にした.
%    ただし,Renzoのノートの定義は一般化されすぎている.
%    Renzoのノートでは$\mathcal{P}$を
%    ``Let $\mathcal{P}$ define a class of objects in some category $\mathcal{C}$."
%    としているが,
%    これでは写像$\mathcal{F} \to B$が定義できるか怪しい.
%    なので私のこのノートでは$\mathcal{P}$を集合のクラスに限定している.
    ``family"を上のように解釈して不整合が生じたことは,
    私の経験の中ではない.

    \begin{Remark}
        moduli theory以外で``family of $\mathcal{C}$"と言えば,
        単に$\mathcal{C}$の部分集合であろう.
        ``family parametrized by $S$"の様に言えば,
        $S$-indexed family (or set)のことを想像するであろう.
        しかし$S$-indexed family :: $\famF \subset \mathcal{C}$は
        $S \to \famF$という写像で定まるから,
        ここでの``family"とは写像の向きが逆である.
        
        上の定義を無心に読めば分かる通り,
        「$\mathcal{C}$のfamily :: $\famF$」と言った時には,
        $\mathcal{C}$に属すのは$\famF$の部分集合である.
        属すのは(一般に)$\famF$の元ではない.
        また$\famF$は$\mathcal{C}$の元の和集合とみなせる.
        (正確には$\mathcal{C}$の元を$S$に沿って並べたものである.)
    \end{Remark}

    \begin{Example}
        $X, B$ :: scheme,
        $f: X \to B$ :: morphism of schemesをとる.
        $X$は$f$によって$B$上のfamilyとなる.
        射のfibreとして実現される,
        scheme(例えばsmooth curve)のfamilyは
        deformation theoryの対象である.
    \end{Example}

    \begin{Example}\label{example:grassmannian}
        $k$を体,$S$を適当なschemeとする.
        $\affine^2_k$の原点を通る直線の$S$上のfamilyとして,
        line bundle :: $\shL \subset \affine^2 \times_k S$を
        考えることが出来る.
        $\shL \to S$は射影写像で与えられる.
        同様に$\affine^n$の$r$次元線形空間の$S$上のfamilyは
        $r$次元vector bundle :: $\shE \subset \affine^n \times S$である.
    \end{Example}

    \begin{Example}
        $k$を適当な体とし,
        $\proj^1_k$の点$O_i~(i=1,2,3)$を順に$(0:1), (1:0), (1:1)$とする.
        この時,$PGL_2(k)$は
        次の全単射で$\proj^1_k$の自己同型写像の$(\proj^1_k)^{\oplus 3}$上のfamilyになる.
        \begin{defmap}
            \pi:& PGL_2(k)& \to& (\proj^1_k)^{\oplus 3} \\
            {}& \phi& \mapsto& (\phi^{-1}(O_i))_{i=1}^3.
        \end{defmap}
    \end{Example}

    \begin{Remark}
        familyに要請される性質として,
        特に``flat"がある.
        projective flat familyは,
        base schemeに適切な条件をつけると
        各fiber :: $X_t$のHilbert多項式が$t$に依らない,
        という特徴がある(\cite{HarAG} III, Thm9.9).
        詳細は\cite{HarAG} III, 9を参照せよ.
    \end{Remark}

    \subsection{Moduli Functor}
    以下の定義は\cite{HaMo}など,
    Moduli問題に関する殆どの入門書で述べられている.
    \begin{Def}
        \textbf{moduli functor}(またはfunctor of families)とは,
        各scheme :: $S$に対して,
        $\ftorM(S)$が代数幾何学的対象の$S$上のfamily達を
        familyの間の同値関係で割ったもの
        (``$\{ \text{families over }S \}/\sim_S$" in \cite{Hos})である
        ような $\ftorM : \Sch \to \Set$のことである.
        morphism :: $f : S \to T$は,
        $\ftorM$によってpullbackに写される.
        すなわち,$\phi: \famF \to T$は$\ftorM(f)$によって
        $\famF \times_T S \to S$に写される.
        \[\xymatrix{
                \famF \times_T S \ar[d]_-{\ftorM(f)(\phi)}\ar[r]& \famF \ar[d]^-{\phi}\\
            S \ar[r]_-{f}& T
        }\]
    \end{Def}
    moduli functorの定義はあえて曖昧に述べられている.
    これは「出来る限り多くのものをmoduli theoryの範疇に取り込みたい」
    という思いがあるからである(\cite{HaMo}).

    \subsection{Fine Moduli Space}
    \begin{Def}
        scheme :: $M$が
        moduli fuctor :: $\ftorM$に対するfine moduli spaceであるとは,
        $M$が$\ftorM$を表現する(represent)ということである.
        言い換えれば,
        関手$\func{M}=\Hom_{\Sch}(-, M)$が$\ftorM$と自然同型,ということである.
    \end{Def}

    \begin{Remark}
        moduli functor :: $\ftorM$のfine moduli space :: $M$が存在したとしよう.
        この時,任意の$X \in \Sch$について$\ftorM(X) \iso \func{M}(X)$.
        これは
        $X$上の代数幾何学的対象が成す同値類が
        $M$の$X$-value pointと一対一に対応していることを意味する.
        したがって,
        $\ftorM$が指定する代数幾何学的対象の集合の同値類を
        $M$が「パラメトライズ」していると考えられる.
    \end{Remark}

    \begin{Def}
        moduli fuctor :: $\ftorM$に対する
        fine moduli spaceを$M$であるとする.
        また$\Psi: \ftorM \to \func{M}$を自然同型とする.
        $u=\Psi_M^{-1}(\id[M]): \famU \to M$をuniversal familyと呼ぶ.
    \end{Def}

    universal familyの名前の由来は次の命題に拠る.
    \begin{Prop}\label{prop:univfamily}
        任意のfamily :: $\phi: \famF \to B \in \ftorM(B)$は,
        $\chi=\Psi(\phi): B \to M$とuniversal family :: $u: \famU \to M$の
        pullback (fiber product)として得られる.
    \end{Prop}
    \begin{proof}
        $\Psi: \ftorM \to \func{M}$は自然同型であるから,
        $\chi=\Psi(\phi): B \to M$から次の可換図式が得られる.
        \[\xymatrix@=30pt{
                \ftorM(B) \ar[d]_-{\Psi_B}& \ftorM(M) \ar[l]_-{\ftorM(\chi)}\ar[d]^-{\Psi_M}\\
                \func{M}(B) & \ar[l]^-{\func{M}(\chi)} \func{M}(M) \\
        }\]
        $u \in \ftorM(M)$を$\ftorM(\chi)$で写すと$\famU \times_M B \to B$になる.
        同じ$u$を$\func{M}(B)$まで写すと,$\Psi_M(u) \circ \chi=\chi$になる.
        これを$\Psi_B^{-1}$で写せば$\phi:\famF \to B$.
        上の図式は可換図式であったから,
        $\phi=\famU \times_M B \to B$.
    \end{proof}

    \begin{Example}[\cite{Hos}, Exercise 2.20]
        例\ref{example:grassmannian}で述べた
        $\affine^n$の$r$次元線形空間の$S$上のfamily
        (vector bundle over $S$)の集合を,
        vector bundleの同型で割った集合を$\ftorM(S)$とする.
        $f: T \to S$に対する$\ftorM(f)$は,
        vector bundleへのpost-compositionで自然に定まる.

        このmoduli functorはfine moduli spaceを持つことが知られている.
        これがGrassmannian varietyである.
    \end{Example}

    残念ながら,多くのmoduli functorに対してfine moduli spaceが存在し得ない.
    (このあたりの議論は\cite{HaMo} p.3や\cite{HarDef} p.150にある.
    この節の終わりでも理由と例を示す.)
    そのためMumfordは(おそらくGIT本で)
    fine moduli spaceの代わりとしてcoarse moduli spaceを提唱した.

    \subsection{Coarse Moduli Space}
    \begin{Def}
        moduli functor :: $\ftorM$に対して,
        以下を満たすscheme :: $M$を$\ftorM$のcoarse moduli spaceと呼ぶ.
        \begin{enumerate}[label=(\roman*), leftmargin=*]
            \item
                自然変換$\eta: \ftorM \to \func{M}$が存在する.
            \item
                $\eta$は自然変換$\ftorM \to \func{\tilde{M}}$の中で最も普遍的である:
                \[
                \xymatrix
                {
                    {} & \ar[ld]_-{{}^{\forall} \tau}\ftorM \ar[rd]^-{\eta}& {} \\
                    {}^{\forall}\func{\tilde{M}} \ar[rr]_-{{}^{\exists!} \func{f}}& {} & \func{M}
                }
                \]
                この図式で$\tilde{M}$ :: scheme, $f: M \to \tilde{M}$.
            \item
                任意の代数閉体 :: $k$について
                $\eta_{\Spec k}: \ftorM(\Spec k) \to \func{M}(\Spec k)$は全単射である.
        \end{enumerate}
    \end{Def}

    \begin{Example}
        楕円曲線の$j$-不変量.
        後に示すとおり,これはfineでない.
    \end{Example}

    \subsection{Properties of Fine / Coarse Moduli Spaces}
    \begin{Prop}
        moduli functor :: $\ftorM$に対して
        coarse moduli spaceは同型を除いて一意である.
    \end{Prop}

    \begin{Prop}[\cite{HarDef}, Prop23.6]
        scheme :: $M$が
        moduli functor :: $\ftorM$に対する
        fine moduli spaceであるならば,
        $M$は$\ftorM$のcoarse moduli spaceでもある.
    \end{Prop}
    この二つをまとめると次の図式に成る.
    \[\xymatrix{
        \text{Fine moduli} \ar@{=>}[r]& \text{Coarse moduli} \ar@{=>}[r]& \text{Universality} \ar@{=>}[d]\\
        {} & {} & \text{Uniqueness}
    }\]
    
    \begin{Prop}[\cite{HarDef}, Prop23.5]
        $S$ :: schemeのopen subschemeと包含写像が成す圏を
        $\OpenSubSch(S)$と書くことにする.
        これは$\Sch/S$のfull subcategoryである.

        moduli functor :: $\ftorM$が
        fine moduli spaceをもつならば,
        任意の$S$ :: schemeについて
        $\ftorM|_{\OpenSubSch(S)}$は$S$上のsheafである.
        言い換えれば,
        $\ftorM$はZariski topology上のsheafである.
    \end{Prop}
    \begin{proof}
        $M$ :: fine moduli scheme for $\ftorM$とし,
        $S$ :: schemeを固定する.
        $\shF:=\func{M}|_{\OpenSubSch(S)}$は
        開集合系からのcontravariant functorだから
        presheafであることは定義から従う.
        また$\shF$の元はschemeのmorphismである.
        このことからsheafの公理Identity AxiomとGluability Axiomを
        満たすことも簡単に分かる.
        (一応,\cite{HarAG} II, Thm3.3 Step3を参考に挙げる.)
    \end{proof}

\section{Pathological behaviour.}
    \begin{question}
        Fine/Coarse Moduli Spaceはいつ存在し,いつ存在しないのか?
    \end{question}

    Moduli問題の対象と一対一に対応するschemeが見つかったからと言って,
    それがfine moduli spaceであるとは言えない.
    問題と成るのは,familyの同型である.
    ここではfine moduli spaceが存在するということから導かれる必要条件を二つ考え,
    それから例を見る.fine moduli spaceが存在するならば
    例では特にautomorphismの存在とjump phenomenonが
    fine moduli spaceが存在するための障害と成ることを見る.

    \subsection{Fine moduli spaceが存在することの必要条件.}
    $\famF, \famG \to S$をfiber of morphismで実現されるfamilyだとする.
    (したがって$\famF, \famG$はschemeである.)
    $\famF, \famG$の同値関係を,schemeとしての同型で定めよう.
    $M$をfine moduli space, 
    $\eta$をmoduli functorから$\func{M}$への自然同型だとする.

    $\eta_S(\famF): S \to M$は$\famF$のfiberを$M$の点に対応させる.
    (添字の$S$は以降略す.)
    \[
    \xymatrix@R=10pt
    {
        \famF \ar[r]& S \ar[r]^-{\eta(\famF)}& M \\
        \famF_s \ar@{{|}->}[r]& s \ar@{{|}->}[r]& m
    }
    \]
    $\eta(\famG): S \to M$についても同様である.
    したがって$\famF, \famG$が
    fiber毎に同型であれば,$\eta(\famF)=\eta(\famG)$となる.

    それぞれのfiberが互いに同型である
    (i.e. $\Forall{t,s \in S} \famF_t \iso \famF_s$)ような
    familyをfiberwise trivial family, 
    対象$X$(なめらかな曲線など)を用いて
    $X \times S \to S$の形に書けるfamilyをtrivial familyと呼ぶ.
    上での議論から,fine moduli spaceが存在するならば,
    fiberwise trivial familyはtrivial familyである
    (cf. \cite{HarDef} Remark23.1.1).
%    これはfiberwise trivialを自然同型$\eta$で写すと定値写像に成り,
%    定値写像を$\eta^{-1}$で戻せばtrivial familyになるからである.

    \subsection{\texorpdfstring{$j$}{j}-invariant is not a fine moduli space.}
    一つ例を見よう.

    $S=\affine^1_k-\{0\}$とする.
    $S$上の楕円曲線のfamily :: $\famF$を次で定める.
    \[
        \famF=\zerosa(y^2-x^3-s) \subseteq \affine^2_k \times_k S
        \xrightarrow{\pr} S.
    \]
    $\eta(\famF)$を$j$不変量を用いて$s \mapsto j(\shF_s)$で定める.
    $j$不変量がcoarse moduliであることは既に見た.
    計算すると分かる通り,$\eta(\famF)$は定値写像である.
    したがって$\famF$のそれぞれのfiberは互いに同型である.
    一方,$\famF'=\zerosa(y^2-x^3-1) \times S$について
    同様に$\eta(\famF')$を定めると,
    これも自明に定値写像である.
    しかし,$\famF \not \iso \famF'$であることが示せる.
    よって$j$不変量はfine moduliにならない.
    fine/coarse moduliの一意性から,
    楕円曲線はfine moduliを持たない.
    \begin{proof}[proof of $\famF \not \iso \famF'$]
        \cite{HarAG} I, Ex6.2を参考にする.
        我々が調べるのは次の二つの環である.
        それぞれ$\famF, \famF'$である.
        \[
            A=k[x,y, t,t^{-1}]/(y^2-x^3-t),
            \qquad
            B=k[x,y]/(y^2-x^3-1) \otimes_k k[t,t^{-1}],
        \]
        $A$はUFDであるが$B$はUFDでない(GCD domainでさえない),
        ということを示す.

        $A$は$k[x,y]_{y^2-x^3}$(1元での局所化)と同型である.
        $k[x,y]$はUFDであり,irreducible elementでの局所化でこれは保たれる.
        すなわち$A$はUFD.

        $B$がUFDでないことを示すために,
        $\bar{x}=x \bmod (y^2-x^3-1)$がnot primeだがirreducibleであることを示す.

        $\bar{x}$ :: not primeを示すために次の等式を考える.
        \[ \bar{x}^3=\bar{x} \cdot \bar{x} \cdot \bar{x}=(\bar{y}+1) \cdot (\bar{y}-1). \]
        $\bar{x}$ :: primeと仮定すると,
        $\bar{y}+1 \mor \bar{y}-1 \in (\bar{x})$となる.
        そこで例えば
        \[ \bar{y}+1=a \bar{x} \]
        なる$\bar{a} \in B$が存在するとしよう.
        すると$y+1-a x \in I$が得られる.
        これは楕円曲線$y^2=x^3+1$が$y+1-a x=0$という曲線に含まれていることを意味する.
        したがって$x=0$と楕円曲線の交点は,
        存在しても$(x,y)=(0,-1)$の一つのみ,ということになる.
        しかし実際は$(0,-1)$もこの楕円曲線に属すので矛盾.

        $\bar{x}$ :: irreducibleを示すために$\sigma$と$N$を準備する.
        $\sigma: k[x,y] \to k[x,y]$を$y \mapsto -y$で他の元は変化させないものとする.
        すると$\sigma(I) \subseteq I$なので$\sigma: B \to B$が誘導される.
        さらに$N(a)=a \cdot \sigma(a)$で$N: B \to k[\bar{x}]$を定める.
        $N$は積について準同型であることに注意せよ.

        $\bar{x}$がirreducibleでないならば,
        $\bar{x}=fg \bmod I$なる$f,g \in k[x,y]$が存在する.
        $f \bmod I, g \bmod I$はどちらも単元でない.
        両辺を$N$で写すと次のように成る.
        \[ (x^2-N(f)N(g)) \bmod I=0. \]

        したがって$x^2-N(f)N(g)=a(y^2-x^3-1)$なる$a \in k[x,y]$が存在する.
        左辺は$k[x]$に属すから,$y$の次数を考えると$a=0$が示される.
        また$N(f), N(g)$の次数は$2$以上であるから,
        $N(f), N(g)$のいずれかは$k^{\times}$の元である.
        しかし$N(f)=f \cdot \sigma(f)$ (resp. $N(g)$)が単元ならば$f$ (resp. $g$)も単元であり,
        $f,g$についての仮定に反する.
        よって$\bar{x}$ :: irreducible.
    \end{proof}

    したがってmoduli functorは必ずしもfine moduli spaceを持たない.

    \subsection{Automorphism is an obstruction to the existence of fine moduli space.}
    Moduli問題の対象が非自明な自己同型写像をもつなら,
    多くの場合でfine moduli spaceが存在し得ない.

    例を二つ考える.
    最初の例はschemeの例ではないが,
    直観的である.
    \begin{Example}
        $k$ :: fieldとし,
        $\affine^2_k$の原点を通る直線を,\kenten{同型を無視して}分類する.
        直線は全て同型であるから,これは一つしか無い.
        したがってこの問題に対するfine moduli spaceが存在すれば,
        それは一点空間である.
        したがって任意のscheme :: $B$について,
        $B$上のfamilyは全てtrivial familyと同型である.
        
        $L$を$\affine^2$の原点を通る直線とし,
        その非自明な自己同型$\sigma: L \to L$をとる.
        $[0,1]$上のtrivial fiber :: $[0,1] \times L$を,
        次の同値関係で割って商空間を作る.
        \[ (t, Q) \sim (s, Q) \iff |t-s|=1 \land P=\sigma(Q) \mwhere s,t \in [0,1], P,Q \in L. \]
        例えば$\sigma$を$(x,y) \mapsto (-x,-y)$と置くと,
        これは丁度メビウスの和である.
        そしてこれは$S^1$上のfamilyとなっている.

        今$S^1$上のfamilyとして,
        $\sigma$を使って構成したものとtrivial family (斜めになった円筒)がある.
        これらは明らかに同型ではない.
    \end{Example}

    次は\cite{HaMo} 2.Aで挙げられている例である.
    \ref{prop:univfamily}の存在(と,fiber productの普遍性)に対して
    矛盾が生じるように構成する.
    構成には(two )pullback lemmaを用いる.
    (生じる矛盾がわかりやすく述べられていないので,再構成した.)
    \begin{Example}
        
    \end{Example}

    ただし,automorphismが存在するにも関わらずfine moduli spaceが
    存在することもある.
    nLab参照.

    \subsection{jump phenomenon.}
    coarse moduli spaceさえ持ち得ないmoduli functorもある.
%    あるいはmoduli fuctorが``unbounded"であるときに起きる.
%    このノートでは深追いしない.
%    詳しくは\cite{Hos} \S 2.4を参照せよ.

\bibliographystyle{jplain}
\bibliography{reference}
\end{document}
