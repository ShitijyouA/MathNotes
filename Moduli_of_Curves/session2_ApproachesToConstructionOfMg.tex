\documentclass[a4paper]{jsarticle}
\usepackage[]{../math_note}
\usepackage[all]{xy}
\usepackage{calc} 
\usepackage[]{enumitem}
\usepackage[]{macros}

%% for hyperref {{{
\usepackage[dvipdfmx, colorlinks=true, linkcolor=black]{hyperref}
\usepackage{pxjahyper}
%% }}}

\begin{document}
\title{ゼミノート \#3 \\ $\M_g$の構成方針 }
\author{七条彰紀}
\maketitle
    以降は,curveと言えば
    \begin{center}
        smooth, complete, reduced and connected scheme of dimension $1$ over $\C$
    \end{center}
    のことである.
    \cite{HarAG} II, 6.7より,以上の意味でのcurveはprojectiveである.
    (geometric) genus of curveは通常$g$で表す.

\section{ Moduli spaces we'll be concerned with }
    以降で考えていくmoduli spaceを簡単に紹介する.

    \subsection{\tp{$\modM_g$}{Mg}
        :: the coarse moduli space of curves of genus \tp{$g$}{g}. }
    これまで議論してきた.
    まだ存在は示されていない.
    trivial automorphismしか持たないcurveに対応する$\modM_g$の点全体を
    $\modM_g^0$と書くことにする.
    これは$\modM_g$の開集合であることが知られている.

    \subsection{\tp{$\modM_{g,n}$}{Mgn}
        :: the coarse moduli space of pairs of curve of genus \tp{$g$}{g} and \tp{$n$}{n} distinct points. }
    $\modC_g=\modM_{g,1}$もここで述べる.

    corve of genus $g$ :: $C$と
    $C$の$n$個の\kenten{互いに異なる}点 :: $p_1,\dots, p_n$を合わせた\kenten{順序}組
    $(C, p_1,\dots, p_n)$のmoduli spaceを$\modM_{g,n}$と呼ぶ.

    \cite{HaMo}によれば,
    圏点をつけた条件(互いに異なる点の順序組)は,
    $\modM_{d,g}$のcompactificationを考える上で必要である.
    また,curve :: $C$と,互いに異なるとは限らない点の順序無し組の組
    $(C, \{p_1,\dots,p_n\})$のcoarse moduli spaceを構成することも出来る.

    $(C, p_1,\dots,p_n)$から$n$点$p_1,\dots,p_n$の情報を忘れると,
    標準的な射$\modM_{g,n} \to \modM_g$が得られる.

    $\modM_{g,n}$は$\modM_g$に比べて次元が$n$大きく,
    元の$\modM_g$の情報を得づらいという問題がある.
    しかし,$\modM_{g,n}$はしばしば自然に現れるので,
    文脈によっては大きな意味を持つ.

    \subsection{\tp{$\modP_{d,g}$}{Pdg}
        :: the coarse moduli space of pairs of curve of genus \tp{$g$}{g} and line bundle of degree \tp{$d$}{d}. }
    $\modP_{d,g}$は,
    curve of genus $g$とその上のline bundle of degree $d$の組$(C, \shL)$から
    $\shL$の情報を忘れれば,
    標準的な射$\phi: \modP_{d,g} \to \modM_g$が得られる.

    \paragraph{}
    次の同型が存在する.
    \[
    \begin{aligned}
        \modP_{d,g}& &\iso& &\modP_{d+(2g-2), g},&
        && \quad
        \modP_{d,g}& &\iso& &\modP_{-d, g}& \\
        (C, \shL)& &\mapsto& &(C, L \otimes K_C),&
        && \quad
        (C, \shL)& &\mapsto& &(C, L^{-1})&.
    \end{aligned}
    \]
    このこととExercise 2.6から,
    互いに同型にならない$\modP_{d,g}$は,
    各$g$に対して丁度$g-1$個ある
    \footnote
    {
        $\Z$を$s: d \mapsto d+(2g-2)$と$t: d \mapsto -d$の二つの自己同型で生成される群で割る.
        $s$が生成する群は$(2g-2)\Z (< \Z)$と同型で,$t$が生成する群は$\Z/2\Z$と同型.
        よって$\#( \Z/(2g-2)\Z \times (\Z/2\Z)) )=(2g-2)/2=g-1$.
    }.
    \[ \modP_{0,g}, \modP_{1,g}, \dots, \modP_{d-1,g}. \]

    $\modP_{d,g}$のうち,
    $\modP_{g-1, g}$は``Theta characteristic"と
    呼ばれるものを付加構造としたmoduli spaceである.
    参考文献:
    Gavril Farkas
    ``Theta characteristics and their moduli"
    \footnote{\url{https://arxiv.org/abs/1201.2557}}
    
\section{Constructions of \tp{$\modM_g$}{Mg}}

    \subsection{Generally Steps of Construction of Moduli Space.}
    moduli spaceの構成方法はある程度決まった手順がある.
    ここではそれを述べる.

    まず,対象$X$と付随する情報(extra data)の組たちを,
    何らかのparameter space :: $W$の点に対応させる.
    parameter spaceは対象と付加情報の組そのもの(同値類でなく)が成す空間である.
    例えば平面上の原点を通る直線のparameter spaceは$\proj^1$である.
    $1$つの対象の同型類$[X]$に対応する$W$の点たちが成す集合$S_{[X]}$を観察する.
    この集合$S_{[X]}$を何らかの群$G$の$W$への作用に拠る軌道と考えることが出来れば
    ($S_{[X]}=Gw$なる$w \in W$が存在すれば),
    求めるmoduli spaceは商空間$W/G$として実現できる.

    まとめると,
    moduli spaceを構成する際には以下の$4$つの要素を中心に考えることに成る.
    \begin{description}[leftmargin=!,labelwidth=\widthof{\bfseries Container Space}]
        \item[Extra Data] 分類対象(Object)に付随させる情報.
        \item[Parameter Space] 組(Object, Extra Data)が成す空間.
        \item[Group] Parameter Spaceに作用し,$1$つのObjectに対応する点の集合が$1$つの軌道である群.
    \end{description}
    
    \begin{Example}(\cite{Muk1})
        $k$ :: fieldとし,
        moduli space of hypersurface of degree $d$ in $\proj_{k}^n$を構成しよう.
        $H$ :: hypersurface of degree $d$ in $\proj_{k}^n$は,
        次のような形の$k[x_1,\dots,x_n]$の斉次$d$次多項式で定まる.
        \[ \sum_{|\alpha|=d} a_{\alpha} x^{\alpha}. \]
        ただし$\alpha$は多重添字である.
        そして多項式はその係数$a$で定まる.
        $a$は$k^{\oplus N}$($N:=\binom{n+d}{d}$)の元である.
        したがって$H$は$\affine_k^N$(Parameter Space)の
        点$(a_{(d,0,\dots,0)}, \dots, a_{(0,\dots,0,d)})$に対応する.

        しかし,$a$に正則行列$g \in GL_{n+1}(k)$ (Group)を作用させた$a'$も,
        $H$と同型なhypersurfaceに対応する
        ($g$の作用のさせ方はここで述べない).
        逆に$H$の同型なhypersurfaceに対応する$\affine^N$の点の全体は,
        $GL_{n+1}(k)$による$a$の軌道として得られる.
        よって$\affine_k^N/GL_{n+1}(k)$がもとめるmoduli spaceである.
    \end{Example}

    以下では$\modM_g$ :: the coarse moduli space of smooth curves of genus $g$
    の構成方法の概略を述べる.分類対象(Object)に付随させる情報.
    方法は大きく分けて$3$つある.
    最初の二つは解析的な方法で,最後のものは完全に代数的である.

    \subsection{The Teichm\tp{\"u}{u}ller approach}
    \begin{description}[leftmargin=!,labelwidth=\widthof{\bfseries Parameter Space}]
        \item[Extra Data]
            Normalized set of generators for $\pi_1(C)$, \\
            or Homeomorphism which $\an{C}$ to standard compact orientable surface $X_0$.
        \item[Parameter Space] Teichm\"uller space :: $T_g \subseteq \C^{3g-3}$.
        \item[Group] $\Gamma_G$ :: Group of diffeomorphisms of $X_0$, modulo isotopy.
    \end{description}
    この方法で構成された$\modM_g$はanalytic varietyになる.

    この方法の利点は,
    $\modM_g$の位相を扱いやすいことと,
    $\modM_g$に自然な計量を入れられることである.

    Teichm\"uller space :: $T_g$は,open ballであることが知られている.
    すなわち,contractable spaceとなっている.
    そこで$T_g$の代わりに扱いやすいcontractable spaceを考え,
    その$\Gamma_g$による商を考えることで$\modM_g$のcohomologyについて調べることが出来る.
    主にHarerがこの方法で成果をあげた.
    この成果についてはこのセミナーでものちに取り上げる.

    $\modM_g$に計量を入れて,
    それをもちいてprojective varietyへの埋め込みを与える,
    ということをWolpertが行った.
    この埋め込み先のprojective varietyは
    Deligne--Mumford compactificationと共通の良い性質を多く持っている.
    なお,計量の入れ方は複数存在する.
    参考文献は
    Kefeng Liu, Xiaofeng Sun, Shin-Tung Yau
    ``Geometric Aspects of the Moduli Space of Riemann Surfaces"
    \footnote{\url{ https://arxiv.org/abs/math/0411247 }}.

    \subsection{The Hodge theory approach}
    \begin{description}[leftmargin=!,labelwidth=\widthof{\bfseries Parameter Space}]
        \item[Extra Data]
            \begin{enumerate}[leftmargin=*]
                \item Symplectic basis of $H_1(C, \Z)$ :: $\{a_1,\dots,a_g, b_1,\dots,b_g\}$,
                \item Basis of $H^0(C, K_C)$ :: $\{\omega_1,\dots,\omega_g\}$,
                \item The intersection pairing.
            \end{enumerate}
        \item[Parameter Space] $\mathfrak{c}_g \subseteq \mathfrak{h}_g$.
        \item[Correspondance] $P=[\int_{b_i} \omega_j ]_{i,j} \in \mathfrak{h}_g$
        \item[Group] $Sp_{2g}(\Z)$ :: Symplectic group.
    \end{description}
    ここで$\mathfrak{h}_g$は次のように定義される.
    \[
        \mathfrak{h}_g=
        \left\{
            \tau \in M_{g \times g}(\C)
            ~\middle|~
            \tau^T=\tau, \Im(\tau)\text{ :: positive difinite.}
        \right\}
    \]
    これはSiegel upper-halfspace of dimension $g$と呼ばれている.
    $\mathfrak{h}_1$が通常のupper-half planeと一致することに注意.

    $\{a_1,\dots,a_g\} \subset H_1(C, \Z)$は
    $[\int_{b_i} \omega_j ]_{i,j}=I_g$(単位行列)であるように選ばれる.
    $b_1,\dots,b_g$の選び方によって$P=[\int_{b_i} \omega_j ]_{i,j} \in \mathfrak{c}_g$は変わるが,
    これは以下の$Sp_{2g}(\Z)$による作用に対応する.
    \[
        Sp_{2g}(\Z)=
        \left\{
            \gamma \in GL_{2g}(\Z)
            ~\middle|~
            \gamma^T \Omega \gamma=\Omega
        \right\},
        \mwhere
        \Omega=\mat{ 0 & I_g \\ -I_g & 0 }.
    \]
    構成方法から,
    $\modM_g$は$\modA_g=\mathfrak{h}_g/Sp_{2g}(\Z)$に含まれる.
    $\modA_g$はcoarse moduli space for abelian varieties of dimension $g$である.

    この方法は$Sp_{2g}(\Z)$が$\Gamma_g$よりも分かりやすいという点で
    Teichm\"uller approachに優っている.
    しかし$\mathfrak{c}_g$の方は把握が難しく,
    「$\mathfrak{c}_g$はどのようなものか」という問はthe Schottky problemと呼ばれている.
    これについては様々な考察がなされているが,
    $\mathfrak{c}_g$の具体的な記述は得られていない.

    この方法の別の利点は,
    compactification of $\modA_g$ :: $\tilde{\modA_g}$
    \footnote{ $\modA_g$をanalytic open subsetとして含むcompact analytic varietyの事. }
    が自然に得られるということである.
    compactification of $\modM_g$ :: $\tilde{\modM_g}$は
    Statake compactificationと呼ばれ.
    $\tilde{\modA_g}$での$\modM_g$の閉包として得られる.
    
    しかし,$\tilde{\modM_g}$はいかなるmoduli functorのcoarse moduli spaceでもないため,
    (以下で述べる例を除いては)$\modM_g$自体の研究には役立てられない.
    実際,$\tilde{\modM_g}-\modM_g$は種数が$g$より小さいsmooth curveに対応しているため,
    $\tilde{\modM_g}$上のfamilyを考えるということは出来ない.
    種数$g$の曲線と種数が$g$未満の曲線の両方をfiberにもつfamilyは,
    どこかでsingularなfiberを持つからである
    (種数はhomotopy/birational不変量であったことを想起せよ).

    $\tilde{\modM_g}$を用いた議論によって,
    $\modM_g$がprojectiveでもaffineでも無いことが分かる
    (TODO: ここでの$\modM_g$ってschemeではないでのは?).

    \subsection{The geometric invariant theory (G.I.T.) approach}
    $n \geq 3$を任意にとって固定する.
    \begin{description}[leftmargin=!,labelwidth=\widthof{\bfseries Parameter Space}]
        \item[Extra Data] (Nothing.)
        \item[Parameter Space] $K \subseteq \modH_{2(g-1)n, g, N}$ ($N:=(2n-1)(g-1)-1$).
        \item[Group] $PGL_{N+1}(\C)$.
    \end{description}
    $\modH_{2(g-1)n, g, N}$は
    subscheme of degree $2(g-1)n$ and genus $g$ in $\proj^N$の
    Hilbert schemeである.

    この方法の利点は,代数的であることの他に二つある.
    \begin{enumerate}
        \item $\modM_g$がquasiprojective algebraic varietyとして得られる.
        \item compactification of $\modM_g$についての考察が自然に得られる.
    \end{enumerate}

    \subsubsection{Compactification of $\modM_g$ and Stable Curve.}
    compactification of $\modM_g$(ここでは$\modM_g$を含むprojetive scheme)を得る方法として,
    $K$の$\modH_{2(g-1)n, g, N}(=:\modH)$での閉包を取って$PGL_{N+1}(\C)$で割る,
    ということが思いつく.
    しかしこれで得られるのは$K$のcompactificationでなく,
    $K$を含む集合$\tilde{K}$の商$\tilde{K}/PGL_{N+1}(\C)$のcompactificationである.
    これらの包含関係は$K \subset \tilde{K} \subset \cl_{\modH}(K)$となる.
    
    この拡張が必要な理由は,次のように説明される:
    次のような$t \in \affine^1-\{0\}$で
    パラメトライズされるfamily of smooth curvesを考える.
    has only nodes as singularities and has only finitely many automor-
    phisms.
    \[ C: y^2z=x^3-t^2axz-t^3bz^3 \mwhere a,b,t \in \C, t \neq 0. \]
    $t \neq 0$ならば$C_t \iso C_1$となる.
    しかし$C_0$はcuspidal curveとなる.
    $C \to \affine^1-\{0\}$に対応する$j$-invariant mapを
    $\chi: \affine^1-\{0\} \to \affine^1$とすると,
    $t \to 0$で$\chi$の値は$\affine^1_j$の外側の点に収束してしまう.
    なので$\modM_1=\affine^1_j$をコンパクト化するには,
    $C_0$に対応する点を$\modM_1$に加えなければならない.
    なお,この曲線族は$a,b$の値を変えることで任意の楕円曲線を含むものに成る.

    では$\tilde{K}$に含まれる曲線は何だろうか,ということになるが,
    これは(Deligne-Mumford) stable curveと呼ばれるものである.
    次のsessionで詳しく述べる.

\bibliographystyle{jplain}
\bibliography{reference}
\end{document}
