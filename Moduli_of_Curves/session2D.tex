\documentclass[a4paper]{jsarticle}
\usepackage[]{../math_note}
\usepackage[all]{xy}
\usepackage{calc}  
\usepackage[]{enumitem}
\usepackage[]{macros}

%% for hyperref {{{
\usepackage[dvipdfmx, colorlinks=true, linkcolor=black]{hyperref}
\usepackage{pxjahyper}
%% }}}

\newcommand{\tp}[2]{\texorpdfstring{#1}{#2}}

%% moduli spaces
\newcommand{\modA}{\mathcal{A}}
\newcommand{\modC}{\mathcal{C}}
\newcommand{\modH}{\mathcal{H}}
\newcommand{\modP}{\mathcal{P}}
\newcommand{\modT}{\mathcal{T}}
\newcommand{\an}[1]{{#1}^{\,\mathrm{an}}}
\newcommand{\Pic}{\operatorname{Pic}}

\newcommand{\M}{\mathcal{M}}
\newcommand{\K}{\mathcal{K}}
\newcommand{\barM}{\overline{\mathcal{M}}}

\newcommand{\Sing}{\operatorname{Sing}}
\newcommand{\Sm}{\operatorname{Sm}}

\begin{document}
\title{ゼミノート \#3}
\author{七条彰紀}
\maketitle

以下では主に$g \geq 1$の場合を考える.
また,考えるのは$\C$上のschemeのみである.

\begin{problem}
    What properties $\modM_g$ and $\bar{\modM}_g$ does have?
\end{problem}

\section{Basic Properties}
    \paragraph{$\dim \modM_g=3g-3$.}
    $\M_g$はHilbert schemeの開集合の$PGL(N+1, \C)$による作用での商であった.
    Hilbert schemeについてはその次元を含めよく分かっている.
    そこで,$\dim \M_g$を用いてHilbert schemeの次元を記述する,
    という考え方で$\dim \M_g$を計算する.

    \paragraph{$\modM_g$ :: Irreducible.}
    

\section{Local Properties.}
    \subsection{Singular Locus.}
        singular locus of $\barM_g$を$\Sing \M_g$と書く.
        smooth locus of $\M_g$を$\Sm \M_g \ (=\M_g-\Sing \M_g)$と書く.
        \paragraph{$g \geq 4$.}
            $\Sm \M_g$はnon-trivial automorphismを持たないcurveのlocusに一致する.

        \paragraph{$g=2$.}
        
        \paragraph{$g=3$.}

    \subsection{Local Looks.}
    $[C] \in \Sing \M_g$の解析的近傍は,
    $\C^{3g-3}$の開集合を有限群の線形作用で割ったようなものに成る.

    \subsection{\tp{$\Delta=\barM_g-\M_g$}{D=barMg-Mg}.}
        nodeを$\delta$個持つstable curveが成すlocusを考える.
        \begin{Claim}
            nodeを$\delta$個持つstable curveが成すlocusを
            $N_{\delta} \subset \bar{\modM}_g$とする.
            この時,
            \[ \dim N_{\delta}=3g-3-\delta \quad (\implies \codim N_{\delta}=\delta). \]
            また,$\cl_{\bar{\modM}_g}(N_{\delta})$は
            nodeを$\delta$個以上持つstable curveが成すlocusに一致する.
        \end{Claim}
        このことは\cite{HaMo} Thm3.150直後の段落でも触れられている.

        nodeを$1$個以上持つcurveのlocus :: $\Delta=\barM_g-\M_g$は,
        $\M_g$が$\barM_g$の開集合であるから,これはclosed in $\barM_g$.
        上の主張から,$\Delta$はnodeを丁度$1$つ持つcurveのlocusのclosureである.
        そこで$\Delta_0$と$\Delta_i \ (i=1,\dots,\lfloor g/2\rfloor)$を次のように定める.
        \begin{itemize}
            \item $\Delta_0=\cl_{\barM_g}(\{ [C] \in \barM_g \mid
                    C \text{ :: irreducible curve with $1$ node } \}).$
            \item $\Delta_0=\cl_{\barM_g}(\{ [C] \in \barM_g \mid
                    C \text{ :: union of two smooth curves of genus $i$ and $g-i$, meeting at $1$ pt } \})$
                    for $i=1,\dots,\lfloor g/2\rfloor.$
        \end{itemize}
        (TODO: なぜ$\Delta_0$はsmooth curvesの和でないのか?)

        $\Delta_0,\dots,\Delta_{\lfloor g/2 \rfloor}$はirreducibleである.
        これは以下のように証明する.
        まず$\Delta_0$を考える.
        $C$ :: irreducible curve with $1$ nodeとする.
        これのnormalizationを$\tilde{C}$とすると,$C$のnodeは$\tilde{C}$の$2$点に対応する.
        そこで$\tilde{C}$とこの$2$点を組にして$\M_{g-1,2}$の点とする.
        こうして$\phi_0: \M_{g-1, 2} \to \Delta_0$が得られる.
        $\Delta_i (i>0)$の場合,
        $C$のnormalizationはgenus $i$, genus $g-i$のcomponentからなる.
        交点に対応する点をそれぞれ一つずつ持つから,
        これをdistinguished pointとして
        $\phi_i: \M_{i, 1} \times_k \M_{g-i, 1} \to \Delta_i$が得られる.
        こうして得られる$\Delta_0,\dots,\Delta_{\lfloor g/2 \rfloor}$は連続である(FACT).
        
        $\barM_{g,n}$はirreducibleである(Thm2.15).
        したがってその開集合$\M_{g,n}$もirreducibleである($\Delta$ :: closedより).
        $\M_{g,n}$は代数閉体上のscheme(実際にはvariety, Thm2.15)なので,
        これらのfiber productもirreducible.
        連続写像で写す操作と閉包をとる操作でirreducibilityが保たれるので,
        $\Delta_i=\cl(\im \phi_i) \ (i=0,\dots,\lfloor g/2 \rfloor)$はirreducibleである.
        
\section{How Close is \tp{$\modM_g$}{Mg} to Projective ?}

\section{Cohomology of \tp{$\modM_g$}{Mg}.}

\section{Cohomology of \tp{$\modC_g:=\modM_{g, 1}$}{Cg=Mg,1}.}

\bibliographystyle{jplain}
\bibliography{reference}
\end{document}
