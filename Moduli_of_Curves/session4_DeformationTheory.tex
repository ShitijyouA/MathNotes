\documentclass[a4paper]{jsarticle}
\usepackage[]{macros}

\newcommand{\Isom}{\operatorname{Isom}}
\newcommand{\ftorIsom}{\mathcal{I}\!som}
\newcommand{\hilb}{\mathcal{H}}
\newcommand{\dualnum}{\mathbb{I}}
\newcommand{\shDer}{\Omega}

\begin{document}
\title{ゼミノート \#4 \\ Deformation Theory}
\author{七条彰紀}
\maketitle

\section{Automorphism Group of Stable Curve}
    \cite{HaMo} 3.A, \cite{IrrOfMg} \S 1を参照する.

    $C, D$ :: stable curves of genus $g$ over a scheme $S$の間の
    isomorphism groupのschemeとしての構造を与える.
    このschemeを$\Isom(C, D)$と書く.
    そして$\Aut(C)=\Isom(C,C)$と定義し,
    これのschemeとしての特徴を調べる.
    
    $\Isom(C, D)$の特徴付けをするため,次の関手を考える.
    \begin{defmap}
        \ftorIsom_S(C, D):& \text{(Scheme over $\C$)}& \to& \text{(Set)} \\
        {}& S'& \mapsto& \{ \ C \times_{\C} S' \to D \times_{\C} S' \text{ :: $S'$-isomorphism} \}
    \end{defmap}
    $\iota \in \ftorIsom(C, D)(S')$から得られる$\iota^*$は
    $\dualsh_{C \times S'/S'}=\iota^*(\dualsh_{D \times S'/S'})$を満たす.
    また$\otimes$と交換する
    (すなわちPicard群の間の準同型である.\cite{HarAG} Ex II.6.8).
    このことから$\Isom(C, D)$が適当な$r$をとると
    $PGL(r+1)$の部分群として書けることが分かる.

    もう少し詳しく$\Isom(C, D)$を書く.
    $n \geq 3$を整数とする.次のように$r,d$をとる.
    \[
        r+1=h^0((\dualsh_{C/\C})^{\otimes n})=(2n-1)(g-1),
        \qquad
        d=\deg (\dualsh_{C/\C})^{\otimes n}=2n(g-1).
    \]
    すると\cite{HarAG} II.7より,
    $C, D$は$\proj_{\C}^r$の次数$d$, arithmetic genus $g$のclosed curveとみなせる
    ($\proj^r$に埋め込める).
    なのでHildert scheme :: $\hilb=\hilb_{d,g,r}$の点として扱うことが出来る.
    ここで次のように射を定める.
    \begin{defmap}
        \mu:& PGL(r+1)& \to& \hilb \times \hilb \\
        {}& \alpha& \mapsto& (\alpha \cdot [C], [D])
    \end{defmap}
    すると,$\ftorIsom(C, D)$は$\mu^{-1}(\Delta)$によって表現される
    \footnote
    {
        $\Delta$は$\hilb \times \hilb$のdiagonal set.
        $\mu^{-1}(\Delta)$は
        \[ \Delta \cap \im \mu=\{ (\alpha \cdot [C], [D]) \mid \alpha \cdot [C]=[D] \} \]
        の$PGL(r+1)$への逆像なので,
        この点と$C, D$の間の同型と対応することが分かるだろう.
    }.
    これをgroup scheme over $\C$ :: $\Isom(C, D)$とする.

    scheme over $\C$ :: $X$について少々一般の理論を述べる.
    $\dualnum=\Spec \C[\epsilon]/(\epsilon^2)$とおく(ref. \cite{HaMo} 1).
    \cite{HarAG} Ex II.2.8より,
    $t \in \ftor{X}(\dualnum)$は$X$の$\C$-rational point :: $x$と
    $T_x(X)=\I{m}_x/\I{m}_x^2=\shT_x$の元に対応する.
    ここで$\shT$はtangent sheaf :: $\shT=\shHom_{\shO_X}(\shDer_X, \shO_X)$のことである.
    \cite{HaMo}でいうregular vector fieldとは$\shT$のsectionのこと(と思われる).

    \begin{Thm}
        $C$ :: stable curve of genus $g \geq 2$について,
        \[ \operatorname{Ext}^0(\shDer_C, \shO_C)=\shT(C)=0. \]
    \end{Thm}
    \begin{proof}
        \cite{IrrOfMg} \S1.

        $\pi: \tilde{C} \to C$をnormalization of $C$とする.
        $D \in \shT_C(C)$は$\pi^*$によって
        $\tilde{D} \in \shT_{\tilde{C}}(\tilde{C})$に対応する.
        $D$は$C$のdouble point :: $P$で$0$になるから,
        $\tilde{D}$は$\pi^{-1}(P)$で$0$になる.

        そしてsmooth curve上のtangent sheaf :: 
        $\shT_{\tilde{C}}=\shHom(\shDer_{\tilde{C}}, \shO_{\tilde{C}})$のsupportを考える.
        $\shT_{\tilde{C}}$はcoherent sheafだから,
        $\Supp \shT_{\tilde{C}}$はclosed in $\tilde{C}$(\cite{HarAG} ExII.5.6).
        したがって$(\shT_{\tilde{C}})_{Q} \neq 0$となる点は有限個しか無い.
        (TODO: $(\shT_{\tilde{C}})_{Q}=0$なる点が存在するなら$\shT_{\tilde{C}}(U)=0$を示したいのだが.)
    \end{proof}

    \begin{Prop}
        $\Aut(C)$ :: reduced scheme.
    \end{Prop}
    \begin{proof}
        $X=\Aut(C)$はgroup schemeであるから,
        $X$のある点でのlocalな性質は
        transitionを用いて単位元$I$での性質と言い換えられる.
        reducedはlocalな性質であるから,
        $X$がreducedであることを示すには$\shO_{X, I}$ :: reduced ringを示せば良い.

        $\shO_{X, I}$がnon-reducedであると仮定すると,
        non-zero section :: $t \in \shT_C(C)$がとれ,
        これは$C$のregular vector field :: $v \in \shT(C)$に対応する(TODO).
        したがって定理から$X$ :: reduced.
    \end{proof}

\section{}
    \cite{HarAG} Example III.9.13.1

\bibliographystyle{jplain}
\bibliography{reference}
\end{document}
