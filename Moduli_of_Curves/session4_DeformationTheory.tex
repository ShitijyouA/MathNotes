\documentclass[a4paper]{jsarticle}
\usepackage[]{macros}

\newcommand{\Isom}{\operatorname{Isom}}
\newcommand{\ftorIsom}{\mathcal{I}\!som}
\newcommand{\hilb}{\mathcal{H}}
\newcommand{\dualnum}{\mathbb{I}}
\newcommand{\famX}{\mathcal{X}}
\newcommand{\famY}{\mathcal{Y}}
\newcommand{\cvU}{\mathfrak{U}}
\newcommand{\der}{\mathrm{d}}
\newcommand{\Der}{\mathrm{Der}}

\begin{document}
\title{ゼミノート \#4 \\ Deformation Theory}
\author{七条彰紀}
\maketitle

\section{Automorphism Group of Stable Curve}
    \cite{HaMo} 3.A, \cite{IrrOfMg} \S 1を参照する.

    $C, D$ :: stable curves of genus $g$ over a scheme $S$の間の
    isomorphism groupのschemeとしての構造を与える.
    このschemeを$\Isom(C, D)$と書く.
    そして$\Aut(C)=\Isom(C,C)$と定義し,
    これのschemeとしての特徴を調べる.
    
    $\Isom(C, D)$の特徴付けをするため,次の関手を考える.
    \begin{defmap}
        \ftorIsom_S(C, D):& \text{(Scheme over $\C$)}& \to& \text{(Set)} \\
        {}& S'& \mapsto& \{ \ C \times_{\C} S' \to D \times_{\C} S' \text{ :: $S'$-isomorphism} \}
    \end{defmap}
    $\iota \in \ftorIsom(C, D)(S')$から得られる$\iota^*$は
    $\shDual_{C \times S'/S'}=\iota^*(\shDual_{D \times S'/S'})$を満たす.
    また$\otimes$と交換する
    (すなわちPicard群の間の準同型である.\cite{HarAG} Ex II.6.8).
    このことから$\Isom(C, D)$が適当な$r$をとると
    $PGL(r+1)$の部分群として書けることが分かる.

    もう少し詳しく$\Isom(C, D)$を書く.
    $n \geq 3$を整数とする.次のように$r,d$をとる.
    \[
        r+1=h^0((\shDual_{C/\C})^{\otimes n})=(2n-1)(g-1),
        \qquad
        d=\deg (\shDual_{C/\C})^{\otimes n}=2n(g-1).
    \]
    すると\cite{HarAG} II.7より,
    $C, D$は$\proj_{\C}^r$の次数$d$, arithmetic genus $g$のclosed curveとみなせる
    ($\proj^r$に埋め込める).
    なのでHildert scheme :: $\hilb=\hilb_{d,g,r}$の点として扱うことが出来る.
    ここで次のように射を定める.
    \begin{defmap}
        \mu:& PGL(r+1)& \to& \hilb \times \hilb \\
        {}& \alpha& \mapsto& (\alpha \cdot [C], [D])
    \end{defmap}
    すると,$\ftorIsom(C, D)$は$\mu^{-1}(\Delta)$によって表現される
    \footnote
    {
        $\Delta$は$\hilb \times \hilb$のdiagonal set.
        $\mu^{-1}(\Delta)$は
        \[ \Delta \cap \im \mu=\{ (\alpha \cdot [C], [D]) \mid \alpha \cdot [C]=[D] \} \]
        の$PGL(r+1)$への逆像なので,
        この点と$C, D$の間の同型と対応することが分かるだろう.
    }.
    これをgroup scheme over $\C$ :: $\Isom(C, D)$とする.

    scheme over $\C$ :: $X$について少々一般の理論を述べる.
    $\dualnum=\Spec \C[\epsilon]/(\epsilon^2)$とおく(ref. \cite{HaMo} 1).
    \cite{HarAG} Ex II.2.8より,
    $t \in \ftor{X}(\dualnum)$は$X$の$\C$-rational point :: $x$と
    $T_x(X)=\I{m}_x/\I{m}_x^2=\shT_x$の元に対応する.
    ここで$\shT$はtangent sheaf :: $\shT=\shHom_{\shO_X}(\shDer_X, \shO_X)$のことである.
    \cite{HaMo}でいうregular vector fieldとは$\shT$のsectionのこと(と思われる).

    \begin{Thm}\label{thm:TC(C)=0}
        $C$ :: stable curve of genus $g \geq 2$について,
        \[ \operatorname{Ext}^0(\shDer_C, \shO_C)=H^0(C, \shT_C)=\shT_C(C)=0. \]
    \end{Thm}
    \begin{proof}
        \cite{IrrOfMg} \S1.

        $\pi: \tilde{C} \to C$をnormalization of $C$とする.
        また$\tilde{C}$のconnected componentの
        個数を$\nu$,それぞれのgenusを$g_i \ (i=1,\dots,\nu)$とする.
        
        今,
        $D \in \shT_C(C)$は
        pullback :: $\pi^*: \shT_{\tilde{C}} \to \pi^* \shT_{C}$によって
        \footnote
        {
            $R$ :: ring, $A, B$ :: ring over $R$とする.
            一般に,$k$-homomorphism :: $\phi: A \to B$があるとき,
            $D \in \Der_{R}(B)$は$\phi^*: D \mapsto D \circ \phi$によって
            $\Der_R(A)$へ写すことが出来る.
        }.
        $\tilde{D} \in \shT_{\tilde{C}}(\tilde{C})$
        $C$のdouble pointに$\pi$で対応する点
        (point laying over double point, plodp)で$0$になるような
        regular vector field :: $\tilde{D} \in \shT_{\tilde{C}}(\tilde{C})$
        に対応する(TODO).
        このような$\tilde{D}$は$0$しかないことを確かめれば,
        $\shT_C(C)=0$がわかる.

        \begin{Claim}
            $1$点$P \in \tilde{C}$で$\tilde{D}_P=0$ならば,
            $\tilde{D}=0$である.
        \end{Claim}
        \begin{proof}
            $C$ :: reduced connected schemeに注意する.
            $P \in C$において$\tilde{D} \in \shT_C(C)$が$\tilde{D}_P=0$を満たすとしよう.
            $C$のirreducible affine open cover :: $\cvU$をとり,
            $P \in U$なる$U=\Spec A \in \cvU$をとって固定する.
            すると$C$ :: reducedより$A$ :: integral domain.
            $\tilde{D}|_U \in \shT_C(U)$が$P \in U$で$0$になるのだから,
            次が成立する.
            \[ \Exists{u \in A-\I{p}_P} u \cdot (\tilde{D}|_U)=0. \]
            $A$ :: integralより,これは$\tilde{D}|_U=0$を意味する.
            $U$と交わるirreducible affine open subset of $C$ :: $V \in \cvU$についても,
            $\tilde{D}|_{U \cap V}=0$なので$\tilde{D}|_{V}=0$.
            $C$ :: connectedなので,このように$V$を取り続けることで,
            全ての$V \in \cvU$について$\tilde{D}|_V=0$であることがわかる.
            sheafのIdentity Axiomから,$C$全体で$t=0$.
        \end{proof}

        したがって我々は
        $\tilde{C}$の各componentは少なくとも一つずつ
        plodpをもつこと示せば良い.

        $\shT_{\tilde{C}}=\shHom(\shDer_{\tilde{C}/\C}, \shO_{\tilde{C}})$なので,
        $\shT_{\tilde{C}}$に対応するdivisorは$K_{\tilde{C}}$.
        $\deg K_{\tilde{C}}=2\tilde{g}-2$なので,
        $\tilde{g}>1$ならば$\deg (-K_{\tilde{C}})<0$.
        したがって\cite{HarAG} Lemma IV.1.2から$\dim_{\C} H^0(\tilde{C}, \shT_{\tilde{C}})=0$.
        すなわち$\shT_{\tilde{C}}(\tilde{C})=0$.
        なので以下では$\tilde{g}_i=0,1$とする.

        $\tilde{g}_i=0,1$であるとき,
        $\tilde{C}$の各connected componentは必ずplodpをもつ.
        実際,genus formulaで$\delta=0$とすると
        \[ g=\sum_{i}(\tilde{g}_i-1)+1 \geq 2 \]
        したがって$\sum_{i}(\tilde{g}_i-1)>0$ということになる.
        しかし仮定から$\tilde{g}_i-1 \leq 0$なので,$\delta>0$.
        すなわち$C$は必ずnodeをもつ.
        $\tilde{C}$の各componentはsmoothであることと
        $C$がconnectedであることも踏まえて考えると,
        $\tilde{C}$の各componentは少なくとも一つずつ
        plodpをもつことが分かる.
        (この辺りは\cite{IrrOfMg} Lemma1.4で詳しく述べられている).
    \end{proof}

    \begin{Prop}
        任意の閉点$P \in \Aut(C)$について,
        $\shO_{\Aut(C), P} \iso \C$.
        特に$\Aut(C)$ :: reduced scheme.
    \end{Prop}
    \begin{proof}
        $X=\Aut(C)$はgroup scheme over $\C$であるから,
        $X$のある点でのlocalな性質は
        transitionを用いて単位元$e$での性質と言い換えられる.
        なので$A:=\shO_{X, e}$のみを考える.
        $X$ :: group scheme over $\C$より
        $e$ :: $\C$-rational pointなので,
        $A$が体ならばそれは$\C(=A/\I{m}_A)$と同型である.
        よって我々は$A$が体であることのみ示せば良い.

        上記の定理(\ref{thm:TC(C)=0})から,$\shT_C(C)=0$.
        これは$C \times \dualnum$の$\dualnum$-automorphismは
        自明なものしか無いことを意味する(後述).
        さらに$\Aut(C)$の定義から,
        これは射$\dualnum \to \Aut(C)$としては自明なものしか存在しないことを意味する.
        さらに\cite{HarAG} Ex II.2.8より,
        これは$\I{m}_A/\I{m}_A^2=0$を意味する.
        中山の補題から$\I{m}_A=0$.
        よって$A$は体である.
    \end{proof}

\section{Definitions of Deformations and Versal Deformation.}
\begin{Def}[$\C$-pointed scheme \cite{DefAS} \S 1.2.1]
    scheme :: $Y$と
    $\C$-rational point :: $y_0 \in Y$の組を
    $\C$-pointed schemeを呼び,$(Y, y_0)$と書く.

    morphism of $\C$-pointed schemes :: $(S, s_0) \to (T, t_0)$とは,
    moephism of schemes :: $\phi: S \to T$であって,
    $\phi(s_0)=t_0$を満たすもののこと.
\end{Def}

\begin{Def}[Deformation of Scheme \cite{DefAS} \S 1.2.1, \cite{HaMo} \S 3.B]
    \begin{enumerate}[label=(\roman*), leftmargin=*]
    \item 
        deformation of $X$とは,以下のようなpullback diagramのことである.
        $\psi$から$X \iso \famX \times_{Y} \C$が誘導される.
        \[
            \xi:
        \vcenter{\xymatrix{
            X \ar[r] \ar[d]& \famX \ar[d]^-{\text{flat, surj.}} \\
            \C \ar[r]_-{s}& Y
            \ar@{}[lu]|{\ulcorner}
        }}
        \]
        $A'$ :: local artinian ring with $A'/\Nil(A') \iso \C$
        を用いて$Y=\Spec A'$と書ける場合,
        これはabstruct lifting of $X$ to $A'$とも呼ばれる(\cite{DefLCI} 4.2).

    \item
        上のdeformation of $X$ :: $\xi$について,
        $S$のことを$\xi$のparameter space,
        $\famX$を$\xi$のtotal spaceと呼ぶ.

    \item
        任意のscheme :: $X$と$\C$-pointed scheme :: $(S, s_0)$に対して,
        $S$がparameter spaceであるようなdeformation of $X$が存在する:
        \[\xymatrix{
                X \ar[r]\ar[d]& X \times_{\C} S \ar[d]\\
                \C \ar[r]_-{s}& S
        }\]
        これをproduct familyまたはtrivial familyと呼ぶ.

    \item
        morphism of $\C$-pointed schemes :: $(T, t_0) \to (S, s_0)$は,
        parameter spaceが$S$であるdeformation :: $\xi$から
        base changeによって次のdeformationを誘導する.
        \[\xymatrix{
            X \ar[r]\ar[d]& \famX \times_S T \ar[d]\\
            \C \ar[r]& T
        }\]
        これを元のdeformationの$f: (T, t_0) \to (S, s_0)$による
        pullbackと呼び,$f^* \xi$と書く(このノート独自?).

    \item
        isomorphism of deformations of $X$ :: $\xi \to \eta$とは,
        以下の可換図式が成立する同型$\famX \iso \famY, S \iso T$のこと.
        \[\xymatrix@R=8pt{
                {} & {} & \famY \ar[dd]\\
            X \ar[r]\ar[dd]\ar[rru]& \famX \ar[dd]\ar[ru]_-{\iso}& {} \\
            {} & {} & T \\
            \C \ar[r]\ar[rru]& S \ar[ru]_-{\iso}& {}
        }\]
        isomorphism of parameter spaces :: $(S, s_0) \to (T, t_0)$と
        deformationから誘導されるdeformationは
        元のdeformationと同型である.
    \end{enumerate}
    \end{Def}

    \begin{Def}[Universal Deformation, \cite{HaMo} 3.B, \cite{HarDef} \S15]
        universal deformation for $X$とは,
        次の性質を満たすdeformation of $X$ :: $\xi$ (parameter space :: $S$):
        任意のdeformation of $X$ ::$\eta$ (parameter space :: $T$)にたいし,
        morphism of pointed schemes :: $f: T \to S$が一意に存在し,
        $f^* \xi \iso \eta$となる.
    \end{Def}
    Universal Deformationは,
    次の関手の表現対象であると言える.
    \[ \Sch/\C \ni S \mapsto \{ \text{Deformation of $X$} \}. \]
    したがって全てのDeformationはuniversal deformationから得られる.
    しかし,当然ながらというべきか,
    universal deformationは殆どの場合で存在しない.
    そこでuniversal deformationへの要求を
    \begin{itemize}
        \item $S' \to S$をlocally about $S'$にとるものとし,
        \item $U \to S$の一意性は要求しない
    \end{itemize}
    と弱める.
    一意(uni-)ではないので,これをversal deformationと呼ぶ.

    \begin{Def}[Versal Deformation]
        (
        versal deformationの定式化が見つからないので保留.
        見つけた限りではversal deformation for schemeは
        次で意義するformal derormationでのみ定義されている.
        \cite{GACII}ではversal deformation for (complex) manifoldが定義されているのみである.
        )
    \end{Def}

    \begin{Def}[First Order Deformation]
        $D=\C[x]/(x^2), \epsilon=x \bmod (x^2)$とする.
        $\dualnum=\Spec D$の唯一の閉点を$0$で表す.
        $(\dualnum, 0)$上のdeformationを,
        first order deformation (or infinitesimal deformation)と呼ぶ.
    \end{Def}

    \begin{Lemma}
        次のfirst order deformation of $X$を考える.
        \[\vcenter{\xymatrix{
            X \ar[r]^-{\psi} \ar[d]& \famX \ar[d] \\
            \C \ar[r]_-{s}& Y
            \ar@{}[lu]|{\ulcorner}
        }}\]
        この時,$\psi$は同相写像である.
    \end{Lemma}
    \begin{proof}
        (TODO)
    \end{proof}

    \begin{Def}[Restriction of First Order Deformation]
        上の命題にあるfirst order deformation of $X$について,
        $U$ :: open subset of $X$をとる.
        locally ringed space :: $(\psi(U), \shO_{\famX}|_{\psi(U)})$を
        $\famX|_U$と書く.
    \end{Def}

\section{First Order Deformation of a Nonsingular Variety.}
    \begin{Lemma}[\cite{Eisen} Cor6.2]
        $D$-module :: $M$がflatであることは,
        $M/\epsilon M \xrightarrow{\times \epsilon} \epsilon M$
        が同型であることと同値.
    \end{Lemma}

    \begin{Lemma}[\cite{GlimpseDefTh} Prop5.1]
        $X$ :: affine, nonsingular, finite type scheme over a field $k$とする.
        この時,$X$のfirst order deformationは
        自明なdeformation :: $X \times_k \Spec D$しか存在しない.
    \end{Lemma}
    \begin{proof}
        $\phi: \famX \to \dualnum, \psi: X \to \famX$を
        $X$のfirst order deformationとする.
        $\phi$ :: flatと上の補題を用いると,
        $X \isomap \famX \times_{D} \Spec \C$から
        $\psi^{\#}: \shO_{\famX}/\epsilon \shO_{\famX} \to \shO_X$が
        同型であることが得られる.
        逆にこの同型が存在する時$\famX \to \dualnum$がflatであることが言える.
        したがって$X$のfirst order deformationは
        infinitesimal extension of $X$ by $\shO_X$ (\cite{HarAG} Ex II.8.7)に対応する.
        しかし\cite{HarAG} Ex II.8.7より,これは自明なものしか存在しない.
    \end{proof}

    \begin{Lemma}[\cite{GlimpseDefTh} Prop5.2]
        $X$ :: nonsingular scheme of finite type over $k$とする.
        この時,次のsheafを考える.
        \[ X \supseteq U \mapsto \{ \text{$\dualnum$-automorphisms of $D \times_k \dualnum$} \}. \]
        するとこのsheafはtangent sheaf of $X$ :: $\shT_X$と同型である.
    \end{Lemma}
    \begin{proof}
    \end{proof}

    \begin{Thm}[\cite{GlimpseDefTh} p.7]
        $X$ :: separated nonsingular scheme of finite type over $k$とする.
        特に,$X$ :: nonsingular (abstruct) variety over $K$であればよい.
        この時,first oder deformation of $X$の同値類は
        $H^1(X, \shT_X)$の元と一対一に対応する.
    \end{Thm}
    \begin{proof}
    \end{proof}

\section{Extension of Sheaves}
    \begin{Def}[Extension of Sheaves]
        \begin{enumerate}[label=(\roman*), leftmargin=*]
        \item
            $\shF, \shG$ :: $\shO_X$-module on ringed space $X$とする.
            extension of $\shF$ by $\shG$とは,次のような完全列のこと.
            \[
            (\shE, \iota, \kappa):
                \xymatrix{
                    0 \ar[r]& \shG \ar[r]^-{\iota}& \shE \ar[r]^-{\kappa}& \shF \ar[r]& 0
            }\]

        \item
            $(\shE, \iota, \kappa) \to (\shE', \iota', \kappa')$
            :: homomorphism of extensions of $\shF$ by $\shG$とは,
            次の図式を可換にする
            homomorphism of sheaves :: $\phi: \shE \to \shE'$のこと.
            \[\xymatrix@R=10pt{
                    {}& {}& \shE \ar[dd]^-{\phi} \ar[rd]^-{\kappa}& {}& {} \\
                    0 \ar[r]& \shG \ar[ru]^-{\iota}\ar[rd]_-{\iota'}& {} & \shF \ar[r]& 0 \\
                    {}& {}& \shE' \ar[ru]_-{\kappa'}& {}& {}
            }\]
        
        \item
            $g: \shG \to \shG'$と
            $(\shE, \iota, \kappa)$ :: extension of $\shF$ by $\shG$について,
            $g_* \shE$ :: pushfoward of $\shF$ by $\shG$を次で定める.
            これがextension of $\shF$ by $\shG'$になっていることは簡単に確かめられる.

            まず,sheafとしては$g_* \shE$は次の準同型のcokernelである.
            \[
                \shG \to \shG' \oplus \shE;
                \qquad
                \sect{U}{y} \mapsto \sect{U}{g_U(y)} \oplus \sect{U}{-\iota_U(y)}.
            \]
            $\iota_{g_*\shE}, \kappa_{g_*\shE}$は次で定める.
            \begin{align*}
                \iota_{g_*\shE}:& \shG' \to g_*\shE; && y' \mapsto [y', 0] \\
                \kappa_{g_*\shE}:& g_*\shE \to \shF; && [y', e] \mapsto \kappa_{\shE}(e)
            \end{align*}
    \end{enumerate}
    \end{Def}

    extension of $\shF$ by $\shG$が成す集合を$E(\shF, \shG)$と書く.
    \begin{Thm}\label{thm:ExtAndExt1}
        $\shF, \shG$
        :: $\shO_X$-modules on ringed scheme :: $X$とする.
        この時,
        全単射$\Phi: E(\shF, \shG) \to \Ext^1_{\shO_X}(\shE, \shG)$が存在する.
    \end{Thm}
    これは有名な結果なので証明を書かない.
    例えば\cite{HarAG} ExIII.6.1に証明の方針が述べられているし,
    加群の場合の類似の結果としては\cite{Shiho} pp.259-264に詳しい証明がある.

    \begin{Def}
        \begin{enumerate}[label=(\arabic*), leftmargin=*]
        \item
            split extesion of $\shF$ by $\shG$を$0_{\shF, \shG}$あるいは単に$0$と書く.
            \[
            0_{\shF, \shG}:
                \xymatrix{
                    0 \ar[r]& \shG \ar[r]& \shF \oplus \shG \ar[r]& \shF \ar[r]& 0
            }\]
            
        \item
            $(\shE, \iota, \kappa)$ :: extension of sheavesについて,
            $-\shE:=(\shE, -\iota, \kappa)$と定める.

        \item
            (The Baer sum)
            $(\shE, \iota, \kappa), (\shE', \iota', \kappa')$
            :: extensions of $\shF$ by $\shG$に対し,
            $\shE+\shE'$を以下のように定める.
            まずsheafとしては,次のよう.
            \[
                \shE+\shE'
                =\frac
                {\{(e, e') \in \shE \oplus \shE \mid \kappa(e)=\kappa'(e') \in \shG\}}
                {\{ (\iota(f), 0)-(0, \iota'(f)) \mid f \in \shF \}}.
            \]
            $\iota_{\shE+\shE'}, \kappa_{\shE+\shE'}$は次のように定める.
            \begin{align*}
                \iota_{\shE+\shE'}:& \shG \to \shE+\shE'; && y \mapsto [\iota(y), 0]=[0, \iota'(y)] \\
                \kappa_{\shE+\shE'}:& \shE+\shE' \to \shF; && [e, e'] \mapsto \kappa(e)=\kappa'(e')
            \end{align*}
    \end{enumerate}
    \end{Def}
    
    \begin{Lemma}
        $\shF, \shG$ :: $\shO_X$-modules on ringed scheme :: $X$とする.
        この時,$E(\shF, \shG)$には加法群の構造が定まる.
    \end{Lemma}

    \begin{Lemma}\label{lemma:PshiIsIso}
        全単射$\Phi: E(\shF, \shG) \to \Ext^1_{\shO_X}(\shE, \shG)$は
        加法群の間の同型である.
    \end{Lemma}

\section{First Order Deformation of a Local Complete Intersection.}
    この節は\ref{thm:SpaceOf1stOrdDefOfLCI}を
    証明するための必要最低限の定義と命題のまとめである.
    元の命題と比較して,このノートでは$X$を$\C$上のものに限定し,
    defoemationも一般のlocal artinian ringではなく$D=\C[\epsilon]$に限定している.
    したがって
    formal deformation(\cite{DefLCI} 6.1),
    abstruct lifting(\cite{DefLCI} 4.2), 
    first order deformationが一致している.

    以下,この節では$X$を以下のようなものとする(\cite{DefLCI} Hypotheses4.1).
    \begin{screen}
    \begin{itemize}
        \item flat,
        \item generically smooth,
        \item local complete intersection,
        \item finite type
    \end{itemize}
        scheme over $\C$.
    \end{screen}
    特にstable curve over $\C$はこれらの条件を満たす.

    ``local complete intersection"の定義を改めて書き下しておく.
    \begin{Def}[(local) complete intersection \cite{DefLCI} p.21, \cite{HarAG} p.185]
        $X$ :: scheme of finite type over $\C$が
        complete intersectionであるとは次が成立すること:
        $X$は$P$ :: smooth scheme over $\C$ (\cite{HarAG} III.10)に埋め込まれ,
        さらにideal sheaf :: $\ker(X \inclmap P)$が
        $\codim(P, X)$個のglobal sectionで生成されること(\cite{HarAG} p.121).

        $X$ :: scheme of finite type over $\C$が
        locally complete intersectionであるとは
        $\cvU$ :: open covering of $X$が存在し,
        任意の$U \in \cvU$がcomplete intersectionであること.
    \end{Def}
    
    議論は,
    $\nu(\famX_1, \famX_2)$,$\shE(\famX_1, \famX_2)$,$e(\famX_1, \famX_2)$の対応の連鎖である.
    これらはいずれもfirst order deformations :: $\famX_1, \famX_2$の
    「差」を表現する量である.
    ここでの「差」の意味を理解するには,
    最初に命題(\ref{prop:key1}), (\ref{prop:key2})のステートメントを見るのが良い.
    そして自明なfirst order deformation of $X$ :: $X \times D$と
    与えられたfirst order deformationの「差」である$e(\famX, X \times D)$によって
    first order deformation of $X$を分類する(定義\ref{def:KSclass-map}).

    \subsection{\tp{$\nu(\famX_1, \famX_2)$}{v(X1,X2)}}
    \begin{Def}
        $X$ :: \textbf{complete intersection} over $\C$ embedded in $P$とする.
        これのideal sheafを$\shI \subseteq \shO_P$とする.
        $\famX_1, \famX_2$ :: first order deformation of $X$とすると,
        $\famX_1, \famX_1$ :: embedded in $P$となる.
        そこでideal sheafをそれぞれ$\shI_1, \shI_2 \subseteq \shO_P$とする.
        first order deformation of $X$の定義から,
        $\shI_i/\epsilon \shI_i \iso \shI \ (i=1,2)$.

        写像$\shI \to (\epsilon D) \otimes_{\C} \shO_X$を
        以下のように定める.
        まず,$\sect{U, f} \in \shI$に対し,
        \[ \sect{U}{\tilde{f}_i} \bmod \epsilon \shI_i=\sect{U}{f} \ (i=1,2) \]
        となる$\sect{U}{\tilde{f}_i}$が存在する.
        そこで
        \[
            \sect{U}{f}
            \mapsto
            \sect{U}{\tilde{f}_1-\tilde{f}_2} \bmod (\epsilon D) \otimes_{\C} \shI
            \in (\epsilon D) \otimes_{\C} (\shO_P/\shI)=(\epsilon D) \otimes_{\C} \shO_X
        \]
        と写す.
        これは$\tilde{f}_i$のとり方に依らず,well-defined.
        この写像を$\nu(\famX_1, \famX_1)$と書く.

        以下の$\C$-moduleとしての同型があるため,
        $\nu(\famX_1, \famX_2)$を以下のいずれの集合の元ともみなす.
        \begin{align*}
            {}&   \Hom_{\shO_P}(\shI, (\epsilon D) \otimes_{\C} \shO_X) \\
            \iso& \Hom_{\shO_P}(\shI/\shI^2, (\epsilon D) \otimes_{\C} \shO_X) \\
            \iso& H^0(X, (\epsilon D) \otimes_{\C} (\shI/\shI^2)\sidehat) \\
            \iso& (\epsilon D) \otimes_{\C} H^0(X, (\shI/\shI^2)\sidehat) \\
            \iso& H^0(X, (\shI/\shI^2)\sidehat)
        \end{align*}
    \end{Def}

    \begin{Prop}[\cite{DefLCI} Prop2.8a,b,c,d,e]
        $X$ :: \textbf{complete intersection} embedded in $P$とする.
        \begin{enumerate}[label=(\alph*), leftmargin=*]
            \item $\nu(\famX_1, \famX_2)=0 \iff \famX_1=\famX_2$.

            \item $\nu(\famX_1, \famX_2)+\nu(\famX_2, \famX_3)=\nu(\famX_1, \famX_3)$.

            \item $\nu(\famX_2, \famX_1)=-\nu(\famX_1, \famX_2)$.

            \item
                任意の$\famX$ :: first order deformation of $X$, 
                任意の$\nu \in H^0(X, (\shI/\shI^2)\sidehat)$に対し,
                $\famY$ :: first order deformation of $X$が存在して,
                $\nu=\nu(\famX, \famY)$となる.

            \item
                $U$ :: open subset of $X$について,
                $\nu(\famX_1|_U, \famX_2|_U)=\nu(\famX_1, \famX_2)|_U:
                (\shI/\shI^2)|_U \to (\epsilon D) \otimes_{\C} \shO_U$.
        \end{enumerate}
    \end{Prop}
    \begin{proof}
        (d)のみ証明する.他は自明であろう.
    \end{proof}

    \subsection{\tp{$\shE(\famX_1, \famX_2)$}{E(X1,X2)}}
    \begin{Lemma}[First Fundamental Exact Sequence]
        $X$ :: \textbf{complete intersection} over $\C$ embedded in $P$とし,
        ideal sheafは$\shI \subseteq \shO_P$であるとする.
        この時,以下はexact.
        \[\xymatrix{
            0 \ar[r]& \shI/\shI^2 \ar[r]& (\shDer_{P/\C})|_{X} \ar[r]& \shDer_{X/\C} \ar[r]& 0
        }\]
        すなわち,
        $(\shDer_{P/\C})|_{X}$ :: extension of $\shDer_{X/\C}$ by $\shI/\shI^2$.
    \end{Lemma}
    \begin{proof}
        (TODO)
        よく知られている通り,
        最初の射が単射であることを示しさえすれば良い.
    \end{proof}

    \begin{Def}
        $U \subseteq X$ :: \textbf{complete intersection} over $\C$ embedded in $P$とする.
        $\famX_1, \famX_2$ :: first order deformation of $X$について,
        \[ \shE(\famX_1|_U, \famX_2|_U):=\nu(\famX_1|_U, \famX_2|_U)_*((\shDer_{P/\C})|_{X}) \]
        をextension of $\shDer_{U/\C}$ by $(\epsilon) \otimes_{\C} \shO_{U}$として定める.
    \end{Def}
    (TODO: 張り合わせが出来ることに言及.)
    貼りあわせによって$\shE(\famX_1, \famX_2)$を定める.

    \begin{Prop}[\cite{DefLCI} Prop4.9a,b,c,f]\label{prop:key1}
        $\famX_1, \famX_2$ :: first order deformations of $X$とする.
        \begin{enumerate}[label=(\alph*), leftmargin=*]
        \item
            $\shE(\famX, \famX) \iso 0_{\shDer_{X/ \C}, (\epsilon D) \otimes \shO_X}$.
        \item
            $\shE(\famX_2, \famX_1)=-\shE(\famX_1, \famX_2)$.
        \item 
            $\shE(\famX_1, \famX_2)+\shE(\famX_2, \famX_3) \iso \shE(\famX_1, \famX_3)$.

        \item
            任意の$\shE$ :: extension of $\shDer_{X/\C}$ by $(\epsilon) \otimes_{\C} \shO_{X}$と
            任意の$\famX$ :: first order deformation of $X$に対し,
            \[ \shE \iso \shE(\famX, \famY) \]
            なる$\famY$ :: first order deformation of $X$が存在する.
        \end{enumerate}
    \end{Prop}
    \begin{proof}
        (TODO)
    \end{proof}

    \begin{Prop}[\cite{DefLCI} Prop4.10]\label{prop:key2}
        $\famX_1, \famX_2$ :: first order deformations of $X$とする.
        この時,以下は同値.
        \begin{enumerate}
            \item $\famX_1$と$\famX_2$は同型.
            \item $\shE(\famX_1, \famX_2)$ :: split extension.
        \end{enumerate}
    \end{Prop}
    \cite{DefLCI} pp.22-23では更に詳しく,
    同型$\famX_1 \isomap \famX_2$が
    $\Hom_{\shO_X}(\shDer_{X/\C}, \shO_X)$の元に一対一対応することが述べられている.
    この集合は$\Ext^1_{\shO_X}(\shDer_{X/\C}, \shO_X)$の中で$0$に潰れていることに注意.
    \begin{proof}
        (TODO)
    \end{proof}

    \subsection{\tp{$e(\famX_1, \famX_2)$}{e(X1,X2)}}
    \begin{Def}
        $T^i(X)=\Ext^i_{\shO_X}(\shDer_{X/\C}, \shO_X)$とおく.
        $\famX$ :: first order deformation of $X$に対し,
        \[
            e(\famX_1, \famX_2)
                \in
                \Ext^1(\shDer_{X/\C}, (\epsilon D) \otimes_{\C} \shO_X)
                \iso (\epsilon D) \otimes T^1(X)
                \iso T^1(X)
                \iso \Hom((\epsilon D)^*, T^1(X))
        \]
        を,$\shE(\famX, X \times D)$に対応する元とする(\ref{thm:ExtAndExt1}).
    \end{Def}
    補題(\ref{lemma:PshiIsIso})より,
    命題(\ref{prop:key1}), (\ref{prop:key2})と同様の命題が
    $e(\famX_1, \famX_2)$についても性質する.

    \begin{Def}[Kodaira-Spencer class/map, \cite{DefLCI}]\label{def:KSclass-map}
        $T^i(X)=\Ext^i_{\shO_X}(\shDer_{X/\C}, \shO_X)$とおく.
        $\famX$ :: first order deformation of $X$に対し,
        \[
            k_{\famX}=e(\famX, X \times D)
                \in T^1(X) \iso \Hom((\epsilon D)^*, T^1(X))
        \]
        とおく.
        この$k_{\famX}$をKodaira-Spencer class of $\famX$と呼ぶ.
        対応する写像$K_{\famX}: (\epsilon D)^*, T^1(X)$を
        Kodaira-Spencer map of $\famX$と呼ぶ.
    \end{Def}
    
    \begin{Thm}\label{thm:SpaceOf1stOrdDefOfLCI}
        First order deformation of $X$の同値類と
        $\Ext^1_{\shO_X}(\shDer_{X/\C}, \shO_X)$の元は一対一に対応する.
        (cf. \cite{DefLCI} Prop 6.12)
    \end{Thm}
    \begin{proof}
        命題(\ref{prop:key1}), (\ref{prop:key2})より,
        $\famX \mapsto k_{\famX}$がこの対応を与えることは明らか.
    \end{proof}

\bibliographystyle{jplain}
\bibliography{reference}
\end{document}
