\documentclass[a4paper]{jsarticle}
\usepackage[]{macros}
\usepackage{graphics}

\newcommand{\Isom}{\operatorname{Isom}}
\newcommand{\ftorIsom}{\mathcal{I}\!som}
\newcommand{\hilb}{\mathcal{H}}
\newcommand{\dualnum}{\mathbb{I}}
\newcommand{\cvU}{\mathfrak{U}}
\newcommand{\der}{\mathrm{d}}
\newcommand{\Der}{\mathrm{Der}}
\newcommand{\viso}{\rotatebox{-90}{$\iso$}}

\newcommand{\LA}{\mathrm{(LA)}}
\newcommand{\LN}{\mathrm{(LN)}}
\newcommand{\CLN}{\mathrm{(CLN)}}

\newcommand{\centerpb}{\ar@{}[lu]|{\text{p.b.}}}
\newcommand{\centerpo}{\ar@{}[lu]|{\text{p.o.}}}

\newcommand{\diag}{\Delta}
\newcommand{\codiag}{\nabla}

\newcommand{\defX}{\mathcal{X}}
\newcommand{\defY}{\mathcal{Y}}
\newcommand{\defU}{\mathcal{U}}
\newcommand{\defP}{\mathcal{P}}
\newcommand{\defM}{\mathcal{M}}

\begin{document}
\title{ゼミノート \#4 \\ Deformation Theory}
\author{七条彰紀}
\maketitle

\section{Automorphism Group of Stable Curve}
    \cite{HaMo} 3.A, \cite{IrrOfMg} \S 1を参照する.

    $C, D$ :: stable curves of genus $g$ over a scheme $S$の間の
    isomorphism groupのschemeとしての構造を与える.
    このschemeを$\Isom(C, D)$と書く.
    そして$\Aut(C)=\Isom(C,C)$と定義し,
    これのschemeとしての特徴を調べる.
    
    $\Isom(C, D)$の特徴付けをするため,次の関手を考える.
    \begin{defmap}
        \ftorIsom_S(C, D):& \text{(Scheme over $\C$)}& \to& \text{(Set)} \\
        {}& S'& \mapsto& \{ \ C \times_{\C} S' \to D \times_{\C} S' \text{ :: $S'$-isomorphism} \}
    \end{defmap}
    $\iota \in \ftorIsom(C, D)(S')$から得られる$\iota^*$は
    $\shDual_{C \times S'/S'}=\iota^*(\shDual_{D \times S'/S'})$を満たす.
    また$\otimes$と交換する
    (すなわちPicard群の間の準同型である.\cite{HarAG} Ex II.6.8).
    このことから$\Isom(C, D)$が適当な$r$をとると
    $PGL(r+1)$の部分群として書けることが分かる.

    もう少し詳しく$\Isom(C, D)$を書く.
    $n \geq 3$を整数とする.次のように$r,d$をとる.
    \[
        r+1=h^0((\shDual_{C/\C})^{\otimes n})=(2n-1)(g-1),
        \qquad
        d=\deg (\shDual_{C/\C})^{\otimes n}=2n(g-1).
    \]
    すると\cite{HarAG} II.7より,
    $C, D$は$\proj_{\C}^r$の次数$d$, arithmetic genus $g$のclosed curveとみなせる
    ($\proj^r$に埋め込める).
    なのでHildert scheme :: $\hilb=\hilb_{d,g,r}$の点として扱うことが出来る.
    ここで次のように射を定める.
    \begin{defmap}
        \mu:& PGL(r+1)& \to& \hilb \times \hilb \\
        {}& \alpha& \mapsto& (\alpha \cdot [C], [D])
    \end{defmap}
    すると,$\ftorIsom(C, D)$は$\mu^{-1}(\Delta)$によって表現される
    \footnote
    {
        $\Delta$は$\hilb \times \hilb$のdiagonal set.
        $\mu^{-1}(\Delta)$は
        \[ \Delta \cap \im \mu=\{ (\alpha \cdot [C], [D]) \mid \alpha \cdot [C]=[D] \} \]
        の$PGL(r+1)$への逆像なので,
        この点と$C, D$の間の同型と対応することが分かるだろう.
    }.
    これをgroup scheme over $\C$ :: $\Isom(C, D)$とする.

    scheme over $\C$ :: $X$について少々一般の理論を述べる.
    $\dualnum=\Spec \C[\epsilon]/(\epsilon^2)$とおく(ref. \cite{HaMo} 1).
    \cite{HarAG} Ex II.2.8より,
    $t \in \ftor{X}(\dualnum)$は$X$の$\C$-rational point :: $x$と
    $T_x(X)=\I{m}_x/\I{m}_x^2=\shT_x$の元に対応する.
    ここで$\shT$はtangent sheaf :: $\shT=\shHom_{\shO_X}(\shDer_X, \shO_X)$のことである.
    \cite{HaMo}でいうregular vector fieldとは$\shT$のsectionのこと(と思われる).

    \begin{Thm}\label{thm:TC(C)=0}
        $C$ :: stable curve of genus $g \geq 2$について,
        \[ \operatorname{Ext}^0(\shDer_C, \shO_C)=H^0(C, \shT_C)=\shT_C(C)=0. \]
    \end{Thm}
    \begin{proof}
        \cite{IrrOfMg} \S1.

        $\pi: \tilde{C} \to C$をnormalization of $C$とする.
        また$\tilde{C}$のconnected componentの
        個数を$\nu$,それぞれのgenusを$g_i \ (i=1,\dots,\nu)$とする.
        
        今,
        $D \in \shT_C(C)$は
        pullback :: $\pi^*: \shT_{\tilde{C}} \to \pi^* \shT_{C}$によって
        \footnote
        {
            $R$ :: ring, $A, B$ :: ring over $R$とする.
            一般に,$k$-homomorphism :: $\phi: A \to B$があるとき,
            $D \in \Der_{R}(B)$は$\phi^*: D \mapsto D \circ \phi$によって
            $\Der_R(A)$へ写すことが出来る.
        }.
        $\tilde{D} \in \shT_{\tilde{C}}(\tilde{C})$
        $C$のdouble pointに$\pi$で対応する点
        (point laying over double point, plodp)で$0$になるような
        regular vector field :: $\tilde{D} \in \shT_{\tilde{C}}(\tilde{C})$
        に対応する(TODO).
        このような$\tilde{D}$は$0$しかないことを確かめれば,
        $\shT_C(C)=0$がわかる.

        \begin{Claim}
            $1$点$P \in \tilde{C}$で$\tilde{D}_P=0$ならば,
            $\tilde{D}=0$である.
        \end{Claim}
        \begin{proof}
            $C$ :: reduced connected schemeに注意する.
            $P \in C$において$\tilde{D} \in \shT_C(C)$が$\tilde{D}_P=0$を満たすとしよう.
            $C$のirreducible affine open cover :: $\cvU$をとり,
            $P \in U$なる$U=\Spec A \in \cvU$をとって固定する.
            すると$C$ :: reducedより$A$ :: integral domain.
            $\tilde{D}|_U \in \shT_C(U)$が$P \in U$で$0$になるのだから,
            次が成立する.
            \[ \Exists{u \in A-\I{p}_P} u \cdot (\tilde{D}|_U)=0. \]
            $A$ :: integralより,これは$\tilde{D}|_U=0$を意味する.
            $U$と交わるirreducible affine open subset of $C$ :: $V \in \cvU$についても,
            $\tilde{D}|_{U \cap V}=0$なので$\tilde{D}|_{V}=0$.
            $C$ :: connectedなので,このように$V$を取り続けることで,
            全ての$V \in \cvU$について$\tilde{D}|_V=0$であることがわかる.
            sheafのIdentity Axiomから,$C$全体で$t=0$.
        \end{proof}

        したがって我々は
        $\tilde{C}$の各componentは少なくとも一つずつ
        plodpをもつこと示せば良い.

        $\shT_{\tilde{C}}=\shHom(\shDer_{\tilde{C}/\C}, \shO_{\tilde{C}})$なので,
        $\shT_{\tilde{C}}$に対応するdivisorは$K_{\tilde{C}}$.
        $\deg K_{\tilde{C}}=2\tilde{g}-2$なので,
        $\tilde{g}>1$ならば$\deg (-K_{\tilde{C}})<0$.
        したがって\cite{HarAG} Lemma IV.1.2から$\dim_{\C} H^0(\tilde{C}, \shT_{\tilde{C}})=0$.
        すなわち$\shT_{\tilde{C}}(\tilde{C})=0$.
        なので以下では$\tilde{g}_i=0,1$とする.

        $\tilde{g}_i=0,1$であるとき,
        $\tilde{C}$の各connected componentは必ずplodpをもつ.
        実際,genus formulaで$\delta=0$とすると
        \[ g=\sum_{i}(\tilde{g}_i-1)+1 \geq 2 \]
        したがって$\sum_{i}(\tilde{g}_i-1)>0$ということになる.
        しかし仮定から$\tilde{g}_i-1 \leq 0$なので,$\delta>0$.
        すなわち$C$は必ずnodeをもつ.
        $\tilde{C}$の各componentはsmoothであることと
        $C$がconnectedであることも踏まえて考えると,
        $\tilde{C}$の各componentは少なくとも一つずつ
        plodpをもつことが分かる.
        (この辺りは\cite{IrrOfMg} Lemma1.4で詳しく述べられている).
    \end{proof}
    別証明として\cite{HarDef} Prop27.4がある.

    \begin{Prop}
        任意の閉点$P \in \Aut(C)$について,
        $\shO_{\Aut(C), P} \iso \C$.
        特に$\Aut(C)$ :: reduced scheme.
    \end{Prop}
    \begin{proof}
        $X=\Aut(C)$はgroup scheme over $\C$であるから,
        $X$のある点でのlocalな性質は
        transitionを用いて単位元$e$での性質と言い換えられる.
        なので$A:=\shO_{X, e}$のみを考える.
        $X$ :: group scheme over $\C$より
        $e$ :: $\C$-rational pointなので,
        $A$が体ならばそれは$\C(=A/\I{m}_A)$と同型である.
        よって我々は$A$が体であることのみ示せば良い.

        上記の定理(\ref{thm:TC(C)=0})から,$\shT_C(C)=0$.
        これは$C \times \dualnum$の$\dualnum$-automorphismは
        自明なものしか無いことを意味する(後述).
        さらに$\Aut(C)$の定義から,
        これは射$\dualnum \to \Aut(C)$としては自明なものしか存在しないことを意味する.
        さらに\cite{HarAG} Ex II.2.8より,
        これは$\I{m}_A/\I{m}_A^2=0$を意味する.
        中山の補題から$\I{m}_A=0$.
        よって$A$は体である.
    \end{proof}

    \section{Definitions of Deformations and (Uni-)Versal Deformation.}
\begin{Def}[$\C$-pointed scheme \cite{DefAS} \S 1.2.1]
    scheme :: $Y$と
    $\C$-rational point :: $y_0 \in Y$の組を
    $\C$-pointed schemeを呼び,$(Y, y_0)$と書く.

    morphism of $\C$-pointed schemes :: $(S, s_0) \to (T, t_0)$とは,
    moephism of schemes :: $\phi: S \to T$であって,
    $\phi(s_0)=t_0$を満たすもののこと.
\end{Def}

\begin{Def}[Deformation of Scheme \cite{DefAS} \S 1.2.1, \cite{HaMo} \S 3.B]
    \begin{enumerate}[label=(\roman*), leftmargin=*]
    \item 
        deformation of $X$とは,以下のようなpullback diagramのことである.
        $\psi$から$X \iso \defX \times_{Y} \C$が誘導される.
        \[
            \xi:
        \vcenter{\xymatrix{
            X \ar[r] \ar[d]& \defX \ar[d]^-{\text{flat, surj.}} \\
            \C \ar[r]_-{s}& Y
            \centerpb
        }}
        \]
        $A'$ :: local artinian ring with $A'/\Nil(A') \iso \C$
        を用いて$Y=\Spec A'$と書ける場合,
        これはabstruct lifting of $X$ to $A'$とも呼ばれる(\cite{DefLCI} 4.2).

    \item
        上のdeformation of $X$ :: $\xi$について,
        $S$のことを$\xi$のparameter space,
        $\defX$を$\xi$のtotal spaceと呼ぶ.

    \item
        任意のscheme :: $X$と$\C$-pointed scheme :: $(S, s_0)$に対して,
        $S$がparameter spaceであるようなdeformation of $X$が存在する:
        \[\xymatrix{
                X \ar[r]\ar[d]& X \times_{\C} S \ar[d]\\
                \C \ar[r]_-{s}& S
        }\]
        これをproduct familyまたはtrivial familyと呼ぶ.

    \item
        morphism of $\C$-pointed schemes :: $(T, t_0) \to (S, s_0)$は,
        parameter spaceが$S$であるdeformation :: $\xi$から
        base changeによって次のdeformationを誘導する.
        \[\xymatrix{
            X \ar[r]\ar[d]& \defX \times_S T \ar[d]\\
            \C \ar[r]& T
        }\]
        これを元のdeformationの$f: (T, t_0) \to (S, s_0)$による
        pullbackと呼び,$f^* \xi$と書く(このノート独自?).

    \item
        isomorphism of deformations of $X$ :: $\xi \to \eta$とは,
        以下の可換図式が成立する同型$\defX \iso \defY, S \iso T$のこと.
        \[\xymatrix@R=8pt{
                {} & {} & \defY \ar[dd]\\
            X \ar[r]\ar[dd]\ar[rru]& \defX \ar[dd]\ar[ru]_-{\iso}& {} \\
            {} & {} & T \\
            \C \ar[r]\ar[rru]& S \ar[ru]_-{\iso}& {}
        }\]
        isomorphism of parameter spaces :: $(S, s_0) \to (T, t_0)$と
        deformationから誘導されるdeformationは
        元のdeformationと同型である.
    \end{enumerate}
    \end{Def}

    \begin{Def}[Universal Deformation, \cite{HaMo} 3.B, \cite{HarDef} \S15]
        universal deformation for $X$とは,
        次の性質を満たすdeformation of $X$ :: $\xi$ (parameter space :: $S$):
        任意のdeformation of $X$ ::$\eta$ (parameter space :: $T$)にたいし,
        morphism of pointed schemes :: $f: T \to S$が一意に存在し,
        $f^* \xi \iso \eta$となる.
    \end{Def}
    Universal Deformationは,
    次の関手の表現対象であると言える.
    \[ \Sch/\C \ni S \mapsto \{ \text{Deformation of $X$} \}. \]
    したがって全てのDeformationはuniversal deformationから得られる.
    しかし,当然ながらというべきか,
    universal deformationは殆どの場合で存在しない.
    そこでuniversal deformationへの要求を
    \begin{itemize}
        \item $S' \to S$をlocally about $S'$にとるものとし,
        \item $U \to S$の一意性は要求しない
    \end{itemize}
    と弱める.
    一意(uni-)ではないので,これをversal deformationと呼ぶ.

    \begin{Def}[Versal Deformation]
        (
        versal deformationの定式化が見つからないので保留.
        見つけた限りではversal deformation for schemeは
        次で意義するformal derormationでのみ定義されている.
        \cite{GACII}ではversal deformation for (complex) manifoldが定義されているのみである.
        )
    \end{Def}

    \begin{Def}[First Order Deformation]
        $D=\C[x]/(x^2), \epsilon=x \bmod (x^2)$とする.
        $\dualnum=\Spec D$の唯一の閉点を$0$で表す.
        $(\dualnum, 0)$上のdeformationを,
        first order deformation (or infinitesimal deformation)と呼ぶ.
    \end{Def}

    \begin{Remark}
        $X$ :: stable curve of genus $g$とする.
        first order deformation :: $\defX \to \dualnum$は,
        moduli spaceの定義から,
        $0 \in \dualnum$を$[X]=[\defX_0] \in \barM_g$へ写す$\dualnum \to \barM_g$に対応する.
        そしてこの射は,
        既に知られている通りZariski tangent space at $[X]$ :: $T_{[X]}$の元に対応する.
        よって$X$のfirst order deformationから$T_{[X]}$の元への対応がある.
        この対応は一対一であろうか?
    \end{Remark}

    \begin{Lemma}\label{lemma:psi_is_imm_homeo}
        次のfirst order deformation of $X$を考える.
        \[\vcenter{\xymatrix{
            X \ar[r]^-{\psi} \ar[d]& \defX \ar[d] \\
            \C \ar[r]_-{0}& \dualnum
            \centerpb
        }}\]
        この時,$\psi$はclosed imm.かつ同相写像である.
    \end{Lemma}
    \begin{proof}
        まず$\C \to \dualnum$がclosed imm.であり,
        closed imm.がstable under base extensionであることから,
        $\psi$もclosed imm.
        またclosed imm.ならばfiniteである(\cite{HarAG} Ex II.5.5b).
        $\psi$がhomeomorphismであることはlocal on codomain(target)なもの
        \footnote{ local on codomainに考えれば,特に全単射性が示せる. }なので,
        $\defX$ :: affineと仮定して証明すれば十分.
        
        以上から,
        $\defX=\Spec A, X=\Spec R$,
        $R$ :: $A$-algebra, finitely generated as module
        と仮定して良い.
        また,$\psi$ :: closed imm.であるから,
        $A \iso R/I$なる$I$ :: ideal of $R$が存在する.
        また仮定から$A \otimes_D \C \iso R$かつ$A$ :: flat $D$-module.
        そこで以下の$D$-module 完全列に$\otimes_D R$とする.
        \[\xymatrix{
            0 \ar[r]& \epsilon D \ar[r]& D \ar[r]& \C \ar[r]& 0
        }\]
        すると次のように成る.
        \[\xymatrix@R=10pt{
            0 \ar[r]& (\epsilon D)\otimes_D R \ar[r]& D \otimes_D R \ar[r]& \C \otimes_D R \ar[r]& 0 \\
            0 \ar[r]& I \ar@<5pt>@{}[u]_-{\viso} \ar[r]& R \ar@<5pt>@{}[u]_-{\viso}\ar[r]& A \ar@<5pt>@{}[u]_-{\viso}\ar[r]& 0 \\
        }\]
        よって$(\epsilon D)\otimes_D R$同様$I$はnilpotent ideal (i.e.$I^2=0$).

        $X=\Spec R$の閉集合として
        \[ \defX=\Spec A=V(I)=V(I^2)=V((0))=X \]
        なので$\im \psi=\defX$.
        $\psi$ :: closed imm.なので,これで$\psi$ :: homeoが証明できた.
    \end{proof}

    \begin{Def}[Restriction of First Order Deformation]
        上の命題にあるfirst order deformation of $X$について,
        $U$ :: open subset of $X$をとる.
        locally ringed space :: $(\psi(U), \shO_{\defX}|_{\psi(U)})$を
        $\defX|_U$と書く.
    \end{Def}

\section{First Order Deformation of a Nonsingular Variety.}
    \begin{Lemma}\label{lemma:def_of_affine_is_affine}
        $A$ :: ring, $X=\Spec A$とする.
        この時,first order deformation of $X$ :: $\defX$も
        affine schemeである.
    \end{Lemma}
    \begin{proof}
        \cite{HarAG} Ex II.3.1への回答でもある.
        
        $\shI=\ker(X \inclmap \defX)$とし,
        $\shF$ :: quasi-coherent sheaf on $X$とする.
        今,補題(\ref{lemma:psi_is_imm_homeo})の証明から$\shI^2=0$が得られる.
        また,明らかに$\shO_{\defX}/\shI \iso \shO_{X}$.
        したがって次のSES(short exact sequence)が存在する.
        \[\xymatrix{
            0 \ar[r]& \shI^{d+1} \shF \ar[r]& \shI^{d} \shF \ar[r]&
                \shI^{d}\shF \otimes_{\shO_{\defX}} \shO_X \ar[r]& 0
        }\]
        次のLESが誘導される.
        \[\xymatrix{
            0 \ar[r]& H^0(\defX, \shI^{d+1} \shF) \ar[r]& H^0(\defX, \shI^{d} \shF) \ar[r]& H^0(\defX, \shI^{d}\shF \otimes \shO_X) \\
            {}\ar[r]& H^1(\defX, \shI^{d+1} \shF) \ar[r]& H^1(\defX, \shI^{d} \shF) \ar[r]& H^1(\defX, \shI^{d}\shF \otimes \shO_X) \\
            \dots
        }\]
        
        sheaf cohomologyはabelian groupのcohomologyとして構成されており,
        module structureとは無関係に定まっている.
        そして$X$と$\defX$はhomeo.
        したがって
        \[ H^i(\defX, \shI^{d}\shF \otimes \shO_X)=H^i(X, \shI^{d}\shF \otimes \shO_X)=0. \]
        最後の等号は$\shI^{d}\shF \otimes \shO_X$ :: quasi-coherent $\shO_X$-moduleと
        \cite{HarAG} Thm III.3.7から得られる.
        
        $d=1 (\implies \shI^{d+1}=0)$からはじめて$d$についての帰納法により
        \[ H^i(\defX, \shI^{d+1} \shF)=H^i(\defX, \shI^{d}\shF \otimes \shO_X)=0 \ (i>0). \]
        よってLESから$H^i(\defX, \shI^{d} \shF)=0 (i>0)$.
        \cite{HarAG} Thm III.3.7から$\defX$ :: affine.
    \end{proof}

    \begin{Lemma}[\cite{DefAS} Thm1.2.4]\label{thm:aff_is_rigid}
        $X$ :: affine, nonsingular, finite type scheme over a field $k$とする.
        この時,$X$のfirst order deformationは
        自明なdeformation :: $X \times_k \Spec D$しか存在しない.
    \end{Lemma}
    \begin{proof}
        上の補題から,deformation of $X=\Spec A$はaffine.
        そこで$\defX=\Spec B$とする.
        \[\vcenter{\xymatrix{
            B \ar@{->>}[r]& A \\
            D \ar@{->>}[r] \ar[u]^-{f}& k \ar[u]
            \centerpb
        }}\]
        $f$ :: flatと,$f$の唯一のfiber :: $X=\Spec A$がsmoothであることから,
        \cite{HarAG} Thm III.10.2より$f$ :: smooth.

        次のcommutative diagramを考える.
        \[\vcenter{\xymatrix{
            B \ar@{->>}[r]& A \\
            D \ar@{->}[r] \ar[u]^-{f}& A \otimes_{k} D \ar[u]_-{\bmod \epsilon A}
            \ar@{}[lu]|{\circlearrowleft}
        }}\]
        $f$ :: ($\epsilon D$-)smooth over $D$なので,
        図式を可換にする射$\phi: B \to A \otimes D$が存在する.
        以下の主張から$\phi$ :: isoなので,
        任意のdeformation of $X$ :: $\Spec B$は
        自明なdeformation :: $\Spec A \otimes D=X \times \dualnum$と同型である.
    \end{proof}

    \begin{Claim}[\cite{DefAS} Lemma A.4]
        $R$ :: ring,
        $I$ :: ideal of $R$,
        $F, G$ :: $R$-module, $G$ :: flat,
        $f: F \to G$ :: homomorphism of $R$-modules.

        $I$ :: nilponentとし,
        誘導されるhomomorphism
        $f \otimes_{R} \id[R/I]: F/IF \to G/IF$が同型であるとする.
        この時,$f$ :: iso.
    \end{Claim}
    \begin{proof}
        $C=\coker f$とする.
        完全列$F \to G \to C \to 0$に$\otimes_{R} (R/I)$を作用させる.
        \[\xymatrix{
            F/IF \ar[r]& G/IG \ar[r]& C/IC \ar[r]& 0
        }\]
        仮定から$C/IC=0$.
        今$I$ :: nilponentなので$I \subset \Jac(R)$.
        したがって中山の補題から$C=0$.
        すなわち$f$ :: surj.

        $K=\ker f$とする.
        完全列$0 \to K \to F \to G \to 0$に$\otimes_{R} (R/I)$を作用させる.
        \[\xymatrix{
                0 \ar[r]& K/IK \ar[r]& F/IF \ar[r]& G/IG \ar[r]& 0
        }\]
        今,$G$ :: flatから$\Tor_{R}(G, R/I)=0$.
        なのでこのSESから誘導される$\Tor_{R}(-, R/I)$のLESを考えると,
        $K \otimes(R/I) \iso K/IK=0$が得られる.
        再び中山の補題から$K=0$.
        よって$f$ :: inj.
    \end{proof}

    \begin{Lemma}[\cite{DefAS} Lemma1.2.6] \label{lemma:inf_auto_and_Der}
        任意の$k$-algebra :: $A$について,次の群同型がある.
        \[
            \left\{\parbox{4.4cm}{
                $D$-automorphism of $A \otimes_k D$ \\
                inducing identity on $A$
            }\right\}
            \iso
            \Der_{k}(A).
        \]
        $D$-automorphism of $A \otimes_k D$のことを
        infinitesimal automorphismと呼ぶ.
    \end{Lemma}
    \begin{proof}
        仮定からautomorphism of $A \otimes_k D=A[\epsilon]$は
        $D$-module homomorphismで,
        $\bmod \epsilon A \otimes (\epsilon D)$を合成するとidentityになる.
        したがって次のように書ける.
        \[ \theta(x)=x+\epsilon D(x). \]
        $\theta$が積を保つことと$D$-module homo.であることから,
        $D: A \otimes D \to A$ :: $D$-derivation.

        $\modDer_{A \otimes_{k} D/D} \iso \modDer_{A/k} \otimes_{k} D$に注意すると,
        $\theta$と$\Der_{k}(A)$の対応が分かる.
        この対応が群準同型であることは明らか.
    \end{proof}

    \begin{Thm}[\cite{DefAS} Prop1.2.9]
        $X$ :: separated nonsingular scheme of finite type over $k$とする.
        特に,$X$ :: nonsingular (abstruct) variety over $K$であればよい.
        この時,first oder deformation of $X$の同値類は
        $H^1(X, \shT_X)$の元と一対一に対応する.
    \end{Thm}
    \begin{proof}
        $\defX$ :: first oder deformation of $X$を任意の取る.
        そしてaffine open cover of $X$:: $\{U_i\}_{i \in I}$を任意に取る.
%        ただし添字集合$I$は有限である.
        このcoverについての\v{C}ech cohomologyを考えていく.

        今,$\defX|_{U_i}$ :: first order deformation of $U_i$.
        定理(\ref{thm:aff_is_rigid})より,
        $\theta_{i}: U_i \times_{k} \dualnum \to \defX|_{U_i}$が得られる.
        これを用いて,各$i,j \in I$について
        \[ \theta_{ij}=\theta_{i}^{-1} \circ \theta_{j}:U_{ij} \times \dualnum \to U_{ij} \times \dualnum \]
        が得られる.
        ただし$U_{ij}=U_i \cap U_j$(以降の$U_{ijk}$なども同様).
        補題(\ref{lemma:inf_auto_and_Der})から,
        これは$d_{ij} \in \Gamma(U_{ij}, \shT_{X})$に対応する.
        
        $\theta_{ij}$は貼り合わせることが出来るのだから,
        Gluing Lemmaを参照すれば
        $\theta_{ij} \theta_{jk} \theta_{ik}^{-1}=\id[U_{ijk} \times \dualnum]$が得られる.
        補題(\ref{lemma:inf_auto_and_Der})の準同型で写せば,
        \[ d_{ij}+d_{jk}-d_{ik}=0 \]
        すなわち\v{C}eck $1$-cocycle conditionが得られる.

        first order deformation of $X$が$2$つあり,
        その間に同型があるとしよう: $\Psi: \defX \to \defX'$.
        $\defX'$について$\theta'_{ij}, d'_{ij}$を$\defX$同様に定める.
        次のinfiniterimal automorphismを考える.
        \[
            \alpha_{i}=\theta'_{i} \circ \Psi|_{U_i} \circ \theta_{i}:
            U_i \times \dualnum \to \defX|_{U_i} \to \defX'|_{U_i} \to U_i \times \dualnum.
        \]
        $\alpha_i$に$a_i \in \Gamma(U_i, \shT_X)$が対応しているとする.
        計算すると
        $(\alpha_{i}|_{U_{ij}})^{-1} \theta'_{ij} (\alpha_{j}|_{U_{ij}})=\theta_{ij}$が得られる.
        すなわち,
        \[ d'_{ij}-d_{ij}=a_{i}-a_{j}. \]
        よって$\{d_{ij}\}$の同値類と$\{d'_{ij}\}$の同値類は
        $\check{H}^1(X, \shT_X)$の中で等しい.

        以上より,$\defX$から$\check{H}^1(X, \shT_X)$の元への対応は単射的である.
        逆に$\{d_{ij}\}$から$\{\theta_{ij}\}$の対応,
        $\theta_{ij}$による$U_{i} \times \dualnum$の貼り合わせへと手順を遡れば,
        $\check{H}^1(X, \shT_X)$の元とfirst order deformation of $X$への対応が全射だと分かる.

        最後に,\cite{HarAG} Thm III.4.5から
        $\check{H}^1(X, \shT_X) \iso H^1(X, \shT_X)$.
    \end{proof}

\section{Extension of Sheaves}
    \subsection{Definitions}
    \begin{Def}[Extension of Sheaves]
        \begin{enumerate}[label=(\roman*), leftmargin=*]
        \item
            $\shF, \shG$ :: $\shO_X$-module on ringed space $X$とする.
            extension of $\shF$ by $\shG$とは,次のような完全列のこと.
            \[
            (\shE, \iota, \kappa):
                \xymatrix{
                    0 \ar[r]& \shG \ar[r]^-{\iota}& \shE \ar[r]^-{\kappa}& \shF \ar[r]& 0
            }\]

        \item
            $(\shE, \iota, \kappa) \to (\shE', \iota', \kappa')$
            :: homomorphism of extensions of $\shF$ by $\shG$とは,
            次の図式を可換にする
            homomorphism of sheaves :: $\phi: \shE \to \shE'$のこと.
            \[\xymatrix@R=10pt{
                    {}& {}& \shE \ar[dd]^-{\phi} \ar[rd]^-{\kappa}& {}& {} \\
                    0 \ar[r]& \shG \ar[ru]^-{\iota}\ar[rd]_-{\iota'}& {} & \shF \ar[r]& 0 \\
                    {}& {}& \shE' \ar[ru]_-{\kappa'}& {}& {}
            }\]
        
        \item
            $f: \shF' \to \shF$と
            $(\shE, \iota, \kappa)$ :: extension of $\shF$ by $\shG$について,
            $\shE f^*$ :: pullback of $\shF$ by $\shG$を次で定める.
            これがextension of $\shF'$ by $\shG$になっていることは簡単に確かめられる.
            まず,sheafとしては$\shE f^*$は
            \[ \shE f^*=\{ \sect{U}{e \oplus x'} \in \shE \oplus \shF' \mid \kappa_U(e)=f_U(x') \}. \]
            $\iota_{\shE f^*}, \kappa_{\shE f^*}$は次で定める.
            \begin{align*}
                \iota_{\shE f^*}:&  \shG \to \shE f^*; \qquad \sect{U}{y} \mapsto \sect{U}{\iota(y) \oplus 0} \\
                \kappa_{\shE f^*}:& \shE f^* \to \shF'; \qquad \sect{U}{e \oplus x'} \mapsto \sect{U}{y'}
            \end{align*}
            定義を終えた後に,$\shE f^*$が実際にpullback of $f$ and $\kappa$であることを示す.
        \item
            $g: \shG \to \shG'$と
            $(\shE, \iota, \kappa)$ :: extension of $\shF$ by $\shG$について,
            $g_* \shE$ :: pushout of $\shF$ by $\shG$を次で定める.
            これがextension of $\shF$ by $\shG'$になっていることは簡単に確かめられる.
            まず,sheafとしては$g_* \shE$は次の準同型のcokernelである.
            \[
                \shG \to \shG' \oplus \shE;
                \qquad
                \sect{U}{y} \mapsto \sect{U}{g_U(y) \oplus (-\iota_U(y))}.
            \]
            $\iota_{g_*\shE}, \kappa_{g_*\shE}$は次で定める.
            \begin{align*}
                \iota_{g_*\shE}:&   \shG' \to g_*\shE; \qquad \sect{U}{y'} \mapsto \sect{U}{[y', 0]} \\
                \kappa_{g_*\shE}:&  g_*\shE \to \shF;  \qquad \sect{U}{[y', e]} \mapsto \kappa_U(e)
            \end{align*}
            定義を終えた後に,$g_*\shE$が実際にpushout of $g$ and $\iota$であることを示す.
    \end{enumerate}
    \end{Def}

    extension of $\shF$ by $\shG$が成す集合を$E(\shF, \shG)$と書く.
    \begin{Thm}\label{thm:ExtAndExt1}
        $\shF, \shG$
        :: $\shO_X$-modules on ringed scheme :: $X$とする.
        この時,extension of $\shF$ by $\shG$
            :: $\shE: 0 \to \shG \to \shE \to \shF \to 0$から誘導される
        doundary map
        \[ d_{\shE}: \Hom_{\shO_X}(\shG, \shG') \to \Ext^1(\shF, \shG') \]
        は,$g \in \Hom_{\shO_X}(\shG, \shG')$を$g_*\shE$の同型類に写す.
        特に,
        \[ \Phi: E(\shF, \shG) \ni \shE \mapsto d_{\shE}(\id[\shG]) \in \Ext^1_{\shO_X}(\shE, \shG) \]
        は全単射である.
    \end{Thm}
    「特に」以降は特に有名で,
    例えば\cite{HarAG} ExIII.6.1に証明の方針が述べられているし,
    加群の場合の類似の結果としては\cite{Shiho} pp.259-264に詳しい証明がある.

    \begin{Def}
        \begin{enumerate}[label=(\arabic*), leftmargin=*]
        \item
            split extesion of $\shF$ by $\shG$を$0_{\shF, \shG}$あるいは単に$0$と書く.
            \[
            0_{\shF, \shG}:
                \xymatrix{
                    0 \ar[r]& \shG \ar[r]& \shF \oplus \shG \ar[r]& \shF \ar[r]& 0
            }\]
            
        \item
            $(\shE, \iota, \kappa)$ :: extension of sheavesについて,
            $-\shE:=(\shE, -\iota, \kappa)$と定める.

        \item
            (The Baer sum)
            $(\shE, \iota, \kappa), (\shE', \iota', \kappa')$
            :: extensions of $\shF$ by $\shG$に対し,
            $\shE+\shE'$を以下のように定める.
            \[\vcenter{\xymatrix{
                0 \ar[r]& \shG \oplus \shG \ar[r]^-{\iota \oplus \iota'}\ar[d]_-{\codiag}& \shE \oplus \shE' \ar[r]^-{\kappa \oplus \kappa'}\ar[d]& \shF \oplus \shF \ar[r]\ar@{=}[d]& 0\\
                0 \ar[r]& \shG \ar[r]& \shM \ar[r]\centerpo& \shF \oplus \shF\ar[r]& 0\\
                0 \ar[r]& \shG \ar[r]\ar@{=}[u]& \shE+\shE' \ar[r]\ar[u]& \shF \ar[r]\ar[u]_-{\diag}\centerpb& 0 \\
            }}\]
            ただし$\diag: a \mapsto (a,a), \codiag: (a,b) \mapsto a+b$.
    \end{enumerate}
    \end{Def}
    
    \subsection{Propositions.}
    %% {{{
    \begin{Lemma}\label{lemma:two_exact_seq_raise_pb}
        以下の図式が可換であり,各行は完全であるとする.
        \[\vcenter{\xymatrix{
            0 \ar[r]& \shG \ar@{=}[d]\ar[r]& \shP \ar[d]\ar[r]& \shF' \ar[d]^-{f}\ar[r]& 0 \\
            0 \ar[r]& \shG \ar[r]_-{\iota}& \shE \ar[r]_-{\kappa}& \shF \ar[r]& 0
        }}\]
        この時,$\shP$はpullback of $f$ and $\kappa$.
    \end{Lemma}
    \begin{proof}
        以下,$x \in \shX$と書いたら,
        $x$は適当な開集合$U$上の$\shX$のsection :: $x \in \shX(U)$を意味する.
        各射に次のように名前を付ける.
        \[\vcenter{\xymatrix{
            {} & \shX \ar@/^10pt/[rrd]^(.2){\alpha}\ar@/_5pt/[rdd]_(.2){\beta}& {} & {} & {} \\
            0 \ar[r]& \shG \ar@{=}[d]\ar[r]^-{\bar{\iota}}& \shP \ar[d]^-{\bar{f}}\ar[r]^-{\bar{\kappa}}& \shF' \ar[d]^-{f}\ar[r]& 0 \\
            0 \ar[r]& \shG \ar[r]_-{\iota}& \shE \ar[r]_-{\kappa}& \shF \ar[r]& 0
        }}\]

        \paragraph{$\phi: \shX \to \shP$の構成.}
        任意の$x \in \shX$をとり,これに対して$y \in \shP$を以下のように定める.
        \begin{enumerate}[labelindent=1cm, leftmargin=*]
            \item $x' \in \shP$を$\bar{\kappa}(x')=\alpha(x)$なるものとする.\mbox{} \\
                  $\bar{\kappa}$ :: surjゆえ$x'$が存在することに注意.
            \item $t' \in \shG$を$\iota(t')=\beta(x)-\bar{f}(x')$なるものとする. \mbox{} \\
                  $\beta(x)-\bar{f}(x')\in \ker \kappa=\im \iota$ゆえ$t'$が存在することに注意.
            \item $p=x'+\bar{\iota}(t')$とする.
        \end{enumerate}
        こうして得られる写像$x \mapsto p$が$\phi: \shX \to \shP$を与える.

        可換性を確認しよう.
        \begin{align*}
            \bar{\kappa}(\psi(x))&=\bar{\kappa}(x')+\bar{\kappa}(\bar{\iota}(t'))=\bar{\kappa}(x')=\alpha(x), \\
            \bar{f}(\psi(x))&=\bar{f}(x')+\bar{f}(\bar{\iota}(t'))
                =\bar{f}(x')+\iota(t')=\bar{f}(x')+\beta(x)-\bar{f}(x')=\beta(x).
        \end{align*}

        \paragraph{$\phi$ :: well-defined.}
        $x=0$の時$\phi(x)=0$であることを見れば十分.
        $x=0$ならば$\bar{\kappa}(x')=0$すなわち
        $x' \in \ker \bar{\kappa}=\im \bar{\iota}$.
        なので,$\bar{\iota}(h)=x'$となる$h \in \shG$がとれる.
        \[ \iota(t')=\beta(0)-\bar{f}(x')=-\bar{f}\bar{\iota}(h)=\iota(-h). \]
        $\iota$ :: inj.より$t'=-h$.
        したがって
        \[ p=x'+\bar{\iota}(t')=\bar{\iota}(h)+\bar{\iota}(-h)=0. \]

        \paragraph{$\shX \to \shP$の一意性.}
        最後に$\phi, \phi': \shZ \to \shE f^*$が
        同じ可換性を持つと仮定して$\psi=\phi-\phi'=0$を示す.
        仮定から$\bar{\kappa} \psi=0, \bar{f}\psi=0$が成立する.
        まず前者から
        \[ \im \psi \subseteq \ker \bar{\kappa}=\im \bar{\iota} \]
        なので任意の$x \in \shX$に対して$g \in \shG$が存在し,
        $\bar{\iota}(g)=\psi(x)$となる.
        図式の可換性から次が成立する.
        \[ \iota(g)=\bar{f}\bar{\iota}(g)=\bar{f}\psi(x)=0. \]
        行の完全性から$\iota$は単射なので$g=0$.
        任意の$x$に対して$\psi(x)=\bar{\iota}(g)=\bar{\iota}(0)=0$.
        すなわち$\psi=0$.
    \end{proof}

    \begin{Lemma}\label{lemma:two_exact_seq_raise_po}
        以下の図式が可換であり,各行は完全であるとする.
        \[\vcenter{\xymatrix{
            0 \ar[r]& \shG \ar[d]_-{g}\ar[r]^-{\iota}& \shE \ar[d]\ar[r]^-{\kappa}& \shF \ar@{=}[d]\ar[r]& 0 \\
            0 \ar[r]& \shG' \ar[r]& \shP \ar[r]& \shF \ar[r]& 0
        }}\]
        この時,$\shP$はpushout of $g$ and $\iota$.
    \end{Lemma}
    \begin{proof}
        各射に次のように名前を付ける.
        \[\vcenter{\xymatrix{
            0 \ar[r]& \shG \ar[d]_-{g}\ar[r]^-{\iota}& \shE
            \ar[d]^-{\bar{g}}\ar[r]^-{\kappa}\ar@/^10pt/[rdd]^(.8){\alpha}& \shF \ar@{=}[d]\ar[r]& 0 \\
            0 \ar[r]& \shG' \ar[r]_-{\bar{\iota}}\ar@/_5pt/[rrd]_-{\beta}& \shP \ar[r]_-{\bar{\kappa}}& \shF \ar[r]& 0 \\
            {} & {} & {} & \shX
        }}\]

        \paragraph{$\shP=\bar{g}(\shE)+\bar{\iota}(\shG')$.}
        $p \in \shP$を任意に取る.
        \begin{enumerate}
            \item
                $\kappa(e)=\bar{\kappa}(p)$となる$e \in \shE$をとる. 
                $\kappa$ :: surj.ゆえ$e$が存在することに注意.
            \item
                $\bar{\iota}(g')=p-\bar{g}(e)$となる$g' \in \shG'$をとる. 
                $p-\bar{g}(e) \in \ker \bar{\kappa}=\im \bar{\iota}$ゆえ
                $g'$が存在することに注意.
        \end{enumerate}
        すると$p=\bar{g}(e)+\bar{\iota}(g')$となる.
        実際,
        \[ \bar{g}(e)+\bar{\iota}(g')=\bar{g}(e)+(p-\bar{g}(e))=p. \]

        \paragraph{$\phi: \shP \to \shX$の構成.}
        $p \in \shP$に対して,
        $p=\bar{g}(e)+\bar{\iota}(g')$となる$e \in \shE, g' \in \shG'$をとる.
        これを元に$\phi(p)=\alpha(e)+\beta(g')$とする.
        すると明らかに
        $\phi \circ \bar{g}=\alpha, \phi \circ \bar{\iota}=\beta$が成立する.
        $\phi$がwell-definedならmodule homomorphismになることは明らか.
        
        \paragraph{$\phi$ :: well-defined.}
        $(p=)\bar{g}(e)+\bar{\iota}(g')=0$となる$e \in \shE, g'\in \shG'$をとる.
        $\alpha(e)+\beta(g')=0$となることを示せば良い.
        まず,$0=\bar{\kappa}(\bar{g}(e)+\bar{\iota}(g'))=\kappa(e)$.
        したがって$e \in \ker \kappa=\im \iota$であり,
        $\iota(h)=e$を満たす$h \in \shG$が存在する.
        \[ 0=\bar{g}(e)+\bar{\iota}(g')=\bar{g}\iota(h)+\bar{\iota}(g')=\bar{\iota}(g(h)+g'). \]
        $\bar{\iota}$ :: injより$g(h)+g'=0$.
        よって
        \[ \alpha(e)+\beta(g')=\alpha \iota(h)+\beta(-g(h))=0. \]

        \paragraph{$\phi$ :: unique.}
        $\phi':\shP \to \shX$も$\phi$と同様の条件を満たすとする.
        $P=\bar{\iota}(\shG')+\bar{g}(\shE)$なので,
        \[ \phi'(p)=\phi'(\bar{g}(e)+\bar{\iota}(g'))=\alpha(e)+\beta(g')=\phi(p). \]
    \end{proof}
    %% }}}
    \begin{Lemma} \label{lemma:peoperties_of_pushout_of_ext}
        $g, g': \shG \to \shG'$と,
        extension of $\shF$ by $\shG$をとる.
        \[\xymatrix{
            0 \ar[r]& \shG \ar[r]& \shE \ar[r]& \shF \ar[r]& 0
        }\]
        以下が成り立つ.
        \begin{enumerate}[label=(\alph*), leftmargin=*, labelindent=1cm]
            \item $0_*\shE=0_{\shF, \shG'}$,
            \item $g_*(-\shE)=-g_*\shE$,
            \item $g_*\shE+g'_*\shE=(g+g')_*\shE$.
        \end{enumerate}
    \end{Lemma}
    \begin{proof}
        \paragraph{proof of (a).}
        以下の図式は可換である.
        \[\xymatrix{
            0 \ar[r] & \shG \ar[d]_-{0}\ar[r]^{\iota}&
                    \shE \ar[r]^-{\kappa}\ar[d]^-{i_2 \circ \kappa}& \shF \ar[r]\ar@{=}[d]& 0 \\ 
            0 \ar[r]& \shG' \ar[r]_-{i_1}& \shG' \oplus \shF \ar[r]^-{\pr_2}& \shF \ar[r]& 0
        }\]
        よってpushoutの一意性から$0_*\shE \iso \shG' \oplus \shF=0_{\shF, \shG'}$.

        \paragraph{proof of (b).}
        以下の可換図式を見よ.
        \[\xymatrix{
            0 \ar[r]& \shG \ar[r]^{(-1)}\ar[d]_-{g}& \shG \ar[r]^{\iota}& \shE \ar[r]^-{\kappa}\ar[d]& \shF \ar[r]\ar@{=}[d]& 0 \\ 
            0 \ar[r]& \shG' \ar@{=}[r]& \shG' \ar[r]& g_*(-\shE) \ar[r]& \shF \ar[r]& 0 \\ 
        }\]
        $(-1)$は同型であることと,
        $g \circ (-1)=(-1) \circ g$から,以下も可換図式.
        \[\vcenter{\xymatrix{
            0 \ar[r]& \shG \ar@{=}[r]\ar[d]_-{g}& \shG \ar[r]^{\iota}& \shE \ar[r]^-{\kappa}\ar[d]& \shF \ar[r]\ar@{=}[d]& 0 \\ 
            0 \ar[r]& \shG' \ar[r]_-{(-1)}& \shG' \ar[r]& g_*(-\shE) \centerpo\ar[r]& \shF \ar[r]& 0 \\ 
        }}\]
        よってpushoutの一意性から$g_*(-\shE) \iso -g_*\shE$.
        また$g_*(\shE) \iso (-g)_* \shE$も分かる.

        \paragraph{proof of (c).}
        \[\vcenter{\xymatrix{
            0 \ar[r]& \shG \ar[r]^-{\iota}\ar[d]_-{\diag}& \shE \ar[r]^-{\kappa}\ar[d]_-{\diag}& \shF \ar[r]\ar[d]_-{\diag}& 0\\
            0 \ar[r]& \shG \oplus \shG \ar[r]^-{\iota \oplus \iota}&
                      \shE \oplus \shE \ar[r]^-{\kappa \oplus \kappa}& \shF \oplus \shF\ar[r]\ar@<5pt>[d]^-{\pr_1}& 0\\
            0 \ar[r]& \shG \oplus \shG\ar[r]\ar@{=}[u]& (\shE \oplus \shE)\diag^* \ar[r]\ar[u]& \shF \ar[r]\ar[u]^-{\diag}\centerpb& 0 \\
        }}\]
        今,この図式は可換であり,各行は完全列である.
        $(\kappa \oplus \kappa) \circ \diag=\diag \circ \kappa$なので,
        pullback $(\shE \oplus \shE)\diag^*$の普遍性から,
        図式を可換に保つ$\phi=\langle \diag, \kappa \rangle: \shE \to (\shE \oplus \shE)\diag^*$が存在する.
        \[\vcenter{\xymatrix{
            0 \ar[r]& \shG \ar[r]^-{\iota}\ar[d]_-{\diag}& \shE \ar[r]^-{\kappa}\ar[dd]_-{\phi}& \shF \ar[r]\ar[d]_-{\diag}& 0\\
            & \shG \oplus \shG&{}& \shF \oplus \shF \ar[d]^-{\pr_1}&\\
            0 \ar[r]& \shG \oplus \shG\ar[r]\ar@{=}[u]& (\shE \oplus \shE)\diag^* \ar[r]& \shF \ar[r]& 0 \\
        }}\]
        右の縦の射は合成すると恒等射.
        したがって左の四角形はpushout diagram.
        pushoutの一意性から一意性から$(\shE \oplus \shE)\diag^* \iso \diag^* \shE$.

        このことから求める同型が得られる.
        \begin{align*}
            g_*\shE+g'_*\shE
            =&  \codiag_*(g_*\shE \oplus g'_*\shE)\diag^*         \\
            =&  \codiag_*(g \oplus g')_*(\shE \oplus \shE)\diag^* \\
            =&  (\codiag_*(g \oplus g')_*\diag_*)\shE             \\
            =&  (\codiag \circ (g \oplus g') \circ \diag)_*\shE   \\
            =&  (g+g')_*\shE.
        \end{align*}
        (2つめの等号は自明.)
    \end{proof}
    
    \begin{Lemma}
        $\shF, \shG$ :: $\shO_X$-modules on ringed scheme :: $X$とする.
        この時,$E(\shF, \shG)$には加法群の構造が定まる.
    \end{Lemma}
    \begin{proof}
        $0_{\shF, \shG}$がBear sumについての単位元である:
        \[ \shE+0=\id[*]\shE+0_*\shE=(\id+0)_*\shE=\id[*]\shE=\shE. \]
        $-\shE$が逆元である:
        \[ \shE+(-\shE)=\id[*]\shE+(-\id)_*\shE=(\id-\id)_*\shE=0_*\shE=0. \]
        演算$+$は可換:
        \[ \shE+\shE'=\codiag_*(\shE \oplus \shE')\diag^*=\codiag_*(\shE' \oplus \shE)\diag^*=\shE'+\shE. \]
        結合律が成り立つ:
        \begin{align*}
            {}& (\shE_1+\shE_2)+\shE_3 \\
            =&  \codiag_*((\shE_1+\shE_2) \oplus \shE_3) \diag^* \\
            =&  \codiag_*(\codiag \oplus \id)_*((\shE_1 \oplus \shE_2) \oplus \shE_3)(\diag \oplus \id)^* \diag^* \\
            =&  \codiag_*(\id \oplus \codiag)_*(\shE_1 \oplus (\shE_2 \oplus \shE_3))(\id \oplus \diag )^* \diag^* \\
            =&  \codiag_*(\shE_1 \oplus (\shE_2+\shE_3))\diag^* \\
            =&  \shE_1+(\shE_2+\shE_3).
        \end{align*}
    \end{proof}

    \begin{Lemma}\label{lemma:PhiIsIso}
        全単射$\Phi: E(\shF, \shG) \to \Ext^1_{\shO_X}(\shE, \shG)$は
        加法群の間の同型である.
    \end{Lemma}
    \begin{proof}
        最初に次のことに注意する: 
        $f+f'=\codiag \circ (f \oplus f') \circ \diag$.
        したがって$\Hom(\shF, \shG)$の加法は次のように定まる.
        \[
            +: \Hom(\shF, \shG)^{\oplus 2}=\Hom(\shF^{\oplus 2}, \shG^{\oplus 2}) \xrightarrow{(\circ \diag)}
            \Hom(\shF^{\oplus 2}, \shG) \xrightarrow{(\codiag \circ)}
            \Hom(\shF, \shG)
        \]
        ここから$\Ext^1(-, \shG)$の間の射が誘導され,
        それぞれ$\codiag_*, \diag^*$に対応することを見る.
        (TODO)
    \end{proof}

\section{First Order Deformation of a Local Complete Intersection.}
    この節は\ref{thm:SpaceOf1stOrdDefOfLCI}を
    証明するための必要最低限の定義と命題のまとめである.
    元の命題と比較して,このノートでは$X$を$\C$上のものに限定し,
    defoemationも一般のlocal artinian ringではなく$D=\C[\epsilon]$に限定している.
    したがって
    formal deformation(\cite{DefLCI} 6.1),
    abstruct lifting(\cite{DefLCI} 4.2), 
    first order deformationが一致している.

    以下,この節では$X$を以下のようなものとする(\cite{DefLCI} Hypotheses4.1).
    この条件を($\dagger$)と呼ぶ.
    \begin{screen}
    \begin{itemize}
        \item flat,
        \item generically smooth,
        \item local complete intersection,
        \item finite type
    \end{itemize}
        scheme over $\C$.
    \end{screen}
    特にstable curve over $\C$はこれらの条件を満たす.

    ``local complete intersection"の定義を改めて書き下しておく.
    \begin{Def}[(local) complete intersection \cite{DefLCI} p.21, \cite{HarAG} p.185]
        $X$ :: scheme of finite type over $\C$が
        complete intersectionであるとは次が成立すること:
        $X$は$P$ :: smooth scheme over $\C$ (\cite{HarAG} III.10)に埋め込まれ,
        さらにideal sheaf :: $\ker(X \inclmap P)$が
        $\codim(P, X)$個のglobal sectionで生成されること(\cite{HarAG} p.121).

        $X$ :: scheme of finite type over $\C$が
        locally complete intersectionであるとは
        $\cvU$ :: open covering of $X$が存在し,
        任意の$U \in \cvU$がcomplete intersectionであること.
    \end{Def}
    
    議論は,
    $\nu(\defX_1, \defX_2)$,$\shE(\defX_1, \defX_2)$,$e(\defX_1, \defX_2)$の対応の連鎖である.
    これらはいずれもfirst order deformations :: $\defX_1, \defX_2$の
    「差」を表現する量である.
    ここでの「差」の意味を理解するには,
    最初に命題(\ref{prop:key1}), (\ref{prop:key2})のステートメントを見るのが良い.
    そして自明なfirst order deformation of $X$ :: $X \times D$と
    与えられたfirst order deformationの「差」である$e(\defX, X \times D)$によって
    first order deformation of $X$を分類する(定義\ref{def:KSclass-map}).

    \begin{Def}
        $\defM$ :: flat scheme of finite type over $D$とし,
        $M=\defM_{0}$ (fiber of $\defM \to \Spec D$ at $0$)とする.
        $X \subseteq M$に対し,
        $\defX \subseteq \defM$であるようなdeformation of $X$ :: $\defX$を,
        $X$のembedded deformation (in $\defM$)と呼ぶ.
    \end{Def}
    $X$ :: affine scheme of finite type over $\C$とすると,
    明らかに$X \inclmap \affine^n_{\C}$ :: closed embeddingが存在する.
    $\defX$ :: first order deformation of $X$も
    補題(\ref{lemma:def_of_affine_is_affine})よりaffine scheme of finite type over $D$.
    したがって$(\dagger)$を満たすならば

    \begin{Thm}
        $X \subseteq M$が($\dagger$)を満たす時,
        $\defX \subseteq \defM$ :: first order deformation of $X$も
        local complete intersectionである.
    \end{Thm}

    \subsection{\tp{$\nu(\defX_1, \defX_2)$}{v(X1,X2)}}
    \begin{Def}
        $X$ :: \textbf{complete intersection} over $\C$ embedded in $P$とする.
        これのideal sheafを$\shI \subseteq \shO_P$とする.
        $\defX_1, \defX_2$ :: first order deformation of $X$とすると,
        $\defX_1, \defX_1$ :: embedded in $P$となる.
        そこでideal sheafをそれぞれ$\shI_1, \shI_2 \subseteq \shO_P$とする.
        first order deformation of $X$の定義から,
        $\shI_i/\epsilon \shI_i \iso \shI \ (i=1,2)$.

        写像$\shI \to (\epsilon D) \otimes_{\C} \shO_X$を
        以下のように定める.
        まず,$\sect{U}{f} \in \shI$に対し,
        \[ \sect{U}{\tilde{f}_i} \bmod \epsilon \shI_i=\sect{U}{f} \ (i=1,2) \]
        となる$\sect{U}{\tilde{f}_i}$が存在する.
        そこで
        \[
            \sect{U}{f}
            \mapsto
            \sect{U}{\tilde{f}_1-\tilde{f}_2} \bmod (\epsilon D) \otimes_{\C} \shI
            \in (\epsilon D) \otimes_{\C} (\shO_P/\shI)=(\epsilon D) \otimes_{\C} \shO_X
        \]
        と写す.
        これは$\tilde{f}_i$のとり方に依らず,well-defined.
        この写像を$\nu(\defX_1, \defX_1)$と書く.

        以下の$\C$-moduleとしての同型があるため,
        $\nu(\defX_1, \defX_2)$を以下のいずれの集合の元ともみなす.
        \begin{align*}
            {}&   \Hom_{\shO_P}(\shI, (\epsilon D) \otimes_{\C} \shO_X) \\
            \iso& \Hom_{\shO_P}(\shI/\shI^2, (\epsilon D) \otimes_{\C} \shO_X) \\
            \iso& H^0(X, (\epsilon D) \otimes_{\C} (\shI/\shI^2)\sidehat) \\
            \iso& (\epsilon D) \otimes_{\C} H^0(X, (\shI/\shI^2)\sidehat) \\
            \iso& H^0(X, (\shI/\shI^2)\sidehat)
        \end{align*}
    \end{Def}

    \begin{Prop}[\cite{DefLCI} Prop2.8a,b,c,d,e] \label{prop:properties_of_nu}
        $P'$ :: flat scheme of finite type over $D$, $P=P'_{0}$, 
        $X$ :: \textbf{complete intersection} embedded in $P$とする.
        \begin{enumerate}[label=(\alph*), leftmargin=*]
            \item $\nu(\defX_1, \defX_2)=0 \iff \defX_1=\defX_2$.

            \item $\nu(\defX_1, \defX_2)+\nu(\defX_2, \defX_3)=\nu(\defX_1, \defX_3)$.

            \item $\nu(\defX_2, \defX_1)=-\nu(\defX_1, \defX_2)$.

            \item
                任意の$\defX$ :: first order deformation of $X$, 
                任意の$\nu \in H^0(X, (\shI/\shI^2)\sidehat)$に対し,
                $\defY$ :: first order deformation of $X$が存在して,
                $\nu=\nu(\defX, \defY)$となる.

            \item
                $U$ :: open subset of $X$について,
                $\nu(\defX_1|_U, \defX_2|_U)=\nu(\defX_1, \defX_2)|_U:
                (\shI/\shI^2)|_U \to (\epsilon D) \otimes_{\C} \shO_U$.
        \end{enumerate}
    \end{Prop}
    \begin{proof}
        (d)のみ証明する.他は自明であろう.
        
        $\shI$ :: ideal sheaf of $X$,
        $\shI'$ :: ideal sheaf of $\defX$とし,
        sheaf of ideal :: $\shJ \subseteq \shO_{P'}$を次のように定める.
        ただし$\shI_P=\ker (P \inclmap P')$.
        sheafのsectionは全てopen subset in $P'$ :: $U$上のものである.
        (これがsheaf of idealであることは自明.)
        \[
            \shJ=\{ \tilde{f}' \in \shO_{P'} \mid
                \tilde{f}' \bmod \epsilon \shI_{P}=:f \in \shI \mand 
                \Exists{f' \in \shI'}
                (f'-\tilde{f}') \bmod \epsilon \shI=v_U(f) \}
        \]

        $\shJ$で定まる$P$のsubscheme :: $\defY$が$X$のdeformationであることを示す.
        これは自然な全射$(\times \epsilon): \shJ/\epsilon \shJ \to \epsilon \shJ$が
        単射(したがって同型)であることを示せば良い(\cite{DefLCI} Lemma 2.6).
        (TODO)
    \end{proof}

    \subsection{\tp{$\shE(\defX_1, \defX_2)$}{E(X1,X2)}}
    \begin{Lemma}
        $X$ :: \textbf{complete intersection} over $\C$ embedded in $P$とし,
        ideal sheafは$\shI \subseteq \shO_P$であるとする.
        この時,$\shI/\shI^2$ :: locally free sheaf of rank $n:=\dim X$.
    \end{Lemma}
    \begin{proof}
        localな問題なので,
        $x \in X \subset P$での$\shI/\shI^2$のstalkがfree moduleであることを示す.
        $A:=\shO_{P, x}$とし,
        $I:=\shI_x$を生成するregular sequenceを$x_1, \dots, x_r$とする.
        \cite{IntTh} Lemma A.6.1より(この文献の証明は同値な命題\cite{Mat} Thm16.2のものより美しい),
        graded ringとして$(A/I)[t_1,\dots,t_r] \iso \bigoplus_{d \geq 0}(I^{d}/I^{d+1})$.
        $1$次成分の同型から$(A/I)^{\oplus r} \iso I/I^2$.
    \end{proof}

    \begin{Lemma}[First Fundamental Exact Sequence]
        $X$ :: \textbf{complete intersection} over $\C$ embedded in $P$とし,
        ideal sheafは$\shI \subseteq \shO_P$であるとする.
        この時,以下はexact.
        \[\xymatrix{
                0 \ar[r]& \shI/\shI^2 \ar[r]^-{d}& (\shDer_{P/\C})|_{X} \ar[r]& \shDer_{X/\C} \ar[r]& 0
        }\]
        すなわち,
        $(\shDer_{P/\C})|_{X}$ :: extension of $\shDer_{X/\C}$ by $\shI/\shI^2$.
    \end{Lemma}
    \begin{proof}
        よく知られている通り,
        最初の射が単射であることを示しさえすれば良い.
        
        $\shK=\ker d$とすると,$\shK \subseteq \shI/\shI^2$.
        したがって$\shI/\shI^2$同様$\shK$もlocally free.
        一方,\cite{HarAG} ThmII.8.17(2)の証明より
        $d$はirreducible pointの近傍でinjective.
        したがって$\Supp \shK$ :: support of $\shK$は
        $\Sing X$ :: $X$のsingular pointsである.
        $X$についての仮定からこれは離散集合で,すなわち開集合を含まない.
        もし$\shK_x \neq 0$ならば,
        $\shK$はtrivialization open coverを持たないので,
        $\shK$ :: locally freeに反する.
        よって$\shK=\ker d=0$.
    \end{proof}

    \begin{Def}
        $U \subseteq X$ :: \textbf{complete intersection} over $\C$ embedded in $P$とする.
        $\defX_1, \defX_2$ :: first order deformation of $X$について,
        \[
            \shE(\defX_1|_U, \defX_2|_U)
            :=\nu(\defX_1|_U, \defX_2|_U)_*((\shDer_{P/\C})|_{U})
            =(\nu(\defX_1, \defX_2)|_U)_*((\shDer_{P/\C})|_{U})
        \]
        をextension of $\shDer_{U/\C}$ by $(\epsilon D) \otimes_{\C} \shO_{U}$として定める.
    \end{Def}

    \begin{Lemma}
        $U \subseteq X$ :: \textbf{affine complete intersection} over $\C$ embedded in $P$とする.
        この時$X$ :: finite type over $\C$なので
        $U \inclmap \affine^n_{\C}$ :: closed embeddingが存在する.
        $\defU$ :: first order deformation of $U$ 
        \footnote{ 補題(\ref{lemma:def_of_affine_is_affine})よりこれもaffine. }
        について,$U \inclmap \affine^{n}_{\C}$の
        拡張$\defU \inclmap \affine^{n}_{D}$が存在する.
    \end{Lemma}
    \begin{proof}
        \cite{DefLCI} Lemma 4.8.
    \end{proof}

    \begin{Lemma}
        $U \subseteq X$ :: \textbf{affine complete intersection} over $\C$ embedded in $P$とする.
        $P_1, P_2, P_3$ :: nonsingular affine scheme over $\C$と
        $U \inclmap P_i$が与えられているとする.
        $\defP_i=P_i \times_{\C} D$への
        $U$のlifting (first order deformation) :: $\defU_i \to \defP_i$
        を任意にとり,
        対応するextension of $\shDer_{U/\C}$ by $(\epsilon D) \otimes_{\C} \shO_{U}$を
        $\shE_i$とする.
        
        この時同型$\alpha_{j, i}: \shE_i \isomap \shE_j$が存在し,
        cocycle condition :: $\alpha_{1,3}=\alpha_{1,2} \circ \alpha_{2,3}$が成立する.
    \end{Lemma}
    \begin{proof}
        \cite{DefLCI} p.24.
    \end{proof}

    したがって$\{U_i\}$ :: open affine, complete intersection covering of $X$について,
    $\shE(\defX_1|_{U_i}, \defX_2|_{U_i})$の貼り合わせることが出来
    \footnote
    {
        次のことに注意して上の二つの補題を使う:
        restriction of sheaves to open subsetはleft adjoint functorであるから,
        pushout of extensions (colimit)を保つ,
        よって$V \subseteq U$について
        \[
            \shE(\defX_1|_U, \defX_2|_U)|_V
            \iso (\nu(\defX_1, \defX_2)|_V)_*(\shDer_{P/\C}|_{V})
            \iso \nu(\defX_1|_V, \defX_2|_V)_*(\shDer_{P/\C}|_{V}).
        \]
    },
    こうして$\shE(\defX_1, \defX_2)$を得る.


    \begin{Prop}[\cite{DefLCI} Prop4.9a,b,c,f]\label{prop:key1}
        $\defX_1, \defX_2$ :: first order deformations of $X$とする.
        \begin{enumerate}[label=(\alph*), leftmargin=*]
        \item
            $\shE(\defX, \defX) \iso 0_{\shDer_{X/ \C}, (\epsilon D) \otimes \shO_X}$.
        \item
            $\shE(\defX_2, \defX_1)=-\shE(\defX_1, \defX_2)$.
        \item 
            $\shE(\defX_1, \defX_2)+\shE(\defX_2, \defX_3) \iso \shE(\defX_1, \defX_3)$.

        \item
            任意の$\shE$ :: extension of $\shDer_{X/\C}$ by $(\epsilon D) \otimes_{\C} \shO_{X}$と
            任意の$\defX$ :: first order deformation of $X$に対し,
            \[ \shE \iso \shE(\defX, \defY) \]
            なる$\defY$ :: first order deformation of $X$が存在する.
        \end{enumerate}
    \end{Prop}
    (d)を命題(\ref{prop:splittings_and_isos})の後に証明する.
    他は命題(\ref{prop:properties_of_nu})の対応する命題と
    補題(\ref{lemma:peoperties_of_pushout_of_ext})から得られる.

    \begin{Prop}[\cite{DefLCI} Prop3.9, Prop4.10] \label{prop:splittings_and_isos} \label{prop:key2}
    \begin{enumerate}[label=(\alph*), leftmargin=*]
    \item 
        $\defX_1, \defX_2$ :: first order deformation of $X$とする.
        この時,splittings of $\shE(\defX_1, \defX_2)$と
        isomorphisms :: $\defX_1 \iso \defX_2$の間に一対一対応がある.
        さらにこの対応を通して,extensionsの加法と同型の合成が対応する
        \footnote
        {
            すなわち$\phi_1: \defX_1 \iso \defX_2, \phi_2: \defX_2 \iso \defX_3$について,
            $\phi_2 \phi_1: \defX_1 \iso \defX_3$は$\shE(\defX_1, \defX_3)$に対応する.
        }.

    \item
        また,$\shE_1, \shE_2$ :: extension of sheavesとする.
        この時splittings of $\shE_1-\shE_2$とisomorphisms :: $\shE_1 \iso \shE_2$の間に
        一対一対応がある.
    \end{enumerate}
    \end{Prop}

    \begin{proof}[命題(\ref{prop:key1})(d)の証明]
        まず$\{X_{\alpha}\}_{\alpha}$を$X$のaffine open coverとし,
        $\defX_{\alpha}=\defX|_{X_{\alpha}}$とおく.
        $\defX$ :: finite type over $\C$なので,
        うまくcoverを取れば$\defX_{\alpha} \inclmap \affine^{n}_{\C}$が存在するように出来る.
        このembeddingに対応するconormal bundle of $X_{\alpha}$を$\shC_{\alpha}$とする.
        すると以下のSESが存在する.
        \[
            E:
        \xymatrix{
            0 \ar[r]& \shC_{\alpha} \ar[r]& \shDer_{\affine^{n_{\alpha}}}|_{X_{\alpha}} \ar[r]& 
            \shDer_{X_{\alpha}} \ar[r]& 0
        }\]
        このextensionを$E:=\shDer_{\affine^{n_{\alpha}}}|_{X_{\alpha}}$と略す.
        ここから$\Ext$のLESが誘導される.
        \[\xymatrix{
            \Hom_{\shO_{X_{\alpha}}}(\shC_{\alpha}, (\epsilon D) \otimes \shO_{X_{\alpha}})
            \ar[r]^-{d_{E}}&
            \Ext^1_{\shO_{X_{\alpha}}}(\shDer_{X_{\alpha}}, (\epsilon D) \otimes \shO_{X_{\alpha}})
            \ar[r]&
            \Ext^1_{\shO_{X_{\alpha}}}(\shDer_{\affine^{n_{\alpha}}}|_{X_{\alpha}},
                (\epsilon D) \otimes \shO_{X_{\alpha}})
            =0
        }\]
        右の$=0$は\cite{HarAG} Prop III.6.7, Prop III.6.3, Thm III.3.7から得られる.
        したがって$d_{E}$ :: surj.
        
        なので定理(\ref{thm:ExtAndExt1})より,
        $f_{\alpha}: \shC_{\alpha} \to (\epsilon D) \otimes \shO_{X_{\alpha}}$と
        同型$(f_{\alpha})_*E \iso \shE|_{X_{\alpha}}$が存在する.
        一方命題(\ref{prop:properties_of_nu})より,
        $f_{\alpha}=\nu(\tilde{\defX}_{\alpha}, \defX_{\alpha})$を満たす
        first order deformation of $X_{\alpha}$
        :: $\tilde{\defX}_{\alpha} \subseteq \affine^{n_{\alpha}}_D$が存在する.
        
        こうして得られる$\{\defX_{\alpha}\}_{\alpha}$を貼り合わせる.
        証明には命題(\ref{prop:splittings_and_isos})を利用する.
        (TODO)
    \end{proof}

    \subsection{\tp{$e(\defX_1, \defX_2)$}{e(X1,X2)}}
    \begin{Def}
        $T^i(X)=\Ext^i_{\shO_X}(\shDer_{X/\C}, \shO_X)$とおく.
        $\defX$ :: first order deformation of $X$に対し,
        \[
            e(\defX_1, \defX_2)
                \in
                \Ext^1(\shDer_{X/\C}, (\epsilon D) \otimes_{\C} \shO_X)
                \iso (\epsilon D) \otimes T^1(X)
                \iso T^1(X)
                \iso \Hom((\epsilon D)^*, T^1(X))
        \]
        を,$\shE(\defX_1, \defX_2)$に対応する元とする(\ref{thm:ExtAndExt1}).
    \end{Def}
    補題(\ref{lemma:PhiIsIso})より,
    命題(\ref{prop:key1}), (\ref{prop:key2})と同様の命題が
    $e(\defX_1, \defX_2)$についても性質する.

    \begin{Def}[Kodaira-Spencer class/map, \cite{DefLCI}]\label{def:KSclass-map}
        $T^i(X)=\Ext^i_{\shO_X}(\shDer_{X/\C}, \shO_X)$とおく.
        $\defX$ :: first order deformation of $X$に対し,
        \[
            k_{\defX}=e(\defX, X \times D)
                \in T^1(X) \iso \Hom((\epsilon D)^*, T^1(X))
        \]
        とおく.
        この$k_{\defX}$をKodaira-Spencer class of $\defX$と呼ぶ.
        対応する写像$K_{\defX}: (\epsilon D)^*, T^1(X)$を
        Kodaira-Spencer map of $\defX$と呼ぶ.
    \end{Def}
    
\subsection{Complete Classification.}
    \begin{Thm}\label{thm:SpaceOf1stOrdDefOfLCI}
        First order deformation of $X$の同値類と
        $\Ext^1_{\shO_X}(\shDer_{X/\C}, \shO_X)$の元は一対一に対応する.
        (cf. \cite{DefLCI} Prop 6.12)
    \end{Thm}
    \begin{proof}
        命題(\ref{prop:key1}), (\ref{prop:key2})より,
        $\defX \mapsto k_{\defX}$がこの対応を与えることは明らか.
    \end{proof}

    \begin{Cor} \label{cor:nonsing-fod}
        $X$ :: nonsingular and have finite dimentionならば,
        First order deformation of $X$の同値類と
        $H^1(X, \shT_X)$の元は一対一に対応する.
    \end{Cor}
    \begin{proof}
        仮定より,
        $\shDer_{X/\C}$ :: locally free sehaf of finite rank.
        したがって\cite{HarAG} PropIII.6.7, Prop II.6.3から次の同型が成立する.
        \[
            \Ext^1_{\shO_X}(\shO_X \otimes \shDer_{X/\C}, \shO_X)
            \iso \Ext^1_{\shO_X}(\shO_X, \shT_X \otimes \shO_X)
            \iso H^1(X, \shT_X)
        \]
    \end{proof}

\section{Two Examples of Other Deformation Theories}
    deformation theoryの対象はschemeの他にもある.
    例えば,次の二つがある.

    \begin{itembox}[l]{Deformation of a coherent sheaf :: $F$ on a scheme $X$, over a fixed scheme}
        $\defX$ :: deformation of $X$ over $(S, s_0)$とする.
        deformation of $F$ over $X$とは,
        $\shF$ :: flat coherent sheaf on $\defX$と
        homomorphism :: $\phi: \shF \to F$の組であって,
        誘導される射
        \[ \phi \otimes_{\shO_{\defX}} 1_{\shO_{X}}: \shF \otimes_{\shO_{\defX}} \shO_{X} \to F \]
        が同型であるもの.
        (ref. \cite{HarDef} \S 7, p.53)
    \end{itembox}

    \begin{itembox}[l]{Deformation of a map $f: X \to Y$ with both $X$ and $Y$ fixed}
        $X, Y$ :: scheme,
        $(S, s_0)$ :: $\C$-pointed scheme,
        $f: X \to Y$ :: morphismとする.
        Deformation of a map $f: X \to Y$ with both $X$ and $Y$ fixedとは
        morphism :: $\bar{f} : X \times S \to Y \times S$であって,
        $\bar{f}|_{X \times \{s_0\}}=f$であるもの.
        (ref. \cite{HaMo}p.93)
    \end{itembox}

    それぞれ,first order deformationが成す空間が分かっている.
    \begin{Thm}[\cite{HarDef} Thm2.7]
        $X$ :: scheme over $\C$,
        $F$ :: coherent sheaf on $X$とする.
        この時,$F$のfirst order deformationとは,
        $\shF$ :: cohenrent sheaf on $\defX=X \times_{\C} D$と
        homomorphism :: $\phi: F \to \shF$の組であって
        誘導される射$\phi \otimes_{D} 1_{\C}: \shF \otimes \C \to F$が同型であるものとする.

        この時,
        first order deformation of $F$ over $\shX$の同型類と
        $\Ext^1_{\shO_X}(F, F)$の元とが,一対一対応する.
    \end{Thm}

    \begin{Cor}[\cite{HarDef} Prop2.6]
        上の定理で$F$をinvertible sheafに限定すると,
        first order deformation of $F$ over $\shX=X \times D$の同型類は
        $H^1(X, \shO_X)$の元と一対一対応する.
    \end{Cor}
    \begin{proof}
        系(\ref{cor:nonsing-fod})の証明と全く同様.
    \end{proof}
    
    \begin{Thm}
        $X, Y$ :: fiexed scheme, 
        $f: X \to Y$をとる.
        この時,
        first order deformation of a map $f$ with both $X$ and $Y$ fixedの同型類と,
        $H^0(X, f^*\shT_Y)$の元とが一対一対応する.
    \end{Thm}
    こちらについては詳しい文献が見つかっていない.
    しかし,``Deformation of a map $f: X \to Y$ with only $Y$ fixed"については,
    解析的な場合について\cite{GACII} \S 8で述べられている.

\section{Functor of Artin Rings - Abstruct Deformation Theory}
    $3$つの圏を次のように定める.
    $L$ :: local noetherian $\C$-algebras with residue field $\C$とする.
    \begin{description}[labelindent=1cm]
        \item[$\LA_{L}$:]  the category of local artinian $L$-algebras with residue field $\C$.
        \item[$\CLN_{L}$:] the category of complete local noetherian $L$-algebras with residue field $\C$.
        \item[$\LN_{L}$:]  the category of local noetherian $L$-algebras with residue field $\C$.
    \end{description}
    $L=\C$の時は添字を略す.
    (ref. \cite{DefAS} p.1)

    $\LA_L \subset \CLN_L \subset \LN_L$という包含関係があることに注意.

    \begin{Def}[\cite{DefAS} \S2.2]
        以下のようなfunctorをfunctor of artin ringsと呼ぶ.
        \[ F: \LA_L \to \mathrm{(Sets)}. \]
        ここで$L \in \CLN$.

        $F(\C)$が$1$元集合(singleton)ならば,
        $F$は特にpredeformation functorと呼ばれる.

        $F$がfunctor
        \[ \Hom_{\CLN_L}(R, -) \qquad R \in \CLN_L \]
        と同型である時,$F$ :: prorepresentableと言う.
    \end{Def}

    \begin{Def}[\cite{DefAS} \S2.2]
        (TODO)
    \end{Def}

    predeformation functor :: $F$が
    (semi)universal formal elementを持つか,
    ということについては,以下の定理が大変有用である.
    逆に以下の定理がpredeformation functorを考える重要性を示している.

    \begin{Thm}[\cite{DefAS} Thm 2.3.2, p.56]
        (TODO)
    \end{Thm}
%    M. Schlessinger, ``Functors of Artin rings"

\bibliographystyle{jplain}
\bibliography{reference}
\end{document}
