    \section{第1,7,8章/作用素}
    \subsection{定義}
    \subsubsection{一般の作用素}
    \begin{Def}
        線形空間$\spX$の部分集合$\dom$から
        線形空間$\spY$への写像$T$を,
        $\spX$から$\spY$への\textbf{作用素}と呼ぶ.
        $\dom$は$T$の定義域と呼ばれ,$\dom T$で表す.
        $\{v \in \spY : \Exists{u \in \dom T} (v=Tu)\}$
        は$T$の値域と呼ばれ,$\ran T$で表す.
    \end{Def}
    一般に,\kenten{作用素はその原像全体で定義されているとは限らない}.

    \begin{Def}
        ノルム空間$\spX, \spY$について,
        線形作用素$T:\spX \to \spY$が
        \[ \Forall{ u \in \spX} \| Tu \|=\|u\| \]
        を満たすとき,$T$は\textbf{等長}であるという.
        等長な作用素は単射である.
    \end{Def}

    \begin{Def}
        Hilbert空間$\spX, \spY$について,
        $\dom T=\spX, \ran T=\spY$かつ等長な作用素$T:\spX \to \spY$を\textbf{ユニタリ作用素}と呼ぶ.
        $\spX$から$\spY$へのユニタリ作用素が存在するとき,
        $\spX$と$\spY$は\textbf{Hilbert空間として同型}であるという.
    \end{Def}

    \begin{Def}
        作用素$S,T$を写像と見た時,すなわち,どちらも空間全体で定義されている時の合成写像を\textbf{作用素の積}と呼ぶ.
        また,$ST, TS$のどちらも恒等写像であるとき,
        $S$と$T$は互いに\textbf{逆作用素}であると言い,$S=T^{-1}, T=S^{-1}$と書く.
    \end{Def}

    \begin{Def}
        2つの作用素$S,T:\spX \to \spY$が
        \[ \dom S \subseteq \dom T; \Forall{ u \in \dom S} Tu=Su \]
        を満たすとき,$T$は$S$の\textbf{拡張}である,または$S$は$T$の\textbf{縮少}であるという.
    \end{Def}

    \subsubsection{線形作用素}
    \begin{Def}
        作用素$T:\spX \to \spY$が
        \[ \Forall{ u,v \in \dom} \Forall{\alpha, \beta \in \C} T(\alpha u+\beta v)=\alpha Tu+\beta Tv \]
        を満たすとき,$T$は\textbf{線形作用素}であると言う.
    \end{Def}

    \begin{Def}
        2つの線形作用素$T:\spX \to \spY$と$S:\spY \to \mathscr{Z}$について,
        積$ST$を
        \[ (ST)u:=S(Tu);~ \dom ST=\{ u \in \dom T : Tu \in \dom S \} \]
        定める.この時,結合律が成り立つ.

        さらに$P, Q:\spX \to \spY$について
        和$P+Q$を以下のように定める.
        \[ (P+Q)u:=Pu+Qu; \dom P+Q=\dom P \cap \dom Q.  \]
        これについて,$(P_1+P_2)Q=P_1 Q+P_2 Q$は成り立つが,
        $P(Q_1+Q_2)=P Q_1+P Q_2$が成り立つとは限らない\footnote{教科書p.154}.

        スカラー倍$\alpha P$を以下で定める.
        \[ (\alpha T)u:=\alpha (Tu); \dom (\alpha T)=\dom T. \]
    \end{Def}
    次節で定める$B(\spX)$では,これらが環を成す.
    しかも$\spX$がBanach空間であれば,$B(\spX)$もBanach空間になる.

    \subsubsection{有界線形作用素}
    \begin{Def}
        線形作用素$T:\spX \to \spY$が
        \[ \Exists{M \geq 0} \Forall{u \in \dom T} \|Tu\|_{\spY} \leq M \|u\|_{\spX} \]
        を満たすとき,$T$は\textbf{有界}であるという.
    \end{Def}

    \begin{Def}
        ノルム空間$\spX$からノルム空間$\spY$への作用素$T$について,
        $u \in \dom T$において連続であるとは,
        \[ \Forall{\{ u_n\}_{n \in \N} \subset \dom T} \lim_{n \to \infty}u_n=u \implies \lim_{n \to \infty}T u_n=Tu  \]
        が成り立つことである.
        $T$が任意の$u \in \dom T$で連続であれば,単に$T$は\textbf{連続}であるという.
    \end{Def}
    $T$が有界ならば$\|T(u_n-u)\| \leq M\|u_n-u\| \to 0(n \to \infty)$より$T$は連続である.
    線形作用素ならば逆が成り立つことも言える.
    定理\ref{them7:1}参照.

    \begin{Def}
        ノルム空間$\spX(\neq \{0\})$からノルム空間$\spY$への\kenten{有界線形作用素}で,
        $\spX$全体で定義されているもの全体の集合を$B(\spX, \spY)$と書く.
        また,$B(\spX):=B(\spX, \spX)$とする.
    \end{Def}
    おそらく$B$はBoundedから来ているのだろう.
    定理\ref{them7:1}から,$B(\spX, \spY)$は連続な線形作用素の集合とも言える.
    以降では$B(\spX, \spY)$の元及び空間自体が主な考察対象となる.

    \begin{Def}
        $\spB{X}{Y}$の元$T$のノルムを
        \[
            \|T\|
            :=\sup_{\|u\| \leq 1}{\|Tu\|_{\spY}}
            =\sup_{\|u\|=1}{\|Tu\|_{\spY}}
            =\sup_{\|u\| \neq 0}{ \frac{\|Tu\|_{\spY}}{\|u\|_{\spX}} }
        \]
        で定める.
        これらが等しいことは$T$の線形性とノルムの定義から容易に示される.
        この時,任意の$u \in \spX$で$\|Tu\| \leq \|T\| \|u\|$
        \footnote{当然ながら$\|Tu\|_{\spY} \leq \|T\|_{\spB{X}{Y}} \|u\|_{\spX} $の事}
        が成り立つ.
    \end{Def}
    以上で定められた,有界線形作用素同士の演算とノルムについて,定理\ref{them7:6}が重要である.

    \begin{Def}
        $T_n (n=1,2,\dots),~ T \in \spB{X}{Y}$とする.
        \[ \|T_n-T\|_{\spB{X}{Y}} \to 0 ~(n \to \infty) \]
        が成り立つとき,
        $T_n$は$T$に\textbf{一様収束}する,あるいは\textbf{ノルム収束}するという.
    \end{Def}
    \begin{Def}
        $T_n (n=1,2,\dots),~ T \in \spB{X}{Y}$とする.
        \[ \|T_n u-T u\|_{\spY} \to 0 ~(n \to \infty) \]
        が成り立つとき,$T_n$は$T$に\textbf{強収束}すると言い,
        \[ \operatorname{s-lim}_{n \to \infty}{T_n}=T,~~ T_n \to T \mbox{(強)} \]
        などと書く.
        $T$は$T_n$の\textbf{強極限}という.これは存在すれば一意である.
    \end{Def}

    \subsubsection{閉作用素}
    \begin{Def}[閉作用素/1]
        線形作用素$T: \spX \to \spY$が以下の条件を満たすとき,
        $T$は\textbf{閉作用素}であるという.
        \begin{center}
            グラフ$\Gamma(T):=\{ (u, Tu) : u \in \spX \}$は$\spX \times \spY$の閉集合である.
        \end{center}
    \end{Def}
    これについては閉グラフ定理(定理\ref{them7:33})が重要である.
    上に述べたのは意味が取りやすい閉作用素の定義であるが,
    他に証明をする際に使われる定義が有る.
    \begin{Def}[閉作用素/2]
        線形作用素$T: \spX \to \spY$が以下の条件を満たすとき,
        $T$は閉作用素であるという.
        \begin{center}
            $\dom T$は\textbf{グラフ・ノルム}$\|u\|_{\spX}+\|Tu\|_\spY$に関して完備である.
        \end{center}
    \end{Def}

    \begin{Def}
        閉作用素であるような拡張を持つ作用素を\textbf{前閉作用素}と呼ぶ.
        また,その閉作用素であるような拡張を\textbf{閉拡大}と呼ぶ.
        最小 \footnote{定義域の包含関係について順序を定めている.} の閉拡大を\textbf{閉包}と呼ぶ.
    \end{Def}
    より詳しく,どのような特徴を持つ作用素が前閉作用素なのか,
    ということは定理\ref{them7:20}が明らかにしている.

    \subsubsection{共役空間}
    \begin{Def}
        $\spX^{\ast}=B(\spX, \C)$はすでに定めたノルム
        \[ \|f\|=\sup_{u \in \spX, \| u \| \neq 0}{ \frac{|f(u)|}{\|u\|} } \]
        によってBanach空間となる
        \footnote{$\spX^{\ast}$の元$f$による$u \in \spX$の像は$fu$でなく$f(u)$や$\langle u,f \rangle$と書く.}.
        $\spX^{\ast}$を$\spX$の\textbf{双対空間}あるいは\textbf{共役空間}と呼ぶ.
        また,$\spX^{\ast}$の元は\textbf{汎関数}と呼ばれる.
    \end{Def}
    \begin{Def}
        ノルム空間$\spX, \spY$について
        $T \in \spB{X}{Y}$と以下のような関係を持つ$T^{\ast} \in \spB{X^{\ast}}{Y^{\ast}}$を
        $T$の\textbf{共役作用素}と呼ぶ.
        \begin{align*}
            \|T^{\ast}\|_{\spB{X^{\ast}}{Y^{\ast}}}=\|T\|_{\spB{X}{Y}} \\
            \Forall{u \in \spX} \Forall{g \in \mathscr{Y^{\ast}}} (T^{\ast}g)(u)=g(Tu)
        \end{align*}
    \end{Def}
    \begin{Def}
        Banach空間$\spX$に対し,作用素$\kappa$を
        \begin{align*}
            \kappa: \spX \to \spX^{\ast \ast}     &;u \mapsto \phi_{u} \\
            \phi_u: \mathscr{X^{\ast}} \to \C \hspace{3truemm}&;f \mapsto f(u)
        \end{align*}
        のように定める.
        $\kappa$により$\spX$は$\spX^{\ast \ast}$の中に同型に埋め込まれる(定理\ref{them8:23}).
        $\ran \kappa=\spX^{\ast \ast}$であった時,
        すなわち$\kappa$が$\spX$から$\spX^{\ast \ast}$への同型写像となるとき,
        $\spX$を\textbf{反射的}あるいは\textbf{回帰的}であると言う.
    \end{Def}

    \subsection{例}
    % Fourier変換

    \subsection{定理・命題・補題・系}
    \begin{Them}[定理7.1, p.148] \label{them7:1}
        線形作用素$T$が連続ならば$T$は有界である.
        \footnote{線形作用素$T$は0で連続ならば定義域全体で連続.
        実際,任意の点$a$について,0へ収束する列$\{u_n\}$を元に$a$へ収束する列を$\{a+u_n\}$の様に作れる.
        $T$が0で連続であれば,$T$の線形性と三角不等式から$\|T(a+u_n)\| \leq \|Ta\|+\|Tu_n\| \to \|Ta\|~(n \to \infty)$.
        よって$T$は任意の$a$で連続.}
    \end{Them}

    \begin{Them}[定理7.6, p.150] \label{them7:6}
        $\spY$がBanach空間であるとき,$\spB{X}{Y}$はBanach空間になる.
    \end{Them}

    \begin{Them}[定理7.8, p.153] \label{them7:8}
        $T_n, T \in \spB{X}{Y}$について$\{T_n\}$が一様収束することは次の命題と同値である.
        \[ \Forall{\varepsilon>0} \Exists{N \in \N} \Forall{u \in \mathrm{UnitBall}} n>N \implies \| T_n u-T u \| \leq \varepsilon. \]
        ただし$\mathrm{UnitBall}$は$\spX$の閉単位球$\{u \in \spX ~|~ \|u\| \leq 1\}$である.
        すなわち,$\{T_n\}$は$\spX$の閉単位球上で一様収束する.
    \end{Them}

    \begin{Them}[定理7.20 (i), p.166] \label{them7:20}
        $T$が前閉作用素であるための必要十分条件は,
        \[ \Forall{\{ u_n \}_{n \in \N} \subset \dom T} \bigg[\Big[\lim_{n \to \infty}u_n=0  \land \lim_{n \to \infty}Tu_n=v \Big] \implies v=0 \bigg]. \]
    \end{Them}

    \begin{Them}[定理7.21, p.166, 一様有界性の原理] \label{them7:21}
        $\spX$を\kenten{Banach空間},$\spY$を\kenten{ノルム空間}であるとし,
        $\{ T_{\lambda} \}_{\lambda \in \Lambda} \subseteq \spB{X}{Y}$を作用素の族とする.
        この時,$\spX$の各点$u$で$\{ T_{\lambda}u \}_{\lambda \in \Lambda} \subset \spY$が
        有界ならば,$\{ T_{\lambda} \}_{\lambda \in \Lambda}$は一様に有界である.
    \end{Them}
    これはthree basic principlesの第1のもの.

    \begin{Them}[定理7.23, p.167, Baireのカテゴリー定理] \label{them7:23}
        $\spX$を\kenten{完備な距離空間}であるとする.
        高々加算個の$\spX$の閉集合$\spX_n ~(n=1,2,\dots)$が
        $\spX$を覆う($\spX=\bigcup_{n=1}^{\infty}{\spX_n}$)ならば,
        少なくとも1つの$\spX_n$は$\spX$の開球を含む.
    \end{Them}
    これを元に次が示される.

    \begin{Them}[定理7.30, p.170, 開写像原理] \label{them7:30}
        $\spX, \spY$をBanach空間とし,$T \in \spB{X}{Y}$とする.
        もし$\ran T=\spY$ならば,$T$は開写像である.
    \end{Them}
    これはthree basic principlesの第2のもの.

    \begin{Them}[定理7.33, p.172, 閉グラフ定理] \label{them7:33}
        $\spX, \spY$はBanach空間,
        $T$は$\spX$から$\spY$への閉作用素とする.
        この時,$\dom T=\spX$ならば$T \in \spB{X}{Y}$.
    \end{Them}

    \begin{Cor}[系7.34, p.172] \label{cor7:34}
        $\spX, \spY$はBanach空間,
        $T$は$\spX$から$\spY$への閉作用素とする.
        $T$が1対1かつ$\ran T=\spY$ならば$T^{-1} \in \spB{X}{Y}$
    \end{Cor}

%    \begin{Them}[定理7., p.] \label{them7:}
%    \end{Them}

    \begin{Them}[定理8.3, p.176] \label{them8:3}
        $\spX$をノルム空間,$f \in \spX^{\ast}$とするとき,次のことが成り立つ.
        \begin{enumerate}[i)]
            \setlength{\leftskip}{5truemm}
            \item $\ker f=\{ u \in \spX : f(u)=0 \}$は$\spX$の閉部分空間.
            \item $f\neq 0$とし,$u_0 \not \in \ker f$とすると,任意の$u \in \spX$は
                  \[ u=u'+\alpha u_0 ~~ u' \in \ker f,~ \alpha \in \C \]
                  と一意に表される.ここで$\alpha$は$\alpha = f(u)/f(u_0)$で与えられる.
              \item $\spX$が\kenten{Hilbert空間}で$f \neq 0$ならば,$(\ker f)^{\perp}$は1次元である.
        \end{enumerate}
    \end{Them}
    \begin{Them}[定理8.5, p.177, Rieszの表現定理] \label{them8:5}
        $\mathscr{H}$をHilbert空間とすると,任意の$f \in \mathscr{H}^{\ast}$は
        有る$v \in \mathscr{H}$によって$f_v(\cdot)=(\cdot,v)$と表される.
        $v$は$f$によって一意に定まる.
    \end{Them}
    このことからHilbert空間が反射的である($\mathscr{H} \simeq \mathscr{H}^{\ast\ast}$)ことが示される.

    Rieszの表現定理は$\mathscr{H}^{\ast}$が十分広いことも言っている.
    Banach空間$\spX$の双対空間$\spX^{\ast}$の場合,
    1次元の部分空間を作ることは簡単に出来る
    \footnote{p.181参照.適当に$u_0 \in \spX, c \in \C$をとり,$f(\alpha u_0)=c \alpha$とする.}.
    しかし$\spX^{\ast}$が十分広い空間であることはまったく自明でない.
    これは以下の定理によって示される.
    \begin{Them}[定理8.11, p.182, Hahn-Banachの拡張定理] \label{them8:11}
        $\spX$を\kenten{実線形空間}とする.
        $p:\spX \to \R$は(線形とは限らない)汎関数とし,以下を満たすとする.
        \begin{enumerate}[i)]
            \setlength{\leftskip}{5truemm}
        \item $\Forall{u, v \in \spX} (p(u+v) \leq p(u)+p(v))$ 
        \item $\Forall{x \in \spX} \Forall{\alpha \in \R_{\geq 0}} (p(\alpha x) = \alpha p(x))$
        \end{enumerate}
        この条件はまとめて劣線形性と呼ばれる.
        $f$は$\spX$の部分空間$\mathscr{M}$で定義された線形汎関数で,
        \[ \Forall{u \in \mathscr{M}} (f(u) \leq p(u)) \]
        を満たすものとする.
        その時$f$はこの不等式と線形性を保ったまま,$\spX$全体に拡張される.
    \end{Them}
    これはthree basic principlesの第3のもの.
    複素線形空間でも同様の定理が成立する.

    \begin{Them}[定理8.13, p.184] \label{them8:13}
        $\spX$を\kenten{複素線形空間}とする.
        $p:\spX \to \R$は(線形とは限らない)汎関数とし,以下を満たすとする.
        \begin{enumerate}[i)]
            \setlength{\leftskip}{5truemm}
            \item $p(u) \geq 0$
            \item $\Forall{ u, v \in \spX} (p(u+v) \leq p(u)+p(v))$ 
            \item $\Forall{ x \in \spX} \Forall{\alpha \in \R_{\geq 0}} (p(\alpha x) = \alpha p(x))$
        \end{enumerate}
        この性質を持つ$p$を$\spX$上のセミノルム(semi-norm)と言う.
        $f$は$\spX$の部分空間$\mathscr{M}$で定義された線形汎関数で,
        \[ \Forall{u \in \mathscr{M}} (0 \leq |f(u)| \leq p(u)) \]
        を満たすものとする.
        その時$f$はこの不等式と線形性を保ったまま,$\spX$全体に拡張される.
    \end{Them}

    \begin{Them}[定理8.23, p.189] \label{them8:23}
        Banach空間$\spX$に対し,作用素$\kappa$を
        \begin{align*}
            \kappa: \spX \to \spX^{\ast \ast}     &;u \mapsto \phi_{u} \\
            \phi_u: \mathscr{X^{\ast}} \to \C \hspace{3truemm}&;f \mapsto f(u)
        \end{align*}
        のように定める.
        $\kappa$により$\spX$は$\spX^{\ast \ast}$の中への等長な線形作用素である.
        したがって,$\spX$は$\spX^{\ast \ast}$の部分空間とみなせる.
    \end{Them}

%    \begin{Them}[定理8., p.] \label{them8:}
%    \end{Them}

