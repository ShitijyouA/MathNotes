\section{第11章 / コンパクト作用素,Fredholm作用素}
    \subsection{定義}
    \subsubsection{直和分解と補空間}
    \begin{Def}
        $\spX$をBanach空間,$\spM, \spN$を$\spX$の\kenten{閉部分空間}とする.
        $\spM \cap \spN=\{0\}$であるとき,
        \[ \spM \oplus \spN=\set{m+n}{m \in \spM, n \in \spN}. \]
        とし,これを$\spM$と$\spN$の\textbf{直和}と呼ぶ.
        Banach空間$\spX$が2つの閉部分空間$\spM, \spN$の直和であるとき,
        $\spX$は$\spM$と$\spN$に\textbf{直和分解}されると言い,
        $\spM, \spN$は互いに\textbf{補空間}であると言う.
    \end{Def}
    $\spM \oplus \spN$の元を一つとって$\spM, \spN$の元の和に分解するとき,
    その分解の仕方は一意である.
    $\spX$はHilbert空間ならば,
    任意の閉部分空間$\spM$は補空間$\spM^{\perp}$を持つ.
    $\spX$がBanach空間である時の補空間の存在については定理\ref{them1103}で,
    一意性については定理\ref{them1102}で述べられる.
    以上の定義は2個以上の空間の直和へ一般化される.
    \begin{Def}
        $\spX$をBanach空間,$\spM_1,\dots,\spM_n$をその部分空間とする.
        これらが以下の条件を満たすとする.
        \[ u_k \in \spM_k,~~ u_1+\dots+u_n=0 \implies u_1=\dots=u_n=0. \]
        このとき$\spM_1,\dots,\spM_n$の直和を以下で定める.
        \[
            \spM_1 \oplus \dots \oplus \spM_n
            =\set{u_1+\dots+u_n}{\Forall{k} u_k \in \spM_k}.
        \]
    \end{Def}
    前提条件から,
    $\spM \oplus \spN$の元一つとって$\spM, \spN$の元の和に分解するとき,
    その分解の仕方は一意である.

    次の集合は教科書p.197で一度定義されたもので,Fredholm作用素の定義にも現れる.
    \begin{Def}
        Banach空間$\spX$の部分集合$\spM$に対して$\spM^{\perp}$を以下で定める.
        \[ \spM^{\perp}=\set{f \in \spX^*}{\Forall{u \in \spM} f(u)=0}. \]
    \end{Def}

    \subsubsection{コンパクト作用素}
    \begin{Def}
        $\spX, \spY$をBanach空間,
        $K \in B(\spX, \spY), \dom(K)=\spX$とする.
        $\spX$の任意の有界点列$\{u_n\}$に対して,
        $\{Ku_n\}$が$\spY$で収束する部分列を持つとき,
        $K$を\textbf{コンパクト作用素}(または完全連続作用素)と呼ぶ.
    \end{Def}
    教科書では最初コンパクト作用素に有界であることを求めないが,
    教科書の定理11.10より,そのように定義したコンパクト作用素も有界である.
    次の定理はコンパクト作用素の別の定義として使える.
    \begin{Def}[定理11.9, p.257]
        $\spX, \spY$をBanach空間,
        $K \in B(\spX, \spY), \dom(K)=\spX$とする.
        $\spX$の任意の有界集合$\spM$について$\overline{K \spM}$が
        $\spY$のコンパクト集合であるとき,
        $K$を\textbf{コンパクト作用素}と呼ぶ.
    \end{Def}
    $\spX$から$\spY$へのコンパクト作用素全体を$B_0(\spX,\spY)$と書く.
    \begin{Def}
        $F \in B(\spX, \spY)$について値域$\ran F$が有限次元であるとき,
        $F$を\textbf{有限次元作用素}と呼ぶ.
    \end{Def}
    有界な部分空間$\spM$について$\overline{F \spM}$は
    $\dim \ran F<\infty$ならば有界閉集合になる.
    よって有限次元作用素はコンパクト作用素である.

    \subsubsection{Fredholm作用素}
    \begin{Def}
        $\spX, \spY$をBanach空間,$T \in B(\spX, \spY)$とする.
        $T$が以下の3条件を満たすとき,$T$を\textbf{Fredholm作用素}と呼ぶ.
        \begin{enumerate}
            \item $\dim \ker T < \infty$.
            \item $\dim \ker T^* < \infty$.
            \item $\ran{T} \subset \spY$は閉部分空間.
        \end{enumerate}
    \end{Def}
    $\spX$から$\spY$へのFredholm作用素全体[2]を$F(\spX,\spY)$で表す.
    \begin{Def}
        $T \in F(\spX,\spY)$に対して\textbf{指数}$\ind T$を,
        \[ \ind T=\dim \ker T-\dim \ker T^* \]
        で定義する.
    \end{Def}
    教科書の定理8.43 i) (p.197)より,
    $\dim \ker T^*=\dim (\ran T)^{\perp}:=\codim \ran T$が成り立つ.

    \subsubsection{自己共役なコンパクト作用素}
%    \begin{Def}
%        Hilbert空間$\spX$の有界線形作用素$T \in B(\spX)$に対して,
%        Hilbert共役作用素を$T^{\star}$ \footnote{$(Tu,v)=(u,T^{\star}v)$なるもの.}とする.
%        $T=T^{\star}$であるとき,$T$は\textbf{自己共役}であるという.
%    \end{Def}

    \subsection{例(前半)}
    \subsubsection{コンパクト作用素}
    \begin{Example}[問, p.258]
        $\spX=l^p, 1 \leq p<\infty, a_n \in C, a_n \to 0$とする.
        $K \in B(l^p)$を$K(u_1,u_2,\dots)=(a_1u_1,a_2u_2,\dots)$で定義する.
        $\|Ku\| \leq \sup_i|a_i| \cdot \|u\|$なので確かにこれは有界作用素.
        この時,$\{u^{(i)}\}$を$M$を上限とする有界点列とすると,
        $\|K u^{(i)}\| \leq \|K\|M$なので像も有界点列.
        よって$K$はコンパクト作用素である.
        後にこれがコンパクトであることの別証明を与える.
    \end{Example}

    \newpage
    \subsection{定理・命題・補題・系}
    \subsubsection{直和分解と補空間}
    この節では$\spX$をBanach空間,$\spM,\spN$をその\kenten{閉部分空間}とする.
    \begin{Them}[定理11.2, p.252] \label{them1102}
        $\spM$が補空間を持つならば,
        それらはBanach空間としての同型を除いて一意.
    \end{Them}
    \begin{Them}[定理11.3, p.253] \label{them1103}
        $\spM$は有限次元ならば補空間を持つ.
    \end{Them}
    \begin{Them}[定理11.4, p.253] \label{them1104}
        $\spM^{\perp}$が有限次元であることと,$\spM$は有限次元な補空間$\spN$を持つことは同値.
        しかもその時$\dim \spM^{\perp}=\dim \spN$.
    \end{Them}
    \begin{Them}[定理11.7, p.255] \label{them1107}
        $\spM,\spN$が閉部分空間であり,$\spN$は有限次元であるとする.
        この時,$\spM \cap \spN=\{0\}$ならば
        \footnote{つまり$\spM \oplus \spN$が存在すれば.}
        $\spM \oplus \spN$も閉部分空間である.
    \end{Them}

    \subsubsection{コンパクト作用素}
    この節では$\spX, \spY$をBanach空間とする.
    \begin{Prop}[p.257,p.261] \label{prop-cmp}
        任意の$K \in B(\spX,\spY)$は$\ran K$または$\spY$が有限次元ならばコンパクトである.
        一方,$\spX$が無限次元ならば恒等作用素$I$はコンパクトでない.
    \end{Prop}
    \begin{Them}[定理11.12, p.257] \label{them1112}
        $B_0(\spX,\spY)$は$B(\spX,\spY)$の閉部分空間である.
    \end{Them}
    \begin{Them}[定理11.13, p.258] \label{them1113}
        $S \in B(\spX, \spY), T \in B(\spY,\spZ)$とする.
        $S,T$のどちらか一方でもコンパクトであれば$ST$もコンパクトである.
    \end{Them}
    \begin{Them}[Schauderの定理, 定理11.15, p.258] \label{them1115}
        $K \in B(\spX,\spY)$について,以下が成り立つ.
        \[ K \in B_0(\spX,\spY) \iff K^* \in B_0(\spX^*,\spY^*). \]
    \end{Them}

    \subsubsection{コンパクト作用素のスペクトル理論}
%    $T \in F(\spX), \zeta \neq 0$について
%    $\zeta I-T \in F(\spX), \ind \zeta I-T=0$が成り立つことに注意しておく.
%    これらは教科書のp.268で述べられている.
    \begin{Them}[定理11.29, p.269] \label{them1129}
        $K \in B_0(\spX)$のスペクトル$\sigma(K)$について以下が成り立つ.
        \begin{enumerate}[i)]
            \item $\sigma(K)=\sigma_p(K) \mor \sigma_p(K) \cup \{0\}$.\footnote{$0 \in \sigma_p(K)$はあり得る.}
            \item $\dim \spX=\infty$ならば$\sigma(K)=\sigma_p(K) \cup \{0\}$.
            \item $\sigma_p(K)$は高々加算な集合$\{\zeta_k\}$.
            \item $\sigma_p(K)=\{\zeta_k\}$が加算集合ならば$\lim_{k \to \infty}\zeta_k=0$.
            \item 各$\zeta_k$の多重度は有限.
            \item 各$\zeta_k$は$K^*$の固有値でもある.
            \item 各$\zeta_k$の$K$の固有値としての多重度は$K^*$の固有値としての多重度に等しい.
        \end{enumerate}
    \end{Them}

    \subsubsection{Fredholm作用素}
    \begin{Them}[定理11.20, p.262] \label{them1120}
        $\spX$をBanach空間とし,$K \in B_0(\spX), T=I-K$とする.
        この時$T$はFredholm作用素である.
    \end{Them}
    \begin{Them}[定理11.24, p.264] \label{them1124}
        $T \in B(\spX,\spY)$がFredholm作用素であるための必要十分条件は,
        以下が成り立つこと.
        \begin{align*}
            &\Big[\Exists{A_1 \in B(\spY,\spX), K_1 \in B_0(\spX)} A_1 T+K_1=I \Big] \\
            &\land \\
            &\Big[\Exists{A_2 \in B(\spY,\spX), K_2 \in B_0(\spY)} T A_2+K_2=I \Big]
            .
        \end{align*}
    \end{Them}
    この定理は大雑把に言えば「Fredholm作用素とはコンパクト作用素の違いを無視すれば可逆なもの」ということを言っている.
%    実際,Masamichi Takesaki ``Theory of OPerator Algebra I''の
%    p.55で述べられているFredholm作用素の定義は次のようになっている.
%    「
%    $C^*$代数として$B(\spX), B_0(\spX)$を見て,標準的全射$\pi: B(\spX) \to B(\spX)/B_0(\spX)$を定める.
%    $\spX$のFredholm作用素とは,$\pi(K)$が可逆であるような$K \in B(\spX)$.
%    」

    \begin{Them}[定理11.25, p.265] \label{them1125}
        $S \in F(\spX, \spY), T \in F(\spY,\spZ)$とする.
        この時$ST \in F(\spX,\spZ)$であり,指数について$\ind ST=\ind S+\ind T$となる.
    \end{Them}

%    \begin{Them}[定理11., p.2] \label{them11}
%    \end{Them}

    \subsubsection{自己共役なコンパクト作用素}

    \subsection{例(後半)}
    \subsubsection{コンパクト作用素}
    \begin{Example}[問, p.258]
        (前半で与えた主張の別証明.)
        $\spX=l^p, 1 \leq p<\infty, a_n \in C, a_n \to 0$とする.
        $K \in B(l^p)$を$K(u_1,u_2,\dots)=(a_1u_1,a_2u_2,\dots)$で定義する.
        $K_n(u_1,u_2,\dots)=(a_1u_1,a_2u_2,\dots,a_nu_n,0,0,\dots)$と定めると,
        明らかに$\dim \ran K_n=n<\infty$で,しかもノルム収束の意味で$K_n \to K$.
        したがって命題\ref{prop-cmp}と定理\ref{them1112}より,
        $K$はコンパクト作用素である.
    \end{Example}
