\documentclass[a4j]{jsarticle}
\usepackage{../../math_note}

\newcommand{\codim}{\operatorname{codim}}
\newcommand{\dom}{\operatorname{dom}}
\newcommand{\ran}{\operatorname{ran}}
\newcommand{\ind}{\operatorname{ind}}

\newcommand{\spX}{\mathscr{X}}
\newcommand{\spY}{\mathscr{Y}}
\newcommand{\spZ}{\mathscr{Z}}
\newcommand{\spM}{\mathscr{M}}
\newcommand{\spN}{\mathscr{N}}
\newcommand{\spB}[2]{B(\mathscr{#1}, \mathscr{#2})}
\newcommand{\spBX}{B(\spX)}

\newcommand{\set}[2]{\left\{#1~\middle|~#2\right\}}

\begin{document}
    これは黒田成俊著「関数解析」(以下,教科書)の内容を
    抜き出し,一部補完して流れを明確にしたノート/cheat-sheetである.

    \section{第9章 / レゾルベント・スペクトル}
    \subsection{レゾルベント・スペクトルを考える動機}
        $\spX$をBanach空間,$I$を$\spX$上の恒等作用素,$T \in B(\spX)$とする.
        次のような$u$についての方程式を考えよう.
        \[ \zeta u-Tu=v ~~~(u,v \in \spX). \label{eq905}\]
        これは形式的に$v=(\zeta I-T)^{-1} u$と解くことが出来る.
        さらに形式的に,$(\zeta I-T)^{-1}$は$\sum_{k=0}^{\infty}T^k$と変形出来るだろう.
        $T^k$は既知だから,この変形が正当化できさえすれれば
        上のような形の方程式は解けたことになり,大変嬉しい.
        そして実際に,$\|T\|<|\zeta|$ならば上の形式的解法が
        正当化出来ることが定理\ref{them902}よりわかる.
        また,$T$が有界でない場合でも$(\zeta I-T)^{-1}$が有界になる場合は多い.
        なので有界作用素の理論を用いて非有界作用素$T$を調べられることが出来る.

    \subsection{定義}
    \begin{Prop}[p.210]
        $T$はBanach空間$\spX$の線形作用素とする.
        $(\zeta I-T)^{-1}$が$B(\spX)$ \footnote{$\spX$上の有界線形作用素.}の元になり,
        しかも1対1になるような$\zeta \in \C$が存在するためには,
        $T$が閉作用素であることが必要である.
    \end{Prop}
    \begin{proof}
        教科書の定理7.16 i)より$(\zeta I-T)^{-1}$は閉で,
        定理7.17より$\zeta I-T$も閉.
        よって定理7.18より$T=(\zeta I)-(\zeta I-T)$は閉.
    \end{proof}
    なのでスペクトル理論の興味の対象は閉作用素に限る.
    \begin{Def}
        $T$はBanach空間$\spX$の\kenten{閉線形作用素}とする.
        \begin{description}
            \item[レゾルベント集合$\rho(T)$] \mbox{}\\
                $\rho(T)=\{ \zeta \in \C ~|~ (\zeta I-T) \text{が1対1,かつ}(\zeta I-T)^{-1} \in B(\spX). \}$
                \footnote{$\rho$はresolventのr.}

            \item[レゾルベント$R(\zeta;T)$] \mbox{}\\
                $\zeta \in \rho(T)$について$R(\zeta;T)=(\zeta I-T)^{-1}$.

            \item[スペクトル $\sigma(T)$] \mbox{}\\
                $\sigma(T)=\C \setminus \rho(T)$.
                \footnote{$\sigma$はspectrumのs.}

            \item[スペクトル半径$r(T)$] \mbox{}\\
                $T \in B(\spX)$に対して,$r(T)=\sup_{\zeta \in \sigma(T)}|\zeta|$.
                \footnote{定理\ref{them902}から$r(T) \leq \|T\|$が成り立つ.}

            \item[固有値]\mbox{}\\
                $(\zeta I-T)$が1対1でないような$\zeta \in \C$を$T$の固有値と呼ぶ.

            \item[点スペクトル$\sigma_p(T)$]\mbox{}\\
                $T$の固有値全体を$\sigma_p(T)$で表す.

            \item[固有値$\zeta$に属する固有ベクトル・固有空間]\mbox{}\\
                $B(\spX)$の閉部分空間$\{u \in \spX ~|~ Tu=\zeta u. \}$を固有値$\zeta$に属する固有空間と呼び,
                その元を固有値$\zeta$に属する固有ベクトルと呼ぶ.

            \item[固有値$\zeta$の多重度] \mbox{}\\
                固有値$\zeta$に属する固有空間の次元を固有値$\zeta$の多重度と呼ぶ.
        \end{description}
    \end{Def}

    $T \in B(\spX)$について,$r(T)=0$であるとき$T$を\textbf{準ベキ零作用素}と呼び,
    ある$n \in \N$について$T^n=0$となるとき$T$を\textbf{ベキ零作用素}と呼ぶ.
    後に証明することとして,以下の等式がある(\ref{them912}).
    \[ r(T)=\lim_{n \to \infty} \|T^n\|^{1/n}=\inf_{n \in \N} \|T^n\|^{1/n}. \]
    この等式から2つの概念は一致するように思われる.
    しかし$r(T)$は$\sup$を取るという極限操作によって定まっており,
    実際には$\spX$が無限次元の時一致しない.
    一方有限次元の時には一致することが確かめられる.


    \subsection{定理・命題・補題・系}
    \subsubsection{Neumann級数}
    教科書の定理7.16 i) (p.165)より,$B(\spX)$の元は閉作用素であることに注意.

    \begin{Them}[定理9.2, p.209] \label{them902}
        $\spX$はBanach空間,$T \in B(\spX), \zeta \in \C$とする.
        $|\zeta| > \|T\|$ならば$\zeta \in \rho(T)$であり,
        以下が成り立つ.
        \begin{align}
            R(\zeta; T)=&\sum_{k=0}^{\infty} \zeta^{-(k+1)} T^k \label{eq906}\\
            \|R(\zeta; T)\| \leq& |\zeta|^{-1} (1-|\zeta|^{-1} \|T\|)^{-1}
        \end{align}
        ここで式(\ref{eq906})の右辺は$B(\spX)$で絶対収束する.
        特に方程式$\zeta u-Tu=v$は一意的に解$u=R(\zeta;T)v$を持つ.
    \end{Them}
    式(\ref{eq906})の右辺はNeumann級数と呼ばれる.
    例で見るようにこの定理は$(\zeta I-T)^{-1}$がNeumann級数で表せることの十分条件を示しているに過ぎない.
    証明も$\sum_{k=0}^{\infty} \zeta^{-(k+1)} T^k$が存在するならば両辺が一致する,
    というものになっている.
    また,この定理の前半(と$T$が有界作用素であること)から,
    $\rho(T)$は$\{ \zeta ~|~ |\zeta| > \|T\| \}$を\kenten{含む}ことがわかる.

    \subsubsection{レゾルベント方程式と正則性}
    この節では$T$を$\spX$の\kenten{閉線形作用素}とする.
    まず$\zeta \in \rho(T)$ならば
    $\zeta I-T$は1対1なので$\dom(T)$を$\spX$全体に写す.
    したがって$R(\zeta;T):\spX \to \dom(T)$である.

    \begin{Them}
        $\zeta_1, \zeta_2 \in \rho(T)$について,以下が成り立つ.
        \begin{align*}
            R(\zeta_2;T)-R(\zeta_1;T)
            =&(\zeta_1-\zeta_2)R(\zeta_2;T)R(\zeta_1;T) \\
            =&(\zeta_1-\zeta_2)R(\zeta_1;T)R(\zeta_2;T)
        \end{align*}
    \end{Them}
    これを(第一)\textbf{レゾルベント方程式}と呼ぶ.
    \footnote
    {
        第二レゾルベント方程式は$R(\zeta;S)-R(\zeta;T)=R(\zeta;S)(S-T)R(\zeta;T)$(演習問題6).
        証明は第一レゾルベント方程式を真似れば良い.
    }
    この等式から$R(\zeta_1;T), R(\zeta_2;T)$が可換であること,
    及び$\zeta_1 \neq \zeta_2$ならば
    \[ \frac{R(\zeta_2;T)-R(\zeta_1;T)}{\zeta_2-\zeta_1}=-R(\zeta_2;T)R(\zeta_1;T)=-R(\zeta_1;T)R(\zeta_2;T). \label{eq-ori901} \]
    となることが得られる.
    すぐさま$\zeta_1 \to \zeta_2$として「微分」したくなるが,
    その極限が存在することは自明でなく,次の定理で述べられる.

    \begin{Them}[定理9.5, p.211]
        $\zeta_0 \in \rho(T)$をとり,$R(\zeta):=R(\zeta;T)$と略記する.
        このとき以下が成り立つ.
        \[
            \mathrm{Disc} \left(\zeta_0, \|R(\zeta_0)\|^{-1} \right)
            =\left\{ \zeta \in \C ~\middle|~ |\zeta-\zeta_0|<\|R(\zeta_0)\|^{-1} \right\}
            \subset \rho(T).
        \]
        この円盤の中では$R(\zeta)$は次のようにべき級数展開される.
        \[ R(\zeta)=\sum_{k=0}^{\infty} (-1)^kR(\zeta_0)^{k+1} \cdot (\zeta-\zeta_0)^k. \]
        また,この等式で両辺のノルムを評価することで,以下が得られる.
        \[
            \|R(\zeta)\|
            \leq \|R(\zeta_0)\| \sum_{k=0}^{\infty}\left( |\zeta-\zeta_0|\|R(\zeta_0)\| \right)^k
            =\|R(\zeta_0)\| (1-|\zeta-\zeta_0|\|R(\zeta_0)\|)^{-1}.
        \]
    \end{Them}
    定理の最初から一般の閉線形作用素$T$について$\rho(T)$が開集合であることがわかる.
    そして最後の不等式から$R(\zeta;T)$の「微分」$\frac{d}{d \zeta}R(\zeta;T):=\lim_{h \to 0}\frac{1}{h} (R(\zeta+h;T)-R(\zeta;T))$
    が存在することが示される.

    \begin{Them}[定理9.6, p.212]
        $R(\zeta;T)$は$\rho(T)$で正則,
        すなわち$\frac{d}{d \zeta}R(\zeta;T)$が存在し,次の等式が成り立つ.
        \[ \frac{d}{d \zeta}R(\zeta;T)=-R(\zeta;T)^2. \]
    \end{Them}

    \subsubsection{スペクトル半径}
    この節では$T$を$\spX$上の\kenten{有界線形作用素}とする.
    定理\ref{them902}によると,$\rho(T)$は空でない.
    実は次も成り立つ.
    \begin{Them}[定理9.8, p.213]
        $T \in B(\spX)$ならばスペクトル$\sigma(T)$は空でない.
    \end{Them}
    したがってスペクトル半径$r(T)=\sup_{\zeta \in \sigma(T)}|\zeta|$は存在する.
    スペクトル半径について,以下の等式が成り立つ.
    \begin{Them}[定理9.12, p.215] \label{them912}
        $\lim_{n \to \infty} \|T^n\|^{1/n}$が存在し,以下が成り立つ.
        \[ r(T)=\lim_{n \to \infty} \|T^n\|^{1/n}=\inf_{n \in \N} \|T^n\|^{1/n}. \]
    \end{Them}

    \subsubsection{双対作用素のレゾルベント}
    $T$が閉作用素ならば,教科書の定理8.42より共役作用素$T^*$も閉作用素.
    なので$\rho(T^*)$が考えられる.
    \begin{Them}[定理9.9, p.213] \label{them909}
        $T$が$\spX$上の閉作用素で,$\dim(T)$が$\spX$で稠密であるとする.
        この時,まず以下が成り立つ.
        \[ \rho(T^*)=\rho(T), R(\zeta;T^*)=R(\zeta;T)^*. \]
        さらに$\spX$がHilbert空間ならば通常Hilbert space adjoint(p.201参照)$T^{\star}$を
        共役作用素として扱うが,これについては以下が成り立つ.
        \[ \rho(T^{\star})=\{ \zeta ~|~ \bar{\zeta} \in \rho(T)\}, R(\zeta;T^{\star})=R(\bar{\zeta};T)^{\star}. \]
    \end{Them}

    \subsubsection{擬レゾルベント}

    \subsection{例}
        \subsubsection{(準)ベキ零作用素}
        \begin{Example}[例9.14, p.216]
            $l^p$の元$u=(u_1,u_2,\dots)$に対して
            $Tu=(2^{-1}u_2, 3^{-1}u_3, \dots)$と定める.
            直ちに以下が得られる.
            \[ T^n u=\left( \frac{1}{(n+1)!} u_{n+1}, \frac{2!}{(n+2)!} u_{n+2}, \dots \right). \]
            $l^p$のノルムは$\sup$ノルムなので$\|T^n\|=\frac{1}{(n+1)!}$.
            定理\ref{them912}より$r(T)=0$.
            しかし明らかに任意の$n$について$T^n \neq 0$なので,
            $T$は準ベキ零作用素だがベキ零作用素でない.
        \end{Example}

        \begin{Example}[問題9.2, p.223]
            $k(x,y)$を正方形$[a,b]^2 (-\infty<a<b<\infty)$上定義された連続関数とする.
            そして作用素$T \in B(C[a,b])$を以下で定める.
            \[ (Tu)(x):=\int_a^x k(x,y)u(y) dy \mwhere u \in C[a,b],~ x \in [a,b]. \]

            \paragraph{(i)}
            $M=\sup_{(x,y) \in [a,b]^2}|k(x,y)|$とすると,
            \footnote{教科書では$M=\sup_{(x,y) \in [a,1]^2}|k(x,y)|$となっているが明らかに誤植である.}
            $|(T u)(x)| \leq M \|u\| \int_a^x dy=M \|u\| (x-a)$.
            帰納的に$|(T^n u)(x)| \leq \frac{M^n}{n!} (x-a)^n \|u\|$が示されるので,
            $\|T\| \leq \frac{M^n}{n!} (b-a)^n \to 0 ~~(n \to \infty)$.
            しかし明らかに$T^n \neq 0$なので,これも準ベキ零作用素だがベキ零作用素でない.
            
            \paragraph{(ii)}
            $\sum_{k=0}^{\infty}T^k$は
            $\leq \sum_{k=0}^{\infty} \|T^k\|=\exp(M(a-b))<\infty$より,絶対収束する.
            なので$(I-T)^{-1}$は定理9.1 (p.209)の証明から$\sum_{k=0}^{\infty}T^k$で表せて,
            方程式$(I-T)u=f$は任意の$f \in C[a,b]$について一意的な解$u=(I-T)^{-1}f$を持つ.
        \end{Example}

        \subsubsection{レゾルベント・スペクトル}
        \begin{Example}[例9.15, p.217]
            $\dim \spX=n < \infty,T \in B(\spX)$とする.
            $\spX$の基底をひとつ取ると,$T$は$n \times n$行列$\tilde{T}$で表示できる.
            この時$\zeta I-T$が1対1であることと$(\zeta I-T)^{-1} \in B(\spX)$は同値.
            $\zeta I-T$が1対1でないことは
            $\zeta I-\tilde{T}$が可逆でないことと同値であることがわかるので,
            $\sigma(T)=\{\zeta ~|~ \det(\zeta I-\tilde{T})=0 \}=\text{行列$\tilde{T}$の固有値全体.}$
        \end{Example}

        \begin{Example}[例9.17, p.217]
            $\spX=l^p, 1 \leq p \leq \infty$とする.
            作用素$S(u_1,u_2,\dots)=(u_2,u_3,\dots)$を考えよう.
            これは明らかに非可逆.
            $\|S\|=1$なので,スペクトルは円盤$|\zeta| \leq 1$に含まれる.

            まず固有値を調べよう.
            $S u=\zeta u$の両辺で成分を見ると$u_{n+1}=\zeta u_{n}$なので$u_n=\zeta^{n-1} u_1$.
            したがって$|\zeta|<1$の時$\zeta$は固有値で,付随する固有空間は$\{(t, \zeta t, \dots) ~|~ t \in \C \}$.
            一方$|\zeta|=1$の時は$|u_n|=|u_1|$.
            故に$p<\infty$の時は同じように固有ベクトルを作っても$l^p$の元にならず,$p=\infty$ならば$l^p$の元になる.
            よって$\sigma_p(S)$は
            $1 \leq p < \infty$ならば$\{ |\zeta| < 1 \}$,
            $p=\infty$ならば$\{|\zeta| \leq 1\}$である.

            さらに,以上から$\{|\zeta| \leq 1\}$とレゾルベント集合は交わらない.
            よって$p$によらず$\sigma(S)=\{|\zeta| \leq 1\}$.
            (あるいは,$\sigma(S)=(\rho(S))^c$は$\sigma_p(S)$を含む閉集合であることを用いてもわかる.)

            $1 \leq p < \infty$の時は共役作用素も考えられる.
            p.202より$S^* (u_1,u_2,\dots)=(0,u_1,u_2,\dots)$.
            まず定理\ref{them909}より$\sigma(S^*)=\sigma(S)$.
            固有値を$S$と同様にして考えると,
            $u_{n-1}=\zeta u_n, 0=\zeta u_1$となるので,$\zeta=0$または$u_1=u_2=\dots=0$が必要になる.
            しかも$\zeta=0$なら$\zeta I-T^*=T^*$で,これは明らかに1対1.
            よって$\sigma_p(S)=\emptyset$.
        \end{Example}


        \section{第1,7,8章/作用素}
    \subsection{定義}
    \subsubsection{一般の作用素}
    \begin{Def}
        線形空間$\spX$の部分集合$\dom$から
        線形空間$\spY$への写像$T$を,
        $\spX$から$\spY$への\textbf{作用素}と呼ぶ.
        $\dom$は$T$の定義域と呼ばれ,$\dom(T)$で表す.
        $\{v \in \spY : \Exists{u \in \dom(T)} (v=Tu)\}$
        は$T$の値域と呼ばれ,$\range(T)$で表す.
    \end{Def}
    一般に,\kenten{作用素はその原像全体で定義されているとは限らない}.

    \begin{Def}
        ノルム空間$\spX, \spY$について,
        線形作用素$T:\spX \to \spY$が
        \[ \Forall{ u \in \spX} (\| Tu \|=\|u\|) \]
        を満たすとき,$T$は\textbf{等長}であるという.
        等長な作用素は単射である.
    \end{Def}

    \begin{Def}
        Hilbert空間$\spX, \spY$について,
        $\dom(T)=\spX, \range(T)=\spY$かつ等長な作用素$T:\spX \to \spY$を\textbf{ユニタリ作用素}と呼ぶ.
        $\spX$から$\spY$へのユニタリ作用素が存在するとき,
        $\spX$と$\spY$は\textbf{Hilbert空間として同型}であるという.
    \end{Def}

    \begin{Def}
        作用素$S,T$を写像と見た時,すなわち,どちらも空間全体で定義されている時の合成写像を\textbf{作用素の積}と呼ぶ.
        また,$ST, TS$のどちらも恒等写像であるとき,
        $S$と$T$は互いに\textbf{逆作用素}であると言い,$S=T^{-1}, T=S^{-1}$と書く.
    \end{Def}

    \begin{Def}
        2つの作用素$S,T:\spX \to \spY$が
        \[ \dom(S) \subseteq \dom(T); \Forall{ u \in \dom(S)} (Tu=Su) \]
        を満たすとき,$T$は$S$の\textbf{拡張}である,または$S$は$T$の\textbf{縮少}であるという.
    \end{Def}

    \subsubsection{線形作用素}
    \begin{Def}
        作用素$T:\spX \to \spY$が
        \[ \Forall{ u,v \in \dom} \Forall{\alpha, \beta \in \C} (T(\alpha u+\beta v)=\alpha Tu+\beta Tv) \]
        を満たすとき,$T$は\textbf{線形作用素}であると言う.
    \end{Def}

    \begin{Def}
        2つの線形作用素$T:\spX \to \spY$と$S:\spY \to \mathscr{Z}$について,
        積$ST$を
        \[ (ST)u:=S(Tu);~ \dom(ST)=\{ u \in \dom(T) : Tu \in \dom(S) \} \]
        定める.この時,結合律が成り立つ.

        さらに$P, Q:\spX \to \spY$について
        和$P+Q$を以下のように定める.
        \[ (P+Q)u:=Pu+Qu; \dom(P+Q)=\dom(P) \cap \dom(Q)  \]
        これについて,$(P_1+P_2)Q=P_1 Q+P_2 Q$は成り立つが,
        $P(Q_1+Q_2)=P Q_1+P Q_2$が成り立つとは限らない\footnote{教科書p.154}.

        スカラー倍$\alpha P$を以下で定める.
        \[ (\alpha T)u:=\alpha (Tu); \dom(\alpha T)=\dom(T) \]
    \end{Def}
    次節で定める$B(\spX)$では,これらが環を成す.
    しかも$\spX$がBanach空間であれば,$B(\spX)$もBanach空間になる.

    \subsubsection{有界線形作用素}
    \begin{Def}
        線形作用素$T:\spX \to \spY$が
        \[ \Exists{M \geq 0} \Forall{u \in \dom(T)} (\|Tu\|_{\spY} \leq M \|u\|_{\spX}) \]
        を満たすとき,$T$は\textbf{有界}であるという.
    \end{Def}

    \begin{Def}
        ノルム空間$\spX$からノルム空間$\spY$への作用素$T$について,
        $u \in \dom(T)$において連続であるとは,
        \[ \Forall{\{ u_n\}_{n \in \N} \subset \dom(T)} (\lim_{n \to \infty}u_n=u \implies \lim_{n \to \infty}T u_n=Tu)  \]
        が成り立つことである.
        $T$が任意の$u \in \dom(T)$で連続であれば,単に$T$は\textbf{連続}であるという.
    \end{Def}
    $T$が有界ならば$\|T(u_n-u)\| \leq M\|u_n-u\| \to 0(n \to \infty)$より$T$は連続である.
    線形作用素ならば逆が成り立つことも言える.
    定理\ref{them7:1}参照.

    \begin{Def}
        ノルム空間$\spX(\neq \{0\})$からノルム空間$\spY$への\kenten{有界線形作用素}で,
        $\spX$全体で定義されているもの全体の集合を$B(\spX, \spY)$と書く.
        また,$B(\spX):=B(\spX, \spX)$とする.
    \end{Def}
    おそらく$B$はBoundedから来ているのだろう.
    定理\ref{them7:1}から,$B(\spX, \spY)$は連続な線形作用素の集合とも言える.
    以降では$B(\spX, \spY)$の元及び空間自体が主な考察対象となる.

    \begin{Def}
        $\spB{X}{Y}$の元$T$のノルムを
        \[
            \|T\|
            :=\sup_{\|u\| \leq 1}{\|Tu\|_{\spY}}
            =\sup_{\|u\|=1}{\|Tu\|_{\spY}}
            =\sup_{\|u\| \neq 0}{ \frac{\|Tu\|_{\spY}}{\|u\|_{\spX}} }
        \]
        で定める.
        これらが等しいことは$T$の線形性とノルムの定義から容易に示される.
        この時,任意の$u \in \spX$で$\|Tu\| \leq \|T\| \|u\|$
        \footnote{当然ながら$\|Tu\|_{\spY} \leq \|T\|_{\spB{X}{Y}} \|u\|_{\spX} $の事}
        が成り立つ.
    \end{Def}
    以上で定められた,有界線形作用素同士の演算とノルムについて,定理\ref{them7:6}が重要である.

    \begin{Def}
        $T_n (n=1,2,\dots),~ T \in \spB{X}{Y}$とする.
        \[ \|T_n-T\|_{\spB{X}{Y}} \to 0 ~(n \to \infty) \]
        が成り立つとき,
        $T_n$は$T$に\textbf{一様収束}する,あるいは\textbf{ノルム収束}するという.
    \end{Def}
    \begin{Def}
        $T_n (n=1,2,\dots),~ T \in \spB{X}{Y}$とする.
        \[ \|T_n u-T u\|_{\spY} \to 0 ~(n \to \infty) \]
        が成り立つとき,$T_n$は$T$に\textbf{強収束}すると言い,
        \[ \operatorname{s-lim}_{n \to \infty}{T_n}=T,~~ T_n \to T \mbox{(強)} \]
        などと書く.
        $T$は$T_n$の\textbf{強極限}という.これは存在すれば一意である.
    \end{Def}

    \subsubsection{閉作用素}
    \begin{Def}[閉作用素/1]
        線形作用素$T: \spX \to \spY$が以下の条件を満たすとき,
        $T$は\textbf{閉作用素}であるという.
        \begin{center}
            グラフ$\Gamma(T):=\{ (u, Tu) : u \in \spX \}$は$\spX \times \spY$の閉集合である.
        \end{center}
    \end{Def}
    これについては閉グラフ定理(定理\ref{them7:33})が重要である.
    上に述べたのは意味が取りやすい閉作用素の定義であるが,
    他に証明をする際に使われる定義が有る.
    \begin{Def}[閉作用素/2]
        線形作用素$T: \spX \to \spY$が以下の条件を満たすとき,
        $T$は閉作用素であるという.
        \begin{center}
            $\dom(T)$は\textbf{グラフ・ノルム}$\|u\|_{\spX}+\|Tu\|_\spY$に関して完備である.
        \end{center}
    \end{Def}

    \begin{Def}
        閉作用素であるような拡張を持つ作用素を\textbf{前閉作用素}と呼ぶ.
        また,その閉作用素であるような拡張を\textbf{閉拡大}と呼ぶ.
        最小 \footnote{定義域の包含関係について順序を定めている.} の閉拡大を\textbf{閉包}と呼ぶ.
    \end{Def}
    より詳しく,どのような特徴を持つ作用素が前閉作用素なのか,
    ということは定理\ref{them7:20}が明らかにしている.

    \subsubsection{共役空間}
    \begin{Def}
        $\spX^{\ast}=B(\spX, \C)$はすでに定めたノルム
        \[ \|f\|=\sup_{u \in \spX, \| u \| \neq 0}{ \frac{|f(u)|}{\|u\|} } \]
        によってBanach空間となる
        \footnote{$\spX^{\ast}$の元$f$による$u \in \spX$の像は$fu$でなく$f(u)$や$\langle u,f \rangle$と書く.}.
        $\spX^{\ast}$を$\spX$の\textbf{双対空間}あるいは\textbf{共役空間}と呼ぶ.
        また,$\spX^{\ast}$の元は\textbf{汎関数}と呼ばれる.
    \end{Def}
    \begin{Def}
        ノルム空間$\spX, \spY$について
        $T \in \spB{X}{Y}$と以下のような関係を持つ$T^{\ast} \in \spB{X^{\ast}}{Y^{\ast}}$を
        $T$の\textbf{共役作用素}と呼ぶ.
        \begin{align*}
            \|T^{\ast}\|_{\spB{X^{\ast}}{Y^{\ast}}}=\|T\|_{\spB{X}{Y}} \\
            \Forall{u \in \spX} \Forall{g \in \mathscr{Y^{\ast}}} ((T^{\ast}g)(u)=g(Tu))
        \end{align*}
    \end{Def}
    \begin{Def}
        Banach空間$\spX$に対し,作用素$\kappa$を
        \begin{align*}
            \kappa: \spX \to \spX^{\ast \ast}     &;u \mapsto \phi_{u} \\
            \phi_u: \mathscr{X^{\ast}} \to \C \hspace{3truemm}&;f \mapsto f(u)
        \end{align*}
        のように定める.
        $\kappa$により$\spX$は$\spX^{\ast \ast}$の中に同型に埋め込まれる(定理\ref{them8:23}).
        $\range(\kappa)=\spX^{\ast \ast}$であった時,
        すなわち$\kappa$が$\spX$から$\spX^{\ast \ast}$への同型写像となるとき,
        $\spX$を\textbf{反射的}あるいは\textbf{回帰的}であると言う.
    \end{Def}

    \subsection{例}
    % Fourier変換

    \subsection{定理・命題・補題・系}
    \begin{Them}[定理7.1, p.148] \label{them7:1}
        線形作用素$T$が連続ならば$T$は有界である.
        \footnote{線形作用素$T$は0で連続ならば定義域全体で連続.
        実際,任意の点$a$について,0へ収束する列$\{u_n\}$を元に$a$へ収束する列を$\{a+u_n\}$の様に作れる.
        $T$が0で連続であれば,$T$の線形性と三角不等式から$\|T(a+u_n)\| \leq \|Ta\|+\|Tu_n\| \to \|Ta\|~(n \to \infty)$.
        よって$T$は任意の$a$で連続.}
    \end{Them}

    \begin{Them}[定理7.6, p.150] \label{them7:6}
        $\spY$がBanach空間であるとき,$\spB{X}{Y}$はBanach空間になる.
    \end{Them}

    \begin{Them}[定理7.8, p.153] \label{them7:8}
        $T_n, T \in \spB{X}{Y}$の時,$\|T_n - T\| \to 0$(一様収束)は次のことと同値である:
        $T_n$は$\spX$の閉単位球上で一様収束する.すなわち,
        \[ \Forall{\epsilon>0} \Exists{n_0 \in \N} ( \Forall{n>n_0} \Forall{u \in \spX} (\|u\| \leq 1 \implies \|T_n u - T u\|<\epsilon)) \]
    \end{Them}
    \begin{Them}[定理7.20 (i), p.166] \label{them7:20}
        $T$が前閉作用素であるための必要十分条件は,
        \[ \Forall{\{ u_n \}_{n \in \N} \subset \dom(T)} ((\lim_{n \to \infty}u_n=0  \land \lim_{n \to \infty}Tu_n=v) \implies v=0) \]
    \end{Them}

    \begin{Them}[定理7.21, p.166, 一様有界性の原理] \label{them7:21}
        $\spX$を\kenten{Banach空間},$\spY$を\kenten{ノルム空間}であるとし,
        $\{ T_{\lambda} \}_{\lambda \in \Lambda} \subseteq \spB{X}{Y}$を作用素の族とする.
        この時,$\spX$の各点$u$で$\{ T_{\lambda}u \}_{\lambda \in \Lambda} \subset \spY$が
        有界ならば,$\{ T_{\lambda} \}_{\lambda \in \Lambda}$は一様に有界である.
    \end{Them}
    これはthree basic principlesの第1のもの.

    \begin{Them}[定理7.23, p.167, Baireのカテゴリー定理] \label{them7:23}
        $\spX$を\kenten{完備な距離空間}であるとする.
        高々加算個の$\spX$の閉集合$\spX_n ~(n=1,2,\dots)$が
        $\spX$を覆う($\spX=\bigcup_{n=1}^{\infty}{\spX_n}$)ならば,
        少なくとも1つの$\spX_n$は$\spX$の開球を含む.
    \end{Them}
    これを元に次が示される.

    \begin{Them}[定理7.30, p.170, 開写像原理] \label{them7:30}
        $\spX, \spY$をBanach空間とし,$T \in \spB{X}{Y}$とする.
        もし$\range(T)=\spY$ならば,$T$は開写像である.
    \end{Them}
    これはthree basic principlesの第2のもの.

    \begin{Them}[定理7.33, p.172, 閉グラフ定理] \label{them7:33}
        $\spX, \spY$はBanach空間,
        $T$は$\spX$から$\spY$への閉作用素とする.
        この時,$\dom(T)=\spX$ならば$T \in \spB{X}{Y}$.
    \end{Them}

    \begin{Cor}[系7.34, p.172] \label{cor7:34}
        $\spX, \spY$はBanach空間,
        $T$は$\spX$から$\spY$への閉作用素とする.
        $T$が1対1かつ$\range(T)=\spY$ならば$T^{-1} \in \spB{X}{Y}$
    \end{Cor}

%    \begin{Them}[定理7., p.] \label{them7:}
%    \end{Them}

    \begin{Them}[定理8.3, p.176] \label{them8:3}
        $\spX$をノルム空間,$f \in \spX^{\ast}$とするとき,次のことが成り立つ.
        \begin{enumerate}[i)]
            \setlength{\leftskip}{5truemm}
            \item $\mKer f=\{ u \in \spX : f(u)=0 \}$は$\spX$の閉部分空間.
            \item $f\neq 0$とし,$u_0 \not \in \mKer f$とすると,任意の$u \in \spX$は
                  \[ u=u'+\alpha u_0 ~~ u' \in \mKer f,~ \alpha \in \C \]
                  と一意に表される.ここで$\alpha$は$\alpha = f(u)/f(u_0)$で与えられる.
              \item $\spX$が\kenten{Hilbert空間}で$f \neq 0$ならば,$(\mKer f)^{\perp}$は1次元である.
        \end{enumerate}
    \end{Them}
    \begin{Them}[定理8.5, p.177, Rieszの表現定理] \label{them8:5}
        $\mathscr{H}$をHilbert空間とすると,任意の$f \in \mathscr{H}^{\ast}$は
        有る$v \in \mathscr{H}$によって$f_v(\cdot)=(\cdot,v)$と表される.
        $v$は$f$によって一意に定まる.
    \end{Them}
    このことからHilbert空間が反射的である($\mathscr{H} \simeq \mathscr{H}^{\ast\ast}$)ことが示される.

    Rieszの表現定理は$\mathscr{H}^{\ast}$が十分広いことも言っている.
    Banach空間$\spX$の双対空間$\spX^{\ast}$の場合,
    1次元の部分空間を作ることは簡単に出来る
    \footnote{p.181参照.適当に$u_0 \in \spX, c \in \C$をとり,$f(\alpha u_0)=c \alpha$とする.}.
    しかし$\spX^{\ast}$が十分広い空間であることはまったく自明でない.
    これは以下の定理によって示される.
    \begin{Them}[定理8.11, p.182, Hahn-Banachの拡張定理] \label{them8:11}
        $\spX$を\kenten{実線形空間}とする.
        $p:\spX \to \R$は(線形とは限らない)汎関数とし,以下を満たすとする.
        \begin{enumerate}[i)]
            \setlength{\leftskip}{5truemm}
        \item $\Forall{u, v \in \spX} (p(u+v) \leq p(u)+p(v))$ 
        \item $\Forall{x \in \spX} \Forall{\alpha \in \R_{\geq 0}} (p(\alpha x) = \alpha p(x))$
        \end{enumerate}
        この条件はまとめて劣線形性と呼ばれる.
        $f$は$\spX$の部分空間$\mathscr{M}$で定義された線形汎関数で,
        \[ \Forall{u \in \mathscr{M}} (f(u) \leq p(u)) \]
        を満たすものとする.
        その時$f$はこの不等式と線形性を保ったまま,$\spX$全体に拡張される.
    \end{Them}
    これはthree basic principlesの第3のもの.
    複素線形空間でも同様の定理が成立する.

    \begin{Them}[定理8.13, p.184] \label{them8:13}
        $\spX$を\kenten{複素線形空間}とする.
        $p:\spX \to \R$は(線形とは限らない)汎関数とし,以下を満たすとする.
        \begin{enumerate}[i)]
            \setlength{\leftskip}{5truemm}
            \item $p(u) \geq 0$
            \item $\Forall{ u, v \in \spX} (p(u+v) \leq p(u)+p(v))$ 
            \item $\Forall{ x \in \spX} \Forall{\alpha \in \R_{\geq 0}} (p(\alpha x) = \alpha p(x))$
        \end{enumerate}
        この性質を持つ$p$を$\spX$上のセミノルム(semi-norm)と言う.
        $f$は$\spX$の部分空間$\mathscr{M}$で定義された線形汎関数で,
        \[ \Forall{u \in \mathscr{M}} (0 \leq |f(u)| \leq p(u)) \]
        を満たすものとする.
        その時$f$はこの不等式と線形性を保ったまま,$\spX$全体に拡張される.
    \end{Them}

    \begin{Them}[定理8.23, p.189] \label{them8:23}
        Banach空間$\spX$に対し,作用素$\kappa$を
        \begin{align*}
            \kappa: \spX \to \spX^{\ast \ast}     &;u \mapsto \phi_{u} \\
            \phi_u: \mathscr{X^{\ast}} \to \C \hspace{3truemm}&;f \mapsto f(u)
        \end{align*}
        のように定める.
        $\kappa$により$\spX$は$\spX^{\ast \ast}$の中への等長な線形作用素である.
        したがって,$\spX$は$\spX^{\ast \ast}$の部分空間とみなせる.
    \end{Them}

%    \begin{Them}[定理8., p.] \label{them8:}
%    \end{Them}


\end{document}
