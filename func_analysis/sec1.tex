    \section{第1,2章/空間}
    \subsection{定義}
        \begin{Def}
            空でない集合$V$,体$k$,
            2つの演算子$+:V \times V \to V$と$\ast: k \times V \to V$の組$(V,k,+,\ast)$
            であって以下の条件を満たすもの\textbf{線形空間}と呼ぶ.
            ただし$\alpha, \beta \in k$と$u,v \in V$とする.
            \begin{description}
                \setlength{\leftskip}{0.5truecm}
                \item[V1] $V$は加法$+$について群を成す
                \item[V2] $\alpha(u+v)=\alpha u+\alpha v$
                \item[V3] $(\alpha+\beta)u=\alpha u+\beta u$
                \item[V4] $(\alpha \beta)u=\alpha (\beta u)$
                \item[V5] $1 u=u$~(ただし左辺の1は$k$の乗法単位)
            \end{description}
            スカラー倍の演算子$\ast$は略記した.
            また,加法の演算子+には
            $+_{V}:V \times V \to V$と$+_{k}: k \times k \to k$の2つがあるが,
            混同のおそれが無いためどちらも+で表した.
        \end{Def}

        \begin{Def}
            $\C$上の線形空間$V$に対し,以下の条件を満たす
            対応$\| \cdot \|: V \to \R $を\textbf{ノルム}と呼ぶ.
            ただし$\alpha \in \C$と$x, y \in V$とする.
            \begin{description}
                \setlength{\leftskip}{0.5truecm}
                \item[N1] $\|x\|=0 \iff x=0$
                \item[N2] $\|\alpha x\| =|\alpha| \|x\|$
                \item[N3] $\|x+y\| \leq \|x\|+\|y\|$
            \end{description}
            $\|x\| \geq 0$を仮定することも多いが,これは上の(N3)で$y=-x$とすれば導出できる.
            ノルムを備えた線形空間を\textbf{ノルム空間}と呼ぶ.
        \end{Def}
        これは大きさの概念を抽象化したものである.
        これを利用し,差の大きさとして標準的な距離を定めることが出来る.
        すなわち,任意の2点$x,y$の距離を$d(x,y)=\|x-y\|$とすると
        これは距離の公理を満たす.

        \begin{Def}
            $\C$上の線形空間$V$に対し,以下の条件を満たす
            対応$(\cdot, \cdot): V \times V \to \C $を\textbf{内積}と呼ぶ.
            ただし$\alpha \in \C$と$u, v, w \in V$が取れるものとする.
            \begin{description}
                \setlength{\leftskip}{0.5truecm}
                \item[I1] $(u,v)=\overline{(v,u)}$
                \item[I2] $(\alpha u, v) =\alpha (u,v)$
                \item[I4] $(u+v,w)=(u,w)+(v,w)$
                \item[I5] $(u,u) \geq 0$
                \item[I6] $(u,u)=0 \iff u=0$
            \end{description}
            内積を備えた線形空間を\textbf{内積空間}(あるいは前Hilbert空間)と呼ぶ.
            また,\[ \|x\|=(x,x)^{1/2} \]と置くと,これはノルムの定義を満たす.
            したがって内積空間はノルム空間とすることが出来る.
            このノルムは\textbf{内積が定めるノルム}を呼ばれる.
        \end{Def}
        ノルムが大きさの概念を抽象化したものであるのに対し,
        内積は大きさと角度の概念を抽象化したものである.
        別の言い方をすれば,これはより一般的な「近さ」を抽象化したものである.
        % グラフにノルムを入れることは出来るか?

        \begin{Def}
            ノルム空間$(V, \| \cdot \|)$について,
            点列$\{ u_n \}_{n \in \N}$が以下の論理式を満たすとき,
            点列$\{ u_n \}$を\textbf{Cauchy列}と呼ぶ.
            \[ \forall \varepsilon>0,~ \exists N \in \N (\forall m,n>N,~ \| u_m-u_n \|<\varepsilon) \]
        \end{Def}

        \begin{Def}
            空間$X$に含まれる任意のCauchy列が$X$の元に収束するとき,空間$X$は\textbf{完備}であると呼ばれる.
        \end{Def}

        \begin{Def}
            完備なノルム空間を\textbf{Banach空間},
            内積が定めるノルムについて完備な内積空間を\textbf{Hilbert空間}と呼ぶ.
        \end{Def}

        \begin{Def}
            線形空間$\spX$の空でない部分集合$\mathscr{M}$が
            \[ u, v \in \mathscr{M},~ \alpha, \beta \in \C,~ \alpha u+ \beta v \in \mathscr{M} \]
            を満たすとき,$\mathscr{M}$は$\spX$の(線形)\textbf{部分空間}と呼ぶ.
            特にノルム空間の部分空間であって閉集合であるものを\textbf{閉部分空間}と呼ぶ.
        \end{Def}
        これに関しては後の定理\ref{them1:28}, 系\ref{cor1:29}が重要である.

        \begin{Def}
            $\C$上の線形空間として有限次元を持つノルム空間を\textbf{有限次元ノルム空間}と呼ぶ.
        \end{Def}
        これに関しては定理\ref{them1:37}が重要である.
        すなわち,有限次元ノルム空間は完備である.

        \begin{Def}
            Hilbert空間$\mathscr{H}$の空でない\kenten{閉部分集合}$\mathscr{M}$に対して,
            \[ \mathscr{M}^{\perp}=\{ u \in \mathscr{H} : \forall v \in \mathscr{M},~ (u,v)=0\} \]
            を$\mathscr{M}$の\textbf{直交補空間}と呼ぶ.
        \end{Def}

        \begin{Def}
            任意の有界閉部分集合がコンパクトであるようなノルム空間を\textbf{局所コンパクト}であると言う.
        \end{Def}

    \subsection{例}
        \begin{Example}[例1.17, p.8]
            閉区間$[a,b]( \subset \R)$から$\C$への連続な関数全体を$C[a,b]$と表す.
            ノルムを\[ \|u\|=\displaystyle{\sup_{x \in [a,b]}{|u(x)|}} \]として入れると,
            $C[a,b]$はBanach空間である.
        \end{Example}
        上のことを証明する際は教科書の例1.9(p.5), 1.13(p.6)も参照のこと.

        \begin{Example}[例1.18, p.9]
            点列$\{u_n\}_{n \in \N}$で
            $\sum_{n \in \N}{|u_n|}<\infty$であるもの全体の集合を$l^1$と書く.
            スカラー集合は$\C$,線形演算は項毎のものとし,ノルムを\[ \|u\|=\sum_{n \in \N}{|u_n|} \]で定義する.
            この時,$l^1$はBanach空間である.
        \end{Example}
        \begin{Example}[例1.19, p.10]
            点列$\{u_n\}_{n \in \N} \subset \C$で
            $\sup_{n \in \N}{|u_n|}<\infty$であるもの全体の集合を$l^{\infty}$と書く.
            $l^{\infty}$に於けるノルムを\[ \|u\|=\sup_{n \in \N}{|u_n|} \]で定義する.
            この時,$l^{\infty}$はBanach空間である.
        \end{Example}
        \begin{Example}[例1.33, p.19]
            閉区間$[a,b] \subset \R$について,
            $[a,b]$から$\C$への$C^m$級関数全体を$C^m[a,b]$と表す.
            これは$C[a,b]$の部分空間であり,$C[a,b]$の中で稠密.
            $C^m[a,b]$にノルムを
            \[ \|u\|=\sum_{j=0}^{m}{ \sup_{x \in [a,b]}{\left| \frac{d^j u}{dx^j}(x) \right|} } \]
            として入れると,
            $C^m[a,b]$はBanach空間である.
        \end{Example}

        次の二つは特に関数解析学で重要である.
        \begin{Example}[\S 2.3, p.37-40]
            $\Omega$を$\R^n$の可測集合とし,$|\Omega|>0$とする.
            $\Omega$上の可積分関数全体を,
            $u=v (a.e.)$な元を同一視するという同値関係で割ったものを$L^1(\Omega)$と呼ぶ.
            これはノルム\[ \|u\|_{L^1}:=\int_{\Omega}{|u(x)|dx} \]によってBanach空間となる.
        \end{Example}
        \begin{Example}[\S 2.4, p.40-45]
            $\Omega$を前と同じものとする.また,$p$は$1 \leq p < \infty$を満たす実数とする.
            $\Omega$上の可測関数で,
            \[ \|u\|_{L^p}:=\left( \int_{\Omega}{|u(x)|^p dx} \right)^{\frac{1}{p}} < \infty \]
            を満たすもの全体を$\mathscr{L}^p(\Omega)$と書く.
            このノルム$\|u\|_{L^p}$について
            Minkowshiの不等式(三角不等式)及び
            H\"olderの不等式
            \footnote{$p,q \in [1,\infty]$が$1/p+1/q=1$を満たすとき
                $f \in \mathscr{L}^p, g \in \mathscr{L}^q$について$\|fg\|_{L^1} \leq \|f\|_{L^p} \|g\|_{L^q}$}
            が成立する.
            $\mathscr{L}^p(\Omega)$を$u=v (a.e.)$な元を同一視するという同値関係で割ったものを$L^p(\Omega)$と呼ぶ.
            これはノルム$\|u\|_{L^p}$によってBanach空間となる.
        \end{Example}

    \subsection{定理・命題・補題・系}
    \subsubsection{ノルム・内積の基本的性質}
    \begin{Them}
        ノルム空間$(V, \| \cdot \|)$について,ノルム$\|u\|: V \to \C$は連続関数である.
    \end{Them}
    \begin{Them}[定理1.24, p.14]
        内積空間$(V, (\cdot, \cdot))$について,
        内積$(u, v): V \times V \to \C$は$u,v$両方について連続関数である.
    \end{Them}
    \begin{Them}[Schwarzの不等式, 定理1.21, p.13]
        内積空間$(V, (\cdot, \cdot))$に対して,
        \[ \forall u, v \in V,~ |(u,v)| \leq \|u\|\|v\| \]
        ただしここに有る$\| \cdot \|$は内積が定めるノルムである.
    \end{Them}

    \subsubsection{閉部分空間}
    \begin{Them}[定理1.28, p17] \label{them1:28}
        Banach空間$\spX$の部分空間$\mathscr{M}$を考える.
        $\mathscr{M}$のノルムを$\spX$のノルムを$\mathscr{M}$に制限したものとした時に
        $\mathscr{M}$もBanach空間になるための必要十分条件は,$\mathscr{M}$が$\spX$の閉部分空間であること.
    \end{Them}
    \begin{Cor}[系1.29, p.18] \label{cor1:29}
        Hilbert空間$\spX$の部分空間$\mathscr{M}$を考える.
        $\mathscr{M}$の内積を$\spX$の内積を$\mathscr{M}$に制限したものとした時に
        $\mathscr{M}$もHilbert空間になるための必要十分条件は,$\mathscr{M}$が$\spX$の閉部分空間であること.
    \end{Cor}

    \subsubsection{有限次元ノルム空間}
    \begin{Them}[補題1.38, p.22] \label{them1:38}
        ノルム空間$X$の元$u$を適当な$X$の基底$\{ \phi_1, \dots, \phi_n \}$を用いて
        $u=\alpha_1 \phi_1+\dots+\alpha_n \phi_n (\alpha_k \in \C)$と表したとする.
        $||| u |||:=\sup_{k=1,\dots,n}{|\alpha_k|}$とおくと,これはノルムであり,
        しかも任意の有限次元ノルム空間に備えられた任意のノルムと$||| \cdot |||$は同値である.
    \end{Them}
    このノルム$||| \cdot |||$について任意の有限次元ノルム空間が完備であることを示すことが出来る.
    \begin{Them}[定理1.37, p.22] \label{them1:37}
        有限次元ノルム空間は完備である.
    \end{Them}

    \subsubsection{局所コンパクト性}
    \begin{Them}
        ノルム空間$\spX$の
        単位球$\mathcal{S}=\{u \in \spX : \|u\|=1\}$がコンパクト集合ならば,
        $\spX$は有限次元である.
    \end{Them}

