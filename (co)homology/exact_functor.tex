\documentclass[a4j]{jarticle}
\usepackage{../math_note}
\usepackage[all]{xy}

\newcommand{\Hom}{\operatorname{Hom}}
%\newcommand{\ker}{\operatorname{ker}}
\newcommand{\coker}{\operatorname{coker}}
\newcommand{\im}{\operatorname{im}}
\newcommand{\coim}{\operatorname{coim}}
\newcommand{\arr}[1]{\overset{#1}{\longrightarrow}}

\begin{document}
\section{アーベル圏}
一般的な圏の定義、(始,終,零)対象、(余)極限、$\Hom$関手、米田の補題については既知とする。

\begin{Def}[前加法圏/preadditive category (or Ab-category)]
    圏$\cat{C}$が前加法圏であるとは、
    任意の対象$X, Y \in \cat{C}$について$\Hom_{\cat{C}}(X,Y)$が非可換環の構造を持つことである。
    ただし加法は余積が誘導するもの、乗法は射の合成、加法単位元は零射、乗法単位元は単位射である。
\end{Def}

前加法圏においては、積(product)と余積(coproduct)は(存在すれば)一致する。
なのでこれを双積(biproduct)と呼ぶ。双積は加群における直和の一般化である。
一致することの証明は自明である。ただしこの「自明」は「すぐ分かる」を意味しない。
丁寧な説明は\url{https://ncatlab.org/nlab/show/additive+category#properties}にある。

\begin{Def}[加法圏/additive category]
    圏$\cat{C}$が加法圏であるとは、
    $\cat{C}$が前加法圏であり、かつ有限個の任意の対象について双積が存在することである。
\end{Def}

\begin{Def}[(ver.I)アーベル圏/abelian category]
    圏$\cat{C}$がアーベル圏であるとは、
    $\cat{C}$が加法圏であり、以下を満たすものである。
    \begin{itemize}
    \item 任意の射が核を持つ
    \item 任意の射が余核を持つ
    \item 任意のmono射はある射の核である
    \item 任意のepi射はある射の余核である
    \end{itemize}
\end{Def}

\begin{Def}[(ver.II)アーベル圏/abelian category]
    圏$\cat{C}$がアーベル圏であるとは、以下を満たすということである。
    \begin{itemize}
    \item 零対象を持つ
    \item 任意の有限個の対象について、双積が存在する
    \item 任意の射が核を持つ
    \item 任意の射が余核を持つ
    \item 任意のmono射はある射の核である
    \item 任意のepi射はある射の余核である
    \end{itemize}
\end{Def}

\begin{Def}[核/kernel, 余核/cokernel, 像/image, 余像/coimage]
    アーベル圏に於ける射$f:X \to Y$について定義する。
    ~~~~~~~~~~~~~~~~~~~~~~~~~~~~~~~~~~~~~~~~~
    \begin{itemize}
        \item 射$f$の核($\ker f$)とは、$f$と零射$0_{XY}$のequalizerのことである
        \item 射$f$の余核($\coker f$)とは、$f$と零射$0_{XY}$のcoequalizerのことである
        \item 射$f$の像($\im f$)とは、$\ker \coker f$のことである
        \item 射$f$の余像($\coim f$)とは、$\coker \ker f$のことである
    \end{itemize}
\end{Def}

像の定義は以下のように述べることも出来る。
\begin{Def}[像/image] \label{image}
    射$f:X \to Y$の像とは、
    射$f$を$X \to I \hookrightarrow Y$と分解する対象$I$とmono射$I \hookrightarrow Y$の組であって、
    そのような分解が作る圏に於ける始対象である。
    \[
    \begin{xy}
    \xymatrix
    {
    X \ar[rr]^{f} \ar[rd] \ar@/_14pt/[rdd]_{\forall} &   & Y \\
    & I \ar@{^{(}->}[ru] \ar@{-->}[d]_{\exists_1} &   \\
    & {}^\forall Z \ar@{^{(}->}@/_14pt/[ruu]_{\forall} &   
    }
    \end{xy}
    \]
\end{Def}

\section{完全関手}
\begin{Def}[完全列/exact sequence]
    アーベル圏の対象と射の列
    \[ \cdots \to X_{n-1} \arr{f_{n-1}} X_{n} \arr{f_{n}} X_{n+1} \to \cdots \]
    であって、
    \[ \im f_{n-1}=\ker f_{n}\]
    を満たすものを完全列と呼ぶ。
    したがって$f_{n} \circ f_{n-1}$は零射である。
    特に完全列
    \[ 0 \to A \to B \to C \to 0 \]
    は短完全列と呼ばれ、
    左側にしか0が無いものは左完全列、右側にしか0が無いものは右完全列と呼ばれる。
\end{Def}

\begin{Def}[加法的関手/additive functor]
    アーベル圏同士の間の関手であって、任意の有限双積を保つものを加法的関手と呼ぶ。
\end{Def}

\begin{Def}[左(右)完全関手/left(right) exact functor]
    アーベル圏同士の間の関手であって、任意の左(resp.右)完全列を保つものを左(resp. 右)完全関手と呼ぶ。
\end{Def}

\begin{Them}
    加法的関手$F: \cat{A} \to \cat{B}$について、以下は同値。
    \begin{enumerate}[(i)]
        \item $F$は左完全関手
        \item $F$は核を保つ
        \item $F$は有限極限を保つ
        \item $F$は左短完全列を保つ
    \end{enumerate}
\end{Them}
\begin{proof}
    (i)$\implies$(ii):
    アーベル圏$\cat{A}$に属す任意の射$f:X \to Y$を取る。
    すると
    \[ 0 \to \ker f \arr{i} X \arr{f} Y \]
    は左完全列である。ただし$i$は埋め込み射(したがって単射)。
    これを$F$で写すと、
    \[ 0 \to F(\ker f) \arr{F(i)} F(X) \arr{F(f)} F(Y) \]
    となる。
    これが左完全列であることから、$ \im F(i) \cong \ker F(f) $が得られ、
    また$F(i)$がmonicであることから、$ F(\ker f) \cong \im F(i) $となる
    \footnote{$\im$の定義(\ref{image})にある可換図式において$Z=X, X \to Z=\mathrm{1_{X}}$として確認できる。}。 %% この行怪しい
    合わせて\[ F(\ker f) \cong \ker F(f). \]

    (ii)$\implies$(iii):
    $F$は加法的関手であるから、有限双積を保つ。さらに仮定から核も保つ。
    一般の圏において、任意の有限極限は有限積とイコライザで
    表すことができる\footnote{Awodey本等にある}が、
    アーベル圏においては有限積とは有限双積のことであり、
    またイコライザは核で表される(イコライザがmono射であることに注意)。
    したがってアーベル圏において任意の有限極限は有限双積とイコライザで表され、
    よって関手$F$は任意の有限極限を保つ。

    (iii)$\implies$(iv):
    アーベル圏$\cat{A}$における任意の左短完全列を取る。
    \[ 0 \to X \arr{f} Y \arr{g} Z. \]
    これを$F$で写す。
    \[ 0 \to F(X) \arr{F(f)} F(Y) \arr{F(g)} F(Z). \]る
    これが再び左短完全列であることを確かめるのは容易である。
    \begin{align*}
        &\ker F(f) \cong F(\ker f) = F(0) = 0 \\
        &\ker F(g) \cong F(\ker g) = F(\im f) \cong \im F(f).
    \end{align*}
    $\im=\ker \coker$を有限極限として表せることに注意。

    (iv)$\implies$(i): 自明。 %%本当に?

\end{proof}

\end{document}
