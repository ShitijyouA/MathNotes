\documentclass[a4paper]{jsarticle}
\usepackage[]{../math_note}
\usepackage[dvipdfmx, colorlinks=true]{hyperref}
\usepackage{pxjahyper}
\usepackage[]{enumitem}
\usepackage[all]{xy}

\newcommand{\Sch}{\mathbf{Sch}}
\newcommand{\Set}{\mathbf{Set}}
\newcommand{\Ring}{\mathbf{Ring}}
\newcommand{\Alg}{\mathbf{Alg}}
\newcommand{\Open}{\mathrm{Open}}
\newcommand{\OpenSubSch}{\mathbf{OpenSubSch}}
\newcommand{\GrpSch}{\mathbf{GrpSch}}
\newcommand{\famF}{\mathcal{F}}
\newcommand{\famG}{\mathcal{G}}

\newcommand{\func}[1]{\underline{#1}}
\newcommand{\ftorM}{\mathcal{M}}
\newcommand{\ftorGL}{\mathcal{GL}}

\newcommand{\acton}{\,\curvearrowright\,}

\newcommand{\Ga}{\mathbb{G}_a}
\newcommand{\Gm}{\mathbb{G}_m}
\newcommand{\GL}{GL}

\begin{document}
\title{Group Schemes}
\author{七条 彰紀}
\maketitle
\tableofcontents

\section{Preface}
    このノートの想定読者は,
    \cite{HarAG}のII, \S 3までを読んだ,
    Geometric Invariant TheoryとModuli Problemに興味がある者である.
    主な参考文献は\cite{Muk1},\cite{AV},\cite{Hos}である.
    大まかな議論の流れは前者の流れを採用し,
    用語などの定義は\cite{Muk1}で述べられているものより
    一般的なものを\cite{AV}と\cite{Hos}から採る.
    \cite{Muk1}で使われる定義は素朴すぎるからである.
    一般的な定義で概念を導入した後,
    特別な場合では\cite{Muk1}での定義と同値に成ることを確かめる,
    という方針を採る.

    このノートでは,次の順に定義していく.
    \begin{enumerate}
        \item $T$-valued point (where $T$ :: scheme),
        \item group scheme,
        \item fine/corse moduli,
        \item categorical/good/geometric/affine GIT quotient,
        \item representation of group (scheme),
        \item linearly reductive group,
        \item closure equivalence,
        \item unstable/semi-stable/stable.
    \end{enumerate}

    目標とする命題は次のものである.
    \begin{Thm}
        $X$ :: affine scheme,
        $G$ :: linearly reductive group scheme acting on $X$とする.
        \begin{enumerate}[label=(\roman*), leftmargin=*]
        \item 
        affine GIT quotient of $X$ by $G$ :: $X \sslash G$はgood quotientである.

        \item
        $X$のstable pointsを$X^s$とすると,
        $X \sslash G$の制限 :: $X^s \slash G$は
        geometric quotient of $X^s$ by $G$である.

        \item
        $X \sslash G$はquotient functor :: $\func{X} \slash G$の最良近似である.

        \item
        $k$を体とする.
        $X$がfinite type/$k$ならば$X \sslash G$もそうである.
%        さらに,$X$がaffine varietyならば$X \sslash G$もそうである.
        \end{enumerate}
    \end{Thm}

\subsection*{Notation}
    $S$ :: scheme上のschemeと$S$-morphismが成す圏を
    \textbf{$\Sch/S$}で表す.
    これはslice categoryの一般的なnotationから来ている.
    
    affine scheme :: $\Spec R$は,時々$R$と略す.

    affine scheme over a ring $R$ :: $X$の
    affine coordinate ringを$R[X]$と書く.
    特に$k[G]$は群環ではないことに注意.
    群環は$kG$と書く.

\section{\texorpdfstring{$T$}{T}-valued Points}
    圏論で言う``generalized point"の概念を,
    名前を変えて用いる.

    \begin{Def}
    \enumfix
    \begin{enumerate}[label=(\roman*),leftmargin=*]
    \item 
    $X, T \in \Sch/S$に対し,
    $\func{X}(T)=\Hom_{\Sch/S}(T,X)$を\textbf{$X$の$T$-valued points}と呼ぶ.
    $T=\Spec R$と書けるときは$\func{X}(T)$を$\func{X}(R)$と書く.
    この関手$\func{X}$はfunctor of pointsと呼ばれる.

    \item
    体$k$上のscheme :: $X$ ($S=\Spec k, X \in \Sch/S$)と
    field extension :: $k \subseteq K$について,
    $\func{X}(K)$を$X$の\textbf{$K$-rational points}と呼ぶ.

    \item
    morphism :: $h: G \to H$について
    自然変換$\func{h}: \func{G} \to \func{H}$は
    $\phi \mapsto h \circ \phi$のように射を写す.
    \end{enumerate}
    \end{Def}

    \begin{Remark}
        $\Sch$はlocally small categoryである.
        すなわち,任意の$X, T \in \Sch$について$\func{X}(T)$は集合である.
        これを確かめるために,
        $X, Y \in \Sch$を任意にとり,
        $\Hom(X,Y)$の濃度がある濃度で抑えられることを見よう.
        射$X \to Y$の作られ方に沿って考える.
        \begin{enumerate}[label=(\arabic*), leftmargin=*]
        \item
            base spaceの間の写像$f: \basesp X \to \basesp Y$をとる.
            このような写像全体の濃度は高々$|\basesp Y|^{|\basesp X|}$.
        \item
            $|Y|$の開集合$U$をとる.
            開集合全体の濃度は高々$2^{|\basesp Y|}$.
        \item
            写像$f^{\#}_U: \shO_Y(U) \to (f_* \shO_X)(U)$を定める.
            このような写像全体の濃度は高々$|(f_* \shO_X)(U)|^{|\shO_Y(U)|}$.
        \end{enumerate}
        したがって$\Hom(X,Y)$の濃度は高々
        \[|\basesp Y|^{|\basesp X|} \times \prod_{U \in 2^{\basesp Y}} |(f_* \shO_X)(U)|^{|\shO_Y(U)|} \]
        となる.
        濃度の上限が存在する(すなわち,ある集合への単射を持つ)から,
        $\Hom(X,Y)$は集合である.
%        presheaf on $Y$の圏は
%        $Y$の開集合が成す圏(small)から
%        集合の圏(locally small)の圏への
%        関手圏である.
%        これがlocally smallであることを示しても良いと思う.
    \end{Remark}

    \begin{Remark}
        上の注意から,Yoneda Lemmaが成立する.
        したがって自然変換$\func{G} \to \func{H}$と
        射$G \to H$が一対一対応する.
        このため,
        schemeの間の射についての議論と
        functor of pointsの間の射の議論は
        (ある程度)互いに翻訳することが出来る.
    \end{Remark}

    \begin{Remark}
        $K$-rational pointについては,
        $\func{X}(K)=\{ x \in X \mid k(x) \subseteq K \}$とおく定義もある.
        ここで$k(x)$は$x$でのresidue fieldである.
        しかし\cite{HarAG} Chapter.2 Ex2.7から分かる通り,
        この二つの定義は翻訳が出来る.
        すなわち,
        $k(x) \subseteq K$を満たす$x \in X$と,
        $\Spec k$-morpsihm :: $\Spec K \to X$は一対一に対応する.

        また$X$ :: finite type /$k$であるとき,
        closed point :: $x \in X$について,
        $k(x)$は$k$の有限次代数拡大体である.
        これはZariski's Lemmaの帰結である.
        したがって$\func{X}(\bar{k})$は$X$のclosed point全体に対応する.
        ただし$\bar{k}$は$k$の代数閉包である.
    \end{Remark}

    \begin{Example}
        $\R$上のaffine scheme $X=\Spec \R[x,y]/(x^2+y^2)$の
        $\R$-rational pointと$\C$-rational pointを考えよう.

        $\Spec \R \to X$の射は環準同型 $\R[x,y]/(x^2+y^2) \to \R$と一対一に対応する.
        しかし直ちに分かる通り,
        このような環準同型は
        \[ (\bar{x}, \bar{y}) \mapsto (0, 0) \]
        で定まるものしか存在し得ない.
        ここで$\bar{x}=x \bmod (x^2+y^2), \bar{y}=y \bmod (x^2+y^2)$と置いた.
        よって$\func{X}(\R)$は1元集合.
        また,この環準同型が誘導する$\Spec R \to X$の射は
        1点空間$\Spec \R$を原点へ写す.

        一方,環準同型 $\R[x]/(x^2+1) \to \C$は
        \[ (\bar{x}, \bar{y}) \mapsto (a, \pm ia) \]
        (ここで$i=\sqrt{-1}, a \in \R$)で定まることが分かる.
        すなわち,$\zerosa(x^2+y^2) \subseteq \affine^2_{\C}$の点に対応して,
        $\R[x]/(x^2+1) \to \C$の環準同型が定まる.
        逆の対応も明らか.
        よって$\func{X}(\C)$の元は
        $\zerosa(x^2+y^2) \subseteq \affine^2_{\C}$の点に
        対応している.
    \end{Example}

    \begin{Example}
        体$k$上のaffine variety :: 
        $X \subseteq \affine^n_k$を
        多項式系 :: $F_1,\dots,F_n \in k[x_1,\dots,x_n]$で定まるものとする.
        すると$k$上の環$R$に対して,次の集合が考えられる.
        \[ V_R=\left\{ p=(r_1,\dots,r_n) \in R^{\oplus n} ~\middle|~ F_1(p)=\dots=F_n(p)=0 \right\}. \]
        この集合の元も$R$-value pointと呼ばれる.
        (\cite{Muk1}ではこちらのみを$R$-value pointと呼んでいる.
        実際,こちらのほうが字句``value point"の意味が分かりやすいだろう.)
        $V_R$の点が$\func{X}(R)$の元と一対一に対応することを見よう.

        $X$のaffine coordinate ringを
        $A=k[x_1,\dots,x_n]/(F_1,\dots,F_n)$とし,
        $\bar{x}_i=x_i \bmod (F_1,\dots,F_n) ~(i=1,\dots,n)$とおく.
        $\phi: A \to R$を考えてみると,
        これは次のようにして定まる.
        \[ (\bar{x}_1,\dots,\bar{x}_n) \mapsto (r_1,\dots,r_n) \in V_R. \]
        すなわち,$V_R$の点に対して$\Hom_{\Ring/k}(A,R)$の元が定まる.
        逆の対応は明らか.
        そして,$\Hom_{\Ring/k}(A,R)$が
        $\Hom_{\Sch/\Spec k}(\Spec R, X)=\func{X}(R)$と一対一対応することはよく知られている.
    \end{Example}

\section{Moduli Functor and Fine/Corse Moduli Space}
    \subsection{Families}
    moduli問題を語るには用語``family"が必要である.

    \begin{Def}
        $\mathcal{P}$を集合のクラス
        \footnote
        {
            集合$X$を変数とする
            述語$X \in \mathcal{C}$の意味を
            「$X$はある条件を満たす対象である」と定義した,
            と考えて良い.
            「属す」の意味は集合と同様に定める.
        }
        とする.
        集合$B$について,
        $B$の構造と整合的な構造を持った集合$\famF$と
        全射写像$\pi: \famF \to B$の組が
        $\mathcal{P}$の$B$上の\textbf{family}であるとは,
        各$b \in B$について集合$\pi^{-1}(b) \subseteq \famF$が
        $\mathcal{P}$に属すということ.
%        $B$はfamily $\pi: \mathcal{F} \to B$のbaseと呼ばれる.
    \end{Def}
    「$B$の構造と整合的な構造」というのは,
    例えば,
    $S$が位相空間であって
    写像$\famF \to S$を連続にするような位相が$\famF$に入っている,
    ということである.
    familyの構造は場合毎に明示されなくてはならない.

    用語``family"を厳密に定義しているものは全くと言っていいほど無いが,
    ここではRenzoのノート
    \footnote{ \url{http://www.math.colostate.edu/~renzo/teaching/Topics10/Notes.pdf} }
    の定義を参考にした.
%    ただし,Renzoのノートの定義は一般化されすぎている.
%    Renzoのノートでは$\mathcal{P}$を
%    ``Let $\mathcal{P}$ define a class of objects in some category $\mathcal{C}$."
%    としているが,
%    これでは写像$\mathcal{F} \to B$が定義できるか怪しい.
%    なので私のこのノートでは$\mathcal{P}$を集合のクラスに限定している.
    ``family"を上のように解釈して不整合が生じたことは,
    私の経験の中ではない.

    \begin{Remark}
        moduli theory以外で``family of $\mathcal{C}$"と言えば,
        単に$\mathcal{C}$の部分集合であろう.
        ``family parametrized by $S$"の様に言えば,
        $S$-indexed family (or set)のことを想像するであろう.
        しかし$S$-indexed family :: $\famF \subset \mathcal{C}$は
        $S \to \famF$という写像で定まるから,
        ここでの``family"とは写像の向きが逆である.
        
        上の定義を無心に読めば分かる通り,
        「$\mathcal{C}$のfamily :: $\famF$」と言った時には,
        $\mathcal{C}$に属すのは$\famF$の部分集合である.
        属すのは(一般に)$\famF$の元ではない.
        また$\famF$は$\mathcal{C}$の元の和集合とみなせる.
        (正確には$\mathcal{C}$の元を$S$に沿って並べたものである.)
    \end{Remark}

    \begin{Example}
        $X, B$ :: scheme,
        $f: X \to B$ :: morphism of schemesをとる.
        $X$は$f$によって$B$上のfamilyとなる.
        射のfibreとして実現される,
        scheme(例えばsmooth curve)のfamilyは
        deformation theoryの対象である.
    \end{Example}

    \begin{Example}\label{example:grassmannian}
        $k$を体,$S$を適当なschemeとする.
        $\affine^2_k$の原点を通る直線の$S$上のfamilyとして,
        line bundle :: $\shL \subset \affine^2 \times_k S$を
        考えることが出来る.
        $\shL \to S$は射影写像で与えられる.
        同様に$\affine^n$の$r$次元線形空間の$S$上のfamilyは
        $r$次元vector bundle :: $\shE \subset \affine^n \times S$である.
    \end{Example}

    \begin{Example}
        $k$を適当な体とし,
        $\proj^1_k$の点$O_i~(i=1,2,3)$を順に$(0:1), (1:0), (1:1)$とする.
        この時,$PGL_2(k)$は
        次の全単射で$\proj^1_k$の自己同型写像の$(\proj^1_k)^{\oplus 3}$上のfamilyになる.
        \begin{defmap}
            \pi:& PGL_2(k)& \to& (\proj^1_k)^{\oplus 3} \\
            {}& \phi& \mapsto& (\phi^{-1}(O_i))_{i=1}^3.
        \end{defmap}
    \end{Example}

    \subsection{Moduli Functor}
    以下の定義は\cite{HaMo}など,
    Moduli問題に関する殆どの入門書で述べられている.
    \begin{Def}
        contravariant functor :: $\ftorM : \Sch \to \Set$が
        \textbf{moduli functor}(またはfunctor of families)であるとは,
        各scheme :: $S$に対して,
        $\ftorM(S)$が代数幾何学的対象の$S$上のfamily達を
        familyの間の同値関係で割ったもの
        (``$\{ \text{families over }S \}/\sim_S$" in \cite{Hos})である,
        ということ.
    \end{Def}
    moduli functorの定義はあえて曖昧に述べられている.
    これは「出来る限り多くのものをmoduli theoryの範疇に取り込みたい」
    という思いがあるからである(\cite{HaMo}).

    \subsection{Fine Moduli Space}
    \begin{Def}
        scheme :: $M$が
        moduli fuctor :: $\ftorM$に対するfine moduli spaceであるとは,
        $M$が$\ftorM$を表現する(represent)ということである.
        言い換えれば,
        関手$\func{M}=\Hom_{\Sch}(-, M)$が$\ftorM$と自然同型,ということである.
    \end{Def}

    \begin{Remark}
        moduli functor :: $\ftorM$のfine moduli space :: $M$が存在したとしよう.
        この時,任意の$X \in \Sch$について$\ftorM(X) \iso \func{M}(X)$.
        これは
        $X$上の代数幾何学的対象が成す同値類が
        $M$の$X$-value pointと一対一に対応していることを意味する.
        したがって,
        $\ftorM$が指定する代数幾何学的対象の集合の同値類を
        $M$が「パラメトライズ」していると考えられる.
    \end{Remark}

    \begin{Example}[\cite{Hos}, Exercise 2.20]
        例\ref{example:grassmannian}で述べた
        $\affine^n$の$r$次元線形空間の$S$上のfamily
        (vector bundle over $S$)の集合を,
        vector bundleの同型で割った集合を$\ftorM(S)$とする.
        $f: T \to S$に対する$\ftorM(f)$は,
        vector bundleへのpost-compositionで自然に定まる.

        このmoduli functorはfine moduli spaceを持つことが知られている.
        これがGrassmannian varietyである.
    \end{Example}

    残念ながら,多くのmoduli functorに対してfine moduli spaceが存在し得ない.
    (このあたりの議論は\cite{HaMo} p.3や\cite{HarDef} p.150にある.
    この節の終わりでも理由と例を示す.)
    そのためMumfordは(おそらくGIT本で)
    fine moduli spaceの代わりとしてcoarse moduli spaceを提唱した.

    \subsection{Coarse Moduli Space}
    \begin{Def}
        moduli functor :: $\ftorM$に対して,
        以下を満たすscheme :: $M$を$\ftorM$のcoarse moduli spaceと呼ぶ.
        \begin{enumerate}[label=(\roman*), leftmargin=*]
            \item
                自然変換$\eta: \ftorM \to \func{M}$が存在する.
            \item
                $\eta$は自然変換$\ftorM \to \func{\tilde{M}}$の中で最も普遍的である:
                \[
                \xymatrix
                {
                    {} & \ar[ld]_-{{}^{\forall} \tau}\ftorM \ar[rd]^-{\eta}& {} \\
                    {}^{\forall}\func{\tilde{M}} \ar[rr]_-{{}^{\exists!} \func{f}}& {} & \func{M}
                }
                \]
                この図式で$\tilde{M}$ :: scheme, $f: M \to \tilde{M}$.
            \item
                任意の代数閉体 :: $k$について
                $\eta_{\Spec k}: \ftorM(\Spec k) \to \func{M}(\Spec k)$は全単射である.
        \end{enumerate}
    \end{Def}

    \begin{Example}
        楕円曲線の$j$-不変量.
        後に示すとおり,これはfineでない.
    \end{Example}

    \subsection{Properties of Fine / Coarse Moduli Spaces}
    \begin{Prop}
        moduli functor :: $\ftorM$に対して
        coarse moduli spaceは同型を除いて一意である.
    \end{Prop}

    \begin{Prop}[\cite{HarDef}, Prop23.6]
        scheme :: $M$が
        moduli functor :: $\ftorM$に対する
        fine moduli spaceであるならば,
        $M$は$\ftorM$のcoarse moduli spaceでもある.
    \end{Prop}
    
    \begin{Prop}[\cite{HarDef}, Prop23.5]
        $S$ :: schemeのopen subschemeと包含写像が成す圏を
        $\OpenSubSch(S)$と書くことにする.
        これは$\Sch/S$のfull subcategoryである.

        moduli functor :: $\ftorM$が
        fine moduli spaceをもつならば,
        任意の$S$ :: schemeについて
        $\ftorM|_{\OpenSubSch(S)}$は$S$上のsheafである.
    \end{Prop}
    \begin{proof}
        $M$ :: fine moduli scheme for $\ftorM$とし,
        $S$ :: schemeを固定する.
        $\shF:=\func{M}|_{\OpenSubSch(S)}$は
        開集合系からのcontravariant functorだから
        presheafであることは定義から従う.
        また$\shF$の元はschemeのmorphismである.
        このことからsheafの公理Identity AxiomとGluability Axiomを
        満たすことも簡単に分かる.
        (一応,\cite{HarAG} II, Thm3.3 Step3を参考に挙げる.)
    \end{proof}

    \subsection{Pathological behaviour.}
    $\famF, \famG \to S$をfiber of morphismで実現されるfamilyだとする.
    (したがって$\famF, \famG$はschemeである.)
    $\famF, \famG$の同値関係を,schemeとしての同型で定めよう.
    $M$をcoarse moduli space, 
    $\eta$をmoduli functorから$\func{M}$への自然変換だとする.

    $\eta_S(\famF): S \to M$は$\famF$のfiberを$M$の点に対応させる.
    (添字の$S$は以降略す.)
    \[
    \xymatrix@R=10pt
    {
        \famF \ar[r]& S \ar[r]^-{\eta(\famF)}& M \\
        \famF_s \ar@{{|}->}[r]& s \ar@{{|}->}[r]& m
    }
    \]
    $\eta(\famG): S \to M$についても同様である.
    したがって$\famF, \famG$が
    fiber毎に同型であれば,$\eta(\famF)=\eta(\famG)$となる.

    $\eta$は全単射であるから,
    これは$\famF \iso \famG$を意味する.
    しかし,対象が非自明な自己同型写像をもつときには
    このようにならないfamilyが構成できてしまう.
    % 構成はmarrten.pdf Example 3.2

    \begin{Example}[\cite{HarDef} \S 26]
        $S=\affine^1_k-\{0\}$とする.
        $S$上の楕円曲線のfamily :: $\famF$を次で定める.
        \[
            \famF=\zerosa(y^2-x^3-s) \subseteq \affine^2_k \times_k S
            \xrightarrow{\pr} S.
        \]
        $\eta(\famF)$を$j$不変量を用いて$s \mapsto j(\shF_s)$で定める.
        $j$不変量がcoarse moduliであることは既に見た.
        計算すると分かる通り,$\eta(\famF)$は定値写像である.
        したがって$\famF$のそれぞれのfiberは互いに同型である.
        一方,$\famF'=\zerosa(y^2-x^3-1) \times S$について
        同様に$\eta(\famF')$を定めると,
        これも自明に定値写像である.
        しかし,$\famF \not \iso \famF'$であることが示せる(TODO).
        よって$j$不変量はfine moduliにならない.
        fine/coarse moduliの一意性から,
        楕円曲線はfine moduliを持たない.
    \end{Example}

    それぞれのfiberが互いに同型である
    (i.e. $\Forall{t,s \in S} \famF_t \iso \famF_s$)ような
    familyをfiberwise trivial family, 
    $X \times S$の形に書けるfamilyをtrivial familyと呼ぶ.
    一般に,fine moduli spaceが存在するならば,
    fiberwise trivial familyはtrivial familyである
    (\cite{HarDef} Remark23.1.1).

    また,coarse moduli spaceさえ持ち得ないmoduli functorもある.
    これはjump phenomenonと呼ばれる性質を持つfamilyが存在する場合や,
    あるいはmoduli fuctorが``unbounded"であるときに起きる.
    このノートでは深追いしない.
    詳しくは\cite{Hos} \S 2.4を参照せよ.

\section{Definition of Group Schemes}
    familyの同値関係は,
    しばしば群作用の軌道分解で与えられる.
    そのためmoduli問題の理解のために,
    group schemeを知ることは不可欠である.

    \subsection{Definition}
    group schemeは圏論的に定義される.
    まずは圏論の言葉で述べよう.
    \begin{Def}
        $S$ :: schemeとする.
        $G$ :: scheme over $S$がgroup scheme (over $S$)であるとは,
        $G$が$\Sch/S$におけるgroup objectであるということである.
        group scheme over $S$とhomomorphismsが成す圏を
        $\GrpSch(S)$と書く.
    \end{Def}
    group objectとhomomorphismsの定義を展開すれば次のよう.
    \begin{Def}
    \enumfix
    \begin{enumerate}[label=(\roman*),leftmargin=*]
        \item
        $S$ :: schemeとする.
        $G$ :: scheme over $S$と次の3つの射から成る4つ組が
        \textbf{group scheme (over $S$)}であるとは,
        任意の$T \in \Sch/S$について
        $\func{G}(T)$の群構造が誘導されるということである.
        \begin{alignat*}{2}
            \mu&:       G \times G \to G    && \quad \text{multiplication} \\
            \epsilon&:  S \to G             && \quad \text{identity section}\\
            \iota&:     G \to G             && \quad \text{inverse}
        \end{alignat*}
        $\mu$はgroup lawとも呼ばれる.
        なお,$x, y \in \func{G}(T)$の
        積$x \ast y \in \func{G}(T)$は$\func{\mu}(\langle x, y \rangle)$,
        すなわち次の射である.
        \[
            x \ast y:
        \xymatrix
        {
            T \ar[r]^-{\langle x, y \rangle}& G \times G \ar[r]^-{\mu}& G
        }
        \]
        ここで$\langle x, y \rangle$は
        $G \xleftarrow{x} T \xrightarrow{y} G$から
        productの普遍性により誘導される射である.
        単位元は$\epsilon!: T \to S \xrightarrow{\epsilon} G$
        \footnote{ この射は$T \to G \to S \to G$と書いても同じである. },
        $x \in \func{G}(T)$の逆元は$\func{\iota}(x)=\iota \circ x$である.

        \item
        group scheme over $S$ :: $G,H$の間の射
        $h: G \to H$が\textbf{homomorphism}であるとは,
        任意の$T \in \Sch/S$について
        $\func{h}(T):\func{G}(T) \to \func{H}(T)$が群準同型であることである.

        \item 
            group schemes over $S$とその間のhomomorphismsが成す圏を\textbf{$\GrpSch(S)$}とする.
    \end{enumerate}
    \end{Def}

    しかしながら,ここで述べたgroup schemeの定義は実用に向かない.
    定義にどこの馬の骨とも知れないscheme :: $T$が現れるからである.
    以上の定義は以下と同値であることを言っておこう.
    \begin{Prop}[\cite{Awodey}, p.76]
    \begin{enumerate}[label=(\roman*),leftmargin=*]
        \item 
        $S$ :: scheme,
        $G$ :: scheme over $S$とし,
        更に3つの射$\mu, \epsilon, \iota$が与えられているとする.
        \begin{alignat*}{1}
            \mu&:       G \times G \to G \\
            \epsilon&:  S \to G          \\
            \iota&:     G \to G         
        \end{alignat*}
        この時,$(G, \mu, \epsilon, \iota)$が
        group schemeであることと,
        以下の3つの可換図式が成立することは同値である.
        \begin{gather}
            \xymatrix
            {
                (G \times G) \times G \ar[d]_-{\mu \times \id} \ar[rr]^-{\iso}&
                {} & G \times (G \times G) \ar[d]^-{\id \times \mu}\\
                G \times G \ar[dr]_-{\mu}& {} & \ar[dl]^-{\mu} G \times G \\
                {} & G & {}
            } \\
            \xymatrix@C=40pt
            {
                G
                    \ar[r]^-{\langle \epsilon!, \id[] \rangle}
                    \ar[d]_-{\langle \id[], \epsilon! \rangle}
                    \ar[rd]_-{\id}&
                G \times G \ar[d]^-{\mu} \\
                G \times G \ar[r]_-{\mu} & G
            } \\
            \xymatrix@R=30pt@C=40pt
            {
                \ar[d]_-{\id[] \times \iota} G \times G &
                G
                    \ar[l]_-{\langle \id[], \id[] \rangle}
                    \ar[d]_-{\epsilon!}
                    \ar[r]^-{\langle \id[], \id[] \rangle}&
                    G \times G \ar[d]^-{\iota \times \id[]}\\
                    G \times G \ar[r]_-{\mu}& G & \ar[l]^-{\mu}G \times G \\
            }
        \end{gather}
        上から順に,結合律,単位元の存在,逆元の存在に対応する.

        \item
        group scheme over $S$ :: $G,H$と,
        射$h: G \to H$が与えられているとする.
        $h$がhomomorphismであることは,
        以下の3つの可換図式が成立することは同値である.
        \begin{gather}
            \xymatrix
            {
                \ar[d]_-{\mu} G \times G \ar[r]^-{h \times h}& H \times H \ar[d]^-{\mu}\\
                G \ar[r]_-{h}& H
            } \\
            \xymatrix
            {
                G \ar[r]^-{h}& H \\
                S \ar[u]^-{\epsilon} \ar[ru]_-{\epsilon}& {}
            } \\
            \xymatrix
            {
                \ar[d]_-{\iota} G \ar[r]^-{h}& H \ar[d]^-{\iota}\\
                G \ar[r]_-{h}& H
            }
        \end{gather}
        上から順に,積の保存,単位元の保存,逆元の保存に対応する.
    \end{enumerate}
    \end{Prop}

    \subsection{Examples}
    以下の例では$k$を適当な体とし,$k$上のaffine group schemeを定義する.
    \begin{Example}[$\Ga$]
        finitely generated $k$-algebra :: $A=k[x]$と次の3つの$k$-linear mapから,
        $k$上のgroup scheme :: $\Ga$ \& $\mu,\epsilon,\iota$が誘導される.
        \begin{alignat*}{3}
            \tilde{\mu}&:
                A \to A \otimes_k A; &&
                \quad x \mapsto (x \otimes 1)+(1 \otimes x) \\
            \tilde{\epsilon}&:
                A \to k; &&
                \quad x \mapsto 1 \\
            \tilde{\iota}&:
                A \to A; &&
                \quad x \mapsto -x
        \end{alignat*}
        群構造を無視すれば$\Ga=\affine^1_k$.
        この$\Ga$はadditive groupと呼ばれる.

        $x_1=x \otimes 1, x_2=1 \otimes x$とすると,
        $A \otimes A \iso k[x_1,x_2]$となる.
        したがって$f \in A$について$\tilde{\mu}(f)(x_1,x_2) \in k[x_1,x_2]$とみなせる.
        そして$k[x]$のalgebraとしての和は
        $\tilde{\mu}(f)(x_1,x_2)=f(x_1+x_2)$のようにco-algebraに反映されている.
        単位元と逆元は$\tilde{\epsilon}(f)(x)=f(1), \tilde{\iota}(f)(x)=f(-x)$のように
        反映されている.
    
        $\Ga$に備わった群構造はclosed point :: $(a,b) \in \affine^2$を$a+b \in \affine^1$に写す.
        これを確かめておこう.
        $\affine^1_k \times_k \affine^1_k \iso \affine^2_k$の
        $\bar{k}$-rational point :: $(a,b)$
        \footnote
        {
            $\bar{k}$は$k$の代数閉体.
            rational pointについての注意で触れたとおり,
            varietyの$\bar{k}$-rational point全体は
            closed point全体と一致するのであった.
        }
        は極大イデアル
        \[ \I{p}=(x_1-a, x_2-b)=\{ f \in k[x_1,x_2] \mid f(a,b)=0 \} \]に対応する.
        したがって$\mu(\I{p})=\tilde{\mu}^{-1}(\I{p})$は次のよう.
        \[ \tilde{\mu}^{-1}(\I{p})=\{ g \in A=k[x] \mid \tilde{\mu}(g)(a,b)=g(a+b)=0 \}. \]
        これは$a+b$に対応する極大イデアル$(x-(a+b))$に他ならない.
    \end{Example}

    \begin{Example}
        finitely generated $k$-algebra :: $A=k[x,x^{-1}]$と次の3つの$k$-linear mapから,
        $k$上のgroup scheme :: $\Gm$ \& $\mu,\epsilon,\iota$が誘導される.
        \begin{alignat*}{3}
            \tilde{\mu}&:
                A \to A \otimes_k A; &&
                \quad x \mapsto (x \otimes 1) \cdot (1 \otimes x) \\
            \tilde{\epsilon}&:
                A \to k; &&
                \quad x \mapsto 1 \\
            \tilde{\iota}&:
                A \to A; &&
                \quad x \mapsto -x
        \end{alignat*}
        群構造を無視すれば$\Gm=\affine^1_k-\{0\}$.

        こちらも$\tilde{\mu}(f)(x_1,x_2)=f(x_1x_2)$の様に積が入っている.
        $\mu: \Gm \times \Gm \to \Gm$が
        $\bar{k}$-rational point :: $(a,b) \in \affine^2-\{(a,b) \mid ab=0\}$を
        $ab \in \affine^1$に写すことは$\Ga$の場合と同様である.
    \end{Example}
    \begin{Example}
        正整数$n$に対し
        finitely generated $k$-algebra :: $A=k[x_{ij}]_{i,j=1}^n [\det^{-1}]$とおく.
        ここで$\det$は不定元が成す$n$次正方行列$X=\tatev{x_{ij}}_{i,j=1}^n$のdeterminantである.
        $A$と次の3つの$k$-linear mapから,
        $k$上のgroup scheme :: $\GL_n$ \& $\mu,\epsilon,\iota$が誘導される.
        \begin{alignat*}{3}
            \tilde{\mu}&:
                A \to A \otimes_k A; &&
                \quad X \mapsto (X \otimes 1) \cdot (1 \otimes X) \\
            \tilde{\epsilon}&:
                A \to k; &&
                \quad X \mapsto I \\
            \tilde{\iota}&:
                A \to A; &&
                \quad X \mapsto X^{-1}
        \end{alignat*}
        $I$は$n$次単位行列.
        ここで$\tilde{\iota}: X \mapsto I$は
        ($X$の$(i,j)$成分)$\mapsto$($I$の$(i,j)$成分)という意味である.
        $\tilde{\mu}, \tilde{\iota}$の定義も同様である.

        $X_1=X \otimes 1=\tatev{x_{ij} \otimes 1}_{i,j=1}^n,
        X_2=1 \otimes X=\tatev{1 \otimes x_{ij}}_{i,j=1}^n$
        とおけば,
        $f \in k[x_{ij}]$について
        $\tilde{\mu}(f)(X_1, X_2)=f(X_1X_2)$となっている.
        $\mu: \GL_n \times \GL_n \to \GL_n$が
        $\bar{k}$-rational point :: $(M,N) \in \GL_n \times \GL_n$を
        $MN \in \GL_n$へ写すことは$\Ga$での議論と同様である.
        $n=1$の時$\GL_n=\Gm$であることに留意せよ.
    \end{Example}

    3つの例に現れた準同型$\tilde{\mu},\tilde{\epsilon},\tilde{\iota}$は
    それぞれco-multiplication,co-unit, co-inversionと呼ばれる.
    この3つの準同型によってそれぞれの$k$-finitely generatedに
    Hopf algebra
    \footnote
    {
        algebra, co-algebraの構造をもつfinitely generated $k$-moduleであって
        antipodeと呼ばれる自己準同型射を備えるもの.
    }
    の構造が入る.
    一般にaffine group schemeとHopf algbraが
    一対一に対応する(\cite{MilneAGS} II,Thm5.1).

    \begin{Thm}[\cite{Hos} Thm3.9]
        Any affine group scheme over a field 
        is a linear algebraic group
        (i.e. subgroup of $GL_n$ for some $n$).
    \end{Thm}
    
    \subsection{Action on Scheme}
    最後に作用の定義を述べる.
    \begin{Def}
        scheme/$S$ :: $X$と,
        group scheme/$S$ :: $G$が与えられたとする.
        $G$の$X$への\textbf{(left) action}とは,
        次を満たす射$\alpha: G \times_S X \to X$である.
        すなわち,任意の$T \in \Sch/S$について,
        $\alpha$は群$\func{G}(T)$から集合$\func{X}(T)$への作用を誘導する.

        なお,$g \in \func{G}(T)$を
        $x \in \func{X}(T)$に作用させた$g \cdot x \in \func{X}(T)$は
        $\func{\alpha}(\langle g,x \rangle)$,すなわち
        \[
        \xymatrix
        {
            T \ar[r]^-{\langle g,x \rangle}& G \times_S X \ar[r]^-{\alpha}& X
        }
        \]
        である.
        ここで$\langle g, x \rangle$は
        $G \xleftarrow{g} T \xrightarrow{x} X$から
        productの普遍性により誘導される射である.

        action :: $G \times X \to X$のことを$G \acton X$と表す.
        $g \in G$を$x \in X$に作用させたものをしばしば$g \cdot x$と書く.
        $Gx:=\alpha(G \times \{x\})$と置き,
        これを点$x$の\textbf{orbit}と呼ぶ.
    \end{Def}

\section{Categorical/Good/Geometric/Affine GIT Quotients}
    % Moduli Theory and Classification Theory of Algebraic Varieties(H_Popp)
    % のp.55からのlecture 5も参考に成る.
    % 特にRemark5.3にcategorialだがgoodでないquotientの例がある.
    % 同様にgoodだがgeometricでないquotientの例もある.
    % graph mapがproperの時,
    % geometric quotientは単に|Q|=|X|/Gなるgood quotientだと言える.


    \textbf
    {
    以下,
    scheme ::$S$,
    scheme/$S$ :: $X$,
    group scheme/$S$ :: $G$,
    action :: $\alpha: G \acton X$が与えられているとする.
    }

    \subsection{Categorical Quotient}
    圏論的立場から「schemeのgroup schemeによるquotient」と呼べるschemeは,
    categorical quotientであろう.
    \begin{Def}
    \begin{enumerate}[label=(\roman*), leftmargin=*]
        \item
            schemeの射 :: $q: X \to Y$は,
            $q \circ \alpha=q \circ \pr_X: G \times X \to Y$であるとき,
            \textbf{$G$-invariant morphism}と呼ばれる.
            この条件は次のように翻訳できる:
            任意の$T \in \Sch/S$と
            任意の$g \in \func{G}(T), x \in \func{X}(T)$について,
            \[
                \func{q}(\alpha \circ \langle g,x\rangle)
                =\func{q}(g \cdot x)
                =\func{q}(x)
                =\func{q}(\pr_X \circ \langle g,x \rangle).
            \]

        \item
            schemeの射$q: X \to Y$は,
            $q$が$\alpha, \pr_X: G \times X \rightrightarrows X$の
            coequalizerであるとき,
            $X$の$G$による\textbf{categorial quotient}と呼ばれる.
            言い換えれば,
            $X$からの$G$-invariant morphismとして普遍的なものが$q$である.

%        \item
%            $S, S' \in \Sch$の間に射$S' \to S$が存在するとしよう.
%            この時,fiber productは関手$- \times_S S': \Sch/S \to \Sch/S'$とみなすことが出来る.

%            $X$の$G$によるcategorical quotient :: $q: X \to Y$は,
%            次を満たす時,$X$の$G$による\textbf{universal categorical quotient}と呼ばれる:
%            任意の$S' \in \Sch/S$について,
%            $q \times S': X \times S' \to Y \times S'$は
%            $X \times S'$の$G \times S'$によるcategorical quotientである.
    \end{enumerate}
    \end{Def}
    すぐに分かる通り,
    この定義は$\Sch$を「finite productを持つcategory」に書き換えても良い.
    この意味で,
    categorical quotientは最も普遍的な「群作用での商」を定義していると言える.

    \begin{Remark}
        categorical quotientは普遍性を持つもの
        と定義されていることから分かる通り,
        同型を除いて\kenten{高々}一つしか無い.
        存在するかどうかは分からない.
    \end{Remark}

    さて,categorical quotientは確かに「商らしい」が,
    幾何学的にも「商らしい」と言えるとは限らない.

    \begin{Example}\label{example:P,A/Gm}
        $\proj^1_k, \affine^1_k$の$\Gm$による群作用の商.
    \end{Example}

    そのために,他の意味で「商らしい商」もいくつか定義する.
    どういった意味で「商らしい」かによって定義は異なり,
    以後は「商らしい商」が存在するかどうか・構成できるかどうかが問題に成る.
    後に示すが,
    以下の「商らしい商」はいずれもcategorical quotientである.
    なので一つ「商らしい商」が存在すれば,
    その商はより弱い意味でも「商らしい」ことに注意せよ.

    \subsection{\texorpdfstring{$G$}{G}-invariant Function}
    \begin{Def}[\cite{AV}]
        $U$ :: open in $X$について,
        $f \in \shO_X(U)$は
        \[ \alpha^{\#}(f)|_{G \times_S U}=\pr_X^{\#}(f) \in \shO_{G \times_S X}(G \times_S U) \]
        が成り立つとき\textbf{$G$-invariant function}と呼ばれる.
        ここで,
        $\alpha^{-1}(U) \supseteq G \times_S U=\pr_X^{-1}(U)$であるために
        左辺にrestrictionが必要であることに注意.
    \end{Def}

    \begin{Claim}
        $G$-invariant functionを集めてできる
        $\shO_X$のsub-presheaf :: $\shO_X^G$は,
        sheafである.
    \end{Claim}
    \begin{proof}
        sub-presheafであることは明らか.
        また$\shO_X^G$がidentity axiomを満たすことは
        sub-presheafであることから従う.
        なのでgluability axiomを満たすことを示そう.

        $U$ :: open in $X$をとり,
        $U=\bigcup_i U_i$をそのopen coverとする.
        section達$t_i \in (\shO_X^G)(U_i)$が次を満たすとしよう.
        \[
            \Forall{i,j}
            t_i|_{U_i \cap U_j}=t_j|_{U_i \cap U_j}
            \in \shO_X(U_i \cap U_j).
            \eqno{(@)}
        \]
        この時,$t|_{U_i}=t_i$を満たす$t \in \shO_X(U)$が存在する.
        $t \in \shO_X^G(U)$を示したい.
        $V_i=\pr_X^{-1}(U_i)=G \times U_i$とおく.
        上の条件$(@)$で$t|_{U_i}=t_i$と,
        sheaf morphismがrestriction mapと可換であることを用いる.
        \[
            \Forall{i,j}
            \alpha^{\#}(t)|_{V_i \cap V_j}
            =\alpha^{\#}(t|_{U_i \cap U_j})|_{V_i \cap V_j}
            =
            \pr_X^{\#}(t|_{U_i \cap U_j})
            =\pr_X^{\#}(t)|_{V_i \cap V_j}
            \in \shO_{G \times X}(V_i \cap V_j).
        \]
        $\shO_{G \times X}$はsheafだから,
        identity axiomを満たす.
        よって$\alpha^{\#}(t)=\pr_X^{\#}(t)$,
        すなわち$t \in \shO_X^G(U)$.
    \end{proof}

    \begin{Remark}
        affine scheme :: $X:=\Spec A$のstructure sheafの元は,
        morphism :: $U \to \affine^1_{\Quot(A)} \in \shO_X(U)$とみなすことが出来る.
        そこで$G$-invariant functions :: $(\shO_X(U))^G$を
        $\shO_X(U)$に属す$G$-invariant morphismの全体と定めることも出来る.
    \end{Remark}
    
    \subsection{Good and Geometric Quotient.}
    \begin{Def}[\cite{Hos}]
        scheme morphism :: $q: X \to Y$が
        $X$の$G$による\textbf{good quotient}であるとは,
        以下の条件が満たされるということ.
        \begin{enumerate}[label=(\roman*), leftmargin=*]
        \item $q$ :: $G$-invariant.
        \item $q$ :: surjective.
        \item $q$ :: affine morphism
              \footnote{ すなわち,$Y$の任意のaffine open subschemeの$q$による逆像がaffine. }.
        \item $(q_*\shO_X)^G \iso \shO_Y$.
        \item
            $W$ :: $G$-invariant closed subset of $X$について,
            $q(W)$ :: closed subset of $Y$.
        \item
            $W_1, W_2$ :: disjoint $G$-invariant closed subsets of $X$について,
            $q(W_1), q(W_2)$ :: disjoint closed subsets of $Y$.
    \end{enumerate}
    \end{Def}
    $q$ :: surjectiveと$q$ :: affineの2条件は,
    GIT \cite{GIT}でなく\cite{Ses}で導入された.
    実際,次の命題はこの2条件がなくとも成立する.
    しかしsurjectivityはquotientをfamilyとして利用するために
    望ましい性質である.
    (affinenessについてはよくわからない.しかし我々が扱う範囲で成立する.)

    \begin{Prop}
        good quotient is categorical quotient also.
    \end{Prop}

    \begin{Def}[\cite{Hos}]
        \textit{good quotient} :: $q: X \to Y$が
        各点$y \in Y$について$q^{-1}(y)$がただひとつのorbitからなる時,
        $q$は$X$の$G$による\textbf{categorical quotient}と呼ばれる.
    \end{Def}

    % \cite{Hos} Moduli spaces and quotients.

    \subsection{Affine GIT Quotient}
    field :: $k$
    affine scheme/$k$ :: $X$,
    affine group scheme/$k$ :: $G$,
    action :: $\alpha: G \acton X$が与えられているとする.
    この場合には,直接構成できるquotient schemeがある.
    それがGIT(Geometric Invariant Theory) quotientである.
    Mumfordが構成した.
    % \cite{Hos} p.23に$G \acton X=\Spec A$の時の$A^G$の定義がある.
    % これは通常の$\shO_X(X)^G$の定義を使えば分かる.
    
    最初に,affine $\acton$ affineの場合の作用について述べておこう.
    $\alpha: G \acton X$に対応する環準同型を
    $\tilde{\alpha}: k[X] \to k[X] \otimes_k k[G]$とする.
    一方,$\pr_X: G \times X \to X$に対応する環準同型は
    \begin{defmap}
        {}& k[X]& \to& k[X] \otimes_k k[G] \\
        {}& x & \mapsto& x \otimes_k 1
    \end{defmap}
    である.
    したがって$\shO_X^G( \subseteq \shO_X)$のglobal sectionは次のように成る.
    \[
        k[X]^G
        :=\{ f \in k[X] \mid \tilde{\alpha}(f)=f \otimes 1 \}
        \subseteq \Gamma(X, \shO_X).
    \]

    \begin{Def}
        $X \sslash G:=\Spec k[X]^G$を
        $G$による$X$のaffine GIT quotientと呼ぶ.
    \end{Def}
    $X \sslash G$という記号は,
    affine GIT quotientは必ずしも(affine) geometric quotientでない,
    ということを意味している(例\ref{example:P,A/Gm}).
    しかしどちらもcategorical quotientであるから,
    両方共存在するならばそれらは同型を除いて一致する.

    affine GIT quotientを張り合わせていくことで,
    projective GIT quotientが構成される.
    \cite{Muk1} chapter 6を参照せよ.

    \subsection{Relations of Quotients.}
    \[
    \xymatrix@C=10pt
    {
        \text{Geometric} \ar@{=>}[rd]& {} & \ar@{=>}[ld]^{\text{Thm} \ref{thm:git=good}} \text{GIT*} \\
        {} & \text{Good} \ar@{=>}[d]& {}\\
        {} & \text{Categorical} & {}
    }
    \]

\section{Linear Reductivity.}
%    $X,G$共にfinite type/$k$とする
%    \footnote{ この条件は$A^G$ :: finitely generatedを示す為に必要である. }.
    % 作用が局所有限=rational.
    % \cite{Hos} p.15, Lemma3.8より,今の状況では$G \acton X$ :: rational.
%    \begin{Thm}[\cite{Hos} Thm4.25]\label{thm:finiteness}
%        $k$ :: field,
%        $A$ :: finitely generated $k$-algebra,
%        $G$ :: linearly reductive (abstruct) group acting on $A$,
%        とする.
%        $G$の$A$への作用がrationalならば,
%        invariant ring :: $A^G$はfinitely generated.
%    \end{Thm}
    % \cite{Hos} Thm 4.26
%    An affine algebraic group G over k is reductive
%    if and only if
%    for every rational G-action on a finitely generated k-algebra A,
%    the subalgebra A G of G-invariants is finitely generated.
%    $G$がnon-reductive groupであるとき,
%    $k[X]^G$がnot finitely generatedであるようなaffine scheme :: $X$が
%    存在することを示した.

    \textbf
    {
    引き続き,
    field :: $k$
    affine scheme/$k$ :: $X$,
    affine group scheme/$k$ :: $G$,
    action :: $\alpha: G \acton X$が与えられているとする.
    }
    まず現状の確認をしよう.
    我々はaffine GIT quotientを定義した.
    affine GIT quotientは不変式環で与えられるため,
    多くの場合で具体的に計算することが出来る.
    しかしこれがcategorical/good/geometric quotient
    であるかどうかはまだ我々には分からない.
    
    我々はgroup scheme :: $G$が
    ``linearly reductive"という性質を備えている場合に
    $G$によるaffine schemeのaffine GIT quotientが
    categorical/good/geometric quotientであることを示す.
    (実はより弱い``reductive"で十分なのだが,
    これを定義するだけでも骨が折れるので,
    このノートでは扱わない,)
    このセクションでは``linearly reductive"を定義し,
    調べていく.

    \subsection{Linear Representation} 
    group schemeのlinear representationとして最も一般的なものは次のものである.
    \begin{Def}[\cite{MilneiAG} 4,a]
        $V$ :: vectoe space over $k$に対し,
        $k$代数の圏から群の圏への関手$\ftorGL_V$を
        \[ R \mapsto \Aut_R(V \otimes_k R) \]
        で定める.
        $G$の$V$への\textbf{linear representation}とは,
        (群の圏への関手の)準同型$\rho: \func{G} \to \ftorGL_V$のことである.
    \end{Def}
%    \begin{Def}[\cite{MilneiAG} 4,aの別の定義]
%        $V$ :: vectoe space over $k$に対し,
%        有限生成$k$代数の圏から群の圏への関手$\func{V}$を
%        \[ R \mapsto V \otimes_k R \]
%        で定める.
%        $G$の$V$へのlinear representationとは,
%        次の条件を満たす自然変換
%        $\rho: \func{G} \times \func{V} \to \func{V}$のことである.
%        すなわち,
%        任意の有限生成$k$代数$R$について
%        $\rho_R$は作用$\func{G}(R) \acton \func{V}(R)=V \otimes_k R$である.
%        この準同型は$R$加群としてのものである
%        \footnote
%        {
%            $\func{G}(R)=\Hom_{\Sch}(\Spec R, G)$は,
%            $G$ :: affine schemeより環準同型 :: $k[G] \to R$に対応する.
%            これは$(r \cdot \phi)(*)=r \phi(*)$のようにして自然に$R$加群の構造を持つ.
%        }.
%    \end{Def}

    \begin{Remark}
        $V$が有限次元である場合には$\ftorGL_V$は
        affine group scheme :: $GL_{\dim V}$で表現できる.
        また,この時$\rho$はhomomorphism of group schemeである.
    \end{Remark}
    \begin{Remark}
        直前の注意で述べたことを使って
        有限次元$k$-vector space :: $V$への$G$の表現が
        \cite{Muk1}での定義と一致することを確かめよう.
        
        $n=\dim V$とする.
        直前の注意から,$G$の$V$への表現は
        group scheme homomorphism :: $h: G \to GL_n$である.
        対応する環準同型を$\tilde{h}: k[X, (\det X)^{-1}] \to k[G]$としよう.
        ここで$X$は不定元からなる行列$\tatev{ x_{ij} }_{i,j=1}^n$である.
        (group schemeの例にある$GL_n$の定義も参照せよ.)
        $\tilde{X}:=\tilde{h}(X)$とおいて,
        $\bar{h}: V \to V \otimes_k k[G]$を次で定める.
        \[ v \mapsto (v \otimes 1)(1 \otimes \tilde{X}). \]
        
        逆に$\mu: V \to V \otimes_k k[G]$が与えられているとして
        $h: G \to GL_n$を構成する.
        これには$\bar{h}$から$\tilde{X}$の情報を取り出せば良い.
        $V$の基底を$\{ e_k \}_{k=1}^n$とし,その双対基底を$\{ e^k \}$とする.
        また$\iota: k \otimes k[G] \mapsto k[G]$を標準的同型とする.
        以上の準備の下で$\tilde{X}$の$(i,j)$成分$\tilde{x}_{ij}$は次の様に取り出せる.
        \[
        \xymatrix
        {
            e_i \ar@{|->}[r]^-{\bar{h}}& 
            \sum_{k=1}^n e_k \otimes \tilde{x}_{ij} \ar@{|->}[r]^-{e^j \otimes \id}&
            1 \otimes \tilde{x}_{ij} \ar@{|->}[r]^-{\iota} &
            \tilde{x}_{ij}
        }
        \]
        こうして取り出した$\tilde{X}$から再び$\mu$を構成できることは明らか.
    \end{Remark}
    \begin{Example}
    \end{Example}

    \subsection{Linear Reductivity}
    \begin{Def}
        $G$のfinite dimensional linear representationの間にある
        任意の全射$\phi: W \to V$に対し,
        $\phi$から誘導される写像
        $\phi^G=\phi|_{W^G}: W^G \to V^G$も全射である時,
        $G$は\textbf{linearly reductive}であると呼ばれる.
    \end{Def}

\section{Affine GIT Quotient of Variety is Variety Again }

\section{Affine GIT Quotient is a Good Quotient}
    \begin{Thm}[\cite{Hos} Thm4.30]\label{thm:git=good}
        Affine GIT Quotient is Good Quotient.
    \end{Thm}

\begin{thebibliography}{99}
    \bibitem{Muk1}
    向井茂(2008)『モジュライ理論 I』岩波書店
    
    \bibitem{Hos}
    Victoria Hoskins (2016)
    ``Moduli Problems and Geometric Invariant Theory"
    \url{https://userpage.fu-berlin.de/hoskins/M15_Lecture_notes.pdf}

    \bibitem{AV}
    Gerard van der Geer, Ben Moonen
    ``Abelian Varieties"
    \url{https://www.math.ru.nl/~bmoonen/research.html}
    (Preliminary Version. 2017/12/31参照)

    \bibitem{HarAG}
    Robin Hartshorne(1977)
    ``Algebraic Geometry"
    Springer

    \bibitem{HarDef}
    Robin Hartshorne
    ``Deformation Theory"
    Springer

    \bibitem{Eisen}
    David Eisenbud(1999)
    ``Commutative Algebra: with a View Toward Algebraic Geometry"
    Springer
    
    \bibitem{MilneAGS}
    J. Milne, ``The basic theory of affine group schemes", 
    \url{http://www.jmilne.org/math/CourseNotes/AGS.pdf}
    
    \bibitem{MilneiAG}
    J. Milne, ``Algebraic Groups", 
    \url{http://www.jmilne.org/math/CourseNotes/iAG200.pdf}

    \bibitem{HaMo}
    J. Harris,I. Morrison ``Moduli of Curves"

    \bibitem{GIT}
    D.Mumford,J.Forgarty ``Geometric Invariant Theory"

    \bibitem{Awodey}
    S.Awodey ``Category Theory" 2nd ed.

    \bibitem{Ses}
    C.S. Seshadri (1972)
    ``Quotient spaces modulo reductive algebraic groups"

\end{thebibliography}
\end{document}
